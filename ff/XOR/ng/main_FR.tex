\clearpage
\subsection{Norton Guide: chiffrement XOR à 1 octet le plus simple possible}
\label{norton_guide}

Norton Guide était très
populaire à l'époque de MS-DOS, c'était un programme résident qui fonctionnait comme
un manuel de référence hypertexte.

Les bases de données de Norton Guide étaient des fichiers avec l'extension .ng, dont
le contenu avait l'air chiffré:

\begin{figure}[H]
\centering
\myincludegraphics{ff/XOR/ng/ng1.png}
\caption{Aspect très typique}
\end{figure}

Pourquoi pensons-nous qu'il est chiffré mais pas compressé?

Nous voyons que l'octet 0x1A (ressemblant à \q{$\rightarrow$}) est très fréquent,
ça ne serait pas possible dans un fichier compressé.

Nous voyons aussi de longues parties constituées seulement de lettres latines, et
qui ressemble à des chaînes de caractères dans un langage inconnu.

\clearpage
Puisque l'octet 0x1A revient si souvent, nous pouvons essayer de décrypter le fichier,
en supposant qu'il est chiffré avec le chiffrement XOR le plus simple.

Si nous appliquons un XOR avec la constante 0x1A à chaque octet dans Hiew, nous voyons
des chaînes de texte familières en anglais:

\begin{figure}[H]
\centering
\myincludegraphics{ff/XOR/ng/ng2.png}
\caption{XOR dans Hiew avec 0x1A}
\end{figure}

Le chiffrement XOR avec un seul octet constant est la méthode de chiffrement la plus
simple, que l'on rencontre néanmoins parfois.

Maintenant, nous comprenons pourquoi l'octet 0x1A revenait si souvent: parce qu'il
y a beaucoup d'octets à zéro et qu'ils sont remplacés par 0x1A dans la forme chiffrée.

Mais la constante pourrait être différente.
Dans ce cas, nous pourrions essayer chaque constante dans l'intervalle 0..255 et
chercher quelque chose de familier dans le fichier déchiffré. 256 n'est pas si grand.

Plus d'informations sur le format de fichier de Norton Guide: \url{http://www.davep.org/norton-guides/file-format/}.

\subsubsection{Entropie}
\myindex{Wolfram Mathematica}
\myindex{Entropy}

Une propriété très importante de tels systèmes de chiffrement est que l'entropie
des blocs chiffrés/déchiffrés est la même.

Voici mon analyse faite dans Wolfram Mathematica 10.

\begin{lstlisting}[caption=Wolfram Mathematica 10,style=custommath]
In[1]:= input = BinaryReadList["X86.NG"];

In[2]:= Entropy[2, input] // N
Out[2]= 5.62724

In[3]:= decrypted = Map[BitXor[#, 16^^1A] &, input];

In[4]:= Export["X86_decrypted.NG", decrypted, "Binary"];

In[5]:= Entropy[2, decrypted] // N
Out[5]= 5.62724

In[6]:= Entropy[2, ExampleData[{"Text", "ShakespearesSonnets"}]] // N
Out[6]= 4.42366
\end{lstlisting}

Ici, nous chargeons le fichier, obtenons son entropie, le déchiffrons, le sauvons
et obtenons à nouveau son entropie (la même!).

Mathematica fourni également quelques textes en langue anglaise bien connus pour
analyse.

Nous obtenons ainsi l'entropie de sonnets de Shakespeare, et elle est proche de l'entropie
du fichier que nous venons d'analyser.

Le fichier que nous avons analysé consiste en des phrases en langue anglaise, qui
sont proches du langage de Shakespeare.

Et le texte en langue anglaise XOR-é possède la même entropie.

% I checked!
Toutefois, ceci n'est pas vrai lorsque le fichier est XOR-é avec un pattern de plus
d'un octet.

Le fichier qui vient d'être analysé peut être téléchargé ici: \url{http://beginners.re/examples/norton_guide/X86.NG}.

\myparagraph{Encore un mot sur la base de l'entropie}

\newcommand{\FNENTURL}{\footnote{\url{http://www.fourmilab.ch/random/}}}

Wolfram Mathematica calcule l'entropie avec une base $e$ (base des logarithmes naturels),
et l'utilitaire\FNENTURL UNIX \emph{ent} utilise une base 2.

Donc, nous avons mis explicitement une base 2 dans la commande \TT{Entropy}, donc
Mathematica nous donne le même résultat que l'utilitaire \emph{ent}.

