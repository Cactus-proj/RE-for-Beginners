\subsection{Entropy of various files}

Entropy of random data is close to 8:

\begin{lstlisting}
% dd bs=1M count=1 if=/dev/urandom | ent
Entropy = 7.999803 bits per byte.
\end{lstlisting}

This means, almost all available space inside of byte is filled with information.

256 bytes in range of 0..255 gives exact value of 8:

\begin{lstlisting}[style=custompy]
#!/usr/bin/env python
import sys

for i in range(256):
    sys.stdout.write(chr(i))
\end{lstlisting}

\begin{lstlisting}
% python 1.py | ent
Entropy = 8.000000 bits per byte.
\end{lstlisting}

Order of bytes doesn't matter.
This means, all available space inside of byte is filled.

Entropy of any block filled with zero bytes is 0:

\begin{lstlisting}
% dd bs=1M count=1 if=/dev/zero | ent
Entropy = 0.000000 bits per byte.
\end{lstlisting}

Entropy of a string constisting of a single (any) byte is 0:

\begin{lstlisting}
% echo -n "aaaaaaaaaaaaaaaaaaa" | ent
Entropy = 0.000000 bits per byte.
\end{lstlisting}

\myindex{base64}
Entropy of base64 string is the same as entropy of source data, but multiplied by $\frac{3}{4}$.
This is because base64 encoding uses 64 symbols instead of 256.

\begin{lstlisting}
% dd bs=1M count=1 if=/dev/urandom | base64 | ent
Entropy = 6.022068 bits per byte.
\end{lstlisting}

Perhaps, 6.02 is slightly bigger than 6 because padding symbols (\TT{=}) spoils our statistics for a little.

\myindex{Uuencode}
Uuencode also uses 64 symbols:

\begin{lstlisting}
% dd bs=1M count=1 if=/dev/urandom | uuencode - | ent
Entropy = 6.013162 bits per byte.
\end{lstlisting}

This means, any base64 and Uuencode strings can be transmitted using 6-bit bytes or characters.

Any random information in hexadecimal form has entropy of 4 bits per byte:

\begin{lstlisting}
% openssl rand -hex $\$$(( 2**16 )) | ent
Entropy = 4.000013 bits per byte.
\end{lstlisting}

Entropy of randomly picked English language text from Gutenberg library has entropy $\approx 4.5$.
The reason of this is because English texts uses mostly 26 symbols, and $log_2(26)=\approx 4.7$, i.e., you would need
5-bit bytes to transmit uncompressed English texts, that would be enough (it was indeed so in teletype era).

Randomly chosen Russian language text from \url{http://lib.ru}
library is F.M.Dostoevsky ``Idiot''\footnote{\url{http://az.lib.ru/d/dostoewskij_f_m/text_0070.shtml}},
internally encoded in CP1251 encoding.

And this file has entropy of $\approx 4.98$.
Russian language has 33 characters, and $log_2(33)=\approx 5.04$.
But it has unpopular and rare ``ё'' character.
% FIXME YO letter isn't rendered in Eng version
And $log_2(32)=5$ (Russian alphabet without this rare character)---now this close to what we've got.

However, the text we studying uses ``ё'' letter, but, probably, it's still rarely used there.

\myindex{UTF-8}
The very same file transcoded from CP1251 to UTF-8 gave entropy of $\approx 4.23$.
Each Cyrillic character encoded in UTF-8 is usually encoded as a pair,
and the first byte is always one of: 0xD0 or 0xD1.
Perhaps, this caused bias.

Let's generate random bits and output them as ``T'' and ``F'' characters:

\begin{lstlisting}[style=custompy]
#!/usr/bin/env python
import random, sys

rt=""
for i in range(102400):
    if random.randint(0,1)==1:
        rt=rt+"T"
    else:
        rt=rt+"F"
print rt
\end{lstlisting}

Sample: \TT{...TTTFTFTTTFFFTTTFTTTTTTFTTFFTTTFTFTTFTTFFFFFF...}.

Entropy is very close to 1 (i.e., 1 bit per byte).

Let's generate random decimal digits:

\begin{lstlisting}[style=custompy]
#!/usr/bin/env python
import random, sys

rt=""
for i in range(102400):
    rt=rt+"%d" % random.randint(0,9)
print rt
\end{lstlisting}

Sample: \TT{...52203466119390328807552582367031963888032...}.

Entropy will be close to 3.32, indeed, this is $log_2(10)$.

