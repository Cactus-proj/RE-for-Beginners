\mysection{Fichier de sauvegarde du jeu Millenium}
\label{Millenium_DOS_game}
\myindex{MS-DOS}

\q{Millenium Return to Earth} est un ancien jeu DOS (1991), qui vous permet d'extraire
des ressources, de construire des vaisseaux, de les équiper et de les envoyer sur d'autres
planêtes, et ainsi de suite\footnote{Il peut être téléchargé librement \href{http://thehouseofgames.org/index.php?t=10&id=110}{ici}}.

Comme beaucoup d'autres jeux, il vous permet de sauvegarder l'état du jeu dans un fichier.

Regardons si l'on peut y trouver quelque chose.

\clearpage
Donc, il y a des mines dans le jeu.
Sur certaines planêtes, les mines rapportent plus vite, sur d'autres, moins vite. 
L'ensemble des ressources est aussi différent.

Ici, nous pouvons voir quelles ressources sont actuellement extraites.

\begin{figure}[H]
\centering
\myincludegraphics{ff/millenium/1.png}
\caption{Mine: état 1}
\label{fig:mill_1}
\end{figure}

Sauvegardons l'état du jeu.
C'est un fichier de 9538 octets.

Attendons quelques \q{jours} dans le jeu, et maintenant, nous avons plus de ressources
extraites des mines.

\begin{figure}[H]
\centering
\myincludegraphics{ff/millenium/2.png}
\caption{Mine: état 2}
\label{fig:mill_2}
\end{figure}

Sauvegardons à nouveau l'état du jeu.

Maintenant, essayons juste de comparer au niveau binaire les fichiers de sauvegarde
en utilisant le simple utilitaire DOS/Windows FC:

\lstinputlisting{ff/millenium/fc_result.txt}

La sortie est incomplète ici, il y a plus de différences, mais j'ai tronqué le résultat
pour montrer ce qu'il y a de plus intéressant.

Dans le premier état, nous avons 14 \q{unités} d'hydrogène et 102 \q{unités} d'oxygène.

Nous avons respectivement 22 et 155 \q{unités} dans le second état.
Si ces valeurs sont sauvées dans le fichier de sauvegarde, nous devrions les voir
dans la différence.
Et en effet, nous les voyons.
Il y a 0x0E (14) à la position 0xBDA et cette valeur est à 0x16 (22) dans la nouvelle
version du fichier.
Ceci est probablement l'hydrogène.
Il y a 0x66 (102) à la position 0xBDC dans la vieille version et x9B (155) dans la
nouvelle version du fichier.
Il semble que ça soit l'oxygène.

Les deux fichiers sont disponibles sur le site web pour ceux qui veulent les inspecter
(ou expérimenter) plus:
\href{http://beginners.re/examples/millenium_DOS_game/}{beginners.re}.

\clearpage
Voici la nouvelle version du fichier ouverte dans Hiew, j'ai marqué les valeurs relatives
aux ressources extraites dans le jeu:

\begin{figure}[H]
\centering
\myincludegraphics{ff/millenium/hiew3.png}
\caption{Hiew: état 1}
\label{fig:mill_hiew3}
\end{figure}

Vérifions chacune d'elles.

Ce sont clairement des valeurs 16-bits: ce n'est pas étonnant pour un logiciel DOS
16-bit où le type \Tint fait 16-bit.

\clearpage
Vérifions nos hypothèses.
Nous allons écrire la valeur 1234 (0x4D2) à la première position (ceci doit être
l'hydrogène):

\begin{figure}[H]
\centering
\myincludegraphics{ff/millenium/hiew4.png}
\caption{Hiew: écrivons 1234 (0x4D2) ici}
\label{fig:mill_hiew4}
\end{figure}

Puis nous chargeons le fichier modifié dans le jeu et jettons un coup d'\oe{}il aux
statistiques des mines:

\begin{figure}[H]
\centering
\myincludegraphics{ff/millenium/5.png}
\caption{Vérifions la valeur pour l'hydrogène}
\label{fig:mill_5}
\end{figure}

Donc oui, c'est bien ça.

\clearpage
Maintenant essayons de finir le jeu le plus vite possible, mettons les valeurs maximales
partout:

\begin{figure}[H]
\centering
\myincludegraphics{ff/millenium/hiew7.png}
\caption{Hiew: mettons les valeurs maximales}
\label{fig:mill_hiew7}
\end{figure}

0xFFFF représente 65535, donc oui, nous avons maintenant beaucoup de ressources:

\begin{figure}[H]
\centering
\myincludegraphics{ff/millenium/6.png}
\caption{Toutes les ressources sont en effet à 65535 (0xFFFF)}
\label{fig:mill_6}
\end{figure}

\clearpage
Laissons passer quelques \q{jours} dans le jeu et oups!
Nous avons un niveau plus bas pour quelques ressources:

\begin{figure}[H]
\centering
\myincludegraphics{ff/millenium/8.png}
\caption{Dépassement des variables de ressource}
\label{fig:mill_8}
\end{figure}

C'est juste un dépassement.

Le développeur du jeu n'a probablement pas pensé à un niveau aussi élevé de ressources,
donc il n'a pas dû mettre des tests de dépassement, mais les mines \q{travaillent}
dans le jeu, des ressources sont extraites, c'est pourquoi il y a des dépassements.
Apparemment, c'est une mauvaise idée d'être aussi avide.

Il y a sans doute beaucoup plus de valeurs sauvées dans ce fichier.

Ceci est donc une méthode très simple de tricher dans les jeux.
Les fichiers des meilleurs scores peuvent souvent être modifiés comme ceci.

Plus d'informations sur la comparaison des fichiers et des snapshots de mémoire:
\myref{snapshots_comparing}.
