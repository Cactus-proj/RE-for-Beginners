\mysection{\oracle: \EN{.SYM-files}\RU{.SYM-файлы}}
\myindex{\oracle}
\label{Oracle_SYM_files_example}

\RU{Когда процесс в \oracle терпит серьезную ошибку (crash), он записывает массу информации в лог-файлы,
включая состояние стека, вроде:}
\EN{When an \oracle process experiences some kind of crash, it writes a lot of information into log files,
including stack trace, like this:}

\begin{lstlisting}
----- Call Stack Trace -----
calling              call     entry                argument values in hex      
location             type     point                (? means dubious value)     
-------------------- -------- -------------------- ----------------------------
_kqvrow()                     00000000             
_opifch2()+2729      CALLptr  00000000             23D4B914 E47F264 1F19AE2
                                                   EB1C8A8 1
_kpoal8()+2832       CALLrel  _opifch2()           89 5 EB1CC74
_opiodr()+1248       CALLreg  00000000             5E 1C EB1F0A0
_ttcpip()+1051       CALLreg  00000000             5E 1C EB1F0A0 0
_opitsk()+1404       CALL???  00000000             C96C040 5E EB1F0A0 0 EB1ED30
                                                   EB1F1CC 53E52E 0 EB1F1F8
_opiino()+980        CALLrel  _opitsk()            0 0
_opiodr()+1248       CALLreg  00000000             3C 4 EB1FBF4
_opidrv()+1201       CALLrel  _opiodr()            3C 4 EB1FBF4 0
_sou2o()+55          CALLrel  _opidrv()            3C 4 EB1FBF4
_opimai_real()+124   CALLrel  _sou2o()             EB1FC04 3C 4 EB1FBF4
_opimai()+125        CALLrel  _opimai_real()       2 EB1FC2C
_OracleThreadStart@  CALLrel  _opimai()            2 EB1FF6C 7C88A7F4 EB1FC34 0
4()+830                                            EB1FD04
77E6481C             CALLreg  00000000             E41FF9C 0 0 E41FF9C 0 EB1FFC4
00000000             CALL???  00000000             
\end{lstlisting}

\RU{Но конечно, для этого исполняемые файлы \oracle должны содержать некоторую отладочную информацию,
либо map-файлы с информацией о символах или что-то в этом роде.}
\EN{But of course, \oracle's executables must have some kind of debug information or map files with symbol
information included or something like that.}

\RU{\oracle для Windows NT содержит информацию о символах в файлах с расширением .SYM, но его формат закрыт.}
\EN{Windows NT \oracle has symbol information in files with .SYM extension, but the format is proprietary.}
\EN{(Plain text files are good, but needs additional parsing, hence offer slower access.)}
\RU{(Простые текстовые файлы --- это хорошо, но они требуют дополнительной обработки (парсинга), и из-за этого доступ
к ним медленнее.)}

\RU{Посмотрим, сможем ли мы разобрать его формат}\EN{Let's see if we can understand its format}.
\RU{Выберем самый короткий файл \TT{orawtc8.sym}, поставляемый с файлом \TT{orawtc8.dll} в Oracle 8.1.7}%
\EN{We will pick the shortest \TT{orawtc8.sym} file that comes with the \TT{orawtc8.dll} file in Oracle 8.1.7}
\footnote{\RU{Будем использовать древнюю версию \oracle сознательно, из-за более короткого размера его модулей}%
\EN{We can chose an ancient \oracle version intentionally due to the smaller size of its modules}}.

\clearpage
\RU{Вот я открываю этот файл в}\EN{Here is the file opened in} Hiew:

\begin{figure}[H]
\centering
\myincludegraphics{ff/Oracle_SYM/whole1.png}
\caption{\RU{Весь файл в Hiew}\EN{The whole file in Hiew}}
\label{fig:oracle_SYM_whole1}
\end{figure}

\RU{Сравнивая этот файл с другими .SYM-файлами, мы можем быстро заметить, что \TT{OSYM} всегда является
заголовком (и концом), так что это, наверное, сигнатура файла.}
\EN{By comparing the file with other .SYM files, we can quickly see that \TT{OSYM} is always header (and footer),
so this is maybe the file's signature.}

\RU{Мы также видим, что в общем-то, формат файла это: OSYM + какие-то бинарные данные + 
текстовые строки разделенные нулем + OSYM.}
\EN{We also see that basically, the file format is: OSYM + some binary data + zero delimited text strings + OSYM.}
\RU{Строки --- это, очевидно, имена функций и глобальных переменных}\EN{The strings are, obviously, function and global 
variable names}.

\clearpage
\RU{Отметим сигнатуры OSYM и строки здесь}\EN{We will mark the OSYM signatures and strings here}: 

\begin{figure}[H]
\centering
\myincludegraphics{ff/Oracle_SYM/whole2.png}
\caption{\RU{Сигнатура OSYM и текстовые строки}\EN{OSYM signature and text strings}}
\label{fig:oracle_SYM_whole2}
\end{figure}

\RU{Посмотрим}\EN{Well, let's see}. 
\RU{В Hiew отметим весь блок со строками (исключая оконечивающую сигнатуру OSYM) и сохраним его в отдельный
файл.}%
\EN{In Hiew, we will mark the whole strings block (except the trailing OSYM signatures) and put it into a separate file.}
\RU{Затем запустим UNIX-утилиты \emph{strings} и \emph{wc} для подсчета текстовых строк:}%
\EN{Then we run UNIX \emph{strings} and \emph{wc} utilities to count the text strings:}

\begin{lstlisting}
strings strings_block | wc -l
66
\end{lstlisting}

\RU{Так что здесь 66 текстовых строк}\EN{So there are 66 text strings}.
\RU{Запомните это число}\EN{Please note that number}.

\RU{Можно сказать, что в общем, как правило, количество \emph{чего-либо} часто сохраняется в бинарном
файле отдельно.}
\EN{We can say, in general, as a rule, the number of \emph{anything} is often stored separately in binary files.}
\RU{Это действительно так, мы можем найти значение 66 (0x42) в самом начале файла, прямо после сигнатуры OSYM:}
\EN{It's indeed so, we can find the 66 value (0x42) at the file's start, right after the OSYM signature:}

\lstinputlisting{ff/Oracle_SYM/dump1.txt}

\RU{Конечно, 0x42 здесь это не байт, но скорее всего, 32-битное значение, запакованное как little-endian,
поэтому мы видим 0x42 и затем как минимум 3 байта.}
\EN{Of course, 0x42 here is not a byte, but most likely a 32-bit value packed as little-endian, hence we see
0x42 and then at least 3 zero bytes.}

\RU{Почему мы полагаем, что оно 32-битное}\EN{Why do we believe it's 32-bit}?
\RU{Потому что файлы с символами в \oracle могут быть очень большими}\EN{Because, \oracle's symbol 
files may be pretty big}.
\RU{oracle.sym для главного исполняемого файла oracle.exe (версия 10.2.0.4) содержит \TT{0x3A38E} (238478) 
символов.}
\EN{The oracle.sym file for the main oracle.exe (version 10.2.0.4) executable contains \TT{0x3A38E} (238478) symbols.}
\RU{16-битного значения тут недостаточно}\EN{A 16-bit value isn't enough here}.

\RU{Проверим другие .SYM-файлы как этот и это подтвердит нашу догадку: значение после 32-битной сигнатуры OSYM
всегда отражает количество текстовых строк в файле.}%
\EN{We can check other .SYM files like this and it proves our guess: the value after the 32-bit OSYM signature always
reflects the number of text strings in the file.}

\RU{Это общая особенность почти всех бинарных файлов: заголовок с сигнатурой плюс некоторая дополнительная
информация о файле.}
\EN{It's a general feature of almost all binary files: a header with a signature plus some other information 
about the file.}

\RU{Рассмотрим бинарный блок поближе}\EN{Now let's investigate closer what this binary block is}.
\RU{Снова используя Hiew, сохраним блок начиная с адреса 8 (т.е. после 32-битного значения,
отражающего количество) до блока со строками, в отдельный файл.}%
\EN{Using Hiew again, we put the block starting at address 8 (i.e., after the 32-bit \emph{count} value) 
ending at the strings block, into a separate binary file.}

\clearpage
\RU{Посмотрим этот блок в}\EN{Let's see the binary block in} Hiew:

\begin{figure}[H]
\centering
\myincludegraphics{ff/Oracle_SYM/binary1.png}
\caption{\RU{Бинарный блок}\EN{Binary block}}
\label{fig:oracle_SYM_binary1}
\end{figure}

\RU{Тут явно есть какая-то структура}\EN{There is a clear pattern in it}. 

\clearpage
\RU{Добавим красные линии, чтобы разделить блок}\EN{We will add red lines to divide the block}: 

\begin{figure}[H]
\centering
\myincludegraphics{ff/Oracle_SYM/binary2.png}
\caption{\RU{Структура бинарного блока}\EN{Binary block patterns}}
\label{fig:oracle_SYM_binary2}
\end{figure}

\RU{Hiew, как и многие другие шестнадцатеричные редакторы, показывает 16 байт на строку.}
\EN{Hiew, like almost any other hexadecimal editor, shows 16 bytes per line.}
\RU{Так что структура явно видна: здесь 4 32-битных значения на строку}\EN{So the pattern is clearly visible: 
there are 4 32-bit values per line}.

\RU{Эта структура видна визуально потому что некоторые значения здесь (вплоть до адреса \TT{0x104}) 
всегда в виде \TT{0x1000xxxx}, так что начинаются с байт 0x10 и 0.}
\EN{The pattern is visually visible because some values here (till address \TT{0x104}) 
are always in \TT{0x1000xxxx} form, 
started with 0x10 and zero bytes.}
\RU{Другие значения (начинающиеся на \TT{0x108}) всегда в виде \TT{0x0000xxxx}, так что начинаются с двух
нулевых байт.}
\EN{Other values (starting at \TT{0x108}) are in \TT{0x0000xxxx} form, so always started with two zero bytes.}

\RU{Посмотрим на этот блок как на массив 32-битных значений}\EN{Let's dump the block as an array of 32-bit values}:

\lstinputlisting[caption=\RU{первый столбец --- это адрес}\EN{first column is address}]{ff/Oracle_SYM/dump2.txt}

\RU{Здесь 132 значения, а это 66*2}\EN{There are 132 values, that's 66*2}.
\RU{Может быть здесь 2 32-битных значения на каждый символ, а может быть здесь два массива}\EN{Probably, 
there are two 32-bit values for each symbol, but maybe there are two arrays}? 
\RU{Посмотрим}\EN{Let's see}.

\RU{Значения, начинающиеся с}\EN{Values starting with} \TT{0x1000} \RU{могут быть адресами}\EN{may be addresses}.
\RU{В конце концов, этот .SYM-файл для DLL, а базовый адрес для DLL в win32 это \TT{0x10000000}, и сам код
обычно начинается по адресу \TT{0x10001000}.}
\EN{This is a .SYM file for a DLL after all, and the default base address of
win32 DLLs is \TT{0x10000000}, and the code usually starts at \TT{0x10001000}.}

\RU{Когда открываем файл orawtc8.dll в \IDA, базовый адрес другой, но тем не менее, первая функция это:}%
\EN{When we open the orawtc8.dll file in \IDA, the base address is different, but nevertheless, the first function is:}

\begin{lstlisting}
.text:60351000 sub_60351000    proc near
.text:60351000
.text:60351000 arg_0    = dword ptr  8
.text:60351000 arg_4    = dword ptr  0Ch
.text:60351000 arg_8    = dword ptr  10h
.text:60351000
.text:60351000          push    ebp
.text:60351001          mov     ebp, esp
.text:60351003          mov     eax, dword_60353014
.text:60351008          cmp     eax, 0FFFFFFFFh
.text:6035100B          jnz     short loc_6035104F
.text:6035100D          mov     ecx, hModule
.text:60351013          xor     eax, eax
.text:60351015          cmp     ecx, 0FFFFFFFFh
.text:60351018          mov     dword_60353014, eax
.text:6035101D          jnz     short loc_60351031
.text:6035101F          call    sub_603510F0
.text:60351024          mov     ecx, eax
.text:60351026          mov     eax, dword_60353014
.text:6035102B          mov     hModule, ecx
.text:60351031
.text:60351031 loc_60351031:    ; CODE XREF: sub_60351000+1D
.text:60351031          test    ecx, ecx
.text:60351033          jbe     short loc_6035104F
.text:60351035          push    offset ProcName ; "ax_reg"
.text:6035103A          push    ecx             ; hModule
.text:6035103B          call    ds:GetProcAddress
...
\end{lstlisting}

\RU{Ух ты}\EN{Wow}, \q{ax\_reg} \RU{звучит знакомо}\EN{string sounds familiar}. 
\RU{Действительно, это самая первая строка в блоке строк!}
\EN{It's indeed the first string in the strings block!}
\RU{Так что имя этой функции, похоже}\EN{So the name of this function seems to be} \q{ax\_reg}.

\RU{Вторая функция}\EN{The second function is}:

\begin{lstlisting}
.text:60351080 sub_60351080    proc near
.text:60351080
.text:60351080 arg_0    = dword ptr  8
.text:60351080 arg_4    = dword ptr  0Ch
.text:60351080
.text:60351080          push    ebp
.text:60351081          mov     ebp, esp
.text:60351083          mov     eax, dword_60353018
.text:60351088          cmp     eax, 0FFFFFFFFh
.text:6035108B          jnz     short loc_603510CF
.text:6035108D          mov     ecx, hModule
.text:60351093          xor     eax, eax
.text:60351095          cmp     ecx, 0FFFFFFFFh
.text:60351098          mov     dword_60353018, eax
.text:6035109D          jnz     short loc_603510B1
.text:6035109F          call    sub_603510F0
.text:603510A4          mov     ecx, eax
.text:603510A6          mov     eax, dword_60353018
.text:603510AB          mov     hModule, ecx
.text:603510B1
.text:603510B1 loc_603510B1:    ; CODE XREF: sub_60351080+1D
.text:603510B1          test    ecx, ecx
.text:603510B3          jbe     short loc_603510CF
.text:603510B5          push    offset aAx_unreg ; "ax_unreg"
.text:603510BA          push    ecx             ; hModule
.text:603510BB          call    ds:GetProcAddress
...
\end{lstlisting}

\RU{Строка \q{ax\_unreg} также это вторая строка в строке блок!}
\EN{The \q{ax\_unreg} string is also the second string in the strings block!}
\RU{Адрес начала второй функции это \TT{0x60351080}, а второе значение в бинарном блоке это \TT{10001080}.}
\EN{The starting address of the second function is \TT{0x60351080}, and the second value in the binary 
block is \TT{10001080}.}
\RU{Так что это адрес, но для DLL с базовым адресом по умолчанию}\EN{So this is the address, 
but for a DLL with the default base address}.

\RU{Мы можем быстро проверить и убедиться, что первые 66 значений в массиве (т.е. первая половина)
это просто адреса функций в DLL, включая некоторые метки, итд.}
\EN{We can quickly check and be sure that the first 66 values in the array (i.e., the first half of the array) 
are just function addresses in the DLL, including some labels, etc.}
\RU{Хорошо, что же тогда остальная часть массива}\EN{Well, what's the other part of array then}? 
\RU{Остальные 66 значений, начинающиеся с}\EN{The other 66 values that start with} \TT{0x0000}? 
\RU{Они похоже в пределах}\EN{These seem to be in range} \TT{[0...0x3F8]}. 
\RU{И не похоже, что это битовые поля: ряд чисел возрастает}\EN{And they do not look like bitfields: 
the series of numbers is increasing}.
\RU{Последняя шестнадцатеричная цифра выглядит как случайная, так что, не похоже, что это
адрес чего-либо (в противном случае, он бы делился, может быть, на 4 или 8 или 0x10).}
\EN{The last hexadecimal digit seems to be random, so, it's unlikely the address of something 
(it would be divisible by 4 or maybe 8 or 0x10 otherwise).}

\RU{Спросим себя: что еще разработчики \oracle хранили бы здесь, в этом файле?}
\EN{Let's ask ourselves: what else \oracle's developers would save here, in this file?}
\RU{Случайная догадка: это может быть адрес текстовой строки (название функции).}
\EN{Quick wild guess: it could be the address of the text string (function name).}
\RU{Это можно легко проверить, и да, каждое число --- это просто позиция первого символа в блоке строк.}
\EN{It can be quickly checked, and yes, each number is just the position of the first character in the strings block.}

\RU{Вот и всё! Всё закончено.}\EN{This is it! All done.}

\myindex{IDA}
\RU{Напишем утилиту для конвертирования .SYM-файлов в \IDA-скрипт, 
так что сможем загружать .idc-скрипт и он выставит имена функций:}%
\EN{We will write an utility to convert these .SYM files into \IDA script, 
so we can load the .idc script and it sets the function names:}

\lstinputlisting[style=customc]{ff/Oracle_SYM/unpacker.c}

\RU{Пример его работы}\EN{Here is an example of its work}:

\begin{lstlisting}
#include <idc.idc>

static main() {
	MakeName(0x60351000, "_ax_reg");
	MakeName(0x60351080, "_ax_unreg");
	MakeName(0x603510F0, "_loaddll");
	MakeName(0x60351150, "_wtcsrin0");
	MakeName(0x60351160, "_wtcsrin");
	MakeName(0x603511C0, "_wtcsrfre");
	MakeName(0x603511D0, "_wtclkm");
	MakeName(0x60351370, "_wtcstu");
...
}
\end{lstlisting}

\RU{Файлы, использованные в этом примере, здесь}\EN{The example files were used in this example are here}: 
\href{http://beginners.re/examples/oracle/SYM/}{beginners.re}.

\clearpage
\RU{О, можно еще попробовать \oracle для win64}\EN{Oh, let's also try \oracle for win64}.
\RU{Там ведь должны быть 64-битные адреса, верно}\EN{There has to be 64-bit addresses instead, right}?

\RU{8-байтная структура здесь видна даже еще лучше}\EN{The 8-byte pattern is visible even easier here}:

\begin{figure}[H]
\centering
\myincludegraphics{ff/Oracle_SYM/whole64.png}
\caption{\RU{пример .SYM-файла из \oracle для win64}\EN{.SYM-file example from \oracle for win64}}
\label{fig:oracle_SYM_whole64}
\end{figure}

\RU{Так что да, все таблицы здесь имеют 64-битные элементы, даже смещения строк!}
\EN{So yes, all tables now have 64-bit elements, even string offsets!}
\RU{Сигнатура теперь \TT{OSYMAM64}, чтобы отличить целевую платформу, очевидно.}%
\EN{The signature is now \TT{OSYMAM64}, to distinguish the target platform, apparently.}

\RU{Вот и всё}\EN{This is it}!
\RU{Вот также библиотека в которой есть функция для доступа к .SYM-файлам \oracle}
\EN{Here is also library which has functions to access \oracle .SYM-files}:
\href{https://github.com/DennisYurichev/porg/blob/master/lib/oracle_sym.c}{GitHub}.
