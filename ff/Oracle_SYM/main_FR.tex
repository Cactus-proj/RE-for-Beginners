\mysection{\oracle: fichiers .SYM}
\myindex{\oracle}
\label{Oracle_SYM_files_example}

Lorsqu'un processus d'\oracle rencontre un sorte de crash, il écrit beaucoup d'information
dans les fichiers de trace, incluant la trace de la pile, comme ceci:

\begin{lstlisting}
----- Call Stack Trace -----
calling              call     entry                argument values in hex      
location             type     point                (? means dubious value)     
-------------------- -------- -------------------- ----------------------------
_kqvrow()                     00000000             
_opifch2()+2729      CALLptr  00000000             23D4B914 E47F264 1F19AE2
                                                   EB1C8A8 1
_kpoal8()+2832       CALLrel  _opifch2()           89 5 EB1CC74
_opiodr()+1248       CALLreg  00000000             5E 1C EB1F0A0
_ttcpip()+1051       CALLreg  00000000             5E 1C EB1F0A0 0
_opitsk()+1404       CALL???  00000000             C96C040 5E EB1F0A0 0 EB1ED30
                                                   EB1F1CC 53E52E 0 EB1F1F8
_opiino()+980        CALLrel  _opitsk()            0 0
_opiodr()+1248       CALLreg  00000000             3C 4 EB1FBF4
_opidrv()+1201       CALLrel  _opiodr()            3C 4 EB1FBF4 0
_sou2o()+55          CALLrel  _opidrv()            3C 4 EB1FBF4
_opimai_real()+124   CALLrel  _sou2o()             EB1FC04 3C 4 EB1FBF4
_opimai()+125        CALLrel  _opimai_real()       2 EB1FC2C
_OracleThreadStart@  CALLrel  _opimai()            2 EB1FF6C 7C88A7F4 EB1FC34 0
4()+830                                            EB1FD04
77E6481C             CALLreg  00000000             E41FF9C 0 0 E41FF9C 0 EB1FFC4
00000000             CALL???  00000000             
\end{lstlisting}

Mais bien sûr, les exécutables d'\oracle doivent avoir une sorte d'information de
débogage ou de fichiers de carte avec l'information sur les symboles incluse ou
quelque chose comme ça.
% TODO: better translation for map file

\oracle sur Windows NT a l'information sur les symboles incluse dans des fichiers
avec l'extension .SYM, mais le format est propriétaire. (Les fichiers texte en clair
sont bons, mais nécessite une analyse supplémentaire, d'où un accès plus lent.)

Voyons si nous pouvons comprendre son format.

Nous allons choisir le plus petit fichier \TT{orawtc8.sym} qui vient avec le fichier
\TT{orawtc8.dll} dans Oracle 8.1.7\footnote{Nous pouvons choisir une version plus
ancienne d'\oracle intentionnellement à cause de la plus petite taille de ses modules.}.

\clearpage
Voici le fichier ouvert dans Hiew:

\begin{figure}[H]
\centering
\myincludegraphics{ff/Oracle_SYM/whole1.png}
\caption{Le fichier entier dans Hiew}
\label{fig:oracle_SYM_whole1}
\end{figure}

En comparant le fichier avec d'autres fichiers .SYM, nous voyons rapidement que
\TT{OSYM} est toujours en entête (et en fin de fichier), donc c'est peut-être la signature du fichier.

Nous voyons que, en gros, le format de fichier est: OSYM + des données binaires
+ un zéro délimiteur de chaîne de texte + OSYM.
Les chaînes sont évidemment les noms des fonctions et des variables globales.

\clearpage
Nous allons marquer les signatures OSYM et les chaînes ici:

\begin{figure}[H]
\centering
\myincludegraphics{ff/Oracle_SYM/whole2.png}
\caption{Signatures OSYM et chaînes de texte}
\label{fig:oracle_SYM_whole2}
\end{figure}

Bon, voyons.
Dans Hiew, nous allons marquer le bloc de chaînes complet (excepté les signatures
OSYM) et les mettre dans un fichier séparé.
Puis, nous lançons les utilitaires UNIX \emph{strings} et \emph{wc} pour compter
les chaînes de texte:

\begin{lstlisting}
strings strings_block | wc -l
66
\end{lstlisting}

Donc, il y a 66 chaînes de texte.
Veuillez noter ce nombre.

Nous pouvons dire, en général, comme une règle, que le nombre de \emph{quelque chose}
est souvent stocké séparément dans des fichiers binaires.

C'est en effet ainsi, nous pouvons trouver la valeur 66 (0x42) au début du fichier,
juste après la signature OSYM:

\lstinputlisting{ff/Oracle_SYM/dump1.txt}

Bien sûr, 0x42 n'est pas ici un octet, mais plus probablement une velaur 32-bit
packée en petit-boutiste, ainsi nous pouvons voir 0x42 et ensuite au moins 3 octets
à zéro.

Pourquoi croyons-nous que c'est 32-bit?
Car les fichiers de symboles d'\oracle peuvent être plutôt gros.

Le fichier oracle.sym pour l'exécutable principal oracle.exe (version 10.2.0.4)
contient \TT{0x3A38E} (238478) symboles.

Nous pouvons vérifier d'autres fichiers .SYM comme ceci et ça prouve notre supposition:
la valeur après la signature 32-bit OSYM représente toujours le nombre de chaînes
de texte dans le fichier.

C'est une caractéristique générale de presque tous les fichiers binaires: un entête
avec une signature ainsi que d'autres informations sur le fichier.

Maintenant, examinons de plus près ce qu'est ce bloc binaire.

En utilisant Hiew, nous extrayons le bloc débutant à l'offset 8 (i.e., après la valeur
32-bit \emph{count}) se terminant au bloc de chaînes, dans un fichier binaire
séparé.

\clearpage
Voyons le bloc binaire dans Hiew:

\begin{figure}[H]
\centering
\myincludegraphics{ff/Oracle_SYM/binary1.png}
\caption{Bloc binaire}
\label{fig:oracle_SYM_binary1}
\end{figure}

Il y un motif clair dedans.

\clearpage
Nous ajoutons des lignes rouges pour diviser le bloc:

\begin{figure}[H]
\centering
\myincludegraphics{ff/Oracle_SYM/binary2.png}
\caption{Schéma de bloc binaire}
\label{fig:oracle_SYM_binary2}
\end{figure}

Hiew, comme presque n'importe quel autre éditeur hexadécimal, montre 16 octets
par ligne.
Donc le motif est clairement visible:
il y a 4 valeurs 32-bit par ligne.

Le schéma est visible visuellement car certaines valeurs ici (jusqu'à l'adresse
\TT{0x104} sont toujours de la forme \TT{0x1000xxxx}, commençant par 0x10 et
des octets à zéro.

D'autres valeurs (commençant à \TT{0x108}) sont de la forme \TT{0x0000xxxx},
donc commencent toujours par deux octets à zéro.

Affichons le bloc comme un tableau de valeurs 32-bit:

\lstinputlisting[caption=la première colonne est l'adresse]{ff/Oracle_SYM/dump2.txt}

Il y a 132 valeurs, qui vaut 66*3.
Peut-être qu'il y a deux valeurs 32-bit pour chaque symbole, mais peut-être y
a-t-il deux tableaux?
Voyons

Les valeurs débutant par \TT{0x1000} peuvent être une adresse.

Ceci est un fichier .SYM pour une DLL après tout, et l'adresse de base par défaut
des DLL win32 est \TT{0x10000000}, et le code débute en général en \TT{0x10001000}.

Lorsque nous ouvrons le fichier orawtc8.dll dans \IDA, l'adresse de base est
différente, mais néanmoins, la première fonction est:

\begin{lstlisting}[style=customasmx86]
.text:60351000 sub_60351000    proc near
.text:60351000
.text:60351000 arg_0    = dword ptr  8
.text:60351000 arg_4    = dword ptr  0Ch
.text:60351000 arg_8    = dword ptr  10h
.text:60351000
.text:60351000          push    ebp
.text:60351001          mov     ebp, esp
.text:60351003          mov     eax, dword_60353014
.text:60351008          cmp     eax, 0FFFFFFFFh
.text:6035100B          jnz     short loc_6035104F
.text:6035100D          mov     ecx, hModule
.text:60351013          xor     eax, eax
.text:60351015          cmp     ecx, 0FFFFFFFFh
.text:60351018          mov     dword_60353014, eax
.text:6035101D          jnz     short loc_60351031
.text:6035101F          call    sub_603510F0
.text:60351024          mov     ecx, eax
.text:60351026          mov     eax, dword_60353014
.text:6035102B          mov     hModule, ecx
.text:60351031
.text:60351031 loc_60351031:    ; CODE XREF: sub\_60351000+1D
.text:60351031          test    ecx, ecx
.text:60351033          jbe     short loc_6035104F
.text:60351035          push    offset ProcName ; "ax\_reg"
.text:6035103A          push    ecx             ; hModule
.text:6035103B          call    ds:GetProcAddress
...
\end{lstlisting}

Ouah, la chaîne \q{ax\_reg} me dit quelque chose.

C'est en effet la première chaîne dans le bloc de chaîne!
Donc le nom de cette fonction semble être \q{ax\_reg}.

La seconde fonction est:

\begin{lstlisting}[style=customasmx86]
.text:60351080 sub_60351080    proc near
.text:60351080
.text:60351080 arg_0    = dword ptr  8
.text:60351080 arg_4    = dword ptr  0Ch
.text:60351080
.text:60351080          push    ebp
.text:60351081          mov     ebp, esp
.text:60351083          mov     eax, dword_60353018
.text:60351088          cmp     eax, 0FFFFFFFFh
.text:6035108B          jnz     short loc_603510CF
.text:6035108D          mov     ecx, hModule
.text:60351093          xor     eax, eax
.text:60351095          cmp     ecx, 0FFFFFFFFh
.text:60351098          mov     dword_60353018, eax
.text:6035109D          jnz     short loc_603510B1
.text:6035109F          call    sub_603510F0
.text:603510A4          mov     ecx, eax
.text:603510A6          mov     eax, dword_60353018
.text:603510AB          mov     hModule, ecx
.text:603510B1
.text:603510B1 loc_603510B1:    ; CODE XREF: sub\_60351080+1D
.text:603510B1          test    ecx, ecx
.text:603510B3          jbe     short loc_603510CF
.text:603510B5          push    offset aAx_unreg ; "ax\_unreg"
.text:603510BA          push    ecx             ; hModule
.text:603510BB          call    ds:GetProcAddress
...
\end{lstlisting}

La chaîne \q{ax\_unreg} est aussi la seconde chaîne dans le bloc de chaîne!

L'adresse de début de la seconde fonction est \TT{0x60351080}, et la seconde
valeur dans le bloc binaire est \TT{10001080}.
Donc ceci est l'adresse, mais pour une DLL avec l'adresse de base par défaut.

Nous pouvons rapidement vérifier et être sûr que les 66 premières valeurs dans
le tableau (i.e., la première moitié du tableau) sont simplement les adresses des
fonctions dans la DLL, incluant quelques labels, etc.
Bien, qu'est-ce que l'autre partie du tableau?
Les autres 66 valeurs qui commencent par \TT{0x0000}?
Elles semblent être dans l'intervalle \TT{[0...0x3F8]}.
Et elles ne ressemblent pas à des champs de bits:
la série de nombres augmente.

Le dernier chiffre hexadécimal semble être aléatoire, donc, il est peu probable
que ça soit l'adresse de quelque chose (il serait divisible par 4 ou peut-être
8 ou 0x10 autrement).

Demandons-nous: qu'est-ce que les développeurs d'\oracle pourraient avoir sauvegardé
ici, dans ce fichier?

Supposition rapide: ça pourrait être l'adresse de la chaîne de texte (nom de la
fonction).

Ça peut être vérifié rapidement, et oui, chaque nombre est simplement la position
du premier caractère dans le bloc de chaînes.

Ça y est! C'est fini.

\myindex{IDA}
Nous allons écrire un utilitaire pour convertir ces fichiers .SYM en un script
\IDA, donc nous pourrons charger le script .idc et mettre les noms de fonction:

\lstinputlisting[style=customc]{ff/Oracle_SYM/unpacker.c}

Voici un exemple de comment ça fonctionne:

\begin{lstlisting}[style=customc]
#include <idc.idc>

static main() {
	MakeName(0x60351000, "_ax_reg");
	MakeName(0x60351080, "_ax_unreg");
	MakeName(0x603510F0, "_loaddll");
	MakeName(0x60351150, "_wtcsrin0");
	MakeName(0x60351160, "_wtcsrin");
	MakeName(0x603511C0, "_wtcsrfre");
	MakeName(0x603511D0, "_wtclkm");
	MakeName(0x60351370, "_wtcstu");
...
}
\end{lstlisting}

Les fichiers d'exemple qui ont été utilisés dans cet exemple sont ici:
\href{http://go.yurichev.com/17216}{beginners.re}.

\clearpage
Oh, essayons avec \oracle pour win64.
Les adresses devraient faire 64-bit, n'est-ce pas?

Le motif sur 8 octets est visible encore plus facilement ici:

\begin{figure}[H]
\centering
\myincludegraphics{ff/Oracle_SYM/whole64.png}
\caption{Exemple de fichier .SYM d'\oracle pour win64}
\label{fig:oracle_SYM_whole64}
\end{figure}

Donc oui, toutes les tables ont maintenant des éléments 64-bit, même les offsets
de chaîne!

La signature est maintenant \TT{OSYMAM64}, pour distinguer la plate-forme cible,
apparemment.

C'est fini!

Voici aussi la bibliothèque qui a des fonctions pour accéder les fichiers .SYM
d'\oracle:
\href{http://go.yurichev.com/17007}{GitHub}.
