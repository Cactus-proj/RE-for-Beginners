% TODO resync with EN version
\chapter{Libros/blogs que merecen lectura}

\mysection{Libros}

\subsection{Reverse Engineering}

\input{RE_books}

% TBT

\subsection{Windows}

\begin{itemize}
\item \Russinovich
\item Peter Ferrie -- The ``Ultimate'' Anti-Debugging Reference\footnote\url{http://pferrie.host22.com/papers/antidebug.pdf}}
\end{itemize}

\EN{Blogs}\ES{Blogs}\RU{Блоги}\FR{Blogs}\DE{Blogs}\PL{Blogi}:

\begin{itemize}
\item \href{http://go.yurichev.com/17025}{Microsoft: Raymond Chen}
\item \href{http://go.yurichev.com/17026}{nynaeve.net}
\end{itemize}



\subsection{\CCpp}

\label{CCppBooks}

\begin{itemize}

\item \KRBook

\item \CNineNineStd\footnote{\AlsoAvailableAs \url{http://www.open-std.org/jtc1/sc22/WG14/www/docs/n1256.pdf}}

\item \TCPPPL

\item \CppOneOneStd\footnote{\AlsoAvailableAs \url{http://www.open-std.org/jtc1/sc22/wg21/docs/papers/2013/n3690.pdf}.}

\item \AgnerFogCPP\footnote{\AlsoAvailableAs \url{http://agner.org/optimize/optimizing_cpp.pdf}.}

\item \ParashiftCPPFAQ\footnote{\AlsoAvailableAs \url{http://www.parashift.com/c++-faq-lite/index.html}}

\item \CNotes\footnote{\AlsoAvailableAs \url{http://yurichev.com/C-book.html}}

\item JPL Institutional Coding Standard for the C Programming Language\footnote{\AlsoAvailableAs \url{https://yurichev.com/mirrors/C/JPL_Coding_Standard_C.pdf}}

\RU{\item Евгений Зуев --- Редкая профессия\footnote{\AlsoAvailableAs \url{https://yurichev.com/mirrors/C++/Redkaya_professiya.pdf}}}

\end{itemize}



\label{x86_manuals}
\begin{itemize}
\item Intel manuals\footnote{\AlsoAvailableAs \url{http://www.intel.com/content/www/us/en/processors/architectures-software-developer-manuals.html}}

\item AMD manuals\footnote{\AlsoAvailableAs \url{http://developer.amd.com/resources/developer-guides-manuals/}}

\item \AgnerFog{}\footnote{\AlsoAvailableAs \url{http://agner.org/optimize/microarchitecture.pdf}}

\item \AgnerFogCC{}\footnote{\AlsoAvailableAs \url{http://www.agner.org/optimize/calling_conventions.pdf}}

\item \IntelOptimization

\item \AMDOptimization
\end{itemize}

\subsection{ARM}

\begin{itemize}
\item Manuales de ARM\footnote{\AlsoAvailableAs \url{http://infocenter.arm.com/help/index.jsp?topic=/com.arm.doc.subset.architecture.reference/index.html}}

\item \ARMSevenRef

\item \ARMSixFourRefURL

\item \ARMCookBook\footnote{\AlsoAvailableAs \url{https://yurichev.com/ref/ARM%20Cookbook%20(1994)/}}
\end{itemize}

% TBT

\subsection{Java}

\JavaBook.

\subsection{UNIX}

\TAOUP

% subsection:
\input{crypto_reading}

\mysection{Otros}

\HenryWarren.

Existen dos excelentes subreddits relacionados con \ac{RE} en reddit.com:
\href{http://www.reddit.com/r/ReverseEngineering/}{reddit.com/r/ReverseEngineering/} \ESph{}
\href{http://www.reddit.com/r/remath}{reddit.com/r/remath}
(en los t\'opicos de la intersecci\'on de \ac{RE} y matem\'aticas).

Tambi\'en hay una secci\'on sobre \ac{RE} en el sitio web de Stack Exchange:

\par
\href{http://reverseengineering.stackexchange.com/}{reverseengineering.stackexchange.com}.

En IRC hay un canal \#\#re en
FreeNode\footnote{\href{https://freenode.net/}{freenode.net}}.

% TBT
