\subsection*{mini-FAQ}

\par Q: Czy ta książka jest prostsza niż inne?
\par A: Nie, poziom trudności jest mniej więcej taki sam jak innych książek na ten temat.

\par Q: Obawiam się zacząć czytać tę książkę, ma ponad 1000 stron.
"... dla początkujących" w nazwie brzmi nieco ironicznie.
\par A: Wszelkiego rodzaju kody źródłowe stanowią większość tej książki.
Ta książka naprawdę jest dla początkujących, wiele w niej (jeszcze) brakuje.

\par Q: Co trzeba wiedzieć zanim się przystąpi do czytania książki?
\par A: Umiejętności С/С++ są pożądane, ale nie są niezbędne.

\par Q: Czy powinienem uczyć się jednocześnie x86/x64/ARM i MIPS? Czy to nie za dużo?
\par A: Myślę, że na początek wystarczy czytać tylko o x86/x64, części o ARM i MIPS można pominąć.

\par Q: Czy można zakupić książki w wersji papierowej w języku rosyjskim lub angielskim?
\par A: Niestety nie, żaden wydawca jeszcze się nie zainteresował wydaniem rosyjskiej lub angielskiej wersji. Natomiast można ją wydrukować i zbindować w każdym ksero.
\url{https://yurichev.com/news/20200222_printed_RE4B/}.

\par Q: Czy istnieje wersja epub/mobi?
\par A: Nie. W wielu miejscach książka korzysta z hacków specyficznych dla TeXa/LaTeXa, dlatego przerobienie jej na HTML
(epub/mobi to jest HTML) nie jest łatwe.

\par Q: Po co uczyć się asemblera w dzisiejszych czasach?
\par A: Jeśli się nie jest developerem \ac{OS}, to prawdopodobnie nie trzeba pisać nic w asemblerze: współczesne kompilatory optymalizują kod lepiej niż człowiek \footnote{Bardzo ciekawy artykuł na ten temat: \InSqBrackets{\AgnerFog}}.

Do tego, współczesne \ac{CPU} są bardzo skomplikowanymi urządzeniami i znajomość asemblera nie pomoże poznać ich mechanizmów wewnętrznych.

Jednak zostają dwa obszary, w których dobra znajomość asemblera może być pomocna:
1) badanie malware (złośliwego oprogramowania) w celu jego analizy ; 2) lepsze zrozumienie skompilowanego kodu w trakcie debuggowania.

Wobec tego ta książka jest napisana dla tych ludzi, którzy raczej chcą rozumieć assembler, a nie w nim pisać. Stąd jest w niej bardzo dużo przykładów - wyjść kompilatora.

\par Q: Kliknąłem w odnośnik wewnątrz pliku PDF, jak teraz wrócić?
\par A: W Adobe Acrobat Reader trzeba wcisnąć Alt+LeftArrow. W Evince wcisnąć "<".

\par Q: Czy mogę wydrukować tę książkę? Korzystać z niej do nauczania?
\par A: Oczywiście, właśnie dlatego ta książka ma licencję Creative Commons (CC BY-SA 4.0).

\par Q: Dlaczego ta książka jest darmowa? Wykonałeś świetną robotę. To podejrzane, podobnie jak z innymi rzeczami za darmo.
\par A: Moim zdaniem autorzy literatury technicznej robią to dla autoreklamy. Taka praca nie przynosi za dużo pieniędzy.

\par Q: Jak znaleźć pracę w zawodzie reverse engineer-а?
\par A: Na reddit (RE\FNURLREDDIT), od czasu od czasu pojawiają się wątki poszukiwania pracowników.
Można spróbować tam poszukać.


\par Q: Wersje kompilatorów z tej książki są już przestarzałe...
\par A: Nie ma potrzeby by dokładnie wykonywać wszystkie kroki opisane w książce.
Użyj kompilatorów, które masz już zainstalowane w swoim \ac{OS}.
Poza tym, istnieje również: \href{https://godbolt.org/}{Compiler Explorer}.

\par Q: Mam pytanie...
\par A: Napisz do mnie maila (\EMAILS).

