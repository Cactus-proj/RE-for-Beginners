\clearpage
\subsection{Ensemble de Mandelbrot}
\label{Mandelbrot_demo}

\epigraph{Vous savez que si vous agrandissez la ligne du littoral, elle ressemble
toujours à une côte, et de nombreuses autres choses ont cette propriété. La nature
a des algorithmes récursifs qu'elle utilise pour générer les nuages, le fromage
Suisse et d'autres choses comme ça.}
{Donald Knuth, interview (1993)}

L'ensemble de Mandelbrot est un ensemble fractal, qui présente de l'auto-similarité.

Lorsque vous augmentez l'échelle, vous voyez que la forme principale se répète indéfiniment.

Voici une démo\footnote{Téléchargeable \href{http://www.pouet.net/prod.php?which=53287} {ici},}
écrite par \q{Sir\_Lagsalot} en 2009, qui dessine l'ensemble de Mandelbrot, qui est
juste un programme x86, dont l'exécutable a une taille de seulement 64 octets.
Il y a seulement 30 instructions x86 16-bit.

Voici ce qu'elle génère:

\begin{figure}[H]
\centering
\myincludegraphics{examples/demos/mandelbrot/1.png}
\end{figure}

Essayons de comprendre comment ça fonctionne.

\subsubsection{Théorie}

\myparagraph{Un mot à propos des nombres complexes}

Un nombre complexe est un nombre qui est constitué de deux parties---réelle (Re) et
imaginaire (Im).


Le plan complexe est un plan à deux dimensions où chaque nombre complexe peut être
placé: la partie réelle est une coordonnée, et la partie imaginaire est l'autre.

Quelques règles de base que nous devons garder à l'esprit:

\begin{itemize}
\item Addition: $(a+bi) + (c+di) = (a+c) + (b+d)i$

Autrement dit:

$\operatorname{Re}(sum) = \operatorname{Re}(a) + \operatorname{Re}(b)$

$\operatorname{Im}(sum) = \operatorname{Im}(a) + \operatorname{Im}(b)$

\item Multiplication: $(a+bi) (c+di) = (ac-bd) + (bc+ad)i$

Autrement dit:

$\operatorname{Re}(product) = \operatorname{Re}(a) \cdot \operatorname{Re}(c) - \operatorname{Re}(b) \cdot \operatorname{Re}(d)$

$\operatorname{Im}(product) = \operatorname{Im}(b) \cdot \operatorname{Im}(c) + \operatorname{Im}(a) \cdot \operatorname{Im}(d)$

\item Carré: $(a+bi)^2 = (a+bi) (a+bi) = (a^2-b^2) + (2ab)i$

Autrement dit:

$\operatorname{Re}(square) = \operatorname{Re}(a)^2-\operatorname{Im}(a)^2$

$\operatorname{Im}(square) = 2 \cdot \operatorname{Re}(a) \cdot \operatorname{Im}(a)$

\end{itemize}

\myparagraph{Comment dessiner l'ensemble de Mandelbrot?}

L'ensemble de Mandelbrot est l'ensemble des points pour lesquels la séquence récursive
$z_{n+1} = {z_n}^2 + c$ (où $z$ et $c$ sont des nombres complexes et $c$ est la valeur
de départ) ne tend pas vers l'infini.\\
\\
En langage courant:

\begin{itemize}
\item Parcourir tous les points de l'écran.
\item Vérifier si le point courant est dans l'ensemble de Mandelbrot.
\item Voici comment le vérifier:

  \begin{itemize}
  \item Représenter le point comme un nombre complexe.
  \item Calculer son carré.
  \item Lui ajouter la valeur initiale du point.
  \item Vérifier si le résultat dépasse les limites. Si oui, arrêter.
  \item Déplacer le point à la coordonnées que nous venons de calculer.
  \item Répéter tout ceci pour un nombre raisonnable d'itérations.
  \end{itemize}

\item Le point est-il toujours dans les limites? Alors dessiner le point.

\item Le point a-t-il dépassé les limites?

  \begin{itemize}
    \item (pour une image en noir et blanc) ne rien dessiner.
    \item 
(pour une image en couleur) transformer le nombre d'itération en une couleur.
     De façon à ce que la couleur montre la vitesse à laquelle le point est sorti
des limites.
  \end{itemize}

\end{itemize}

%
Voici un algorithme en Python pour les représentations en nombres complexes et entiers:

\lstinputlisting[caption=Pour les nombres complexes]{examples/demos/mandelbrot/algo_cplx_FR.lst}


La version avec les entiers est celle où les opérations sur les nombres complexes
sont remplacées par des opérations sur des entiers, en suivant les règles qui ont
été expliquées plus haut.

\lstinputlisting[caption=Pour les nombres entiers]{examples/demos/mandelbrot/algo_int_FR.lst}

Voici aussi un source en C\# qui se trouve dans l'article\footnote{\href{http://ru.wikipedia.org/wiki/Множество_Мандельброта}{Wikipédia}}
de Wikipédia, mais nous allons le modifier afin qu'il affiche le nombre d'itérations au lieu d'un symbole
\footnote{Voici aussi le fichier exécutable: 
\href{http://beginners.re/examples/mandelbrot/dump_iterations.exe}{beginners.re}}:

\lstinputlisting[style=customc]{examples/demos/mandelbrot/dump_iterations.cs}

Voici le fichier résultant, qui est trop large pour être inclus ici:\\
\href{http://beginners.re/examples/mandelbrot/result.txt}{beginners.re}.

Le nombre maximal d'itérations est 40, donc lorsque vous voyez 40 dans ce dump, ça
signifie que ce point s'est baladé pendant 40 itérations, sans sortir des limites.

Un nombre $n$ inférieur à 40 signifie que le point est resté dans les limites pendant
$n$ itérations, puis en est sorti.

\clearpage
Il y a une démo cool disponible ici \url{http://demonstrations.wolfram.com/MandelbrotSetDoodle/}, qui montre
visuellement comment le point se déplace dans le plan à chaque itération pour un
point spécifique.
Voici deux copies d'écran.

%
Premièrement, j'ai cliqué dans la zone jaune et vu que la trajectoire (ligne verte)
fini par tourner autour d'un point à l'intérieur:

\begin{figure}[H]
\centering
\includegraphics[width=0.7\textwidth]{examples/demos/mandelbrot/demo1.png}
\caption{Click à l'intérieur de la surface jaune}
\end{figure}


Ceci implique que le point que nous avons cliqué appartient à l'ensemble de Mandelbrot.

\clearpage

Puis j'ai cliqué en dehors de la surface jaune et vu un mouvement du point bien plus
chaotique, qui sort rapidement des limites:

\begin{figure}[H]
\centering
\includegraphics[width=0.7\textwidth]{examples/demos/mandelbrot/demo2.png}
\caption{Click en dehors de la surface jaune}
\end{figure}

Ceci signifie que le point n'appartient pas à l'ensemble de Mandelbrot.

Une autre démo est disponible ici:
\url{http://demonstrations.wolfram.com/IteratesForTheMandelbrotSet/}.

\clearpage
\subsubsection{Revenons à la démo}


La démo, bien que minuscule (seulement 64 octets ou 30 instructions), implémente
l'algorithme standard décrit ici, mais utilise des astuces de programmation.

%
Le code source est facile à télé-charger, donc le voici, mais j'ai ajouté des commentaires:

\lstinputlisting[caption=Code source commenté,numbers=left,style=customasmx86]{examples/demos/mandelbrot/Microbrot_commented_FR.asm}

Algorithme:

\begin{itemize}
\item Change le mode vidéo à VGA 320*200 VGA, 256 couleurs. 
$320*200=64000$ (0xFA00). 

Chaque pixel est encodé par un octet, donc la taille du buffer est de 0xFA00 octets.
Il est accédé en utilisant la paire de registres ES:DI.

\myindex{x86!\Registers!ES}
ES doit être 0xA000 ici, car c'est l'adresse du segment du buffer vidéo VGA, mais
mettre 0xA000 dans ES nécessite au moins 4 octets (\TT{PUSH 0A000h / POP ES}).
Plus d'information sur le modèle de mémoire 16-bit de MS-DOS ici:
\myref{8086_memory_model}.

\myindex{x86!\Instructions!LES}

En supposant que BX vaut zéro ici, et que le Program Segment Prefix est à l'adresse
zéro, l'instruction en 2 octets \TT{LES AX,[BX]} stocke 0x20CD dans AX et 0x9FFF
dans ES.

Donc le programme commence à écrire 16 pixels (ou octets) avant le buffer vidéo.
Mais c'est MS-DOS,

Il n'y a pas de protection de la mémoire, donc l'écriture se produit tout à la fin
de la mémoire conventionnelle, et en général, il n'y a rien d'important.
C'est pourquoi vous voyez une bande rouge de 16 pixels de large sur le côté droit.
L'image complète est décalée à gauche de 16 pixels.
C'est le prix pour économiser 2 octets.

\item Une boucle infinie traite chaque pixel.

La façon la plus courante de parcourir tous les pixels de l'écran est d'utiliser deux boucles:
une pour la coordonnée X, une autre pour la coordonnée Y.
Mais vous devrez multiplier une coordonnée pour accéder à un octet dans le buffer vidéo VGA.

L'auteur de cette démo a décidé de faire autrement: énumérer tous les octets dans
le buffer vidéo en utilisant une seule boucle au lieu de deux, et obtenir les coordonnées
du point courant en utilisant une division
Les coordonnées résultantes sont: X dans l'intervalle $-256..63$ et Y dans l'intervalle
$-100..99$.
Vous pouvez voir sur la copie d'écran que l'image est quelque peu décalée sur la
droite de l'écran.

C'est parce que la surface en forme de c\oe{}ur apparaît en général aux coordonnées
0,0 et elles sont décalées vers la droite.
Est-ce que l'auteur aurait pu soustraire 160 de la valeur pour avoir X dans l'intervalle
$-160..159$?
Oui, mais l'instruction \TT{SUB DX, 160} nécessite 4 octets, tandis que
\TT{DEC DH}---2 octets (ce qui soustrait 0x100 (256) de DX).
Donc l'ensemble de l'image est décalée pour économiser 2 octets de code.

    \begin{itemize}
    \item Vérifier si le point courant
est dans l'ensemble de Mandelbrot.
          L'algorithme est celui qui a été décrit ici.
\myindex{x86!\Instructions!LOOP}
     \item La boucle 
est organisée en utilisant l'instruction \TT{LOOP}, qui utilise le registre CX comme compteur.

L'auteur pourrait mettre le nombre d'itérations à une valeur spécifique, mais il
ne l'a pas fait: 320 est déjà présent dans CX (il a été chargé à la ligne 35), et
de toutes façons une bonne valeur d'itération maximale.
Nous économisons encore un peu d'espace ici en ne rechargeant pas le registre CX
avec une autre valeur.

\myindex{x86!\Instructions!SAR}
     \item 
\TT{IMUL} est utilisé ici au lieu de \TT{MUL}, car nous travaillons avec des valeurs
signées: gardez à l'esprit que les coordonnées 0,0 doivent être proches du centre de
l'écran.

C'est la même chose avec \TT{SAR} (décalage arithmétique pour des valeurs signées):
c'est utilisé au lieu de \TT{SHR}.

     \item Une autre idée est de simplifier le test des limites.
Nous devons tester une paire de coordonnées, i.e., deux variables.
Ce que je fais est de vérifier trois fois le débordement: deux opérations de mise
au carré et une addition.  %% FIXME

En effet, nous utilisons des registres 16-bit, qui peuvent contenir des valeurs signées
dans l'intervalle -32768..32767, donc si l'une des coordonnées est plus grande que
32767 lors de la multiplication signée, ce point est définitivement hors limites:
nous sautons au label \TT{MandelBreak}.

     \item 
Il y a aussi une division par 64 (instruction SAR). 64 met à l'échelle.

Il est possible d'augmenter la valeur pour regarder de plus près, ou de la diminuer
pour voir de plus loin.

    \end{itemize}

\item Nous sommes au label \TT{MandelBreak}, il y a deux façons d'arriver ici:
la boucle s'est terminée avec CX=0 (le point est à l'intérieur de l'ensemble de Mandelbrot);
ou parce qu'un débordement s'est produit (CX contient toujours une valeur non zéro.)
Nous écrivons la partie 8-bit basse de CX (CL) dans le buffer vidéo.

La palette par défaut est grossière, néanmoins, 0 est noir: ainsi nous voyons du noir
aux endroits où les points sont dans l'ensemble de Mandelbrot.
La palette peut être initialisée au début du programme, mais gardez à l'esprit que
ceci est un programme de seulement 64 octets!

\item le programme tourne en boucle infinie, car un test supplémentaire, ou une interface
utilisateur quelconque nécessiterait des instructions supplémentaires.

\end{itemize}

D'autres astuces d'optimisation:

\myindex{x86!\Instructions!CWD}
\begin{itemize}
\item L'instruction en 1-octet CWD est utilisée ici
pour effacer DX au lieu de celle en 2-octets \TT{XOR DX, DX} ou même celle en 3-octets \TT{MOV DX, 0}.

\item L'instruction en 1-octet \TT{XCHG AX, CX} est utilisée ici au lieu de celle en 2-octets
\TT{MOV AX,CX}. 
La valeur courante de AX n'est plus nécessaire ici.

\item DI (position dans le buffer vidéo) n'est pas initialisée, et contient 0xFFFE au début
\footnote{Plus d'information sur la valeur initiale des registres:
\url{https://code.google.com/p/corkami/wiki/InitialValues#DOS}}.

C'est OK, car le programme travaille pour tout DI dans l'intervalle 0..0xFFFF éternellement,
et l'utilisateur ne peut pas remarquer qu'il a commencé en dehors de l'écran (le
dernier pixel d'un buffer vidéo de 320*200 est à l'adresse 0xF9FF).
Donc du travail est en fait effectué en dehors des limites de l'écran.

Autrement, il faudrait une instruction supplémentaire pour mettre DI à 0 et tester
la fin du buffer vidéo.

\end{itemize}

\newcommand{\MyFixedVersion}{Ma version \q{corrigée}}
\subsubsection{\MyFixedVersion}

\lstinputlisting[caption=\MyFixedVersion,numbers=left,style=customasmx86]{examples/demos/mandelbrot/my_version_FR.asm}


J'ai essayé de corriger toutes ces bizarreries: maintenant la palette est un dégradé
de gris, le buffer vidéo est à la bonne place (lignes 19..20),
l'image est dessinée au centre de l'écran (ligne 30), le programme se termine et
attend qu'une touche soit pressée (lignes 58..68).

Mais maintenant c'est bien plus gros: 105 octets (ou 54 instructions)
\footnote{
Vous pouvez tester par vous-même: prenez DosBox et NASM et compilez-le avec:
\TT{nasm file.asm -fbin -o file.com}}.

\begin{figure}[H]
\centering
\myincludegraphics{examples/demos/mandelbrot/fixed.png}
\caption{\MyFixedVersion}
\label{fig:mandelbrot_fixed}
\end{figure}

Voir aussi: petit programme C qui affiche l'ensemble de Mandelbrot en ASCII:
\url{https://people.sc.fsu.edu/~jburkardt/c_src/mandelbrot_ascii/mandelbrot_ascii.html} \\
\url{https://miyuki.github.io/2017/10/04/gcc-archaeology-1.html}.
