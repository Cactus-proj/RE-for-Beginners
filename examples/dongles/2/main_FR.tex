\subsection{Exemple \#2: SCO OpenServer}

\label{examples_SCO}
\myindex{SCO OpenServer}
Un ancien logiciel pour SCO OpenServer de 1997 développé par une société qui a disparue
depuis longtemps.

Il y a un driver de dongle special à installer dans le système, qui contient les
chaînes de texte suivantes:
\q{Copyright 1989, Rainbow Technologies, Inc., Irvine, CA}
et
\q{Sentinel Integrated Driver Ver. 3.0 }.

Après l'installation du driver dans SCO OpenServer, ces fichiers apparaissent dans
l'arborescence /dev:

\begin{lstlisting}
/dev/rbsl8
/dev/rbsl9
/dev/rbsl10
\end{lstlisting}

Le programme renvoie une erreur lorsque le dongle n'est pas connecté, mais le message
d'erreur n'est pas trouvé dans les exécutables.

\myindex{COFF}

Grâce à \ac{IDA}, il est facile de charger l'exécutable COFF utilisé dans SCO OpenServer.

Essayons de trouver la chaîne \q{rbsl} et en effet, elle se trouve dans ce morceau
de code:

\lstinputlisting[style=customasmx86]{examples/dongles/2/1.lst}

Oui, en effet, le programme doit communiquer d'une façon ou d'une autre avec le driver.

\myindex{thunk-functions}
Le seul endroit où la fonction \TT{SSQC()} est appelée est dans la \glslink{thunk
 function}{fonction thunk}:

\lstinputlisting[style=customasmx86]{examples/dongles/2/2.lst}

SSQ() peut être appelé depuis au moins 2 fonctions.

L'une d'entre elles est:

\lstinputlisting[style=customasmx86]{examples/dongles/2/check1_EN.lst}

\q{\TT{3C}} et \q{\TT{3E}} semblent familiers: il y avait un dongle Sentinel Pro de
Rainbow sans mémoire, fournissant seulement une fonction de crypto-hachage secrète.

Vous pouvez lire une courte description de la fonction de hachage dont il s'agit
ici: \myref{hash_func}.

Mais retournons au programme.

Donc le programme peut seulement tester si un dongle est connecté ou s'il est absent.

Aucune autre information ne peut être écrite dans un tel dongle, puisqu'il n'a pas
de mémoire.
Les codes sur deux caractères sont des commandes (nous pouvons voir comment les commandes
sont traitées dans la fonction \TT{SSQC()}) et toutes les autres chaînes sont hachées
dans le dongle, transformées en un nombre 16-bit.
L'algorithme était secret, donc il n'était pas possible d'écrire un driver de remplacement
ou de refaire un dongle matériel qui l'émulerait parfaitement.

Toutefois, il est toujours possible d'intercepter tous les accès au dongle et de
trouver les constantes auxquelles les résultats de la fonction de hachage sont comparées.

Mais nous devons dire qu'il est possible de construire un schéma de logiciel de protection
de copie robuste basé sur une fonction secrète de hachage cryptographique: il suffit
qu'elle chiffre/déchiffre les fichiers de données utilisés par votre logiciel.

Mais retournons au code:

Les codes 51/52/53 sont utilisés pour choisir le port imprimante LPT.
3x/4x sont utilisés pour le choix de la \q{famille} (c'est ainsi que les dongles
Sentinel Pro sont différenciés les uns des autres: plus d'un dongle peut être connecté
sur un port LPT).

La seule chaîne passée à la fonction qui ne fasse pas 2 caractères est "0123456789".

Ensuite, le résultat est comparé à l'ensemble des résultats valides.

Si il est correct, 0xC ou 0xB est écrit dans la variable globale \TT{ctl\_model}.%

Une autre chaîne de texte qui est passée est
"PRESS ANY KEY TO CONTINUE: ", mais le résultat n'est pas testé.
Difficile de dire pourquoi, probablement une erreur\footnote{C'est un sentiment
étrange de trouver un bug dans un logiciel aussi ancien.}.

Voyons où la valeur de la variable globale \TT{ctl\_model} est utilisée.

Un tel endroit est:

\lstinputlisting[style=customasmx86]{examples/dongles/2/4.lst}

Si c'est 0, un message d'erreur chiffré est passé à une routine de déchiffrement
et affiché.

\myindex{x86!\Instructions!XOR}

La routine de déchiffrement de la chaîne semble être un simple \glslink{xoring}{xor}:

\lstinputlisting[style=customasmx86]{examples/dongles/2/err_warn.lst}

C'est pourquoi nous étions incapable de trouver le message d'erreur dans les fichiers
exécutable, car ils sont chiffrés (ce qui est une pratique courante).

Un autre appel à la fonction de hachage \TT{SSQ()} lui passe la chaîne \q{offln}
et le résultat est comparé avec \TT{0xFE81} et \TT{0x12A9}.

Si ils ne correspondent pas, ça se comporte comme une sorte de fonction \TT{timer()}
(peut-être en attente qu'un dongle mal connecté soit reconnecté et re-testé?) et ensuite
déchiffre un autre message d'erreur à afficher.

\lstinputlisting[style=customasmx86]{examples/dongles/2/check2_EN.lst}

Passer outre le dongle est assez facile: il suffit de patcher tous les sauts après
les instructions \CMP pertinentes.

Une autre option est d'écrire notre propre driver SCO OpenServer, contenant une table
de questions et de réponses, toutes celles qui sont présentent dans le programme.

\subsubsection{Déchiffrer les messages d'erreur}

À propos, nous pouvons aussi essayer de déchiffrer tous les messages d'erreurs.
L'algorithme qui se trouve dans la fonction \TT{err\_warn()} est très simple, en effet:

\lstinputlisting[caption=Decryption function,style=customasmx86]{examples/dongles/2/decrypting_FR.lst}

Comme on le voit, non seulement la chaîne est transmise à la fonction de déchiffrement
mais aussi la clef:

\lstinputlisting[style=customasmx86]{examples/dongles/2/tmp1_EN.asm}

L'algorithme est un simple \glslink{xoring}{xor}: chaque octet est xoré avec la clef, mais
la clef est incrémentée de 3 après le traitement de chaque octet.

Nous pouvons écrire un petit script Python pour vérifier notre hypothèse:

\lstinputlisting[caption=Python 3.x]{examples/dongles/2/decr1.py}

Et il affiche: \q{check security device connection}.
Donc oui, ceci est le message déchiffré.

Il y a d'autres messages chiffrés, avec leur clef correspondante.
Mais inutile de dire qu'il est possible de les déchiffrer sans leur clef.
Premièrement, nous voyons que le clef est en fait un octet.
C'est parce que l'instruction principale de déchiffrement (\XOR) fonctionne au niveau
de l'octet.
La clef se trouve dans le registre \ESI, mais seulement une partie de \ESI d'un octet
est utilisée.
Ainsi, une clef pourrait être plus grande que 255, mais sa valeur est toujours arrondie.

En conséquence, nous pouvons simplement essayer de brute-forcer, en essayant toutes
les clefs possible dans l'intervalle 0..255.
Nous allons aussi écarter les messages comportants des caractères non-imprimable.

\lstinputlisting[caption=Python 3.x]{examples/dongles/2/decr2.py}

Et nous obtenons:

\lstinputlisting[caption=Results]{examples/dongles/2/decr2_result.txt}

Ici il y a un peu de déchet, mais nous pouvons rapidement trouver les messages en
anglais.

À propos, puisque l'algorithme est un simple chiffrement xor, la même fonction peut
être utilisée pour chiffrer les messages.
Si besoin, nous pouvons chiffrer nos propres messages, et patcher le programme en les insérant.
