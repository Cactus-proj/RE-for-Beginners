\newcommand{\PPC}{\InSqBrackets{\emph{PowerPC(tm) Microprocessor Family: The Programming Environments for 32-Bit Microprocessors}, (2000)}\footnote{\AlsoAvailableAs \url{http://yurichev.com/mirrors/PowerPC/6xx_pem.pdf}}}

\newcommand{\PPCABI}{[Steve Zucker, SunSoft and Kari Karhi, IBM, \emph{SYSTEM V APPLICATION BINARY INTERFACE: PowerPC Processor Supplement}, (1995)]\footnote{\AlsoAvailableAs \url{http://yurichev.com/mirrors/PowerPC/elfspec_ppc.pdf}}}

\subsection{Exemple \#1: MacOS Classic et PowerPC}

\myindex{PowerPC}
\myindex{Mac OS Classic}
Il y a ici un exemple de programme pour MacOS Classic\footnote{pre-UNIX MacOS}, pour PowerPC.
La société qui a développé le logiciel a disparu il y a longtemps, donc le client
(légal) avait peur d'un problème matériel sur le dongle.

Lorsqu'on le lançait sans le dongle connecté, une boite de dialogue avec le message
"Invalid Security Device" apparaissait.

Par chance, cette chaîne de texte pût facilement être trouvée dans le fichier exécutable binaire.

Prétendons que nous ne sommes pas très familier, à la fois avec le Mac OS Classic et le PowerPC, mais essayons tout de même.

\ac{IDA} 
ouvre le fichier exécutable sans problème, indique son type comme
"PEF (Mac OS or Be OS executable)" (en effet, c'est un format de fichier Mac OS Classic
standard).

En cherchant la chaîne de texte avec le message d'erreur, nous trouvons ce morceau
de code:

\lstinputlisting[style=customasmPPC]{examples/dongles/1/1.lst}

\myindex{ARM}
\myindex{MIPS}
Oui, ceci est du code PowerPC.

Le CPU est un \ac{RISC} 32-bit très typique des années 1990s.

Chaque instruction occupe 4 octets (tout comme MIPS et ARM) et les noms ressemblent
quelque peu aux noms des instructions MIPS.

\TT{check1()} est une fonction donc nous allons donner le nom plu tard.
\TT{BL} est l'instruction \emph{Branch Link}, e.g., destinée à appeler des sous-programmes.

Le point crucial est l'instruction \ac{BNE} qui saute si la vérification du dongle
de protection passe ou pas si une erreur se produit: alors l'adresse de la chaîne
de texte est chargée dans le registre r3 pour la passer à la routine de la boite de
dialogue subséquente.

Dans \PPCABI nous trouvons que le registre r3 es utilisé pour la valeur de retour
(et r4, dans le cas de valeurs 64-bit).

\myindex{x86!\Instructions!MOVZX}
Une autre instruction inconnue est \TT{CLRLWI}.
Dans \PPC nous apprenons que cette instruction effectue la mise à zéro et le chargement.
Dans notre cas, elle efface les 24 bits haut de la valeur dans r3 et la met dans
r0, donc elle est analogue à \MOVZX en x86 (\myref{movzx}), mais elle met aussi les
flags, donc \ac{BNE} peut ensuite les tester.

Jetons un \oe{}il à la fonction \TT{check1()}:

\lstinputlisting[style=customasmPPC]{examples/dongles/1/check1.lst}

Comme on peut le voir dans \ac{IDA}, cette fonction est appelée depuis de nombreux
points du programme, mais seule la valeur du registre r3 est testée après chaque
appel.
\myindex{thunk-functions}

Tout ce que fait cette fonction est d'appeler une autre fonction, donc c'est une
\glslink{thunk function}{fonction thunk}:
il y a un prologue et un épilogue de fonction, mais le registre r3 n'est pas touché,
donc \TT{checkl()} renvoie ce que \TT{check2()} renvoie.

\ac{BLR} semble être le retour de la fonction, mais vu comment \ac{IDA} dispose la
fonction, nous ne devons probablement pas nous en occuper.

Puisque c'est un \ac{RISC} typique, il semble que les sous-programmes soient appelés
en utilisant un \glslink{link register}{registre de lien}, tout comme en ARM.

La fonction \TT{check2()} est plus complexe:

\lstinputlisting[style=customasmPPC]{examples/dongles/1/check2.lst}

\myindex{USB}

Nous sommes encore chanceux: quelques noms de foncions ont été laissés dans l'exécutable
(section de symboles de débogage?\\
difficile à dire puisque nous ne sommes pas très familier de ce format de fichier,
peut-être est-ce une sorte d'export PE? (\ref{PE_exports_imports})),\\ % due to \vref bug
comme \TT{.RBEFINDNEXT()} et \TT{.RBEFINDFIRST()}.

Enfin ces fonctions appellent d'autres fonctions avec des noms comme \TT{.GetNextDeviceViaUSB()},
\TT{.USBSendPKT()}, donc elles travaillent clairement avec un dispositif USB.

Il y a même une fonction appelée \TT{.GetNextEve3Device()}---qui semble familière,
il y avait un dongle Sentinel Eve3 pour le port ADB (présent sue les MACs) dans les 1990s.

Regardons d'abord comment le registre r3 est mis avant le retour, en ignorant tout
le reste.

Nous savons qu'une \q{bonne} valeur de r3 doit être non-nulle, r3 à zéro conduit
à l'exécution de la partie avec un message d'erreur dans une boite de dialogue.

Il y a deux instructions \TT{li \%r3, 1} présentes dans la fonction et une \TT{li \%r3, 0}
(\emph{Load Immediate}, i.e., charger une valeur dans un registre).
La première instruction est en \TT{0x001186B0}---et franchement, il est difficile
de dire ce qu'elle signifie.

Ce que l'on voit ensuite, toutefois, est plus facile à comprendre:
\TT{.RBEFINDFIRST()} est appelé:
si elle échoue, 0 est écrit dans r3 et nous sautons à \emph{exit}, autrement une
autre fonction est appelée(\TT{check3()})---si elle échoue aussi, \TT{.RBEFINDNEXT()}
est appelée, probablement afin de chercher un autre dispositif USB.

N.B.: \TT{clrlwi. \%r0, \%r3, 16} est analogue à ce que nous avons déjà vu, mais
elle éfface 16 bits, i.e., \\
\TT{.RBEFINDFIRST()}
renvoie probablement une valeur 16-bit.

\TT{B} (signifie \emph{branch}) saut inconditionnel.

\ac{BEQ} est l'instruction inverse de \ac{BNE}.

Regardons \TT{check3()}:

\lstinputlisting[style=customasmPPC]{examples/dongles/1/check3.lst}

Il y a de nombreux appels à \TT{.RBEREAD()}.

Peut-être que la fonction renvoie une valeur du dongle, donc elles sont comparées
ici avec des variables codées en dur en utilisant \TT{CMPLWI}.

Nous voyons aussi que le registre r3 est aussi rempli avant chaque appel à \TT{.RBEREAD()}
avec une de ces valeurs: 0,1, 8, 0xA, 0xB, 0xC, 0xD, 4, 5.
Probablement une adresse mémoire ou quelque chose comme ça?

Oui, en effet, en googlant ces noms de fonction il est facile de trouver le manuel
du dongle Sentinel Eve3!

Peut-être n'avons nous pas besoin d'apprendre aucune autre instruction PowerPC: tout
ce que fait cette fonction est seulement d'appeler \TT{.RBEREAD()}, de comparer ses
résultats avec les constantes et de renvoyer 1 si les comparaisons sont justes ou
0 autrement.

Ok, tout ce dont nous avons besoin est que la fonction \TT{check1()} renvoie toujours
1 ou n'importe quelle valeur autre que zéro.

Mais puisque nous ne sommes pas très sûrs de nos connaissances des instructions PowerPC,
nous allons être prudents: nous allons patcher le saut dans \TT{check2()} en \TT{0x001186FC}
et en \TT{0x00118718}.

En \TT{0x001186FC} nous allons écrire les octets 0x48 et 0, convertissant ainsi l'instruction
\ac{BEQ} en un \TT{B} (saut inconditionnel): nous pouvons repérer son opcode sans
même nous référer à \PPC.

En \TT{0x00118718} nous allons écrire  0x60 et 3 octets à zéro, la convertissant
ainsi en une instruction \ac{NOP}:
Nous pouvons aussi repérer son opcode dans le code.

Et maintenant, tout fonctionne sans un dongle connecté.

En résumé, des petites modification telles que celles-ci peuvent être effectuées
avec \ac{IDA} et un minimum de connaissances en langage d'assemblage.
