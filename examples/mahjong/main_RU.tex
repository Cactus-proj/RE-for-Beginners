\section{Шутка с Маджонгом (Windows 7)}

Маджонг это замечательная игра, однако, можем ли мы усложнить её, запретив пункт меню \emph{Hint} (\emph{подсказка})?

В Windows 7, я могу найти файлы Mahjong.dll и Mahjong.exe в:\\
\verb|C:\Windows\winsxs\| \\
\verb|x86_microsoft-windows-s..inboxgames-shanghai_31bf3856ad364e35_6.1.7600.16385_none\| \\
\verb|c07a51d9507d9398|.

Также, файл \verb|Mahjong.exe.mui| в:\\
\verb|C:\Windows\winsxs\| \\
\verb|x86_microsoft-windows-s..-shanghai.resources_31bf3856ad364e35_6.1.7600.16385_en-us| \\
\verb|_c430954533c66bf3|

и в

\verb|C:\Windows\winsxs\| \\
\verb|x86_microsoft-windows-s..-shanghai.resources_31bf3856ad364e35_6.1.7600.16385_ru-ru| \\
\verb|_0d51acf984cb679a|.

Я использую англоязычную Windows, но с поддержкой русского языка, так что тут могут быть файлы ресурсов для двух языков.
\myindex{Resource Hacker}
Открыв файл Mahjong.exe.mui в Resource Hacker, можно увидеть определения меню:

\begin{lstlisting}[caption=Ресурсы меню в Mahjong.exe.mui]
103 MENU
LANGUAGE LANG_ENGLISH, SUBLANG_ENGLISH_US
{
  POPUP "&Game"
  {
    MENUITEM "&New Game\tF2",  40000
    MENUITEM SEPARATOR
    MENUITEM "&Undo\tCtrl+Z",  40001
    MENUITEM "&Hint\tH",  40002
    MENUITEM SEPARATOR
    MENUITEM "&Statistics\tF4",  40003
    MENUITEM "&Options\tF5",  40004
    MENUITEM "Change &Appearance\tF7",  40005
    MENUITEM SEPARATOR
    MENUITEM "E&xit",  40006
  }
  POPUP "&Help"
  {
    MENUITEM "&View Help\tF1",  40015
    MENUITEM "&About Mahjong Titans",  40016
    MENUITEM SEPARATOR
    MENUITEM "Get &More Games Online",  40020
  }
}
\end{lstlisting}

Подменю \emph{Hint} имеет код 40002.
Открываю Mahjong.exe в IDA и ищу значение 40002.

(Я пишу это в ноябре 2019. Почему-то, IDA не может выкачать PDB-файлы с серверов Microsoft. Может быть, Windows 7 больше не поддерживается?
Так или иначе, имен ф-ций я не смог установить...)

\lstinputlisting[style=customasmx86,caption=Mahjong.exe]{examples/mahjong/1_RU.lst}

Эта часть кода разрешает или запрещает пункт меню \emph{Hint}.

\myindex{Windows!EnableMenuItem}
И согласно MSDN\footnote{\url{https://docs.microsoft.com/en-us/windows/win32/api/winuser/nf-winuser-enablemenuitem}}:

\verb$MF_DISABLED | MF_GRAYED = 3$ и \verb|MF_ENABLED = 0|.

Я думаю, эта ф-ция разрешает или запрещает несколько пунктов меню (\emph{Hint}, \emph{Undo}, итд), исходя из значения в \verb|arg_0|.
Потому что при старте, когда пользователь выбирает тип игры, пункты \verb|Hint| и \verb|Undo| запрещены.
Они разрешены, когда игра начинается.

Так что я изменяю в файле Mahjong.exe по адресу 0x01020637 байт 0x75 на 0xEB, делая так, что переход \INS{JNZ} всегда будет работать.
Эффект в том, что ф-ция всегда будет вызываться как \verb|EnableMenuItem(..., ..., 3)|.
Теперь подменю \emph{Hint} всё время запрещено.

Также, почему-то, ф-ция \verb|EnableMenuItem()| вызывается дважды, для \verb|hMenu| и для \verb|hmenu|.
Может быть, в программе два меню, и может быть, они переключаются?

В качестве домашней работы, попробуйте запретить подменю \emph{Undo}, чтобы сделать игру еще труднее.
