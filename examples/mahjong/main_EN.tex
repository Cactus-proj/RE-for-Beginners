\section{Mahjong solitaire prank (Windows 7)}

Mahjong solitaire is a great game, but can we make it harder, by disabling the \emph{Hint} menu item?

In my Windows 7, I can find Mahjong.dll and Mahjong.exe in:\\
\verb|C:\Windows\winsxs\| \\
\verb|x86_microsoft-windows-s..inboxgames-shanghai_31bf3856ad364e35_6.1.7600.16385_none\| \\
\verb|c07a51d9507d9398|.

Also, the \verb|Mahjong.exe.mui| file in:\\
\verb|C:\Windows\winsxs\| \\
\verb|x86_microsoft-windows-s..-shanghai.resources_31bf3856ad364e35_6.1.7600.16385_en-us| \\
\verb|_c430954533c66bf3|

and

\verb|C:\Windows\winsxs\| \\
\verb|x86_microsoft-windows-s..-shanghai.resources_31bf3856ad364e35_6.1.7600.16385_ru-ru| \\
\verb|_0d51acf984cb679a|.

I'm using English Windows, but with Russian language support, so there are might be resource files for two languages.
\myindex{Resource Hacker}
By opening Mahjong.exe.mui in Resource Hacker, there we can see a menu definition:

\begin{lstlisting}[caption=Menu resources from Mahjong.exe.mui]
103 MENU
LANGUAGE LANG_ENGLISH, SUBLANG_ENGLISH_US
{
  POPUP "&Game"
  {
    MENUITEM "&New Game\tF2",  40000
    MENUITEM SEPARATOR
    MENUITEM "&Undo\tCtrl+Z",  40001
    MENUITEM "&Hint\tH",  40002
    MENUITEM SEPARATOR
    MENUITEM "&Statistics\tF4",  40003
    MENUITEM "&Options\tF5",  40004
    MENUITEM "Change &Appearance\tF7",  40005
    MENUITEM SEPARATOR
    MENUITEM "E&xit",  40006
  }
  POPUP "&Help"
  {
    MENUITEM "&View Help\tF1",  40015
    MENUITEM "&About Mahjong Titans",  40016
    MENUITEM SEPARATOR
    MENUITEM "Get &More Games Online",  40020
  }
}
\end{lstlisting}

The \emph{Hint} submenu has the 40002 code.
Now I'm opening Mahjong.exe in IDA and find the 40002 value.

(I'm writing this in November 2019. Somehow, IDA can't get PDBs from Microsoft servers. Maybe Windows 7 is unsupported anymore?
Anyway, I can't get function names...)

\lstinputlisting[style=customasmx86,caption=Mahjong.exe]{examples/mahjong/1.lst}

This piece of code enables or disables the \emph{Hint} menu item.

\myindex{Windows!EnableMenuItem}
And according to MSDN\footnote{\url{https://docs.microsoft.com/en-us/windows/win32/api/winuser/nf-winuser-enablemenuitem}}:

\verb$MF_DISABLED | MF_GRAYED = 3$ and \verb|MF_ENABLED = 0|.

I think, this function enables or disables several menu items (\emph{Hint}, \emph{Undo}, etc), according to the value in \verb|arg_0|.
Because at start, when a user choose solitaire type, \verb|Hint| and \verb|Undo| are disabled.
They are enabled when the game has begun.

So I'm patching the Mahjong.exe file at 0x01020637 by replacing 0x75 with 0xEB byte, making this \INS{JNZ} jump working always.
Effectively, this will make calling \verb|EnableMenuItem(..., ..., 3)| always.
Now the \emph{Hint} submenu is always disabled.

Also, somehow, \verb|EnableMenuItem()| called twice, for \verb|hMenu| and for \verb|hmenu|.
Perhaps, the program has two menus, and maybe switching them?

As a homework, try to disable \emph{Undo} menu item, to make the game even harder.
