\section{Blague avec le solitaire Mahjong (Windows 7)}

Le solitaire Mahjong est un bon jeu, mais pouvons-nous le rendre plus difficile,
en désactivant l'élément de menu \emph{Hint}?

Dans mon Windows 7, je peux trouver Mahjong.dll et Mahjong.exe dans:\\
\verb|C:\Windows\winsxs\| \\
\verb|x86_microsoft-windows-s..inboxgames-shanghai_31bf3856ad364e35_6.1.7600.16385_none\| \\
\verb|c07a51d9507d9398|.

Aussi le fichier \verb|Mahjong.exe.mui| dans:\\
\verb|C:\Windows\winsxs\| \\
\verb|x86_microsoft-windows-s..-shanghai.resources_31bf3856ad364e35_6.1.7600.16385_en-us| \\
\verb|_c430954533c66bf3|

et

\verb|C:\Windows\winsxs\| \\
\verb|x86_microsoft-windows-s..-shanghai.resources_31bf3856ad364e35_6.1.7600.16385_ru-ru| \\
\verb|_0d51acf984cb679a|.

J'utilise Windows en anglais, mais avec le support du langage russe, donc il peut
y avoir des fichiers de ressource pour ces deux langages.
\myindex{Resource Hacker}
En ouvrant Mahjong.exe.mui dans Resource Hacker, nous pouvons y voir une définition
de menu:

\begin{lstlisting}[caption=Ressource de menu de Mahjong.exe.mui]
103 MENU
LANGUAGE LANG_ENGLISH, SUBLANG_ENGLISH_US
{
  POPUP "&Game"
  {
    MENUITEM "&New Game\tF2",  40000
    MENUITEM SEPARATOR
    MENUITEM "&Undo\tCtrl+Z",  40001
    MENUITEM "&Hint\tH",  40002
    MENUITEM SEPARATOR
    MENUITEM "&Statistics\tF4",  40003
    MENUITEM "&Options\tF5",  40004
    MENUITEM "Change &Appearance\tF7",  40005
    MENUITEM SEPARATOR
    MENUITEM "E&xit",  40006
  }
  POPUP "&Help"
  {
    MENUITEM "&View Help\tF1",  40015
    MENUITEM "&About Mahjong Titans",  40016
    MENUITEM SEPARATOR
    MENUITEM "Get &More Games Online",  40020
  }
}
\end{lstlisting}

Le sous-menu \emph{Hint} a le code 40002.
Maintenant, j'ouvre Mahjong.exe dans IDA et trouve la valeur 40002.

(J'écris ceci en novembre 2019. Il semble qu'IDA ne puisse pas obtenir les PDBs
depuis les serveurs de Microsoft. Peut-être que Windows 7 n'est plus supporté?
En tout cas, je ne peux pas obtenir les noms de fonction...)

\begin{lstlisting}[style=customasmx86,caption=Mahjong.exe]
.text:010205C8 6A 03                             push    3
.text:010205CA 85 FF                             test    edi, edi
.text:010205CC 5B                                pop     ebx

...

.text:01020625 57                                push    edi             ; uIDEnableItem
.text:01020626 FF 35 C8 97 08 01                 push    hmenu           ; hMenu
.text:0102062C FF D6                             call    esi ; EnableMenuItem
.text:0102062E 83 7D 08 01                       cmp     [ebp+arg_0], 1
.text:01020632 BF 42 9C 00 00                    mov     edi, 40002
.text:01020637 75 18                             jnz     short loc_1020651 ; must be always
.text:01020639 6A 00                             push    0               ; uEnable
.text:0102063B 57                                push    edi             ; uIDEnableItem
.text:0102063C FF 35 B4 8B 08 01                 push    hMenu           ; hMenu
.text:01020642 FF D6                             call    esi ; EnableMenuItem
.text:01020644 6A 00                             push    0               ; uEnable
.text:01020646 57                                push    edi             ; uIDEnableItem
.text:01020647 FF 35 C8 97 08 01                 push    hmenu           ; hMenu
.text:0102064D FF D6                             call    esi ; EnableMenuItem
.text:0102064F EB 1A                             jmp     short loc_102066B
.text:01020651                   ; ---------------------------------------------------------------------------
.text:01020651
.text:01020651                   loc_1020651:                            ; CODE XREF: sub\_1020581+B6
.text:01020651 53                                push    ebx             ; 3
.text:01020652 57                                push    edi             ; uIDEnableItem
.text:01020653 FF 35 B4 8B 08 01                 push    hMenu           ; hMenu
.text:01020659 FF D6                             call    esi ; EnableMenuItem
.text:0102065B 53                                push    ebx             ; 3
.text:0102065C 57                                push    edi             ; uIDEnableItem
.text:0102065D FF 35 C8 97 08 01                 push    hmenu           ; hMenu
.text:01020663 FF D6                             call    esi ; EnableMenuItem
\end{lstlisting}

Ce morceau de code active ou désactive l'élément de menu \emph{Hint}.

\myindex{Windows!EnableMenuItem}
Et d'après MSDN\footnote{\url{https://docs.microsoft.com/en-us/windows/win32/api/winuser/nf-winuser-enablemenuitem}}:

\verb$MF_DISABLED | MF_GRAYED = 3$ et \verb|MF_ENABLED = 0|.

Je pense que cette fonction active ou désactive plusieurs élément de menu (\emph{Hint},
\emph{Undo}, etc), suivant la valeur dans \verb|arg_0|.
Car au début, lorsqu'un utilisateur choisi le type de solitaire \verb|Hint| et
\verb|Undo| sont désactivés.
Ils sont activés lorsque le jeu a commencé.

Je modifie le fichier Mahjong.exe en 0x01020637, en remplaçant l'octet 0x75 avec
0xEB, rendant ce saut \INS{JNZ} toujours pris.
Pratiquement, ceci va toujours appeler \verb|EnableMenuItem(..., ..., 3)|.
Maintenant le sous-menu \emph{Hint} est toujours désactivé.

Aussi, de manière ou d'une autre, \verb|EnableMenuItem()| est appelé deux fois,
pour \verb|hMenu| et pour \verb|hmenu|.
Peut-être que le programme a deux menus, et peut-être les échange-t-il?

À titre d'exercice, essayez de désactiver l'élément de menu \emph{Undo}, pour rendre
le jeu encore plus difficile.
