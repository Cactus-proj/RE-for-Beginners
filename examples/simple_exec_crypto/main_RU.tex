\mysection{Взлом простого шифровальщика исполняемого кода}

У нас есть исполняемый файл, зашифрованный относительно простым шифрованием.
\href{\RepoURL/examples/simple_exec_crypto/files/cipher.bin}{Вот он} (здесь осталась только исполняемая секция).

Прежде всего, всё что делает ф-ция шифрования это прибавляет номер позиции в буфере к байту.
Вот как это можно описать:

\begin{lstlisting}[caption=Скрипт на Питоне,style=custompy]
#!/usr/bin/env python
def e(i, k):
    return chr ((ord(i)+k) % 256)

def encrypt(buf):
    return e(buf[0], 0)+ e(buf[1], 1)+ e(buf[2], 2) + e(buf[3], 3)+ e(buf[4], 4)+ e(buf[5], 5)+ e(buf[6], 6)+ e(buf[7], 7)+
           e(buf[8], 8)+ e(buf[9], 9)+ e(buf[10], 10)+ e(buf[11], 11)+ e(buf[12], 12)+ e(buf[13], 13)+ e(buf[14], 14)+ e(buf[15], 15)
\end{lstlisting}

Так, если вы зашифровываете буфер с 16-ю нулями, вы получаете \emph{0, 1, 2, 3 ... 12, 13, 14, 15}.

\myindex{Propagating Cipher Block Chaining}
Также здесь используется ``Propagating Cipher Block Chaining'' (PCBC) (режим распространяющегося сцепления блоков шифра),
и вот как он работает:

\begin{figure}[H]
\centering
\myincludegraphics{examples/simple_exec_crypto/601px-PCBC_encryption.png}
\caption{Шифрование в режиме распространяющегося сцепления блоков шифра (иллюстрация взята из Wikipedia)}
\end{figure}

Проблема восстановить \ac{IV}.
Полный перебор не подходит, потому что \ac{IV} слишком длинный (16 байт).
Посмотрим, сможем ли мы восстановить \ac{IV} для произвольного зашифрованного исполняемого файла?

Давайте попробуем простой частотный анализ.
Это 32-битный исполняемый x86 код, так что попробуем собрать статистику о наиболее частых байтах и опкодах.
Я пробовал огромный файл oracle.exe из Oracle RDBMS версии 11.2 для windows x86 и нашел, что самый часто встречающийся байт 
(и это не удивляет) это ноль (~10\%).
Другой часто встречающийся байт (снова, не удивительно) 0xFF (~5\%).
Следующий 0x8B (~5\%).

\myindex{x86!\Instructions!MOV}
0x8B это опкод инструкции \INS{MOV}, и это действительно одна из самых используемых инструкций в x86.
А что насчет популярности нулевого байта?
Если компилятору нужно закодировать значение более 127, он будет использовать 32-битные значения (displacement) вместо 8-битных,
но большие значения редки (\myref{MostPopularNumbers}), так что они дополняются нулями.
\myindex{x86!\Instructions!LEA}
\myindex{x86!\Instructions!PUSH}
\myindex{x86!\Instructions!CALL}
Это справедливо как минимум для \INS{LEA}, \INS{MOV}, \INS{PUSH}, \INS{CALL}.

Например:

\begin{lstlisting}[style=customasmx86]
8D B0 28 01 00 00                 lea     esi, [eax+128h]
8D BF 40 38 00 00                 lea     edi, [edi+3840h]
\end{lstlisting}

Смещения (displacements) более 127 очень популярны, но они редко превышают 0x10000
(действительно, настолько большие буфера/структуры также редки).

Та же история с \INS{MOV}, большие константы редки, наиболее часто используемые это 0, 1, 10, 100, $2^n$, итд.
Компилятору приходится дополнять маленькие константы нулями, чтобы представить их в виде 32-битных значений.

\begin{lstlisting}[style=customasmx86]
BF 02 00 00 00                    mov     edi, 2
BF 01 00 00 00                    mov     edi, 1
\end{lstlisting}

Теперь о 00 и FF вместе: переходы (включая условные) и вызовы могут переходить по адресам вперед или назад, но очень часто,
в пределах текущего исполняемого модуля.
Если вперед, то смещение (displacement) не очень большое и дополняется нулями.
Если назад, то смещение представляется отрицательным значением, так что дополняется байтами 0xFF.
Например, перейти вперед:

\begin{lstlisting}[style=customasmx86]
E8 43 0C 00 00                    call    _function1
E8 5C 00 00 00                    call    _function2
0F 84 F0 0A 00 00                 jz      loc_4F09A0
0F 84 EB 00 00 00                 jz      loc_4EFBB8
\end{lstlisting}

Назад:

\begin{lstlisting}[style=customasmx86]
E8 79 0C FE FF                    call    _function1
E8 F4 16 FF FF                    call    _function2
0F 84 F8 FB FF FF                 jz      loc_8212BC
0F 84 06 FD FF FF                 jz      loc_FF1E7D
\end{lstlisting}

Байт 0xFF часто встречается в отрицательных смещения вроде этих:

\begin{lstlisting}[style=customasmx86]
8D 85 1E FF FF FF                 lea     eax, [ebp-0E2h]
8D 95 F8 5C FF FF                 lea     edx, [ebp-0A308h]
\end{lstlisting}

Пока всё понятно. Нам теперь нужно пробовать разные 16-байтные ключи, дешифровать исполняемую секцию и измерять, как часто встречаются
байты 0, 0xFF и 0x8B.
Также будем держать на виду схему, как работает дешифрование в режиме PCBC:

\begin{figure}[H]
\centering
\myincludegraphics{examples/simple_exec_crypto/640px-PCBC_decryption.png}
\caption{Дешифрование в режиме распространяющегося сцепления блоков шифра (иллюстрация взята из Wikipedia)}
\end{figure}

Хорошие новости в том, что нам не нужно дешифровать весь кусок данных, но только ломтик за ломтиком, точно так же, как я это делал
в моем предыдущем примере: \myref{XOR_mask_2}.

Теперь я пробую все возможные байты (0..255) для каждого байта в ключе и просто выбираю тот байт, от которого будет максимальное число
байтов 0x0/0xFF/0x8B в дешифрованном ломтике:

% TODO translate to Russian:
\begin{lstlisting}[style=custompy]
#!/usr/bin/env python
import sys, hexdump, array, string, operator

KEY_LEN=16

def chunks(l, n):
    # split n by l-byte chunks
    # https://stackoverflow.com/q/312443
    n = max(1, n)
    return [l[i:i + n] for i in range(0, len(l), n)]

def read_file(fname):
    file=open(fname, mode='rb')
    content=file.read()
    file.close()
    return content

def decrypt_byte (c, key):
    return chr((ord(c)-key) % 256)

def XOR_PCBC_step (IV, buf, k):
    prev=IV
    rt=""
    for c in buf:
	new_c=decrypt_byte(c, k)
        plain=chr(ord(new_c)^ord(prev))
	prev=chr(ord(c)^ord(plain))
	rt=rt+plain
    return rt

each_Nth_byte=[""]*KEY_LEN

content=read_file(sys.argv[1])
# split input by 16-byte chunks:
all_chunks=chunks(content, KEY_LEN)
for c in all_chunks:
    for i in range(KEY_LEN):
        each_Nth_byte[i]=each_Nth_byte[i] + c[i]

# try each byte of key
for N in range(KEY_LEN):
    print "N=", N
    stat={}
    for i in range(256):
        tmp_key=chr(i)
	tmp=XOR_PCBC_step(tmp_key,each_Nth_byte[N], N)
        # count 0, FFs and 8Bs in decrypted buffer:
	important_bytes=tmp.count('\x00')+tmp.count('\xFF')+tmp.count('\x8B')
	stat[i]=important_bytes
    sorted_stat = sorted(stat.iteritems(), key=operator.itemgetter(1), reverse=True)
    print sorted_stat[0]
\end{lstlisting}

(Исходный код можно скачать \href{\RepoURL/examples/simple_exec_crypto/files/decrypt.py}{здесь}.)

Запускаю и вот ключ, для которого присутствие байт 0/0xFF/0x8B в дешифрованном буфере максимально:

\begin{lstlisting}
N= 0
(147, 1224)
N= 1
(94, 1327)
N= 2
(252, 1223)
N= 3
(218, 1266)
N= 4
(38, 1209)
N= 5
(192, 1378)
N= 6
(199, 1204)
N= 7
(213, 1332)
N= 8
(225, 1251)
N= 9
(112, 1223)
N= 10
(143, 1177)
N= 11
(108, 1286)
N= 12
(10, 1164)
N= 13
(3, 1271)
N= 14
(128, 1253)
N= 15
(232, 1330)
\end{lstlisting}

Напишем утилиту для дешифрования для полученного ключа:

% TODO translate to Russian:
\begin{lstlisting}[style=custompy]
#!/usr/bin/env python
import sys, hexdump, array

def xor_strings(s,t):
    # \verb|https://en.wikipedia.org/wiki/XOR_cipher#Example_implementation|
    """xor two strings together"""
    return "".join(chr(ord(a)^ord(b)) for a,b in zip(s,t))

IV=array.array('B', [147, 94, 252, 218, 38, 192, 199, 213, 225, 112, 143, 108, 10, 3, 128, 232]).tostring()

def chunks(l, n):
    n = max(1, n)
    return [l[i:i + n] for i in range(0, len(l), n)]

def read_file(fname):
    file=open(fname, mode='rb')
    content=file.read()
    file.close()
    return content

def decrypt_byte(i, k):
    return chr ((ord(i)-k) % 256)

def decrypt(buf):
    return "".join(decrypt_byte(buf[i], i) for i in range(16))

fout=open(sys.argv[2], mode='wb')

prev=IV
content=read_file(sys.argv[1])
tmp=chunks(content, 16)
for c in tmp:
    new_c=decrypt(c)
    p=xor_strings (new_c, prev)
    prev=xor_strings(c, p)
    fout.write(p)
fout.close()
\end{lstlisting}

(Исходный код можно скачать \href{\RepoURL/examples/simple_exec_crypto/files/decrypt2.py}{здесь}.)

Проверим итоговый файл:

\lstinputlisting{examples/simple_exec_crypto/objdump_result.txt}

Да, выглядит как корректно дизассемблированный кусок x86-кода.
Весь дешифрованный файл можно скачать \href{\RepoURL/examples/simple_exec_crypto/files/decrypted.bin}{здесь}.

На самом деле, это исполняемая секция из regedit.exe из Windows 7.
Но этот пример основан на реальном случае из моей практики, так что только исполняемый файл другой (и ключ), а алгоритм тот же.

\subsection{Еще идеи для рассмотрения}

Что если бы не получилось с простым частотным анализом?
Есть и другие идеи о том, как измерить корректность дешифрованного/разжатого x86-кода:

\begin{itemize}

\item Многие современные компиляторы выравнивают начало ф-ций по 16-байтной границе.
Так что пространство оставленное перед ними заполняется NOP-ами (0x90) или иными NOP-инструкциями с известными опкодами: \myref{sec:npad}.
Либо же инструкциями INT3 (0xCC).

\item Наверное, самый частый шаблонный код в ассемблере это вызов ф-ции:\\
\TT{PUSH chain / CALL / ADD ESP, X}.
Эту последовательность легко обнаруживать и находить.
Я даже собирал статистику о среднем количестве аргументов ф-ций: \myref{args_stat}.
(Т.е., это средняя длина цепи PUSH-инструкций.)

\end{itemize}

Читайте еще о некорректно/корректно дизассмеблированном коде: \myref{ISA_detect}.

