\subsection{Fonctions de vérification de mot de passe de SAP 6.0}

\myindex{SAP}

Lorsque je suis retourné sur SAP 6.0 IDES installé sur une machine VMware, je me
suis aperçu que j'avais oublié le mot de passe pour le compte SAP*, puis je m'en 
suis souvenu, mais j'ai alors eu ce message
\emph{<<Password logon no longer possible - too many failed attempts>>},
car j'ai fait trop de tentatives avant de m'en rappeler.

\myindex{Windows!PDB}

La première très bonne nouvelle fût que le fichier \gls{PDB} complet de \emph{disp+work.pdb}
était fourni avec SAP, et il contient presque tout: noms de fonction, structures,
types, variable locale et nom d'arguments, etc. Quel cadeau somptueux!

Il y a l'utilitaire TYPEINFODUMP\footnote{\url{http://www.debuginfo.com/tools/typeinfodump.html}} pour
convertir les fichiers \gls{PDB} en quelque chose de lisible et grepable.

Voici un exemple d'information d'une fonction + ses arguments + ses variables locales:

\lstinputlisting{examples/SAP/pw/1.txt}

Et voici un exemple d'une structure:

\lstinputlisting{examples/SAP/pw/2.txt}

Wow!

Une autre bonne nouvelle: les appels de \emph{debugging} (il y en a beaucoup) sont
très utiles.

Ici, vous pouvez remarquer la variable globale \emph{ct\_level}\footnote{Plus d'information
sur le niveau de trace: \url{http://help.sap.com/saphelp_nwpi71/helpdata/en/46/962416a5a613e8e10000000a155369/content.htm}}, qui reflète le niveau
actuel de trace.

Il y a beaucoup d'ajout de débogage dans le fichier \emph{disp+work.exe}:

\lstinputlisting[style=customasmx86]{examples/SAP/pw/3.asm}

Si le niveau courant de trace est plus élevé ou égal à la limite défini dans le code
ici, un message de débogage est écrit dans les fichiers de log comme \emph{dev\_w0},
\emph{dev\_disp}, et autres fichiers \emph{dev*}.

\myindex{\GrepUsage}

Essayons de grepper dans le fichier que nous avons obtenu à l'aide de l'utilitaire
TYPEINFODUMP:

\begin{lstlisting}
cat "disp+work.pdb.d" | grep FUNCTION | grep -i password
\end{lstlisting}

Nous obtenons:

\lstinputlisting{examples/SAP/pw/4.txt}

Essayons aussi de chercher des messages de debug qui contiennent les mots \emph{<<password>>}
et \emph{<<locked>>}.
L'un d'entre eux se trouve dans la chaîne \emph{<<user was locked by subsequently
failed password logon attempts>>}, référencé dans\\
la fonction \emph{password\_attempt\_limit\_exceeded()}.

D'autres chaînes que cette fonction peut écrire dans le fichier de log sont:
\emph{<<password logon attempt will be rejected immediately (preventing dictionary attacks)>>},
\emph{<<failed-logon lock: expired (but not removed due to 'read-only' operation)>>},
\emph{<<failed-logon lock: expired => removed>>}.

Après avoir joué un moment avec cette fonction, nous remarquons que le problème se
situe exactement dedans.
Elle est appelée depuis la fonction \emph{chckpass()}~---une des fonctions de vérification
u mot de passe.

d'abord, nous voulons être sûrs que nous sommes au bon endroit:

Lançons \tracer:
\myindex{tracer}

\begin{lstlisting}
tracer64.exe -a:disp+work.exe bpf=disp+work.exe!chckpass,args:3,unicode
\end{lstlisting}

\begin{lstlisting}
PID=2236|TID=2248|(0) disp+work.exe!chckpass (0x202c770, L"Brewered1                               ", 0x41) (called from 0x1402f1060 (disp+work.exe!usrexist+0x3c0))
PID=2236|TID=2248|(0) disp+work.exe!chckpass -> 0x35
\end{lstlisting}

L'enchaînement des appels est: \emph{syssigni()} -> \emph{DyISigni()} -> \emph{dychkusr()} -> \emph{usrexist()} -> \emph{chckpass()}.

Le nombre 0x35 est une erreur renvoyée dans \emph{chckpass()} à cet endroit:

\lstinputlisting[style=customasmx86]{examples/SAP/pw/5.asm}

Bien, vérifions:

\begin{lstlisting}
tracer64.exe -a:disp+work.exe bpf=disp+work.exe!password_attempt_limit_exceeded,args:4,unicode,rt:0
\end{lstlisting}

\begin{lstlisting}
PID=2744|TID=360|(0) disp+work.exe!password_attempt_limit_exceeded (0x202c770, 0, 0x257758, 0) (called from 0x1402ed58b (disp+work.exe!chckpass+0xeb))
PID=2744|TID=360|(0) disp+work.exe!password_attempt_limit_exceeded -> 1
PID=2744|TID=360|We modify return value (EAX/RAX) of this function to 0
PID=2744|TID=360|(0) disp+work.exe!password_attempt_limit_exceeded (0x202c770, 0, 0, 0) (called from 0x1402e9794 (disp+work.exe!chngpass+0xe4))
PID=2744|TID=360|(0) disp+work.exe!password_attempt_limit_exceeded -> 1
PID=2744|TID=360|We modify return value (EAX/RAX) of this function to 0
\end{lstlisting}

Excellent! Nous pouvons nous connecter avec succès maintenant.

À propos, nous pouvons prétendre que nous avons oublier le mot de passe, modifier
la fonction \emph{chckpass()} afin qu'elle renvoie toujours une valeur de 0, ce qui
est suffisant pour passer outre la vérification:

\begin{lstlisting}
tracer64.exe -a:disp+work.exe bpf=disp+work.exe!chckpass,args:3,unicode,rt:0
\end{lstlisting}

\begin{lstlisting}
PID=2744|TID=360|(0) disp+work.exe!chckpass (0x202c770, L"bogus                                   ", 0x41) (called from 0x1402f1060 (disp+work.exe!usrexist+0x3c0))
PID=2744|TID=360|(0) disp+work.exe!chckpass -> 0x35
PID=2744|TID=360|We modify return value (EAX/RAX) of this function to 0
\end{lstlisting}

Ce que l'on peut aussi dire en analysant la fonction\\
\emph{password\_attempt\_limit\_exceeded()},
c'est qu'à son tout début, on voit cet appel:

\lstinputlisting[style=customasmx86]{examples/SAP/pw/6.asm}

Étonnement, la fonction \emph{sapgparam()} est utilisée pour chercher la valeur de
certains paramètres de configuration. Cette fonction peut être appelée depuis 1768
endroits différents.
Il semble qu'avec l'aide de cette information, nous pouvons facilement trouver les
endroits dans le code, où le contrôle du flux est affecté par des configurations
spécifiques de paramètres.

C'est vraiment agréable. Le nom des fonctions est très clair, bien plus que dans \oracle.
\myindex{\Cpp}
Il semble que le processus \emph{disp+work} est écrit en \Cpp. A-t-il été récrit il y a quelques temps?
