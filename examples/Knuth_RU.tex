Вместо эпиграфа:

\begin{framed}
\begin{quotation}

\textbf{Питер Сейбел:} Как вы читаете исходный код? Ведь непросто читать даже то, что написано на известном вам языке программирования.

\textbf{Дональд Кнут:} Но это действительно того стоит, если говорить о том, что выстраивается в вашей голове. Как я читаю код? Когда-то была машина под названием Bunker Ramo 300, и кто-то мне однажды сказал, что компилятор Фортрана для этой машины работает чрезвычайно быстро, но никто не понимает почему. Я заполучил копию его исходного кода. У меня не было руководства по этому компьютеру, поэтому я даже не был уверен, какой это был машинный язык.

Но я взялся за это, посчитав интересной задачей. Я нашел BEGIN и начал разбираться. В кодах операций есть ряд двухбуквенных мнемоник, поэтому я мог начать анализировать: “Возможно, это инструкция загрузки, а это, возможно, инструкция перехода”. Кроме того, я знал, что это компилятор Фортрана, и иногда он обращался к седьмой колонке перфокарты - там он мог определить, комментарий это или нет.

Спустя три часа я кое-что понял об этом компьютере. Затем обнаружил огромные таблицы ветвлений. То есть это была своего рода головоломка, и я продолжал рисовать небольшие схемы, как разведчик, пытающийся разгадать секретный шифр. Но я знал, что программа работает, и знал, что это компилятор Фортрана — это не был шифр, в том смысле что программа не была написана с сознательной целью запутать. Все дело было в коде, поскольку у меня не было руководства по компьютеру.

В конце концов мне удалось выяснить, почему компилятор работал так быстро. К сожалению, дело было не в гениальных алгоритмах - просто там применялись методы неструктурированного программирования и код был максимально оптимизирован вручную.

По большому счету, именно так и должна решаться головоломка: составляются таблицы, схемы, информация извлекается по крупицам, выдвигается гипотеза. В общем, когда я читаю техническую работу, это такая же сложная задача. Я пытаюсь влезть в голову автора, понять, в чем состоял его замысел. Чем больше вы учитесь читать вещи, написанные другими, тем более способны изобретать что-то свое - так мне кажется. 

\end{quotation}
\end{framed}

( Сейбел Питер --- Кодеры за работой. Размышления о ремесле программиста )

