% TODO translate
% TODO: OpenSSL tool, URLs, etc
\mysection{Cas de base de données chiffrée \#1}
\label{encrypted_DB1}

(Cette partie est apparue initialement dans mon blog le 26 août 2015.
Discussion: \url{https://news.ycombinator.com/item?id=10128684}.)

\subsection{Base64 et entropie}

\myindex{XML}
J'ai un fichier \ac{XML} contenant des données chiffrées.
Peut-être est-ce relatif à des commandes et/ou des information clients.

\begin{lstlisting}
<?xml version = "1.0" encoding = "UTF-8"?>
<Orders>
	<Order>
		<OrderID>1</OrderID>
		<Data>yjmxhXUbhB/5MV45chPsXZWAJwIh1S0aD9lFn3XuJMSxJ3/E+UE3hsnH</Data>
	</Order>
	<Order>
		<OrderID>2</OrderID>
		<Data>0KGe/wnypFBjsy+U0C2P9fC5nDZP3XDZLMPCRaiBw9OjIk6Tu5U=</Data>
	</Order>
	<Order>
		<OrderID>3</OrderID>
		<Data>mqkXfdzvQKvEArdzh+zD9oETVGBFvcTBLs2ph1b5bYddExzp</Data>
	</Order>
	<Order>
		<OrderID>4</OrderID>
		<Data>FCx6JhIDqnESyT3HAepyE1BJ3cJd7wCk+APCRUeuNtZdpCvQ2MR/7kLXtfUHuA==</Data>
	</Order>
...
\end{lstlisting}

Le fichier est disponible \href{https://raw.githubusercontent.com/DennisYurichev/RE-for-beginners/master/examples/encrypted_DB1/encrypted.xml}{ici}.

\myindex{base64}
Ce sont clairement des données encodées en base64, car toutes les chaînes consistent
en des caractères Latin, chiffres, plus (+) et symbole slash (/).
Il peut y avoir 1 ou 2 symboles de remplissage (=), mais ils ne se trouvent jamais
au milieu d'une chaîne.
Gardez à l'esprit ces propriétés du base64, il est très facile de les reconnaître.

Décodons les et calculons l'entropie (\myref{entropy}) de ces blocs dans Wolfram Mathematica:

\begin{lstlisting}
In[]:= ListOfBase64Strings =
  Map[First[#[[3]]] &, Cases[Import["encrypted.xml"], XMLElement["Data", _, _], Infinity]];

In[]:= BinaryStrings =
  Map[ImportString[#, {"Base64", "String"}] &, ListOfBase64Strings];

In[]:= Entropies = Map[N[Entropy[2, #]] &, BinaryStrings];

In[]:= Variance[Entropies]
Out[]= 0.0238614
\end{lstlisting}

\myindex{Variance}
La variance est basse.
Cela signifie que l'entropie des valeurs ne sont pas très différentes les unes des autres.
Ceci est visible sur le graphique:

\begin{lstlisting}
In[]:= ListPlot[Entropies]
\end{lstlisting}

\begin{figure}[H]
\centering
\myincludegraphics{examples/encrypted_DB1/entropy.png}
\end{figure}

La plupart des valeurs sont entre 5.0 et 5.4.
Ceci est un signe que les données sont compressées et/ou chiffrées

Pour comprendre la variance, calculons l'entropie de toutes les liens du livre de
Conan Doyle \emph{The Hound of the Baskervilles}:

\begin{lstlisting}
In[]:= BaskervillesLines = Import["http://www.gutenberg.org/cache/epub/2852/pg2852.txt", "List"];

In[]:= EntropiesT = Map[N[Entropy[2, #]] &, BaskervillesLines];

In[]:= Variance[EntropiesT]
Out[]= 2.73883

In[]:= ListPlot[EntropiesT]
\end{lstlisting}

\begin{figure}[H]
\centering
\myincludegraphics{examples/encrypted_DB1/conan_doyle.png}
\end{figure}

La plupart des valeurs sont regroupées autour de 4, mais il y a aussi des valeurs
qui sont plus petites, et elles influencent la valeur finale de la variance.

Peut-être que les chaînes courtes ont une entropie plus petite, prenons les chaînes
courtes du livre de Conan Doyle.

\begin{lstlisting}
In[]:= Entropy[2, "Yes, sir."] // N
Out[]= 2.9477
\end{lstlisting}

Essayons encore plus petit:

\begin{lstlisting}
In[]:= Entropy[2, "Yes"] // N
Out[]= 1.58496

In[]:= Entropy[2, "No"] // N
Out[]= 1.
\end{lstlisting}

\subsection{Est-ce que les données sont compressées?}

OK, donc nos données sont compressées et/ou chiffrées.
Sont-elles compressées? Presque tous les compresseurs de données ajoutent un entête
au début, une signature ou quelque chose comme ça.
Comme on peut le voir, il n'y a pas de motifs communs au début de chaque bloc.
Il est toujours possible qu'il s'agisse d'un compresseur de données écrit à la main,
mais c'est très rare.
D'un autre côté, les algorithmes de chiffrement maison sont plus répandus, car il
est facile d'en faire un.
\myindex{memfrob()}
\myindex{ROT13}
Même des systèmes de chiffrement sans clef primitifs comme \emph{memfrob()}\footnote{\url{http://linux.die.net/man/3/memfrob}}
et ROT13 fonctionnent bien sans erreur.
C'est un gros défi d'écrire un compresseur depuis zéro, en utilisant seulement sa
fantaisie et son imagination de façon à ce qu'il n'ait pas de bugs évidents.
Certains programmeurs implémentent des fonctions de compression de données en lisant
des livres, mais ceci est aussi rare.
Les deux moyens les plus fréquents sont:
\myindex{zlib}
1) utiliser simplement la bibliothèque open-source zlib;
2) copier/coller quelque chose de quelque part.
Les algorithmes de compression open-source mettent en général une sorte d'en-tête,
ainsi que les algorithmes de sites comme \url{http://www.codeproject.com/}.

\subsection{Est-ce que les données sont chiffrées?}

Les algorithmes majeurs de chiffrement de données traitent les données par bloc.
DES---8 octets, AES---16 octets. 
Si le buffer en entrée n'est pas divisible par la taille du bloc, des zéros sont
ajoutés (ou quelque chose d'autre), afin que les donnés chiffrées soient alignées
sur la taille du bloc de l'algorithme.
Ce n'est pas notre cas.

En utilisant Wolfram Mathematica, j'ai analysé la longueur des blocs:

\begin{lstlisting}
In[]:= Counts[Map[StringLength[#] &, BinaryStrings]]
Out[]= <|42 -> 1858, 38 -> 1235, 36 -> 699, 46 -> 1151, 40 -> 1784,
 44 -> 1558, 50 -> 366, 34 -> 291, 32 -> 74, 56 -> 15, 48 -> 716,
 30 -> 13, 52 -> 156, 54 -> 71, 60 -> 3, 58 -> 6, 28 -> 4|>
\end{lstlisting}

1858 blocs ont une taille de 42 octets, 1235 blocs ont une taille de 38 octets, etc.

J'ai fait un graphe:

\begin{lstlisting}
ListPlot[Counts[Map[StringLength[#] &, BinaryStrings]]]
\end{lstlisting}

\begin{figure}[H]
\centering
\myincludegraphics{examples/encrypted_DB1/lengths.png}
\end{figure}

Donc, la plupart des blocs ont une taille entre $\textasciitilde{}36$ et $\textasciitilde{}48$.
Il y a un autre chose à remarquer: tous les blocs ont une taille paire.
Pas un bloc n'a une taille impaire.

Il y a, toutefois, des flux de chiffrement qui opèrent au niveau de l'octet ou même
du bit.

\subsection{CryptoPP}
\myindex{CryptoPP}

Le programme qui peut parcourir cette base de données chiffrées est écrit en C\#
et le code .NET est fortement obscurci.
Néanmoins, il y a une DLL avec du code x86, qui, après un bref examen, contient des
parties de la bibliothèque open-source connue CryptoPP!
(J'ai juste repéré des chaînes \q{CryptoPP} dedans.)
Maintenant, c'est très facile de trouver toutes les fonctions à l'intérieur de la
DLL car la bibliothèque CryptoPP est open-source.

\myindex{AES}
La bibliothèque CryptoPP contient beaucoup de fonctions de chiffrement, AES inclus (AKA Rijndael).
Les CPUs x86 récents possèdent des instructions dédiées à AES comme \INS{AESENC}, \INS{AESDEC}
et \INS{AESKEYGENASSIST}\footnote{\url{https://en.wikipedia.org/wiki/AES_instruction_set}}.
Elles ne font pas le chiffrement/déchiffrement complètement, mais elles font une
part significative du travail.
Et les nouvelles versions de CryptoPP les utilisent.
Par exemple, ici:
\href{https://github.com/mmoss/cryptopp/blob/2772f7b57182b31a41659b48d5f35a7b6cedd34d/src/rijndael.cpp#L1034}{1},
\href{https://github.com/mmoss/cryptopp/blob/2772f7b57182b31a41659b48d5f35a7b6cedd34d/src/rijndael.cpp#L1000}{2}.
\myindex{x86!\Instructions!AESENC}
\myindex{x86!\Instructions!AESDEC}
\myindex{tracer}
À ma surprise, lors du déchiffrement, \INS{AESENC} est exécutée, tandis que \INS{AESDEC}
ne l'est pas (j'ai vérifié avec mon utilitaire tracer, mais n'importe quel débogueur
peut être utilisé).
J'ai vérifié, si mon CPU supporte réellement les instructions AES. Certains CPUs
Intel i3 ne les supportent pas.
Et si non, la bibliothèque CryptoPP se rabat sur les fonctions implémentées de l'ancienne façon
\footnote{\url{https://github.com/mmoss/cryptopp/blob/2772f7b57182b31a41659b48d5f35a7b6cedd34d/src/rijndael.cpp#L355}}.
Mais mon CPU les supporte.
Pourquoi \INS{AESDEC} n'est pas exécuté?
Pourquoi le programme utilise le chiffrement AES pour déchiffrer la base de données?

OK, ce n'est pas un problème de trouver la fonction qui chiffre les blocs.
Elle est appelée \\
\emph{CryptoPP::Rijndael::Enc::ProcessAndXorBlock}:
\href{https://github.com/mmoss/cryptopp/blob/2772f7b57182b31a41659b48d5f35a7b6cedd34d/src/rijndael.cpp#L349}{src},
et elle peut être appelée depuis une autre fonction: \\
\emph{Rijndael::Enc::AdvancedProcessBlocks()}
\href{https://github.com/mmoss/cryptopp/blob/2772f7b57182b31a41659b48d5f35a7b6cedd34d/src/rijndael.cpp#L1179}{src},
qui, à son tour, appelle les deux fonctions: (
\href{https://github.com/mmoss/cryptopp/blob/2772f7b57182b31a41659b48d5f35a7b6cedd34d/src/rijndael.cpp#L1000}{AESNI\_Enc\_Block}
et
\href{https://github.com/mmoss/cryptopp/blob/2772f7b57182b31a41659b48d5f35a7b6cedd34d/src/rijndael.cpp#L1012}{AESNI\_Enc\_4\_Blocks}
)
qui ont les instructions  \INS{AESENC}.

Donc, a en juger par les entrailles de CryptoPP \\
\emph{CryptoPP::Rijndael::Enc::ProcessAndXorBlock()} chiffre un bloc 16-octet.
Mettons un point d'arrêt dessus et voyons ce qui se produit pendant le déchiffrement.
J'utilise à nouveau mon petit outil tracer.
Le logiciel doit déchiffrer le premier bloc de données maintenant.
Oh, à propos, voici le premier bloc de données converti de l'encodage en base64 vers
des données hexadécimale, faisons le manuellement:

\lstinputlisting{examples/encrypted_DB1/1.lst}

Voici les arguments de la fonction d'après les fichiers sources de CryptoPP:

\begin{lstlisting}
size_t Rijndael::Enc::AdvancedProcessBlocks(const byte *inBlocks, const byte *xorBlocks, byte *outBlocks, size_t length, word32 flags);
\end{lstlisting}

Donc, il y a 5 arguments. Les flags possibles sont:

\begin{lstlisting}
enum {BT_InBlockIsCounter=1, BT_DontIncrementInOutPointers=2, BT_XorInput=4, BT_ReverseDirection=8, BT_AllowParallel=16} FlagsForAdvancedProcessBlocks;
\end{lstlisting}

OK, lançons tracer sur la fonction \emph{ProcessAndXorBlock()}:

\lstinputlisting{examples/encrypted_DB1/2.lst}

Ici nous pouvons voir l'entrée de la fonction \emph{ProcessAndXorBlock()}, et sa sortie.

Ceci est la sortie de la fonction lors du premier appel:

\begin{lstlisting}
00000000: C7 39 4E 7B 33 1B D6 1F-B8 31 10 39 39 13 A5 5D ".9N{3....1.99..]"
\end{lstlisting}

Puis la fonction \emph{ProcessAndXorBlock()} est appelée avec un bloc de longueur
zéro, mais avec le flag 8 (\emph{BT\_ReverseDirection}).

Second appel:

\begin{lstlisting}
00000000: 45 00 20 00 4A 00 4F 00-48 00 4E 00 53 00 00 00 "E. .J.O.H.N.S..."
\end{lstlisting}

Maintenant, il y a des chaînes qui nous sont familières!

Troisième appel:

\begin{lstlisting}
00000000: B1 27 7F E4 9F 01 E3 81-CF C6 12 FB B9 7C F1 BC ".'...........|.."
\end{lstlisting}

La première sortie est très similaire aux 16 premiers octets du buffer chiffré.

Sortie du premier appel à \emph{ProcessAndXorBlock()}:

\begin{lstlisting}
00000000: C7 39 4E 7B 33 1B D6 1F-B8 31 10 39 39 13 A5 5D ".9N{3....1.99..]"
\end{lstlisting}

16 premiers octets du buffer chiffré:

\begin{lstlisting}
00000000: CA 39 B1 85 75 1B 84 1F F9 31 5E 39 72 13 EC 5D  .9..u....1^9r..]
\end{lstlisting}

Il y a trop d'octets égaux!
Comment le résultat du chiffrement AES peut-il être aussi similaire au buffer chiffré
alors que ceci n'est pas du chiffrement mais bien du déchiffrement?!

\subsection{Mode Cipher Feedback}

\myindex{Cipher Feedback mode}
\myindex{XOR}
La réponse est \ac{CFB}:
Dans ce mode, l'algorithme AES n'est pas utilisé comme un algorithme de chiffrement,
mais comme un dispositif qui génère des données aléatoires cryptographiquement sûres.
Le chiffrement effectif est obtenu en utilisant une simple opération XOR.

Voici l'algorithme de chiffrement (les images proviennent de Wikipédia):

\begin{figure}[H]
\centering
\myincludegraphics{examples/encrypted_DB1/601px-CFB_encryption.png}
\end{figure}

Et le déchiffrement:

\begin{figure}[H]
\centering
\myincludegraphics{examples/encrypted_DB1/601px-CFB_decryption.png}
\label{fig:CFB_decryption}
\end{figure}

Maintenant regardons: le chiffrement AES génère 16 octets (ou 128 bits) de données
\emph{aléatoires} destinées à être utilisées lors du XOR, qui nous oblige à utiliser
tous les 16 octets?
Si à la dernière itération nous n'avons qu'un octet de données, nous ne chiffrons
qu'un octet avec un octet de données \emph{aléatoires} générée.
Ceci conduit à une propriété importante du mode \ac{CFB}: les données ne doivent
pas être adaptées à une taille, des données de taille arbitraire peuvent être chiffrées
et déchiffrées.

Oh, c'est pour ça que les blocs chiffrés ne sont pas complétés.
Et c'est pourquoi l'instruction \INS{AESDEC} n'est jamais appelée.

Essayons de déchiffrer le premier bloc manuellement, en utilisant Python.
Le mode \ac{CFB} utilise aussi un \ac{IV}, comme \emph{semence} pour \ac{CSPRNG}.
Dans notre cas, l'\ac{IV} est le bloc qui est chiffré à la première itération:

\begin{lstlisting}
0038B920: 01 00 00 00 FF FF FF FF-79 C1 69 0B 67 C1 04 7D "........y.i.g..}"
\end{lstlisting}

Oh, et nous devons aussi retrouver la clef de chiffrement.
\myindex{x86!\Instructions!AESKEYGENASSIST}
Il y a \INS{AESKEYGENASSIST} dans la DLL, et elle est appelée, et elle est utilisée dans la fonction \\
\href{https://github.com/mmoss/cryptopp/blob/2772f7b57182b31a41659b48d5f35a7b6cedd34d/src/rijndael.cpp#L198}{src}.
C'est facile de la trouver dans \IDA et de mettre un point d'arrêt. Voyons:

\begin{lstlisting}
... tracer.exe -l:filename.exe bpf=filename.exe!0x435c30,args:3,dump_args:0x10

Warning: no tracer.cfg file.
PID=2068|New process software.exe
no module registered with image base 0x77320000
no module registered with image base 0x76e20000
no module registered with image base 0x77320000
no module registered with image base 0x77220000
Warning: unknown (to us) INT3 breakpoint at ntdll.dll!LdrVerifyImageMatchesChecksum+0x96c (0x776c103b)
(0) software.exe!0x435c30(0x15e8000, 0x10, 0x14f808) (called from software.exe!.text+0x22fa1 (0x13d3fa1))
Argument 1/3
015E8000: CD C5 7E AD 28 5F 6D E1-CE 8F CC 29 B1 21 88 8E "..~.(_m....).!.."
Argument 3/3
0014F808: 38 82 58 01 C8 B9 46 00-01 D1 3C 01 00 F8 14 00 "8.X...F...<....."
Argument 3/3 +0x0: software.exe!.rdata+0x5238
Argument 3/3 +0x8: software.exe!.text+0x1c101
(0) software.exe!0x435c30() -> 0x13c2801
PID=2068|Process software.exe exited. ExitCode=0 (0x0)
\end{lstlisting}

Donc, ceci est la clef: \emph{CD C5 7E AD 28 5F 6D E1-CE 8F CC 29 B1 21 88 8E}.

Durant le déchiffrement manuel, nous obtenons ceci:

\begin{lstlisting}
00000000: 0D 00 FF FE 46 00 52 00  41 00 4E 00 4B 00 49 00  ....F.R.A.N.K.I.
00000010: 45 00 20 00 4A 00 4F 00  48 00 4E 00 53 00 66 66  E. .J.O.H.N.S.ff
00000020: 66 66 66 9E 61 40 D4 07  06 01                    fff.a@....
\end{lstlisting}

Maintenant, c'est quelque chose de lisible!
Et nous comprenons pourquoi il y avait autant d'octets égaux dans la première
itération de déchiffrement:
car le text en clair a beaucoup d'octet à zéro!
Déchiffrons le second bloc:

\begin{lstlisting}
00000000: 17 98 D0 84 3A E9 72 4F  DB 82 3F AD E9 3E 2A A8  ....:.rO..?..>*.
00000010: 41 00 52 00 52 00 4F 00  4E 00 CD CC CC CC CC CC  A.R.R.O.N.......
00000020: 1B 40 D4 07 06 01                                 .@....
\end{lstlisting}

Les troisième, quatrième et cinquième:

\begin{lstlisting}
00000000: 5D 90 59 06 EF F4 96 B4  7C 33 A7 4A BE FF 66 AB  ].Y.....|3.J..f.
00000010: 49 00 47 00 47 00 53 00  00 00 00 00 00 C0 65 40  I.G.G.S.......e@
00000020: D4 07 06 01                                       ....
\end{lstlisting}

\begin{lstlisting}
00000000: D3 15 34 5D 21 18 7C 6E  AA F8 2D FE 38 F9 D7 4E  ..4]!.|n..-.8..N
00000010: 41 00 20 00 44 00 4F 00  48 00 45 00 52 00 54 00  A. .D.O.H.E.R.T.
00000020: 59 00 48 E1 7A 14 AE FF  68 40 D4 07 06 02        Y.H.z...h@....
\end{lstlisting}

\begin{lstlisting}
00000000: 1E 8B 90 0A 17 7B C5 52  31 6C 4E 2F DE 1B 27 19  .....{.R1lN...'.
00000010: 41 00 52 00 43 00 55 00  53 00 00 00 00 00 00 60  A.R.C.U.S.......
00000020: 66 40 D4 07 06 03                                 f@....
\end{lstlisting}

Tous les blocs déchiffrés semblent correct, à l'exception des 16 premiers octets.

\subsection{Initializing Vector}

Qu'est-ce qui peut affecter les 16 premiers octets?

Revenons à nouveau à l'algorithme de déchiffrement \ac{CFB}: \myref{fig:CFB_decryption}.

Nous pouvons voir que l'\ac{IV} peut affecter le déchiffrement de la première opération
de déchiffrement, mais pas la seconde, car lors de la seconde itération, le texte
chiffré de la première itération est utilisé, et en cas de déchiffrement, c'est le
même, quelque soit l'\ac{IV}!

Donc, l'\ac{IV} est sans doute différent à chaque fois.
En utilisant mon tracer, j'ai regardé la première entrée lors du déchiffrement du
second bloc du fichier \ac{XML}:

\begin{lstlisting}
0038B920: 02 00 00 00 FE FF FF FF-79 C1 69 0B 67 C1 04 7D "........y.i.g..}"
\end{lstlisting}

\dots troisième:

\begin{lstlisting}
0038B920: 03 00 00 00 FD FF FF FF-79 C1 69 0B 67 C1 04 7D "........y.i.g..}"
\end{lstlisting}

Il semble que le premier et le cinquième octet changent à chaque fois.
J'en ai finalement conclu que le premier entier 32-bit est simplement OrderID du fichier
\ac{XML}, et le second entier 32-bit est aussi OrderID, mais multiplié par -1. Tous
les 8 autres octets sont les mêmes pour chaque opération.
Maintenant, j'ai déchiffré la base de données entière:
\url{https://raw.githubusercontent.com/DennisYurichev/RE-for-beginners/master/examples/encrypted_DB1/decrypted.full.txt}.

Le script Python utilisé pour ceci est:
\url{https://github.com/DennisYurichev/RE-for-beginners/blob/master/examples/encrypted_DB1/decrypt_blocks.py}.

Peut-être que l'auteur voulait chiffrer chaque bloc différemment, donc il a utilisé
OrderID comme une partie de la clef.
Il aurait aussi été possible de créer une clef AES différente, au lieu de l'\ac{IV}.

Dinc maintenant nous savons que l'\ac{IV} affecte seulement le premier bloc lors
du déchiffrement en mode \ac{CFB}, ceci en est une caractéristique.
Tous les autres blocs peuvent être déchiffrés sans connaître l'\ac{IV}, mais en utilisant
la clef.

OK, donc pourquoi le mode \ac{CFB}? Apparemment, parce que le tout premier exemple
sur le wiki de CryptoPP utilise le mode \ac{CFB}:
\url{http://www.cryptopp.com/wiki/Advanced_Encryption_Standard#Encrypting_and_Decrypting_Using_AES}.
On peut aussi supposer que le développeur l'a choisi pour sa simplicité:
l'exemple peut chiffrer/déchiffrer des chaînes de texte de longueur arbitraire, sans
remplissage.

Il est aussi probable que l'auteur du programme a juste copié/collé l'exemple depuis
la page wiki de CryptoPP.
Beaucoup de programmeurs font ça.

La seule différence est que l'\ac{IV} est choisi aléatoirement dans l'exemple du
wiki de CryptoPP, alors que cet indéterminisme n'était pas permis aux programmeurs
du logiciel que nous disséquons maintenant, donc ils ont choisi d'initialiser l'\ac{IV}
en utilisant OrderID.

Nous pouvons maintenant procéder à l'analyse du cas de chaque octet dans le bloc
déchiffré.

\subsection{Structure du buffer}

Prenons les quatre premier bloc déchiffrés:

\begin{lstlisting}
00000000: 0D 00 FF FE 46 00 52 00  41 00 4E 00 4B 00 49 00  ....F.R.A.N.K.I.
00000010: 45 00 20 00 4A 00 4F 00  48 00 4E 00 53 00 66 66  E. .J.O.H.N.S.ff
00000020: 66 66 66 9E 61 40 D4 07  06 01                    fff.a@....

00000000: 0B 00 FF FE 4C 00 4F 00  52 00 49 00 20 00 42 00  ....L.O.R.I. .B.
00000010: 41 00 52 00 52 00 4F 00  4E 00 CD CC CC CC CC CC  A.R.R.O.N.......
00000020: 1B 40 D4 07 06 01                                 .@....

00000000: 0A 00 FF FE 47 00 41 00  52 00 59 00 20 00 42 00  ....G.A.R.Y. .B.
00000010: 49 00 47 00 47 00 53 00  00 00 00 00 00 C0 65 40  I.G.G.S.......e@
00000020: D4 07 06 01                                       ....

00000000: 0F 00 FF FE 4D 00 45 00  4C 00 49 00 4E 00 44 00  ....M.E.L.I.N.D.
00000010: 41 00 20 00 44 00 4F 00  48 00 45 00 52 00 54 00  A. .D.O.H.E.R.T.
00000020: 59 00 48 E1 7A 14 AE FF  68 40 D4 07 06 02        Y.H.z...h@....
\end{lstlisting}

On voit clairement des chaînes de textes encodées en UTF-16, ce sont les noms et
noms de famille.
Le premier octet (ou mot de 16-bit) semble être la longueur de la chaîne, nous pouvons
vérifier visuellement.
\emph{FF FE} semble être le \ac{BOM} Unicode.

Il y a 12 autres octets après chaque chaîne.

En utilisant ce script
(\url{https://github.com/DennisYurichev/RE-for-beginners/blob/master/examples/encrypted_DB1/dump_buffer_rest.py})
j'ai obtenu une sélection aléatoire de \emph{fins} (de bloc):

\lstinputlisting{examples/encrypted_DB1/tails.lst}

Nous voyons tout d'abord que les octets 0x40 et 0x07 sont présent dans chaque \emph{fin}.
Le tout dernier octet est toujours dans l'intervalle 1..0x1F (1..31), j'ai vérifié.
Le pénultième octet est toujours dans l'intervalle 1..0xC (1..12).
Ouah, ça ressemble à une date!
L'année peut être représentée comme une valeur 16-bit, et peut-être que les 4 derniers
octets sont une date (16 bits pour l'année, 8 bits pour le mois et les 8 restants
pour le jour)?
0x7DD est 2013, 0x7D5 est 2005, etc. Ça semble juste. Ceci est une date.
Il y a 8 octets supplémentaires.
À en juger par le fait que ceci est une base de données appelée \emph{orders}, peut-être
s'agit-il d'une sorte de somme ici?
J'ai essayé de les interpréter comme des réels en double précision IEEE 754 et ai
affiché toutes les valeurs!

Certaines sont:

\begin{lstlisting}
71.0
134.0
51.95
53.0
121.99
96.95
98.95
15.95
85.95
184.99
94.95
29.95
85.0
36.0
130.99
115.95
87.99
127.95
114.0
150.95
\end{lstlisting}

Ça ressemble à des nombres réels

Maintenant, nous pouvons afficher les noms, sommes et dates.

\begin{lstlisting}
plain:
00000000: 0D 00 FF FE 46 00 52 00  41 00 4E 00 4B 00 49 00  ....F.R.A.N.K.I.
00000010: 45 00 20 00 4A 00 4F 00  48 00 4E 00 53 00 66 66  E. .J.O.H.N.S.ff
00000020: 66 66 66 9E 61 40 D4 07  06 01                    fff.a@....
OrderID= 1 name= FRANKIE JOHNS sum= 140.95 date= 2004 / 6 / 1

plain:
00000000: 0B 00 FF FE 4C 00 4F 00  52 00 49 00 20 00 42 00  ....L.O.R.I. .B.
00000010: 41 00 52 00 52 00 4F 00  4E 00 CD CC CC CC CC CC  A.R.R.O.N.......
00000020: 1B 40 D4 07 06 01                                 .@....
OrderID= 2 name= LORI BARRON sum= 6.95 date= 2004 / 6 / 1

plain:
00000000: 0A 00 FF FE 47 00 41 00  52 00 59 00 20 00 42 00  ....G.A.R.Y. .B.
00000010: 49 00 47 00 47 00 53 00  00 00 00 00 00 C0 65 40  I.G.G.S.......e@
00000020: D4 07 06 01                                       ....
OrderID= 3 name= GARY BIGGS sum= 174.0 date= 2004 / 6 / 1

plain:
00000000: 0F 00 FF FE 4D 00 45 00  4C 00 49 00 4E 00 44 00  ....M.E.L.I.N.D.
00000010: 41 00 20 00 44 00 4F 00  48 00 45 00 52 00 54 00  A. .D.O.H.E.R.T.
00000020: 59 00 48 E1 7A 14 AE FF  68 40 D4 07 06 02        Y.H.z...h@....
OrderID= 4 name= MELINDA DOHERTY sum= 199.99 date= 2004 / 6 / 2

plain:
00000000: 0B 00 FF FE 4C 00 45 00  4E 00 41 00 20 00 4D 00  ....L.E.N.A. .M.
00000010: 41 00 52 00 43 00 55 00  53 00 00 00 00 00 00 60  A.R.C.U.S.......
00000020: 66 40 D4 07 06 03                                 f@....
OrderID= 5 name= LENA MARCUS sum= 179.0 date= 2004 / 6 / 3
\end{lstlisting}

En voir plus: \url{https://raw.githubusercontent.com/DennisYurichev/RE-for-beginners/master/examples/encrypted_DB1/decrypted.full.with_data.txt}.
Ou filtré: \url{https://github.com/DennisYurichev/RE-for-beginners/blob/master/examples/encrypted_DB1/decrypted.short.txt}.
Ça semble correct.

Ceci est une sorte de sérialisation \ac{OOP}, i.e., stockant différents types de
valeurs dans un buffer binaire pour le stocker et/ou le transmettre.

\subsection{Bruit en fin de buffer}

La seule question qui reste est que, parfois, la \emph{fin} est plus longue:

\begin{lstlisting}
00000000: 0E 00 FF FE 54 00 48 00  45 00 52 00 45 00 53 00  ....T.H.E.R.E.S.
00000010: 45 00 20 00 54 00 55 00  54 00 54 00 4C 00 45 00  E. .T.U.T.T.L.E.
00000020: 66 66 66 66 66 1E 63 40  D4 07 07 1A 00 07 07 19  fffff.c@........
OrderID= 172 name= THERESE TUTTLE sum= 152.95 date= 2004 / 7 / 26
\end{lstlisting}

(Les octets \emph{00 07 07 19} ne sont pas utilisés et servent de remplissage.)

\begin{lstlisting}
00000000: 0C 00 FF FE 4D 00 45 00  4C 00 41 00 4E 00 49 00  ....M.E.L.A.N.I.
00000010: 45 00 20 00 4B 00 49 00  52 00 4B 00 00 00 00 00  E. .K.I.R.K.....
00000020: 00 20 64 40 D4 07 09 02  00 02                    . d@......
OrderID= 286 name= MELANIE KIRK sum= 161.0 date= 2004 / 9 / 2
\end{lstlisting}

(\emph{00 02} ne sont pas utilisés.)

Après un examen rigoureux, on peut voir que le but à la fin de la \emph{fin} est
juste le reste d'un chiffrement précédent!

Voici deux buffers consécutifs:

\begin{lstlisting}
00000000: 10 00 FF FE 42 00 4F 00  4E 00 4E 00 49 00 45 00  ....B.O.N.N.I.E.
00000010: 20 00 47 00 4F 00 4C 00  44 00 53 00 54 00 45 00   .G.O.L.D.S.T.E.
00000020: 49 00 4E 00 9A 99 99 99  99 79 46 40 D4 07 07 19  I.N......yF@....
OrderID= 171 name= BONNIE GOLDSTEIN sum= 44.95 date= 2004 / 7 / 25

00000000: 0E 00 FF FE 54 00 48 00  45 00 52 00 45 00 53 00  ....T.H.E.R.E.S.
00000010: 45 00 20 00 54 00 55 00  54 00 54 00 4C 00 45 00  E. .T.U.T.T.L.E.
00000020: 66 66 66 66 66 1E 63 40  D4 07 07 1A 00 07 07 19  fffff.c@........
OrderID= 172 name= THERESE TUTTLE sum= 152.95 date= 2004 / 7 / 26
\end{lstlisting}

(Les derniers octets \emph{07 07 19} sont copiés du buffer précédent.) 

Un autre exemple de deux buffers consécutifs:

\begin{lstlisting}
00000000: 0D 00 FF FE 4C 00 4F 00  52 00 45 00 4E 00 45 00  ....L.O.R.E.N.E.
00000010: 20 00 4F 00 54 00 4F 00  4F 00 4C 00 45 00 CD CC   .O.T.O.O.L.E...
00000020: CC CC CC 3C 5E 40 D4 07  09 02                    ...<^@....
OrderID= 285 name= LORENE OTOOLE sum= 120.95 date= 2004 / 9 / 2

00000000: 0C 00 FF FE 4D 00 45 00  4C 00 41 00 4E 00 49 00  ....M.E.L.A.N.I.
00000010: 45 00 20 00 4B 00 49 00  52 00 4B 00 00 00 00 00  E. .K.I.R.K.....
00000020: 00 20 64 40 D4 07 09 02  00 02                    . d@......
OrderID= 286 name= MELANIE KIRK sum= 161.0 date= 2004 / 9 / 2
\end{lstlisting}

Le dernier octet 02 a été copié du buffer en texte clair précédent.

C'est possible si le buffer utilisé lors du chiffrement est global et/ou s'il n'est
pas mis à zéro entre chaque chiffrement.
La taille du buffer final est aussi chaotique, néanmoins, le bogue reste sans conséquence
car il n'affecte pas le processus de déchiffrement, qui ignore le bruit à la fin.
C'est une erreur courante.
\myindex{OpenSSL}
\myindex{Heartbleed}
Il était présent dans OpenSSL (Heartbleed bug).

\subsection{Conclusion}

Résumé:
Chaque rétro-ingénieur pratiquant doit être familier avec la majorité des algorithmes
ainsi que la majorité des modes de chiffrement.
Quelques livres à ce sujet: \myref{crypto_books}.

Le contenu \emph{chiffré} de la base de données a été artificiellement construit
par moi, pour les besoins de la démonstration.
J'ai obtenu les nom et noms de famille les plus répandus au USA ici: \url{http://stackoverflow.com/questions/1803628/raw-list-of-person-names},
et les ai combiné aléatoirement.
Les dates et montants ont aussi été générés aléatoirement.

Tous les fichiers utilisés dans cette partie sont ici: \url{https://github.com/DennisYurichev/RE-for-beginners/tree/master/examples/encrypted_DB1}.

Néanmoins, j'ai observé de telles caractéristiques dans des logiciels réels.
Cet exemple est basé dessus.

\subsection{Post Scriptum: brute-force \ac{IV}}

Le cas que vous venez de voir a été construit artificiellement, mais il est basé
sur des logiciels réels que j'ai rétro-ingénièré.
Lorsque j'ai travaillé dessus, j'ai d'abord remarqué que l'\ac{IV} avait été généré
en utilisant un nombre 32-bit, et je n'ai pas été capable de trouver un lien entre
cette valeur et OrderID.
Donc j'ai utilisé le brute-force, ce qui est aussi possible ici.

Ce n'est pas un problème d'énumérer toutes les valeurs 32-bit et d'essayer chacune
d'elle comme base pour l'\ac{IV}.
Ensuite vous déchiffrez le premier bloc de 16 octets et vérifiez les octets à zéro,
qui sont toujours à des places fixes.
