Instead of epigraph:

\begin{framed}
\begin{quotation}

\textbf{Peter Seibel:} How do you tackle reading source code? Even reading something in
a programming language you already know is a tricky problem.

\textbf{Donald Knuth:} But it’s really worth it for what it builds in your brain. So how do I
do it? There was a machine called the Bunker Ramo 300 and somebody told
me that the Fortran compiler for this machine was really amazingly fast, but
nobody had any idea why it worked. I got a copy of the source-code listing
for it. I didn’t have a manual for the machine, so I wasn’t even sure what the
machine language was.

But I took it as an interesting challenge. I could figure out BEGIN and then I
would start to decode. The operation codes had some two-letter
mnemonics and so I could start to figure out “This probably was a load
instruction, this probably was a branch.” And I knew it was a Fortran
compiler, so at some point it looked at column seven of a card, and that was
where it would tell if it was a comment or not.

After three hours I had figured out a little bit about the machine. Then I
found these big, branching tables. So it was a puzzle and I kept just making
little charts like I’m working at a security agency trying to decode a secret
code. But I knew it worked and I knew it was a Fortran compiler—it wasn’t
encrypted in the sense that it was intentionally obscure; it was only in code
because I hadn’t gotten the manual for the machine.

Eventually I was able to figure out why this compiler was so fast.
Unfortunately it wasn’t because the algorithms were brilliant; it was just
because they had used unstructured programming and hand optimized the
code to the hilt.

It was just basically the way you solve some kind of an unknown puzzle—
make tables and charts and get a little more information here and make a
hypothesis. In general when I’m reading a technical paper, it’s the same
challenge. I’m trying to get into the author’s mind, trying to figure out what
the concept is. The more you learn to read other people’s stuff, the more
able you are to invent your own in the future, it seems to me.

\end{quotation}
\end{framed}

( Peter Seibel --- Coders at Work: Reflections on the Craft of Programming )

