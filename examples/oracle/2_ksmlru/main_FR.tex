\subsection{Table \TT{X\$KSMLRU} dans \oracle}
\myindex{\oracle}

Il y a une mention d'une table spéciale dans la note \emph{Diagnosing and Resolving
Error ORA-04031 on the Shared Pool or Other Memory Pools [Video] [ID 146599.1]}:

\begin{framed}
\begin{quotation}
Il y a une table fixée appelée X\$KSMLRU qui suit les différentes allocations dans
le pool partagé qui force les autres objets du pool partagé à vieillir. Cette table
fixée peut être utilisée pour identifier ce qui cause une grosse allocation.

Si plusieurs objets sont supprimés périodiquement du pool partagé, alors ceci va poser des problèmes de temps de réponse
et va probablement provoquer des problèmes de contention du verrou de cache de bibliothèque lorsque
les objets seront rechargés dans le pool partagé.

Une chose inhabituelle à propos de la table fixée X\$KSMLRU est que le contenu de la table fixée est écrasé à chaque fois que
quelqu'uni effectue requête dans la table fixée. Ceci est fait puisque la table fixée ne contient que l'allocation la plus large
qui s'est produite. Les valeurs sont réinitialisées après avoir été sélectionnées, de sorte que les allocations importantes
suivantes puissent ête inscrites, même si elles ne sont pas aussi laarges que celles qui se sont produites précédemment.
À cause de cette réinitialisation, la sortie produite par la sélection de cette table doit être soigneusement conservée
puisqu'elle ne peut plus être récupérée après que la requête a été faite.
\end{quotation}
\end{framed}

Toutefois, comme on peut le vérifier facilement, le contenu de cette table est effacé à chaque fois qu'on l'interroge.
Pouvons-nous trouver pourquoi?
Retournons aux tables que nous connaissons déjà: \TT{kqftab} et \TT{kqftap} qui
sont générées avec l'aide d'\oracletables, qui a toutes les informations concernant les table X\$-.
Nous pouvons voir ici que la fonction \TT{ksmlrs()} est appelée pour préparer les éléments de cette table:

\begin{lstlisting}[caption=Résultat de \OracleTablesName]
kqftab_element.name: [X$KSMLRU] ?: [ksmlr] 0x4 0x64 0x11 0xc 0xffffc0bb 0x5
kqftap_param.name=[ADDR] ?: 0x917 0x0 0x0 0x0 0x4 0x0 0x0
kqftap_param.name=[INDX] ?: 0xb02 0x0 0x0 0x0 0x4 0x0 0x0
kqftap_param.name=[INST_ID] ?: 0xb02 0x0 0x0 0x0 0x4 0x0 0x0
kqftap_param.name=[KSMLRIDX] ?: 0xb02 0x0 0x0 0x0 0x4 0x0 0x0
kqftap_param.name=[KSMLRDUR] ?: 0xb02 0x0 0x0 0x0 0x4 0x4 0x0
kqftap_param.name=[KSMLRSHRPOOL] ?: 0xb02 0x0 0x0 0x0 0x4 0x8 0x0
kqftap_param.name=[KSMLRCOM] ?: 0x501 0x0 0x0 0x0 0x14 0xc 0x0
kqftap_param.name=[KSMLRSIZ] ?: 0x2 0x0 0x0 0x0 0x4 0x20 0x0
kqftap_param.name=[KSMLRNUM] ?: 0x2 0x0 0x0 0x0 0x4 0x24 0x0
kqftap_param.name=[KSMLRHON] ?: 0x501 0x0 0x0 0x0 0x20 0x28 0x0
kqftap_param.name=[KSMLROHV] ?: 0xb02 0x0 0x0 0x0 0x4 0x48 0x0
kqftap_param.name=[KSMLRSES] ?: 0x17 0x0 0x0 0x0 0x4 0x4c 0x0
kqftap_param.name=[KSMLRADU] ?: 0x2 0x0 0x0 0x0 0x4 0x50 0x0
kqftap_param.name=[KSMLRNID] ?: 0x2 0x0 0x0 0x0 0x4 0x54 0x0
kqftap_param.name=[KSMLRNSD] ?: 0x2 0x0 0x0 0x0 0x4 0x58 0x0
kqftap_param.name=[KSMLRNCD] ?: 0x2 0x0 0x0 0x0 0x4 0x5c 0x0
kqftap_param.name=[KSMLRNED] ?: 0x2 0x0 0x0 0x0 0x4 0x60 0x0
kqftap_element.fn1=ksmlrs
kqftap_element.fn2=NULL
\end{lstlisting}

\myindex{tracer}
En effet, avec l'aide de \tracer, il est facile de voir que cette fonction est appelée
à chaque fois que nous interrogeons la table \TT{X\$KSMLRU}.

\myindex{\CStandardLibrary!memset()}
Ici nous voyons une référence aux fonctions \TT{ksmsplu\_sp()} et \TT{ksmsplu\_jp()},
chacune d'elles appelle \TT{ksmsplu()} à la fin.
À la fin de la fonction \TT{ksmsplu()} nous voyons un appel à \TT{memset()}:

\begin{lstlisting}[caption=ksm.o,style=customasmx86]
...

.text:00434C50 loc_434C50:    ; DATA XREF: .rdata:off\_5E50EA8
.text:00434C50         mov     edx, [ebp-4]
.text:00434C53         mov     [eax], esi
.text:00434C55         mov     esi, [edi]
.text:00434C57         mov     [eax+4], esi
.text:00434C5A         mov     [edi], eax
.text:00434C5C         add     edx, 1
.text:00434C5F         mov     [ebp-4], edx
.text:00434C62         jnz     loc_434B7D
.text:00434C68         mov     ecx, [ebp+14h]
.text:00434C6B         mov     ebx, [ebp-10h]
.text:00434C6E         mov     esi, [ebp-0Ch]
.text:00434C71         mov     edi, [ebp-8]
.text:00434C74         lea     eax, [ecx+8Ch]
.text:00434C7A         push    370h            ; Size
.text:00434C7F         push    0               ; Val
.text:00434C81         push    eax             ; Dst
.text:00434C82         call    __intel_fast_memset
.text:00434C87         add     esp, 0Ch
.text:00434C8A         mov     esp, ebp
.text:00434C8C         pop     ebp
.text:00434C8D         retn
.text:00434C8D _ksmsplu  endp
\end{lstlisting}

Des constructions comme \TT{memset (block, 0, size)} sont souvent utilisées pour
mettre à zéro un bloc de mémoire.
Que se passe-t-il si nous prenons le risque de bloquer l'appel à \TT{memset (block, 0, size)}
et regardons ce qui se produit?

\myindex{tracer}

Lançons \tracer avec les options suivantes: mettre un point d'arrêt en \TT{0x434C7A}
(le point où les arguments sont passés à \TT{memset()}), afin que \tracer mette le compteur de programme \TT{EIP} au point
où les arguments passés à \TT{memset()} sont éffacés (en \TT{0x434C8A}).
On peut dire que nous simulons juste un saut inconditionnel de l'adresse \TT{0x434C7A} à \TT{0x434C8A}.

\begin{lstlisting}
tracer -a:oracle.exe bpx=oracle.exe!0x00434C7A,set(eip,0x00434C8A)
\end{lstlisting}

(Important: toutes ces adresses sont valides seulement pour la version win32 de \oracle 11.2)

En effet, nous pouvons maintenant interroger la table \TT{X\$KSMLRU} autant de fois
que nous voulons et elle n'est plus du tout effacée!

% \sout{Do not try this at home ("MythBusters")}
Au cas où, n'essayez pas ceci sur vos serveurs de production.

Ce n'est probablement pas un comportement très utile ou souhaité, mais comme une
expérience pour déterminer l'emplacement d'un bout de code dont nous avons besoin,
ça remplit parfaitement notre besoin!

