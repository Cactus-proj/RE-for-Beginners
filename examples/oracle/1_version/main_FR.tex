\subsection{Table \TT{V\$VERSION} dans \oracle}

\myindex{\oracle}
\myindex{Linux}
\myindex{Windows!ntoskrnl.exe}
\oracle 11.2 est un programme gigantesque, son module principal \TT{oracle.exe} contient
environ 124000 fonctions. Par comparaison, le noyau de Windows 7 x86 (ntoskrnl.exe)
contient environ 11000 fonctions et le noyau Linux 3.9.8 (avec les drivers par défaut
compilés)---31000 fonctions.

Commençons par une question facile. Où \oracle trouve-t-il toutes ces informations,
lorsque l'on exécute une expression simple dans SQL*Plus:

\lstinputlisting{examples/oracle/1_version/0.txt}

Et nous obtenons:

\lstinputlisting{examples/oracle/1_version/1.txt}

Allons-y. Où \oracle trouve-t-il la chaîne \TT{V\$VERSION}?

Dans la version win32, le fichier \TT{oracle.exe} contient la chaîne, c'est facile
à voir.
Mais nous pouvons aussi utiliser les fichiers objet (.o) de la version Linux d'\oracle,
puisque contrairement à la version win32 \TT{oracle.exe}, les noms de fonctions (et
aussi les variables globales) y sont préservés.

Donc, le fichier \TT{kqf.o} contient la chaîne \TT{V\$VERSION}.
Le fichier objet se trouve dans la bibliothèque Oracle principale \TT{libserver11.a}.

On trouve une référence à ce texte dans la table \TT{kqfviw} stockée dans le même
fichier, \TT{kqf.o}:

\lstinputlisting[caption=kqf.o,style=customasmx86]{examples/oracle/1_version/2.lst}

À propos, souvent, en analysant les entrailles d\oracle, vous pouvez vous demander
pourquoi les noms de fonctions et de variables globales sont si étranges.

Sans doute parce qu'\oracle est un très vieux produit et a été développé en C dans
les années 80.

Et c'était un temps où le standard C garantissait que les noms de fonction et de variable
pouvaient supporter seulement jusqu'à 6 caractères incluant: <<6 charactères significatifs
dans un identifiant externe>>\footnote{\href{http://go.yurichev.com/17142}{Draft
ANSI C Standard (ANSI X3J11/88-090) (May 13, 1988) (yurichev.com)}}

Probablement que la table \TT{kqfviw} contient la plupart (peut-être même toutes)
des vues préfixées avec V\$, qui sont des \emph{vues fixées}, toujours présentes.
Superficiellement, en remarquant la récurrence cyclique des données, nous pouvons
facilement voir que chaque élément de la table \TT{kqfviw} a 12 champs de 32-bit.
C'est très facile de créer une structure de 12 éléments dans \IDA et de l'appliquer
à tous les éléments de la table.
Depuis \oracle version 11.2, il y a 1023 éléments dans la table, i.e., dans celle-ci
sont décrites 1023 de toutes les \emph{vues fixées} possible.

Nous reviendrons à ce nombre plus tard.

Comme on le voit, il n'y a pas beaucoup d'information sur les nombres dans les champs.
Le premier nombre est toujours égal au nom de la vue (sans le zéro de fin).

Nous savons aussi que l'information sur toutes ces vues fixes peut être récupérée
depuis une \emph{vue fixée} appelée \TT{V\$FIXED\_VIEW\_DEFINITION} (à propos, l'information
pour cette vue est aussi prise dans les tables \TT{kqfviw} et \TT{kqfvip}.)
Au fait, il y a aussi 1023 éléments dans celle-ci. Coïncidence? Non.

\lstinputlisting{examples/oracle/1_version/3.txt}

Donc, \TT{V\$VERSION} est une sorte de \emph{vue terminale} pour une autre vue appelée
\TT{GV\$VERSION}, qui est, à son tour:

\lstinputlisting{examples/oracle/1_version/4.txt}

Les tables préfixées par X\$ dans \oracle sont aussi des tables de service, non documentées,
qui ne peuvent pas être modifiées par l'utilisateur et qui sont rafraîchies dynamiquement.

Si nous cherchons le texte \\
\lstinputlisting{examples/oracle/1_version/5.txt}
...  dans le fichier \TT{kqf.o}, nous le trouvons dans la table \TT{kqfvip}:

\lstinputlisting[caption=kqf.o,style=customasmx86]{examples/oracle/1_version/6.lst}

La table semble avoir 4 champs dans chaque élément. À propos, elle a 1023 éléments,
encore, le nombre que nous connaissons déjà.

Le second champ pointe sur une autre table qui contient les champs de la table pour
cette \emph{vue fixée}.
Comme pour \TT{V\$VERSION}, cette table a seulement deux éléments, le premier est
6 et le second est la chaîne \TT{BANNER} (le nombre 6 est la longueur de la chaîne)
et après, un élément \emph{de fin} qui contient 0 et une chaîne C \emph{null}:

\begin{lstlisting}[caption=kqf.o,style=customasmx86]
.rodata:080BBAC4 kqfv133_c_0 dd 6     ; DATA XREF: .rodata:08019574
.rodata:080BBAC8             dd offset _2__STRING_5017_0 ; "BANNER"
.rodata:080BBACC             dd 0
.rodata:080BBAD0             dd offset _2__STRING_0_0
\end{lstlisting}

En joignant les données des deux tables \TT{kqfviw} et \TT{kqfvip}, nous pouvons
obtenir la déclaration SQL qui est exécutée lorsque l'utilisateur souhaite faire
une requête sur une \emph{vue fixée} spécifique.

Ainsi nous pouvons écrire un programme \oracletables, pour collecter toutes ces informations
d'un fichier objet d'\oracle pour Linux.

\begin{lstlisting}[caption=Résultat de \OracleTablesName]
kqfviw_element.viewname: [V$VERSION] ?: 0x3 0x43 0x1 0xffffc085 0x4
kqfvip_element.statement: [select  BANNER from GV$VERSION where inst_id = USERENV('Instance')]
kqfvip_element.params:
[BANNER] 
\end{lstlisting}

Et:

\begin{lstlisting}[caption=Résultat de \OracleTablesName]
kqfviw_element.viewname: [GV$VERSION] ?: 0x3 0x26 0x2 0xffffc192 0x1
kqfvip_element.statement: [select inst_id, banner from x$version]
kqfvip_element.params:
[INST_ID] [BANNER] 
\end{lstlisting}

La \emph{vue fixée} \TT{GV\$VERSION} est différente de \TT{V\$VERSION} seulement
parce qu'elle a un champ de plus avec l'identifiant de l'\emph{instance}.

Quoiqu'il en soit, nous allons rester avec la table \TT{X\$VERSION}. Tout comme les
autres tables X\$, elle n'est pas documentée, toutefois, nous pouvons y effectuer
des requêtes:

\lstinputlisting{examples/oracle/1_version/7.txt}

Cette table a des champs additionnels, comme \TT{ADDR} et \TT{INDX}.

En faisant défiler \TT{kqf.o} dans \IDA, nous pouvons repérer une autre table qui
contient un pointeur sur la chaîne \TT{X\$VERSION}, c'est \TT{kqftab}:

\lstinputlisting[caption=kqf.o,style=customasmx86]{examples/oracle/1_version/8.lst}

Il y a beaucoup de référence aux noms de X\$-table, visiblement, à toutes les X\$-tables
d'\oracle 11.2.
Mais encore une fois, nous n'avons pas assez d'information.

Ce que signifie la chaîne \TT{kqvt} n'est pas clair. 

Le préfixe \TT{kq} peut signifier \emph{kernel} ou \emph{query}.

\TT{v} signifie apparemment \emph{version} et \TT{t}---\emph{type}?
Difficile à dire.

Une table avec un nom similaire se trouve dans \TT{kqf.o}:

\lstinputlisting[caption=kqf.o,style=customasmx86]{examples/oracle/1_version/9.lst}

Elle contient des informations à propos de tous les champs de la table \TT{X\$VERSION}.
La seule référence à cette table est dans la table \TT{kqftap}:

\begin{lstlisting}[caption=kqf.o]
.rodata:08042680                 kqftap_element <0, offset kqvt_c_0, offset kqvrow, 0> ; element 0x1f6
\end{lstlisting}

Il est intéressant de voir que cet élément ici est \TT{0x1f6th} (502nd), tout comme
le pointeur sur la chaîne \TT{X\$VERSION} dans la table \TT{kqftab}.

Sans doute que les tables \TT{kqftap} et \TT{kqftab} sont complémentaires l'une de
l'autre, tout comme \TT{kqfvip} et \TT{kqfviw}.

Nous voyons aussi un pointeur sur la fonction \TT{kqvrow()}. Enfin, nous obtenons
quelque chose d'utile!

Nous ajoutons donc ces tables à notre utilitaire \oracletables. Pour \TT{X\$VERSION}
nous obtenons:

\begin{lstlisting}[caption=Résultat de \OracleTablesName]
kqftab_element.name: [X$VERSION] ?: [kqvt] 0x4 0x4 0x4 0xc 0xffffc075 0x3
kqftap_param.name=[ADDR] ?: 0x917 0x0 0x0 0x0 0x4 0x0 0x0
kqftap_param.name=[INDX] ?: 0xb02 0x0 0x0 0x0 0x4 0x0 0x0
kqftap_param.name=[INST_ID] ?: 0xb02 0x0 0x0 0x0 0x4 0x0 0x0
kqftap_param.name=[BANNER] ?: 0x601 0x0 0x0 0x0 0x50 0x0 0x0
kqftap_element.fn1=kqvrow
kqftap_element.fn2=NULL
\end{lstlisting}

\myindex{tracer}
Avec l'aide de \tracer, il est facile de vérifier que cette fonction est appelée
6 fois par ligne (depuis la fonction \TT{qerfxFetch()}) lorsque l'on fait une requête
sur la table \TT{X\$VERSION}.

Lançons \tracer en mode \TT{cc} (il commente chaque instruction exécutée):

\begin{lstlisting}
tracer -a:oracle.exe bpf=oracle.exe!_kqvrow,trace:cc
\end{lstlisting}

\lstinputlisting[style=customasmx86]{examples/oracle/1_version/VERSION_kqvrow.asm}

Maintenant, il est facile de voir que le nombre est passé de l'extérieur. La fonction
renvoie une chaîne, construite comme ceci:

\begin{center}
\begin{tabular}{ | l | l | }
\hline                        
String 1	& Using \TT{vsnstr}, \TT{vsnnum}, \TT{vsnban} global variables. \\
                                & Calls \TT{sprintf()}. \\
String 2	& Calls \TT{kkxvsn()}. \\
String 3	& Calls \TT{lmxver()}. \\
String 4	& Calls \TT{npinli()}, \TT{nrtnsvrs()}. \\
String 5	& Calls \TT{lxvers()}. \\
\hline  
\end{tabular}
\end{center}

C'est ainsi que les fonctions correspondantes sont appelées pour déterminer la version de chaque module.

