\subsection{\TT{V\$VERSION} table in the \oracle}

\myindex{\oracle}
\myindex{Linux}
\myindex{Windows!ntoskrnl.exe}
\oracle 11.2 is a huge program, its main module \TT{oracle.exe} contain approx. 124,000 functions. For comparison, the Windows 7 x86 kernel (ntoskrnl.exe) contains approx. 11,000 functions and the Linux 3.9.8 kernel
(with default drivers compiled)---31,000 functions.

Let's start with an easy question. Where does \oracle get all this information, when we execute this simple statement in SQL*Plus:

\lstinputlisting{examples/oracle/1_version/0.txt}

And we get:

\lstinputlisting{examples/oracle/1_version/1.txt}

Let's start. Where in the \oracle can we find the string \TT{V\$VERSION}?

In the win32-version, \TT{oracle.exe} file contains the string,
it's easy to see.
But we can also use the object (.o) files from the Linux version of \oracle since, unlike the win32 version \TT{oracle.exe}, the function names (and global variables as well) are preserved there.

So, the \TT{kqf.o} file contains the \TT{V\$VERSION} string.
The object file is in the main Oracle-library \TT{libserver11.a}.

A reference to this text string can find in the \TT{kqfviw} table stored in the same file, \TT{kqf.o}:

\lstinputlisting[caption=kqf.o,style=customasmx86]{examples/oracle/1_version/2.lst}

By the way, often, while analyzing \oracle's internals, you may ask yourself, why are the names of the functions and global variable so weird.

Probably, because \oracle is a very old product and was developed in C in the 1980s.

And that was a time when the C standard guaranteed that the function names/variables can support only up to 6 characters inclusive: <<6 significant initial characters in an external identifier>>\footnote{\href{http://go.yurichev.com/17142}{Draft ANSI C Standard (ANSI X3J11/88-090) (May 13, 1988) (yurichev.com)}}

Probably, the table \TT{kqfviw} contains most (maybe even all) views prefixed with V\$, these are \emph{fixed views}, present all the time.
Superficially, by noticing the cyclic recurrence of data, we can easily see that each \TT{kqfviw} table element has 12 32-bit fields.
It is very simple to create a 12-elements structure in \IDA and apply it to all table elements.
As of \oracle version 11.2, there are 1023 table elements, i.e., in it are described 1023 of all possible \emph{fixed views}.

We are going to return to this number later.

As we can see, there is not much information in these numbers in the fields. The first field is always equals to the name of the view (without the terminating zero).
This is correct for each element. But this information is not very useful.

We also know that the information about all fixed views can be retrieved from a \emph{fixed view} named \TT{V\$FIXED\_VIEW\_DEFINITION}
(by the way, the information for this view is also taken from the \TT{kqfviw} and \TT{kqfvip} tables.)
Incidentally, there are 1023 elements in those too. Coincidence? No.

\lstinputlisting{examples/oracle/1_version/3.txt}

So, \TT{V\$VERSION} is some kind of a \emph{thunk view} for another view, named \TT{GV\$VERSION}, which is, in turn:

\lstinputlisting{examples/oracle/1_version/4.txt}

The tables prefixed with X\$ in the \oracle are service tables too, undocumented, cannot be changed by the user and are refreshed dynamically.

If we search for the text \\
\lstinputlisting{examples/oracle/1_version/5.txt}
...  in the \TT{kqf.o} file, we find it in the \TT{kqfvip} table:

\lstinputlisting[caption=kqf.o,style=customasmx86]{examples/oracle/1_version/6.lst}

The table appear to have 4 fields in each element. By the way, there are 1023 elements in it, again, the number we already know.

The second field points to another table that contains the table fields for this \emph{fixed view}.
As for \TT{V\$VERSION}, this table has only two elements, the first is 6 and the second is 
the \TT{BANNER} string (the number 6 is this string's length) and after, a \emph{terminating} element that contains 
0 and a \emph{null} C string:

\begin{lstlisting}[caption=kqf.o,style=customasmx86]
.rodata:080BBAC4 kqfv133_c_0 dd 6     ; DATA XREF: .rodata:08019574
.rodata:080BBAC8             dd offset _2__STRING_5017_0 ; "BANNER"
.rodata:080BBACC             dd 0
.rodata:080BBAD0             dd offset _2__STRING_0_0
\end{lstlisting}

By joining data from both \TT{kqfviw} and \TT{kqfvip} tables, we can get the SQL statements which are executed when the user wants to query information from a specific \emph{fixed view}.

So we can write an \oracletables program, to gather all this information from \oracle for Linux's object files.
For \TT{V\$VERSION}, we find this:

\begin{lstlisting}[caption=Result of \OracleTablesName]
kqfviw_element.viewname: [V$VERSION] ?: 0x3 0x43 0x1 0xffffc085 0x4
kqfvip_element.statement: [select  BANNER from GV$VERSION where inst_id = USERENV('Instance')]
kqfvip_element.params:
[BANNER] 
\end{lstlisting}

And:

\begin{lstlisting}[caption=Result of \OracleTablesName]
kqfviw_element.viewname: [GV$VERSION] ?: 0x3 0x26 0x2 0xffffc192 0x1
kqfvip_element.statement: [select inst_id, banner from x$version]
kqfvip_element.params:
[INST_ID] [BANNER] 
\end{lstlisting}

The \TT{GV\$VERSION} \emph{fixed view} is different from \TT{V\$VERSION} only in that it has one more field with the identifier \emph{instance}.

Anyway, we are going to stick with the \TT{X\$VERSION} table. Just like any other X\$-table, it is undocumented, however, we can query it:

\lstinputlisting{examples/oracle/1_version/7.txt}

This table has some additional fields, like \TT{ADDR} and \TT{INDX}.

While scrolling \TT{kqf.o} in \IDA we can spot another table that contains a pointer to the \TT{X\$VERSION} string, this is \TT{kqftab}:

\lstinputlisting[caption=kqf.o,style=customasmx86]{examples/oracle/1_version/8.lst}

There are a lot of references to the X\$-table names, apparently, to all \oracle 11.2 X\$-tables.
But again, we don't have enough information.

It's not clear what does the \TT{kqvt} string stands for. 

The \TT{kq} prefix may mean \emph{kernel} or \emph{query}. 

\TT{v} apparently stands for \emph{version} and \TT{t}---\emph{type}? 
Hard to say.

A table with a similar name can be found in \TT{kqf.o}:

\lstinputlisting[caption=kqf.o,style=customasmx86]{examples/oracle/1_version/9.lst}

It contains information about all fields in the \TT{X\$VERSION} table.
The only reference to this table is in the \TT{kqftap} table:

\begin{lstlisting}[caption=kqf.o]
.rodata:08042680                 kqftap_element <0, offset kqvt_c_0, offset kqvrow, 0> ; element 0x1f6
\end{lstlisting}

It is interesting that this element here is \TT{0x1f6th} (502nd), just like the pointer to the \TT{X\$VERSION} string in 
the \TT{kqftab} table.

Probably, the \TT{kqftap} and \TT{kqftab} tables complement each other, just like \TT{kqfvip} and \TT{kqfviw}.

We also see a pointer to the \TT{kqvrow()} function. Finally, we got something useful!

So we will add these tables to our \oracletables utility too. For \TT{X\$VERSION} we get:

\begin{lstlisting}[caption=Result of \OracleTablesName]
kqftab_element.name: [X$VERSION] ?: [kqvt] 0x4 0x4 0x4 0xc 0xffffc075 0x3
kqftap_param.name=[ADDR] ?: 0x917 0x0 0x0 0x0 0x4 0x0 0x0
kqftap_param.name=[INDX] ?: 0xb02 0x0 0x0 0x0 0x4 0x0 0x0
kqftap_param.name=[INST_ID] ?: 0xb02 0x0 0x0 0x0 0x4 0x0 0x0
kqftap_param.name=[BANNER] ?: 0x601 0x0 0x0 0x0 0x50 0x0 0x0
kqftap_element.fn1=kqvrow
kqftap_element.fn2=NULL
\end{lstlisting}

\myindex{tracer}
With the help of \tracer, it is easy to check that this function is called 6 times in row (from the \TT{qerfxFetch()} function) while querying the \TT{X\$VERSION} table.

Let's run \tracer in \TT{cc} mode (it comments each executed instruction):

\begin{lstlisting}
tracer -a:oracle.exe bpf=oracle.exe!_kqvrow,trace:cc
\end{lstlisting}

\lstinputlisting[style=customasmx86]{examples/oracle/1_version/VERSION_kqvrow.asm}

Now it is easy to see that the row number is passed from outside. The function returns the string, constructing it as follows:

\begin{center}
\begin{tabular}{ | l | l | }
\hline                        
String 1	& Using \TT{vsnstr}, \TT{vsnnum}, \TT{vsnban} global variables. \\
                                & Calls \TT{sprintf()}. \\
String 2	& Calls \TT{kkxvsn()}. \\
String 3	& Calls \TT{lmxver()}. \\
String 4	& Calls \TT{npinli()}, \TT{nrtnsvrs()}. \\
String 5	& Calls \TT{lxvers()}. \\
\hline  
\end{tabular}
\end{center}

That's how the corresponding functions are called for determining each module's version.

