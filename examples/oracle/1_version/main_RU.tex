\subsection{Таблица \TT{V\$VERSION} в \oracle}

\myindex{\oracle}
\myindex{Linux}
\myindex{Windows!ntoskrnl.exe}
\oracle 11.2 это очень большая программа, основной модуль \TT{oracle.exe} содержит около 124 тысячи функций. Для сравнения, ядро Windows 7 x64 (ntoskrnl.exe) --- около 11 тысяч функций, а ядро Linux 3.9.8 (с драйверами по умолчанию) --- 31 тысяч функций.

Начнем с одного простого вопроса. Откуда \oracle берет информацию, когда мы в SQL*Plus пишем вот такой вот простой запрос:

\lstinputlisting{examples/oracle/1_version/0.txt}

И получаем:

\lstinputlisting{examples/oracle/1_version/1.txt}

Начнем. Где в самом \oracle мы можем найти строку \TT{V\$VERSION}?

Для win32-версии, эта строка имеется в файле \TT{oracle.exe}, это легко увидеть.

Но мы также можем использовать объектные (.o) файлы от версии \oracle для Linux, потому что в них сохраняются имена функций и глобальных переменных, а в \TT{oracle.exe} для win32 этого нет.

Итак, строка \TT{V\$VERSION} имеется в файле \TT{kqf.o}, в самой главной Oracle-библиотеке \TT{libserver11.a}.

Ссылка на эту текстовую строку имеется в таблице \TT{kqfviw}, размещенной в этом же файле \TT{kqf.o}:

\lstinputlisting[caption=kqf.o,style=customasmx86]{examples/oracle/1_version/2.lst}

Кстати, нередко, при изучении внутренностей \oracle, появляется вопрос, почему имена функций и глобальных переменных такие странные.
Вероятно, дело в том, что \oracle очень старый продукт сам по себе и писался на Си еще в 1980-х.

А в те времена стандарт Си гарантировал поддержку имен переменных длиной только до шести символов включительно:
<<6 significant initial characters in an external identifier>>\footnote{\href{https://yurichev.com/ref/Draft%20ANSI%20C%20Standard%20(ANSI%20X3J11-88-090)%20(May%2013,%201988).txt}{Draft ANSI C Standard (ANSI X3J11/88-090) (May 13, 1988) (yurichev.com)}}

Вероятно, таблица \TT{kqfviw} содержащая в себе многие (а может даже и все) view с префиксом V\$, это служебные view (fixed views), присутствующие всегда.
Бегло оценив цикличность данных, мы легко видим, что в каждом элементе таблицы \TT{kqfviw} 12 32-битных полей.
В \IDA легко создать структуру из 12-и элементов и применить её ко всем элементам таблицы.
Для версии \oracle 11.2, здесь 1023 элемента в таблице, то есть, здесь описываются 1023 всех возможных \emph{fixed view}.
Позже, мы еще вернемся к этому числу.

Как видно, мы не очень много можем узнать чисел в этих полях. Самое первое поле всегда равно длине строки-названия view (без терминирующего ноля).

Это справедливо для всех элементов. Но эта информация не очень полезна.

Мы также знаем, что информацию обо всех fixed views можно получить из \emph{fixed view} под названием
\TT{V\$FIXED\_VIEW\_DEFINITION}
(кстати, информация для этого view также берется из таблиц \TT{kqfviw} и \TT{kqfvip}).
Между прочим, там тоже 1023 элемента. Совпадение? Нет.

\lstinputlisting{examples/oracle/1_version/3.txt}

Итак, \TT{V\$VERSION} это как бы \emph{thunk view} для другого, с названием \TT{GV\$VERSION}, который, в свою очередь:

\lstinputlisting{examples/oracle/1_version/4.txt}

Таблицы с префиксом X\$ в \oracle --- это также служебные таблицы, они не документированы,
не могут изменятся пользователем, и обновляются динамически.

Попробуем поискать текст \\
\lstinputlisting{examples/oracle/1_version/5.txt}
... в файле \TT{kqf.o} и находим ссылку на него в таблице \TT{kqfvip}:

\lstinputlisting[caption=kqf.o,style=customasmx86]{examples/oracle/1_version/6.lst}

Таблица, по всей видимости, имеет 4 поля в каждом элементе. Кстати, здесь так же 1023 элемента, уже знакомое нам число.

Второе поле указывает на другую таблицу, содержащую поля этого \emph{fixed view}.

Для \TT{V\$VERSION}, эта таблица только из двух элементов, первый это 6 и второй это строка 
\TT{BANNER} (число 6 это длина строки) и далее \emph{терминирующий} элемент содержащий 0 и \emph{нулевую} 
Си-строку:

\begin{lstlisting}[caption=kqf.o,style=customasmx86]
.rodata:080BBAC4 kqfv133_c_0 dd 6     ; DATA XREF: .rodata:08019574
.rodata:080BBAC8             dd offset _2__STRING_5017_0 ; "BANNER"
.rodata:080BBACC             dd 0
.rodata:080BBAD0             dd offset _2__STRING_0_0
\end{lstlisting}

Объединив данные из таблиц \TT{kqfviw} и \TT{kqfvip}, мы получим SQL-запросы, которые исполняются, когда пользователь хочет получить информацию из какого-либо \emph{fixed view}.

Напишем программу \oracletables, которая собирает всю эту информацию из объектных файлов от \oracle под Linux.

Для \TT{V\$VERSION}, мы можем найти следующее:

\begin{lstlisting}[caption=Результат работы \OracleTablesName]
kqfviw_element.viewname: [V$VERSION] ?: 0x3 0x43 0x1 0xffffc085 0x4
kqfvip_element.statement: [select  BANNER from GV$VERSION where inst_id = USERENV('Instance')]
kqfvip_element.params:
[BANNER] 
\end{lstlisting}

И:

\begin{lstlisting}[caption=Результат работы \OracleTablesName]
kqfviw_element.viewname: [GV$VERSION] ?: 0x3 0x26 0x2 0xffffc192 0x1
kqfvip_element.statement: [select inst_id, banner from x$version]
kqfvip_element.params:
[INST_ID] [BANNER] 
\end{lstlisting}

\emph{Fixed view} \TT{GV\$VERSION} отличается от \TT{V\$VERSION} тем, что содержит еще и поле отражающее идентификатор \emph{instance}.

Но так или иначе, мы теперь упираемся в таблицу \TT{X\$VERSION}. Как и прочие X\$-таблицы, она не документирована, однако, мы можем оттуда что-то прочитать:

\lstinputlisting{examples/oracle/1_version/7.txt}

Эта таблица содержит дополнительные поля вроде \TT{ADDR} и \TT{INDX}.

Бегло листая содержимое файла \TT{kqf.o} в \IDA мы можем увидеть еще одну таблицу где есть ссылка на строку \TT{X\$VERSION}, это \TT{kqftab}:

\lstinputlisting[caption=kqf.o,style=customasmx86]{examples/oracle/1_version/8.lst}

Здесь очень много ссылок на названия X\$-таблиц, вероятно, на все те что имеются в \oracle этой версии.

Но мы снова упираемся в то что не имеем достаточно информации.
Не ясно, что означает строка \TT{kqvt}.
 
Вообще, префикс \TT{kq} может означать \emph{kernel} и \emph{query}.
 
\TT{v}, может быть, \emph{version}, а \TT{t} --- \emph{type}?
 
Сказать трудно.

Таблицу с очень похожим названием мы можем найти в \TT{kqf.o}:

\lstinputlisting[caption=kqf.o,style=customasmx86]{examples/oracle/1_version/9.lst}

Она содержит информацию об именах полей в таблице \TT{X\$VERSION}.
Единственная ссылка на эту таблицу имеется в таблице \TT{kqftap}:

\begin{lstlisting}[caption=kqf.o]
.rodata:08042680                 kqftap_element <0, offset kqvt_c_0, offset kqvrow, 0> ; element 0x1f6
\end{lstlisting}

Интересно что здесь этот элемент проходит так же под номером \TT{0x1f6} (502-й), как и ссылка на строку 
\TT{X\$VERSION} в таблице \TT{kqftab}.

Вероятно, таблицы \TT{kqftap} и \TT{kqftab} дополняют друг друга, как и \TT{kqfvip} и \TT{kqfviw}.

Мы также видим здесь ссылку на функцию с названием \TT{kqvrow()}. А вот это уже кое-что!

Сделаем так чтобы наша программа \oracletables могла дампить и эти таблицы. Для \TT{X\$VERSION} получается:

\begin{lstlisting}[caption=Результат работы \OracleTablesName]
kqftab_element.name: [X$VERSION] ?: [kqvt] 0x4 0x4 0x4 0xc 0xffffc075 0x3
kqftap_param.name=[ADDR] ?: 0x917 0x0 0x0 0x0 0x4 0x0 0x0
kqftap_param.name=[INDX] ?: 0xb02 0x0 0x0 0x0 0x4 0x0 0x0
kqftap_param.name=[INST_ID] ?: 0xb02 0x0 0x0 0x0 0x4 0x0 0x0
kqftap_param.name=[BANNER] ?: 0x601 0x0 0x0 0x0 0x50 0x0 0x0
kqftap_element.fn1=kqvrow
kqftap_element.fn2=NULL
\end{lstlisting}

\myindex{tracer}
При помощи \tracer, можно легко проверить, что эта функция вызывается 6 раз кряду (из функции \TT{qerfxFetch()}) при получении строк из \TT{X\$VERSION}.

Запустим \tracer в режиме \TT{cc} (он добавит комментарий к каждой исполненной инструкции):

\begin{lstlisting}
tracer -a:oracle.exe bpf=oracle.exe!_kqvrow,trace:cc
\end{lstlisting}

\lstinputlisting[style=customasmx86]{examples/oracle/1_version/VERSION_kqvrow.asm}

Так можно легко увидеть, что номер строки таблицы задается извне. Сама функция возвращает строку, формируя её так:

\begin{center}
\begin{tabular}{ | l | l | }
\hline                        
Строка 1	& Использует глобальные переменные \TT{vsnstr}, \TT{vsnnum}, \TT{vsnban}. \\
                                & Вызывает \TT{sprintf()}. \\
Строка 2	& Вызывает \TT{kkxvsn()}. \\
Строка 3	& Вызывает \TT{lmxver()}. \\
Строка 4	& Вызывает \TT{npinli()}, \TT{nrtnsvrs()}. \\
Строка 5	& Вызывает \TT{lxvers()}. \\
\hline  
\end{tabular}
\end{center}

Так вызываются соответствующие функции для определения номеров версий отдельных модулей.

