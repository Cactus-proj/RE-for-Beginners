\newcommand{\AcronymsUsed}{Acronymes utilisés}

\newcommand{\TitleRE}{Rétro-ingénierie pour Débutants}

\newcommand{\TitleUAL}{Comprendre le langage d'assemblage}

\newcommand{\AUTHOR}{Dennis Yurichev}

\newcommand{\figname}{fig.\xspace}
\newcommand{\listingname}{listado.\xspace}
\newcommand{\Sourcecode}{Code source\xspace}
\newcommand{\Seealso}{Voir également\xspace}
\newcommand{\tableheader}{\headercolor{} offset & \headercolor{} description }
% instructions descriptions
\newcommand{\ASRdesc}{décalage arithmétique vers la droite}

% x86 registers tables
\newcommand{\RegHeaderTop}{ \multicolumn{8}{ | c | }{ Octet d'indice } }
% TODO: non-overlapping color!
\newcommand{\RegHeader}{7 & 6 & 5 & 4 & 3 & 2 & 1 & 0}

\newcommand{\ReturnAddress}{Adresse de retour}

\newcommand{\localVariable}{variable locale}

\newcommand{\savedValueOf}{valeur enregistrée de}

% for index
\newcommand{\GrepUsage}{Utilisation de grep}
\newcommand{\SyntacticSugar}{Sucre syntaxique}
\newcommand{\CompilerAnomaly}{Anomalies du compilateur}
\newcommand{\CLanguageElements}{Éléments du langage C}
\newcommand{\CStandardLibrary}{Bibliothèque standard C}
\newcommand{\Instructions}{Instructions}
\newcommand{\Pseudoinstructions}{Pseudo-instructions}
\newcommand{\Prefixes}{Préfixes}

\newcommand{\Flags}{Flags}
\newcommand{\Registers}{Registres}
\newcommand{\registers}{registres}
\newcommand{\Stack}{Pile}
\newcommand{\Recursion}{Récursivité}
\newcommand{\RAM}{RAM}
\newcommand{\ROM}{ROM}
\newcommand{\Pointers}{Pointeurs}
\newcommand{\BufferOverflow}{Débordement de tampon}

\newcommand{\Conclusion}{Conclusion}

\newcommand{\Exercise}{Exercice}
\newcommand{\Exercises}{Exercices\xspace}
\newcommand{\Arrays}{Tableaux}
\newcommand{\Cpp}{C++\xspace}
\newcommand{\CCpp}{C/C++\xspace}
\newcommand{\NonOptimizing}{sans optimisation\xspace}
\newcommand{\Optimizing}{avec optimisation\xspace}
\newcommand{\ARMMode}{Mode ARM\xspace}
\newcommand{\ThumbMode}{Mode Thumb\xspace}
\newcommand{\ThumbTwoMode}{Mode Thumb-2\xspace}

\newcommand{\DataProcessingInstructionsFootNote}{Ces instructions sont également appelées \q{instructions de traitement de données}}

% for .bib files
\newcommand{\AlsoAvailableAs}{Aussi disponible en\xspace}

% section names
\newcommand{\ShiftsSectionName}{Décalages}
\newcommand{\SignedNumbersSectionName}{Représentations des nombres signés}
\newcommand{\HelloWorldSectionName}{Hello, world!}
\newcommand{\SwitchCaseDefaultSectionName}{switch()/case/default}
\newcommand{\PrintfSeveralArgumentsSectionName}{printf() avec plusieurs arguments}
\newcommand{\BitfieldsChapter}{Manipulation de bits spécifiques}
\newcommand{\ArithOptimizations}{Remplacement de certaines instructions arithmétiques par d'autres}

\newcommand{\FPUChapterName}{Unité à virgule flottante}
\newcommand{\MoreAboutStrings}{Plus d'information sur les chaînes}
\newcommand{\DivisionByMultSectionName}{Division par la multiplication}
\newcommand{\Answer}{Réponse}
\newcommand{\WhatThisCodeDoes}{Que fait ce code ?}
\newcommand{\WorkingWithFloatAsWithStructSubSubSectionName}{Travailler avec le type float comme une structure}

\newcommand{\MinesweeperWinXPExampleChapterName}{Démineur (Windows XP)}
\newcommand{\StructurePackingSectionName}{Organisation des champs dans la structure}
\newcommand{\StructuresChapterName}{Structures}
\newcommand{\PICcode}{code indépendant de la position}
\newcommand{\CapitalPICcode}{Code indépendant de la position}
\newcommand{\Loops}{Boucles}

% C
\newcommand{\PostIncrement}{Post-incrémentation}
\newcommand{\PostDecrement}{Post-décrémentation}
\newcommand{\PreIncrement}{Pré-incrémentation}
\newcommand{\PreDecrement}{Pré-décrémentation}

% MIPS
\newcommand{\GlobalPointer}{Pointeur Global}

\newcommand{\garbage}{déchets}
\newcommand{\IntelSyntax}{Syntaxe Intel}
\newcommand{\ATTSyntax}{Syntaxe AT\&T}
\newcommand{\randomNoise}{bruit aléatoire}
\newcommand{\Example}{Exemple}
\newcommand{\argument}{argument}
\newcommand{\MarkedInIDAAs}{marqué dans \IDA comme}
\newcommand{\stepover}{enjamber}
\newcommand{\ShortHotKeyCheatsheet}{Anti-sèche des touches de raccourci}

\newcommand{\assemblyOutput}{résultat en sortie de l'assembleur}

% ML prefix is for multi-lingual words and sentences:
\newcommand{\MLHeap}{Heap}
\newcommand{\MLStack}{Pile}
\newcommand{\MLStackOverflow}{Débordement de pile}
\newcommand{\MLStartOfHeap}{Début du heap}
\newcommand{\MLStartOfStack}{Début de la pile}
\newcommand{\MLinputA}{entrée A}
\newcommand{\MLinputB}{entrée B}
\newcommand{\MLoutput}{sortie}
\newcommand{\SoftwareCracking}{cracking de logiciel}

%\newcommand{\EMAILPRI}{<first\_name @ last\_name . com>}
%\newcommand{\EMAILS}{<first\_name @ last\_name . com> / <first\_name . last\_name @ gmail . com>}
%\newcommand{\EMAILPRI}{<dennis@yurichev.com>}
%\newcommand{\EMAILS}{<dennis@yurichev.com> / <dennis.yurichev@gmail.com>}
\newcommand{\EMAILPRI}{<book@beginners.re>}
\newcommand{\EMAILS}{<book@beginners.re>}
