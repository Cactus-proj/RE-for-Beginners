\mysection{Exemple de calcul de CRC32}
\myindex{CRC32}
\label{sec:CRC32}

\newcommand{\URLCRCSRC}{\url{http://burtleburtle.net/bob/c/crc.c}}

C'est une technique très répandue de calcul de hachage basée sur une table
CRC32\footnote{Le code source provient d'ici: \URLCRCSRC}.

\lstinputlisting[style=customc]{\CURPATH/CRC.c}

\myindex{\CLanguageElements!for}

Nous sommes seulement intéressés par la fonction \TT{crc()}.
À propos, faîtes attention aux deux déclarations d'initialisation dans la boucle
\TT{for()}: \TT{hash=len, i=0}.
Le standard \CCpp permet ceci, bien sûr.
Le code généré contiendra deux opérations dans la partie d'initialisation de la boucle,
au lieu d'une.

Compilons-le dans MSVC avec l'optimisation (\Ox).
Dans un soucis de concision, seule la fonction \TT{crc()} est listée ici, avec mes
commentaires.

\lstinputlisting[style=customasmx86]{\CURPATH/CRC_2_FR.asm}

Essayons la même chose dans GCC 4.4.1 avec l'option \Othree:

\lstinputlisting[style=customasmx86]{\CURPATH/CRC_gcc_O3_FR.asm}

\myindex{x86!\Instructions!NOP}
\myindex{x86!\Instructions!LEA}

% \vref bug, please ignore "\\"...
GCC a aligné le début de la boucle sur une limite de 8-octet en ajoutant \NOP et
\TT{lea esi, [esi+0]} (qui est aussi une opération sans effet).\\
\\
Vous pouvez en lire plus à ce sujet dans la section npad~(\myref{sec:npad}).
