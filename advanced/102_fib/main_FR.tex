\mysection{Suite de Fibonacci}

Un autre exemple très utilisé dans les livres de programmation est la fonction récursive
qui génère les termes de la suite de Fibonacci\footnote{\url{http://oeis.org/A000045}}.
Cette suite est très simple: chaque nombre consécutif est la somme des deux précédents.
Les deux premiers termes sont 0 et 1 ou 1 et 1.

La suite commence comme ceci:

\begin{center}
$0, 1, 1, 2, 3, 5, 8, 13, 21, 34, 55, 89, 144, 233, 377, 610, 987, 1597, 2584, 4181 ...$
\end{center}

\subsection{Exemple \#1}

L'implémentation est simple. Ce programme génère la suite jusqu'à 21.

\lstinputlisting[style=customc]{\CURPATH/fib.c}

\lstinputlisting[caption=MSVC 2010 x86,style=customasmx86]{\CURPATH/fib.asm}

Nous allons illustrer les frames de pile avec ceci.

\clearpage

Chargeons cet exemple dans \olly et traçons jusqu'au dernier appel de \ttf{}:

\begin{figure}[H]
\centering
\myincludegraphics{\CURPATH/olly.png}
\caption{\olly: dernier appel de \ttf{}}
\label{fig:fib_olly}
\end{figure}

\clearpage
Examinons plus attentivement la pile.
Les commentaires ont été ajoutés par l'auteur de ce livre
\footnote{ À propos, il est possible de sélectionner plusieurs entrées dans \olly
et de les copier dans le presse-papier (Crtl-C).
C'est ce qui a été fait par l'auteur pour cet exemple.}:

\lstinputlisting{\CURPATH/stack_FR.txt}

La fonction est récursive \footnote{i.e., elle s'appelle elle-même}, donc la pile
ressemble à un \q{sandwich}.

Nous voyons que l'argument \emph{limit} est toujours le même (\TT{0x14} ou 20), mais
que les arguments $a$ et $b$ sont différents pour chaque appel.

Il y a aussi ici le \ac{RA}-s et la valeur sauvée de \EBP.
\olly est capable de déterminer les frames basés sur EBP, donc il les dessine ces
accolades. Les valeurs dans chaque accolade constituent la \glslink{stack frame}{frame de pile},
autrement dit, la zone de la pile qu'un appel de fonction utilise comme espace dédié.

Nous pouvons aussi dire que chaque appel de fonction ne doit pas accéder les éléments
de la pile au delà des limites de son bloc (en excluant les arguments de la fonction),
bien que cela soit techniquement possible.

C'est généralement vrai, à moins que la fonction n'ait des bugs.

Chaque valeur sauvée de \EBP est l'adresse de la \glslink{stack frame}{structure
de pile locale} précédente:
c'est la raison pour laquelle certains débogueurs peuvent facilement diviser la pile
en blocs et afficher chaque argument de la fonction.
% TODO add about StackWalk (MSDN)

Comme nous le voyons ici, chaque exécution de fonction prépare les arguments pour
l'appel de fonction suivant.

À la fin, nous voyons les 3 arguments de \main.
\TT{argc} vaut 1 (oui, en effet, nous avons lancé le programme sans argument sur
la ligne de commande).

Ceci peut conduire facilement à un débordement de pile: il suffit de supprimer (ou
commenter) le test de la limite et ça va planter avec l'exception \TT{0xC00000FD} (stack overflow).

\subsection{Exemple \#2}

Ma fonction a quelques redondances, donc ajoutons une nouvelle variable locale \emph{next}
et remplaçons tout les \q{a+b} avec elle:

\lstinputlisting[style=customc]{\CURPATH/fib2.c}

C'est la sortie de MSVC sans optimisation, donc la variable \emph{next} est allouée
sur la pile locale:

\lstinputlisting[caption=MSVC 2010 x86,style=customasmx86]{\CURPATH/fib2.asm}

\clearpage
Chargeons-le à nouveau dans \olly:

\begin{figure}[H]
\centering
\myincludegraphics{\CURPATH/olly2.png}
\caption{\olly: dernier appel de \ttf{}}
\label{fig:fib_olly2}
\end{figure}

Maintenant, la variable \emph{next} est présente dans chaque frame.

\clearpage

Examinons plus attentivement la pile. L'auteur a de nouveau ajouté ses commentaires:

\lstinputlisting{\CURPATH/stack2_FR.txt}

% FIXME what a word! hmmm...
Voici ce que l'on voit: la valeur \emph{next} est calculée dans chaque appel de la fonction,
puis passée comme argument $b$ au prochain appel.

\subsection{Résumé}

\label{Recursion_and_tail_call}
\myindex{\Recursion}
Les fonctions récursives sont esthétiquement jolies, mais techniquement elles
peuvent dégrader les performances à cause de leur usage intensif de la pile.
Quiconque qui écrit du code dont la perfomance est critique devrait probablement
éviter la récursion.

Par exemple, j'ai écrit une fois une fonction pour chercher un n\oe{}ud particulier
dans un arbre binaire.
Bien que la fonction récursive avait l'air élégante, il y avait du temps passé à
chaque appel de fonction pour le prologue et l'épilogue, elle fonctionnait deux ou
trois fois plus lentement que l'implémentation itérative (sans récursion).

\newcommand{\FnFP}{\footnote{LISP, Python, Lua, etc.}}
\myindex{\Recursion!Tail recursion}

À propos, c'est la raison pour laquelle certains compilateurs fonctionnels \ac{PL}\FnFP{}
(où la récursion est très utilisée) utilisent les \glslink{tail call}{appels terminaux}.
Nous parlons d'appel terminal lorsqu'une fonction a un seul appel à elle-même, situé
à sa fin, comme:

\begin{lstlisting}[caption=Scheme{,} exemple copié/collé depuis Wikipédia]
;; factorial : nombre -> nombre
;; pour calculer le produit de tous les entiers
;; positifs inférieurs ou égaux à n.
(define (factorial n)
 (if (= n 1)
     1
	 (* n (factorial (- n 1)))))
\end{lstlisting}

Les appels terminaux sont importants car le compilateur peut retravailler facilement
ce code en un code itératif, pour supprimer la récursion.

