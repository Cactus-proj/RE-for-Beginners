\mysection{Exercice: un peu plus loin avec les pointeur et les unions}

Ce code a été copié/collé de \emph{dwm}\footnote{\url{https://dwm.suckless.org/}},
probablement le plus petit window manager sur Linux de tous les temps.

Le problème: les frappes de l'utilisateur au clavier doivent être réparties aux diverses
fonctions de \emph{dwm}.
Ceci est en général résolu en utilisant un gros \emph{switch()}.
Apparemment, le créateur de \emph{dwm} a voulu rendre le code soigné et modifiable
par les utilisateurs:

\lstinputlisting[style=customc]{\CURPATH/dwm.c}

Pour chaque touche frappée (ou raccourci), une fonction est définie.
Encore mieux: un paramètre (ou argument) peut être passé à une fonction dans chaque
cas.
Mais les paramètres peuvent avoir des types variés.
Donc une \emph{union} est utilisée ici.
Une valeur du type requis est mise dans la table.
Chaque fonction prend ce dont elle a besoin.

À titre d'exercice, essayez d'écrire un code comme cela, ou plongez-vous dans \emph{dwm}
et voyez comment l'union est passée aux fonctions et gérée.

