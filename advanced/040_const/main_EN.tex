\mysection{const correctness}
\myindex{\CLanguageElements!const}
\label{const_in_rdata}

This is undeservedly underused feature of many programming languages.
Read here about its importance:
\href{https://isocpp.org/wiki/faq/const-correctness}{1},
\href{https://stackoverflow.com/questions/136880/sell-me-on-const-correctness}{2}.

Ideally, everything you don't modify should have \emph{const} modifier.

Interestingly, how \emph{const correctness} is implemented at low level.
There are no runtime checks of local \emph{const} variables and function arguments (only compile-time checks).
But global variables of such a type are to be allocated in read-only data segments.

This is example is to be crashed, because if compiled by MSVC for win32,
the $a$ global variable is allocated in \verb|.rdata| read-only segment:

\lstinputlisting[style=customc]{\CURPATH/ex1.c}

C strings also have \verb|const char*| type.
You can't modify them:

\lstinputlisting[style=customc]{\CURPATH/ex2.c}

This code will crash on Linux (``segmentation fault'') and on Windows if compiled by MinGW.

GCC for Linux places all text strings info \TT{.rodata} data segment, which is explicitly read-only
(``read only data''):

\lstinputlisting{\CURPATH/ex2.txt}

When the \verb|alter_string()| function tries to write there, exception occurred.

Things are different in the code generated by MSVC, strings are located in \TT{.data} segment, which has no \TT{READONLY} flag.
Is this the MSVC's problem?

\lstinputlisting{\CURPATH/ex22.txt}

However, MinGW hasn't this fault and allocates text strings in \verb|.rdata| segment.

\subsection{Overlapping const strings}
\label{use_parts_of_C_strings}

The fact that an \emph{anonymous} C-string has \emph{const} type (\myref{string_is_const_char}), 
and that C-strings allocated in constants segment are guaranteed to be immutable, has an interesting consequence:
the compiler may use a specific part of the string.

Let's try this example:

\begin{lstlisting}[style=customc]
#include <stdio.h>

int f1()
{
	printf ("world\n");
}

int f2()
{
	printf ("hello world\n");
}

int main()
{
	f1();
	f2();
}
\end{lstlisting}

Common \CCpp{}-compilers (including MSVC) allocate two strings, but let's see what GCC 4.8.1 does:

\lstinputlisting[caption=GCC 4.8.1 + IDA listing,style=customasmx86]{\CURPATH/two_strings.asm}

Indeed: when we print the \q{hello world} string
these two words are positioned in memory adjacently and \puts called from \GTT{f2()}
function is not aware that this string is divided. 
In fact, it's not divided; it's divided only \emph{virtually}, in this listing.

When \puts is called from \GTT{f1()}, it uses the \q{world} string plus a zero byte. \puts is not aware that there is something before this string!

This clever trick is often used by at least GCC and can save some memory.
This is close to \emph{string interning}.

Another related example is here: \myref{strstr_example}.



