\mysection{const correctness}
\myindex{\CLanguageElements!const}
\label{const_in_rdata}

Ceci est une fonctionnalité indûment sous-utilisée de nombreux langages de programmation.
Pour en savoir plus à ce sujet:
\href{https://isocpp.org/wiki/faq/const-correctness}{1},
\href{https://stackoverflow.com/questions/136880/sell-me-on-const-correctness}{2}.

Idéalement, tout ce que vous ne modifiez pas devrait avoir le modificateur \emph{const}.
Il est intéressant de savoir comment le \emph{const correctness} est implémenté
à bas niveau.
Il n'y a pas de vérification des variables ni des arguments de fonction  \emph{const}
lors de l'exécution (seulement des vérifications lors de la compilation).
Mais les variables globales de ce type sont allouées dans le segment de données en
lecture seule.

Cet exemple va planter, car il est compilé par MSVC pour win32, la variable globale
$a$ est allouée dans le segment de données \verb|.rdata| en lecture seule:

\lstinputlisting[style=customc]{\CURPATH/ex1.c}

Les chaînes C \emph{anonymes} (non liées à un nom de variable) ont aussi un type
\verb|const char*|.
Vous ne pouvez pas les modifier:

\lstinputlisting[style=customc]{\CURPATH/ex2.c}

Ce code va planter sur Linux (``segmentation fault'') et sur Windows si il compilé
par MinGW.

GCC pour Linux met toutes les chaînes de texte dans le segment de données \TT{.rodata},
qui est explicitement en lecture seule (``read only data''):

\lstinputlisting{\CURPATH/ex2.txt}

Lorsque la fonction \verb|alter_string()| essaye d'y écrire, une exception se produit.

Les choses sont différentes dans le code généré par MSVC, les chaînes sont situées
dans le segment \TT{.data}, qui n'a pas de flag \TT{READONLY}.
Les développeurs de MSVC ont-ils fait un faux pas?

\lstinputlisting{\CURPATH/ex22.txt}

Toutefois, MinGW n'a pas cette erreur et alloue les chaînes de texte dans le segment
\verb|.rdata|.

% TBT \subsection{Overlapping const strings}
\subsection{GCC---encore une chose}
\label{use_parts_of_C_strings}

Le fait est qu'une chaîne C \emph{anonyme} a un type \emph{const} (\myref{string_is_const_char}),
et que les chaînes C allouées dans le segment des constantes sont garanties d'être immuables,
a cette conséquence intéressante:
Le compilateur peut utiliser une partie spécifique de la chaîne.

Voyons cela avec un exemple:

\begin{lstlisting}[style=customc]
#include <stdio.h>

int f1()
{
	printf ("world\n");
}

int f2()
{
	printf ("hello world\n");
}

int main()
{
	f1();
	f2();
}
\end{lstlisting}

La plupart des compilateurs \CCpp{} (MSVC inclus) allouent deux chaînes, mais voyons ce que fait GCC 4.8.1:

\lstinputlisting[caption=GCC 4.8.1 + IDA,style=customasmx86]{\CURPATH/two_strings.asm}

Effectivement: lorsque nous affichons la chaîne \q{hello world} ses deux mots sont positionnés
consécutivement en mémoire et l'appel à \puts depuis la fonction \GTT{f2()}
n'est pas au courant que la chaîne est divisée.
En fait, elle n'est pas divisée; elle l'est \emph{virtuellement}, dans ce listing.

Lorsque \puts est appelé depuis \GTT{f1()}, il utilise la chaîne \q{world} ainsi qu'un
octet à zéro. \puts ne sait pas qu'il y a quelque chose avant cette chaîne!

Cette astuce est souvent utilisée, au moins par GCC, et permet d'économiser de la mémoire.
C'est proche du \emph{string interning}. % TODO clarify imbrication de chaînes ?

Un autre exemple concernant ceci se trouve là: \myref{strstr_example}.



