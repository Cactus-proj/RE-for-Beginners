\subsection{Еще одна оптимизация циклов}

Если вы обрабатываете все элементы некоторого массива, который находится в глобальной памяти, компилятор может оптимизировать
это.
Например, вычисляем сумму всех элементов массива из 128-и \emph{int}-ов:

\begin{lstlisting}[style=customc]
#include <stdio.h>

int a[128];

int sum_of_a()
{
	int rt=0;
	
	for (int i=0; i<128; i++)
		rt=rt+a[i];

	return rt;
};

int main()
{
	// инициализация
	for (int i=0; i<128; i++)
		a[i]=i;
	
	// вычисляем сумму
	printf ("%d\n", sum_of_a());
};
\end{lstlisting}

Оптимизирующий GCC 5.3.1 (x86) может сделать так (\IDA):

\lstinputlisting[style=customasmx86]{advanced/500_loop_optimizations/tmp1.lst}

И что же такое \TT{\_\_libc\_start\_main@@GLIBC\_2\_0} на \TT{0x080484C5}?
Это метка, находящаяся сразу за концом массива \TT{a[]}.
Эта ф-ция может быть переписана так:

\begin{lstlisting}[style=customc]
int sum_of_a_v2()
{
	int *tmp=a;
	int rt=0;
	
	do
	{
		rt=rt+(*tmp);
		tmp++;
	}
	while (tmp<(a+128));

	return rt;
};
\end{lstlisting}

Первая версия имеет счетчик \emph{i}, и адрес каждого элемента массива вычисляется на каждой итерации.
Вторая версия более оптимизирована: указатель на каждый элемент массива всегда готов, и продвигается на 4 байта вперед
на каждой итерации.
Как проверить, закончился ли цикл?
Просто сравните указатель с адресом сразу за концом массива, который, как случилось в нашем случае, это просто адрес
импортируемой из Glibc 2.0 ф-ции \TT{\_\_libc\_start\_main()}.
Такой код иногда сбивает с толку, и это очень популярный оптимизационный трюк, поэтому я сделал этот пример.

Моя вторая версия очень близка к тому, что сделал GCC, и когда я компилирую её, код почти такой как и в первой версии,
но две первых инструкции поменены местами:

\lstinputlisting[style=customasmx86]{advanced/500_loop_optimizations/tmp2.lst}

Надо сказать, эта оптимизация возможна если компилятор, во время компиляции, может расчитать адрес за концом массива.
Это случается если массив глобальный и его размер фиксирован.

Хотя, если адрес массива не известен во время компиляции, но его размер фиксирован, адрес метки за концом массива
можно вычислить в начале цикла.
% FIXME <!-- \ref{} to example -->

