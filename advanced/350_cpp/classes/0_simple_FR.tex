\subsubsection{Un exemple simple}

En interne, la représentation des classes \Cpp est presque la même que les structures.

Essayons un exemple avec deux variables, deux constructeurs et une méthode:

\lstinputlisting[style=customc]{\CURPATH/classes/simple_FR.cpp}

\myparagraph{MSVC: x86}

Voici à quoi ressemble la fonction \main, traduite en langage d'assemblage:

\lstinputlisting[caption=MSVC,style=customasmx86]{\CURPATH/classes/simple_1_msvc.asm}

Voici ce qui se passe.
Pour chaque objet (instance de la classe $c$) 8 octets sont alloués, exactement la
taille requise pour stocker les deux variables.

Pour \emph{c1} un constructeur par défaut sans argument \TT{??0c@@QAE@XZ} est appelé.
Pour \emph{c2} un autre constructeur \TT{??0c@@QAE@HH@Z} est appelé et deux nombres
sont passés comme arguments.

\myindex{thiscall}
\label{thiscall}
\myindex{x86!\Registers!ECX}

Un pointeur sur l'objet (\ITthis en terminologie \Cpp) est passé dans le registre \ECX.
Ceci est appelé thiscall~(\myref{thiscall})---la méthode pour passer un pointeur
à l'objet.

MSVC le fait en utilisant le regsitre \ECX. Inutile de le dire, ce n'est pas une
méthode standardisée, d'autres compilateurs peuvent le faire différemment, e.g.,
par le premier argument de la fonction (comme GCC).

\label{namemangling}
\myindex{Name mangling}
\myindex{OOP!Polymorphism}

Pourquoi est-ce que ces fonctions ont un nom aussi étrange?
C'est le \gls{name mangling}.

Une classe \Cpp peut contenir plusieurs méthodes partageant le même nom mais ayant
des arguments différents---c'est le polymorphisme.
Et bien sûr, différentes classes peuvent avoir leurs propres méthodes avec le même nom.

\myindex{Linker}

Le \emph{Name mangling} nous permet d'encoder le nom de la classe + le nom de la méthode +
tous les types des arguments de la méthode dans une chaîne ASCII, qui est ensuite
utilisée comme le nom interne de la fonction.
C'est ainsi car ni le linker, ni le chargeur de DLL de l'\ac{OS} (les mangled names
peuvent aussi se trouver parmi les exports de DLL) n'ont conscience de \Cpp ou de
l'\ac{OOP}.

La fonction \TT{dump()} est appelée deux fois.

Maintenant, regardons le code du constructeur:

\lstinputlisting[caption=MSVC,style=customasmx86]{\CURPATH/classes/simple_2_msvc.asm}

Les constructeurs sont juste des fonctions, ils utilisent un pointeur sur la structure
dans \ECX, en copiant le pointeur dans leur propre variable locale, toutefois, ce
n'est pas nécessaire.

D'après le standard (C++11 12.1) nous savons que les constructeurs n'ont pas l'obligation
de renvoyer une valeur.

En fait, en interne, les constructeurs renvoient un pointeur sur l'objet nouvellement
créé, i.e., \ITthis.

Maintenant, la méthode \TT{dump()}:

\lstinputlisting[caption=MSVC,style=customasmx86]{\CURPATH/classes/simple_3_msvc.asm}

Assez simple: \TT{dump()} prend un pointeur sur la structure qui contient les deux
\Tint dans \ECX, prend les deux valeurs et les passe à \printf.

Le code est bien plus court s'il est compilé avec les optimisations (\Ox):

\lstinputlisting[caption=MSVC,style=customasmx86]{\CURPATH/classes/simple_4_msvc_Ox.asm}

\myindex{x86!\Instructions!RET}

C'est tout. L'autre chose que nous devons noter est que le \glslink{stack pointer}{pointeur de pile}
n'a pas été corrigé avec \TT{add esp, X} après l'appel du constructeur.
En même temps, le constructeur a \TT{ret 8} au lieu de \RET à la fin.

\myindex{thiscall}

C'est parce que la convention d'appel thiscall~(\myref{thiscall}) est utilisée ici,
qui, comme avec la méthode stdcall~(\myref{sec:stdcall}), offre à l'\glslink{callee}{appelée}
la possibilité de corriger la pile au lieu de l'\glslink{caller}{appelante}.
L'instruction \TT{ret x} ajoute \TT{X} à la valeur de \ESP, puis passe le contrôle
à la fonction \glslink{caller}{appelante}.

Regardez la section parlant des conventions d'appel~(\myref{sec:callingconventions}).

Il faut également noter que le compilateur décide quand appeler le constructeur et
le destructeur---mais nous le savons déjà des bases du langage \Cpp.

\myparagraph{MSVC: x86-64}
\label{simple_CPP_MSVC_x64}

Comme nous le savons déjà, les 4 premiers arguments de fonction en x86-64 sont passés
dans les registres \RCX, \RDX, \Reg{8} et \Reg{9}, tous les autres---par la pile.

Néanmoins, le pointeur \ITthis sur l'objet est passé dans \RCX, le premier argument
de la méthode dans \RDX, etc.
Nous pouvons le voir dans les entrailles de la méthode \TT{c(int a, int b)}:

\lstinputlisting[caption=MSVC 2012 x64 \Optimizing,style=customasmx86]{\CURPATH/classes/simple_MSVC_x64_FR.asm}

Le type de donnée \Tint est toujours 32-bit en x64
\footnote{Apparemment, pour faciliter le portage de code 32-bit \CCpp en x64},
c'est donc pourquoi les parties 32-bit des registres sont utilisées ici.

Nous voyons également \TT{JMP printf} au lieu de \RET dans la méthode \TT{dump()},
\emph{astuce} que nous avons déjà vu plus tôt: \myref{JMP_instead_of_RET}.

\myparagraph{GCC: x86}

C'est presque la même chose avec GCC 4.4.1, avec quelques exceptions.

\lstinputlisting[caption=GCC 4.4.1,style=customasmx86]{\CURPATH/classes/simple_GCC.asm}

Ici, nous voyons un autre style de \emph{name mangling}, spécifique à GNU
\footnote{Il y a un bon document à propos des différentes conventions de name mangling
dans différent compilateurs:\\
\InSqBrackets{\AgnerFogCC}.}.
Il peut aussi être noté que le pointeur sur l'objet est passé comme premier argument
de la fonction---invisible au programmeur, bien sûr.


Premier constructeur:

\begin{lstlisting}[style=customasmx86]
                public _ZN1cC1Ev ; weak
_ZN1cC1Ev       proc near               ; CODE XREF: main+10

arg_0           = dword ptr  8

                push    ebp
                mov     ebp, esp
                mov     eax, [ebp+arg_0]
                mov     dword ptr [eax], 667
                mov     eax, [ebp+arg_0]
                mov     dword ptr [eax+4], 999
                pop     ebp
                retn
_ZN1cC1Ev       endp
\end{lstlisting}

Il écrit juste les deux nombres en utilisant le pointeur passé comme premier (et
seul) argument.

Second constructeur:

\begin{lstlisting}[style=customasmx86]
                public _ZN1cC1Eii
_ZN1cC1Eii      proc near

arg_0           = dword ptr  8
arg_4           = dword ptr  0Ch
arg_8           = dword ptr  10h

                push    ebp
                mov     ebp, esp
                mov     eax, [ebp+arg_0]
                mov     edx, [ebp+arg_4]
                mov     [eax], edx
                mov     eax, [ebp+arg_0]
                mov     edx, [ebp+arg_8]
                mov     [eax+4], edx
                pop     ebp
                retn
_ZN1cC1Eii      endp
\end{lstlisting}

Ceci est une fonction, l'analogue d'une qui pourrait ressembler à ceci:

\begin{lstlisting}[style=customc]
void ZN1cC1Eii (int *obj, int a, int b)
{
  *obj=a;
  *(obj+1)=b;
};
\end{lstlisting}

\dots et cela est entièrement prévisible.

Maintenant, la fonction \TT{dump()}:

\lstinputlisting[style=customasmx86]{\CURPATH/classes/tmp1.asm}

La \emph{représentation interne} de cette fonction a un seul argument, utilisé comme
pointeur sur l'objet (\ITthis).

Cette fonction pourrait être récrite en C comme ceci:

\begin{lstlisting}[style=customc]
void ZN1c4dumpEv (int *obj)
{
  printf ("%d; %d\n", *obj, *(obj+1));
};
\end{lstlisting}

Ainsi, si nous basons notre jugement sur ces simples exemples, la différence entre
MSVC et GCC est le style d'encodage des noms de fonctions (\emph{name mangling}) et
la méthode pour passer un pointeur sur l'objet (via le registre \ECX ou via le premier
argument).

\myparagraph{GCC: x86-64}

Les 6 premiers arguments, comme nous le savons déjà, sont passés par les registres
\RDI, \RSI, \RDX, \RCX, \Reg{8} et \Reg{9} (\SysVABI),
et le pointeur sur \ITthis via le premier (\RDI) et c'est ce que l'on voit ici.
Le type de donnée \Tint est aussi 32-bit ici.

L'\emph{astuce} du \JMP au lieu de \RET est aussi utilisée ici.

\lstinputlisting[caption=GCC 4.4.6 x64,style=customasmx86]{\CURPATH/classes/simple_GCC_x64_FR.s}

