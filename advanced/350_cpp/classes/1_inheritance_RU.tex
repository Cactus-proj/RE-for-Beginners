\subsubsection{Наследование классов}
\label{cpp_inheritance}

О наследованных классах можно сказать, что это та же простая структура, которую мы уже рассмотрели, 
только расширяемая в наследуемых классах.

Возьмем очень простой пример:

\lstinputlisting[style=customc]{\CURPATH/classes/classes1_inheritance.cpp}

Исследуя сгенерированный код для функций/методов \TT{dump()}, а также \\
\TT{object::print\_color()},
посмотрим, какая будет разметка памяти для структур-объектов (для 32-битного кода).

\myindex{Inline code}
Итак, методы \TT{dump()} разных классов сгенерированные MSVC 2008 с опциями \Ox и \Obzero
\footnote{опция \Obzero означает отмену inline expansion, 
ведь вставка компилятором тела функции/метода прямо в код где он вызывается, 
может затруднить наши эксперименты.}.

\lstinputlisting[caption=\Optimizing MSVC 2008 /Ob0,style=customasmx86]{\CURPATH/classes/classes1_1.asm}

\lstinputlisting[caption=\Optimizing MSVC 2008 /Ob0,style=customasmx86]{\CURPATH/classes/classes1_2.asm}

\lstinputlisting[caption=\Optimizing MSVC 2008 /Ob0,style=customasmx86]{\CURPATH/classes/classes1_3.asm}

Итак, разметка полей получается следующая:

(базовый класс \emph{object})

\begin{center}
\begin{tabular}{ | l | l | }
\hline
  \tableheader{} \\
\hline
  +0x0 & int color \\
\hline
\end{tabular}
\end{center}

(унаследованные классы)

\emph{box}:

\begin{center}
\begin{tabular}{ | l | l | }
\hline
  \tableheader{} \\
\hline
  +0x0 & int color \\
\hline
  +0x4 & int width \\
\hline
  +0x8 & int height \\
\hline
  +0xC & int depth \\
\hline
\end{tabular}
\end{center}

\emph{sphere}:

\begin{center}
\begin{tabular}{ | l | l | }
\hline
  \tableheader{} \\
\hline
  +0x0 & int color \\
\hline
  +0x4 & int radius \\
\hline
\end{tabular}
\end{center}

Посмотрим тело \main:

\lstinputlisting[caption=\Optimizing MSVC 2008 /Ob0,style=customasmx86]{\CURPATH/classes/classes1_4.asm}

Наследованные классы всегда должны добавлять свои поля после полей базового класса для того, чтобы методы
базового класса могли продолжать работать со своими собственными полями.


Когда метод \TT{object::print\_color()} вызывается, ему в качестве \TT{this} передается указатель и на объект типа \emph{box} 
и на объект типа \emph{sphere}, так как он может легко работать с классами \emph{box} и \emph{sphere}, потому что поле \emph{color} в этих
классах всегда стоит по тому же адресу (по смещению \TT{0x0}).


Можно также сказать, что методу \TT{object::print\_color()} даже не нужно знать,
с каким классом он работает, до тех пор, пока будет соблюдаться условие \emph{закрепления} полей по тем же адресам,
а это условие соблюдается всегда.


А если вы создадите класс-наследник класса \emph{box}, например, 
то компилятор будет добавлять новые поля уже за полем \emph{depth}, оставляя уже имеющиеся поля класса \emph{box} по тем же адресам.


Так, метод \TT{box::dump()} будет нормально работать обращаясь к полям 
\emph{color}, \emph{width}, \emph{height} и \emph{depth}, всегда находящимся по известным адресам.


Код на GCC практически точно такой же, за исключением способа передачи \TT{this} (он, как уже было указано, 
передается в первом аргументе, вместо регистра \ECX).


