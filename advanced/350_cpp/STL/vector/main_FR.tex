\subsubsection{std::vector}
\myindex{\Cpp!STL!std::vector}

Nous appelons \TT{std::vector} un wrapper sûr sur le tableau C \ac{PODT}.
En interne il ressemble à \TT{std::string} (\myref{std_string}):
il a un pointeur sur le buffer alloué, un pointeur sur la fin du tableau, et un pointeur
sur la fin du buffer alloué.

Les éléments du tableau résident en mémoire adjacents les un aux autres, tout comme
un tableau normal (\myref{arrays}).
\myindex{\Cpp!C++11}
En C++11 il y a une nouvelle méthode appelée \TT{.data()}, qui renvoie un pointeur
sur le buffer, comme \TT{.c\_str()} dans \TT{std::string}.

Le buffer alloué dans le \gls{heap} peut être plus large que le tableau lui-même.

Les implémentations de MSVC et GCC sont similaires, seul sont légèrement différents
le nom des champs de la structure\footnote{GCC internals: \url{http://gcc.gnu.org/onlinedocs/libstdc++/libstdc++-html-USERS-4.4/a01371.html}},
donc il y a un seul code source qui fonctionne pour les deux compilateurs.
Voici encore le code pseudo-C pour afficher la structure de \TT{std::vector}:

\lstinputlisting[style=customc]{\CURPATH/STL/vector/2_FR.cpp}

Voici la sortie de ce programme lorsqu'il est compilé dans MSVC:

\lstinputlisting{\CURPATH/STL/vector/MSVC.txt}

On voit qu'il n'y a pas de buffer alloué lorsque \main débute.
Après le premier appel à \TT{push\_back()}, un buffer est alloué.
Et puis, après chaque appel à \TT{push\_back()}, la taille du tableau et la taille
du buffer (\emph{capacity}) sont augmentées.
Mais l'adresse du buffer change aussi, car \TT{push\_back()} ré-alloue le buffer dans
le \gls{heap} à chaque fois.
C'est une opération coûteuse, c'est pourquoi il est très important de prévoir la
taille du tableau dans le futur et de lui réserver assez d'espace avec la méthode
\TT{.reserve()}.

Le dernier nombre est du déchet: il n'y a pas d'élément du tableau à cet endroit,
donc un nombre aléatoire est affiché.
Ceci illustre le fait que \TT{operator[]} de \TT{std::vector} ne vérifie pas si l'index
est dans le limites du tableau.
La méthode plus lente \TT{.at()}, toutefois, fait cette vérification et envoie une
exception \TT{std::out\_of\_range} en cas d'erreur.

Regardons le code:

\lstinputlisting[caption=MSVC 2012 /GS- /Ob1,style=customasmx86]{\CURPATH/STL/vector/MSVC.asm}

Nous voyons que la méthode \TT{.at()} vérifie les limites et envoie une exception
en cas d'erreur.
Le nombre que le dernier appel à \printf affiche est pris de la mémoire, sans aucune
vérification.

On peut se demander pourquoi ne pas utiliser des variables comme \q{size} et \q{capacity},
comme c'est fait dans \TT{std::string}.
Probablement que c'est fait comme cela pour avoir une vérification des limites plus
rapide.

\myindex{Inline code}

Le code que GCC génère est en général presque le même, mais la méthode \TT{.at()}
est mise en ligne:

\lstinputlisting[caption=GCC 4.8.1 -fno-inline-small-functions -O1,style=customasmx86]{\CURPATH/STL/vector/GCC.asm}

\TT{.reserve()} est aussi mise en ligne.
Elle appelle \TT{new()} si le buffer est trop petit pour la nouvelle taille, appelle
\TT{memmove()} pour copier le contenu du buffer et appelle \TT{delete()} pour libérer
l'ancien buffer.

Regardons aussi ce que le programme affiche s'il est compilé avec GCC:

\lstinputlisting{\CURPATH/STL/vector/GCC.txt}

Nous repérons que la taille du buffer grossit d'une manière différente qu'avec MSVC.

Une simple expérimentation montre que l'implémentation de MSVC augmente le buffer de
\textasciitilde{}50\% à chaque fois qu'il a besoin d'être augmenté, tandis que le
code de GCC l'augmente de 100\% à chaque fois, i.e., le double.

