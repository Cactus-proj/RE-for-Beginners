\subsubsection{std::string}
\myindex{\Cpp!STL!std::string}
\label{std_string}

\myparagraph{Internals} % "Entrailles?"

De nombreuses bibliothèques de chaînes \InSqBrackets{\CNotes 2.2} implémentent une
structure qui contient un pointeur sur un buffer de chaîne, une variable qui contient
toujours la longueur actuelle de la chaîne (ce qui est très pratique pour de nombreuses
fonctions: \InSqBrackets{\CNotes 2.2.1}) et une variable qui contient la taille actuelle
du buffer.

La chaîne  dans le buffer est en général terminée par un zéro, afin de pouvoir passer
un pointeur sur le buffer aux fonctions qui prennent une chaîne C \ac{ASCIIZ} standard.

Il n'est pas précisé dans le standard \Cpp comment std::string doit être implémentée,
toutefois, elle l'est en général comme expliqué ci-dessus.

La chaîne \Cpp n'est pas une classse (comme QString dansQt, par exemple) mais un
template (basic\_string), ceci est fait afin de supporter différents types de caractères:
au moins \Tchar et \emph{wchar\_t}.

Donc, std::string est une classe avec \Tchar comme type de base.

Et std::wstring est une classe avec \emph{wchar\_t} comme type de base.

\mysubparagraph{MSVC}

L'implémentation de MSVC peut stocker le buffer en place au lieu d'utiliser un pointeur
sur un buffer (si la chaîne est plus courte que 16 symboles).


Ceci implique qu'une chaîne courte occupe au moins $16 + 4 + 4 = 24$
octets en environnement 32-bit et au moins $16 + 8 + 8 = 32$
octets dans un 64-bit, et si la chaîne est plus longue que 16 caractères, nous devons
ajouter la longueur de la chaîne elle-même.

\lstinputlisting[caption=exemple pour MSVC,style=customc]{\CURPATH/STL/string/MSVC_FR.cpp}

Presque tout est clair dans le code source.

Quelques notes:

Si la chaîne est plus petite que 16 symboles, il n'y a pas de buffer alloué sur le
\gls{heap}.

Ceci est pratique car en pratique, beaucoup de chaînes sont courtes.

Il semble que les développeurs de Microsoft aient choisi 16 caractères comme un bon
compromis.

On voit une chose très importante à la fin de main(): nous n'utilisons pas la méthode
c\_str(), néanmoins, si nous compilons et exécutons ce code, les deux chaînes apparaîtrons
dans la console!

C'est pourquoi ceci fonctionne.

Dans le premier cas, la chaîne fait moins de 16 caractères et le buffer avec la chaîne
se trouve au début de l'objet std::string (il peut-être traité comme une structure).
printf() traite le pointeur comme un pointeur sur un tableau de caractères terminé
par un \\0, donc ça fonctionne.

L'affichage de la seconde chaîne (de plus de 16 caractères) est encore plus dangereux:
c'est une erreur typique de programmeur (ou une typo) d'oublier d'écrire c\_str().

Ceci fonctionne car pour le moment un pointeur sur le buffer est situé au début de
la structure.

Ceci peut passer inaperçu pendant un long moment, jusqu'à ce qu'une chaîne plus longue
apparaisse à un moment, alors le processus plantera.

\mysubparagraph{GCC}

L'implémentation de GCC de cette structure a une variable de plus---le compteur de
références.

Une chose intéressante est que dans GCC un pointeur sur une instance de std::string
ne pointe pas au début de la structure, mais sur le pointeur du buffer.
Dans \emph{libstdc++-v3\textbackslash{}include\textbackslash{}bits\textbackslash{}basic\_string.h}
nous pouvons lire que ça a été fait ainsi afin de faciliter le débogage:

\begin{lstlisting}
   *  The reason you want _M_data pointing to the character %array and
   *  not the _Rep is so that the debugger can see the string
   *  contents. (Probably we should add a non-inline member to get
   *  the _Rep for the debugger to use, so users can check the actual
   *  string length.)
\end{lstlisting}

\href{http://gcc.gnu.org/onlinedocs/libstdc++/libstdc++-html-USERS-4.4/a01068.html}{code source de basic\_string.h}

Considérons ceci dans notre exemple:

\lstinputlisting[caption=exemple pour GCC,style=customc]{\CURPATH/STL/string/GCC_FR.cpp}

Il faut utiliser une astuce pour imiter l'erreur que nous avons vue avant car GCC
a une vérification de type plus forte, cependant, printf() fonctionne ici également
sans c\_str().

\myparagraph{Un exemple plus avancé}

\lstinputlisting[style=customc]{\CURPATH/STL/string/3.cpp}

\lstinputlisting[caption=MSVC 2012,style=customasmx86]{\CURPATH/STL/string/3_MSVC_FR.asm}

Le compilateur ne construit pas les chaînes statiquement: il ne serait de toutes
façons pas possible si le buffer devait être situer dans le \gls{heap}.

au lieu de ça, les chaînes \ac{ASCIIZ} sont stockées dans le segment de données,
et plus tard, au lancement, avec l'aide de la méthode \q{assign}, les chaînes s1
et s2 sont construites.

Veuillez noter qu'il n'y a pas d'appel à la méthode c\_str(), car le code est assez
petit pour que le compilateur le mette en ligne ici: si la chaîne est plus courte
que 16 caractères, un pointeur sur le buffer est laissé dans \EAX, autrement l'adresse
du buffer de la chaîne située dans le \gls{heap} est récupérée.

Ensuite, nous voyons les appels aux 3 destructeurs, is sont appelés si la chaîne
fait plus de 16 caractères: le buffer dans le \gls{heap} doit être libéré.
Autrement, puisque les trois objets std::string sont stockés dans la pile, ils sont
automatiquement libérés lorsque la fonction se termine.

En conséquence, traiter des chaînes courtes est plus rapide, car il y a moins d'accès
au \gls{heap}.

Le code de GCC est encore plus simple (car la manière de GCC, comme nous l'avons
vu ci-dessus, est de ne pas stocker les chaînes courtes directement dans la structure):

% TODO1 comment each function meaning
\lstinputlisting[caption=GCC 4.8.1,style=customasmx86]{\CURPATH/STL/string/3_GCC_FR.s}

On voit que ce n'est pas un pointeur sur l'objet qui est passé aux destructeurs,
mais plutôt une adresse 12 octets (ou 3 mots) avant, i.e., un pointeur sur le début
réel de la structure.

\myparagraph{std::string comme une variable globale}
\label{sec:std_string_as_global_variable}

Les programmeurs \Cpp expérimentés savent que les des variables globales des types
\ac{STL} peuvent être définis sans problème.

Oui, en effet:

\lstinputlisting[style=customc]{\CURPATH/STL/string/5.cpp}

Mais comment et où le constructeur de \TT{std::string} sera appelé?

En fait, cette variable sera initialisée avant le démarrage de \main.

\lstinputlisting[caption=MSVC 2012: voici comment une variable globale est construite et aussi comment sont destructeur est déclaré,style=customasmx86]{\CURPATH/STL/string/5_MSVC_p2.asm}

\lstinputlisting[caption=MSVC 2012: ici une variable globale est utilisée dans \main,style=customasmx86]{\CURPATH/STL/string/5_MSVC_p1.asm}

\lstinputlisting[caption=MSVC 2012: cette fonction destructeur est appelée avant exit,style=customasmx86]{\CURPATH/STL/string/5_MSVC_p3.asm}

\myindex{\CStandardLibrary!atexit()}

En fait, une fonction spéciale avec tous les constructeurs des variables globales
est appelée depuis \ac{CRT}, avant main().

Mieux que ça: avec l'aide de atexit() une autre fonction est enregistrée, qui contient
des appels aux destructeurs de telles variables globales.

GCC fonctionne comme ceci:

\lstinputlisting[caption=GCC 4.8.1,style=customasmx86]{\CURPATH/STL/string/5_GCC.s}

Mais il ne crée pas une fonction séparée pour cela, chaque destructeur est passé
à atexit(), un par un.

% TODO а если глобальная STL-переменная в другом модуле? надо проверить.

