\mysection{Windows 16-bit}
\myindex{Windows!Windows 3.x}

16-bit Windows programs are rare nowadays, but can be used in the cases of retrocomputing
or dongle hacking (\myref{dongles}).

16-bit Windows versions were up to 3.11.
95/98/ME also support 16-bit code, as well as the 32-bit versions of the \gls{Windows NT} line.
The 64-bit versions of \gls{Windows NT} line do not support 16-bit executable code at all.

The code resembles MS-DOS's one.

Executable files are of type NE-type (so-called \q{new executable}).

All examples considered here were compiled by the OpenWatcom 1.9 compiler, using these switches:

\begin{lstlisting}
wcl.exe -i=C:/WATCOM/h/win/ -s -os -bt=windows -bcl=windows example.c
\end{lstlisting}

\input{\CURPATH/ex1.tex}
\input{\CURPATH/ex2.tex}
\input{\CURPATH/ex3.tex}
\input{\CURPATH/ex4.tex}
\input{\CURPATH/ex5.tex}
\input{\CURPATH/ex6_EN.tex}

