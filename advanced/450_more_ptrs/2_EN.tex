\subsection{Passing values as pointers; tagged unions}

Here is an example on how to pass values in pointers:

\begin{lstlisting}[label=unsigned_multiply_C,style=customc]
#include <stdio.h>
#include <stdint.h>

uint64_t multiply1 (uint64_t a, uint64_t b)
{
	return a*b;
};

uint64_t* multiply2 (uint64_t *a, uint64_t *b)
{
	return (uint64_t*)((uint64_t)a*(uint64_t)b);
};

int main()
{
	printf ("%d\n", multiply1(123, 456));
	printf ("%d\n", (uint64_t)multiply2((uint64_t*)123, (uint64_t*)456));
};
\end{lstlisting}

It works smoothly and GCC 4.8.4 compiles both multiply1() and multiply2() functions identically!

\begin{lstlisting}[label=unsigned_multiply_lst,style=customasmx86]
multiply1:
	mov	rax, rdi
	imul	rax, rsi
	ret

multiply2:
	mov	rax, rdi
	imul	rax, rsi
	ret
\end{lstlisting}

As long as you do not dereference pointer (in other words, you don't read any data from the address stored in pointer), everything will work fine.
Pointer is a variable which can store anything, like usual variable.

\myindex{x86!\Instructions!MUL}
\myindex{x86!\Instructions!IMUL}
Signed multiplication instruction (\IMUL) is used here instead of unsigned one (\MUL), read more about it here:
\myref{IMUL_over_MUL}.

\myindex{Tagged pointers}
By the way, it's well-known hack to abuse pointers a little called \emph{tagged pointers}.
In short, if all your pointers points to blocks of memory with size of, let's say, 16 bytes (or it is always aligned on 16-byte boundary), 4 lowest bits of pointer is always zero bits and this space
can be used somehow.
\myindex{LISP}
It's very popular in LISP compilers and interpreters.
They store cell/object type in these unused bits, this can save some memory.
Even more, you can judge about cell/object type using just pointer, with no additional memory access.
Read more about it: \InSqBrackets{\CNotes 1.3}.

% TODO Example of tagged ptrs here

