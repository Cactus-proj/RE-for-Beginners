\subsection{Pointeur sur une fonction: protection contre la copie}
\myindex{\SoftwareCracking}

Un pirate de logiciel peut trouver une fonction qui vérifie la protection et renvoie
\emph{vrai} ou \emph{faux}.

Peut-on vérifier son intégrité?
Il s'avère que cela peut être fait facilement.

D'après objdump, les 3 premiers octets de \verb|check_protection()| sont 0x55 0x89 0xE5
(compte tenu du fait qu'il s'agit de GCC sans optimisation):

\lstinputlisting[style=customc]{advanced/450_more_ptrs/61.c}

\lstinputlisting[style=customc]{advanced/450_more_ptrs/61_objdump.txt}

Si quelqu'un patchait \verb|check_protection()|, votre programme peut faire quelque
chose de méchant, peut-être se terminer brusquement.
(\tracer possède l'option BPMx pour ça.)

