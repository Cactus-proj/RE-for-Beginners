\subsection{Указатель на функцию: защита от копирования}
\myindex{\SoftwareCracking}

Взломщик может найти ф-цию, проверяющую защиту и возвращать \emph{true} или \emph{false}.
Он(а) может вписать там \TT{XOR EAX,EAX / RETN} или \TT{MOV EAX, 1 / RETN}.

Может ли проверить целостность ф-ции?
Оказывается, сделать это легко.

Судя по objdump, первые 3 байта ф-ции \verb|check_protection()| это 0x55 0x89 0xE5 (учитывая, что это неоптимизирующий GCC):

\lstinputlisting[style=customc]{advanced/450_more_ptrs/61.c}

\lstinputlisting[style=customc]{advanced/450_more_ptrs/61_objdump.txt}

Если кто-то пропатчит начало ф-ции \verb|check_protection()|, ваша программа может совершить что-то подлое, например,
внезапно закончить работу.
Чтобы разобраться с таким трюком, взломщик может установить брякпоинт на чтение памяти, по адресу начала ф-ции.
(В \tracer{}-е для этого есть опция BPMx.)

