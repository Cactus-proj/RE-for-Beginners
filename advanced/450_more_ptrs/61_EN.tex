\subsection{Pointer to a function: copy protection}
\myindex{\SoftwareCracking}

A software cracker can find a function that checks protection and return \emph{true} or \emph{false}.
He/she then can put \TT{XOR EAX,EAX / RETN} or \TT{MOV EAX, 1 / RETN} there.

Can you check integrity of the function?
As it turns out, this can be done easily.

According to objdump, the first 3 bytes of \verb|check_protection()| are 0x55 0x89 0xE5 (given the fact this is non-optimizing GCC):

\lstinputlisting[style=customc]{advanced/450_more_ptrs/61.c}

\lstinputlisting[style=customc]{advanced/450_more_ptrs/61_objdump.txt}

If someone would patch the beginning of the \verb|check_protection()| function,
your program can do something mean, maybe exit suddenly.
To find such a trick, a cracker can set a memory read breakpoint on the address of the function's beginning.
(\tracer has BPMx options for that.)

