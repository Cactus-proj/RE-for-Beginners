\mysection{Le cas du return oublié}
\label{ForgottenReturn}

Revoyons la partie ``tentative d'utiliser le résultat d'une fonction renvoyant \Tvoid'': \myref{UseResultOfVoidFunc}.

Ceci est un bug que j'ai rencontré une fois.

Et c'est encore une autre démonstration de la façon dont \CCpp{} met les valeurs de
retour dans le registre \EAX/\RAX.

Dans ce bout de code, j'ai oublié d'ajouter \TT{return}:

\lstinputlisting[style=customc]{\CURPATH/1_EN.c}

GCC 5.4 sans optimisation le compile silencieusement, sans avertissement.
Et le code fonctionne!
Voyons pourquoi:

\lstinputlisting[style=customasmx86,caption=GCC 5.4 sans optimisation]{\CURPATH/O0_no_return_works_EN.lst}

Si j'ajoute \TT{return rt;}, 

\lstinputlisting[style=customasmx86,caption=GCC 5.4 sans optimisation]{\CURPATH/O0_return_works_EN.lst}

Des bogues de ce type sont très dangereux, parfois ils apparaissent, parfois ils
restent invisibles.

Maintenant, j'essaye GCC avec l'optimisation:

\lstinputlisting[style=customasmx86,caption=GCC 5.4 avec optimisation]{\CURPATH/O3_no_return_crash_EN.lst}

Le compilateur en déduit que rien n'est renvoyé de la fonction, donc il l'optimise.
% to be sync: Compiler deducing that nothing returns from the function, so it optimizes the whole function away.
Et il suppose que 0 est renvoyé par défaut. Le zéro est utilisé comme une adresse
sur une structure dans main()..
Bien sûr, ce code plante.

GCC en mode C++ est aussi silencieux à propos de cela.
% to be sync: GCC in C++ mode silent about it as well.

Essayons MSVC 2015 x86 sans optimisation.
Il averti à propos de ce problème:

\begin{lstlisting}
c:\tmp\3.c(19) : warning C4716: 'create_color': must return a value
\end{lstlisting}

Et il génère du code qui plante:

\lstinputlisting[style=customasmx86,caption=MSVC 2015 x86 sans optimisation]{\CURPATH/MSVC_x86_crash_EN.lst}

Maintenant MSVC 2015 x86 avec optimisation, qui génère du code qui plante aussi mais
pour une raison différente.

\lstinputlisting[style=customasmx86,caption=MSVC 2015 x86 avec optimisation]{\CURPATH/MSVC_Ox_x86_crash_EN.lst}

Toutefois, MSVC 2015 x64 sans optimisation génère du code qui fonctionne:

\lstinputlisting[style=customasmx86,caption=MSVC 2015 x64 sans optimisation]{\CURPATH/MSVC_x64_works_EN.lst}

MSVC 2015 x64 avec optimisation met le fonction en ligne, comme dans le cas du x86,
et le code résultant plante.

\myhrule{}

% FIXME URL to octothorpe
Ceci est un morceau de code réel de ma bibliothèque \emph{octothorpe},
qui fonctionnait et dont tous les tests réussissaient.
C'était ainsi, sans \verb|return| pendant un certain temps..

\begin{lstlisting}
uint32_t LPHM_u32_hash(void *key)
{
        jenkins_one_at_a_time_hash_u32((uint32_t)key);
}
\end{lstlisting}

\myhrule{}

La morale de l'histoire: les warnings sont très importants, utilisez \TT{-Wall}, etc, etc...
Lorsque la déclaration \TT{return} est absente, le compilateur peut simplement silencieusement
ne rien faire à ce point.

Un tel bug passé inaperçu peut gâcher une journée.

Aussi, \emph{le débogage shotgun}
est mauvais, car encore une fois, un tel bogue peut passer inaperçu (``tout fonctionne
maintenant, qu'il en soit ainsi'').

% to be sync: Also, \emph{shotgun debugging}
%             is bad, because again, such a bug can left unnoticed (``everything works now, so be it, let's leave it as is'').

% to be sync: See also: discussion at Hacker News\footnote{\url{https://news.ycombinator.com/item?id=18671609}}
%             and archived blog post\footnote{\url{https://web.archive.org/web/20190317231721/https://yurichev.com/blog/no_return/}}.

