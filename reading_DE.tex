\chapter{Bücher / Lesenswerte Blogs}

\mysection{Bücher und andere Materialien}

\subsection{Reverse Engineering}

\input{RE_books}

% TBT
% (Outdated, but still interesting) Pavol Cerven, \emph{Crackproof Your Software: Protect Your Software Against Crackers}, (2002).

Ebenfalls das Buch von Kris Kaspersky.

\subsection{Windows}

\begin{itemize}
\item \Russinovich
\item Peter Ferrie -- The ``Ultimate'' Anti-Debugging Reference\footnote\url{http://pferrie.host22.com/papers/antidebug.pdf}}
\end{itemize}

\EN{Blogs}\ES{Blogs}\RU{Блоги}\FR{Blogs}\DE{Blogs}\PL{Blogi}:

\begin{itemize}
\item \href{http://go.yurichev.com/17025}{Microsoft: Raymond Chen}
\item \href{http://go.yurichev.com/17026}{nynaeve.net}
\end{itemize}



\subsection{\CCpp}

\label{CCppBooks}

\begin{itemize}

\item \KRBook

\item \CNineNineStd\footnote{\AlsoAvailableAs \url{http://www.open-std.org/jtc1/sc22/WG14/www/docs/n1256.pdf}}

\item \TCPPPL

\item \CppOneOneStd\footnote{\AlsoAvailableAs \url{http://www.open-std.org/jtc1/sc22/wg21/docs/papers/2013/n3690.pdf}.}

\item \AgnerFogCPP\footnote{\AlsoAvailableAs \url{http://agner.org/optimize/optimizing_cpp.pdf}.}

\item \ParashiftCPPFAQ\footnote{\AlsoAvailableAs \url{http://www.parashift.com/c++-faq-lite/index.html}}

\item \CNotes\footnote{\AlsoAvailableAs \url{http://yurichev.com/C-book.html}}

\item JPL Institutional Coding Standard for the C Programming Language\footnote{\AlsoAvailableAs \url{https://yurichev.com/mirrors/C/JPL_Coding_Standard_C.pdf}}

\RU{\item Евгений Зуев --- Редкая профессия\footnote{\AlsoAvailableAs \url{https://yurichev.com/mirrors/C++/Redkaya_professiya.pdf}}}

\end{itemize}



\subsection{x86 / x86-64}

\label{x86_manuals}
\begin{itemize}
\item Intel Handbücher\footnote{\AlsoAvailableAs \url{http://www.intel.com/content/www/us/en/processors/architectures-software-developer-manuals.html}}

\item AMD Handbücher\footnote{\AlsoAvailableAs \url{http://developer.amd.com/resources/developer-guides-manuals/}}

\item \AgnerFog{}\footnote{\AlsoAvailableAs \url{http://agner.org/optimize/microarchitecture.pdf}}

\item \AgnerFogCC{}\footnote{\AlsoAvailableAs \url{http://www.agner.org/optimize/calling_conventions.pdf}}

\item \IntelOptimization

\item \AMDOptimization
\end{itemize}

Etwas veraltet aber immer noch interessant zu lesen:

\MAbrash\footnote{\AlsoAvailableAs \url{https://github.com/jagregory/abrash-black-book}}
(Er ist bekannt für seine Arbeiten auf dem Gebiet der Low-Level Optimierung in Projekten wie Windows NT 3.1 und id Quake).

\subsection{ARM}

\begin{itemize}
\item ARM Handbücher\footnote{\AlsoAvailableAs \url{http://infocenter.arm.com/help/index.jsp?topic=/com.arm.doc.subset.architecture.reference/index.html}}

\item \ARMSevenRef

\item \ARMSixFourRefURL

\item \ARMCookBook\footnote{\AlsoAvailableAs \url{https://yurichev.com/ref/ARM%20Cookbook%20(1994)/}}
\end{itemize}

\subsection{Assembler}

Richard Blum --- Professional Assembly Language.

\subsection{Java}

\JavaBook.

\subsection{UNIX}

\TAOUP

\subsection{Programmierung Allgemein}

\begin{itemize}

	\item \RobPikePractice
	
	\item \HenryWarren.
	Einige Leute sagen, die Tricks und Hacks aus diesem Buch sind heute nicht mehr relevant und haben die eigentliche Bedeutung für \ac{RISC} \ac{CPU}s, bei denen Verzweigungsbefehle teuer sind.
	Nichtsdestotrotz können diese immens hilfreich sein um Bool'sche Algebra und die damit zusammenhängende Mathematik zu verstehen.
	
\end{itemize}

% subsection:
\input{crypto_reading}
% TBT
