\subsubsection{Linux}

Voyons maintenant comment une variable globale conservée dans l'espace de stockage propre au thread 
est déclarée avec GCC:

\begin{lstlisting}
__thread uint32_t rand_state=1234;
\end{lstlisting}

Il ne s'agit pas du modificateur standard \CCpp modifier, mais bien d'un modificateur spécifique à 
GCC
\footnote{\url{https://gcc.gnu.org/onlinedocs/gcc-3.3/gcc/C99-Thread-Local-Edits.html}}.

\myindex{x86!\Registers!GS}

Le sélecteur \TT{GS:} est utilisé lui aussi pour accéder au \ac{TLS}, mais d'une manière un peu 
différente:

\lstinputlisting[caption=\Optimizing GCC 4.8.1 x86,style=customasmx86]{OS/TLS/linux/rand.lst}

% FIXME (to be checked) Uninitialized data is allocated in \TT{.tbss} section, initialized --- in \TT{.tdata} section.

Pour en savoir plus: \DrepperTLS.

