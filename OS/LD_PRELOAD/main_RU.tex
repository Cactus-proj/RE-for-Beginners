\subsection{Трюк с \emph{LD\_PRELOAD} в Linux}

\myindex{LD\_PRELOAD}
\label{ld_preload}

Это позволяет загружать свои динамические библиотеки перед другими, даже перед системными, такими как libc.so.6.

Что в свою очередь, позволяет \q{подставлять} написанные нами функции перед оригинальными из системных библиотек.

Например, легко перехватывать все вызовы к time(), read(), write(), итд. \\
\\
\myindex{uptime}
Попробуем узнать, сможем ли мы обмануть утилиту \emph{uptime}.
Как известно, она сообщает, как долго компьютер работает.
\myindex{strace}
При помощи strace(\myref{strace}), можно увидеть, что эту информацию утилита получает из файла \TT{/proc/uptime}:

\begin{lstlisting}
$ strace uptime 
...
open("/proc/uptime", O_RDONLY)          = 3
lseek(3, 0, SEEK_SET)                   = 0
read(3, "416166.86 414629.38\n", 2047)  = 20
...
\end{lstlisting}

Это не реальный файл на диске, это виртуальный файл, содержимое которого генерируется на лету в ядре Linux.

Там просто два числа:

\begin{lstlisting}
$ cat /proc/uptime
416690.91 415152.03
\end{lstlisting}

Из Wikipedia, можно узнать
\footnote{\url{https://en.wikipedia.org/wiki/Uptime}}:

\begin{framed}
\begin{quotation}
The first number is the total number of seconds the system has been up.
The second number is how much of that time the machine has spent idle, in seconds.
\end{quotation}
\end{framed}

\myindex{\CStandardLibrary!open()}
\myindex{\CStandardLibrary!read()}
\myindex{\CStandardLibrary!close()}
Попробуем написать свою динамическую библиотеку, в которой будет open(), read(), close() с нужной нам функциональностью.

Во-первых, наш open() будет сравнивать имя открываемого файла с тем что нам нужно, и если да, 
то будет запоминать дескриптор открытого файла.

Во-вторых, read(), если будет вызываться для этого дескриптора, будет подменять вывод,
а в остальных случаях, будет вызывать настоящий read() из libc.so.6.
А также close(), будет следить, закрывается ли файл за которым мы следим.

\myindex{dlopen()}
\myindex{dlsym()}
Для того чтобы найти адреса настоящих функций в libc.so.6, используем dlopen() и dlsym().

Нам это нужно, потому что нам нужно передавать управление \q{настоящим} функциями.

\myindex{\CStandardLibrary!strcmp()}
С другой стороны, если бы мы перехватывали, скажем, strcmp(), и следили бы за всеми сравнениями строк в программе, 
то, наверное, strcmp() можно было бы и самому реализовать, не пользуясь настоящей функцией
\footnote{Например, посмотрите как обеспечивается простейший перехват strcmp() в статье
\footnote{\href{https://yurichev.com/mirrors/LD_PRELOAD/Yong%20Huang%20LD_PRELOAD.txt}{yurichev.com}} написанной Yong Huang}, так было бы проще.

\lstinputlisting[style=customc]{OS/LD_PRELOAD/fool_uptime.c}
( \href{\RepoURL/OS/LD_PRELOAD/fool_uptime.c}{Исходный код} )

Компилируем как динамическую библиотеку:

\begin{lstlisting}
gcc -fpic -shared -Wall -o fool_uptime.so fool_uptime.c -ldl
\end{lstlisting}

Запускаем \emph{uptime}, подгружая нашу библиотеку перед остальными:

\begin{lstlisting}
LD_PRELOAD=`pwd`/fool_uptime.so uptime
\end{lstlisting}

Видим такое:

\begin{lstlisting}
 01:23:02 up 24855 days,  3:14,  3 users,  load average: 0.00, 0.01, 0.05
\end{lstlisting}

Если переменная окружения \emph{LD\_PRELOAD} 
будет всегда указывать на путь и имя файла нашей библиотеки, то она будет
загружаться для всех запускаемых программ.  \\
\\
Еще примеры:

\begin{itemize}
\item
Перехват time() в Sun Solaris \href{http://yurichev.com/mirrors/LD_PRELOAD/sun_hack.txt}{yurichev.com}

\item
Очень простой перехват strcmp() (Yong Huang) 
\url{https://yurichev.com/mirrors/LD_PRELOAD/Yong%20Huang%20LD_PRELOAD.txt}

\item
Kevin Pulo --- Fun with LD\_PRELOAD. Много примеров и идей.
\href{http://yurichev.com/mirrors/LD_PRELOAD/lca2009.pdf}{yurichev.com}

\item
Перехват функций работы с файлами для компрессии и декомпрессии файлов на лету (zlibc). \url{ftp://metalab.unc.edu/pub/Linux/libs/compression}

\end{itemize}
