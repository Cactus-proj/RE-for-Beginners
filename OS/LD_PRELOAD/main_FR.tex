\subsection{Hack \emph{LD\_PRELOAD} sur Linux}

\myindex{LD\_PRELOAD}
\label{ld_preload}

Cela permet de charger nos propres bibliothèques dynamiques avant les autres, même celle du système, comme libc.so.6.

Cela permet de \q{substituer} nos propres fonctions, avant celles originales des bibliothèques du système.
Par exemple, il est facile d'intercepter tous les appels à
time(), read(), write(), etc. \\
\\
\myindex{uptime}
Regardons si l'on peut tromper l'utilitaire \emph{uptime}.
Comme chacun le sait, il indique depuis combien de temps l'ordinateur est démarré.
\myindex{strace}
Avec l'aide de strace(\myref{strace}), on voit que ces informations proviennent du fichier \TT{/proc/uptime}:

\begin{lstlisting}
$ strace uptime 
...
open("/proc/uptime", O_RDONLY)          = 3
lseek(3, 0, SEEK_SET)                   = 0
read(3, "416166.86 414629.38\n", 2047)  = 20
...
\end{lstlisting}

Ce n'est pas un fichier réel sur le disque, il est virtuel et généré au vol par le noyau Linux.
Il y a seulement deux nombres:

\begin{lstlisting}
$ cat /proc/uptime
416690.91 415152.03
\end{lstlisting}

Ce que l'on peut apprendre depuis Wikipédia
\footnote{\href{http://go.yurichev.com/17043}{Wikipédia}}:

\begin{framed}
\begin{quotation}
Le premier est le nombre total de seconde depuis lequel le système est démarré.
Le second est la part de ce temps pendant lequel la machine était inoccupée, en secondes.
\end{quotation}
\end{framed}

\myindex{\CStandardLibrary!open()}
\myindex{\CStandardLibrary!read()}
\myindex{\CStandardLibrary!close()}

Essayons d'écrire notre propre bibliothèque dynamique, avec les fonctions
open(), read(), close() fonctionnant comme nous en avons besoin.

Tout d'abord, notre fonction open() va comparer le nom du fichier à ouvrir avec celui que l'on veut modifier,
et si c'est le cas, sauver le descripteur du fichier ouvert.

Ensuite, si read() est appelé avec ce descripteur de fichier, nous allons substituer la sortie,
et dans les autres cas, nous appellerons la fonction read() originale de libc.so.6.
Et enfin close() fermera le fichier que nous suivons.

\myindex{dlopen()}
\myindex{dlsym()}

Nous allons utiliser les fonctions dlopen() et dlsym() pour déterminer les adresses des fonctions originales dans libc.so.6.

Nous en avons besoin pour appeler les fonctions \q{réelles}.

\myindex{\CStandardLibrary!strcmp()}

D'un autre côté, si nous interceptons strcmp() et surveillons chaque comparaison de
chaînes dans le programme, nous pouvons alors implémenter notre propre fonction strcmp(),
et ne pas utiliser la fonction originale du tout.

\footnote{Par exemple, voilà comment implémenter une interception simple de strcmp() dans cet article
\footnote{\href{http://go.yurichev.com/17143}{yurichev.com}}
écrit par Yong Huang}.

\lstinputlisting[style=customc]{OS/LD_PRELOAD/fool_uptime.c}
( \href{\GitHubBlobMasterURL/OS/LD_PRELOAD/fool_uptime.c}{Code source sur GitHub} )
% FIXME go.yurichev.com...

Compilons le comme une bibliothèque dynamique standard:

\begin{lstlisting}
gcc -fpic -shared -Wall -o fool_uptime.so fool_uptime.c -ldl
\end{lstlisting}

Lançons \emph{uptime}
en chargeant notre bibliothèque avant les autres:

\begin{lstlisting}
LD_PRELOAD=`pwd`/fool_uptime.so uptime
\end{lstlisting}

Et nous voyons:

\begin{lstlisting}
 01:23:02 up 24855 days,  3:14,  3 users,  load average: 0.00, 0.01, 0.05
\end{lstlisting}

Si la variable d'environnement \emph{LD\_PRELOAD}

pointe toujours sur le nom et le chemin de notre bibliothèque,
elle sera chargée pour tous les programmes démarrés. \\
\\
Plus d'exemples:

\begin{itemize}

\item
Interception simple de strcmp() (Yong Huang)
\url{http://go.yurichev.com/17143}

\item
Kevin Pulo---Jouons avec LD\_PRELOAD. De nombreux exemples et idées.
\href{http://go.yurichev.com/17145}{yurichev.com}

\item
Interception des fonctions de fichier pour compresser/décompresser les fichiers au vol (zlibc) \url{http://go.yurichev.com/17146}

\end{itemize}
