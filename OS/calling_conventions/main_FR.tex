\mysection{Méthodes de transmission d'arguments (calling conventions)}
\label{sec:callingconventions}

\subsection{cdecl}
\myindex{cdecl}
\label{cdecl}

Il s'agit de la méthode la plus courante pour passer des arguments à une fonction en langage \CCpp.

Après que l'\glslink{callee}{appelé} ait rendu la main, l'\glslink{caller}{appelant} doit ajuster la valeur du \gls{stack pointer}
(\ESP) pour lui redonner la valeur qu'elle avait avant l'appel de l'\glslink{callee}{appelé}.

\begin{lstlisting}[caption=cdecl,style=customasmx86]
push arg3
push arg2
push arg1
call function
add esp, 12 ; returns ESP
\end{lstlisting}

\subsection{stdcall}
\label{sec:stdcall}
\myindex{stdcall}

\newcommand{\SIZEOFINT}{La taille d'une variable de type \Tint est de 4 octets sur les systèmes x86 et de 8 octets sur les systèmes x64}

Similaire à \emph{cdecl}, sauf que c'est l'\glslink{callee}{appelé} qui doit réinitialise \ESP à sa valeur d'origine
en utilisant l'instruction \TT{RET x} et non pas \RET, \\
avec \TT{x = nb arguments * sizeof(int)\footnote{\SIZEOFINT}}.
Après le retour du \gls{callee}, l'\glslink{caller}{appelant} ne modifie pas le \gls{stack pointer}. Il ny' a donc pas d'instruction \TT{add esp, x}.

\begin{lstlisting}[caption=stdcall,style=customasmx86]
push arg3
push arg2
push arg1
call function

function:
... do something ...
ret 12
\end{lstlisting}

Cette méthode est omniprésente dans les librairies standard win32, mais pas dans celles win64 (voir ci-dessous pour win64).

\par Prenons par exemple la fonction \myref{src:passing_arguments_ex} et modifions la légèrement en utilisant la convention \TT{\_\_stdcall} :

\begin{lstlisting}[style=customc]
int __stdcall f2 (int a, int b, int c)
{
	return a*b+c;
};
\end{lstlisting}

Le résultat de la compilation sera quasiment identique à celui de \myref{src:passing_arguments_ex_MSVC_cdecl},
mais vous constatez la présence de \TT{RET 12} au lieu de \TT{RET}.
L' \glslink{caller}{appelant} ne met pas à jour {SP} après l'appel.

Avec la convention \TT{\_\_stdcall}, on peut facilement déduire le nombre d'arguments de la fonction
appelée à partir de l'instruction \TT{RETN n}: divisez $n$ par 4.

\lstinputlisting[caption=MSVC 2010,style=customasmx86]{OS/calling_conventions/stdcall_ex.asm}

\subsubsection{Fonctions à nombre d'arguments variables}

Les fonctions du style \printf sont un des rares cas de fonctions à nombre d'arguments variables
en \CCpp. Elles permettent d'illustrer une différence importante entre les conventions \emph{cdecl} et \emph{stdcall}.
Partons du principe que le compilateur connait le nombre d'arguments utilisés à chaque fois que la
fonction \printf est appelée.

En revanche, la fonction \printf est déjà compilée dans MSVCRT.DLL (si l'on parle de Windows) et ne
possède aucune information sur le nombre d'arguments qu'elle va recevoir. Elle peut cependant le
deviner à partir du contenu du paramètre format.

Si la convention \emph{stdcall} était utilisé pour la fonction \printf, elle devrait réajuster le
\glslink{stack pointer}{pointeur de pile} à sa valeur initiale en comptant le nombre d'arguments dans la chaîne de format.
Cette situation serait dangereuse et pourrait provoquer un crash du programme à la moindre faute de
frappe du programmeur.
La convention \emph{stdcall} n'est donc pas adaptée à ce type de fonction. La convention \emph{cdecl} est préférable.

\subsection{fastcall}
\label{fastcall}
\myindex{fastcall}

Il s'agit d'un nom générique pour désigner les conventions qui passent une partie des paramètres dans
des registres et le reste sur la pile.
Historiquement, cette méthode était plus performante que les conventions \emph{cdecl}/\emph{stdcall} -
car elle met moins de pression sur la pile. Ce n'est cependant probablement plus le cas sur les
processeurs actuels qui sont beaucoup plus complexes.

Cette convention n'est pas standardisée. Les compilateurs peuvent donc l'implémenter à leur guise.
Prenez une DLL qui en utilise une autre compilée avec une interprétation différente de la convention
\emph{fastcall}. Vous avez un cas d'école et des problèmes en perspective.

Les compilateurs MSVC et GCC passent les deux premiers arguments dans \ECX et \EDX, et le reste des
arguments sur la pile.

Le \gls{stack pointer} doit être restauré à sa valeur initiale par l'\glslink{callee}{appelé}
(comme pour la convention \emph{stdcall}).

\begin{lstlisting}[caption=fastcall,style=customasmx86]
push arg3
mov edx, arg2
mov ecx, arg1
call function

function:
.. do something ..
ret 4
\end{lstlisting}

Prenons par exemple la fonction \myref{src:passing_arguments_ex} et modifions la légèrement en utilisant la convention \TT{\_\_fastcall} :

\begin{lstlisting}[style=customc]
int __fastcall f3 (int a, int b, int c)
{
	return a*b+c;
};
\end{lstlisting}

Le résultat de la compilation est le suivant:

\lstinputlisting[caption=MSVC 2010 optimisé/Ob0,style=customasmx86]{OS/calling_conventions/fastcall_ex.asm}

Nous voyons que l'\glslink{callee}{appelé} ajuste \ac{SP} en utilisant l'instruction \TT{RETN} suivie d'un opérande.

Le nombre d'arguments peut, encore une fois, être facilement déduit.

\subsubsection{GCC regparm}

\newcommand{\URLREGPARMM}{\url{http://go.yurichev.com/17040}}

Il s'agit en quelque sorte d'une évolution de la convention \emph{fastcall}\footnote{\URLREGPARMM}.
L'option \TT{-mregparm} permet de définir le nombre d'arguments (3 au maximum) qui doivent être passés
dans les registres. \EAX, \EDX et \ECX sont alors utilisés.

Bien entendu si le nombre d'arguments est inférieur à 3, seuls les premiers registres sont utilisés.

C'est l'\glslink{caller}{appelant} qui effectue l'ajustement du \gls{stack pointer}.

Pour un exemple, voire (\myref{regparm}).

\subsubsection{Watcom/OpenWatcom}
\myindex{OpenWatcom}

Dans le cas de ce compilateur, on parle de \q{register calling convention}.
Les 4 premiers arguments sont passés dans les registres \EAX, \EDX, \EBX et \ECX.
Les paramètres suivant sont passés sur la pile.

Un suffixe constitué d'un caractère de soulignement est ajouté par le compilateur au nom de la
fonction afin de les distinguer de celles qui utilisent une autre convention d'appel.

\subsection{thiscall}
\myindex{thiscall}

Cette convention passe le pointeur d'objet \ITthis à une méthode en \Cpp.

Le compilateur MSVC, passe généralement le pointeur \ITthis dans le registre \ECX.

Le compilateur GCC passe le pointeur \ITthis comme le premier argument de la fonction.
% BlueSkeye : don't undertand the underlying idea.
% BlueSkeye : at most this is true for instance method. Class level methods don't have a this pointer.
Thus it will be very visible that internally: all function-methods have an extra argument.

Pour un example, voir (\myref{thiscall}).

\subsection{x86-64}
\myindex{x86-64}

\subsubsection{Windows x64}
\label{sec:callingconventions_win64}

La méthode utilisée pour passer les arguments aux fonctions en environnement Win64 ressemble
beaucoup à la convention \TT{fastcall}.
Les 4 premiers arguments sont passés dans les registres \RCX, \RDX, \Reg{8} et \Reg{9},
les arguments suivants sont passés sur la pile.
L'\glslink{caller}{appelant} doit également préparer un espace sur la pile pour 32 octets, soit 4 mots de 64 bits,
l'\glslink{callee}{appelé} pourra y sauvegarder les 4 premiers arguments.
Les fonctions suffisamment simples peuvent utiliser les paramètres directement depuis les registres.
Les fonctions plus complexes doivent sauvegarder les valeurs de paramètres afin de les utiliser plus tard.

L'\glslink{caller}{appelé} est aussi responsable de rétablir la valeur du \glslink{stack pointer}{pointeur de pile} à la valeur qu'il
avait avant l'appel de la fonction.

Cette convention est utilisée dans les DLLs Windows x86-64 (à la place de \emph{stdcall} en win32).

Exemple:

\lstinputlisting[style=customc]{OS/calling_conventions/x64.c}

\lstinputlisting[caption=MSVC 2012 /0b,style=customasmx86]{OS/calling_conventions/x64_MSVC_Ob.asm}

\myindex{Espace de travail}

Nous vyons ici clairement que des 7 arguments, 4 sont passés dans les registres et les 3 suivants sur
la pile.

Le prologue du code de la fonction f1() sauvegarde les arguments dans le \q{scratch space}---un espace
sur la pile précisément prévu à cet effet.

Le compilateur agit ainsi car il n'est pas certain par avance qu'il disposera de suffisamment de
registres pour toute la fonction en l'absence des 4 utilisés par les paramètres.

L'appelant est responsable de l'allocation du \q{scratch space} sur la pile.

\lstinputlisting[caption=\Optimizing MSVC 2012 /0b,style=customasmx86]{OS/calling_conventions/x64_MSVC_Ox_Ob.asm}

L'exemple compilé en branchant les optimisations, sera quasiment identique, si ce n'est que le
\q{scratch space} ne sera pas utilisé car inutile.

\myindex{x86!\Instructions!LEA}
\label{using_MOV_and_pack_of_LEA_to_load_values}

Notez également la manière dont MSVC 2012 optimise le chargement de certaines valeurs litérales dans
les registres en utilisant \LEA (\myref{sec:LEA}).
L'instruction \INS{MOV} utiliserait 1 octet de plus (5 au lieu de 4).

\myref{TaskMgr_LEA} est un autre exemple de cette pratique.

\myparagraph{Windows x64: Passage de \ITthis (\CCpp)}

Le pointeur \ITthis est passé dans le registre \RCX, le premier argument de la méthode dans \RDX, etc.
Pour un exemple, voir : \myref{simple_CPP_MSVC_x64}.

\subsubsection{Linux x64}

Le passage d'arguments par Linux pour x86-64 est quasiment identique à celui de Windows, si ce n'est
que 6 registres sont utilisés au lieu de 4 (\RDI, \RSI, \RDX, \RCX, \Reg{8}, \Reg{9}) et qu'il n'existe
pas de \q{scratch space}. L'glslink{callee}{appelé} conserve la possibilité de sauvegarder les registres sur la
pile s'il le souhaite ou en a besoin.

\lstinputlisting[caption=GCC 4.7.3 avec optimisation,style=customasmx86]{OS/calling_conventions/x64_linux_O3.s}

\myindex{AMD}

N.B.: Les valeurs sont enregistrées dans la partie basse des registres (e.g., EAX) et non pas dans la
totalité du registre 64 bits (RAX).
Ceci s'explique par le fait que l'écriture des 32 bits de poids faible du registre remet automatiquement
à zéro les 32 bits de poids fort.
On suppose qu'AMD a pris cette décision afin de simplifier le portage du code 32 bits vers l'architecture
x86-64.

\subsection{Valeur de retour de type \Tfloat et \Tdouble}
\myindex{float}
\myindex{double}

Dans toutes les conventions, à l'exception de l'environnement Win64, les valeurs de type \Tfloat ou
\Tdouble sont retournées dans le registre FPU \ST{0}.

En environnement Win64, les valeurs de type \Tfloat et \Tdouble sont respectivement retournées dans
les 32 bits de poids faible et dans les 64 bits du registre \XMM{0}.

\subsection{Modification des arguments}

Il arrive que les programmeurs \CCpp{} (et ceux d'autres \ac{PL}s aussi) se demandent ce qui se passe
s'ils modifient la valeur des arguments dans la fonction appelée.

La réponse est simple. Les arguments sont stockés sur la pile, et c'est elle qui est modifiée.

Une fois que l'\glslink{callee}{appelé} s'achève, la fonction appelante ne les utilise pas.
(Je n'a jamais constaté qu'il en aille autrement).

\lstinputlisting[style=customc]{OS/calling_conventions/change_arguments.c}

\lstinputlisting[caption=MSVC 2012,style=customasmx86]{OS/calling_conventions/change_arguments.asm}

% TODO (OllyDbg) пример как в стеке меняется $a$

Donc, oui, la valeur des arguments peut être modifiée sans problème.
Sous réserver que l'argument ne soit pas une \emph{references} en \Cpp{} (\myref{cpp_references}),
et que vous ne modifiez pas la valeur qui est référencée par un pointeur, l'effet de votre modification
ne se propagera pas au dehors de la fonction.

En théorie, après le retour de l'\glslink{callee}{appelé}, l'\glslink{caller}{appelant} pourrait récupérer l'argument modifié et
l'utiliser à sa guise.
Ceci pourrait peut être se faire dans un programme rédigé en assembleur.

Par exemple, un compilateur \CCpp génèrera un code comme celui-ci :

\begin{lstlisting}[style=customasmx86]
	push	456	; will be b
	push	123	; will be a
	call	f	; f() modifies its first argument
	add	esp, 2*4
\end{lstlisting}

Nous pouvons réécrire le code ainsi :

\begin{lstlisting}[style=customasmx86]
	push	456	; will be b
	push	123	; will be a
	call	f	; f() modifies its first argument
	pop	eax
	add	esp, 4
	; EAX=1st argument of f() modified in f()
\end{lstlisting}

Il est difficile d'imaginer pourquoi quelqu'un aurait besoin d'agir ainsi, mais en pratique
c'est possible.
Toujours est-il que les langages \CCpp ne permettent pas de faire ainsi.

% subsections
\subsection{Recevoir un argument par adresse}
\label{pointer_to_argument}

\dots mieux que cela, il est possible de passer à une fonction l'adresse d'un argument, plutôt que
la valeur de l'argument lui-même:

\lstinputlisting[style=customc]{OS/calling_conventions/ptr_to_argument/10.c}

Il est difficile de comprendre ce fonctionnement jusqu'à ce que l'on regarde le code:

\lstinputlisting[caption=MSVC 2010 optimisé,style=customasmx86]{OS/calling_conventions/ptr_to_argument/MSVC_2010_O3.asm}

L'argument $a$ est placé sur la pile et l'adresse de cet emplacement de pile est passé à une autre
fonction. Celle-ci modifie la valeur à l'adresse référencée par le pointeur, puis \printf affiche
la valeur après modification.

\par Le lecteur attentif se demandera peut-être ce qu'il en est avec les conventions d'appel qui
utilisent les registres pour passer les arguments.

C'est justement une des utilisations du \emph{Shadow Space}.

La valeur en entrée est copiée du registre vers le \emph{Shadow Space} dans la pile locale, puis
l'adresse de la pile locale est passée à la fonction appelée:

\lstinputlisting[caption=MSVC 2012 x64 optimisé,style=customasmx86]{OS/calling_conventions/ptr_to_argument/MSVC_2012_x64_O3.asm}

Le compilateur GCC lui aussi sauvegarde la valeur sur la pile locale:

\lstinputlisting[caption=GCC 4.9.1 optimisé x64,style=customasmx86]{OS/calling_conventions/ptr_to_argument/GCC491_x64_O3.s}

Le compilateur GCC pour ARM64 se comporte de la même manière, mais l'espace de sauvegarde sur la pile
est dénommé \emph{Register Save Area} :

\lstinputlisting[caption=GCC 4.9.1 optimisé ARM64,style=customasmARM]{OS/calling_conventions/ptr_to_argument/GCC49_ARM64_O3.s}

Enfin, nous constatons un usage similaire du \emph{Shadow Space} ici: \myref{variadic_arith_registers}.


%\subsection{Problème des ctypes en Python (devoir à la maison en assembleur x86)}

\myindex{Python!ctypes}
Un module Python ctypes peut appeler des fonctions externes dans des DLLs, .so's, etc.
Mais la convention d'appel (pour l'environnement 32-bit) doit être définie explicitement:

\begin{lstlisting}
"ctypes" exports the *cdll*, and on Windows *windll* and *oledll*
objects, for loading dynamic link libraries.

You load libraries by accessing them as attributes of these objects.
*cdll* loads libraries which export functions using the standard
"cdecl" calling convention, while *windll* libraries call functions
using the "stdcall" calling convention.
\end{lstlisting}
( \url{https://docs.python.org/3/library/ctypes.html} )%
\footnote{NDT:Projet de traduction en français: \url{https://docs.python.org/fr/3/library/ctypes.html} }

En fait, nous pouvons modifier le module ctypes (ou tout autre module d'appel),
afin qu'il appelle avec succès des fonctions externes cdecl ou stdcall, sans
connaître, ce qui se trouve où.
(Le nombre d'arguments, toutefois, doit être spécifié).

Il est possible de le résoudre en utilisant environ 5 à 10 instructions assembleur
x86 dans l'appelant.
Essayez de trouver ça.

% hint: check ESP



