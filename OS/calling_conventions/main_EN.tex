\mysection{Arguments passing methods (calling conventions)}
\label{sec:callingconventions}

\subsection{cdecl}
\myindex{cdecl}
\label{cdecl}

This is the most popular method for passing arguments to functions in the \CCpp languages.

The gls{caller} also must return the value of the \gls{stack pointer} (\ESP) to its initial state after the \gls{callee} function exits.

\begin{lstlisting}[caption=cdecl,style=customasmx86]
push arg3
push arg2
push arg1
call function
add esp, 12 ; returns ESP
\end{lstlisting}

\subsection{stdcall}
\label{sec:stdcall}
\myindex{stdcall}

\newcommand{\SIZEOFINT}{The size of an \Tint type variable is 4 in x86 systems and 8 in x64 systems}

It's almost the same as \emph{cdecl}, with the exception that the \gls{callee} must set \ESP to the initial state by executing the \TT{RET x} instruction instead of \RET, \\
where \TT{x = arguments number * sizeof(int)\footnote{\SIZEOFINT}}.
The \gls{caller} is not adjusting the \gls{stack pointer}, 
there are no \TT{add esp, x} instruction.

\begin{lstlisting}[caption=stdcall,style=customasmx86]
push arg3
push arg2
push arg1
call function

function:
... do something ...
ret 12
\end{lstlisting}

The method is ubiquitous in win32 standard libraries, but not in win64 (see below about win64).

\par For example, we can take the function from \myref{src:passing_arguments_ex} and change it slightly by adding the \TT{\_\_stdcall} modifier:

\begin{lstlisting}[style=customc]
int __stdcall f2 (int a, int b, int c)
{
	return a*b+c;
};
\end{lstlisting}

It is to be compiled in almost the same way as \myref{src:passing_arguments_ex_MSVC_cdecl}, but you will see \TT{RET 12} instead of \TT{RET}.
\ac{SP} is not updated in the \gls{caller}.

As a consequence, 
the number of function arguments can be easily deduced from the \TT{RETN n} instruction: just divide $n$ by 4.

\lstinputlisting[caption=MSVC 2010,style=customasmx86]{OS/calling_conventions/stdcall_ex.asm}

\subsubsection{Functions with variable number of arguments}

\printf-like functions are, probably, the only case of functions with a variable number of arguments in \CCpp,
but it is easy to illustrate an important difference between \emph{cdecl} and \emph{stdcall} with their help.
Let's start with the idea that the compiler knows the argument count of each \printf function call.

However, the called \printf, which is already compiled and located in MSVCRT.DLL (if we talk about Windows),
does not have any information about how much arguments were passed, however it can determine it from the format string.

Thus, if \printf would be a \emph{stdcall} function and restored \gls{stack pointer} to its initial state by counting
the number of arguments in the format string, this could be a dangerous situation, when one programmer's typo can
provoke a sudden program crash.
Thus it is not suitable for such functions to use \emph{stdcall}, \emph{cdecl} is better.

\subsection{fastcall}
\label{fastcall}
\myindex{fastcall}

That's the general naming for the method of passing some arguments via registers and the 
rest via the stack. It worked faster than \emph{cdecl}/\emph{stdcall} on older CPUs 
(because of smaller stack pressure).
It may not help to gain any significant performance on latest (much more complex) \ac{CPU}s, however.

It is not standardized, so the various compilers can do it differently.
It's a well known caveat: if you have two DLLs and the one uses another one, and they are built by different compilers with 
different \emph{fastcall} calling conventions, you can expect problems.

Both MSVC and GCC pass the first and second arguments via \ECX and \EDX and the rest of the arguments via the stack.

The \gls{stack pointer} must be restored to its initial state by the \gls{callee} (like in \emph{stdcall}).

\begin{lstlisting}[caption=fastcall,style=customasmx86]
push arg3
mov edx, arg2
mov ecx, arg1
call function

function:
.. do something ..
ret 4
\end{lstlisting}

For example, we may take the function from \myref{src:passing_arguments_ex} and change it slightly by adding a \TT{\_\_fastcall} modifier:

\begin{lstlisting}[style=customc]
int __fastcall f3 (int a, int b, int c)
{
	return a*b+c;
};
\end{lstlisting}

Here is how it is to be compiled:

\lstinputlisting[caption=\Optimizing MSVC 2010 /Ob0,style=customasmx86]{OS/calling_conventions/fastcall_ex.asm}

We see that the \gls{callee} returns \ac{SP} by using the \TT{RETN} instruction with an operand.

Which implies that the number of arguments can be deduced easily here as well.

\subsubsection{GCC regparm}

\newcommand{\URLREGPARMM}{\url{http://www.ohse.de/uwe/articles/gcc-attributes.html\#func-regparm}}

It is the evolution of \emph{fastcall}\footnote{\URLREGPARMM} in some sense.
With the \TT{-mregparm} option it is possible to set how many arguments are to be passed via registers (3 is the maximum).
Thus, the \EAX, \EDX and \ECX registers are to be used.

Of course, if the number the of arguments is less than 3, not all 3 registers are to be used.

The \gls{caller} restores the \gls{stack pointer} to its initial state.

For example, see (\myref{regparm}).

\subsubsection{Watcom/OpenWatcom}
\myindex{OpenWatcom}

Here it is called \q{register calling convention}.
The first 4 arguments are passed via the \EAX, \EDX, \EBX and \ECX registers.
All the rest---via the stack.

These functions have an underscore appended to the function name in order to distinguish them from 
those having a different calling convention.

\subsection{thiscall}
\myindex{thiscall}

This is passing the object's \ITthis pointer to the function-method, in \Cpp.

In MSVC, \ITthis is usually passed in the \ECX register.

In GCC, the \ITthis pointer is passed as the first function-method argument.
Thus it will be visible that all functions in assembly code have an extra argument, in comparison with the source code.

For an example, see (\myref{thiscall}).

\subsection{x86-64}
\myindex{x86-64}

\subsubsection{Windows x64}
\label{sec:callingconventions_win64}

The method of for passing arguments in Win64 somewhat resembles \TT{fastcall}.
The first 4 arguments are passed via \RCX, \RDX, \Reg{8} and \Reg{9}, the rest---via the stack.
The \gls{caller} also must prepare space for 32 bytes or 4 64-bit values,
so then the \gls{callee} can save there the first 4 arguments.
Short functions may use the arguments' values just from the registers,
but larger ones may save their values for further use.

The \gls{caller} also must return the \gls{stack pointer} into its initial state.

This calling convention is also used in Windows x86-64 system DLLs 
(instead of \emph{stdcall} in win32).

Example:

\lstinputlisting[style=customc]{OS/calling_conventions/x64.c}

\lstinputlisting[caption=MSVC 2012 /0b,style=customasmx86]{OS/calling_conventions/x64_MSVC_Ob.asm}

\myindex{Scratch space}

Here we clearly see how 7 arguments are passed: 4 via registers and the remaining 3 via the stack.

The code of the f1() function's prologue saves the arguments in the \q{scratch space}---a space in the stack
intended exactly for this purpose.

This is arranged so because the compiler cannot be sure that there will be enough registers to use without these 4,
which will otherwise be occupied by the arguments until the function's execution end.

The \q{scratch space} allocation in the stack is the caller's duty.

\lstinputlisting[caption=\Optimizing MSVC 2012 /0b,style=customasmx86]{OS/calling_conventions/x64_MSVC_Ox_Ob.asm}

If we compile the example with optimizations, it is to be almost the same, 
but the \q{scratch space} will not be used, because it won't be needed.

\myindex{x86!\Instructions!LEA}
\label{using_MOV_and_pack_of_LEA_to_load_values}

Also take a look on how MSVC 2012 optimizes the loading of primitive values into registers by using \LEA (\myref{sec:LEA}).
\INS{MOV} would be 1 byte longer here (5 instead of 4).

Another example of such thing is: \myref{TaskMgr_LEA}.

\myparagraph{Windows x64: Passing \ITthis (\CCpp)}

The \ITthis pointer is passed in \RCX, the first argument of the method is in \RDX, etc.
For an example see: \myref{simple_CPP_MSVC_x64}.
 
\subsubsection{Linux x64}

The way arguments are passed in Linux for x86-64 is almost the same as in Windows, but 6 registers are
used instead of 4 (\RDI, \RSI, \RDX, \RCX, \Reg{8}, \Reg{9}) and there is no \q{scratch space}, 
although the \gls{callee} may save the register values in the stack, if it needs/wants to.

\lstinputlisting[caption=\Optimizing GCC 4.7.3,style=customasmx86]{OS/calling_conventions/x64_linux_O3.s}

\myindex{AMD}

N.B.: here the values are written into the 32-bit parts of the registers (e.g., EAX) but not in the whole 64-bit 
register (RAX).
This is because each write to the low 32-bit part of a register automatically clears the high 32 bits.
Supposedly, it was decided in AMD to do so to simplify porting code to x86-64.

\subsection{Return values of \Tfloat and \Tdouble type}
\myindex{float}
\myindex{double}

In all conventions except in Win64, the values of type \Tfloat or \Tdouble are returned via the FPU register \ST{0}.

In Win64, the values of \Tfloat and \Tdouble types are returned 
in the low 32 or 64 bits of the \XMM{0} register.

\subsection{Modifying arguments}

Sometimes, \CCpp{} programmers (not limited to these \ac{PL}s, though),
may ask, what can happen if they modify the arguments?

The answer is simple: the arguments are stored in the stack, 
that is where the modification takes place.

The calling functions is not using them after the \gls{callee}'s exit (the author of these lines has never seen any such case in his practice).

\lstinputlisting[style=customc]{OS/calling_conventions/change_arguments.c}

\lstinputlisting[caption=MSVC 2012,style=customasmx86]{OS/calling_conventions/change_arguments.asm}

% TODO (OllyDbg) пример как в стеке меняется $a$

So yes, one can modify the arguments easily.
Of course, if it is not \emph{references} in \Cpp{} (\myref{cpp_references}),
and if you don't modify data to which a pointer points to, 
then the effect will not propagate outside the current function.

Theoretically, after the \gls{callee}'s return, 
the \gls{caller} could get the modified argument and use it somehow.
Maybe if it is written directly in assembly language.

For example, code like this will be generated by usual \CCpp compiler:

\begin{lstlisting}[style=customasmx86]
	push	456	; will be b
	push	123	; will be a
	call	f	; f() modifies its first argument
	add	esp, 2*4
\end{lstlisting}

We can rewrite this code like:

\begin{lstlisting}[style=customasmx86]
	push	456	; will be b
	push	123	; will be a
	call	f	; f() modifies its first argument
	pop	eax
	add	esp, 4
	; EAX=1st argument of f() modified in f()
\end{lstlisting}

Hard to imagine, why anyone would need this, but this is possible in practice.
Nevertheless, the \CCpp languages standards don't offer any way to do so.

% subsections
\mysection{Task manager practical joke (Windows Vista)}
\myindex{Windows!Windows Vista}

Let's see if it's possible to hack Task Manager slightly so it would detect more \ac{CPU} cores.

\myindex{Windows!NTAPI}

Let us first think, how does the Task Manager know the number of cores?

There is the \TT{GetSystemInfo()} win32 function present in win32 userspace which can tell us this.
But it's not imported in \TT{taskmgr.exe}.

There is, however, another one in \gls{NTAPI}, \TT{NtQuerySystemInformation()}, 
which is used in \TT{taskmgr.exe} in several places.

To get the number of cores, one has to call this function with the \TT{SystemBasicInformation} constant
as a first argument (which is zero
\footnote{\href{http://msdn.microsoft.com/en-us/library/windows/desktop/ms724509(v=vs.85).aspx}{MSDN}}).

The second argument has to point to the buffer which is getting all the information.

So we have to find all calls to the \\
\TT{NtQuerySystemInformation(0, ?, ?, ?)} function.
Let's open \TT{taskmgr.exe} in IDA. 
\myindex{Windows!PDB}

What is always good about Microsoft executables is that IDA can download the corresponding \gls{PDB} 
file for this executable and show all function names.

It is visible that Task Manager is written in \Cpp and some of the function names and classes are really 
speaking for themselves.
There are classes CAdapter, CNetPage, CPerfPage, CProcInfo, CProcPage, CSvcPage, 
CTaskPage, CUserPage.

Apparently, each class corresponds to each tab in Task Manager.

Let's visit each call and add comment with the value which is passed as the first function argument.
We will write \q{not zero} at some places, because the value there was clearly not zero, 
but something really different (more about this in the second part of this chapter).

And we are looking for zero passed as argument, after all.

\begin{figure}[H]
\centering
\myincludegraphics{examples/taskmgr/IDA_xrefs.png}
\caption{IDA: cross references to NtQuerySystemInformation()}
\end{figure}

Yes, the names are really speaking for themselves.

When we closely investigate each place where\\
\TT{NtQuerySystemInformation(0, ?, ?, ?)} is called,
we quickly find what we need in the \TT{InitPerfInfo()} function:

\lstinputlisting[caption=taskmgr.exe (Windows Vista),style=customasmx86]{examples/taskmgr/taskmgr.lst}

\TT{g\_cProcessors} is a global variable, and this name has been assigned by 
IDA according to the \gls{PDB} loaded from Microsoft's symbol server.

The byte is taken from \TT{var\_C20}. 
And \TT{var\_C58} is passed to\\
\TT{NtQuerySystemInformation()} 
as a pointer to the receiving buffer.
The difference between 0xC20 and 0xC58 is 0x38 (56).

Let's take a look at format of the return structure, which we can find in MSDN:

\begin{lstlisting}[style=customc]
typedef struct _SYSTEM_BASIC_INFORMATION {
    BYTE Reserved1[24];
    PVOID Reserved2[4];
    CCHAR NumberOfProcessors;
} SYSTEM_BASIC_INFORMATION;
\end{lstlisting}

This is a x64 system, so each PVOID takes 8 bytes.

All \emph{reserved} fields in the structure take $24+4*8=56$ bytes.

Oh yes, this implies that \TT{var\_C20} is the local stack is exactly the
\TT{NumberOfProcessors} field of the \TT{SYSTEM\_BASIC\_INFORMATION} structure.

Let's check our guess.
Copy \TT{taskmgr.exe} from \TT{C:\textbackslash{}Windows\textbackslash{}System32} 
to some other folder 
(so the \emph{Windows Resource Protection} 
will not try to restore the patched \TT{taskmgr.exe}).

Let's open it in Hiew and find the place:

\begin{figure}[H]
\centering
\myincludegraphics{examples/taskmgr/hiew2.png}
\caption{Hiew: find the place to be patched}
\end{figure}

Let's replace the \TT{MOVZX} instruction with ours.
Let's pretend we've got 64 CPU cores.

Add one additional \ac{NOP} (because our instruction is shorter than the original one):

\begin{figure}[H]
\centering
\myincludegraphics{examples/taskmgr/hiew1.png}
\caption{Hiew: patch it}
\end{figure}

And it works!
Of course, the data in the graphs is not correct.

At times, Task Manager even shows an overall CPU load of more than 100\%.

\begin{figure}[H]
\centering
\myincludegraphics{examples/taskmgr/taskmgr_64cpu_crop.png}
\caption{Fooled Windows Task Manager}
\end{figure}

The biggest number Task Manager does not crash with is 64.

Apparently, Task Manager in Windows Vista was not tested on computers with a large number of cores.

So there are probably some static data structure(s) inside it limited to 64 cores.

\subsection{Using LEA to load values}
\label{TaskMgr_LEA}

Sometimes, \TT{LEA} is used in \TT{taskmgr.exe} instead of \TT{MOV} to set the first argument of \\
\TT{NtQuerySystemInformation()}:

\lstinputlisting[caption=taskmgr.exe (Windows Vista),style=customasmx86]{examples/taskmgr/taskmgr2.lst}

\myindex{x86!\Instructions!LEA}

Perhaps \ac{MSVC} did so because machine code of \INS{LEA} is shorter than \INS{MOV REG, 5} (would be 5 instead of 4).

\INS{LEA} with offset in $-128..127$ range (offset will occupy 1 byte in opcode) with 32-bit registers is even shorter (for lack of REX prefix)---3 bytes.

Another example of such thing is: \myref{using_MOV_and_pack_of_LEA_to_load_values}.

\subsection{Python ctypes problem (x86 assembly homework)}

\myindex{Python!ctypes}
A Python ctypes module can call external functions in DLLs, .so's, etc.
But calling convention (for 32-bit environment) must be specified explicitely:

\begin{lstlisting}
"ctypes" exports the *cdll*, and on Windows *windll* and *oledll*
objects, for loading dynamic link libraries.

You load libraries by accessing them as attributes of these objects.
*cdll* loads libraries which export functions using the standard
"cdecl" calling convention, while *windll* libraries call functions
using the "stdcall" calling convention.
\end{lstlisting}
( \url{https://docs.python.org/3/library/ctypes.html} )

In fact, we can modify ctypes module (or any other caller code), so that it will successfully
call external cdecl or stdcall functions, without knowledge, which is where.
(Number of arguments, however, is to be specified).

This is possible to solve using maybe 5-10 x86 assembly instructions in caller.
Try to find out these.

% hint: check ESP


\subsection{Cdecl example: a DLL}
\label{cdecl_DLL}

Let's back to the fact that this is not very important how to declare the \verb|main()| function: \myref{main_arguments}.

This is a real story: once upon a time I wanted to replace an original DLL file in some software by mine.
First I enumerated names of all DLL exports and made a function in my own replacement DLL for each function in the original DLL, like:

\begin{lstlisting}[style=customc]
void function1 ()
{
	write_to_log ("function1() called\n");
};
\end{lstlisting}

I wanted to see, which functions are called during run, and when.
However, I was in hurry and had no time to deduce arguments count for each function, let alone data types.
So each function in my replacement DLL had no argumnts whatsoever.
But everything worked, because all functions had \emph{cdecl} calling convention.
(It wouldn't work if functions had \emph{stdcall} calling convention.)
It also worked for x64 version.

And then I did a next step: I deduced argument types for some functions.
But I made several mistakes, for example, the original function took 3 arguments, but I knew only about 2, etc.

Still, it worked.
At the beginning, my replacement DLL just ignored all arguments.
Then, it ignored the 3rd argument.



