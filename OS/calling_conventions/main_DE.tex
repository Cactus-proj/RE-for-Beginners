\mysection{Methoden zur Argumentenübergabe (Aufrufkonventionen)}
\label{sec:callingconventions}

\subsection{cdecl}
\myindex{cdecl}
\label{cdecl}

Hierbei handelt es sich um die am weitesten verbreitete Methode um in \CCpp-Sprachen
Argumente an Funktionen zu übergeben.

Der \gls{caller} muss den Wert des \gls{stack pointer} (\ESP) auf den ursprünglichen Stand
bringen, nachdem \gls{callee}-Funktion beendet wurde.

\begin{lstlisting}[caption=cdecl]
push Argument3
push Argument2
push Argument1
call Funktion
add esp, 12 ; gibt ESP zurueck
\end{lstlisting}

\subsection{stdcall}
\label{sec:stdcall}
\myindex{stdcall}

\newcommand{\SIZEOFINT}{Die Größe einer Variablen vom Datentyp \Tint ist 4 in x86-Systemen und 8 in x64-Systemen}

Dies ist fast gleich zu der \emph{cdecl}-Aufrufkonvention, mit Ausnahme, dass die
\gls{callee} den Wert von \ESP auf den ursprünglichen Wert setzen muss. Dies geschieht
durch die \TT{RET x} anstatt \RET, wobei gilt \TT{x = Nummer des Arguments * sizeof(int)\footnote{\SIZEOFINT}}.

Der \gls{caller} passt den \gls{stack pointer} nicht an, es sind also keine \TT{add esp, x}-Anweisungen vorhanden.

\begin{lstlisting}[caption=stdcall]
push Argument3
push Argument2
push Argument1
call Funktion

Funktion:
... tue etwas ...
ret 12
\end{lstlisting}

Diese Methode ist in Win32-Standard-Bibliotheken allgegenwärtig, fehlt jedoch in Win64 (siehe unten).

\par Beispielsweise kann die Funktion von \myref{src:passing_arguments_ex} genommen werden und durch
Hinzufügen des \TT{\_\_stdcall}-Modifizierers leicht verändert werden:

\begin{lstlisting}
int __stdcall f2 (int a, int b, int c)
{
	return a*b+c;
};
\end{lstlisting}

Das Kompilat ist fast das gleiche wie bei \myref{src:passing_arguments_ex_MSVC_cdecl},
jedoch wird \TT{RET 12} anstatt \TT{RET} genutzt. Der \ac{SP} wird im \gls{caller} nicht
aktualisiert.

Als Konsequenz daraus kann die Anzahl der Funktionsargumente einfach von der \TT{RETN n}-Anweisung
abgeleitet werden, indem $n$ durch 4 geteilt wird

%TODO Compilation of german version fails with style=customasm
%\lstinputlisting[caption=MSVC 2010,style=customasm]{OS/calling_conventions/stdcall_ex.asm}
\lstinputlisting[caption=MSVC 2010]{OS/calling_conventions/stdcall_ex.asm}

\subsubsection{Funktionen mit einer variablen Anzahl von Argumenten}

\printf-ähnliche Funktionen sind vielleicht die einzigen Funktionen in \CCpp mit einer Variablen
Anzahl von Argumenten, aber mit ihnen kann ein wichtiger Unterschied zwischen \emph{cdecl} und
\emph{stdcall} veranschaulicht werden.
Beginnen wir mit der Vorstellung, dass der Compiler die Anzahl der Argumente für jeden Aufruf
von \printf kennt.

Die aufgerufene und bereits kompilierte \printf-Funktion befindet sich in der Datei MSVCRT.DLL
(wenn über Windows geredet wird) und hat aber keinerlei Informationen darüber wie viele Argumente
übergeben wurden; dies kann jedoch über den Formatstring herausgefunden werden.

%TODO: The following sentence is much too long.
Wenn \printf eine \emph{stdcall}-Funktion wäre und den \gls{stack pointer} durch Zählen der Zahl
der Argumente im Formatstring auf den ursprünglichen Wert setzen würde, kann eine gefährliche
Situation entstehen, da ein Schreibfehler des Programmierers zu einem plötzliche Programmabsturz
führen könnte.
Aus diesem Grund ist für diese Art von Funktionen \emph{stdcall} ungeeignet und \emph{cdecl} ist
zu bevorzugen.

\subsection{fastcall}
\label{fastcall}
\myindex{fastcall}

Dies ist der allgemeine Name für eine Methode, in der einige Argumente mittels Registern und
der Rest über den Stack übergeben werden. Für ältere CPUs ist \emph{fastcall} schneller als
\emph{cdecl} und \emph{stdcall} wegen der geringeren Stack-Nutzung.
Auf neueren \ac{CPU}s wird dieser Ansatz vermutlich keine signifikante Geschwindigkeitserhöhung
nach sich ziehen.

\emph{fastcall} ist nicht standardisiert, so das verschiedene Compiler eine unterschiedliche
Umsetzung machen können.
Dies kann zu Problemen führen, wenn zwei DLLs genutzt werden, von denen eine die andere nutzt
und durch das Nutzen verschiedener Compiler unterschiedliche \emph{fastcall}-Aufrufkonventionen
genutzt werden.

Sowohl MSVC als auch GCC übergeben das erste und zweite Argument über \ECX und \EDX und den Rest
der Arguments mittels des Stacks.

Der \gls{stack pointer} muss vom \gls{callee} auf den ursprünglichen Wert gesetzt werden,
wie in \emph{stdcall} auch.

\begin{lstlisting}[caption=fastcall]
push Argument3
mov edx, Argument2
mov ecx, Argument1
call Funktion

Funktion:
.. tue etwas ..
ret 4
\end{lstlisting}

Beispielsweise kann die Funktion von \myref{src:passing_arguments_ex} genommen werden und durch
Hinzufügen des \TT{\_\_fastcall}-Modifizierers leicht verändert werden:

\begin{lstlisting}
int __fastcall f3 (int a, int b, int c)
{
	return a*b+c;
};
\end{lstlisting}

Nachfolgend das Kompilat:

%TODO Compilation of german version fails with style=customasm
%\lstinputlisting[caption=\Optimizing MSVC 2010 /Ob0,style=customasm]{OS/calling_conventions/fastcall_ex.asm}
\lstinputlisting[caption=\Optimizing MSVC 2010 /Ob0]{OS/calling_conventions/fastcall_ex.asm}

Es ist erkennbar, dass der \gls{callee} den \ac{SP} mit der \TT{RETN}-Anweisung
und einem Operanden auf den ursprünglichen Wert setzt.

Dies bedeutet, dass die Anzahl der Argumente ebenfalls einfach abgeleitet werden kann.

\subsubsection{GCC regparm}

\newcommand{\URLREGPARMM}{\url{http://www.ohse.de/uwe/articles/gcc-attributes.html\#func-regparm}}

Dies ist in gewisser Weise die Weiterentwicklung von \emph{fastcall}\footnote{\URLREGPARMM}.
Mit der \TT{-mregparm}-Option ist es möglich festzulegen, wieviele Argumente per Register
übergeben werden (maximal 3).
Aus diesem Grund werden die Register \EAX, \EDX und \ECX genutzt.

Natürlich werden, wenn die Anzahl der Argumente kleiner als drei ist, nicht alle drei
Register genutzt.

Der \gls{caller} setzt den \gls{stack pointer} auf den initialen Zustand.

Als Beispiel siehe (\myref{regparm}).

\subsubsection{Watcom/OpenWatcom}
\myindex{OpenWatcom}

Hier erfolgt der Aufruf mit der \q{Register-Aufruf-Konvention}.
Die ersten vier Argumente werden in den Registern \EAX, \EDX, \EBX und \ECX übergeben,
der Rest auf dem Stack.

Diese Funktionen haben einen Unterstrich an den Funktionsnamen angehängt, um sie von
den anderen Aufrufkonventionen unterscheiden zu können.

\subsection{thiscall}
\myindex{thiscall}

Hier wird der \ITthis-Zeiger des Objekts an die Methode in \Cpp übergeben.

In MSVC wird \ITthis üblicherweise im \ECX-Register übergeben.

In GCC wird der \ITthis-Zeiger im ersten Argument der Methode übergeben.
Es ist sehr offensichtlich dass intern alle Methoden ein zusätzliches Argument haben.

Als Beispiel siehe (\myref{thiscall}).

\subsection{x86-64}
\myindex{x86-64}

\subsubsection{Windows x64}
\label{sec:callingconventions_win64}

Die Art Argumente zu Übergeben ähnelt in Win64 in gewisser Weise \TT{fastcall}.
Die ersten vier Argumente werden in den Registern \RCX, \RDX, \Reg{8} und \Reg{9}
übergeben und der Rest auf dem Stack.
Der \gls{caller} muss Platz für 32 Byte, also 4 64-Bit-Werte bereitstellen, so
dass der \gls{callee} dort die ersten vier Argumente speichern kann.
Kurze Funktionen können die Werte der Argumente direkt aus den Registern lesen,
während längere Funktionen diese für späteren Gebrauch zwischenspeichern sollten.

Der \gls{caller} muss den \gls{stack pointer} auf den vorherigen Zustand zurücksetzen.

Diese Aufrufkonvention wird auch in den Windows x86-64-System-DLLs genutzt
(anstatt \emph{stdcall} in Win32).

Beispiel:

\lstinputlisting[style=customc]{OS/calling_conventions/x64.c}

%TODO Compilation of german version fails with style=customasm
%\lstinputlisting[caption=MSVC 2012 /0b,style=customasm]{OS/calling_conventions/x64_MSVC_Ob.asm}
\lstinputlisting[caption=MSVC 2012 /0b]{OS/calling_conventions/x64_MSVC_Ob.asm}

\myindex{Scratch space}

Es ist hier klar erkennbar, wie sieben Argumente übergeben werden: vier in den
Registern und die drei restlichen auf dem Stack.

Der Code des f1()-Funktionsprolog sichert die Argumente in dem \q{Scratch Space},
einer Stelle auf dem Stack, die genau für diese Zwecke existiert.

Dies ist so realisiert, weil der Compiler nicht sicher sein kann, dass genug Register
ohne diese vier genutzt werden können und sie sonst durch die Argumente verändert
werden können, bis die Funktion beendet wird.

Die Bereitstellung des \q{Scratch Space} auf dem Stack ist Aufgabe der aufrufenden Funktion.

%TODO Compilation of german version fails with style=customasm
%\lstinputlisting[caption=\Optimizing MSVC 2012 /0b,style=customasm]{OS/calling_conventions/x64_MSVC_Ox_Ob.asm}
\lstinputlisting[caption=\Optimizing MSVC 2012 /0b]{OS/calling_conventions/x64_MSVC_Ox_Ob.asm}

Wenn das Beispiel ohne Optimierung compiliert wird, ist das Ergebnis fast das gleiche,
lediglich der \q{Scratch Space} ist unnötig und wird nicht genutzt.

\myindex{x86!\Instructions!LEA}
\label{using_MOV_and_pack_of_LEA_to_load_values}

Beachtenswert ist auch, wie MSVC 2012 das Laden von einfachen Werten in Register
durch das Nutzen von \LEA (\myref{sec:LEA}) optimiert.
\INS{MOV} wäre hier ein Byte länger (5 anstatt 4).

Ein weiteres Beispiel so eines Sachverhalts ist: \myref{TaskMgr_LEA}.

\myparagraph{Windows x64: Übergeben von \ITthis (\CCpp)}

Der \ITthis-Zeiger wird in \RCX übergeben, das erste Argument der Methode ist \RDX, usw.
Ein Beispiel ist hier zu sehen: \myref{simple_CPP_MSVC_x64}.

\subsubsection{Linux x64}

Die Art wie Linux für x86-64 Argumente übergibt ist fast die Gleiche wie in Windows,
jedoch werden sechs anstatt vier Register genutzt (\RDI, \RSI, \RDX, \RCX, \Reg{8}, \Reg{9})
und es gibt keinen \q{Scratch Space}, auch wenn der \gls{callee} die Registerwerte auf
dem Stack speichern kann wenn er dies will oder muss.

%TODO Compilation of german version fails with style=customasm
%\lstinputlisting[caption=\Optimizing GCC 4.7.3,style=customasm]{OS/calling_conventions/x64_linux_O3.s}
\lstinputlisting[caption=\Optimizing GCC 4.7.3]{OS/calling_conventions/x64_linux_O3.s}

\myindex{AMD}
Zur Beachtung: die Werte werden hier in die 32-Bit-Teile der Register (z.B.EAX) geschrieben,
aber nicht in die kompletten 64-Bit-Register (RAX).
Dies wird gemacht, weil jeder Schreibzugang auf die niederwertigen 32-Bit-Teile eines
Registers automatisch den höherwertigen Teil zurücksetzt.

Vermutlich wurde dies bei AMD so eingeführt um die Portierung des Codes zu x86-64 zu vereinfachen.

\subsection{Rückgabewerte von \Tfloat- und \Tdouble-Typen}
\myindex{float}
\myindex{double}

In allen Konventionen außer in Win64, werden di Werte vom Typ \Tfloat oder \Tdouble
in dem FPU-Register \ST{0} zurückgegeben.

In Win64 werden die Werte vom Typ \Tfloat oder \Tdouble in den niederwertigen 32 oder
64 Bit des \XMM{0}-Registers zurückgegeben.

\subsection{Verändern von Argumenten}

Manchmal fragen \CCpp{}-Programmierer (obwohl nicht auf diese \ac{PL} beschränkt),
was passieren kann wenn die Funktionsargumente verändert werden.

Die Antwort ist einfach: die Argumente sind auf dem Stack gespeichert und hier
werden auch die Veränderungen vorgenommen.

Die aufzurufende Funktion wird diese nicht nach dem Verlassen des \gls{callee}
nutzen (der Autor dieser Linien hat so einen Fall in der Praxis noch nie gesehen).

\lstinputlisting[style=customc]{OS/calling_conventions/change_arguments.c}

%TODO Compilation of german version fails with style=customasm
%\lstinputlisting[caption=MSVC 2012,style=customasm]{OS/calling_conventions/change_arguments.asm}
\lstinputlisting[caption=MSVC 2012,style=customasmx86]{OS/calling_conventions/change_arguments.asm}

% TODO (OllyDbg) пример как в стеке меняется $a$

Also: ja, die Argumente können einfach modifiziert werden.
Natürlich, wenn diese keine \emph{Referenz} in  \Cpp{} (\myref{cpp_references}) ist,
und nicht die Daten verändert werden auf die der Zeiger zeigt, wird der Effekt nicht
außerhalb der aktuellen Funktion sichtbar sein.

Theoretisch kann der \gls{caller} die modifizierten Argumente auf irgendeine Weise
nutzen nachdem der \gls{callee} beendet wurde.
Vielleicht wenn dies direkt in Assembler programmiert ist.

Beispielsweise wird Code wie folgt von gewöhnlichen \CCpp-Compilern erzeugt:

\begin{lstlisting}[style=customasmx86]
	push	456	; wird gleich b sein
	push	123	; wird gleich a sein
	call	f	; f() modifiziert das erste Argument
	add	esp, 2*4
\end{lstlisting}

Der Code kann wie folgt neu geschrieben werden:

\begin{lstlisting}[style=customasmx86]
	push	456	; wird gleich b sein
	push	123	; wird gleich a sein
	call	f	; f() modifiziert das erste Argument
	pop	eax
	add	esp, 4
	; EAX=Erstes Argument von f() modifiziert in f()
\end{lstlisting}

Es ist scher vorzustellen warum jemand dies tun sollte, aber in der Praxis ist es möglich.
Nichtsdestotrotz bietet der \CCpp-Standard keinen Möglichkeit dies zu tun.

% subsections
\subsection{Text strings}

\subsubsection{\CCpp}

\label{C_strings}

Die normalen C-strings sind NULL-Terminiert (\ac{ASCIIZ}-strings).

Der Grund warum C Stringformatierung so ist wie sie ist (NULL-Terminiert) scheint ein Historischer zu sein.
In [Dennis M. Ritchie, \emph{The Evolution of the Unix Time-sharing System}, (1979)] kann man nach lesen:

\begin{framed}
\begin{quotation}
Ein kleiner Unterschied war das die I/O Einheit ein ``word'' war, nicht ein Byte, weil die PDP-7 eine word-adressierte
Maschine war. In der Praxis bedeutete das lediglich das alle Programme die mit Zeichen Streams arbeiteten, das NULL 
Zeichen ignorieren mussten, weil die NULL benutzt wurde um eine Datei bis zu einer Graden Zahl an Bytes auf zu f\"ullen.

\end{quotation}
\end{framed}

\myindex{Hiew}

In Hiew oder FAR Manager sehen diese Strings so aus:

\begin{lstlisting}[style=customc]
int main()
{
	printf ("Hello, world!\n");
};
\end{lstlisting}

\begin{figure}[H]
\centering
\includegraphics[width=0.6\textwidth]{digging_into_code/strings/C-string.png}
\caption{Hiew}
\end{figure}

% FIXME видно \n в конце, потом пробел

\subsubsection{Borland Delphi}
\myindex{Pascal}
\myindex{Borland Delphi}

Dem String in Passcal und Borland Delphi h\"angt eine 8 oder 32-Bit Zeichenkette an. 

Zum Beispiel:

\begin{lstlisting}[caption=Delphi,style=customasmx86]
CODE:00518AC8                 dd 19h
CODE:00518ACC aLoading___Plea db 'Loading... , please wait.',0

...

CODE:00518AFC                 dd 10h
CODE:00518B00 aPreparingRun__ db 'Preparing run...',0
\end{lstlisting}

\subsubsection{Unicode}

\myindex{Unicode}

Oft, ist das was Unicode genannt wird einfach eine Methode um Strings zu codieren, bei denen jedes Zeichen 2 Byte oder 
16 Bits verbraucht. Das ist ein h\a"ufiger Terminologischer Fehler. Unicode ist ein Standard bei dem eine Nummer 
zu einem der vielen Schreibsysteme der Welt zugeordnet wird, aber es beschreibt nicht die codierungs Methode. 

\myindex{UTF-8}
\myindex{UTF-16LE}

Die bekannteste Methode zu Codieren ist: UTF-8 ( ist weit verteilt im Internet und auf *NIX Systemen) und UTF-16LE ( wird bei Windows benutzt). 

\myparagraph{UTF-8}

\myindex{UTF-8}
UTF-8 ist eine der erfolgreichsten Methoden um Zeichen zu codieren.
Alle Latein Zeichen werden codiert so wie in ASCII, und alle Symbole nach der
ASCII Tabelle wurden codiert mit zus\"atzlichen Bytes. 0 wird codiert als davor,
also arbeiten alle Standard C String Funktionen mit UTF-8 Strings wie mit jedem anderen String auch.

Lasst uns anschauen wie die Symbole in verschiedenen anderen Sprachen nach UTF-8 Codiert werden und 
wie man sie als FAR aussehen lassen kann, durch das benutzen der codepage 437.

\footnote{Beispiel und \"Ubersetzung k\o"nnen von hier bezogen werden:  
\url{http://www.columbia.edu/~fdc/utf8/}}:

\begin{figure}[H]
\centering
\includegraphics[width=0.6\textwidth]{digging_into_code/strings/multilang_sampler.png}
\end{figure}

% FIXME: cut it
\begin{figure}[H]
\centering
\myincludegraphics{digging_into_code/strings/multilang_sampler_UTF8.png}
\caption{FAR: UTF-8}
\end{figure}

Wie man hier sehen kann, der Englische String sieht genauso aus wie sein Gegenst\"uck in ASCII.

Die Ungarische Sprache benutzt Latein Symbole plus ein paar Symbole mit diacritic Markierungen.

Diese Symbole werden mit mehreren Bytes codiert, diese wurden rot unterstrichen.
Das gleiche gilt f\"ur die Isl\"andischen und Polnischen Sprachen.

Es gibt auch das \q{Euro} W\"ahrungs Symbol im Standard, das Symbol wurde mit 3 Bytes Codiert.

Der Rest der Schreibsysteme hat keinen Bezug zu Latein.

Zumindest in Russisch, Arabisch, Hebr\"aisch und Hindu k\"onnen wir wiederkehrende Bytes erkennen und das ist nicht mal \"uberraschend:
Alle Zeichen eines Schreibsystems werden normalerweise in der selben Unicode Tabelle angelegt, also f\"angt ihr code mit den 
immer gleichen nummern an. % <--- Wird anders \"ubersetzt.

Zu Anfang, noch vor dem \q{How much?} String sehen wir 3 Bytes, die tats\"achlich das \ac{BOM} darstellen.
Das \ac{BOM} definiert das Codierungssystem das benutzt werden soll.

\myparagraph{UTF-16LE}

\myindex{UTF-16LE}
\myindex{Windows!Win32}
Viele win32 Funktionen in Windows haben die Suffixe \TT{-A} und \TT{-W}. 
Der erste Typ Funktionen arbeitet mit normalen Strings, der andere Typ mit 
UTF-16LE Strings (\emph{wide}). 

Im zweiten Fall, wird jedes Symbol normal als 16-Bit Wert des Typs \emph{short} gespeichert.

Die Latein Symbole in UFT-16 Strings sehen in Hiew oder FAR aus als w\"aren sie mit Null Bytes verschachtelt:

\begin{lstlisting}[style=customc]
int wmain()
{
	wprintf (L"Hello, world!\n");
};
\end{lstlisting}

\begin{figure}[H]
\centering
\includegraphics[width=0.6\textwidth]{digging_into_code/strings/UTF16-string.png}
\caption{Hiew}
\end{figure}

Wir k\o"nnen das oft auch in gls{Windows NT} System Dateien sehen:

\begin{figure}[H]
\centering
\includegraphics[width=0.6\textwidth]{digging_into_code/strings/ntoskrnl_UTF16.png}
\caption{Hiew}
\end{figure}

\myindex{IDA}
Strings mit Zeichen die exakt 2 Bytes verbrauchen werden \q{Unicode} in \IDA genannt:

\begin{lstlisting}[style=customasmx86]
.data:0040E000 aHelloWorld:
.data:0040E000                 unicode 0, <Hello, world!>
.data:0040E000                 dw 0Ah, 0
\end{lstlisting}

Hier sieht man wie Russische Sprache in UTF-16LE Codiert wird:

\begin{figure}[H]
\centering
\includegraphics[width=0.6\textwidth]{digging_into_code/strings/russian_UTF16.png}
\caption{Hiew: UTF-16LE}
\end{figure}

Was man leicht sehen kann ist das die Symbole durchzogen sind von den Diamant Zeichen (das im ASCII code mit 4 codiert wird).
Tats\"achlich, findet man die Kyrillischen Symbole in der vierten Unicode Tabelle.
Deswegen, alle Kyrillischen Symbole in UTF-16LE findet man im Bereich \TT{0x400-0x4FF}.

Lass uns noch mal zu dem Beispiel gehen mit dem String der in verschiedenen Sprachen geschrieben ist.
Hier sieht man wie der String in UTF-16LE aussieht. 

% FIXME: cut it
\begin{figure}[H]
\centering
\myincludegraphics{digging_into_code/strings/multilang_sampler_UTF16.png}
\caption{FAR: UTF-16LE}
\end{figure}

Hier k\"onnen wir auch das \ac{BOM} am Anfang sehen. 
Alle Latein Zeichen enthalten Null Bytes.

Manche Zeichen mit unterschiedlichen Markierungen (Ungarisch und Isl\"andisch) wurden rot unterstrichen.

% subsection:
\input{digging_into_code/strings/base64_DE}


%\input{OS/calling_conventions/ctypes_DE}
%\input{OS/calling_conventions/DLL_cdecl_DE}

