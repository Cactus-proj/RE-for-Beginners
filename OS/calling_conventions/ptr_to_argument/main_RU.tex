\subsection{Указатель на аргумент функции}
\label{pointer_to_argument}

\dots и даже более того, можно взять указатель на аргумент функции и передать его в другую функцию:

\lstinputlisting[style=customc]{OS/calling_conventions/ptr_to_argument/10.c}

Трудно понять, как это работает, пока мы не посмотрим на код:

\lstinputlisting[caption=\Optimizing MSVC 2010,style=customasmx86]{OS/calling_conventions/ptr_to_argument/MSVC_2010_O3.asm}

Адрес места в стеке где была передана $a$ просто передается в другую функцию.
Она модифицирует переменную по этому адресу, и затем \printf выведет модифицированное значение.

\par Наблюдательный читатель может спросить, а что насчет тех соглашений о вызовах, где аргументы функции
передаются в регистрах?

Это та самая ситуация, где используется \emph{Shadow Space}.

Так что входящее значение копируется из регистра в \emph{Shadow Space} в локальном стеке и затем это адрес
передается в другую функцию:

\lstinputlisting[caption=\Optimizing MSVC 2012 x64,style=customasmx86]{OS/calling_conventions/ptr_to_argument/MSVC_2012_x64_O3.asm}

GCC также записывает входное значение в локальный стек:

\lstinputlisting[caption=\Optimizing GCC 4.9.1 x64,style=customasmx86]{OS/calling_conventions/ptr_to_argument/GCC491_x64_O3.s}

GCC для ARM64 делает то же самое, но это пространство здесь называется \emph{Register Save Area}:

\lstinputlisting[caption=\Optimizing GCC 4.9.1 ARM64,style=customasmARM]{OS/calling_conventions/ptr_to_argument/GCC49_ARM64_O3.s}

Кстати, похожее использование \emph{Shadow Space} разбирается здесь: \myref{variadic_arith_registers}.

