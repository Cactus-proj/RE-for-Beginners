\chapter{Książki/blogi warte przeczytania}

\mysection{Książki i inne materiały}

\subsection{Inżynieria wsteczna}

\input{RE_books}

(Stara, ale wciąż interesująca) Pavol Cerven, \emph{Crackproof Your Software: Protect Your Software Against Crackers}, (2002).

Oraz książki Krisa Kaspersky'ego.

\subsection{Windows}

\begin{itemize}
\item \Russinovich
\item Peter Ferrie -- The ``Ultimate'' Anti-Debugging Reference\footnote\url{http://pferrie.host22.com/papers/antidebug.pdf}}
\end{itemize}

\EN{Blogs}\ES{Blogs}\RU{Блоги}\FR{Blogs}\DE{Blogs}\PL{Blogi}:

\begin{itemize}
\item \href{http://go.yurichev.com/17025}{Microsoft: Raymond Chen}
\item \href{http://go.yurichev.com/17026}{nynaeve.net}
\end{itemize}



\subsection{\CCpp}

\label{CCppBooks}

\begin{itemize}

\item \KRBook

\item \CNineNineStd\footnote{\AlsoAvailableAs \url{http://www.open-std.org/jtc1/sc22/WG14/www/docs/n1256.pdf}}

\item \TCPPPL

\item \CppOneOneStd\footnote{\AlsoAvailableAs \url{http://www.open-std.org/jtc1/sc22/wg21/docs/papers/2013/n3690.pdf}.}

\item \AgnerFogCPP\footnote{\AlsoAvailableAs \url{http://agner.org/optimize/optimizing_cpp.pdf}.}

\item \ParashiftCPPFAQ\footnote{\AlsoAvailableAs \url{http://www.parashift.com/c++-faq-lite/index.html}}

\item \CNotes\footnote{\AlsoAvailableAs \url{http://yurichev.com/C-book.html}}

\item JPL Institutional Coding Standard for the C Programming Language\footnote{\AlsoAvailableAs \url{https://yurichev.com/mirrors/C/JPL_Coding_Standard_C.pdf}}

\RU{\item Евгений Зуев --- Редкая профессия\footnote{\AlsoAvailableAs \url{https://yurichev.com/mirrors/C++/Redkaya_professiya.pdf}}}

\end{itemize}



\subsection{x86 / x86-64}

\label{x86_manuals}
\begin{itemize}
\item manuale Intela\footnote{\AlsoAvailableAs \url{http://www.intel.com/content/www/us/en/processors/architectures-software-developer-manuals.html}}

\item manuale AMD \footnote{\AlsoAvailableAs \url{http://developer.amd.com/resources/developer-guides-manuals/}}

\item \AgnerFog{}\footnote{\AlsoAvailableAs \url{http://agner.org/optimize/microarchitecture.pdf}}

\item \AgnerFogCC{}\footnote{\AlsoAvailableAs \url{http://www.agner.org/optimize/calling_conventions.pdf}}

\item \IntelOptimization

\item \AMDOptimization
\end{itemize}

Trochę stare, ale wciąż interesujące:

\MAbrash\footnote{\AlsoAvailableAs \url{https://github.com/jagregory/abrash-black-book}}
(znany z pracy nad niskopoziomową optymalizacją w takich projektach jak Windows NT 3.1 i id Quake).

\subsection{ARM}

\begin{itemize}
\item manuale ARM\footnote{\AlsoAvailableAs \url{http://infocenter.arm.com/help/index.jsp?topic=/com.arm.doc.subset.architecture.reference/index.html}}

\item \ARMSevenRef

\item \ARMSixFourRefURL

\item \ARMCookBook\footnote{\AlsoAvailableAs \url{https://yurichev.com/ref/ARM%20Cookbook%20(1994)/}}
\end{itemize}

\subsection{Język maszynowy}

Richard Blum --- Professional Assembly Language.

\subsection{Java}

\JavaBook.

\subsection{UNIX}

\TAOUP

\subsection{Programowanie}

\begin{itemize}

\item \RobPikePractice

\item \HenryWarren.
Niektórzy twierdzą, że sztuczki z tej książki nie mają dzisiaj znaczenia, ponieważ miały zastosowanie wyłącznie w procesorach \ac{RISC},
gdzie instrukcje typu branch są kosztowne.
Niemniej jednak, wszystko to znacząco ułatwia zrozumienie algebry Boole'a i całej matematyki wokół tego.

\end{itemize}

% subsection:
\input{crypto_reading}

