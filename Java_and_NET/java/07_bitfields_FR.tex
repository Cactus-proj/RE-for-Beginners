% TODO proof-reading
\subsection{Champs de bit}

Toutes les opérations au niveau des bits fonctionnent comme sur les autres \ac{ISA}:

\begin{lstlisting}[style=customjava]
	public static int set (int a, int b)
	{
		return a | 1<<b;
	}

	public static int clear (int a, int b)
	{
		return a & (~(1<<b));
	}
\end{lstlisting}

\begin{lstlisting}
  public static int set(int, int);
    flags: ACC_PUBLIC, ACC_STATIC
    Code:
      stack=3, locals=2, args_size=2
         0: iload_0
         1: iconst_1
         2: iload_1
         3: ishl
         4: ior
         5: ireturn

  public static int clear(int, int);
    flags: ACC_PUBLIC, ACC_STATIC
    Code:
      stack=3, locals=2, args_size=2
         0: iload_0
         1: iconst_1
         2: iload_1
         3: ishl
         4: iconst_m1
         5: ixor
         6: iand
         7: ireturn
\end{lstlisting}

\TT{iconst\_m1} charge $-1$ sur la pile, c'est la même chose que le nombre \TT{0xFFFFFFFF}.

XORé avec \TT{0xFFFFFFFF} a le même effet qu'inverser tous les bits (\myref{XOR_property}).

Étendons tous les types de données à 64-bit \emph{long}:

\begin{lstlisting}[style=customjava]
	public static long lset (long a, int b)
	{
		return a | 1<<b;
	}

	public static long lclear (long a, int b)
	{
		return a & (~(1<<b));
	}
\end{lstlisting}

\begin{lstlisting}
  public static long lset(long, int);
    flags: ACC_PUBLIC, ACC_STATIC
    Code:
      stack=4, locals=3, args_size=2
         0: lload_0
         1: iconst_1
         2: iload_2
         3: ishl
         4: i2l
         5: lor
         6: lreturn

  public static long lclear(long, int);
    flags: ACC_PUBLIC, ACC_STATIC
    Code:
      stack=4, locals=3, args_size=2
         0: lload_0
         1: iconst_1
         2: iload_2
         3: ishl
         4: iconst_m1
         5: ixor
         6: i2l
         7: land
         8: lreturn
\end{lstlisting}

Le code est le même, mais des instructions avec le préfixe \emph{l} sont utilisées,
qui opèrent avec des valeurs 64-bit.

Ainsi, le second paramètre de la fonction est toujours du type \emph{int}, et lorsque
la valeur 32-bit qu'il contient doit être étendues à une valeur 64-bit, l'instruction
\TT{i2l} est utilisée, 

