% TODO proof-reading
\subsection{Introduction}

\newcommand{\JADURL}{\url{http://varaneckas.com/jad/}}

Il existe des décompilateurs très connus pour Java (ou pour du bytecode \ac{JVM} en général)
\footnote{Par exemple, JAD: \JADURL}.

La raison est que la décompilation du bytecode \ac{JVM} est un peu plus facile
que du code x86 de plus bas niveau:

\begin{itemize}
\item Il y a bien plus d'informations sur les types de données.
\item Le modèle de la mémoire \ac{JVM} est beaucoup plus rigoureux et décrit.
\item Le compilateur Java ne fait pas d'optimisation (\ac{JVM} \ac{JIT} le fait à l'exécution),
      donc le bytecode dans les fichiers de classe est généralement assez lisible.
      
\end{itemize}

Quand est-ce que la connaissance de la \ac{JVM} est utile ?

\newcommand{\URLListOfJVMLangs}{\url{http://en.wikipedia.org/wiki/List_of_JVM_languages}}

\begin{itemize}
\item Créer des patchs "Quick-and-dirty" des fichiers de classe sans avoir besoin de recompiler les résultats du décompilateur.
\item Analyse de code obfusqué.
\item Analyser du code généré par les nouveaux compilateurs Java, pour lesquels
      il n'existe pas encore de décompilateur mis à jour.
\item Construire votre propre obfuscateur.
\item Construire un compilateur générateur de code (back-end) ciblant la \ac{JVM} (comme Scala, Clojure, etc.
      \footnote{Liste complète: \URLListOfJVMLangs}).
      
\end{itemize}

Commençons avec quelques bouts de code.
Le JDK 1.7 est ici utilisé partout, sauf mention contraire.

Voici la commande utilisée partout pour décompiler les fichiers de classe :\\
\GTT{javap -c -verbose}.

Voici le livre que j'ai utilisé pour préparer tous les exemples : \JavaBook.

