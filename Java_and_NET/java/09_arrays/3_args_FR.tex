% TODO proof-reading
\subsubsection{Le seul argument de la fonction \main est aussi un tableau}

Nous allons utiliser le seul argument de la fonction \main, qui est un tableau de
chaînes:

\begin{lstlisting}[style=customjava]
public class UseArgument
{
	public static void main(String[] args)
	{
		System.out.print("Hi, ");
		System.out.print(args[1]);
		System.out.println(". How are you?");
	}
}
\end{lstlisting}

L'argument d'indice zéro est le nom du programme (comme en \CCpp, etc.), donc le
1er argument fourni par l'utilisateur est à l'indice 1.

\begin{lstlisting}
  public static void main(java.lang.String[]);
    flags: ACC_PUBLIC, ACC_STATIC
    Code:
      stack=3, locals=1, args_size=1
         0: getstatic     #2      // Field java/lang/System.out:Ljava/io/PrintStream;
         3: ldc           #3      // String Hi, 
         5: invokevirtual #4      // Method java/io/PrintStream.print:(Ljava/lang/String;)V
         8: getstatic     #2      // Field java/lang/System.out:Ljava/io/PrintStream;
        11: aload_0       
        12: iconst_1      
        13: aaload        
        14: invokevirtual #4      // Method java/io/PrintStream.print:(Ljava/lang/String;)V
        17: getstatic     #2      // Field java/lang/System.out:Ljava/io/PrintStream;
        20: ldc           #5      // String . How are you?
        22: invokevirtual #6      // Method java/io/PrintStream.println:(Ljava/lang/String;)V
        25: return        
\end{lstlisting}

\TT{aload\_0} en 11 charge une \emph{référence} sur le slot zéro du \ac{LVA}
(1er et unique argument de \main).

\TT{iconst\_1} et \TT{aaload} en 12 et 13 prend une \emph{référence} sur l'élément
1 du tableau (en comptant depuis 0).

La \emph{référence} sur l'objet chaîne est sur le \ac{TOS} à l'offset 14, et elle
est prise d'ici par la méthode \TT{println}.
