% TODO proof-reading
\subsubsection{Premier exemple}

Les chaînes sont des objets et sont construites de la même manière que les autres
objets (et tableaux).


\begin{lstlisting}[style=customjava]
	public static void main(String[] args)
	{
		System.out.println("What is your name?");
		String input = System.console().readLine();
		System.out.println("Hello, "+input);
	}
\end{lstlisting}

\begin{lstlisting}
  public static void main(java.lang.String[]);
    flags: ACC_PUBLIC, ACC_STATIC
    Code:
      stack=3, locals=2, args_size=1
         0: getstatic     #2        // Field java/lang/System.out:Ljava/io/PrintStream;
         3: ldc           #3        // String What is your name?
         5: invokevirtual #4        // Method java/io/PrintStream.println:(Ljava/lang/String;)V
         8: invokestatic  #5        // Method java/lang/System.console:()Ljava/io/Console;
        11: invokevirtual #6        // Method java/io/Console.readLine:()Ljava/lang/String;
        14: astore_1      
        15: getstatic     #2        // Field java/lang/System.out:Ljava/io/PrintStream;
        18: new           #7        // class java/lang/StringBuilder
        21: dup           
        22: invokespecial #8        // Method java/lang/StringBuilder."<init>":()V
        25: ldc           #9        // String Hello, 
        27: invokevirtual #10       // Method java/lang/StringBuilder.append:(Ljava/lang/String;)Ljava/lang/StringBuilder;
        30: aload_1       
        31: invokevirtual #10       // Method java/lang/StringBuilder.append:(Ljava/lang/String;)Ljava/lang/StringBuilder;
        34: invokevirtual #11       // Method java/lang/StringBuilder.toString:()Ljava/lang/String;
        37: invokevirtual #4        // Method java/io/PrintStream.println:(Ljava/lang/String;)V
        40: return        
\end{lstlisting}

La méthode \TT{readLine()} est appelée à l'offset 11, une \emph{référence} sur la
chaîne (qui est fournie par l'utilisateur) est stockée sur le \ac{TOS}.

À l'offset 14, la \emph{référence} sur la chaîne est stockée dans le slot 1 du \ac{LVA}.

La chaîne que l'utilisateur a entré est rechargée à l'offset 30 et concaténée avec
la chaîne \q{Hello, } en utilisant la classe \TT{StringBuilder}.

La chaîne construite est ensuite affichée en utilisant \TT{println} à l'offset
37.

