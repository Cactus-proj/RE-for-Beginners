% TODO proof-reading
\subsection{Modèle de mémoire de la \ac{JVM}}

x86 et d'autres environnements de bas niveau utilisent la pile pour le passage des
paramètres et le stockage de variables locales.

La \ac{JVM} est légèrement différente.

Elle a:

\begin{itemize}
\item Le tableau des variables locales, Local Variable Array (\ac{LVA}).
Utilisé comme stockage pour les paramètres en entrée de fonction et les variables
locales.

Des instructions comme \INS{iload\_0} charge une valeur depuis cet espace.

\INS{istore} stocke des valeurs dedans.
Au début les paramètres de la fonction sont stockés: commençant à 0 ou à 1 (si l'indice
0 est occupé par le pointeur \emph{this}).

Ensuite les variables locales sont allouées.


Chaque slot a une taille de 32-bit.

De ce fait, les valeurs de types de données \emph{long} et \emph{double} occupent
deux slots.


\item Pile des opérandes (ou simplement \q{pile}).
Elle est utilisée pour les calculs et la passage de paramètres lors de l'appel d'autres
fonctions.

Contrairement aux environnements bas niveau comme x86, il n'est pas possible d'accèder
à la pile sans utiliser des instructions qui poussent ou prennent des valeurs dans/depuis
la pile.


\item Heap. Est utilisé pour le stockage d'objets et de tableaux.

\end{itemize}

Ces 3 espaces sont isolés les uns des autres.

