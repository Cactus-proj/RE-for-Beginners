\subsection{Systemy liczbowe}

\epigraph{Nowadays octal numbers seem to be
used for exactly one purpose---file permissions on POSIX systems---but hexadecimal numbers
are widely used to emphasize the bit pattern of a number over its numeric value.}
{Alan A. A. Donovan, Brian W. Kernighan ---  The Go Programming Language}

Ludzie przyzwyczaili się do systemu dziesiętnego prawdopodobnie dlatego, że każdy ma 10 palców.
Natomiast liczba 10 nie odgrywa szczególnej roli w nauce i matematyce.
Binarny (dwójkowy) system liczbowy jest naturalny dla techniki cyfrowej i elektroniki: 0 oznacza brak prądu, a 1 --- jego obecność.
10 w systemie binarnym to 2 w dziesiętnym; 100 w binarnym to 4 w dziesiętnym, itd.

Jeżeli w systemie liczbowym jest 10 znaków, jego \emph{podstawa} (ang. \emph{radix} lub \emph{base}) to 10.
System dwójkowy ma \emph{podstawę} 2.

Ważne rzeczy, które warto sobie przypomnieć:

1) \emph{liczba} jest liczbą, natomiast \emph{cyfra} to umowny znak pisarski służący do zapisywania liczb;
2) sama w sobie liczba się nie zmienia przy przeliczaniu z jednego systemu na inny: zmienia się tylko sposób jej zapisu (lub reprezentacja w pamięci).

Jak skonwertować liczbę z jednego systemu na drugi?

Z notacji pozycyjnej korzysta się prawie wszędzie, to znaczy, że każda cyfra posiada swoją wagę w zależności od jej usytuowania wewnątrz liczby.
Jeżeli 2 znajduje sie na ostatnim miejscu od prawej, jest to 2.
Jeżeli jest ona usytuowana w miejscu przedostatnim, jest to 20.

Co oznacza zapis $1234$?

$10^3 \cdot 1 + 10^2 \cdot 2 + 10^1 \cdot 3 + 1 \cdot 4$ = 1234 lub
$1000 \cdot 1 + 100 \cdot 2 + 10 \cdot 3 + 4 = 1234$

Tak samo to wygląda w przypadku liczb binarnych, tyle że podstawą jest 2, a nie 10.
Co oznacza zapis 0b101011?

$2^5 \cdot 1 + 2^4 \cdot 0 + 2^3 \cdot 1 + 2^2 \cdot 0 + 2^1 \cdot 1 + 2^0 \cdot 1 = 43$ lub
$32 \cdot 1 + 16 \cdot 0 + 8 \cdot 1 + 4 \cdot 0 + 2 \cdot 1 + 1 = 43$

Notację pozycyjną można przeciwstawić notacji niepozycyjnej, np. rzymskiej.
\footnote{O ewolucji systemów liczbowych przeczytasz w \InSqBrackets{\TAOCPvolII{}, 195--213.}}.
Prawdopodobnie ludzkość przeszła na notację pozycyjną, ponieważ w ten sposób łatwiej jest wykonywać proste operacje (dodawanie, mnożenie, itd.) na papierze, ręcznie.

Liczby binarne również można pisemnie dodawać, odejmować itd., dokładnie tak samo jak uczy się tego w szkole, tylko z użyciem dwóch cyfr.

Liczby w zapisie binarnym są nieporęczne, kiedy stosuje się je w kodzie źródłowym i zrzutach pamięci, dlatego w tych miejscach używa się systemu szesnastkowego (heksadecymalnego), o podstawie 16.
System szesnastkowy używa szesnastu cyfr - znaków 0-9 oraz A-F.
Każda cyfra zajmuje 4 bity lub 4 cyfry w systemie binarnym, więc łatwo można konwertować z reprezentacji binarnej na szesnastkową i odwrotnie, nawet w pamięci.

\begin{center}
\begin{longtable}{ | l | l | l | }
\hline
\HeaderColor szesnastkowy & \HeaderColor binarny & \HeaderColor dziesiętny \\
\hline
0	&0000	&0 \\
1	&0001	&1 \\
2	&0010	&2 \\
3	&0011	&3 \\
4	&0100	&4 \\
5	&0101	&5 \\
6	&0110	&6 \\
7	&0111	&7 \\
8	&1000	&8 \\
9	&1001	&9 \\
A	&1010	&10 \\
B	&1011	&11 \\
C	&1100	&12 \\
D	&1101	&13 \\
E	&1110	&14 \\
F	&1111	&15 \\
\hline
\end{longtable}
\end{center}

Jak rozpoznać w jakim systemie zapisana jest konkretna liczba?

Liczby dziesiętne z reguły są zapisywane tradycyjnie, czyli 1234. Ale niektóre asemblery pozwalają podkreślić to przez sufiks "d": 1234d.

Do liczb binarnych czasami dodaje się prefiks "0b": 0b100110111
(W \ac{GCC} istnieje do tego niestandardowe rozszerzenie
\footnote{\url{https://gcc.gnu.org/onlinedocs/gcc/Binary-constants.html}}).
Jest jeszcze inny sposób: sufiks "b", np: 100110111b.
W tej książce będę się trzymał konwencji prefiksowej "0b" dla liczb binarnych.

Liczby szesnastkowe mają prefiks "0x" w \CCpp i niektórych innych \ac{PL}: 0x1234ABCD.
Lub mają sufiks "h": 1234ABCDh --- zwykle są w ten sposób reprezentowane w asemblerach lub debuggerach.
W tej konwencji, jeśli liczba zaczyna się od A..F, przed nimi dopisuje się 0: 0ABCDEFh.
Za czasów 8-bitowych komputerów domowych był również sposób zapisu liczb za pomocą prefiksu \$, np., \$ABCD.
W tej książce będę się trzymał prefiksu "0x" dla liczb szesnastkowych.

Czy trzeba umieć konwertować liczby w głowie? Tablicę liczb szesnastkowych składających się z jednej cyfry łatwo zapamiętać,
ale raczej nie warto zapamiętywać większych liczb.

Prawdopodobnie najczęściej liczby szesnastkowe są spotykane w \ac{URL}-ach.
W ten sposób są kodowane litery spoza alfabetu łacińskiego.
Np.:
\url{https://en.wiktionary.org/wiki/na\%C3\%AFvet\%C3\%A9} to \ac{URL} strony w Wiktionary o słowie \q{naïveté}.

\subsubsection{System ósemkowy}

Jeszcze jeden system, z którego często korzystało się w informatyce w przeszłości to system oktalny. System oktalny ma 8 cyfr (0-7), każda
opisująca 3 bity, więc łatwo przeliczać liczbę na inne systemu, w obie strony.
Ten system prawie wszędzie został zastąpiony przez szesnastkowy, ale, o dziwo, w systemach *NIX nadal jest program korzystający z systemu ósemkowego: \TT{chmod}.

\myindex{UNIX!chmod}
Jak wiedzą użytkownicy systemów *NIX, argumentem \TT{chmod} jest liczba składająca się z 3 cyfr. Pierwsza cyfra określa uprawnienia właściciela pliku,
druga - to uprawnienia grupy (do której plik należy), trzecia dla reszty użytkowników.
Każda cyfra może być przedstawiona binarnie:

\begin{center}
\begin{longtable}{ | l | l | l | }
\hline
\HeaderColor dziesiętny & \HeaderColor binarny & \HeaderColor znaczenie \\
\hline
7	&111	&\textbf{rwx} \\
6	&110	&\textbf{rw-} \\
5	&101	&\textbf{r-x} \\
4	&100	&\textbf{r-{}-} \\
3	&011	&\textbf{-wx} \\
2	&010	&\textbf{-w-} \\
1	&001	&\textbf{-{}-x} \\
0	&000	&\textbf{-{}-{}-} \\
\hline
\end{longtable}
\end{center}

Więc każdy bit jest powiązany z flagą: read/write/execute (prawo do odczytu/zapisu/wykonania).

Właśnie dlatego wspomniałem o \TT{chmod} ~--- liczba, będąca argumentem, może być reprezentowana w systemie ósemkowym.
Na przykład weźmy 644.
Kiedy uruchamiasz \TT{chmod 644 file}, ustawiasz uprawnienia read/write (odczyt zapis) dla właściciela, uprawnienia read (zapis) dla grupy i read dla wszystkich innych użytkowników.
Jeśli skonwertujemy liczbę 644 z systemu ósemkowego na binarny, to otrzymamy \TT{110100100}, lub (w grupach po 3 bity) \TT{110 100 100}.

Teraz widzimy, że każda 'trójka' opisuje uprawnienia dla właściciela/grupy/reszty: pierwsza \TT{rw-}, druga to \TT{r--} i trzecia to \TT{r--}.

System ósemkowy był również popularny na starych komputerach, jak PDP-8, dlatego że słowo (podstawowa porcja informacji) mogło składać się 12, 24 lub
36 bitów, a wszystkie te liczby są podzielne przez 3, więc wybór systemu ósemkowego był całkiem logiczny.
Obecnie wszystkie popularne komputery mają słowa/adresy 16-, 32- lub 64-bitowe i wszystkie te liczby są podzielne przez 4,
więc system szesnastkowy jest wygodniejszy.

System ósemkowy jest wspierany przez wszystkie standardowe kompilatory \CCpp{}.
Czasami jest to źródłem nieporozumień, dlatego że liczby ósemkowe są kodowane z zerem z przodu, na przykład: 0377 to 255.
Gdy pomylisz się i napiszesz "09" zamiast 9, to kompilator zgłosi błąd.
GCC może zwrócić podobny komunikat :\\
\TT{error: invalid digit "9" in octal constant}.

System ósemkowy jest również popularny w Javie. Gdy \IDA wyświetla string ze znakami niedrukowalnymi, są one zakodowane w systemie ósemkowym (a nie szesnastkowym)
\myindex{JAD}.
Dekompilator JAD zachowuje się w taki sam sposób.

\subsubsection{Podzielność}

Kiedy widzisz liczbę 120, to można szybko się zorientować, że jest ona podzielna przez 10, dlatego że ostatnią cyfrą jest 0.
Podobnie, 123400 jest podzielne przez 100, bo ostatnie dwie cyfry są zerami.

Analogicznie liczba szesnastkowa 0x1230 jest podzielna przez 0x10 (16 w systemie dziesiętnym), 0x123000 jest podzielne przez 0x1000 (4096 w systemie dziesiętnym), itd.

Liczba binarna 0b1000101000 jest podzielna przez 0b1000 (8), itd.

Tę właściwość można wykorzystać, żeby szybko sprawdzić
czy jakiś adres lub rozmiar bloku jest wyrównany do pewnej granicy (czy rozmiar bloku jest całkowitą krotnością długości słowa, np. krotnością 32 bitów).
Na przykład sekcje w plikach \ac{PE} prawie zawsze zaczynają się od adresów kończących się trzema szesnastkowymi zerami: 0x41000, 0x10001000, itd., gdyż prawie wszystkie sekcje w plikach \ac{PE} są wyrównane do granicy 0x1000 (4096) bajtów.

\subsubsection{Arytmetyka wielokrotnej precyzji a podstawa}

\myindex{RSA}
Arytmetyka wielokrotnej precyzji (multi-precision arithmetic) może operować na dowolnie dużych liczbach,
które mogą być przechowywane w kilku bajtach.
Na przykład klucze RSA, zarówno prywatne jak i publiczne, mogą zajmować 4096 bitów a nawet więcej.

W \InSqBrackets{\TAOCPvolII, 265} przedstawiono ideę: kiedy przechowuje się liczbę wielokrotnej precyzji w kilku bajtach,
cała liczba może być reprezentowana jako zapisana w systemie liczbowym o podstawie $2^8=256$, i każda cyfra reprezentuje jeden bajt.
Podobnie, gdybyś liczbę wielokrotnej precyzji przechowywał w kilku 32-bitowych całkowitoliczbowych wartościach,
każda cyfra byłaby reprezentowana przez 32-bitowy slot (4 bajty), i można by uważać tę liczbę za zapisaną w systemie o podstawie $2^{32}$.

\subsubsection{Wymowa liczb w systemach niedziesiętnych}

Liczby zapisane w systemie o podstawie innej niż 10, zwykle wymawia się cyfra po cyfrze: ``jeden-zero-zero-jeden-jeden-...``. Nie używa się słów jak ``dziesięć`` czy ``tysiąc`` by przez pomyłkę nie potraktowano liczby jak zapisanej w systemie dziesiętnym.

\subsubsection{Liczby zmiennoprzecinkowe}

Żeby odróżniać liczby zmiennoprzecinkowe od całkowitoliczbowych, często na końcu dodaje się ``.0'',
np. $0.0$, $123.0$, itd.


