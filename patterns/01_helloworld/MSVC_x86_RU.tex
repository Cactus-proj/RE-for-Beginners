\subsubsection{MSVC}

Компилируем в MSVC 2010:

\begin{lstlisting}
cl 1.cpp /Fa1.asm
\end{lstlisting}

(Ключ \TT{/Fa} означает сгенерировать листинг на ассемблере)

\begin{lstlisting}[caption=MSVC 2010,style=customasmx86]
CONST	SEGMENT
$SG3830	DB	'hello, world', 0AH, 00H
CONST	ENDS
PUBLIC	_main
EXTRN	_printf:PROC
; Function compile flags: /Odtp
_TEXT	SEGMENT
_main	PROC
	push	ebp
	mov	ebp, esp
	push	OFFSET $SG3830
	call	_printf
	add	esp, 4
	xor	eax, eax
	pop	ebp
	ret	0
_main	ENDP
_TEXT	ENDS
\end{lstlisting}

MSVC выдает листинги в синтаксисе Intel.
Разница между синтаксисом Intel и AT\&T будет рассмотрена немного позже:

Компилятор сгенерировал файл \TT{1.obj}, который впоследствии будет слинкован линкером в \TT{1.exe}.
В нашем случае этот файл состоит из двух сегментов: \TT{CONST} (для данных-констант) и \TT{\_TEXT} (для кода).

\myindex{\CLanguageElements!const}
\label{string_is_const_char}
Строка \TT{hello, world} в \CCpp имеет тип \TT{const char[]}\InSqBrackets{\TCPPPL p176, 7.3.2}, однако не имеет имени.
Но компилятору нужно как-то с ней работать, поэтому он дает ей внутреннее имя \TT{\$SG3830}.

Поэтому пример можно было бы переписать вот так:

\lstinputlisting[style=customc]{patterns/01_helloworld/hw_2.c}

Вернемся к листингу на ассемблере. Как видно, строка заканчивается нулевым байтом~--- это требования стандарта \CCpp для строк.
Больше о строках в \CCpp: \myref{C_strings}.

В сегменте кода \TT{\_TEXT} находится пока только одна функция: \main{}.
Функция \main, как и практически все функции, начинается с пролога и заканчивается эпилогом
\footnote{Об этом смотрите подробнее в разделе о прологе и эпилоге функции ~(\myref{sec:prologepilog}).}.

\myindex{x86!\Instructions!CALL}
Далее следует вызов функции \printf{}: \INS{CALL \_printf}.
\myindex{x86!\Instructions!PUSH}
Перед этим вызовом адрес строки (или указатель на неё) с нашим приветствием (``Hello, world!'') при помощи инструкции \PUSH помещается в стек.

После того, как функция \printf возвращает управление в функцию \main, адрес строки (или указатель на неё) всё ещё лежит в стеке.
Так как он больше не нужен, то \glslink{stack pointer}{указатель стека} (регистр \ESP) корректируется.

\myindex{x86!\Instructions!ADD}
\INS{ADD ESP, 4} означает прибавить 4 к значению в регистре \ESP.

Почему 4? Так как это 32-битный код, для передачи адреса нужно 4 байта. В x64-коде это 8 байт.\\
\INS{ADD ESP, 4} эквивалентно \TT{POP регистр}, но без использования какого-либо регистра\footnote{Флаги процессора, впрочем, модифицируются}.

\myindex{Intel C++}
\myindex{\oracle}
\myindex{x86!\Instructions!POP}

Некоторые компиляторы, например, Intel C++ Compiler, в этой же ситуации могут вместо 
\ADD сгенерировать \TT{POP ECX} (подобное можно встретить, например, в коде \oracle{}, им скомпилированном),
что почти то же самое, только портится значение в регистре \ECX.
Возможно, компилятор применяет \TT{POP ECX}, потому что эта инструкция короче (1 байт у \TT{POP} против 3 у \TT{ADD}).

Вот пример использования \POP вместо \ADD из \oracle{}:

\begin{lstlisting}[caption=\oracle 10.2 Linux (файл app.o),style=customasmx86]
.text:0800029A                 push    ebx
.text:0800029B                 call    qksfroChild
.text:080002A0                 pop     ecx
\end{lstlisting}

Впрочем, MSVC был замечен в подобном же.

\begin{lstlisting}[caption=MineSweeper из Windows 7 32-bit]
.text:0102106F                 push    0
.text:01021071                 call    ds:time
.text:01021077                 pop     ecx
\end{lstlisting}

%О стеке можно прочитать в соответствующем разделе
% ~(\myref{sec:stack}).
\myindex{\CLanguageElements!return}
После вызова \printf в оригинальном коде на \CCpp указано \TT{return 0}~--- вернуть 0 в качестве результата функции \main.

\myindex{x86!\Instructions!XOR}
В сгенерированном коде это обеспечивается инструкцией \\
\INS{XOR EAX, EAX}.

\myindex{x86!\Instructions!MOV}

\XOR, как легко догадаться~--- \q{исключающее ИЛИ}\footnote{\href{http://go.yurichev.com/17118}{wikipedia}}, но компиляторы часто используют его вместо простого
\INS{MOV EAX, 0} --- снова потому, что опкод короче (2 байта у \XOR против 5 у \MOV).

\myindex{x86!\Instructions!SUB}
Некоторые компиляторы генерируют \INS{SUB EAX, EAX}, что значит \emph{отнять значение в} \EAX \emph{от значения в }\EAX, что в любом случае даст 0 в результате.

\myindex{x86!\Instructions!RET}
Самая последняя инструкция \RET возвращает управление в вызывающую функцию. Обычно это код \CCpp \ac{CRT}, который, в свою очередь, вернёт управление в \ac{OS}.

