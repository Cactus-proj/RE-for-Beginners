\subsubsection{GCC}

Skompilujmy teraz ten sam kod za pomocą kompilatora GCC 4.4.1 na systemie Linux: \TT{gcc 1.c -o 1}.
Następnie za pomocą deasemblera \IDA podejrzymy wynik kompilacji funkcji \main.
\IDA, jak i MSVC, pokazują kod w składni Intela\footnote{Można zmusić również GCC do generowania listingów w tym formacie, za pomocą opcji \TT{-S -masm=intel}.}.

\lstinputlisting[caption=Kod w programie \IDA,style=customasmx86]{patterns/01_helloworld/IDA_x86.asm}

\myindex{Function prologue}
\myindex{x86!\Instructions!AND}
Wynik jest prawie taki sam.
Adres łańcucha znaków \TT{hello, world}, leżącego w segmencie danych, najpierw zapisywany jest do \EAX, a później odkładany na stos.
Dodatkowo, w prologu funkcji widzimy \TT{AND ESP, 0FFFFFFF0h}~---
ta instrukcja wyrównuje \ESP do granicy 16 bajtów.
Dzięki temu wszystkie wartości na stosie będą również wyrównane w taki sam sposób (procesor pracuje efektywniej z adresami wyrównanymi do granicy 4 lub 16 bajtów.).

\myindex{x86!\Instructions!SUB}
\INS{SUB ESP, 10h} alokuje na stosie 16 bajtów. Choć tutaj wystarczyłyby 4 bajty, co będzie widoczne dalej.

Dzieje się tak dlatego, że ilość przydzielanego miejsca na stosie jest również wyrównywana do granicy 16 bajtów.

% TODO1: rewrite.
\myindex{x86!\Instructions!PUSH}
Adres łańcucha znaków (lub wskaźnik na niego) jest zatem odkładany prosto na stos, bez wykorzystywania instrukcji \PUSH.
\emph{var\_10} jest jednocześnie zmienną lokalną i argumentem dla funkcji \printf{}.
Więcej na ten temat dowiesz się później.

Następnie jest wywoływana funkcja \printf.

W odróżnieniu od MSVC, GCC przy kompilacji z wyłączoną optymalizacją generuje \TT{MOV EAX, 0} zamiast krótszego kod operacji (opcode).

\myindex{x86!\Instructions!LEAVE}
Ostatnia instrukcja \LEAVE~--- jest analogiczna do pary instrukcji  \TT{MOV ESP, EBP} i \TT{POP EBP}~--- jest to powrót \glslink{stack pointer}{wskaźnika stosu} i rejestru \EBP do stanu początkowego.
Jest to niezbędne, ponieważ na początku funkcji modyfikowaliśmy rejestry \ESP i \EBP\\
(za pomocą \INS{MOV EBP, ESP} / \INS{AND ESP, \ldots}).

\subsubsection{GCC: \ATTSyntax}
\label{ATT_syntax}

Sprawdźmy jak będzie wyglądał listing w składni AT\&T, która jest bardziej popularna świecie UNIXa.

\begin{lstlisting}[caption=Kompilujemy za pomocą GCC 4.7.3]
gcc -S 1_1.c
\end{lstlisting}

Otrzymamy taki plik:

\lstinputlisting[caption=GCC 4.7.3,style=customasmx86]{patterns/01_helloworld/GCC.s}

Listing zawiera wiele makr (rozpoczynających się od kropki). Na razie nie są one dla nas istotne.

Zignorujmy je wszystkie, (za wyjątkiem \emph{.string}, które koduje sekwencję znaków zakończonych znakiem null ~--- tak jak łańcuchy znaków w \Cpp). Otrzymamy wtedy:
\footnote{Eliminację zbędnych makr można uzyskać za pomocą opcji GCC: \emph{-fno-asynchronous-unwind-tables}}:

\lstinputlisting[caption=GCC 4.7.3,style=customasmx86]{patterns/01_helloworld/GCC_refined.s}

\myindex{\ATTSyntax}
\myindex{\IntelSyntax}
Główne różnice między składnią Intela a AT\&T :

\begin{itemize}

\item
Operandy są zapisywane w odwotnej kolejności

W składni Intela: \\
<instrukcja> <operand docelowy> <operand źródłowy>.

W składni AT\&T: \\
<instrukcja> <operand źródłowy> <operand docelowy>.

\myindex{\CStandardLibrary!memcpy()}
\myindex{\CStandardLibrary!strcpy()}
Istnieje łatwy sposób na zapamiętanie tej różnicy:
kiedy pracujecie ze składnią Intela~--- możecie w głowie postawić znak równości ($=$) między operandami,
a z AT\&T~--- strzałkę w prawo ($\rightarrow$)
\footnote{W niektórych standardowych funkcjach biblioteki C (memcpy(), strcpy(), ...) również korzysta się z kolejności argumentów jak w składni Intela: najpierw wskaźnik na miejsce docelowe w pamięci,
następnie wskaźnik na miejsce źródłowe.}.

\item
AT\&T: Przed nazwami rejestrów stawia się symbol (\%), a przed liczbami (\$).
Zamiast nawiasów kwadratowych używa się okrągłych.

\item
AT\&T: Do każdej instrukcji dodaje się przyrostek określający typ danych:

\begin{itemize}
\item q --- quad (64 bity)
\item l --- long (32 bity)
\item w --- word (16 bitów)
\item b --- byte (8 bitów)
\end{itemize}


\end{itemize}

Wracając do wyniku kompilacji: jest on niemal identyczny do tego, który prezentuje \IDA.
Jedna drobnostka: \TT{0FFFFFFF0h} jest zapisywane jako \TT{\$-16}.
Oba zapisy oznaczją dokładnie to samo: \TT{16} w sytemie dziesiętnym to \TT{0x10} w szesnastkowym.
\TT{-0x10} to \TT{0xFFFFFFF0} (dla liczb całkowitych 32-bitowych) \footnote{W kodowaniu U2 - \url{https://pl.wikipedia.org/wiki/Kod_uzupe%C5%82nie%C5%84_do_dw%C3%B3ch} i kolejności bajtów big endian - \url{https://pl.wikipedia.org/wiki/Kolejno%C5%9B%C4%87_bajt%C3%B3w}}.

\myindex{x86!\Instructions!MOV}
Zwracana wartość jest ustawiany na 0 za pomocą zwykłej instrukcji \MOV, a nie \XOR.
\MOV zapisuje wartość do rejestru.
Nazwa tej instrukcji nie jest do końca poprawna (wartości nie są przemieszczane, tylko kopiowane). W innych architekturach instrukcja ta nosi nazwę \q{LOAD}, \q{STORE} lub podobną.

