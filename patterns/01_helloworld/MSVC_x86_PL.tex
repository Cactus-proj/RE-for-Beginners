\subsubsection{MSVC}

Skompilujmy kod w MSVC 2010:

\begin{lstlisting}
cl 1.cpp /Fa1.asm
\end{lstlisting}

(Opcja \TT{/Fa} oznacza wygenerowanie listingu w asemblerze)

\begin{lstlisting}[caption=MSVC 2010,style=customasmx86]
CONST	SEGMENT
$SG3830	DB	'hello, world', 0AH, 00H
CONST	ENDS
PUBLIC	_main
EXTRN	_printf:PROC
; Function compile flags: /Odtp
_TEXT	SEGMENT
_main	PROC
	push	ebp
	mov	ebp, esp
	push	OFFSET $SG3830
	call	_printf
	add	esp, 4
	xor	eax, eax
	pop	ebp
	ret	0
_main	ENDP
_TEXT	ENDS
\end{lstlisting}

MSVC generuje listingi w składni Intela.
Różnica między składnią Intela a AT\&T będzie omówiona w in \myref{ATT_syntax}.

Kompilator wygenerował plik \TT{1.obj}, który następnie będzie połączony konsolidatorem (ang. linker) w \TT{1.exe}.
W naszym przypadku, ten plik składa się z dwóch segmentów: \TT{CONST} (dane-stałe) i \TT{\_TEXT} (kod).

\myindex{\CLanguageElements!const}
\label{string_is_const_char}
Łańcuch znaków \TT{hello, world} w \CCpp ma typ \TT{const char[]}\InSqBrackets{\TCPPPL p176, 7.3.2}, jednak nie posiada nazwy.
Kompilator potrzebuje nazwy, żeby z tym łańcuchem znaków pracować, dlatego nadaje mu własną nazwę -  \TT{\$SG3830}.

Dlatego ten przykład można by zapisać w ten sposób:

\lstinputlisting[style=customc]{patterns/01_helloworld/hw_2.c}

Wróćmy do kodu w asemblerze. Jak widać, łańcuch znaków kończy się bajtem zerowym~--- jest to element standardu \CCpp o łańcuchach znaków.
Więcej o łańcuchach znaków w \CCpp: \myref{C_strings}.

W segmencie kodu \TT{\_TEXT} znajduje się na razie tylko jedna funkcja: \main{} i, jak prawie każda funkcja, zaczyna się od prologu a kończy się epilogiem
\footnote{Więcej o prologu i epilogu przeczytasz w sekcji ~(\myref{sec:prologepilog}).}.

\myindex{x86!\Instructions!CALL}
Po prologu następuje wywołanie funkcji \printf{}:\\
\INS{CALL \_printf}.
\myindex{x86!\Instructions!PUSH}
Przed tym wywołaniem adres łańcucha znaków (lub wskaźnik na niego) z naszym pozdrowieniem (``Hello, world!'') odkładany jest na stos, za pomocą instrukcji \PUSH.

Po tym jak funkcja \printf zwraca sterowanie do funkcji \main, adres łańcucha znaków (lub wskaźnik na niego) wciąż jest na stosie.
Nie jest on dalej potrzebny, więc \glslink{stack pointer}{wskaźnik wierzchołka stosu} (rejestr \ESP) musi zostać poprawiony.

\myindex{x86!\Instructions!ADD}
\INS{ADD ESP, 4} oznacza: dodaj wartość 4 do rejestru \ESP.

Dlaczego 4? Z racji tego, że jest to kod 32-bitowy, do przekazania adresu za pomocą stosu potrzebowaliśmy 4 bajtów. W x64 potrzebowalibyśmy 8 bajtów.\\
\INS{ADD ESP, 4} jest równoważne instrukcji \INS{POP register}, ale nie wykorzystuje dodatkowego rejestru\footnote{Ale za to modyfikowany jest rejestr stanu}. W pierwszym przypadku jedynie bezpośrednio manipulujemy na rejestrze ESP (wskaźniku wierzchołka stosu), a w drugim przypadku zdejmujemy ze stosu jeden element i odkładamy go do rejestru \emph{register} a rejestr ESP zmienia się automatycznie.

\myindex{Intel C++}
\myindex{\oracle}
\myindex{x86!\Instructions!POP}

Niektóre kompilatory, jak Intel C++ Compiler, w tej samej sytuacji mogą wygenerować \TT{POP ECX} zamiast \ADD (można to zaobserwować w kodzie \oracle{}, gdyż jest on kompilowany właśnie tym kompilatorem).
\TT{POP ECX} oznacza zdjęcie elementu ze stosu i umieszczenie go w rejestrze \ECX.
Osiągnęliśmy taki sam efekt jak w przypadku instrukcji \ADD (zmiana wskaźnika stosu), ale skutkiem ubocznym jest nadpisanie \ECX.

Możliwe, że kompilator stosuje tu \TT{POP ECX} dlatego, że ta instrukcja jest krótsza (1 bajt w przypadku \TT{POP} kontra 3 bajty w przypadku \TT{ADD}).

Poniżej przykład wykorzystania \POP zamiast \ADD z \oracle{}:

\begin{lstlisting}[caption=\oracle 10.2 Linux (plik app.o),style=customasmx86]
.text:0800029A                 push    ebx
.text:0800029B                 call    qksfroChild
.text:080002A0                 pop     ecx
\end{lstlisting}

MSVC czasami zachowuje się tak samo.

\begin{lstlisting}[caption=Saper na systemie Windows 7 32-bit]
.text:0102106F                 push    0
.text:01021071                 call    ds:time
.text:01021077                 pop     ecx
\end{lstlisting}

%Więcej informacji o stosie można znaleźć w odpowiednim rozdziale
% ~(\myref{sec:stack}).
\myindex{\CLanguageElements!return}
Po wywołaniu \printf kod w \CCpp zawiera instrukcję \TT{return 0}~--- zwróć 0 jako wynik funcji \main.

\myindex{x86!\Instructions!XOR}
W kodzie wygenerowanym robi to instrukcja \INS{XOR EAX, EAX}.

\myindex{x86!\Instructions!MOV}

\XOR, jak można się domyśleć, to ~--- \q{alternatywa wykluczająca}\footnote{\href{http://en.wikipedia.org/wiki/Exclusive_or}{Wikipedia}}, ale kompilatory często korzystają z niej zamiast z
\INS{MOV EAX, 0} --- znów dlatego, że kod operacji (opcode) jest krótszy (2 bajty w \XOR kontra 5 w \MOV).

\myindex{x86!\Instructions!SUB}
Niektóre kompilatory generują \INS{SUB EAX, EAX}, co oznacza \emph{odejmij wartość w} \EAX \emph{od wartości w }\EAX, co w każdym przypadku da 0.

\myindex{x86!\Instructions!RET}
Ostatnia instrukcja \RET zwraca sterowanie do funkcji wywołującej. Zwykle jest to kod \CCpp \ac{CRT}, który z kolei zwróci sterowanie do systemu operacyjnego.

