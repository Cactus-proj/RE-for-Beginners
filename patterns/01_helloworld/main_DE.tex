\mysection{\HelloWorldSectionName}
\label{sec:helloworld}

Beginnen wir mit dem berühmten Beispiel aus dem Buch [\KRBook]:

\lstinputlisting[style=customc]{patterns/01_helloworld/hw.c}

\subsection{x86}

\input{patterns/01_helloworld/MSVC_x86}
\input{patterns/01_helloworld/GCC_x86}
\subsubsection{String-Patching (Win32)}

Man kann die Zeichenkette ''hello, world'' in der ausführbaren Datei mit Hiew finden:

\begin{figure}[H]
\centering
\myincludegraphics{patterns/01_helloworld/hola_edit1.png}
\caption{Hiew}
\label{}
\end{figure}

Man kann jetzt versuchen die Meldung ins Spanische zu übersetzen:

\begin{figure}[H]
\centering
\myincludegraphics{patterns/01_helloworld/hola_edit2.png}
\caption{Hiew}
\label{}
\end{figure}

Die spanische Version ist ein Byte kürzer als die englische, als muss am Ende ein 0x0A-Byte (\TT{\textbackslash{}n}) und ein Null-Byte eingefügt werden.

Es funktioniert.

Was wenn eine längere Nachricht eingefügt werden soll?
Hinter dem originalen englischen Text befinden sich einige Nullbytes.
Es ist schwierig zu sagen, ob diese überschrieben werden dürfen: es ist möglich, dass die zum Beispiel in dem \ac{CRT}-Code genutz werden. Vielleicht aber auch nicht.
Wie dem auch sei: diese Daten sollten nur überschrieben werden, wenn wirklich klar ist was man tut.

\subsubsection{String-Patching (Linux x64)}

\myindex{\radare}
Nachfolgend wird der Patch einer ausführbaren Datei unter einem 64 Bit-Linux mit \radare{} gezeigt:

\lstinputlisting[caption=\radare{} session]{patterns/01_helloworld/radare.lst}

Was hier passiert ist folgendes: suchen von \q{hello} mit dem \TT{/}-Kommando,
dann Setzen des \emph{cursor} (oder \emph{seek} im \radare{}-Wording) an diese Adresse.
Um sicher zu gehen, dass die richtige Stelle gesetzt ist, kann mit \TT{px} der Datenblock ausgegeben werden.
\TT{oo+} versetzt \radare{} in den \emph{Lese-Schreibe}-Modus.
\TT{w} schreibt einen ASCII string an die aktuelle Adresse.
Hinweis: \TT{\textbackslash{}00} am Ende ist das Null-Byte.
\TT{q} beendet \radare{}.

% TBT
%\subsubsection{This is a real story of software cracking}
%\myindex{\SoftwareCracking}
%
%An image processing software, when not registered, added watermarks,
%like ``This image was processed by evaluation version of [software name]'', across a picture.
%We tried at random: we found that string in the executable file and put spaces instead of it.
%Watermarks disappeared.
%Technically speaking, they continued to appear.
%\myindex{Qt}
%With the help of Qt functions, the watermark was still added to the resulting image.
%But adding spaces didn't alter the image itself...

\subsubsection{Software-\emph{Lokalisation} zu MS-DOS-Zeiten}

Der hier beschriebene Weg war in den 1980ern und  1990ern sehr vebreitet, um MS-DOS-Progamme in die russische Sprache zu übersetzen.
%TBT
%This technique is available even for those who are not aware of machine code and executable file formats.
%The new string has not to be bigger than the old one, because there's a risk of overwriting another value or code
%there.
Russische Wörter und Sätze sind in der Regel etwas länger als ihre englischen Gegenstücke, was der Grund ist, dass
viele \emph{lokalisierte} Programme eine Menge seltsamer Akronyme und Abkürzungen haben.

\begin{figure}[H]
\centering
\includegraphics[width=0.5\textwidth]{patterns/01_helloworld/Norton_Commander_v5_51.png}
\caption{\DEph{}}
\end{figure}

Möglicherweise passierte dies in der Zeit auch in anderen Sprachen.

\myindex{Borland Delphi}
In Delphi müssen die Längen der Zeichenketten falls nötig korrigiert werden.


\subsection{x86-64}
\input{patterns/01_helloworld/MSVC_x64}
\input{patterns/01_helloworld/GCC_x64}
\subsubsection{Adress-Patching (Win64)}

Wenn das Beispiel in MSCV2013 mit der Option \TT{/MD} kompiliert wird
(was zu einer kleineren ausfühbaren Datei durch das linken mit \TT{MSVCR*.DLL} führt),
kommt zuerst die \main-Funktion und kann einfach gefunden werden:

\begin{figure}[H]
\centering
\myincludegraphics{patterns/01_helloworld/hiew_incr1.png}
\caption{Hiew}
\label{}
\end{figure}

Als Experiment kann die Adresse des Pointers um 1 \gls{increment} werden:

\begin{figure}[H]
\centering
\myincludegraphics{patterns/01_helloworld/hiew_incr2.png}
\caption{Hiew}
\label{}
\end{figure}

Hiew zeigt \q{ello, world} als Zeichenkette und beim Ausführen der gepatchten
Datei wird eben dieser Text ausgegeben.

\subsubsection{Aussuchen einer anderen Zeichenkette einer Binärdatei (Linux x64)}

Die Binärdatei die beim Kompilieren des Beispiels mit GCC 5.4.0 unter Linux x64
entsteht, beinhaltet noch viele andere Zeichenketten: die meisten sind importierte
Funktions- und Bibliotheksnamen.

Mit objdump können die Inhalte aller Sektionen der kompilierten Datei ausgegeben werden:

\begin{lstlisting}[basicstyle=\ttfamily, mathescape]
$\$$ objdump -s a.out

a.out:     file format elf64-x86-64

Contents of section .interp:
 400238 2f6c6962 36342f6c 642d6c69 6e75782d  /lib64/ld-linux-
 400248 7838362d 36342e73 6f2e3200           x86-64.so.2.
Contents of section .note.ABI-tag:
 400254 04000000 10000000 01000000 474e5500  ............GNU.
 400264 00000000 02000000 06000000 20000000  ............ ...
Contents of section .note.gnu.build-id:
 400274 04000000 14000000 03000000 474e5500  ............GNU.
 400284 fe461178 5bb710b4 bbf2aca8 5ec1ec10  .F.x[.......^...
 400294 cf3f7ae4                             .?z.

...
\end{lstlisting}

Es ist kein Problem die Adresse der Zeichenkette \q{/lib64/ld-linux-x86-64.so.2}
an \TT{printf()} zu übergeben:

\begin{lstlisting}[style=customc]
#include <stdio.h>

int main()
{
    printf(0x400238);
    return 0;
}
\end{lstlisting}

Schwer zu glauben, aber dieser Code gibt die erwähnte Zeichenkette aus.

Beim Ändern der Adresse zu \TT{0x400260} wird die Zeichenkette \q{GNU} ausgegeben.
Diese Adresse gilt für die hier verwendete GCC-Version, Toolkonfiguration und so weiter.
Auf anderen Systemen kann die ausfühbare Datei leicht unterschiedlich sein, was auch
die Adressen verändern kann.
Auch das Hinzufügen und Entfernen von Quellcode kann Adressen vor- und zurückschieben.


\EN{\mysection{Returning Values: redux}

Again, when we know about function prologue and epilogue, let's recompile an example returning a value
(\ref{ret_val_func}, \ref{lst:ret_val_func}) using non-optimizing GCC:

\lstinputlisting[caption=\NonOptimizing GCC 8.2 x64 (\assemblyOutput),style=customasmx86]{patterns/017_ret_redux/1.s}

Effective instructions here are \INS{MOV} and \INS{RET}, others are -- prologue and epilogue.

}

\EN{\mysection{Returning Values: redux}

Again, when we know about function prologue and epilogue, let's recompile an example returning a value
(\ref{ret_val_func}, \ref{lst:ret_val_func}) using non-optimizing GCC:

\lstinputlisting[caption=\NonOptimizing GCC 8.2 x64 (\assemblyOutput),style=customasmx86]{patterns/017_ret_redux/1.s}

Effective instructions here are \INS{MOV} and \INS{RET}, others are -- prologue and epilogue.

}


\subsection{\Conclusion{}}

Der Hauptunterschied zwischen x86/ARM and x64/ARM64-Code ist das der Zeiger auf den String 64 Bit lang ist.
Moderne \ac{CPU}s haben eine 64-Bit-Architektur um Speicherkosten zu reduzieren und den höheren Bedarf
aktueller Anwendungen erfüllen zu können.
Es ist möglich sehr viel mehr Speicher in dem Computer zu verwenden als 32-Bit-Zeiger adressieren können.
Aus diesem Grund sind alle Zeiger 64 Bit lang.

% sections
\input{patterns/01_helloworld/exercises}

