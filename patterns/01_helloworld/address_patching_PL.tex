\subsubsection{Łatanie (patching) adresów (Win64)}

Jeśli nasz przykład skompilujemy za pomocą MSVC 2013 z opcją \TT{/MD}
(dynamiczne linkowanie, mniejszy plik wykonywalny dzięki zewnętrznym odwoływaniom do bibliotek \TT{MSVCR*.DLL}),
funkcja \main będzie łatwa do znalezienia, gdyż wystąpi jako pierwsza:

\begin{figure}[H]
\centering
\myincludegraphics{patterns/01_helloworld/hiew_incr1.png}
\caption{Hiew}
\label{}
\end{figure}

Możemy spróbować \glslink{increment}{inkrementować} adres:

\begin{figure}[H]
\centering
\myincludegraphics{patterns/01_helloworld/hiew_incr2.png}
\caption{Hiew}
\label{}
\end{figure}

Hiew wyświetla teraz \q{ello, world} (obok instrukcji `\LEA \RCX ...`, która ładuje adres łańucha znaków, przekazywany jako argument do funkcji \printf).
Kiedy uruchomimy plik wykonywalny, właśnie ten ciąg znaków zostnie wypisany na ekran.

\subsubsection{Wyświetlanie różnych ciągów znaków z pliku wykonywalnego (Linux x64)}

Plik wykonywalny po skompilowaniu za pomocą GCC 5.4.0 na systemie Linux x64, ma wiele innych łańcuchów znaków:
głównie są to nazwy importowanych funkcjii oraz nazwy bibliotek.

Uruchamiamy objdump, żeby podejrzeć zawartość wszystkich sekcji skompilowanego pliku:

\begin{lstlisting}
% objdump -s a.out

a.out:     file format elf64-x86-64

Contents of section .interp:
 400238 2f6c6962 36342f6c 642d6c69 6e75782d  /lib64/ld-linux-
 400248 7838362d 36342e73 6f2e3200           x86-64.so.2.
Contents of section .note.ABI-tag:
 400254 04000000 10000000 01000000 474e5500  ............GNU.
 400264 00000000 02000000 06000000 20000000  ............ ...
Contents of section .note.gnu.build-id:
 400274 04000000 14000000 03000000 474e5500  ............GNU.
 400284 fe461178 5bb710b4 bbf2aca8 5ec1ec10  .F.x[.......^...
 400294 cf3f7ae4                             .?z.

...
\end{lstlisting}

Łatwo przekazać adres łańcucha znaków \q{/lib64/ld-linux-x86-64.so.2} do funkcji \TT{printf()}:

\begin{lstlisting}[style=customc]
#include <stdio.h>

int main()
{
    printf(0x400238);
    return 0;
}
\end{lstlisting}

Trudno uwierzyć, ale program wyświetli ten łańcuch znaków na ekran.

Jeśli zmienimy adres na \TT{0x400260}, to wyświetli się napis \q{GNU}.
Adres jest prawidłowy dla konkretnej wersji GCC, GNU toolset, etc.
W waszym systemie plik wykonywalny może wyglądać trochę inaczej i wszystkie adresy także będą inne.
Podobnie, usuwanie lub dodawanie kodu do kodu źródłowego może przesunąć wszystkie adresy w programie wykonywalnym do przodu lub do tyłu.


