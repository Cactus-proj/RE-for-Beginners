\mysection{\HelloWorldSectionName}
\label{sec:helloworld}

[\KRBook]という本の有名な例を使ってみましょう

\lstinputlisting[caption=\CCpp Code,style=customc]{patterns/01_helloworld/hw.c}

\subsection{x86}

\input{patterns/01_helloworld/MSVC_x86}
\input{patterns/01_helloworld/GCC_x86}
\subsubsection{文字列のパッチ(Win32)}

Hiewを使用して、実行可能ファイル内の ``hello、world'' 文字列を簡単に見つけることができます:

\begin{figure}[H]
\centering
\myincludegraphics{patterns/01_helloworld/hola_edit1.png}
\caption{Hiew}
\label{}
\end{figure}

メッセージをスペイン語に翻訳しようとすることができます

\begin{figure}[H]
\centering
\myincludegraphics{patterns/01_helloworld/hola_edit2.png}
\caption{Hiew}
\label{}
\end{figure}

スペイン語のテキストは英語より1バイト短くなっているので、最後に0x0Aバイト(\TT{\textbackslash{}n})に続けてNULLバイトを追加しました。

うまくいきました。

より長いメッセージを挿入する場合はどうすればよいですか?
元の英語テキストの後には、ゼロバイトがいくつかあります。 上書きできるかどうかはなんとも言えません:\ac{CRT}コードのどこかで使われるかもしれないし、そうでないかもしれません。 
とにかく、自分が行っていることを本当に知っていれば、それらを上書きするだけです。

\subsubsection{文字列のパッチ(Linux x64)}

\myindex{\radare}
\radare{}を使ってLinux x64実行ファイルにパッチを当ててみましょう

\lstinputlisting[caption=\radare{} session]{patterns/01_helloworld/radare.lst}

ここでは何が起こっているの:私は\TT{/}コマンドを使用して\q{hello}文字列を検索し、
そのアドレスに\emph{カーソル}(\radare{}用語で\emph{シーク})を設定します。 
次に、これが本当にその場所であることを確かめたい:\TT{px}がダンプします。
\TT{oo+}は\radare{}を\emph{読み書き}モードに切り替えます。 
\TT{w}は現在の\emph{シーク}時にASCII文字列を書き込みます。
最後の\TT{\textbackslash{}00}に注意してください。これはNULLバイトです。
\TT{q}で終了します。

% TBT
%\subsubsection{This is a real story of software cracking}
%\myindex{\SoftwareCracking}
%
%An image processing software, when not registered, added watermarks,
%like ``This image was processed by evaluation version of [software name]'', across a picture.
%We tried at random: we found that string in the executable file and put spaces instead of it.
%Watermarks disappeared.
%Technically speaking, they continued to appear.
%\myindex{Qt}
%With the help of Qt functions, the watermark was still added to the resulting image.
%But adding spaces didn't alter the image itself...

\subsubsection{MS-DOS時代におけるソフトウェアの\emph{ローカライズ}}

このやり方は、1980年代と1990年代にMS-DOSソフトウェアをロシア語に翻訳する一般的な方法でした。 ロシア語の言葉や文章は、英語の文章と比べて通常若干長いので、\emph{ローカライズ}されたソフトウェアには奇妙な頭字語やとても読みにくい略語が含まれています。

\begin{figure}[H]
\centering
\includegraphics[width=0.5\textwidth]{patterns/01_helloworld/Norton_Commander_v5_51.png}
\caption{\JAph{}}
\end{figure}

他の国の他の言語でも、この時代に起きていたことでしょう。


\subsection{x86-64}
\input{patterns/01_helloworld/MSVC_x64}
\input{patterns/01_helloworld/GCC_x64}
\subsubsection{アドレスのパッチ(Win64)}

この例が\TT{/MD}スイッチ(\TT{MSVCR*.DLL}ファイルリンケージのために小さな実行可能ファイルを意味します)を使用してMSVC 2013でコンパイルされた場合、 \main 関数が最初に来て簡単に見つかります

\begin{figure}[H]
\centering
\myincludegraphics{patterns/01_helloworld/hiew_incr1.png}
\caption{Hiew}
\label{}
\end{figure}

実験として、アドレスを1ずつ\gls{increment}することができます:

\begin{figure}[H]
\centering
\myincludegraphics{patterns/01_helloworld/hiew_incr2.png}
\caption{Hiew}
\label{}
\end{figure}

Hiewは \q{ello, world}を示しています。 そしてパッチが適用された実行可能ファイルを実行すると、この文字列が表示されます。

\subsubsection{バイナリイメージから他の文字列を抜き取る(Linux x64)}

Linux x64ボックスでGCC 5.4.0を使用して私たちの例をコンパイルしたときに得たバイナリファイルには、他の多くのテキスト文字列があります。
ほとんどはインポートされた関数名とライブラリ名です。

objdumpを実行して、コンパイル済みファイルのすべてのセクションの内容を取得します。

\begin{lstlisting}[basicstyle=\ttfamily, mathescape]
$\$$ objdump -s a.out

a.out:     file format elf64-x86-64

Contents of section .interp:
 400238 2f6c6962 36342f6c 642d6c69 6e75782d  /lib64/ld-linux-
 400248 7838362d 36342e73 6f2e3200           x86-64.so.2.
Contents of section .note.ABI-tag:
 400254 04000000 10000000 01000000 474e5500  ............GNU.
 400264 00000000 02000000 06000000 20000000  ............ ...
Contents of section .note.gnu.build-id:
 400274 04000000 14000000 03000000 474e5500  ............GNU.
 400284 fe461178 5bb710b4 bbf2aca8 5ec1ec10  .F.x[.......^...
 400294 cf3f7ae4                             .?z.

...
\end{lstlisting}

テキスト文字列\q{/lib64/ld-linux-x86-64.so.2}のアドレスを\TT{printf()}に渡すのは問題ではありません

\begin{lstlisting}[style=customc]
#include <stdio.h>

int main()
{
    printf(0x400238);
    return 0;
}
\end{lstlisting}

信じがたいですが、このコードは前述の文字列を表示します。

アドレスを\TT{0x400260}に変更すると、 \q{GNU}文字列が出力されます。 
このアドレスは、私の特定のGCCバージョン、GNUツールセットなどに当てはまります。
あなたのシステムでは、実行ファイルは若干異なる場合があり、すべてのアドレスも異なります。 
また、このソースコードに/からコードを追加/削除すると、おそらくすべてのアドレスが前後に移動します。


\EN{\mysection{Returning Values: redux}

Again, when we know about function prologue and epilogue, let's recompile an example returning a value
(\ref{ret_val_func}, \ref{lst:ret_val_func}) using non-optimizing GCC:

\lstinputlisting[caption=\NonOptimizing GCC 8.2 x64 (\assemblyOutput),style=customasmx86]{patterns/017_ret_redux/1.s}

Effective instructions here are \INS{MOV} and \INS{RET}, others are -- prologue and epilogue.

}

\EN{\mysection{Returning Values: redux}

Again, when we know about function prologue and epilogue, let's recompile an example returning a value
(\ref{ret_val_func}, \ref{lst:ret_val_func}) using non-optimizing GCC:

\lstinputlisting[caption=\NonOptimizing GCC 8.2 x64 (\assemblyOutput),style=customasmx86]{patterns/017_ret_redux/1.s}

Effective instructions here are \INS{MOV} and \INS{RET}, others are -- prologue and epilogue.

}


\subsection{\Conclusion{}}

x86/ARMとx64/ARM64コードの主な違いは、文字列へのポインタが64ビット長になったことです。
確かに、現代の \ac{CPU} は64ビットになりました。これは、現代のアプリケーションではメモリの節約と大きな需要の両方があるからです。
私たちは32ビットポインタよりもはるかに多くのメモリをコンピュータに追加することができます。
そのため、すべてのポインタは64ビットになりました。

% sections
\input{patterns/01_helloworld/exercises}
