\mysection{\HelloWorldSectionName}
\label{sec:helloworld}

Utilizziamo il famoso esempio dal libro [\KRBook]:

\lstinputlisting[caption=\CCpp Code,style=customc]{patterns/01_helloworld/hw.c}

\subsection{x86}

\input{patterns/01_helloworld/MSVC_x86}
\input{patterns/01_helloworld/GCC_x86}
\subsubsection{String patching (Win32)}

Possiamo facilmente trovare la stringa ``hello, world'' all'interno del file eseguibile utilizzando Hiew:

\begin{figure}[H]
\centering
\myincludegraphics{patterns/01_helloworld/hola_edit1.png}
\caption{Hiew}
\label{}
\end{figure}

E possiamo cercare di tradurre il messaggio in spagnolo:

\begin{figure}[H]
\centering
\myincludegraphics{patterns/01_helloworld/hola_edit2.png}
\caption{Hiew}
\label{}
\end{figure}

Il testo in spagnolo è più corto di un byte rispetto a quello inglese, quindi abbiamo aggiunto anche il byte 0x0A al fondo (\TT{\textbackslash{}n}) con un byte zero.

Funziona.

E se volessimo inserire un messaggio più lungo?
Ci sono alcuni byte a zero dopo il testo in inglese.
E' difficile stabilire se possono essere sovrascritti: potrebbero essere utilizzati da qualche parte all'interno del codice \ac{CRT}, oppure no.
Ad ogni modo, sovrascrivili solo se sai esattamente cosa stai facendo.

\subsubsection{String patching (Linux x64)}

\myindex{\radare}
Proviamo a modificare un eseguibile Linux x64 utilizzando \radare{}:

\lstinputlisting[caption=\radare{} session]{patterns/01_helloworld/radare.lst}

Questo è il procedimento: ho cercato la stringa \q{hello} utilizzando il comando \TT{/},
poi ho impostato il \emph{cursore} (\emph{seek}, in \radare{}) a quell'indirizzo.
Poi voglio essere sicuro di essere veramente nel posto giusto: \TT{px} mostra un dump dei dati locali.
\TT{oo+} imposta \radare{} in modalità \emph{read-write}.
\TT{w} scrive una stringa ASCII nel \emph{seek} corrente.
Nota il \TT{\textbackslash{}00} al fondo---è un byte zero.
\TT{q} esce (quit).

% TBT
%\subsubsection{This is a real story of software cracking}
%\myindex{\SoftwareCracking}
%
%An image processing software, when not registered, added watermarks,
%like ``This image was processed by evaluation version of [software name]'', across a picture.
%We tried at random: we found that string in the executable file and put spaces instead of it.
%Watermarks disappeared.
%Technically speaking, they continued to appear.
%\myindex{Qt}
%With the help of Qt functions, the watermark was still added to the resulting image.
%But adding spaces didn't alter the image itself...

\subsubsection{La \emph{traduzione} del software all'epoca del MS-DOS}

Questo era un metodo comune per tradurre i software per MS-DOS durante gli anni '80 e '90.
%TBT
%This technique is available even for those who are not aware of machine code and executable file formats.
%The new string has not to be bigger than the old one, because there's a risk of overwriting another value or code
%there.
A volte le parole e le frasi sono leggermente più lunghe rispetto ai corrispettivi in inglese, per questo motivo i software \emph{adattati}
hanno molti acronimi strani ed abbreviazioni difficili da comprendere.

\begin{figure}[H]
\centering
\includegraphics[width=0.5\textwidth]{patterns/01_helloworld/Norton_Commander_v5_51.png}
\caption{\ITph{}}
\end{figure}

Probabilmente questo è successo in molti Paesi durante quel periodo.


\subsection{x86-64}
\input{patterns/01_helloworld/MSVC_x64}
\input{patterns/01_helloworld/GCC_x64}
\subsubsection{Address patching (Win64)}

Se il nostro esempio venisse compilato in MSVC 2013 utilizzando lo switch \TT{/MD}
(che significa un eseguibile più piccolo a causa del link dei file \TT{MSVCR*.DLL}), la funzione \main verrebbe prima, e può essere trovata facilmente:

\begin{figure}[H]
\centering
\myincludegraphics{patterns/01_helloworld/hiew_incr1.png}
\caption{Hiew}
\label{}
\end{figure}

Come esperimento, possiamo \glslink{increment}{incrementare} l'indirizzo di 1:

\begin{figure}[H]
\centering
\myincludegraphics{patterns/01_helloworld/hiew_incr2.png}
\caption{Hiew}
\label{}
\end{figure}

Hiew mostra \q{ello, world}.
E quando lanciamo l'eseguibile modificato, viene stampata proprio questa stringa.

\subsubsection{Scegliere un'altra stringa dall'immagine binaria (Linux x64)}

Il file binario che si ottiene compilando il nostro esempio tramite GCC 5.4.0 su Linux x64 contiene molte altre stringhe di testo.
Si tratta principalmente di nomi di funzioni e librerie importate.

Esegui objdump per ottenere il contenuto di tutte le sezioni del file compilato:

\begin{lstlisting}[basicstyle=\ttfamily, mathescape]
$\$$ objdump -s a.out

a.out:     file format elf64-x86-64

Contents of section .interp:
 400238 2f6c6962 36342f6c 642d6c69 6e75782d  /lib64/ld-linux-
 400248 7838362d 36342e73 6f2e3200           x86-64.so.2.
Contents of section .note.ABI-tag:
 400254 04000000 10000000 01000000 474e5500  ............GNU.
 400264 00000000 02000000 06000000 20000000  ............ ...
Contents of section .note.gnu.build-id:
 400274 04000000 14000000 03000000 474e5500  ............GNU.
 400284 fe461178 5bb710b4 bbf2aca8 5ec1ec10  .F.x[.......^...
 400294 cf3f7ae4                             .?z.

...
\end{lstlisting}

Non è un problema passare l'indirizzo della stringa di test \q{/lib64/ld-linux-x86-64.so.2} a \TT{printf()}:

\begin{lstlisting}[style=customc]
#include <stdio.h>

int main()
{
    printf(0x400238);
    return 0;
}
\end{lstlisting}

E' difficile da credere, ma questo codice stampa la stringa citata prima.

Se cambiassi l'indirizzo a \TT{0x400260}, verrebbe stampata la stringa \q{GNU}.
Questo indirizzo è corretto per la mia specifica versione di GCC, GNU toolset, etc.
Sul tuo sistema, l'eseguibile potrebbe essere leggermente differente, e anche tutti gli indirizzi sarebbero differenti.
Inoltre, aggiungendo o rimuovendo del codice in/da questo codice sorgente probabilmente sposterebbe tutti gli indirizzi in avanti o indietro.


\EN{\mysection{Returning Values: redux}

Again, when we know about function prologue and epilogue, let's recompile an example returning a value
(\ref{ret_val_func}, \ref{lst:ret_val_func}) using non-optimizing GCC:

\lstinputlisting[caption=\NonOptimizing GCC 8.2 x64 (\assemblyOutput),style=customasmx86]{patterns/017_ret_redux/1.s}

Effective instructions here are \INS{MOV} and \INS{RET}, others are -- prologue and epilogue.

}

\EN{\mysection{Returning Values: redux}

Again, when we know about function prologue and epilogue, let's recompile an example returning a value
(\ref{ret_val_func}, \ref{lst:ret_val_func}) using non-optimizing GCC:

\lstinputlisting[caption=\NonOptimizing GCC 8.2 x64 (\assemblyOutput),style=customasmx86]{patterns/017_ret_redux/1.s}

Effective instructions here are \INS{MOV} and \INS{RET}, others are -- prologue and epilogue.

}


\subsection{\Conclusion{}}

La differenza principale tra il codice x86/ARM e x64/ARM64 è che il puntatore alla stringa è adesso lungo 64 bit.
Infatti, le moderne \ac{CPU} sono ora a 64-bit grazie ai costi ridotti della memoria e alla sua grande richiesta da parte delle applicazioni moderne.
Possiamo aggiungere ai nostri computer più memoria di quanto i puntatori a 32-bit siano in grado di indirizzare.
Di conseguenza, tutti i puntatori sono adesso a 64-bit.

% sections
\input{patterns/01_helloworld/exercises}
