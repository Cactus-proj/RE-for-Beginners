\mysection{\HelloWorldSectionName}
\label{sec:helloworld}

Skorzystajmy ze sławnego przykładu z książki [\KRBook]:

\lstinputlisting[caption=Kod w \CCpp,style=customc]{patterns/01_helloworld/hw.c}

\subsection{x86}

\input{patterns/01_helloworld/MSVC_x86}
\input{patterns/01_helloworld/GCC_x86}
\subsubsection{Korekcja (patching) łańcuchów znaków (Win32)}

Możemy w łatwy sposób znaleźć linię ``hello, world'' w pliku wykonywalnym za pomocą Hiew:

\begin{figure}[H]
\centering
\myincludegraphics{patterns/01_helloworld/hola_edit1.png}
\caption{Hiew}
\label{}
\end{figure}

Możemy przetłumaczyć naszą wiadomość na język hiszpański:

\begin{figure}[H]
\centering
\myincludegraphics{patterns/01_helloworld/hola_edit2.png}
\caption{Hiew}
\label{}
\end{figure}

Tekst w języku hiszpańskim jest o 1 bajt krótszy od tekstu w języku angielskim, dlatego dodajemy na koniec bajt 0x0A (\TT{\textbackslash{}n}) i bajt zerowy.

Działa.

A co jeśli chcielibyśmy wstawić dłuższy tekst?
Po oryginalnym tekście w języku angielskim widzimy kilka bajtów zerowych.
Trudno powiedzieć czy można je nadpisać: mogą (ale nie muszą!) one być wykorzystywane gdzieś w kodzie \ac{CRT}.
Tak czy inaczej, możemy je nadpisywać tylko jeśli naprawdę wiemy co robimy.

\subsubsection{Korekcja łańcuchów znaków (Linux x64)}

\myindex{\radare}
Spróbujmy edytować plik wykonywalny systemu Linux x64, korzystając z \radare{}:

\lstinputlisting[caption=Sesja w \radare{}]{patterns/01_helloworld/radare.lst}

Co tu  się dzieje: szukam łańcucha znaków \q{hello}, korzystając z komendy \TT{/},
następnie ustawiam \emph{kursor} (\emph{seek} w terminologii \radare{}) pod ten adres.
Następnie chcę się upewnić, że jest to rzeczywiście poszukiwane miejsce: \TT{px} wyświetla bajty pod tym adresem.
\TT{oo+} przełącza \radare{} w tryb \emph{odczytu/zapis}.
\TT{w} zapisuje łańuch znaków ASCII w miejscu kursora (\emph{seek}).
Warto zauważyć \TT{\textbackslash{}00} na końcu, jest to bajt zerowy.
\TT{q} wyłącza \radare.


\subsubsection{Prawdziwa historia crackowania oprogramowania}
\myindex{\SoftwareCracking}

Program do przetwarzania obrazów, niezarejestrowany, dodawał do pliku znak wodny,
na przykład napis \q{This image was processed by evaluation version of [nazwa oprogramowania]}
Spróbowaliśmy najprostszego rozwiązania: znaleźliśmy ten tekst w pliku wykonywalnym i zastąpiliśmy go spacjami
Znak wodny zniknął.
Ogólnie rzecz biorąc, wciąż był nakładany przez program.
\myindex{Qt}
Za pomocą funkcji Qt, znak wodny wciąż był dodawany do obrazu.
Ale dodawanie spacji nie zmieniało go w żaden sposób...

\subsubsection{Tłumaczenie oprogramowania za czasów MS-DOS}
\myindex{MS-DOS}

Sposób przedstawiony wyżej był powszechnie wykorzystywany w latach 80. i 90. przy tłumaczeniu oprogramowania pod MS-DOS na język rosyjski.
Ta technika może być wykorzystywana przy braku wiedzy na temat kodu maszynowego i formatów plików wykonywalnych.
Nowy łańcuch znaków nie powinien być dłuższy niż stary, ponieważ istnieje ryzyko nadpisania innej wartości albo fragmentu kodu wykonywalnego

Rosyjskie słowa i zdania zwykle są trochę dłuższe od angielskich odpowiedników, dlatego \emph{przetłumaczone} oprogramowanie zawierało
sporo dziwnych akronimów (skrótowców) i trudnych do zrozumienia skrótów.

\begin{figure}[H]
\centering
\includegraphics[width=0.5\textwidth]{patterns/01_helloworld/Norton_Commander_v5_51.png}
\caption{Norton Commander 5.51 przetłumaczony na język rosyjski}
\end{figure}

Prawdopodobnie sytuacja wyglądała podobnie z tłumaczeniem na inne języki.

\myindex{Borland Delphi}
W przypadku łańcuch znaków w Delphi, rozmiar również musi być poprawiony, jeśli zachodzi taka potrzeba.





\subsection{x86-64}
\input{patterns/01_helloworld/MSVC_x64}
\input{patterns/01_helloworld/GCC_x64}
\subsubsection{Łatanie (patching) adresów (Win64)}

Jeśli nasz przykład skompilujemy za pomocą MSVC 2013 z opcją \TT{/MD}
(dynamiczne linkowanie, mniejszy plik wykonywalny dzięki zewnętrznym odwoływaniom do bibliotek \TT{MSVCR*.DLL}),
funkcja \main będzie łatwo do znalezienia, gdyż wystąpi jako pierwsza:

\begin{figure}[H]
\centering
\myincludegraphics{patterns/01_helloworld/hiew_incr1.png}
\caption{Hiew}
\label{}
\end{figure}

Możemy spróbować \glslink{increment}{inkrementować} adres o 1:

\begin{figure}[H]
\centering
\myincludegraphics{patterns/01_helloworld/hiew_incr2.png}
\caption{Hiew}
\label{}
\end{figure}

Hiew wyświetla teraz \q{ello, world} (obok instrukcji `\LEA \RCX ...`, która ładuje adres łańucha znaków, przekazywany jako argument do funkcji \printf).
Kiedy uruchomimy plik wykonywalny, właśnie ten ciąg znaków zostnie wypisany na ekran.

\subsubsection{Wyświetlanie różnych ciągów znaków z pliku wykonywalnego (Linux x64)}

Plik wykonywalny po skompilowaniu za pomocą GCC 5.4.0 na systemie Linux x64, ma wiele innych łańcuchów znaków:
głównie są to nazwy importowanych funkcjii oraz nazwy bibliotek.

Uruchamiamy objdump, żeby podejrzeć zawartość wszystkich sekcji skompilowanego pliku:

\begin{lstlisting}
% objdump -s a.out

a.out:     file format elf64-x86-64

Contents of section .interp:
 400238 2f6c6962 36342f6c 642d6c69 6e75782d  /lib64/ld-linux-
 400248 7838362d 36342e73 6f2e3200           x86-64.so.2.
Contents of section .note.ABI-tag:
 400254 04000000 10000000 01000000 474e5500  ............GNU.
 400264 00000000 02000000 06000000 20000000  ............ ...
Contents of section .note.gnu.build-id:
 400274 04000000 14000000 03000000 474e5500  ............GNU.
 400284 fe461178 5bb710b4 bbf2aca8 5ec1ec10  .F.x[.......^...
 400294 cf3f7ae4                             .?z.

...
\end{lstlisting}

Łatwo przekazać adres łańcucha znaków \q{/lib64/ld-linux-x86-64.so.2} do funkcji \TT{printf()}:

\begin{lstlisting}[style=customc]
#include <stdio.h>

int main()
{
    printf(0x400238);
    return 0;
}
\end{lstlisting}

Trudno uwierzyć, ale program wyświetli ten łańcuch znaków na ekran.

Jeśli zmienimy adres na \TT{0x400260}, to wyświetli się napis \q{GNU}.
Adres jest prawidłowy dla konkretnej wersji GCC, GNU toolset, etc.
W waszym systemie plik wykonywalny może wyglądać trochę inaczej i wszystkie adresy także będą inne.
Podobnie, usuwanie lub dodawanie kodu do kodu źródłowego może przesunąć wszystkie adresy w programie wykonywalnym do przodu lub do tyłu.




\EN{\mysection{Returning Values: redux}

Again, when we know about function prologue and epilogue, let's recompile an example returning a value
(\ref{ret_val_func}, \ref{lst:ret_val_func}) using non-optimizing GCC:

\lstinputlisting[caption=\NonOptimizing GCC 8.2 x64 (\assemblyOutput),style=customasmx86]{patterns/017_ret_redux/1.s}

Effective instructions here are \INS{MOV} and \INS{RET}, others are -- prologue and epilogue.

}

\EN{\mysection{Returning Values: redux}

Again, when we know about function prologue and epilogue, let's recompile an example returning a value
(\ref{ret_val_func}, \ref{lst:ret_val_func}) using non-optimizing GCC:

\lstinputlisting[caption=\NonOptimizing GCC 8.2 x64 (\assemblyOutput),style=customasmx86]{patterns/017_ret_redux/1.s}

Effective instructions here are \INS{MOV} and \INS{RET}, others are -- prologue and epilogue.

}


\subsection{\Conclusion{}}

Największa różnica między kodem w x86/ARM a x64/ARM64 polega na tym, że wskaźnik na linię stał się 64-bitowy.
Właśnie, współczesne \ac{CPU} stały się 64-bitowe, dlatego że pamięć stała się znacznie tańsza,
komputery mogą zawierać jej o wiele więcej niż wcześniej i 32-bit już jest za mało, żeby ją zaadresować.

% sections
\input{patterns/01_helloworld/exercises}


