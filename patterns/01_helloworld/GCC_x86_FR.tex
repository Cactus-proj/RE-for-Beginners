\subsubsection{GCC}

Maintenant compilons le même code \CCpp avec le compilateur GCC 4.4.1 sur Linux: \TT{gcc 1.c -o 1}.
Ensuite, avec l'aide du désassembleur \IDA, regardons comment la fonction \main a été créée.
\IDA, comme MSVC, utilise la syntaxe Intel\footnote{GCC peut aussi produire un listing assembleur utilisant la syntaxe Intel en lui passant les options \TT{-S -masm=intel}.}.

\lstinputlisting[caption=code in \IDA,style=customasmx86]{patterns/01_helloworld/IDA_x86.asm}

\myindex{Prologue de fonction}
\myindex{x86!\Instructions!AND}
Le résultat est presque le même.
L'adresse de la chaîne \TT{hello, world} (stockée dans le segment de donnée) est d'abord chargée dans
 le registre \EAX puis sauvée sur la pile.\\
En plus, le prologue de la fonction comprend \INS{AND ESP, 0FFFFFFF0h}~---cette
instruction aligne le registre \ESP sur une limite de 16-octet.
Ainsi, toutes les valeurs sur la pile seront alignées de la même manière (Le CPU
est plus performant si les adresses avec lesquelles il travaille en mémoire sont
alignées sur des limites de 4-octet ou 16-octet).

\myindex{x86!\Instructions!SUB}
\INS{SUB ESP, 10h} réserve 16 octets sur la pile. Pourtant, comme nous allons le voir, seuls 4 sont nécessaires ici.

C'est parce que la taille de la pile allouée est alignée sur une limite de 16-octet.

% TODO1: rewrite.
\myindex{x86!\Instructions!PUSH}
L'adresse de la chaîne est (ou un pointeur vers la chaîne) est stockée directement sur la pile sans utiliser
l'instruction \PUSH.
\emph{var\_10}~---est une variable locale et est aussi un argument pour \printf{}.
Lisez à ce propos en dessous.

Ensuite la fonction \printf est appelée.

Contrairement à MSVC, lorsque GCC compile sans optimisation, il génère \TT{MOV EAX, 0} au lieu d'un opcode plus court.

\myindex{x86!\Instructions!LEAVE}
La dernière instruction, \LEAVE~---est équivalente à la paire d'instruction \TT{MOV ESP, EBP} et \TT{POP EBP}~---en d'autres mots,
cette instruction déplace le \glslink{stack pointer}{pointeur de pile} (\ESP) et remet le registre \EBP dans son état initial.
Ceci est nécessaire puisque nous avons modifié les valeurs de ces registres (\ESP et \EBP) au début de la fonction (en exécutant \INS{MOV EBP, ESP} / \INS{AND ESP, \ldots}).

\subsubsection{GCC: \ATTSyntax}
\label{ATT_syntax}

Voyons comment cela peut-être représenté en langage d'assemblage avec la syntaxe AT\&T.
Cette syntaxe est bien plus populaire dans le monde UNIX.

\begin{lstlisting}[caption=compilons avec GCC 4.7.3]
gcc -S 1_1.c
\end{lstlisting}

Nous obtenons ceci:

\lstinputlisting[caption=GCC 4.7.3,style=customasmx86]{patterns/01_helloworld/GCC.s}

Le listing contient beaucoup de macros (qui commencent avec un point). Cela ne nous intéresse pas pour le moment.

Pour l'instant, dans un souci de simplification, nous pouvons les ignorer (excepté la macro \emph{.string}
qui encode une séquence de caractères terminée par un octet nul, comme une chaîne C).
Ensuite nous verrons cela\footnote{Cette option de GCC peut être utilisée pour éliminer les macros \q{non nécessaires}:
\emph{-fno-asynchronous-unwind-tables}}:

\lstinputlisting[caption=GCC 4.7.3,style=customasmx86]{patterns/01_helloworld/GCC_refined.s}

\myindex{\ATTSyntax}
\myindex{\IntelSyntax}
Quelques-une des différences majeures entre la syntaxe Intel et AT\&T sont:

\begin{itemize}

\item
Opérandes source et destination sont écrites dans l'ordre inverse.

En syntaxe Intel: <instruction> <opérande de destination> <opérande source>.

En syntaxe AT\&T: <instruction> <opérande source> <opérande de destination>.

\myindex{\CStandardLibrary!memcpy()}
\myindex{\CStandardLibrary!strcpy()}
Voici un moyen simple de mémoriser la différence:
lorsque vous avez affaire avec la syntaxe Intel, vous pouvez imaginer qu'il y a un signe
égal ($=$) entre les opérandes et lorsque vous avez affaire avec la syntaxe AT\&T imaginez
qu'il y a un flèche droite ($\rightarrow$)
\footnote{À propos, dans certaine fonction C standard (e.g., memcpy(), strcpy()) les arguments
sont listés de la même manière que dans la syntaxe Intel: en premier se trouve le pointeur du
bloc mémoire de destination, et ensuite le pointeur sur le bloc mémoire source.}.

\item
AT\&T: Avant les noms de registres, un signe pourcent doit être écrit (\%) et avant les nombres, un signe dollar (\$).
Les parenthèses sont utilisées à la place des crochets.

\item
AT\&T: un suffixe est ajouté à l'instruction pour définir la taille de l'opérande:

\begin{itemize}
\item q --- quad (64 bits)
\item l --- long (32 bits)
\item w --- word (16 bits)
\item b --- byte (8 bits)
\end{itemize}

% TODO1 simple example may be? \RU{Например mov\textbf{l}, movb, movw представляют различые версии инсструкция mov} \EN {For example: movl, movb, movw are variations of the mov instruction}
% \FR {Par exemple: movl, movb, movw sont des variantes de l'instruction mov}

\end{itemize}

Retournons au résultat compilé: il est identique à ce que l'on voit dans \IDA.
Avec une différence subtile: \TT{0FFFFFFF0h} est représenté avec \TT{\$-16}.
C'est la même chose: \TT{16} dans le système décimal est \TT{0x10} en hexadécimal.
\TT{-0x10} est équivalent à \TT{0xFFFFFFF0} (pour un type de donnée sur 32-bit).

\myindex{x86!\Instructions!MOV}
Encore une chose: la valeur de retour est mise à 0 en utilisant un \MOV usuel, pas un \XOR.
\MOV charge seulement la valeur dans le registre.
Le nom est mal choisi (la donnée n'est pas déplacée, mais plutôt copiée). Dans d'autres architectures, cette instruction
est nommée \q{LOAD} ou \q{STORE} ou quelque chose de similaire.

