\mysection{\HelloWorldSectionName}
\label{sec:helloworld}

Продолжим, используя знаменитый пример из книги [\KRBook]:

\lstinputlisting[caption=код на Си/Си++,style=customc]{patterns/01_helloworld/hw.c}

\subsection{x86}

\input{patterns/01_helloworld/MSVC_x86}
\input{patterns/01_helloworld/GCC_x86}
\subsubsection{Коррекция (патчинг) строки (Win32)}

Мы можем легко найти строку ``hello, world'' в исполняемом файле при помощи Hiew:

\begin{figure}[H]
\centering
\myincludegraphics{patterns/01_helloworld/hola_edit1.png}
\caption{Hiew}
\label{}
\end{figure}

Можем перевести наше сообщение на испанский язык:

\begin{figure}[H]
\centering
\myincludegraphics{patterns/01_helloworld/hola_edit2.png}
\caption{Hiew}
\label{}
\end{figure}

Испанский текст на 1 байт короче английского, так что добавляем в конце байт 0x0A (\TT{\textbackslash{}n}) и нулевой байт.

Работает.

Что если мы хотим вставить более длинное сообщение?
После оригинального текста на английском есть какие-то нулевые байты.
Трудно сказать, можно ли их перезаписывать: они могут где-то использоваться в \ac{CRT}-коде, а может и нет.
Так или иначе, вы можете их перезаписывать, только если вы действительно знаете, что делаете.

\subsubsection{Коррекция строки (Linux x64)}

\myindex{\radare}
Попробуем пропатчить исполняемый файл для Linux x64 используя \radare{}:

\lstinputlisting[caption=Сессия в \radare{}]{patterns/01_helloworld/radare.lst}

Что я здесь делаю: ищу строку \q{hello} используя команду \TT{/}, 
я затем я выставляю \emph{курсор} (\emph{seek} в терминах \radare{}) на этот адрес.
Потом я хочу удостовериться, что это действительно нужное место: \TT{px} выводит байты по этому адресу.
\TT{oo+} переключает \radare{} в режим \emph{чтения-записи}.
\TT{w} записывает ASCII-строку на месте курсора (\emph{seek}).
Нужно отметить \TT{\textbackslash{}00} в конце --- это нулевой байт.
\TT{q} заканчивает работу.

\subsubsection{Это реальная история взлома ПО}
\myindex{\SoftwareCracking}

Некое ПО обрабатывало изображения, и когда не было зарегистрированно, оно добавляло водяные знаки,
вроде ``This image was processed by evaluation version of [software name]'', поперек картинки.
Мы попробовали от балды: нашли эту строку в исполняемом файле и забили пробелами.
Водяные знаки пропали.
Технически, они продолжали добавляться.
\myindex{Qt}
При помощи соответствующих ф-ций Qt, надпись продолжала добавляться в итоговое изображение.
Но добавление пробелов не меняло само изображение...

\subsubsection{Локализация ПО во времена MS-DOS}

Описанный способ был очень распространен для перевода ПО под MS-DOS на русский язык в 1980-е и 1990-е.
Русские слова и предложения обычно немного длиннее английских, так что \emph{локализованное} ПО содержало
массу странных акронимов и труднопонятных сокращений.

\begin{figure}[H]
\centering
\includegraphics[width=0.5\textwidth]{patterns/01_helloworld/Norton_Commander_v5_51.png}
\caption{Русифицированный Norton Commander 5.51}
\end{figure}

Вероятно, так было и с другими языками в других странах.



\subsection{x86-64}
\input{patterns/01_helloworld/MSVC_x64}
\input{patterns/01_helloworld/GCC_x64}
\subsubsection{Коррекция (патчинг) адреса (Win64)}

Если наш пример скомпилирован в MSVC 2013 используя опцию \TT{/MD}
(подразумевая меньший исполняемый файл из-за внешнего связывания файла \TT{MSVCR*.DLL}),
ф-ция \main идет первой, и её легко найти:

\begin{figure}[H]
\centering
\myincludegraphics{patterns/01_helloworld/hiew_incr1.png}
\caption{Hiew}
\label{}
\end{figure}

В качестве эксперимента, мы можем \glslink{increment}{инкрементировать} адрес на 1:

\begin{figure}[H]
\centering
\myincludegraphics{patterns/01_helloworld/hiew_incr2.png}
\caption{Hiew}
\label{}
\end{figure}

Hiew показывает строку \q{ello, world}.
И когда мы запускаем исполняемый файл, именно эта строка и выводится.

\subsubsection{Выбор другой строки из исполняемого файла (Linux x64)}

Исполняемый файл, если скомпилировать используя GCC 5.4.0 на Linux x64, имеет множество других строк:
в основном, это имена импортированных ф-ций и имена библиотек.

Запускаю objdump, чтобы посмотреть содержимое всех секций скомпилированного файла:

\begin{lstlisting}
% objdump -s a.out

a.out:     file format elf64-x86-64

Contents of section .interp:
 400238 2f6c6962 36342f6c 642d6c69 6e75782d  /lib64/ld-linux-
 400248 7838362d 36342e73 6f2e3200           x86-64.so.2.
Contents of section .note.ABI-tag:
 400254 04000000 10000000 01000000 474e5500  ............GNU.
 400264 00000000 02000000 06000000 20000000  ............ ...
Contents of section .note.gnu.build-id:
 400274 04000000 14000000 03000000 474e5500  ............GNU.
 400284 fe461178 5bb710b4 bbf2aca8 5ec1ec10  .F.x[.......^...
 400294 cf3f7ae4                             .?z.

...
\end{lstlisting}

Не проблема передать адрес текстовой строки \q{/lib64/ld-linux-x86-64.so.2} в вызов \TT{printf()}:

\begin{lstlisting}[style=customc]
#include <stdio.h>

int main()
{
    printf(0x400238);
    return 0;
}
\end{lstlisting}

Трудно поверить, но этот код печатает вышеуказанную строку.

Измените адрес на \TT{0x400260}, и напечатается строка \q{GNU}.
Адрес точен для конкретной версии GCC, GNU toolset, итд.
На вашей системе, исполняемый файл может быть немного другой, и все адреса тоже будут другими.
Также, добавление/удаление кода из исходных кодов, скорее всего, сдвинет все адреса вперед или назад.



\EN{\mysection{Returning Values: redux}

Again, when we know about function prologue and epilogue, let's recompile an example returning a value
(\ref{ret_val_func}, \ref{lst:ret_val_func}) using non-optimizing GCC:

\lstinputlisting[caption=\NonOptimizing GCC 8.2 x64 (\assemblyOutput),style=customasmx86]{patterns/017_ret_redux/1.s}

Effective instructions here are \INS{MOV} and \INS{RET}, others are -- prologue and epilogue.

}

\EN{\mysection{Returning Values: redux}

Again, when we know about function prologue and epilogue, let's recompile an example returning a value
(\ref{ret_val_func}, \ref{lst:ret_val_func}) using non-optimizing GCC:

\lstinputlisting[caption=\NonOptimizing GCC 8.2 x64 (\assemblyOutput),style=customasmx86]{patterns/017_ret_redux/1.s}

Effective instructions here are \INS{MOV} and \INS{RET}, others are -- prologue and epilogue.

}


\subsection{\Conclusion{}}

Основная разница между кодом x86/ARM и x64/ARM64 в том, что указатель на строку теперь 64-битный.
Действительно, ведь для того современные \ac{CPU} и стали 64-битными, потому что подешевела память,
её теперь можно поставить в компьютер намного больше, и чтобы её адресовать, 32-х бит уже
недостаточно.
Поэтому все указатели теперь 64-битные.

% sections
\input{patterns/01_helloworld/exercises}

