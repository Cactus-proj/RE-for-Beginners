\subsubsection{GCC: x86-64}

\myindex{x86-64}
Przyjrzyjmy się wynikom kompilacji GCC na 64-bitowym systemie Linux:

\lstinputlisting[caption=GCC 4.4.6 x64,style=customasmx86]{patterns/01_helloworld/GCC_x64_PL.s}

Na Linuksie, *BSD i \MacOSX w architekturze x86-64 argumenty funkcji także przekazuje się przez rejestry \SysVABI.

6 pierwszych argumentów jest przekazywanych przez rejestry \RDI, \RSI, \RDX, \RCX, \Reg{8} i \Reg{9}, a reszta --- przez stos.
Zgodnie z przytoczonym standardem ABI, argumenty zmiennoprzecinkowe można przekazać również przez rejestry wektorowe \XMM{0}-\XMM{7}.

Wskaźnik na łańcuch znaków jest przekazywany przez \EDI (32-bitową część rejestru). Dlaczego nie użyto 64 bitowego \RDI?

Warto pamiętać, że w 64 bitowym trybie wszystkie instrukcje \MOV, zapisujące cokolwiek do młodszej 32 bitowej części rejestru, zerują starsze 32 bity (jest to opisane w dokumentacji Intel: \myref{x86_manuals}).
Z tego powodu instrukcja \INS{MOV EAX, 011223344h} poprawnie zapisze tę wartość do \RAX, a starsze bity się wyzerują.

Jeśli byśmy podejrzeli w deasemblerze \IDA skompilowany plik (.o), to zobaczylibyśmy kody operacji (opcode) wszystkich instrukcji
\footnote{Trzeba włączyć tę opcję w \textbf{Options $\rightarrow$ Disassembly $\rightarrow$ Number of opcode bytes}}:

\lstinputlisting[caption=GCC 4.4.6 x64,style=customasmx86]{patterns/01_helloworld/GCC_x64.lst}

\label{hw_EDI_instead_of_RDI}
Jak widać, instrukcja \MOV pod adresem \TT{0x4004D4}, która zapisuje do \EDI, zajmuje 5 bajtów.
Ta sama instrukcja, zapisująca 64 bitową wartość do \RDI, zajmuje 7 bajtów.
Najwyraźniej GCC stara się zaoszczędzić trochę miejsca.
GCC jest również pewne, że segment danych, w którym przechowywany jest łańcuch znaków, nigdy nie będzie zaalokowany pod adresem powyżej 4\gls{GiB}.

\label{SysVABI_input_EAX}
Widać również, że rejestr \EAX został wyzerowany przed wywołaniem funkcji \printf.
Zgodnie ze standardem \ac{ABI} opisanym wyżej, w *NIX dla x86-64 w \EAX jest ustawiana liczba użytych rejestrów wektorowych do przekazania argumentów zmiennoprzecinkowych, jeśli funkcja może przyjmować zmienną liczbę argumentów.
Nie został wykorzystany żaden taki rejestr, a \printf jest funkcją o zmiennej liczbie argumentów, więc należy ustawić \EAX na 0.
