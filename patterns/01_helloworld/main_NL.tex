\mysection{\HelloWorldSectionName}
\label{sec:helloworld}

We bekijken het beroemde voorbeeld uit het boek [\KRBook]:

\lstinputlisting[style=customc]{patterns/01_helloworld/hw.c}

\subsection{x86}

\input{patterns/01_helloworld/MSVC_x86}
\input{patterns/01_helloworld/GCC_x86}
% \subsubsection{String patching (Win32)}

We can easily find the ``hello, world'' string in the executable file using Hiew:

\begin{figure}[H]
\centering
\myincludegraphics{patterns/01_helloworld/hola_edit1.png}
\caption{Hiew}
\label{}
\end{figure}

And we can try to translate our message into Spanish:

\begin{figure}[H]
\centering
\myincludegraphics{patterns/01_helloworld/hola_edit2.png}
\caption{Hiew}
\label{}
\end{figure}

The Spanish text is one byte shorter than English, so we also added the 0x0A byte at the end (\TT{\textbackslash{}n}) with a zero byte.

It works.

What if we want to insert a longer message?
There are some zero bytes after original English text.
It's hard to say if they can be overwritten: they may be used somewhere in \ac{CRT} code, or maybe not.
Anyway, only overwrite them if you really know what you're doing.

\subsubsection{String patching (Linux x64)}

\myindex{\radare}
Let's try to patch a Linux x64 executable using \radare{}:

\lstinputlisting[caption=\radare{} session]{patterns/01_helloworld/radare.lst}

Here's what's going on: I searched for the \q{hello} string using the \TT{/} command,
then I set the \emph{cursor} (\emph{seek}, in \radare{} terms) to that address.
Then I want to be sure that this is really that place: \TT{px} dumps bytes there.
\TT{oo+} switches \radare{} to \emph{read-write} mode.
\TT{w} writes an ASCII string at the current \emph{seek}.
Note the \TT{\textbackslash{}00} at the end---this is a zero byte.
\TT{q} quits.

\subsubsection{This is a real story of software cracking}
\myindex{\SoftwareCracking}

An image processing software, when not registered, added watermarks,
like ``This image was processed by evaluation version of [software name]'', across a picture.
We tried at random: we found that string in the executable file and put spaces instead of it.
Watermarks disappeared.
Technically speaking, they continued to appear.
\myindex{Qt}
With the help of Qt functions, the watermark was still added to the resulting image.
But adding spaces didn't alter the image itself...

\subsubsection{Software \emph{localization} of MS-DOS era}
\myindex{MS-DOS}

This method was a common way to translate MS-DOS software to Russian language back to 1980's and 1990's.
This technique is available even for those who are not aware of machine code and executable file formats.
The new string shouldn't be bigger than the old one, because there's a risk of overwriting another value or code
there.
Russian words and sentences are usually slightly longer than its English counterparts, so that is why \emph{localized}
software has a lot of weird acronyms and hardly readable abbreviations.

\begin{figure}[H]
\centering
\includegraphics[width=0.5\textwidth]{patterns/01_helloworld/Norton_Commander_v5_51.png}
\caption{\emph{Localized} Norton Commander 5.51}
\end{figure}
% note to translators: if you know such examples of MS-DOS programs 'localized' to your native language,
% please tell me, maybe I will add more screenshots.

Perhaps this also happened to other languages during that era, in other countries.

\myindex{Borland Delphi}
As for Delphi strings, the string's size must also be corrected, if needed.
 % TODO translate

\subsection{x86-64}
\input{patterns/01_helloworld/MSVC_x64}
\input{patterns/01_helloworld/GCC_x64}
% \subsubsection{Address patching (Win64)}

If our example was compiled in MSVC 2013 using \TT{/MD} switch
(meaning a smaller executable due to \TT{MSVCR*.DLL} file linkage), the \main function comes first, and can be easily found:

\begin{figure}[H]
\centering
\myincludegraphics{patterns/01_helloworld/hiew_incr1.png}
\caption{Hiew}
\label{}
\end{figure}

As an experiment, we can \gls{increment} address by 1:

\begin{figure}[H]
\centering
\myincludegraphics{patterns/01_helloworld/hiew_incr2.png}
\caption{Hiew}
\label{}
\end{figure}

Hiew shows \q{ello, world}.
And when we run the patched executable, this very string is printed.

\subsubsection{Pick another string from binary image (Linux x64)}

The binary file I've got when I compile our example using GCC 5.4.0 on Linux x64 box has many other text strings.
They are mostly imported function names and library names.

Run objdump to get the contents of all sections of the compiled file:

\begin{lstlisting}[basicstyle=\ttfamily, mathescape]
$\$$ objdump -s a.out

a.out:     file format elf64-x86-64

Contents of section .interp:
 400238 2f6c6962 36342f6c 642d6c69 6e75782d  /lib64/ld-linux-
 400248 7838362d 36342e73 6f2e3200           x86-64.so.2.
Contents of section .note.ABI-tag:
 400254 04000000 10000000 01000000 474e5500  ............GNU.
 400264 00000000 02000000 06000000 20000000  ............ ...
Contents of section .note.gnu.build-id:
 400274 04000000 14000000 03000000 474e5500  ............GNU.
 400284 fe461178 5bb710b4 bbf2aca8 5ec1ec10  .F.x[.......^...
 400294 cf3f7ae4                             .?z.

...
\end{lstlisting}

It's not a problem to pass address of the text string \q{/lib64/ld-linux-x86-64.so.2} to \TT{printf()}:

\begin{lstlisting}[style=customc]
#include <stdio.h>

int main()
{
    printf(0x400238);
    return 0;
}
\end{lstlisting}

It's hard to believe, but this code prints the aforementioned string.

If you would change the address to \TT{0x400260}, the \q{GNU} string would be printed.
This address is true for my specific GCC version, GNU toolset, etc.
On your system, the executable may be slightly different, and all addresses will also be different.
Also, adding/removing code to/from this source code will probably shift all addresses back or forward.
 % TODO translate

\EN{\mysection{Returning Values: redux}

Again, when we know about function prologue and epilogue, let's recompile an example returning a value
(\ref{ret_val_func}, \ref{lst:ret_val_func}) using non-optimizing GCC:

\lstinputlisting[caption=\NonOptimizing GCC 8.2 x64 (\assemblyOutput),style=customasmx86]{patterns/017_ret_redux/1.s}

Effective instructions here are \INS{MOV} and \INS{RET}, others are -- prologue and epilogue.

}

\EN{\mysection{Returning Values: redux}

Again, when we know about function prologue and epilogue, let's recompile an example returning a value
(\ref{ret_val_func}, \ref{lst:ret_val_func}) using non-optimizing GCC:

\lstinputlisting[caption=\NonOptimizing GCC 8.2 x64 (\assemblyOutput),style=customasmx86]{patterns/017_ret_redux/1.s}

Effective instructions here are \INS{MOV} and \INS{RET}, others are -- prologue and epilogue.

}


\subsection{\Conclusion{}}

Het grootste verschil tussen x86/ARM en x64/ARM64 code is dat de pointer naar de string nu 64-bits in lengte is.
De meeste moderne \ac{CPU}s zijn tegenwoordig 64-bit wegens zowel de verminderde gebruik van geheugen, als de grote vraag ervoor door moderne applicaties.
We kunnen hierdoor veel meer geheugen aan onze computers toevoegen dan dat 32-bit pointers kunnen aanspreken.
Bijgevolg zijn alle pointers nu 64-bit.

% sections
\input{patterns/01_helloworld/exercises}
