\mysection{\HelloWorldSectionName}
\label{sec:helloworld}

Utilisons le fameux exemple du livre [\KRBook]:

\lstinputlisting[style=customc]{patterns/01_helloworld/hw.c}

\subsection{x86}

\input{patterns/01_helloworld/MSVC_x86}
\input{patterns/01_helloworld/GCC_x86}
\subsubsection{Modification de chaînes (Win32)}

Nous pouvons facilement trouver la chaîne ``hello, world'' dans l'exécutable en utilisant Hiew:

\begin{figure}[H]
\centering
\myincludegraphics{patterns/01_helloworld/hola_edit1.png}
\caption{Hiew}
\label{}
\end{figure}

Et nous pouvons essayer de traduire notre message en espagnol:

\begin{figure}[H]
\centering
\myincludegraphics{patterns/01_helloworld/hola_edit2.png}
\caption{Hiew}
\label{}
\end{figure}

Le texte en espagnol est un octet plus court que celui en anglais, nous ajoutons l'octet 0x0A à la fin
 (\TT{\textbackslash{}n}) ainsi qu'un octet à zéro.

Ça fonctionne.

Comment faire si nous voulons insérer un message plus long ?
Il y a quelques octets à zéro après le texte original en anglais.
Il est difficile de dire s’ils peuvent être écrasés: ils peuvent être utilisés quelque part dans du code \ac{CRT},
ou pas.
De toutes façons, écrasez-les seulement si vous savez vraiment ce que vous faîtes.

\subsubsection{Modification de chaînes (Linux x64)}

\myindex{\radare}
Essayons de modifier un exécutable Linux x64 en utilisant \radare{}:

\lstinputlisting[caption=\radare{} session]{patterns/01_helloworld/radare.lst}

Ce que je fais ici: je cherche la chaîne \q{hello} en utilisant la commande  \TT{/},
ensuite je déplace le \emph{curseur} (ou \emph{seek} selon la terminologie de \radare{}) à cette adresse.
Je veux être certain d'être à la bonne adresse: \TT{px} affiche les octets ici.
\TT{oo+} passe \radare{} en mode \emph{read-write}.
\TT{w} écrit une chaîne ASCII à la \emph{seek} (\emph{position}) courante.
Notez le \TT{\textbackslash{}00} à la fin--c'est l'octet à zéro.
\TT{q} quitte.

subsubsection{Ceci est une histoire vraie de modification de logiciel}
\myindex{\SoftwareCracking}

Un logiciel de traitement d'image, lorsqu'il n'était pas enregistré, ajoutait un
tatouage numérique comme ``Cette image a été traitée par la version d'évaluation
de [nom du logiciel]'', à travers l'image.
Nous avons essayé au hasard: nous avons trouvé cette chaîne dans le fichier exécutable
et avons mis des espaces à la place.
Le tatouage a disparu.
Techniquement parlant, il continuait d'apparaître.
\myindex{Qt}
Avec l'aide des fonctions Qt, le tatouage numérique était toujours ajouté à l'image
résultante.
Mais ajouter des espaces n'altèrait pas l'image elle-même...

\subsubsection{\emph{Traduction} de logiciel à l'ère MS-DOS}

La méthode que je viens de décrire était couramment employée pour traduire des logiciels sous MS-DOS en russe dans les
années 1980 et 1990.
Les mots et les phrases russes sont en général un peu plus longs qu'en anglais, c'est pourquoi les logiciels
\emph{traduits} sont pleins d'acronymes sibyllins et d'abréviations difficilement lisibles.

\begin{figure}[H]
\centering
\includegraphics[width=0.5\textwidth]{patterns/01_helloworld/Norton_Commander_v5_51.png}
\caption{\FRph{}}
\end{figure}

Peut-être que cela s'est produit pour d'autres langages durant cette période.



\subsection{x86-64}
\input{patterns/01_helloworld/MSVC_x64}
\input{patterns/01_helloworld/GCC_x64}
\subsubsection{Modification d'adresse (Win64)}

Lorsque notre exemple est compilé sous MSVC 2013 avec l'option \TT{/MD}
(générant un exécutable plus petit du fait du lien avec \TT{MSVCR*.DLL}), la fonction \main vient en premier et
est trouvée facilement:

\begin{figure}[H]
\centering
\myincludegraphics{patterns/01_helloworld/hiew_incr1.png}
\caption{Hiew}
\label{}
\end{figure}

A titre expérimental, nous pouvons \glslink{increment}{incrémenter} l'adresse du pointeur de 1:

\begin{figure}[H]
\centering
\myincludegraphics{patterns/01_helloworld/hiew_incr2.png}
\caption{Hiew}
\label{}
\end{figure}

Hiew montre la chaîne \q{ello, world}.
Et lorsque nous lançons l'exécutable modifié, la chaîne raccourcie est affichée.

\subsubsection{Utiliser une autre chaîne d'un binaire (Linux x64)}

Le fichier binaire que j'obtiens en compilant notre exemple avec GCC 5.4.0 sur un système Linux x64 contient de
nombreuses autres chaînes:
la plupart sont des noms de fonction et de bibliothèque importées.

Je lance \emph{objdump} pour voir le contenu de toutes les sections du fichier compilé:

\begin{lstlisting}[basicstyle=\ttfamily, mathescape]
$\$$ objdump -s a.out

a.out:     file format elf64-x86-64

Contents of section .interp:
 400238 2f6c6962 36342f6c 642d6c69 6e75782d  /lib64/ld-linux-
 400248 7838362d 36342e73 6f2e3200           x86-64.so.2.
Contents of section .note.ABI-tag:
 400254 04000000 10000000 01000000 474e5500  ............GNU.
 400264 00000000 02000000 06000000 20000000  ............ ...
Contents of section .note.gnu.build-id:
 400274 04000000 14000000 03000000 474e5500  ............GNU.
 400284 fe461178 5bb710b4 bbf2aca8 5ec1ec10  .F.x[.......^...
 400294 cf3f7ae4                             .?z.

...
\end{lstlisting}

Ce n'est pas un problème de passer l'adresse de la chaîne \q{/lib64/ld-linux-x86-64.so.2} à l'appel de \TT{printf()}:

\begin{lstlisting}[style=customc]
#include <stdio.h>

int main()
{
    printf(0x400238);
    return 0;
}
\end{lstlisting}

Difficile à croire, ce code affiche la chaîne mentionnée.

Changez l'adresse en \TT{0x400260}, et la chaîne \q{GNU} sera affichée.
L'adresse est valable pour cette version spécifique de GCC, outils GNU, etc.
Sur votre système, l'exécutable peut être légèrement différent, et toutes les adresses seront différentes.
Ainsi, ajouter/supprimer du code à/de ce code source va probablement décaler les adresses en arrière et avant.



\EN{\mysection{Returning Values: redux}

Again, when we know about function prologue and epilogue, let's recompile an example returning a value
(\ref{ret_val_func}, \ref{lst:ret_val_func}) using non-optimizing GCC:

\lstinputlisting[caption=\NonOptimizing GCC 8.2 x64 (\assemblyOutput),style=customasmx86]{patterns/017_ret_redux/1.s}

Effective instructions here are \INS{MOV} and \INS{RET}, others are -- prologue and epilogue.

}

\EN{\mysection{Returning Values: redux}

Again, when we know about function prologue and epilogue, let's recompile an example returning a value
(\ref{ret_val_func}, \ref{lst:ret_val_func}) using non-optimizing GCC:

\lstinputlisting[caption=\NonOptimizing GCC 8.2 x64 (\assemblyOutput),style=customasmx86]{patterns/017_ret_redux/1.s}

Effective instructions here are \INS{MOV} and \INS{RET}, others are -- prologue and epilogue.

}


\subsection{\Conclusion{}}

La différence principale entre le code x86/ARM et x64/ARM64 est que le pointeur sur la chaîne a une taille de 64 bits.
Le fait est que les \ac{CPU}s modernes sont maintenant 64-bit à cause le la baisse du coût de la mémoire et du grand
besoin de cette dernière par les applications modernes.
Nous pouvons ajouter bien plus de mémoire à nos ordinateurs que les pointeurs 32-bit ne peuvent en adresser.
Ainsi, tous les pointeurs sont maintenant 64-bit.

% sections
\input{patterns/01_helloworld/exercises}

