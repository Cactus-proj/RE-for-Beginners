\subsubsection{ARM64}

\myparagraph{GCC}

Skompilujmy przykład w GCC 4.8.1 dla ARM64:

\lstinputlisting[numbers=left,label=hw_ARM64_GCC,caption=\NonOptimizing GCC 4.8.1 + objdump,style=customasmARM]{patterns/01_helloworld/ARM/hw.lst}

W ARM64 nie ma trybów Thumb i Thumb-2, jest tylko ARM, więc wszystkie instrukcje są 32 bitowe.

Dysponujemy dwukrotnie większą liczbą rejestrów: \myref{ARM64_GPRs}.
64 bitowe rejestry mają prefiks
\TT{X-}, a ich 32 bitowe części --- \TT{W-}.

\myindex{ARM!\Instructions!STP}
Instrukcja \TT{STP} (\emph{Store Pair}) 
odkłada na stos jednocześnie 2 rejestry: \RegX{29} i \RegX{30}.
Oczywiście może ona zapisać tę parę gdziekolwiek w pamięci, ale w tym przypadku miejscem docelowym jest rejestr \ac{SP}, a więc jest ona odkładana na stos.

Rejestry w ARM64 są 64 bitowe (8 bajtowe), dlatego do przechowywania 2 rejestrów potrzeba 16 bajtów.

Wykrzyknik (``!'') po operandzie oznacza, że najpierw od \ac{SP} będzie odjęte 16 i dopiero po tej czynności wartości z obu rejestrów będą odłożone na stos.

Jest to tak zwany \emph{pre-index}.
Więcej o różnicy między \emph{post-index} a \emph{pre-index}: \myref{ARM_postindex_vs_preindex}.

Posługując się terminologią x86 ~--- pierwsza instrukcja jest analogiczna do pary instrukcji \TT{PUSH X29} i \TT{PUSH X30}.
\RegX{29} w ARM64 jest wykorzystywane jako \ac{FP}, a \RegX{30} 
jako \ac{LR}, dlatego są one odkładane na stos w prologu funkcji.

Druga instrukcja kopiuje \ac{SP} do \RegX{29} (\ac{FP}).
Jest to niezbędne do ustawienia ramki stosu (stack frame) funkcji.

\label{pointers_ADRP_and_ADD}
\myindex{ARM!\Instructions!ADRP/ADD pair}
Instrukcje \TT{ADRP} i \ADD ustawiają adres łańcucha znaków \q{Hello!} w rejestrze \RegX{0},
ponieważ pierwszy argument funkcji jest przekazywany przez ten rejestr.
Jednakże w ARM nie ma instrukcji, za pomocą których można zapisać do rejestru dużą liczbę 
(dlatego że długość instrukcji wynosi maksymalnie 4 bajty. Więcej informacji o tym można znaleźć tutaj: \myref{ARM_big_constants_loading}).
Dlatego trzeba skorzystać z kilku instrukcji.
Pierwsza instrukcja (\TT{ADRP}) zapisuje do \RegX{0} adres strony o rozmiarze 4KiB, która zawiera łańcuch znaków,
a druga (\ADD) dodaje do tego adresu resztę (przesunięcie względem początku strony pamięci).
Więcej o tym: \myref{ARM64_relocs}.

\TT{0x400000 + 0x648 = 0x400648}, i możemy zobaczyć, że w segmencie danych \TT{.rodata} pod tym adresem znajduje się nasz
łańcuch znaków \q{Hello!}.

\myindex{ARM!\Instructions!BL}
Następnie za pomocą instrukcji \TT{BL} jest wywoływana funkcja \puts. Zostało to omówione wcześniej: \myref{puts}.

Instrukcja \MOV zapisuje 0 do \RegW{0}. 
\RegW{0} to młodsze 32 bity 64 bitowego rejestru \RegX{0}:

\begin{center}
\begin{tabular}{ | l | l | }
\hline
	Starsze 32 bity & Młodsze 32 bity \\
\hline
\multicolumn{2}{ | c | }{X0} \\
\hline
\multicolumn{1}{ | c | }{} & \multicolumn{1}{ c | }{W0} \\
\hline
\end{tabular}
\end{center}


Wynik funkcji jest zwracany przez \RegX{0} i \main zwraca 0.

Dlaczego 32 bitowa część?
W ARM64, jak i w x86-64, typ \Tint ma rozmiar 32 bitów, dla kompatybilności.

Skoro funkcja zwraca 32 bitowy \Tint, to trzeba wypełnić tylko młodsze 32 bity 64 bitowego rejestru \RegX{0}.

Żeby mieć pewność, zmieńmy przykład i skompilujmy go ponownie.
Teraz \main zwraca 64 bitową wartość:

\begin{lstlisting}[caption=funkcja \main zwracająca wartość typu \TT{uint64\_t},style=customc]
#include <stdio.h>
#include <stdint.h>

uint64_t main()
{
        printf ("Hello!\n");
        return 0;
}
\end{lstlisting}

Wynik jest taki sam, tylko \MOV w tej linii wygląda teraz:

\begin{lstlisting}[caption=\NonOptimizing GCC 4.8.1 + objdump]
  4005a4:       d2800000        mov     x0, #0x0      // #0
\end{lstlisting}

\myindex{ARM!\Instructions!LDP}
Następnie za pomocą instrukcji \INS{LDP} (\emph{Load Pair}) przywracane są rejestry \RegX{29} i \RegX{30}.

Nie ma wykrzyknika po instrukcji: oznacza to, że najpierw wartości są zdejmowane ze stosu a dopiero po tej czynności \ac{SP} jest zwiększany o 16.

Jest to tzw. \emph{post-index}.

\myindex{ARM!\Instructions!RET}
W ARM64 pojawia się nowa instrukcja: \RET. 
Działa tak samo jak \INS{BX LR}, ale zawiera specjalny bit (ang. \emph{hint bit}),
który podpowiada procesorowi, że jest to wyjście z funkcji, a nie kolejna instrukcja skoku - by procesor mógł zoptymalizować jej wykonanie.

Funkcja jest bardzo prosta, GCC z włączoną optymalizacją generuje dokładnie taki sam kod.


