\subsubsection{\OptimizingXcodeIV (\ThumbTwoMode)}

Standardmäßig generiert Xcode 4.6.3 den Thumb-2-Code auf folgende Weise:

\begin{lstlisting}[caption=\OptimizingXcodeIV (\ThumbTwoMode),style=customasmARM]
__text:00002B6C                   _hello_world
__text:00002B6C 80 B5          PUSH            {R7,LR}
__text:00002B6E 41 F2 D8 30    MOVW            R0, #0x13D8
__text:00002B72 6F 46          MOV             R7, SP
__text:00002B74 C0 F2 00 00    MOVT.W          R0, #0
__text:00002B78 78 44          ADD             R0, PC
__text:00002B7A 01 F0 38 EA    BLX             _puts
__text:00002B7E 00 20          MOVS            R0, #0
__text:00002B80 80 BD          POP             {R7,PC}

...

__cstring:00003E70 48 65 6C 6C 6F 20+aHelloWorld  DCB "Hello world!",0xA,0
\end{lstlisting}

% Q: If you subtract 0x13D8 from 0x3E70,
% you actually get a location that is not in this function, or in _puts.
% How is PC-relative addressing done in THUMB2?
% A: it's not Thumb-related. there are just mess with two different segments. TODO: rework this listing.

\myindex{\ThumbTwoMode}
\myindex{ARM!\Instructions!BL}
\myindex{ARM!\Instructions!BLX}

Die \TT{BL}- und \TT{BLX}-Anweisung im Thumb-Mode ist als Paar von 16-Bit-Anweisungen kodiert.
In Thumb-2 sind diese \emph{Ersatz}-Opcodes so erweitert, dass neue Anweisungen hier mit 32 Bit
kodiert werden können.

Offensichtlich beginnen die Opcodes der Thumb-2-Anweisungen immer mit \TT{0xFx} oder \TT{0xEx}.

Im \IDA-Listing jedoch sind die Bytes der Opcodes vertauscht weil für den ARM-Prozessor die
Anweisungen wie folgt kodiert werden:
Das letzte Byte kommt zuerst und danach das erste (für Thumb- und Thum-2-Mode) oder für
Anweisungen im ARM-Mode kommt das vierte Byte zuerst, dann das dritte, dann das zweite und
zum Schluss das erste (aufgrund des unterschiedlichen \gls{endianness}).

Die Bytes sind also im \IDA-Listing wie folgt angeordnet:
\begin{itemize}
\item für ARM und ARM64 Mode: 4-3-2-1;
\item für Thumb Mode: 2-1;
\item für 16-Bit-Anweisungspaar in Thumb-2 Mode: 2-1-4-3.
\end{itemize}

\myindex{ARM!\Instructions!MOVW}
\myindex{ARM!\Instructions!MOVT.W}
\myindex{ARM!\Instructions!BLX}

Wie zu sehen ist, beginnend die Anweisungen \TT{MOVW}, \TT{MOVT.W} und \TT{BLX} mit \TT{0xFx}.

Eine der Thumb-2-Anweisungen ist \TT{MOVW R0, \#0x13D8} ~---sie speichert einen 16-Bit-Wert in den
niederwertigeren Teil des \Reg{0}-Registers und setzt die höherwertigen Bits auf 0.

Des weiteren funktioniert \TT{MOVT.W R0, \#0} genau wie \TT{MOVT} aus dem vorherigen Beispiel,
jedoch nur für Thumb-2.

\myindex{ARM!mode switching}
\myindex{ARM!\Instructions!BLX}

Neben den anderen Unterschieden wird in diesem Fall die \TT{BLX}-Anweisung anstatt \TT{BL} genutzt.

Der Unterschied ist, dass, neben dem Speichern von \ac{RA} in das \ac{LR}-Register und die Übergabe
der Ausführungskontrolle an die \puts-Funktion, der Prozessor auch vom Thumb/Thumb-2-Mode in den
ARM-Mode (oder zurück) wechselt.

Diese Anweisung ist hier eingefügt weil die Anweisung mit der die Kontrolle abgegeben wird wie folgt
aussieht (im ARM-Mode kodiert):

\begin{lstlisting}[style=customasmARM]
__symbolstub1:00003FEC _puts           ; CODE XREF: \_hello\_world+E
__symbolstub1:00003FEC 44 F0 9F E5     LDR  PC, =__imp__puts
\end{lstlisting}

Dies ist im Endeffekt ein Sprung an die Stelle an der die Adresse von \puts in der import-Sektion geschriben wird.

Der aufmerksame Leser mag fragen: warum wird \puts nicht direkt an der Stelle im Code aufgerufen,
an der es benötigt wird? Dies wäre nicht sehr speicherplatzeffizient.

\myindex{Dynamically loaded libraries}
Fast jedes Programm nutzt externe, dynamische Bibliotheken (wie DLL in Windows, .so in *NIX oder.dylib in \MacOSX).
Diese Bibliotheken beinhalten häufig genutzte Funktion wie die Standard-C-Funktion \puts.

\myindex{Relocation}
In einer ausführbaren Binärdatei (Windows PE .exe, ELF oder Mach-O) existiert eine import-Sektion.
Dies ist eine Liste von Symbolen (Funktionen oder globale Variablen) die, zusammen mit den Namen, von
externen Modulen importiert werden.

Der \ac{OS}-Loader läd alle Module die gebraucht werden und bestimmt die korrekten Adressen von jedem Symbol,
während diese in dem primärem Modul aufgelistet werden.

In dem vorliegenden Fall ist \emph{\_\_imp\_\_puts} eine 32-Bit-Variable die vom \ac{OS}-Loader genutzt wird
um die korrekte Adresse der Funktion in der externen Bibliothek zu speichern.
Anschließend liest die \TT{LDR}-Anweisung den 32-Bit-Wert dieser Variable und schreibt ihn in das \ac{PC}-Register
bevor die Ausführkontrolle dorthin übergeben wird.

Um also die Zeit zu reduzieren die der \ac{OS}-Loader für dieses Vorgehen benötigt, ist es eine gute Idee
die Adressen für jedes Symbol einmalig an eine geeignete Stelle zu schreiben.

\myindex{thunk-functions}
Daneben wurde bereits erwähnt, dass es unmöglich ist einen 32-Bit-Wert in ein Register zu laden wenn
nur eine Anweisung ohne Speicher-Zugriff genutzt wird.

Aus diesem Grund ist die optimale Lösung, eine separate Funktion im ARM-Mode zu allozieren die lediglich
die Aufgabe hat die Ausführkontrolle an die dynamische Bibliothek zu übergeben und dann in diese kurze
Funktion mit einer Anweisung (so genannte \gls{thunk function}) aus dem Thumb-Code auszuführen.

\myindex{ARM!\Instructions!BL}
Übrigens: in dem vorherigen Beispiel (für ARM-Mode kompiliert) wird die Ausführkontrolle durch \TT{BL}
an die gleiche \gls{thunk function} übergeben.
Der Prozessor-Modus wird hier jedoch aufgrund des Fehlens eines \q{X} im Anweisungsnamen nicht gewechselt.

\myparagraph{Mehr über Thunk-Funktionen}
\myindex{thunk-functions}

Thunk-Funktionen sind aufgrund der irrtümlichen Bezeichnung schwierig zu verstehen.
Der einfachste Weg ist es sie als Adapter oder Konverter zwischen verschiedenen Anschlüssen aufzufassen.
Zum Beispiel wie einen Adapter zwischen einer britischen und einer amerikanischen Steckdose oder andersherum.
Thunk-Funktionen werden manchmal auch \emph{Wrapper} genannt.

Hier sind einige weitere Beschreibung dieser Funktionstypen:

\begin{framed}
\begin{quotation}
"Ein Teil der Software um Adressen zur Verfügung zu stellen:" nach P. Z. Ingerman,
der 1961 Thunk-Funktionen als Möglichkeit zum Binden von Aktualparametern zu deren
formalen Definitionen in Algol-60-Prozedur-Aufrufen. Wenn eine Prozedur mit einem Ausdruck anstatt
der formalen Parameter aufgerufen wird, generiert der Compiler eine Thunk-Funktion die den Ausdruck
errechnet und die Adresse des Ergebnisses an eine Standard-Stelle speichert.

\dots

% TODO: the english version is kind of misleading -- I could not find the word "bletcherous"
Microsoft und IBM haben beide in ihrem Intel-basierten System eine "16-Bit Umgebung"
und eine "32-Bit-Umgebung" definiert. Beide können auf dem selben Computer und demselben
Betriebssystem laufen (dank dem was Microsoft \q{ Windows On Windows} (WOW) nennt).
Sowohl MS als auch IBM haben entschieden, den Vorgang der zwischen 16- und 32-Bit wechselt
"Thunk" zu nennen; für Windows 95 existiert sogar ein Tool THUNK.EXE, das Thunk-Compiler
genannt wird.\end{quotation}
\end{framed}
% TODO FIXME move to bibliography and quote properly above the quote
( \href{http://www.catb.org/jargon/html/T/thunk.html}{The Jargon File} )
