\subsection{ARM}
\label{sec:hw_ARM}

\myindex{\idevices}
\myindex{Raspberry Pi}
\myindex{Xcode}
\myindex{LLVM}
\myindex{Keil}
Pour mes expérimentations avec les processeurs ARM, différents compilateurs ont été utilisés:

\begin{itemize}
\item Très courant dans le monde de l'embarqué: Keil Release 6/2013.

\item Apple Xcode 4.6.3 IDE avec le compilateur LLVM-GCC 4.2
\footnote{C'est ainsi: Apple Xcode 4.6.3 utilise les composants open-source GCC comme front-end et LLVM
comme générateur de code} % TODO clarify

\item GCC 4.9 (Linaro) (pour ARM64), disponible comme exécutable win32 ici \url{http://www.linaro.org/projects/armv8/}.

\end{itemize}

C'est du code ARM 32-bit qui est utilisé (également pour les modes Thumb et Thumb-2) dans tous les
cas dans ce livre, sauf mention contraire.

% subsections
\input{patterns/01_helloworld/ARM/keil_ARM_FR}
\input{patterns/01_helloworld/ARM/keil_T_FR}
\subsubsection{\OptimizingXcodeIV (\ARMMode)}

Xcode 4.6.3 sans l'option d'optimisation produit beaucoup de code redondant c'est
pourquoi nous allons étudier le code généré avec optimisation, où le nombre
d'instruction est aussi petit que possible, en mettant l'option \Othree du
compilateur.

\begin{lstlisting}[caption=\OptimizingXcodeIV (\ARMMode),style=customasmARM]
__text:000028C4             _hello_world
__text:000028C4 80 40 2D E9   STMFD           SP!, {R7,LR}
__text:000028C8 86 06 01 E3   MOV             R0, #0x1686
__text:000028CC 0D 70 A0 E1   MOV             R7, SP
__text:000028D0 00 00 40 E3   MOVT            R0, #0
__text:000028D4 00 00 8F E0   ADD             R0, PC, R0
__text:000028D8 C3 05 00 EB   BL              _puts
__text:000028DC 00 00 A0 E3   MOV             R0, #0
__text:000028E0 80 80 BD E8   LDMFD           SP!, {R7,PC}

__cstring:00003F62 48 65 6C 6C+aHelloWorld_0  DCB "Hello world!",0
\end{lstlisting}

Les instructions \TT{STMFD} et \TT{LDMFD} nous sont déjà familières.

\myindex{ARM!\Instructions!MOV}

L'instruction \MOV écrit simplement le nombre \TT{0x1686} dans le registre \Reg{0}.
C'est l'offset pointant sur la chaîne \q{Hello world!}.

Le registre \TT{R7} (tel qu'il est standardisé dans \IOSABI) est un pointeur de frame. Voir plus loin.

\myindex{ARM!\Instructions!MOVT}
L'instruction \TT{MOVT R0, \#0} (MOVe Top) écrit 0 dans les 16 bits de poids
fort du registre.
Le problème ici est que l'instruction générique \MOV en mode ARM peut n'écrire
que dans les 16 bits de poids faible du registre.

Il faut garder à l'esprit que tout les opcodes d'instruction en mode ARM sont
limités en taille à 32 bits. Bien sûr, cette limitation n'est pas relative
au déplacement de données entre registres.
C'est pourquoi une instruction supplémentaire existe \TT{MOVT} pour écrire dans
les bits de la partie haute (de 16 à 31 inclus).
Son usage ici, toutefois, est redondant car l'instruction \TT{MOV R0, \#0x1686}
ci dessus a effacé la partie haute du registre.
C'est soi-disant un défaut du compilateur.
% TODO:
% I think, more specifically, the string is not put in the text section,
% ie. the compiler is actually not using position-independent code,
% as mentioned in the next paragraph.
% MOVT is used because the assembly code is generated before the relocation,
% so the location of the string is not yet known,
% and the high bits may still be needed.

\myindex{ARM!\Instructions!ADD}
L'instruction \TT{ADD R0, PC, R0} ajoute la valeur dans \ac{PC} à celle de
\Reg{0}, pour calculer l'adresse absolue de la chaîne \q{Hello world!}.
Comme nous l'avons déjà vu, il s'agit de \q{\PICcode} donc la correction
est essentielle ici.

L'instruction \INS{BL} appelle la fonction \puts au lieu de \printf.

\label{puts}
\myindex{\CStandardLibrary!puts()}
\myindex{puts() instead of printf()}

LLVM a remplacé le premier appel à \printf par un à \puts.
Effectivement: \printf avec un unique argument est presque analogue à \puts.

\emph{Presque}, car les deux fonctions produisent le même résultat uniquement dans
le cas où la chaîne ne contient pas d'identifiants de format débutant par \emph{\%}.
Dans le cas où elle en contient, l'effet de ces deux fonctions est différent\footnote{Il
est à noter que \puts ne nécessite pas un `\textbackslash{}n'
symbole de retour à la ligne à la fin de la chaîne, donc nous ne le voyons pas ici.}.

Pourquoi est-ce que le compilateur a remplacé \printf par \puts? Probablement car
\puts est plus
rapide\footnote{\href{http://www.ciselant.de/projects/gcc_printf/gcc_printf.html}{ciselant.de/projects/gcc\_printf/gcc\_printf.html}}. 

Car il envoie seulement les caractères dans \glslink{stdout}{sortie standard}
sans comparer chacun d'entre eux avec le symbole \emph{\%}.

Ensuite, nous voyons l'instruction familière \TT{MOV R0, \#0} pour mettre le
registre \Reg{0} à 0.

\subsubsection{\OptimizingXcodeIV (\ThumbTwoMode)}

Par défaut Xcode 4.6.3 génère du code pour Thumb-2 de cette manière:

\begin{lstlisting}[caption=\OptimizingXcodeIV (\ThumbTwoMode),style=customasmARM]
__text:00002B6C                   _hello_world
__text:00002B6C 80 B5          PUSH            {R7,LR}
__text:00002B6E 41 F2 D8 30    MOVW            R0, #0x13D8
__text:00002B72 6F 46          MOV             R7, SP
__text:00002B74 C0 F2 00 00    MOVT.W          R0, #0
__text:00002B78 78 44          ADD             R0, PC
__text:00002B7A 01 F0 38 EA    BLX             _puts
__text:00002B7E 00 20          MOVS            R0, #0
__text:00002B80 80 BD          POP             {R7,PC}

...

__cstring:00003E70 48 65 6C 6C 6F 20+aHelloWorld  DCB "Hello world!",0xA,0
\end{lstlisting}

% Q: If you subtract 0x13D8 from 0x3E70,
% you actually get a location that is not in this function, or in _puts.
% How is PC-relative addressing done in THUMB2?
% A: it's not Thumb-related. there are just mess with two different segments. TODO: rework this listing.

\myindex{\ThumbTwoMode}
\myindex{ARM!\Instructions!BL}
\myindex{ARM!\Instructions!BLX}

Les instructions \TT{BL} et \TT{BLX} en mode Thumb, comme on s'en souvient, sont
encodées comme une paire d'instructions 16 bits.
En Thumb-2 ces opcodes \emph{substituts} sont étendus de telle sorte que les nouvelles
instructions puissent être encodées comme des instructions 32-bit.

C'est évident en considérant que les opcodes des instructions Thumb-2 commencent
toujours avec \TT{0xFx} ou \TT{0xEx}.

Mais dans le listing d'\IDA les octets d'opcodes sont échangés car pour le processeur
ARM les instructions sont encodées comme ceci:
dernier octet en premier et ensuite le premier (pour les modes Thumb et Thumb-2)
ou pour les instructions en mode ARM le quatrième octet vient en premier, ensuite
le troisième, puis le second et enfin le premier (à cause des différents \gls{endianness}).

C'est ainsi que les octets se trouvent dans le listing d'IDA:
\begin{itemize}
\item pour les modes ARM et ARM64: 4-3-2-1;
\item pour le mode Thumb: 2-1;
\item pour les paires d'instructions 16-bit en mode Thumb-2: 2-1-4-3.
\end{itemize}

\myindex{ARM!\Instructions!MOVW}
\myindex{ARM!\Instructions!MOVT.W}
\myindex{ARM!\Instructions!BLX}

Donc, comme on peut le voir, les instructions \TT{MOVW}, \TT{MOVT.W} et \TT{BLX}
commencent par \TT{0xFx}.

Une des instructions Thumb-2 est \TT{MOVW R0, \#0x13D8} ~---elle stocke une valeur
16-bit dans la partie inférieure du registre \Reg{0}, effaçant les bits supérieurs.

Aussi, \TT{MOVT.W R0, \#0} ~fonctionne comme \TT{MOVT} de l'exemple précédent
mais il fonctionne en Thumb-2.

\myindex{ARM!mode switching}
\myindex{ARM!\Instructions!BLX}

Parmi les autres différences, l'instruction \TT{BLX} est utilisée dans ce cas à
à la place de \TT{BL}.

La différence est que, en plus de sauver \ac{RA} dans le registre \ac{LR} et de
passer le contrôle à la fonction \puts, le processeur change du mode Thumb/Thumb-2
au mode ARM (ou inversement).

Cette instruction est placée ici, car l'instruction à laquelle est passée le contrôle
ressemble à (c'est encodé en mode ARM):

\begin{lstlisting}[style=customasmARM]
__symbolstub1:00003FEC _puts           ; CODE XREF: \_hello\_world+E
__symbolstub1:00003FEC 44 F0 9F E5     LDR  PC, =__imp__puts
\end{lstlisting}

Il s'agit principalement d'un saut à l'endroit où l'adresse de \puts est écrit
dans la section import.

Mais alors, le lecteur attentif pourrait demander: pourquoi ne pas appeler \puts
depuis l'endroit dans le code où on en a besoin ?

Parce que ce n'est pas très efficace en terme d'espace.

\myindex{Dynamically loaded libraries}
Presque tous les programmes utilisent des bibliothèques dynamiques externes
(comme les DLL sous Windows, les .so sous *NIX ou les .dylib sous \MacOSX).
Les bibliothèques dynamiques contiennent les bibliothèques fréquemment utilisées,
incluant la fonction C standard \puts.

\myindex{Relocation}
Dans un fichier binaire exécutable (Windows PE .exe, ELF ou Mach-O) se trouve
une section d'import.
Il s'agit d'une liste des symboles (fonctions ou variables globales) importées
depuis des modules externes avec le nom des modules eux-même.

Le chargeur de l'\ac{OS} charge tous les modules dont il a besoin, tout en énumérant
les symboles d'import dans le module primaire, il détermine l'adresse correcte de
chaque symbole.

Dans notre cas, \emph{\_\_imp\_\_puts} est une variable 32-bit utilisée par le
chargeur de l'\ac{OS} pour sauver l'adresse correcte d'une fonction dans une
bibliothèque externe.
Ensuite l'instruction \TT{LDR} lit la valeur 32-bit depuis cette variable et
l'écrit dans le registre \ac{PC}, lui passant le contrôle.

Donc, pour réduire le temps dont le chargeur de l'\ac{OS} à besoin pour réaliser
cette procédure, c'est une bonne idée d'écrire l'adresse de chaque symbole une
seule fois, à une place dédiée.

\myindex{thunk-functions}

À côté de ça, comme nous l'avons déjà compris, il est impossible de charger
une valeur 32-bit dans un registre en utilisant seulement une instruction
sans un accès mémoire.

Donc, la solution optimale est d'allouer une fonction séparée fonctionnant en
mode ARM avec le seul but de passer le contrôle à la bibliothèque dynamique et
ensuite de sauter à cette petite fonction d'une instruction (ainsi appelée
\glslink{thunk function}{fonction thunk}) depuis le code Thumb.

\myindex{ARM!\Instructions!BL}
À propos, dans l'exemple précédent (compilé en mode ARM), le contrôle est
passé par \TT{BL} à la même \glslink{thunk function}{fonction thunk}.
Le mode du processeur, toutefois, n'est pas échangé (d'où l'absence d'un \q{X}
dans le mnémonique de l'instruction).

\myparagraph{Plus à propos des fonctions thunk}
\myindex{thunk-functions}

Les fonctions thunk sont difficile à comprendre, apparemment, à cause d'un
mauvais nom.
La manière la plus simple est de les voir comme des adaptateurs ou des
convertisseurs d'un type jack à un autre.
Par exemple, un adaptateur permettant l'insertion d'un cordon électrique
britannique sur une prise murale américaine, ou vice-versa.
Les fonctions thunk sont parfois appelées \emph{wrappers}.

Voici quelques autres descriptions de ces fonctions:

\begin{framed}
\begin{quotation}
“Un morceau de code qui fournit une adresse:”, d'après P. Z. Ingerman, qui inventa
thunk en 1961 comme un moyen de lier les paramètres réels à leur définition
formelle dans les appels de procédures en Algol-60. Si une procédure est appelée
avec une expression à la place d'un paramètre formel, le compilateur génère un
thunk qui calcule l'expression et laisse l'adresse du résultat dans une place
standard.

\dots

Microsoft et IBM ont tous les deux défini, dans systèmes basés sur Intel, un
"environnement 16-bit" (avec leurs horribles registres de segment et la limite des
adresses à 64K) et un "environnement 32-bit" (avec un adressage linéaire et une
gestion semi-réelle de la mémoire). Les deux environnements peuvent fonctionner
sur le même ordinateur et OS (grâce à ce qui est appelé, dans le monde
Microsoft, WOW qui signifie Windows dans Windows). MS et IBM ont tous deux décidé
que le procédé de passer de 16-bit à 32-bit et vice-versa est appelé un "thunk";
pour Window 95, il y a même un outil, THUNK.EXE, appelé un "compilateur thunk".
\end{quotation}
\end{framed}
% TODO FIXME move to bibliography and quote properly above the quote
( \href{http://www.catb.org/jargon/html/T/thunk.html}{The Jargon File} )

\myindex{LAPACK}
\myindex{FORTRAN}
Nous pouvons trouver un autre exemple dans la bibliothèque LAPCAK---un ``Linear Algebra PACKage''
écrit en FORTRAN.
Les développeurs \CCpp veulent aussi utiliser LAPACK, mais c'est un non-sens de
la récrire en \CCpp et de maintenir plusieurs versions.
Donc, il y a des petites fonctions que l'on peut invoquer depuis un environnement
\CCpp, qui font, à leur tour, des appels aux fonctions FORTRAN, et qui font
presque tout le reste:

\begin{lstlisting}[style=customc]
double Blas_Dot_Prod(const LaVectorDouble &dx, const LaVectorDouble &dy)
{
    assert(dx.size()==dy.size());
    integer n = dx.size();
    integer incx = dx.inc(), incy = dy.inc();

    return F77NAME(ddot)(&n, &dx(0), &incx, &dy(0), &incy);
}
\end{lstlisting}

Donc, ce genre de fonctions est appelé ``wrappers''.


\subsubsection{ARM64}

\myparagraph{GCC}

Compilons l'exemple en utilisant GCC 4.8.1 en ARM64:

\lstinputlisting[numbers=left,label=hw_ARM64_GCC,caption=GCC 4.8.1 \NonOptimizing + objdump,style=customasmARM]{patterns/01_helloworld/ARM/hw.lst}

Il n'y a pas de mode Thumb ou Thumb-2 en ARM64, seulement en ARM, donc il n'y a que des
instructions 32-bit.
Le nombre de registres a doublé: \myref{ARM64_GPRs}.
Les registres 64-bit ont le préfixe \TT{X-}, tandis que leurs partie 32-bit basse---\TT{W-}.

\myindex{ARM!\Instructions!STP}
L'instruction \TT{STP} (\emph{Store Pair} stocke une paire)
sauve deux registres sur la pile simultanément: \RegX{29} et \RegX{30}.

Bien sûr, cette instruction peut sauvegarder cette paire à n'importe quelle endroit en mémoire,
mais le registre \ac{SP} est spécifié ici, donc la paire est sauvé sur le pile.

Les registres ARM64 font 64-bit, chacun a une taille de 8 octets, donc il faut 16 octets pour sauver
deux registres.

Le point d'exclamation (``!'') après l'opérande signifie que 16 octets doivent d'abord être soustrait de \ac{SP},
et ensuite les valeurs de la paire de registres peuvent être écrites sur la pile.
Ceci est appelé le \emph{pre-index}.
À propos de la différence entre \emph{post-index} et \emph{pre-index}
lisez ceci: \myref{ARM_postindex_vs_preindex}.

Dans la gamme plus connue du x86, la première instruction est analogue à la paire
\TT{PUSH X29} et \TT{PUSH X30}.
En ARM64, \RegX{29} est utilisé comme \ac{FP} et \RegX{30} comme \ac{LR}, c'est pourquoi ils sont
sauvegardés dans le prologue de la fonction et remis dans l'épilogue.

La seconde instruction copie \ac{SP} dans \RegX{29} (ou \ac{FP}).
Cela sert à préparer la pile de la fonction.

\label{pointers_ADRP_and_ADD}
\myindex{ARM!\Instructions!ADRP/ADD pair}
Les instructions \TT{ADRP} et \ADD sont utilisées pour remplir l'adresse de
la chaîne \q{Hello!} dans le registre \RegX{0},
car le premier argument de la fonction est passé dans ce registre.
Il n'y a pas d'instruction, quelqu'elle soit, en ARM qui puisse stocker un nombre large
dans un registre (car la longueur des instructions est limitée à 4 octets, cf: \myref{ARM_big_constants_loading}).
Plusieurs instructions doivent donc être utilisées. La première instruction (\TT{ADRP}) écrit l'adresse de
la page de 4KiB, où se trouve la chaîne, dans \RegX{0}, et la seconde (\ADD) ajoute simplement
le reste de l'adresse.
Plus d'information ici: \myref{ARM64_relocs}.

\TT{0x400000 + 0x648 = 0x400648}, et nous voyons notre chaîne C \q{Hello!} dans le \TT{.rodata} segment
des données à cette adresse.

\myindex{ARM!\Instructions!BL}

\puts est appelée après en utilisant l'instruction \TT{BL}. Cela a déjà été discuté: \myref{puts}.

\MOV écrit 0 dans \RegW{0}.
\RegW{0} est la partie basse 32 bits du registre 64-bit \RegX{0}:

\begin{center}
\begin{tabular}{ | l | l | }
\hline
	Partie 32 bits haute & Partie 32 bits basse \\
\hline
\multicolumn{2}{ | c | }{X0} \\
\hline
\multicolumn{1}{ | c | }{} & \multicolumn{1}{ c | }{W0} \\
\hline
\end{tabular}
\end{center}


Le résultat de la fonction est retourné via \RegX{0} et main renvoie 0, donc c'est ainsi que la valeur
de retour est préparée.
Mais pourquoi utiliser la partie 32-bit?

Parce que le type de donnée \Tint en ARM64, tout comme en x86-64, est toujours 32-bit, pour une
meilleure compatibilité.

Donc si la fonction renvoie un \Tint 32-bit, seul les 32 premiers bits du registre \RegX{0} doivent
être remplis.

Pour vérifier ceci, changeons un peu cet exemple et recompilons-le.
Maintenant, \main renvoie une valeur sur 64-bit:

\begin{lstlisting}[caption=\main renvoie une valeur de type \TT{uint64\_t} type,style=customc]
#include <stdio.h>
#include <stdint.h>

uint64_t main()
{
        printf ("Hello!\n");
        return 0;
}
\end{lstlisting}

Le résultat est le même, mais c'est à quoi ressemble \MOV à cette ligne maintenant:

\begin{lstlisting}[caption=GCC 4.8.1 \NonOptimizing + objdump]
  4005a4:       d2800000        mov     x0, #0x0      // #0
\end{lstlisting}

\myindex{ARM!\Instructions!LDP}

\INS{LDP} (\emph{Load Pair}) remet les registres \RegX{29} et \RegX{30}.

Il n'y a pas de point d'exclamation après l'instruction: celui signifie que les valeurs sont
d'abord chargées depuis la pile, et ensuite \ac{SP} est incrémenté de 16.
Cela est appelé \emph{post-index}.

\myindex{ARM!\Instructions!RET}
Une nouvelle instruction est apparue en ARM64: \RET.
Elle fonctionne comme \TT{BX LR}, un \emph{hint} bit particulier est ajouté, qui informe le \ac{CPU}
qu'il s'agit d'un retour de fonction, et pas d'une autre instruction de saut, et il peut l'exécuter
de manière plus optimale. 

À cause de la simplicité de la fonction, GCC avec l'option d'optimisation génère le même code.


