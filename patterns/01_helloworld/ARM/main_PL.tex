\subsection{ARM}
\label{sec:hw_ARM}

\myindex{\idevices}
\myindex{Raspberry Pi}
\myindex{Xcode}
\myindex{LLVM}
\myindex{Keil}
Do eksperymentów z ARM skorzystamy z kilku kompilatorów:

\begin{itemize}
\item popularnego w systemach wbudowanych Keil Release 6/2013,

\item Apple Xcode 4.6.3 IDE z kompilatorem LLVM-GCC 4.2
\footnote{W rzeczywistości Apple Xcode 4.6.3 korzysta z GCC jako front-endu i z LLVM jako generatora kodu binarnego},

\item GCC 4.9 (Linaro) (dla ARM64), 
jest dostępny w postaci pliku wykonywalnego dla win32 na \url{http://go.yurichev.com/17325}.

\end{itemize}

Wszędzie w tej książce, jeżeli zaznaczono inaczej, mówimy o 32-bitowym ARM (włączając tryb Thumb i Thumb-2).
64-bitowym ARM będzie oznaczony explicite jako ARM64.

% subsections
\subsubsection{\NonOptimizingKeilVI (\ARMMode)}

Na początek skompilujmy nasz przykład za pomocą Keil:

\begin{lstlisting}
armcc.exe --arm --c90 -O0 1.c 
\end{lstlisting}

\myindex{\IntelSyntax}
Kompilator \emph{armcc} generuje listing w asemblerze w składni Intela.
Listing zawiera niektóre wysokopoziomowe makra, związane z ARM
\footnote{na przykład listing zawiera instrukcję \PUSH/\POP, których nie ma w trybie ARM}.
Nas interesują prawdziwe instrukcje, dlatego zobaczmy jak wygląda skompilowany kod w programie \IDA.

\begin{lstlisting}[caption=\NonOptimizingKeilVI (\ARMMode) \IDA,style=customasmARM]
.text:00000000             main
.text:00000000 10 40 2D E9    STMFD   SP!, {R4,LR}
.text:00000004 1E 0E 8F E2    ADR     R0, aHelloWorld ; "hello, world"
.text:00000008 15 19 00 EB    BL      __2printf
.text:0000000C 00 00 A0 E3    MOV     R0, #0
.text:00000010 10 80 BD E8    LDMFD   SP!, {R4,PC}

.text:000001EC 68 65 6C 6C+aHelloWorld  DCB "hello, world",0    ; DATA XREF: main+4
\end{lstlisting}

Widać, że każda instrukcja ma rozmiar 4 bajtów ~--- zgodnie z oczekiwaniami, ponieważ kompilowaliśmy nasz kod dla trybu ARM, a nie Thumb.

\myindex{ARM!\Instructions!STMFD}
\myindex{ARM!\Instructions!POP}
Pierwsza instrukcja, \INS{STMFD SP!, \{R4,LR\}}\footnote{\ac{STMFD}},
działa podobnie jak instrukcja \PUSH w x86: odkłada wartości dwóch rejestrów (\Reg{4} i \ac{LR}) na stos.

W rzeczy samej, kompilator \emph{armcc}, generując listing, wstawił tam dla uproszczenia instrukcję \INS{PUSH \{r4,lr\}}.
Nie jest to do końca precyzyjne, ponieważ instrukcja \PUSH dostępna jest w trybie Thumb.
By uniknąć dezorientacji, wygenerowany kod maszynowy podglądamy w programie \IDA.

Instrukcja najpierw zmniejsza \ac{SP}, by wskazywał na miejsce na stosie dostępne do zapisu nowych wartości, następnie zapisuje wartości rejestrów \Reg{4} i \ac{LR}
pod adres w pamięci, na który wskazuje zmodyfikowany rejestr \ac{SP}.

Podobnie jak \PUSH w trybie Thumb, ta instrukcja pozwala na odkładanie na stos wartości kilku rejestrów na raz, co może być bardzo wygodne.

Przy okazji, takie zachowanie nie ma swojego odpowiednika w x86.

Można zauważyć, że \TT{STMFD} jest generalizacją instrukcji \PUSH (czyli rozszerza jej możliwości), dlatego że może operować na różnych rejestrach, a nie tylko na \ac{SP}.
Inaczej mówiąc, z \TT{STMFD} można korzystać przy zapisie wartości kilku rejestrów we wskazane miejsce w pamięci.

\myindex{\PICcode}
\myindex{ARM!\Instructions!ADR}
Instrukcja \INS{ADR R0, aHelloWorld} dodaje/odejmuje wartość w rejestrze \ac{PC} (R0) do/od przesunięcia, w którym jest przechowywany łańcuch znaków
\TT{hello, world}.
Dlaczego użyto tutaj rejestru \ac{PC}? Jest to tzw. \q{\PICcode}
\footnote{Jest to szerzej omówione w kolejnym rozdziale ~(\myref{sec:PIC})}.

Taki kod można uruchomić z dowolnego miejsca w pamięci.
Inaczej mówiąc jest to adresowanie względne, względem rejestru \ac{PC}.
W kodzie operacji (opcode) instrukcji \TT{ADR} jest zapisane przesunięcie (offset) między adresem tej instrukcji a adresem łańcucha znaków.
Przesunięcie zawsze jest stałe, niezależnie od tego, w które miejsce \ac{OS} załadował nasz kod.
Dlatego, wszystko czego potrzebujemy~--- to dodanie adresu bieżącej instrukcji (z \ac{PC}), żeby otrzymać adres bezwględny łańcucha znaków.

\myindex{ARM!\Registers!Link Register}
\myindex{ARM!\Instructions!BL}
Instrukcja \INS{BL \_\_2printf}\footnote{Branch with Link} wywołuje funkcję \printf.
Działanie tej instrukcji przebiega w 2 krokach:

\begin{itemize}
\item zapisz adres występujący po instrukcji \INS{BL} (\TT{0xC}) do rejestru \ac{LR},
\item przekaż sterowanie do funkcji \printf, zapisując jej adres do rejestru \ac{PC}.
\end{itemize}

Kiedy funkcja \printf zakończy działanie, musi wiedzieć gdzie zwrócić sterowanie. Dlatego każda funkcja, kończąc pracę, zwraca sterowanie pod adres zapisany w rejestrze \ac{LR}.

Na tym polega główna różnica między \q{czystymi} procesorami \ac{RISC}, jak ARM, i procesorami \ac{CISC} w rodzaju x86,
gdzie adres powrotu zwykle jest odkładany na stos. Przeczytasz o tym więcej w kolejnym rozdziale ~(\myref{sec:stack}).

Dodatkowo nie jest możliwe zakodowanie 32 bitowego adresu bezwzględnego (lub przesunięcia) w 32-bitowej instrukcji \INS{BL}, ponieważ ma ona miejsce tylko dla 24 bitów.
Jak zapewne pamiętasz, wszystkie instrukcje w trybie ARM mają długość 4 bajtów (32 bitów) i mogą się znajdować tylko pod adresem wyrównanym do krotności 4 bajtów.
Oznacza to, że ostatnich 2 bitów (które są zawsze zerowe) można nie kodować.
Ostatecznie zostaje nam 26 bitów na zakodowanie przesunięcia. Odpowiada to zakresowi $current\_PC \pm{} \approx{}32M$.

\myindex{ARM!\Instructions!MOV}
Następna instrukcja \INS{MOV R0, \#0}\footnote{oznacza MOV}
po prostu zapisuje 0 do rejestru \Reg{0}.
To dlatego, że nasza funkcja zwraca 0, a wartości zwracane z funkcji zapisywane są do \Reg{0}.

\myindex{ARM!\Registers!Link Register}
\myindex{ARM!\Instructions!LDMFD}
\myindex{ARM!\Instructions!POP}
Ostatnia instrukcja to \INS{LDMFD SP!, {R4,PC}}\footnote{\ac{LDMFD}~--- jest instrukcją odwrotną do \ac{STMFD}}.
Pobiera ona wartość ze stosu (lub z pamięci), zapisuje do \Reg{4} i \ac{PC} oraz zwiększa \glslink{stack pointer}{wskaźnik stosu} \ac{SP}. Działa podobnie jak instrukcja \POP.\\
Notabene pierwsza instrukcja \TT{STMFD} odłożyła na stos wartości z rejestrów \Reg{4} i \ac{LR}, ale teraz zostały one przywrócone do \Reg{4} i \ac{PC}, dzięki instrukcji \TT{LDMFD}.

Jak już wiemy, rejestr \ac{LR} zawiera adres w pamięci, pod który funkcja zwróci sterowanie po zakończeniu swojej pracy.
Pierwsza instrukcja odkłada tę wartość na stos, ponieważ ten sam rejestr zostanie wykorzystany przez naszą funkcję \main, gdy wywoła ona funkcję \printf.

Na końcu funkcji ta wartość zapisywana jest do rejestru \ac{PC}, w ten sposób przekazując sterowanie tam, skąd została wywołana.

Z reguły funkcja \main jest funkcją główną w \CCpp, więc zarządzanie zostanie zwrócone do loadera w \ac{OS}, lub gdzieś do \ac{CRT}, lub w jeszcze inne, podobne, miejsce.

Wszystko to pozwala pozbyć się ręcznego wywoływania \INS{BX LR} (skok pod adres z rejestru \ac{LR}) na samym końcu funcji.

\myindex{ARM!DCB}
\TT{DCB}~--- dyrektywa asemblera, opisująca tablicę bajtów bądź ciąg znaków ASCII, podobna do dyrektywy DB w x86



\subsubsection{\NonOptimizingKeilVI (\ThumbMode)}

Skompilujmy ten sam przykład za pomocą Keil w trybie
 Thumb:

\begin{lstlisting}
armcc.exe --thumb --c90 -O0 1.c 
\end{lstlisting}

Otrzymamy (w \IDA):

\begin{lstlisting}[caption=\NonOptimizingKeilVI (\ThumbMode) + \IDA,style=customasmARM]
.text:00000000             main
.text:00000000 10 B5          PUSH    {R4,LR}
.text:00000002 C0 A0          ADR     R0, aHelloWorld ; "hello, world"
.text:00000004 06 F0 2E F9    BL      __2printf
.text:00000008 00 20          MOVS    R0, #0
.text:0000000A 10 BD          POP     {R4,PC}

.text:00000304 68 65 6C 6C+aHelloWorld  DCB "hello, world",0    ; DATA XREF: main+2
\end{lstlisting}

Od razu można zauważyć (16-bitowe) opcode --- jest to, jak już było powiedziane, Thumb.

\myindex{ARM!\Instructions!BL}
Oprócz instrukcji \TT{BL}.
Ale tak naprawdę to ona się składa z dwóch 16-bitowych instrukcji.
Tak się dzieje dlatego że w jednym 16-bitowym opcode jest za mało miejsca dla przesunięcia, po którym znajduje się funkcja \printf.
Dlatego pierwsza 16-bitowa instrukcja ładuje starsze 10 bitów przesunięcia, a druga~--- młodsze 11 bitów przesunięcia.

% TODO:
% BL has space for 11 bits, so if we don't encode the lowest bit,
% then we should get 11 bits for the upper half, and 12 bits for the lower half.
% And the highest bit encodes the sign, so the destination has to be within
% \pm 4M of current_PC.
% This may be less if adding the lower half does not carry over,
% but I'm not sure --all my programs have 0 for the upper half,
% and don't carry over for the lower half.
% It would be interesting to check where __2printf is located relative to 0x8
% (I think the program counter is the next instruction on a multiple of 4
% for THUMB).
% The lower 11 bytes of the BL instructions and the even bit are
% 000 0000 0110 | 001 0010 1110 0 = 000 0000 0110 0010 0101 1100 = 0x00625c,
% so __2printf should be at 0x006264.
% But if we only have 10 and 11 bits, then the offset would be:
% 00 0000 0110 | 01 0010 1110 0 = 0 0000 0011 0010 0101 1100 = 0x00325c,
% so __2printf should be at 0x003264.
% In this case, though, the new program counter can only be 1M away,
% because of the highest bit is used for the sign.

jak już było powiedziane, wszystkie instrukcje w trybie Thumb są 2 bajtowe (lub 16 bitowe).
Dlatego sytuacja, w której Thumb-instrukcja zaczyna się pod adresem nieparzystym, jest niemożliwa.

Wniosek z tego jest taki, że ostatniego bitu adresu można nie kodować.
W ten sposób, w Thumb-instrukcji \TT{BL} można zakodować adres $current\_PC \pm{}\approx{}2M$.

\myindex{ARM!\Instructions!PUSH}
\myindex{ARM!\Instructions!POP}
Reszta instrukcji w funkcji (\PUSH i \POP) działa prawie tak samo, jak i opisane wyżej \TT{STMFD}/\TT{LDMFD}, tyle że rejestr \ac{SP} nie jest tu wskazany w sposób jawny.
\INS{ADR} działa dokładnie tak samo, jak i w poprzednim przykładzie.
\INS{MOVS} zapisuje wartość 0 do rejestru \Reg{0}, żeby f-cja zwróciła zero.



\subsubsection{\OptimizingXcodeIV (\ARMMode)}

Xcode 4.6.3 bez włączonego trybu optymalizacji generuje za dużo zbędnego kodu, dlatego włączymy optymalizację (flaga \Othree), dzięki czemu liczba instrukcji będzie tak mała, jak to możliwe.

\begin{lstlisting}[caption=\OptimizingXcodeIV (\ARMMode),style=customasmARM]
__text:000028C4             _hello_world
__text:000028C4 80 40 2D E9   STMFD           SP!, {R7,LR}
__text:000028C8 86 06 01 E3   MOV             R0, #0x1686
__text:000028CC 0D 70 A0 E1   MOV             R7, SP
__text:000028D0 00 00 40 E3   MOVT            R0, #0
__text:000028D4 00 00 8F E0   ADD             R0, PC, R0
__text:000028D8 C3 05 00 EB   BL              _puts
__text:000028DC 00 00 A0 E3   MOV             R0, #0
__text:000028E0 80 80 BD E8   LDMFD           SP!, {R7,PC}

__cstring:00003F62 48 65 6C 6C+aHelloWorld_0  DCB "Hello world!",0
\end{lstlisting}

Instrukcje \TT{STMFD} i \TT{LDMFD} są już nam znane.

\myindex{ARM!\Instructions!MOV}
Instrukcja \MOV zapisuje liczbę \TT{0x1686} do rejestru \Reg{0}~--- jest to przesunięcie, wskazujące na łańcuch znaków \q{Hello world!}.

Rejestr \Reg{7} (wg standardu \IOSABI) przechowuje wskaźnik ramki stosu (frame pointer). Będzie to omówione później.

\myindex{ARM!\Instructions!MOVT}
Instrukcja \TT{MOVT R0, \#0} (MOVe Top) zapisuje 0 do starszych 16 bitów rejestru.
Zwykła instrukcja \MOV w trybie ARM może zapisywać tylko do młodszych 16 bitów rejestru, z uwagi na długość instrukcji.

Warto pamiętać, że w trybie ARM kody operacji (opcode) instrukcji są ograniczone do 32 bitów. Nie ma to oczywiście wpływu na przenoszenie wartości między rejestrami.
Z tego powodu do zapisywania do starszych bitów (od 16 do 31, włącznie) istnieje dodatkowa instrukcja \INS{MOVT}.
Tutaj jej użycie jest zbędne, gdyż \INS{MOV R0, \#0x1686} i tak by wyzerowała starszą część rejestru.
Możliwe, że jest to niedociągnięcie kompilatora.
% TODO:
% I think, more specifically, the string is not put in the text section,
% ie. the compiler is actually not using position-independent code,
% as mentioned in the next paragraph.
% MOVT is used because the assembly code is generated before the relocation,
% so the location of the string is not yet known,
% and the high bits may still be needed.

\myindex{ARM!\Instructions!ADD}
Instrukcja \TT{ADD R0, PC, R0} dodaje \ac{PC} do \Reg{0} żeby wyliczyć adres bezwzględny łańcucha znaków \q{Hello world!}. Jak już wiemy, jest to \q{\PICcode}, dlatego taka korekta jest niezbędna.

Instrukcja \TT{BL} wywołuje \puts zamiast \printf.

\label{puts}
\myindex{\CStandardLibrary!puts()}
\myindex{puts() zamiast printf()}
LLVM zamienił wywołanie \printf na \puts.
Działanie \printf z jednym argumentem jest prawie równoznaczna \puts.
 
\emph{Prawie}, gdyż obie funkcje zadziałają identycznie, jeśli łańcuch znaków nie będzie zawierał sekwencji opisujących format, zaczynających się od znaku \emph{\%}. Jeśli będzie, wtedy wyniki ich pracy będą różne.
\footnote{Należy również zauważyć, że \puts nie potrzebuje znaku nowej linii `\textbackslash{}n' na końcu łańcucha,
dlatego został on pominięty.}.

Dlaczego kompilator zamienił wywoływaną funkcję? Prawdopodobnie dlatego, że funkcja \puts jest szybsza
\footnote{\href{http://www.ciselant.de/projects/gcc_printf/gcc_printf.html}{ciselant.de/projects/gcc\_printf/gcc\_printf.html}}. 
Najwidoczniej dlatego, że \puts przekazuje znaki na \gls{stdout}, nie porównując ich ze znakiem \emph{\%}.

Dalej jest już znana instrukcja \TT{MOV R0, \#0}, ustawiająca 0 jako wartość zwracaną.



\subsubsection{\OptimizingXcodeIV (\ThumbTwoMode)}

Domyślnie Xcode 4.6.3 wygeneruje trybie Thumb-2 podobny kod:

\begin{lstlisting}[caption=\OptimizingXcodeIV (\ThumbTwoMode),style=customasmARM]
__text:00002B6C                   _hello_world
__text:00002B6C 80 B5          PUSH            {R7,LR}
__text:00002B6E 41 F2 D8 30    MOVW            R0, #0x13D8
__text:00002B72 6F 46          MOV             R7, SP
__text:00002B74 C0 F2 00 00    MOVT.W          R0, #0
__text:00002B78 78 44          ADD             R0, PC
__text:00002B7A 01 F0 38 EA    BLX             _puts
__text:00002B7E 00 20          MOVS            R0, #0
__text:00002B80 80 BD          POP             {R7,PC}

...

__cstring:00003E70 48 65 6C 6C 6F 20+aHelloWorld  DCB "Hello world!",0xA,0
\end{lstlisting}

% Q: If you subtract 0x13D8 from 0x3E70,
% you actually get a location that is not in this function, or in _puts.
% How is PC-relative addressing done in THUMB2?
% A: it's not Thumb-related. there are just mess with two different segments. TODO: rework this listing.

\myindex{\ThumbTwoMode}
\myindex{ARM!\Instructions!BL}
\myindex{ARM!\Instructions!BLX}
Instrukcje \TT{BL} i \TT{BLX} w trybie Thumb są kodowane jako para 16-bitowych instrukcji.
W Thumb-2 te zastępcze kody operacji (opcode) rozszerzono tak, by instrukcje mogły być zakodowane jako 32-bitowe.

Można to łatwo zauważyc, gdyż w w trybie Thumb-2 wszystkie 32-bitowe instrukcje zaczynają się od \TT{0xFx} lub \TT{0xEx}.

Jednak na listingu w programie \IDA bajty kodu operacji (opcode) są zamienione miejscami.
W procesorze ARM instrukcje są kodowane w następujący sposób:
najpierw podaje się ostatni bajt, potem pierwszy (dla trybów Thumb i Thumb-2), lub, dla trybu ARM, najpierw czwarty bajt, następnie trzeci, drugi i pierwszy
(z uwagi na różną \glslink{endianness}{kolejność bajtów}).

Bajty są wypisywane w listingach IDA w następującej kolejności:

\begin{itemize}
\item dla trybów ARM i ARM64: 4-3-2-1;
\item dla trybu Thumb: 2-1;
\item dla pary 16-bitowych instrukcji w trybie Thumb-2: 2-1-4-3.
\end{itemize}

\myindex{ARM!\Instructions!MOVW}
\myindex{ARM!\Instructions!MOVT.W}
\myindex{ARM!\Instructions!BLX}
Widzimy, że instrukcje \TT{MOVW}, \TT{MOVT.W} i \TT{BLX} rzeczywiście zaczynają się od \TT{0xFx}.

Jedną z tych instrukcji jest
\TT{MOVW R0, \#0x13D8}~--- zapisuje ona 16-bitową liczbę do młodszych bitów rejestru \Reg{0}, zerując starsze.

Inną instrukcją jest \TT{MOVT.W R0, \#0}~---  działa tak jak \TT{MOVT} z poprzedniego przykładu, ale przeznaczona jest dla trybu Thumb-2

\myindex{ARM!przełączanie trybów}
\myindex{ARM!\Instructions!BLX}
W tym przykładzie wykorzystana została instrukcja \TT{BLX} zamiast \TT{BL}.
Różnica polega na tym, że oprócz zapisania adresu powrotu (\ac{RA}) do rejestru \ac{LR} i przekazania sterowania
do funkcji \puts, odbywa się zmiana trybu procesora z Thumb/Thumb-2 na tryb ARM (lub odwrotnie).

Jest to niezbędne dlatego, że instrukcja, do której zostanie przekazane sterowanie jest zakodowana w trybie ARM i wygląda następująco:

\begin{lstlisting}[style=customasmARM]
__symbolstub1:00003FEC _puts           ; CODE XREF: \_hello\_world+E
__symbolstub1:00003FEC 44 F0 9F E5     LDR  PC, =__imp__puts
\end{lstlisting}

Jest to skok do miejsca, w którym, w sekcji importów, zapisany jest adres funkcji \puts.
Można zadać pytanie: dlaczego nie można wywołać \puts bezpośrednio, tam gdzie jest to potrzebne?

Nie jest to efektywne, z punktu widzenia oszczędności miejsca.

\myindex{Biblioteki łączone dynamicznie (DLL, z ang. Dynamic-Link Library )}
Praktycznie każdy program korzysta z zewnętrznych bibliotek łączonych dynamicznie (jak DLL w Windows, .so w *NIX
czy .dylib w \MacOSX).
W bibliotekach dynamicznych znajdują się często wykorzystywane funkcje biblioteczne, w tym funkcja \puts ze standardu C.

\myindex{Relocation}
W wykonywalnym pliku binarnym (Windows PE .exe, ELF lub Mach-O) istnieje sekcja importów.
Jest to lista symboli (funkcji lub zmiennych globalnych) importowanych z modułów zewnętrznych wraz z nazwami tych modułów.
Program ładujący \ac{OS} iterując po symbolach zaimportowanych w module głównym, ładuje niezbędne moduły i ustawia rzeczywiste adresy każdego z symboli.

W naszym przypadku, \emph{\_\_imp\_\_puts} jest 32-bitową zmienną, w której program ładujący \ac{OS} umieści rzeczywisty adres funkcji z biblioteki zewnętrznej.

Następnie \TT{LDR} odczytuje 32-bitową wartość z tej zmiennej, i zapisując ją do rejestru \ac{PC}, przekazuje tam sterowanie.

Żeby skrócic czas tej procedury programu ładującego, trzeba sprawić aby adres każdego symbolu zapisywał się tylko raz, do specjalnie przydzielonego miejsca.

\myindex{thunk-functions}
Do tego, jak się już upewniliśmy, zapisywanie 32-bitowej liczby do rejestru jest niemożliwe bez odwoływania się do pamięci.

Optymalnym rozwiązaniem jest wydzielenie osobnej funkcji, pracującej w trybie ARM,
której jedynym celem jest przekazywanie sterowania dalej, do bilioteki dynamicznie łączonej. Nastepnie można wywoływać tę jednoinstrukcyjną funkcję z kodu w trybie Thumb.

\myindex{ARM!\Instructions!BL}
Nawiasem mówiąc, w poprzednim przykładzie (skompilowanym dla trybu ARM), instrukcja \TT{BL} przekazuje sterowanie do takiej samej \glslink{thunk function}{thunk-funkcji}, lecz procesor nie przestawia się w inny tryb (stąd brak \q{X} w mnemoniku instrukcji).

\myparagraph{Jeszcze o thunk-funkcjach}
\myindex{thunk-functions}

Thunk-funkcje są trudne do zrozumienia przede wszystkim przez brak spójności w terminologii.
Najprościej jest myśleć o nich jak o adapterach-przejściówkach z jednego typu gniazdek na drugi.
Na przykład, adapter pozwalający włożyć do gniazdka amerykańskiego wtyczkę brytyjską lub na odwrót. Thunk-funkcje również są czasami nazywane \emph{wrapper-ami}. \emph{Wrap} w języku angielskim to \emph{owinąć}, \emph{zawinąć}, \emph{opakować}.
Oto jeszcze kilka definicji tych funkcji:

\begin{framed}
\begin{quotation}
“Kawałek kodu, który dostarcza adres:”, według P. Z. Ingerman,
który wymyślił \emph{thunk} w 1961 roku, jako sposób na powiązanie parametrów rzeczywistych z ich formalnymi
definicjami w wywołaniach procedur, w języku Algol-60. Jeśli procedura jest wywołana z wyrażeniem
w miejscu parametru formalnego, kompilator generuje \emph{thunk}, który oblicza
wartość wyrażenia i pozostawia adres tego wyniku w pewnej standardowej lokalizacji.

\dots

Microsoft i IBM zdefiniowali w ich systemach, opartych na Intelu, “środowisko 16-bitowe“
(z odrażającymi rejestrami segmentowymi i 64k limitem adresów) i “środowisko 32-bitowe“
(z płaskim adresowaniem i półrzeczywistym trybem zarządzania pamięcią). Te dwa środowiska mogą
działać równocześnie na tym samym komputerze i systemie operacyjnym (dzięki temu, co w świecie Microsoftu znane jest jako WOW, co jest skrótowcem od Windows On Windows). MS i IBM zdecydowali, że proces przechodzenia
z trybu 16-bitowego do 32-bitowego (i odwrotnie), nazwany zostanie “thunk”; istnieje nawet
narzędzie na system Windows 95, "THUNK.EXE", nazwane "thunk kompilatorem".
\end{quotation}
\end{framed}
% TODO FIXME move to bibliography and quote properly above the quote
( \href{http://go.yurichev.com/17362}{The Jargon File} )

\myindex{LAPACK}
\myindex{FORTRAN}
Jeszcze jeden przykład możemy znaleźć w bibliotece LAPACK --- (``Linear Algebra PACKage'') napisanej w języku FORTRAN.
Deweloperzy \CCpp również chcą korzystać z LAPACK, ale przepisywanie jej na \CCpp, a następnie utrzymywanie kilku wersji byłoby szaleństwem.
Istnieją wobec tego krótkie funkcje w C, które są wywoływane ze środowiska \CCpp{}, które z kolei wywołują funkcje FORTRAN i prawie nic oprócz tego nie robią:

\begin{lstlisting}[style=customc]
double Blas_Dot_Prod(const LaVectorDouble &dx, const LaVectorDouble &dy)
{
    assert(dx.size()==dy.size());
    integer n = dx.size();
    integer incx = dx.inc(), incy = dy.inc();

    return F77NAME(ddot)(&n, &dx(0), &incx, &dy(0), &incy);
}
\end{lstlisting}

Takie funkcje również są nazywane "wrapperami".



\subsubsection{ARM64}

\myparagraph{GCC}

Skompilujmy przykład w GCC 4.8.1 dla ARM64:

\lstinputlisting[numbers=left,label=hw_ARM64_GCC,caption=\NonOptimizing GCC 4.8.1 + objdump,style=customasmARM]{patterns/01_helloworld/ARM/hw.lst}

W ARM64 nie ma trybów Thumb i Thumb-2, tylko ARM, dlatego tu korzystamy tylko z 32-bitowych instrukcji.

Jest tu dwa razy więcej rejestrów: \myref{ARM64_GPRs}.
64-bitowe rejestry mają prefiks 
\TT{X-}, a ich 32-bitowe części --- \TT{W-}.

\myindex{ARM!\Instructions!STP}
Instrukcja \TT{STP} (\emph{Store Pair}) 
odkłada na stos od razu 2 rejestry: \RegX{29} i \RegX{30}.
Oczywiście, ta instrukcja może zapisać tę parę rejestrów gdziekolwiek w pamięci, ale tu jest wskazany rejestr \ac{SP}, także ta para jest odkładana na stos.

Rejestry w ARM64 są 64-bitowe, każdy o długości 8 bajtów, dlatego do przechowywania 2 rejestrów potrzeba 16 bajtów.

Wykrzyknik (``!'') po operandzie oznacza, że najpierw od \ac{SP} będzie odjęte 16 i dopiero po tej czynności wartości z obu rejestrów będą odłożone na stos.

Jest to nazywane \emph{pre-index}.
Więcej o róznicy między \emph{post-index} a \emph{pre-index} 
można znaleźć tu: \myref{ARM_postindex_vs_preindex}.

W ten sposób, posługując się terminologią x86, pierwsza instrukcja~--- jest analogiczna do \TT{PUSH X29} i \TT{PUSH X30}.
\RegX{29} w ARM64 jest wykorzystywane jako \ac{FP}, a \RegX{30} 
jako \ac{LR}, dlatego są one zapisywane w prologu funkcji.

Druga instrukcja kopiuje \ac{SP} do \RegX{29} (lub \ac{FP}).
Jest to niezbędne do ustawienia stack frame funkcji.

\label{pointers_ADRP_and_ADD}
\myindex{ARM!\Instructions!ADRP/ADD pair}
Instrukcje \TT{ADRP} i \ADD są potrzebne do skonstruowania adresu linii \q{Hello!} w rejestrze \RegX{0}, 
jako że pierwszy argument f-cji jest przekazywany przez ten rejestr.
Jednakże w ARM nie ma instrukcji, za pomocą których można zapisać do rejestru dużą liczbę 
(dlatego że długość instrukcji wynosi maksymalnie 4 bajty. Więcej informacji o tym można znaleźć tutaj: \myref{ARM_big_constants_loading}).
Dlatego trzeba skorzystać z kilku instrukcji.
Pierwsza instrukcja (\TT{ADRP}) zapisuje do \RegX{0} adres 4-kB strony, na której się znajduje linia, 
a druga (\ADD) dodaje do tego adresu resztę.
Więcej o tym tutaj: \myref{ARM64_relocs}.

\TT{0x400000 + 0x648 = 0x400648}, i możemy zobaczyć, że w segmencie danych \TT{.rodata} pod tym adresem znajduje się nasza
linia \q{Hello!}.

\myindex{ARM!\Instructions!BL}
Następnie, za pomocą instrukcji \TT{BL} jest wywoływane \puts. Zostało to omówione wcześniej: \myref{puts}.

Instrukcja \MOV zapisuje 0 do \RegW{0}. 
\RegW{0} to młodsze 32 bity 64-bitowego rejestru \RegX{0}:

\begin{center}
\begin{tabular}{ | l | l | }
\hline
	Starsze 32 bity & Młodsze 32 bity \\
\hline
\multicolumn{2}{ | c | }{X0} \\
\hline
\multicolumn{1}{ | c | }{} & \multicolumn{1}{ c | }{W0} \\
\hline
\end{tabular}
\end{center}


Wynik funkcji jest zwracany przez \RegX{0}, i \main zwraca 0.

Dlaczego 32-bitowa część?
Dlatego że w ARM64, jak i w x86-64, typ \Tint zostawili 32-bitowym, dla kompatybilności.

Odpowiednio, jako że funkcja zwraca 32-bitowy \Tint, to trzeba wypełnić tylko młodsze 32 bity 64-bitowego rejestru \RegX{0}.

Żeby mieć pewność, trochę zmienimy przykład i skompilujemy go ponownie.%

Teraz \main zwraca 64-bitową wartość:

\begin{lstlisting}[caption=\main zwracające wartość typu \TT{uint64\_t},style=customc]
#include <stdio.h>
#include <stdint.h>

uint64_t main()
{
        printf ("Hello!\n");
        return 0;
}
\end{lstlisting}

Wynik jest taki sam, tylko \MOV w tej linii wygląda teraz w ten sposób:

\begin{lstlisting}[caption=\NonOptimizing GCC 4.8.1 + objdump]
  4005a4:       d2800000        mov     x0, #0x0      // #0
\end{lstlisting}

\myindex{ARM!\Instructions!LDP}
Następnie za pomocą instrukcji \INS{LDP} (\emph{Load Pair}) są przywracane rejestry \RegX{29} i \RegX{30}.

Wykrzyknika po instrukcji brak. To oznacza, że najpierw wartości są zdejmowane ze stosu, i dopiero po tej czynności \ac{SP} jest zwiększane o 16.

Jest to nazywane \emph{post-index}.

\myindex{ARM!\Instructions!RET}
W ARM64 pojawia się nowa instrukcja: \RET. 
Ona działa tak samo jak i \INS{BX LR}, ale zawiera bit ,
który podpowiada procesorowi, że jest to wyjście z f-cji, a nie zwykłe przejście, żeby procesor mógł zoptymalizować tę instrukcję.

Jako że ta funkcja jest bardzo prosta, optymalizujący GCC generuje dokładnie taki sam kod.





