\subsubsection{\NonOptimizingKeilVI (\ARMMode)}

Na początek skompilujmy nasz przykład za pomocą Keil:

\begin{lstlisting}
armcc.exe --arm --c90 -O0 1.c 
\end{lstlisting}

\myindex{\IntelSyntax}
Kompilator \emph{armcc} generuje listing w asemblerze w składni Intela.
Listing zawiera niektóre wysokopoziomowe makra, związane z ARM
\footnote{na przykład listing zawiera instrukcję \PUSH/\POP, których nie ma w trybie ARM}.
Nas interesują prawdziwe instrukcje, dlatego zobaczmy jak wygląda skompilowany kod w programie \IDA.

\begin{lstlisting}[caption=\NonOptimizingKeilVI (\ARMMode) \IDA,style=customasmARM]
.text:00000000             main
.text:00000000 10 40 2D E9    STMFD   SP!, {R4,LR}
.text:00000004 1E 0E 8F E2    ADR     R0, aHelloWorld ; "hello, world"
.text:00000008 15 19 00 EB    BL      __2printf
.text:0000000C 00 00 A0 E3    MOV     R0, #0
.text:00000010 10 80 BD E8    LDMFD   SP!, {R4,PC}

.text:000001EC 68 65 6C 6C+aHelloWorld  DCB "hello, world",0    ; DATA XREF: main+4
\end{lstlisting}

Widać, że każda instrukcja ma rozmiar 4 bajtów ~--- zgodnie z oczekiwaniami, ponieważ kompilowaliśmy nasz kod dla trybu ARM, a nie Thumb.

\myindex{ARM!\Instructions!STMFD}
\myindex{ARM!\Instructions!POP}
Pierwsza instrukcja, \INS{STMFD SP!, \{R4,LR\}}\footnote{\ac{STMFD}},
działa podobnie jak instrukcja \PUSH w x86: odkłada wartości dwóch rejestrów (\Reg{4} i \ac{LR}) na stos.

W rzeczy samej, kompilator \emph{armcc}, generując listing, wstawił tam dla uproszczenia instrukcję \INS{PUSH \{r4,lr\}}.
Nie jest to do końca precyzyjne, ponieważ instrukcja \PUSH dostępna jest w trybie Thumb.
By uniknąć dezorientacji, wygenerowany kod maszynowy podglądamy w programie \IDA.

Instrukcja najpierw zmniejsza \ac{SP}, by wskazywał na miejsce na stosie dostępne do zapisu nowych wartości, następnie zapisuje wartości rejestrów \Reg{4} i \ac{LR}
pod adres w pamięci, na który wskazuje zmodyfikowany rejestr \ac{SP}.

Podobnie jak \PUSH w trybie Thumb, ta instrukcja pozwala na odkładanie na stos wartości kilku rejestrów na raz, co może być bardzo wygodne.

Przy okazji, takie zachowanie nie ma swojego odpowiednika w x86.

Można zauważyć, że \TT{STMFD} jest generalizacją instrukcji \PUSH (czyli rozszerza jej możliwości), dlatego że może operować na różnych rejestrach, a nie tylko na \ac{SP}.
Inaczej mówiąc, z \TT{STMFD} można korzystać przy zapisie wartości kilku rejestrów we wskazane miejsce w pamięci.

\myindex{\PICcode}
\myindex{ARM!\Instructions!ADR}
Instrukcja \INS{ADR R0, aHelloWorld} dodaje/odejmuje wartość w rejestrze \ac{PC} (R0) do/od przesunięcia, w którym jest przechowywany łańcuch znaków
\TT{hello, world}.
Dlaczego użyto tutaj rejestru \ac{PC}? Jest to tzw. \q{\PICcode}
\footnote{Jest to szerzej omówione w kolejnym rozdziale ~(\myref{sec:PIC})}.

Taki kod można uruchomić z dowolnego miejsca w pamięci.
Inaczej mówiąc jest to adresowanie względne, względem rejestru \ac{PC}.
W kodzie operacji (opcode) instrukcji \TT{ADR} jest zapisane przesunięcie (offset) między adresem tej instrukcji a adresem łańcucha znaków.
Przesunięcie zawsze jest stałe, niezależnie od tego, w które miejsce \ac{OS} załadował nasz kod.
Dlatego, wszystko czego potrzebujemy~--- to dodanie adresu bieżącej instrukcji (z \ac{PC}), żeby otrzymać adres bezwględny łańcucha znaków.

\myindex{ARM!\Registers!Link Register}
\myindex{ARM!\Instructions!BL}
Instrukcja \INS{BL \_\_2printf}\footnote{Branch with Link} wywołuje funkcję \printf.
Działanie tej instrukcji przebiega w 2 krokach:

\begin{itemize}
\item zapisz adres występujący po instrukcji \INS{BL} (\TT{0xC}) do rejestru \ac{LR},
\item przekaż sterowanie do funkcji \printf, zapisując jej adres do rejestru \ac{PC}.
\end{itemize}

Kiedy funkcja \printf zakończy działanie, musi wiedzieć gdzie zwrócić sterowanie. Dlatego każda funkcja, kończąc pracę, zwraca sterowanie pod adres zapisany w rejestrze \ac{LR}.

Na tym polega główna różnica między \q{czystymi} procesorami \ac{RISC}, jak ARM, i procesorami \ac{CISC} w rodzaju x86,
gdzie adres powrotu zwykle jest odkładany na stos. Przeczytasz o tym więcej w kolejnym rozdziale ~(\myref{sec:stack}).

Dodatkowo nie jest możliwe zakodowanie 32-bitowego adresu bezwzględnego (lub przesunięcia) w 32-bitowej instrukcji \INS{BL}, ponieważ ma ona miejsce tylko dla 24 bitów.
Jak zapewne pamiętasz, wszystkie instrukcje w trybie ARM mają długość 4 bajtów (32 bitów) i mogą się znajdować tylko pod adresem wyrównanym do krotności 4 bajtów.
Oznacza to, że ostatnich 2 bitów (które są zawsze zerowe) można nie kodować.
Ostatecznie zostaje nam 26 bitów na zakodowanie przesunięcia. Odpowiada to zakresowi $current\_PC \pm{} \approx{}32M$.

\myindex{ARM!\Instructions!MOV}
Następna instrukcja \INS{MOV R0, \#0}\footnote{oznacza MOV}
po prostu zapisuje 0 do rejestru \Reg{0}.
To dlatego, że nasza funkcja zwraca 0, a wartości zwracane z funkcji zapisywane są do \Reg{0}.

\myindex{ARM!\Registers!Link Register}
\myindex{ARM!\Instructions!LDMFD}
\myindex{ARM!\Instructions!POP}
Ostatnia instrukcja to \INS{LDMFD SP!, {R4,PC}}\footnote{\ac{LDMFD} jest instrukcją odwrotną do \ac{STMFD}}.
Pobiera ona wartość ze stosu (lub z pamięci), zapisuje do \Reg{4} i \ac{PC} oraz zwiększa \glslink{stack pointer}{wskaźnik stosu} \ac{SP}. Działa podobnie jak instrukcja \POP.\\
Notabene pierwsza instrukcja \TT{STMFD} odłożyła na stos wartości z rejestrów \Reg{4} i \ac{LR}, ale teraz zostały one przywrócone do \Reg{4} i \ac{PC}, dzięki instrukcji \TT{LDMFD}.

Jak już wiemy, rejestr \ac{LR} zawiera adres w pamięci, pod który funkcja zwróci sterowanie po zakończeniu swojej pracy.
Pierwsza instrukcja odkłada tę wartość na stos, ponieważ ten sam rejestr zostanie wykorzystany przez naszą funkcję \main, gdy wywoła ona funkcję \printf.

Na końcu funkcji ta wartość zapisywana jest do rejestru \ac{PC}, w ten sposób przekazując sterowanie tam, skąd została wywołana.

Z reguły funkcja \main jest funkcją główną w \CCpp, więc zarządzanie zostanie zwrócone do loadera w \ac{OS}, lub gdzieś do \ac{CRT}, lub w jeszcze inne, podobne, miejsce.

Wszystko to pozwala pozbyć się ręcznego wywoływania \INS{BX LR} (skok pod adres z rejestru \ac{LR}) na samym końcu funcji.

\myindex{ARM!DCB}
\TT{DCB}~--- dyrektywa asemblera, opisująca tablicę bajtów bądź ciąg znaków ASCII, podobna do dyrektywy DB w x86


