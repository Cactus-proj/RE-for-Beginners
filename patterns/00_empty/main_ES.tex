\mysection{Una Función Vacía}
\label{empty_func}

La función más simple es posiblemente una que no hace nada:

\lstinputlisting[label=lst:empty_func,caption=\ESph{},style=customc]{patterns/00_empty/1.c}

!Vamos a compilarla!

\subsection{x86}

Esto es lo que producen los compiladores GCC y MSVC en la plataforma x86:

\lstinputlisting[caption=\Optimizing GCC/MSVC (\assemblyOutput),style=customasmx86]{patterns/00_empty/1.s}

\myindex{x86!\Instructions!RET}
Solo hay una instrucción: \RET, que devuelve la ejecución al \gls {llamante}.

\subsection{ARM}

\lstinputlisting[caption=\OptimizingKeilVI (\ARMMode) \assemblyOutput,style=customasmARM]{patterns/00_empty/1_Keil_ARM_O3.s}

La dirección de retorno no se guarda en la pila local en el \ac{ISA} de ARM, sino en el registro del enlace, por lo que la instrucción \INS {BX LR} hace que la ejecución salte a esa dirección --- devolviendo efectivamente la ejecución al \gls{llamante}.

\subsection{MIPS}

Hay dos convenciones de denominación utilizadas en el mundo de MIPS al nombrar registros:
por número (de \$0 a \$31) o por pseudo nombre (\$V0, \$A0, etc.).

El resultado de ensamblaje de GCC se muestra a continuación y enumera los registros por número:

\lstinputlisting[caption=\Optimizing GCC 4.4.5 (\assemblyOutput),style=customasmMIPS]{patterns/00_empty/MIPS.s}

\dots, mientras que \IDA lo hace por su  pseudonimo:

\lstinputlisting[caption=\Optimizing GCC 4.4.5 (IDA),style=customasmMIPS]{patterns/00_empty/MIPS_IDA.lst}

\myindex{¡MIPS!\Instructions!J}
La primera instrucción es la instrucción de salto (J o JR) que devuelve el flujo de ejecución al \gls {llamante},
saltando a la dirección en el registro \$ 31 (o \$ RA).

Este es el registro análogo a \ac{LR} en ARM.

La segunda instrucción es \ac{NOP}, que no hace nada.
Podemos ignorarlo por ahora.

\subsubsection{Una nota sobre las instrucciones de MIPS y nombres de registro}

Los nombres de registros e instrucciones en el mundo de MIPS se escriben tradicionalmente en minúsculas.
Sin embargo, en aras de la coherencia, este libro se limitará a usar letras mayúsculas,
ya que es la convención seguida por todos los demás \ac {ISA} presentados en este libro.

\subsection{Funciones vacías en la práctica}

A pesar de que las funciones vacías parecen inútiles, son bastante frecuentes en el código de bajo nivel.

En primer lugar, son bastante populares en las funciones de depuración, como la siguiente:

\lstinputlisting[caption=\CCpp code,style=customc]{patterns/00_empty/dbg_print_EN.c}

En una compilación sin depuración (como en una ``versión (release)''), \TT{\_DEBUG} no está definido,
por lo que la función \TT{dbg\_print()}, a pesar de ser llamada durante la ejecución,
estará vacío

Del mismo modo, un método popular de protección de software es crear una compilación para clientes legales y otra compilación de demostración.
Una compilación de demostración puede carecer de algunas funciones importantes, como en este ejemplo:

\lstinputlisting[caption=\CCpp code,style=customc]{patterns/00_empty/demo_EN.c}

Se puede invocar a la función \TT{save\_file ()} cuando el usuario hace clic en \TT{File->Save} en el menú.
La versión demo puede entregarse con este elemento de menú deshabilitado, pero incluso si un cracker de software lo habilita,
solo se llamará una función vacía sin código útil.

IDA marca tales funciones con nombres como \TT{nullsub\_00}, \TT{nullsub\_01}, etc.
