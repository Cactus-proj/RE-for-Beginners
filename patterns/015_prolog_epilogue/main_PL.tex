\mysection{Prolog i epilog funkcji}
\label{sec:prologepilog}
\myindex{Function epilogue}
\myindex{Function prologue}

Prolog funkcji to sekwencja instrukcji rozpoczynająca funkcję. Zwykle przypomina poniższy fragment kodu:

\begin{lstlisting}[style=customasmx86]
    push    ebp
    mov     ebp, esp
    sub     esp, X
\end{lstlisting}

Co te instrukcje robią: zapisują wartość rejestru  \EBP na stosie,
ustawiają wartość w rejestrze \EBP na wartość z \ESP a następnie alokuję miejsce na stosie na zmienne lokalne

Wartość w \EBP pozostaje niezmieniona przez całą funkcję i używana jest przy dostępie do zmiennych lokalnych i argumentów.
Do tego samego celu można by użyć \ESP, ale byłoby to niewygodne, gdyż wartość \ESP zmienia się w czasie wykonywania funkcji.

Epilog funkcji zwalnia zaalokowane miejsce na stosie, przywraca wartość rejestru \EBP do jego pierwotnego stanu
i zwraca sterowanie do \glslink{caller}{funkcji wywołującej}:

\begin{lstlisting}[style=customasmx86]
    mov    esp, ebp
    pop    ebp
    ret    0
\end{lstlisting}

% what about calling convention?
Prolog i epilog zwykle jest wykrywany przez deasemblery i używany do wyodrębnienia pojedynczych funkcji.

\subsection{\Recursion}

\myindex{\Recursion}
Prolog i epilog funkcji mają negatywny wpływ na wywołania rekurencyjne.

Więcej o rekurencji: \myref{Recursion_and_tail_call}.

