\myparagraph{32-bit ARM}

\mysubparagraph{\NonOptimizingXcodeIV (\ARMMode)}

\lstinputlisting[caption=\NonOptimizingXcodeIV (\ARMMode),label=ARM_leaf_example7,style=customasmARM]{patterns/10_strings/1_strlen/ARM/xcode_ARM_O0_IT.asm}

LLVM non ottimizzata genera troppo codice, tuttavia, qui possiamo vedere come la funzione lavora con 
le variabili locali nello stack.
Ci sono solo due variabili locali nella nostra funzione: \emph{eos} e \emph{str}.
In questo listato , generato da \IDA, abbiamo rinominato manualmente \emph{var\_8} e \emph{var\_4} in \emph{eos} e \emph{str}.

Le prime istruzioni, salvano solamente i valori di input in entrambe \emph{str} e \emph{eos}.

Il corpo del ciclo inizia a \emph{loc\_2CB8}.

Le prime tre istruzioni nel corpo del ciclo(\TT{LDR}, \ADD, \TT{STR}) caricano il valore di \emph{eos} in \Reg{0}. 
Dopodichè il valore viene \glslink{increment}{incrementato} e risalvato in \emph{eos}, che è situata nello stack.

\myindex{ARM!\Instructions!LDRSB}
L' istruzione successiva,  \TT{LDRSB R0, [R0]} (\q{Load Register Signed Byte}), carica un byte dalla memoria all' indirizzo salvato in \Reg{0} e lo extende con segno in 32-bit
\footnote{Il compilatore Keil tratta il tipo \Tchar come con segno, proprio come MSVC e GCC.}.
\myindex{x86!\Instructions!MOVSX}
Ciò è simile all' istruizione \MOVSX in x86.

Il compilatore tratta questo byte come con segno da quando il tipo \Tchar è con segno in accordo con lo standard C.
Riguardo a ciò, se ne è già parlato nella sezione~(\myref{MOVSX}), in relazione a x86.

\myindex{Intel!8086}
\myindex{Intel!8080}
\myindex{ARM}

E' da notare che in ARM è impossibile utilizzare una parte a 8- o 16-bit  
di un registro a 32-bit separatamente dall' intero registro,
come in x86.

Apparentemente, è perché x86 ha un'enorme storia di retrocompatibilità con i suoi antenati
 fino all'8086 a 16 bit e persino all'8080 a 8 bit,
 ma ARM è stato sviluppato da zero come un processore RISC a 32 bit.

Di conseguenza, per elaborare byte separati in ARM, è necessario utilizzare comunque i registri a 32 bit.

Quindi, \TT{LDRSB} carica i byte dalla stringa a \Reg{0}, uno alla volta.
Le seguenti istruzioni \CMP e \ac{BEQ}, controllano se il byte caricato è 0.
Se non è 0, il controllo passa a inizio del corpo del ciclo.
E se è, il ciclo termina.

Alla fune della funzione, la differenza tra 
\emph{eos} e \emph{str} è calcolata, viene sottrato 1 da esso, e il valore risultante iene ritornato
attraverso \Reg{0}.

N.B. I registri non sono stati salavati in questa funzione.
\myindex{ARM!\Registers!scratch registers}

Questo perchè nella covenzione a chiamata ARM, i registri \Reg{0}-\Reg{3} sono \q{scratch registers}, 
utilizati per il passaggio di argomenti,
e non ci è richiesto di ripristinare il loro valore quando la funzione esce, 
in quanto la funzione chiamante non gli userà più.
Di conseguenza, possono essere utilizzati per ciò che vogliamo.

Nessun altro registro viene usato qui, quindi è per questo che non abbiamo nulla da salvare nello stack.

Così, il controllo può essere ritornato alla funzione chiamante con un semplice salto (\TT{BX}),
all' indirizzo nel registro \ac{LR}.

\mysubparagraph{\OptimizingXcodeIV (\ThumbMode)}

\lstinputlisting[caption=\OptimizingXcodeIV (\ThumbMode),style=customasmARM]{patterns/10_strings/1_strlen/ARM/xcode_thumb_O3.asm}

Come conclude LLVM ottimizzata, \emph{eos} e \emph{str} non necessitano spazio nello stack, e possono sempre essere salvate nei registri.

Prima dell' inizio del corpo del ciclo, \emph{str} è sempre in \Reg{0}, 
e \emph{eos}---in \Reg{1}.

\myindex{ARM!\Instructions!LDRB.W}
L' istruzione \TT{LDRB.W R2, [R1],\#1}, carica un byte dalla memoria all' indirizzo salvato in \Reg{1}, in \Reg{2}, estendendo con segno ad un valore 32-bit, ma non solo.
\TT{\#1} alla fine dell' istruzione è implicito \q{Post-indexed addressing}, che significa cje 1 verrà aggiunto a \Reg{1} dopo che il byte è caricato.
Approfondimento: \myref{ARM_postindex_vs_preindex}.

Come potete vedere \CMP e \ac{BNE} sono nel corpo del ciclo, queste istruzioni contineranno a ciclare fino a che 0 non verrà trovato nella stringa.

\myindex{ARM!\Instructions!MVNS}
\myindex{x86!\Instructions!NOT}
Le istruzioni \TT{MVNS}\footnote{MoVe Not} (inverte tutti i bit, come \NOT in x86) e \ADD calcolano $eos - str - 1$.
Infatti, queste due istruzioni eseguono $R0 = ~str + eos$, 
che è effettivamente equivalente a quello che c' era nel codice sorgente, ed il motivo, è stato già spiegato qui
~(\myref{strlen_NOT_ADD}).

Apparentemente, LLVM, proprio come GCC, ha concluso che questo codice può essere più corto (o veloce).

\mysubparagraph{\OptimizingKeilVI (\ARMMode)}

\lstinputlisting[caption=\OptimizingKeilVI (\ARMMode),label=ARM_leaf_example6,style=customasmARM]{patterns/10_strings/1_strlen/ARM/Keil_ARM_O3.asm}

\myindex{ARM!\Instructions!SUBEQ}

Quasi lo stesso di ciò che abbiamo visto prima, con la differenza che l' espressione $str - eos - 1$ 
può non essere calcolata alla fine della funzione, ma direttamente nel corpo del ciclo.
Il suffisso \TT{-EQ}, come possiamo ricordare, implica che l' istruzione viene eseguita solo se gli operandi in
 \CMP, che sono stati eseguiti prima, erano uguali tra loro.
Pertanto, se \Reg{0} contiene 0, entrambe le istruzioni \TT{SUBEQ} vengono eseguite e il risultato viene lasciato nel registro \Reg{0}.
