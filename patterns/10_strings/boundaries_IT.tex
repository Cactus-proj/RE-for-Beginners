\subsection{Delimitazione delle stringhe}

\myindex{win32!GetOpenFileName}
E' interessante notare, come i parametri vengono passati alla funzione win32 \emph{GetOpenFileName()}.
Per chiamarlo, è necessario impostare un elenco di estensioni di file consentite:

\begin{lstlisting}[style=customc]
	OPENFILENAME *LPOPENFILENAME;
	...
	char * filter = "Text files (*.txt)\0*.txt\0MS Word files (*.doc)\0*.doc\0\0";
	...
	LPOPENFILENAME = (OPENFILENAME *)malloc(sizeof(OPENFILENAME));
	...
	LPOPENFILENAME->lpstrFilter = filter;
	...

	if(GetOpenFileName(LPOPENFILENAME))
	{
		...
\end{lstlisting}

Ciò che accade qui, è che la lista di stringhe viene passata a \emph{GetOpenFileName()}.
Non è un problema analizzarla: ogni volta che si incontra un singolo zero byte, questo è un elemento.
Ogni qualvolta si incontrano due zero byte, si è alla fine della lista.
Se passassimo la stringa a \printf, tratterebbe il primo oggetto come una singola stringa.

Quindi, questa è una stringa, oppure...?
Sarebbe meglio dire che questo un buffer contenente diverse stringhe C zero terminate, le quali possono essere salvate e processate
nel loro insieme.

\myindex{\CStandardLibrary!strtok}
Un altro esempio è la funzione \emph{strtok()}. Essa prende una stringa e scrive dei byte zero al suo interno.
Trasforma quindi la stringa di input in un qualche tipo di buffer, che ha diverse stringhe C con lo zero terminatore.
