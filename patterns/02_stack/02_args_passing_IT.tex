\subsubsection{Passaggio di argomenti alle funzioni}

Il modo più diffuso per passare parametri in x86 è detto \q{cdecl}:

\begin{lstlisting}[style=customasmx86]
push arg3
push arg2
push arg1
call f
add esp, 12 ; 4*3=12
\end{lstlisting}

La funzioni chiamate, \Gls{callee}, ricevono i propri argomenti tramite lo stack pointer.

Quindi è così che i valori degli argomenti sono posizionati nello stack prima dell'esecuzione della prima istruzione della funzione \ttf{}:

\begin{center}
\begin{tabular}{ | l | l | }
\hline
ESP & return address \\
\hline
ESP+4 & \argument \#1, \MarkedInIDAAs{} \TT{arg\_0} \\
\hline
ESP+8 & \argument \#2, \MarkedInIDAAs{} \TT{arg\_4} \\
\hline
ESP+0xC & \argument \#3, \MarkedInIDAAs{} \TT{arg\_8} \\
\hline
\dots & \dots \\
\hline
\end{tabular}
\end{center}

Per ulteriori informazioni su altri tipi di convenzioni di chiamata (calling conventions), fare riferimento alla sezione~(\myref{sec:callingconventions}).

\par
A proposito, la funzione \gls{callee}{chiamata} non possiede alcuna informazione su quanti argomenti sono stati passati.
Le funzioni C con un numero variabile di argomenti (come \printf) determinano il loro numero attraverso specificatori di formato stringa (che iniziano con il simbolo \%).

Se scriviamo qualcosa come:

\begin{lstlisting}
printf("%d %d %d", 1234);
\end{lstlisting}

\printf scriverà 1234, e successivamente due numeri casuali\footnote{Non casuali in senso stretto, ma piuttosto non predicibili: \myref{noise_in_stack}}, che si trovavano lì vicino nello stack.

\label{main_arguments}
\par
Per questo motivo non è molto importante come dichiariamo la funzione \main: come \main, \\
\TT{main(int argc, char *argv[])} oppure \TT{main(int argc, char *argv[], char *envp[])}.

Infatti, il codice \ac{CRT} sta chiamando \main circa in questo modo:

\begin{lstlisting}[style=customasmx86]
push envp
push argv
push argc
call main
...
\end{lstlisting}

Se dichiari \main come \main senza argomenti, questi sono, in ogni caso, ancora presenti nello stack, ma non vengono utilizzati.
Se dichiari \main come \TT{main(int argc, char *argv[])},
sarai in grado di utilizzare i primi due argomenti, ed il terzo rimarrà \q{invisibile} per la tua funzione.
In più, è possibile dichiarare \TT{main(int argc)}, e continuerà a funzionare.

% TBT Another related example: \ref{cdecl_DLL}.

\myparagraph{Metodi alternativi per passare argomenti}

Vale la pena notare che non c'è nulla che obbliga il programmatore a passare gli argomenti attraverso lo stack. Non è un requisito necessario.
Si potrebbe implementare un qualunque altro metodo anche senza usare per niente lo stack.

Un metodo abbastanza popolare tra chi inizia a programmare in linguaggio assembly language è di passare argomenti attraverso variabili globali, in questo modo:

\lstinputlisting[caption=Assembly code,style=customasmx86]{patterns/02_stack/global_args.asm}

Tuttavia questo metodo ha un limite evidente: la funzione \emph{do\_something()} non può richiamare sè stessa in modo ricorsivo (o attraverso un'altra funzione),
perchè deve cancellare i suoi stessi argomenti.
Lo stesso accade con le variabili locali: se le tieni in variabili globali, la funzione non può chiamare se stessa.
Inoltre questo non sarebbe thread-safe
\footnote{Implementato correttamente, ciascun thread avrebbe il suo proprio stack con i suoi argomenti/variabili.}.
Il metodo di memorizzare queste informazioni nello stack rende il tutto più semplice---può mantenere quanti argomenti di funzione e/o valori,
quanto spazio è disponibile.

\InSqBrackets{\TAOCPvolI{}, 189} menziona alcuni schemi ancora più strani e particolarmente convenienti su IBM System/360.

\myindex{MS-DOS}
\myindex{x86!\Instructions!INT}

MS-DOS utilizzava un modo per passare tutti gli argomenti di funzione via registri, ad esempio, in questo pezzo
di codice per MS-DOS a 16 bit scrive ``Hello, world!'':

\begin{lstlisting}[style=customasmx86]
mov  dx, msg      ; indirizzo del messaggio
mov  ah, 9        ; 9 indica la funzione "print string"
int  21h          ; "syscall" (chiamata di sistema) DOS

mov  ah, 4ch      ; funzione "termina il programma"
int  21h          ; "syscall" DOS

msg  db 'Hello, World!\$'
\end{lstlisting}

\myindex{fastcall}
Questo è abbastanza simile al metodo \myref{fastcall}.
Ed è inoltre molto simile alle chiamate syscalls in Linux (\myref{linux_syscall}) e Windows.

\myindex{x86!\Flags!CF}
Se una funzione MS-DOS restituisce un valore di tipo boolean (cioè, un singolo bit, di solito per indicare uno stato di errore),
il flag \TT{CF} era spesso utilizzato.

Ad esempio:

\begin{lstlisting}[style=customasmx86]
mov ah, 3ch       ; crea file
lea dx, filename
mov cl, 1
int 21h
jc  error
mov file_handle, ax
...
error:
...
\end{lstlisting}

In caso di errore, il flag \TT{CF} viene innalzato. Altrimenti, l'handle ad un nuovo file creato viene restituito attraverso \TT{AX}.

Questo metodo viene ancora utilizzato dai programmatori assembly.
Nel codice sorgente del Windows Research Kernel (che è abbastanza simile a Windows 2003) possiamo trovare qualcosa tipo:
(file \emph{base/ntos/ke/i386/cpu.asm}):

\begin{lstlisting}[style=customasmx86]
        public  Get386Stepping
Get386Stepping  proc

        call    MultiplyTest            ; Esegue test di moltiplicazione
        jnc     short G3s00             ; se nc, muttest è ok
        mov     ax, 0
        ret
G3s00:
        call    Check386B0              ; Verifica B0 stepping
        jnc     short G3s05             ; se nc, è B1/later
        mov     ax, 100h                ; è B0/earlier stepping
        ret

G3s05:
        call    Check386D1              ; Verifica D1 stepping
        jc      short G3s10             ; se c, non è D1
        mov     ax, 301h                ; è D1/later stepping
        ret

G3s10:
        mov     ax, 101h                ; suppone che sia B1 stepping
        ret

	...

MultiplyTest    proc

        xor     cx,cx                   ; 64K volte è un bel numero tondo
mlt00:  push    cx
        call    Multiply                ; la moltiplicazione funziona in questo chip?
        pop     cx
        jc      short mltx              ; se c, No, esci
        loop    mlt00                   ; se nc, Si, cicla per riprovare
        clc
mltx:
        ret

MultiplyTest    endp
\end{lstlisting}
