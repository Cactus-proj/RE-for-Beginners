\mysection{\Stack}
\label{sec:stack}
\myindex{\Stack}

Der Stack ist eine der fundamentalen Datenstrukturen in der Informatik.
\footnote{\href{http://go.yurichev.com/17119}{wikipedia.org/wiki/Call\_Stack}}.
\ac{AKA} \ac{LIFO}.

Technisch betrachtet ist es ein Stapelspeicher innerhalb des Prozessspeichers der zusammen mit den \ESP (x86), \RSP (x64) oder dem \ac{SP} (ARM) Register als ein Zeiger in diesem Speicherblock fungiert.

\myindex{ARM!\Instructions!PUSH}
\myindex{ARM!\Instructions!POP}
\myindex{x86!\Instructions!PUSH}
\myindex{x86!\Instructions!POP}

Die häufigsten Stack-Zugriffsinstruktionen sind die \PUSH- und \POP-Instruktionen (in beidem x86 und ARM Thumb-Modus). \PUSH subtrahiert vom \ESP/\RSP/\ac{SP} 4 Byte im 32-Bit Modus (oder 8 im 64-Bit Modus) und schreibt dann den Inhalt des Zeigers an die Adresse auf die von \ESP/\RSP/\ac{SP} gezeigt wird.

\POP ist die umgekehrte Operation: Die Daten des Zeigers für die Speicherregion auf die von \ac{SP}
gezeigt wird werden ausgelesen und die Inhalte in den Instruktionsoperanden geschreiben (oft ist das ein Register). Dann werden 4 (beziehungsweise 8) Byte zum \gls{stack pointer} addiert.

Nach der Stackallokation, zeigt der \gls{stack pointer} auf den Boden des Stacks.
\PUSH verringert den \gls{stack pointer} und \POP erhöht ihn.
Der Boden des Stacks ist eigentlich der Anfang der Speicherregion die für den Stack reserviert wurde.
Das wirkt zunächst seltsam, aber so funktioniert es.

ARM unterstützt beides, aufsteigende und absteigende Stacks.

\myindex{ARM!\Instructions!STMFD}
\myindex{ARM!\Instructions!LDMFD}
\myindex{ARM!\Instructions!STMED}
\myindex{ARM!\Instructions!LDMED}
\myindex{ARM!\Instructions!STMFA}
\myindex{ARM!\Instructions!LDMFA}
\myindex{ARM!\Instructions!STMEA}
\myindex{ARM!\Instructions!LDMEA}

Zum Beispiel die \ac{STMFD}/\ac{LDMFD} und \ac{STMED}/\ac{LDMED} Instruktionen sind alle dafür gedacht mit einem absteigendem Stack zu arbeiten ( wächst nach unten, fängt mit hohen Adressen an und entwickelt sich zu niedrigeren Adressen). Die \ac{STMFA}/\ac{LDMFA} und \ac{STMEA}/\ac{LDMEA} Instruktionen sind dazu gedacht mit einem aufsteigendem Stack zu arbeiten (wächst nach oben und fängt mit niedrigeren Adressen an und wächst nach oben).

% It might be worth mentioning that STMED and STMEA write first,
% and then move the pointer, and that LDMED and LDMEA move the pointer first, and then read.
% In other words, ARM not only lets the stack grow in a non-standard direction,
% but also in a non-standard order.
% Maybe this can be in the glossary, which would explain why E stands for "empty".

\subsection{Warum wächst der Stack nach unten?}
\label{stack_grow_backwards}

Intuitiv, würden man annehmen das der Stack nach oben wächst z.B Richtung höherer Adressen, so wie bei jeder anderen Datenstruktur.

Der Grund das der Stack rückwärts wächst ist wohl historisch bedingt. Als Computer so groß waren das sie einen ganzen Raum beansprucht haben war es einfach Speicher in zwei Sektionen zu unterteilen, einen Teil für den \gls{heap} und einen Teil für den Stack. Sicher war zu dieser Zeit nicht bekannt wie groß der \gls{heap} und der Stack wachsen würden, während der Programm Laufzeit, also war die Lösung die einfachste mögliche.

\input{patterns/02_stack/stack_and_heap}

In \RitchieThompsonUNIX können wir folgendes lesen:

\begin{framed}
\begin{quotation}
Der user-core eines Programm Images wird in drei logische Segmente unterteilt. Das Programm-Text Segment beginnt bei 0 im virtuellen Adress Speicher. Während der Ausführung wird das Segment als schreibgeschützt markiert und eine einzelne Kopie des Segments wird unter allen Prozessen geteilt die das Programm ausführen. An der ersten 8K grenze über dem Programm Text Segment im Virtuellen Speicher, fängt der ``nonshared'' Bereich an, der nach Bedarf von Syscalls erweitert werden kann. Beginnend bei der höchsten Adresse im Virtuellen Speicher ist das Stack Segment, das Automatisch nach unten wächst während der Hardware Stackpointer sich ändert.
\end{quotation}
\end{framed}

Das erinnert daran wie manche Schüler Notizen zu  zwei Vorträgen in einem Notebook dokumentieren:
Notizen für den ersten Vortrag werden normal notiert, und Notizen zur zum zweiten Vortrag werden 
ans Ende des Notizbuches geschrieben, indem man das Notizbuch umdreht. Die Notizen treffen sich irgendwann
im Notizbuch aufgrund des fehlenden Freien Platzes.

% I think if we want to expand on this analogy,
% one might remember that the line number increases as as you go down a page.
% So when you decrease the address when pushing to the stack, visually,
% the stack does grow upwards.
% Of course, the problem is that in most human languages,
% just as with computers,
% we write downwards, so this direction is what makes buffer overflows so messy.

\subsection{Für was wird der Stack benutzt?}

% subsections
\subsection{Rückgabe Adresse der Funktion speichern}

\myparagraph{x86}

\myindex{x86!\Instructions!CALL}
Wenn man eine Funktion mit der \CALL Instruktion aufruft, wird die Adresse direkt nach der
\CALL Instruktion auf dem Stack gespeichert und der unbedingte jump wird ausgeführt.

\myindex{x86!\Instructions!PUSH}
\myindex{x86!\Instructions!JMP}
Die \CALL Instruktion ist äquivalent zu dem \INS{PUSH address\_after\_call / JMP operand} Instruktions paar.

\myindex{x86!\Instructions!RET}
\myindex{x86!\Instructions!POP}
\RET ruft die Rückkehr Adresse vom Stack ab und springt zu dieser~---was äquivalent zu einem \TT{POP tmp / JMP tmp} Instruktions
paar ist.

\myindex{\Stack!\MLStackOverflow}
\myindex{\Recursion}

Den Stack zum überlaufen zu bringen ist recht einfach, einfach eine 
endlos rekursive Funktion Aufrufen:


\begin{lstlisting}[style=customc]
void f()
{
	f();
};
\end{lstlisting}


MSVC 2008 hat eine Erkennung für das Problem:


\begin{lstlisting}
c:\tmp6>cl ss.cpp /Fass.asm
Microsoft (R) 32-bit C/C++ Optimizing Compiler Version 15.00.21022.08 for 80x86
Copyright (C) Microsoft Corporation.  All rights reserved.

ss.cpp
c:\tmp6\ss.cpp(4) : warning C4717: 'f' : recursive on all control paths, function will cause runtime stack overflow
\end{lstlisting}

\dots aber der Compiler erzeugt den Code trotzdem:

\lstinputlisting[style=customasmx86]{patterns/02_stack/1.asm}

\dots Auch wenn wir die Compiler Optimierungen einschalten (\TT{/0x} Option) wird der optimierte Code nicht
den Stack zum überlaufen bringen. Stattdessen wird der Code \emph{korrekt}\footnote{Ironie hier} ausgeführt: 

\lstinputlisting[style=customasmx86]{patterns/02_stack/2.asm}

GCC 4.4.1 generiert vergleichbaren Code in beiden Fällen, jedoch ohne über das Overflow Problem zu warnen.

\myparagraph{ARM}

\myindex{ARM!\Registers!Link Register}

ARM Programme benutzen den Stack um Rücksprung Adressen zu speichern, aber anders.
Wie bereits erwähnt in \q{\HelloWorldSectionName}~(\myref{sec:hw_ARM}),
wird der \ac{RA} Wert im \ac{LR} (\gls{link register}) gespeichert.
Wenn nun eine andere Funktion aufgerufen werden muss und auf das \ac{LR} Register 
zu greift, muss der aktuelle Wert im Register irgendwo gespeichert werden.

\myindex{Funktion Prologe}
Normal wird der Wert im Funktion Prolog gespeichert.

\myindex{ARM!\Instructions!PUSH}
\myindex{ARM!\Instructions!POP}

Oft sieht man Instruktionen wie z.B \INS{PUSH {R4-R7,LR}} zusammen mit dieser Instruktion im 
Epilog \INS{POP {R4-R7,PC}}---Somit werden Werte die in den Funktionen benötigt werden auf dem 
Stack gespeichert, inklusive \ac{LR}.

\myindex{ARM!Leaf Funktion}
Wenn eine Funktion nie eine andere Funktion aufruft, nennt man das in der \ac{RISC} Terminologie eine
\emph{\glslink{leaf function}{leaf Funktion}}\footnote{\href{http://infocenter.arm.com/help/index.jsp?topic=/com.arm.doc.faqs/ka13785.html}{infocenter.arm.com/help/index.jsp?topic=/com.arm.doc.faqs/ka13785.html}}.  % <-- attention could be a compilier bug
Als Konsequenz ergibt sich, das leaf Funktionen nicht das \ac{LR} Register speichern (da sie es nicht modifizieren).
Wenn solche Funktionen klein sind und nur eine geringe Anzahl an Registern benutzt, ist es möglich das der Stack
gar nicht benutzt wird. Es ist also möglich leaf Funktionen zu benutzen ohne den Stack zurück zu greifen, die Ausführung
ist hier schneller als auf älteren x86 Maschinen weil kein externer RAM für den Stack benutzt wird 
\footnote{Bis vor einer weile war es sehr teuer auf PDP-11 und VAX Maschinen die CALL Instruktion zu benutzen; bis zu 50\%
der Rechenzeit wurde allein für diese Instruktion verschwendet, man hat dabei festgestellt das eine große Anzahl an kleinen
Funktionen zu haben ein \gls{anti-pattern} \InSqBrackets{\TAOUP Chapter 4, Part II}.} ist.
Diese Eigenschaft kann nützlich sein wenn der Speicher für den Stack noch nicht alloziert oder verfügbar ist.

Ein paar Beispiele für leaf Funktionen:

\myref{ARM_leaf_example1}, \myref{ARM_leaf_example2}, 
\myref{ARM_leaf_example3}, \myref{ARM_leaf_example4}, \myref{ARM_leaf_example5},
\myref{ARM_leaf_example6}, \myref{ARM_leaf_example7}, \myref{ARM_leaf_example10}.


\subsubsection{Funktion Argumente übergeben}

Der übliche weg Argumente in x86 zu übergeben ist die \q{cdecl} Methode:

\begin{lstlisting}[style=customasmx86]
push arg3
push arg2
push arg1
call f
add esp, 12 ; 4*3=12
\end{lstlisting}

Die \Gls{callee} Funktionen bekommen ihre Argumente über den Stackpointer. 

So werden die Argumente auf dem Stack gefunden, noch vor der Ausführung der ersten Instruktion der \ttf{} Funktion:

\begin{center}
\begin{tabular}{ | l | l | }
\hline
ESP & return address \\
\hline
ESP+4 & \argument \#1, \MarkedInIDAAs{} \TT{arg\_0} \\
\hline
ESP+8 & \argument \#2, \MarkedInIDAAs{} \TT{arg\_4} \\
\hline
ESP+0xC & \argument \#3, \MarkedInIDAAs{} \TT{arg\_8} \\
\hline
\dots & \dots \\
\hline
\end{tabular}
\end{center}


Für mehr Informationen über andere Aufrufs Konventionen siehe Sektion:~(\myref{sec:callingconventions}).

\par
Übrigens, die \gls{callee} Funktion hat keine Informationen wie viele Argumente übergeben wurden.
C Funktionen mit einer variablen Anzahl an Argumenten (wie z.B \printf) errechnen die zahl der Argumente anhand der 
Formatstring spezifizier-er (alle spezifizier-er die mit dem \% beginnen).
% to be sync: C functions with a variable number of arguments (like \printf) can determine their number using format string specifiers (which begin with the \% symbol).

Wenn wir etwas schreiben wie z.B:

\begin{lstlisting}
printf("%d %d %d", 1234);
\end{lstlisting}

\printf wird die Zahlen 1234 und zwei zufällige Werte ausgeben, welche direkt neben 1234 auf
dem Stack lagen\footnote{Nicht zufällig im eigentlichen Sinne sondern eher unvorhersehbar: \myref{noise_in_stack}}.

\label{main_arguments}
\par
Das ist auch der Grund warum es nicht wichtig ist wie die \main Funktion definiert ist: Als \main, \\
\TT{main(int argc, char *argv[])} oder \TT{main(int argc, char *argv[], char *envp[])}.

Tatsächlich ruf der \ac{CRT}-Code die \main Funktion um Grunde so auf:
	
\begin{lstlisting}[style=customasmx86]
push envp
push argv
push argc
call main
...
\end{lstlisting}

Wenn man \main als \main Funktion ohne Argumente definiert, dann liegen sie trotzdem auf dem Stack auch wenn sie 
nicht benutzt werden. Wenn man \main als \TT{main(int argc, char *argv[])}, definiert kann man auf die ersten beiden
Argumente der Funktion zugreifen, das dritte bleibt aber weiterhin ``Unsichtbar'' für andere Funktionen.
Es ist aber auch u.a möglich die Main Funktion als \TT{main(int argc)} schreiben und sie wird noch immer funktionieren.

% TBT Another related example: \ref{cdecl_DLL}.

\myparagraph{Alternative Wege Argumente zu übergeben}

Es sollte bemerkt werden das nichts einen Programmierer dazu zwingt Argumente über den Stack zu übergeben. Das ist
keine generelle Anforderung. Jemand könnte auch einfach eine andere Methode implementieren ohne den Stack überhaupt zu benutzen.

Ein ziemlich beliebter Weg Argumente zu übergeben unter Assembler Neulingen ist über globale Variablen wie z.B:

\lstinputlisting[caption=Assembly code,style=customasmx86]{patterns/02_stack/global_args.asm}

Aber diese Methode hat Nachteile: Die \emph{do\_something()} Funktion kann sich selbst nicht rekursiv aufrufen (aber auch keine andere Funktion),
weil sie ihre eigenen Argumente löschen muss.
Die gleiche Geschichte mit lokalen Variablen: Wenn die Werte in globalen Variablen gespeichert sind, kann die Funktion sich nicht selbst aufrufen.
Und das bedeutet wiederum das die Funktion nicht thread-Safe ist.
\footnote{Korrekt implementiert, hat jeder Thread seinen eigenen Stack und seine eigenen Argumente/Variablen}.
Eine Methode solche Informationen auf dem Stack zu speichern macht die Dinge einfacher--- Der Stack kann so viele Funktion Arguemente und/oder Werte speichern,
so viel Speicher wie der Computer hat.

\InSqBrackets{\TAOCPvolI{}, 189} nennt sogar noch verrückter Methoden die speziell auf IBM System/360 benutzt werden.

\myindex{MS-DOS}
\myindex{x86!\Instructions!INT}

Auf MS-DOS gab es einen Weg Funktion Argumente über Register zu übergeben, zum Beispiel dies 
ist ein Stück Code einer veralteten 16-Bit MS-DOS ``Hallo, Welt!'' Funktion:

\begin{lstlisting}[style=customasmx86]
mov  dx, msg      ; Adresse der Naricht
mov  ah, 9        ; 9 bedeutet ``print string''
int  21h          ; DOS "syscall"

mov  ah, 4ch      ; ``Terminiere Programm'' Funktion
int  21h          ; DOS "syscall"

msg  db 'Hello, World!\$' 
\end{lstlisting}

\myindex{fastcall}
Diese Methode ist der \myref{fastcall} Methode sehr ähnlich. Sie ähnelt aber auch der Methode
wie man auf Linux (\myref{linux_syscall}) und Windows syscalls ausführt.

\myindex{x86!\Flags!CF}
Wenn eine MS-DOS Funktion einen Bool'schen Wert zurück gibt (z.B., Single Bit bedeutet ein Fehler ist aufgetreten), wird dafür das \TT{CF} Flag benutzt.

Zum Beispiel:

\begin{lstlisting}[style=customasmx86]
mov ah, 3ch       ; create file
lea dx, filename
mov cl, 1
int 21h
jc  error
mov file_handle, ax
...
error:
...
\end{lstlisting}

Im Falle eines Fehlers, wird das \TT{CF} Flag gesetzt. Anderenfalls wird ein handle für die neu erstellte Datei über \TT{AX} zurück gegeben. 


Diese Methode wird heute immer noch von Assembler Programmierern benutzt.
Im Windows Reseearch Kernel source Code (der sehr ähnlich zum Windows 2003 Kernel ist) können wir folgenden Code
finden (file \emph{base/ntos/ke/i386/cpu.asm}):

% muss noch die kommentare geändert werden
\begin{lstlisting}[style=customasmx86]
        public  Get386Stepping
Get386Stepping  proc

        call    MultiplyTest            ; Muliplikations Test durchführen
        jnc     short G3s00             ; wenn nc, ist muttest ok
        mov     ax, 0
        ret
G3s00:
        call    Check386B0              ; Prüfe das B0 stepping
        jnc     short G3s05             ; wenn nc, ist es B1/later
        mov     ax, 100h                ; It is B0/earlier stepping
        ret

G3s05:
        call    Check386D1              ; Prüfe das D1 stepping
        jc      short G3s10             ; wenn c, iust es NICHT NOT D1
        mov     ax, 301h                ; Es ist das D1/later stepping
        ret

G3s10:
        mov     ax, 101h                ; annahme das es das it is B1 stepping ist
        ret

	...

MultiplyTest    proc

        xor     cx,cx                   ; 64K durchläufe ist eine nette runde Nummer
mlt00:  push    cx
        call    Multiply                ; Funktioniert dis multiplikation auf diesem Chip?
        pop     cx
        jc      short mltx              ; wenn c c, Nein, exit
        loop    mlt00                   ; Wenn nc, Ja, weitere iteration für nächsten versuch
        clc
mltx:
        ret

MultiplyTest    endp
\end{lstlisting}


\input{patterns/02_stack/03_local_vars_DE}
\EN{\mysection{Returning Values: redux}

Again, when we know about function prologue and epilogue, let's recompile an example returning a value
(\ref{ret_val_func}, \ref{lst:ret_val_func}) using non-optimizing GCC:

\lstinputlisting[caption=\NonOptimizing GCC 8.2 x64 (\assemblyOutput),style=customasmx86]{patterns/017_ret_redux/1.s}

Effective instructions here are \INS{MOV} and \INS{RET}, others are -- prologue and epilogue.

}

\input{patterns/02_stack/05_SEH}
\ifdefined\ENGLISH
\subsubsection{Buffer overflow protection}

More about it here~(\myref{subsec:bufferoverflow}).
\fi

\ifdefined\RUSSIAN
\subsubsection{Защита от переполнений буфера}

Здесь больше об этом~(\myref{subsec:bufferoverflow}).
\fi

\ifdefined\BRAZILIAN
\subsubsection{Proteção contra estouro de buffer}

Mais sobre aqui~(\myref{subsec:bufferoverflow}).
\fi

\ifdefined\ITALIAN
\subsubsection{Protezione contro buffer overflow}

Maggiori informazioni qui~(\myref{subsec:bufferoverflow}).
\fi

\ifdefined\FRENCH
\subsubsection{Protection contre les débordements de tampon}

Lire à ce propos~(\myref{subsec:bufferoverflow}).
\fi


\ifdefined\POLISH
\subsubsection{Ochrona przed przepełnieniem bufora}

Więcej o tym tutaj~(\myref{subsec:bufferoverflow}).
\fi

\ifdefined\JAPANESE
\subsubsection{バッファオーバーフロー保護}

詳細はこちら~(\myref{subsec:bufferoverflow})
\fi


\subsubsection{Automatisches deallokieren der Daten auf dem Stack}

Vielleicht ist der Grund warum man lokale Variablen und SEH Einträge auf dem Stack speichert, weil sie beim 
verlassen der Funktion automatisch aufgeräumt werden. Man braucht dabei nur eine Instruktion um die Position
des Stackpointers zu korrigieren (oftmals ist es die \ADD Instruktion). Funktions Argumente, könnte man sagen 
werden auch am Ende der Funktion deallokiert. Im Kontrast dazu, alles was auf dem \emph{heap} gespeichert wird muss
explizit deallokiert werden. 

% sections
\input{patterns/02_stack/07_layout_DE}
\EN{\mysection{Returning Values: redux}

Again, when we know about function prologue and epilogue, let's recompile an example returning a value
(\ref{ret_val_func}, \ref{lst:ret_val_func}) using non-optimizing GCC:

\lstinputlisting[caption=\NonOptimizing GCC 8.2 x64 (\assemblyOutput),style=customasmx86]{patterns/017_ret_redux/1.s}

Effective instructions here are \INS{MOV} and \INS{RET}, others are -- prologue and epilogue.

}

\input{patterns/02_stack/exercises}

