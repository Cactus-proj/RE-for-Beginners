\mysection{\Stack}
\label{sec:stack}
\myindex{\Stack}

Stos w informatyce jest jedną z najbardziej fundamentalnych struktur danych
\footnote{\href{http://go.yurichev.com/17119}{wikipedia.org/wiki/Call\_stack}}.
\ac{AKA} \ac{LIFO}.

Technicznie rzecz biorąc, jest to tylko blok pamięci w pamięci procesora + rejestr \ESP w x86 lub \RSP w x64, lub \ac{SP} w ARM, który wskazuje obszar gdzieś w granicach tego bloku.

\myindex{ARM!\Instructions!PUSH}
\myindex{ARM!\Instructions!POP}
\myindex{x86!\Instructions!PUSH}
\myindex{x86!\Instructions!POP}
Najczęsciej wykorzystywanymi instrukcjami do operowania na stosie są \PUSH i \POP (w x86 i Thumb-trybie ARM). 
\PUSH zmniejsza \ESP/\RSP/\ac{SP} o 4 w trybie 32-bitowym (lub o 8 w 64-bitowym),
następnie zapisuje pod adres, na który wskazuje \ESP/\RSP/\ac{SP}, zawartość swojego operandu.

\POP jest odwrotną operacją- najpierw zdejmuje ze \glslink{stack pointer}{wskaźnika stosu} wartość i umieszcza ją do operandu 
(który często jest rejestrem), a następnie zwiększa wskaźnik o 4 (lub 8).

Przed alokacją pamięci na stosie \glslink{stack pointer}{rejestr-wskaźnik} wskazuje na koniec stosu.
Koniec stosu znajduje się na początku zaalokowanego bloku pamięci, przeznaczonego na stos. Może zabrzmieć to dziwnie, ale tak to działa.
\PUSH zmniejsza \glslink{stack pointer}{rejestr-wskaźnik}, а \POP~--- zwiększa.

W procesorze ARM jest wsparcie dla stosów zarówno rosnących w dół, jak i rosnącyh w górę.

\myindex{ARM!\Instructions!STMFD}
\myindex{ARM!\Instructions!LDMFD}
\myindex{ARM!\Instructions!STMED}
\myindex{ARM!\Instructions!LDMED}
\myindex{ARM!\Instructions!STMFA}
\myindex{ARM!\Instructions!LDMFA}
\myindex{ARM!\Instructions!STMEA}
\myindex{ARM!\Instructions!LDMEA}

Na przykład, instrukcje \ac{STMFD}/\ac{LDMFD}, \ac{STMED}/\ac{LDMED} są przeznaczone dla stosu malejącego (rośnie w dół, zaczynając od adresów wysokich, do adresów niskich).\\
Natomiast instrukcje \ac{STMFA}/\ac{LDMFA}, \ac{STMEA}/\ac{LDMEA} są przeznaczone dla stosu rosnącego (rośnie w górę, zaczynając od niskich adresów, kończąc na adresach wysokich).

% It might be worth mentioning that STMED and STMEA write first,
% and then move the pointer,
% and that LDMED and LDMEA move the pointer first, and then read.
% In other words, ARM not only lets the stack grow in a non-standard direction,
% but also in a non-standard order.
% Maybe this can be in the glossary, which would explain why E stands for "empty".

\subsection{Dlaczego stos rośnie wstecznie?}
\label{stack_grow_backwards}

Intuicyjnie moglibyśmy pomyśleć, że, jak i każda inna struktura danych, stos mogłby rosnąć w górę, tzn. w kierunku zwiększenia adresów.

Powód, dlaczego stos rośnie w dół, jest najprawdobodobniej historyczny.
Kiedy komputery były duże i zajmowały cały pokój, można było bardzo łatwo rozdzielić segment na dwa obszary: dla \glslink{heap}{kopca} i dla stosu.
Z góry nie było wiadomo, jak duża może być \glslink{heap}{sterta} lub stos, dlatego takie rozwiązanie było najbardziej logiczne.

\input{patterns/02_stack/stack_and_heap}

W \RitchieThompsonUNIX można przeczytać:

\begin{framed}
\begin{quotation}
The user-core part of an image is divided into three logical segments. The program text segment begins at location 0 in the virtual address space. During execution, this segment is write-protected and a single copy of it is shared among all processes executing the same program. At the first 8K byte boundary above the program text segment in the virtual address space begins a nonshared, writable data segment, the size of which may be extended by a system call. Starting at the highest address in the virtual address space is a stack segment, which automatically grows downward as the hardware's stack pointer fluctuates.
\end{quotation}
\end{framed}

To trochę przypomina podejście studenta,
który pisze dwie osobne lektury w jednym zeszycie:
pierwsza lektura jest pisana jak zwykle od początku zeszytu, a druga jest pisana od końca zeszytu.
Lektury mogą się "spotkać" gdzieś na środku zeszytu, jeśli zabraknie miejsca.

% I think if we want to expand on this analogy,
% one might remember that the line number increases as as you go down a page.
% So when you decrease the address when pushing to the stack, visually,
% the stack does grow upwards.
% Of course, the problem is that in most human languages,
% just as with computers,
% we write downwards, so this direction is what makes buffer overflows so messy.

\subsection{Do jakich celów służy stos?}

% subsections
\subsubsection{Zapisywanie adresu powrotu}

\myparagraph{x86}

\myindex{x86!\Instructions!CALL}
Przed wywołaniem funkcji za pomocą instrukcji \CALL, na stos odkładany jest adres kolejnej instrukcji (tej bezpośrednio za \CALL). Następnie następuje skok bezwarunkowy pod adres z operandu instrukcji \CALL.

\myindex{x86!\Instructions!PUSH}
\myindex{x86!\Instructions!JMP}
Instrukcja \CALL jest równoważna parze instrukcji \INS{PUSH adres\_docelowy / JMP}.

\myindex{x86!\Instructions!RET}
\myindex{x86!\Instructions!POP}
\RET zdejmuje adres ze stosu i przekazuje tam sterowanie~---
jest to równoważne parze instrukcji \TT{POP tmp / JMP tmp}.

\myindex{\Stack!\MLStackOverflow}
\myindex{\Recursion}
Bardzo łatwo przepełnić stos, poprzez nieskończoną rekurencję:

\begin{lstlisting}[style=customc]
void f()
{
	f();
};
\end{lstlisting}

MSVC 2008 wyświetli ostrzeżenie:

\begin{lstlisting}
c:\tmp6>cl ss.cpp /Fass.asm
Microsoft (R) 32-bit C/C++ Optimizing Compiler Version 15.00.21022.08 for 80x86
Copyright (C) Microsoft Corporation.  All rights reserved.

ss.cpp
c:\tmp6\ss.cpp(4) : warning C4717: 'f' : recursive on all control paths, function will cause runtime stack overflow
\end{lstlisting}

\dots ale wygeneruje plik wykonywalny:

\lstinputlisting[style=customasmx86]{patterns/02_stack/1.asm}

\dots jeśli włączymy optymalizację (opcja \TT{\Ox}), to zoptymalizowany kod nie będzie powodował przepełnienia stosu i działał  \emph{poprawnie}\footnote{o ironio!}

\lstinputlisting[style=customasmx86]{patterns/02_stack/2.asm}

GCC 4.4.1 wygeneruje taki sam kod w obu przypadkach i nie wyświetli żadnego ostrzeżenia.

\myparagraph{ARM}

\myindex{ARM!\Registers!Link Register}
Programy na ARM również korzystają ze stosu do zapisywania adresu powrotu, ale trochę w inny sposób.
Jak już było wspomniane w rozdziale \q{\HelloWorldSectionName}~(\myref{sec:hw_ARM}),
adres powrotu (\ac{RA}) jest zapisywany do rejestru \ac{LR} (\gls{link register}).
Jeśli zajdzie potrzeba wywołania kolejnej funkcji i ponownego użycia  \ac{LR}, to jego zawartość będzie musiała być gdzieś zapisana.

\myindex{Function prologue}
\myindex{ARM!\Instructions!PUSH}
\myindex{ARM!\Instructions!POP}

Zwykle odbywa się to w prologu funkcji, często widzimy tam instrukcję jak \INS{PUSH \{R4-R7,LR\}}, a w epilogu
\INS{POP \{R4-R7,PC\}}~--- rejestry, z których będzie korzystała bieżąca funkcja, w tym rejestr \ac{LR}, odkładane są na stos.

\myindex{ARM!Leaf function}
Jeśli jakaś funkcja nie wywołuje żadnych innych funkcji w trakcie swojej pracy, w terminologii \ac{RISC} nazywana jest
\emph{\glslink{leaf function}{funkcją-liściem}}\footnote{\href{http://infocenter.arm.com/help/index.jsp?topic=/com.arm.doc.faqs/ka13785.html}{infocenter.arm.com/help/index.jsp?topic=/com.arm.doc.faqs/ka13785.html}}.
Z tego powodu funkcja-liść nie odkłada rejestru \ac{LR} na stos, ponieważ go nie zmienia.
Jeśli funkcja jest niewielkich rozmiarów i korzysta z małej liczby rejestrów, to może w ogóle nie korzystać ze stosu.
Stąd w ARM możliwe jest wywoływanie małych funkcji-liści bez używania stosu.
Jest to szybsze niż w starych x86, gdyż nie korzysta się z pamięci zewnętrznej RAM do korzystania ze stosu.
\footnote{Kiedyś, na PDP-11 i VAX, instrukcja CALL (wywołanie innych funkcji) była kosztowna; procesor spędzał na \CALL nawet do 50\% czasu wykonania programu. Z tego powodu posiadanie dużej liczby małych funkcji uchodziło za \glslink{anti-pattern}{antywzorzec}
\InSqBrackets{\TAOUP Chapter 4, Part II}.}.
Ten mechanizm przydaje się również, gdy pamięć pod stos nie została jeszcze zaalokowana albo jest niedostępna.

kilka przykładów takich funkcji:
\myref{ARM_leaf_example1}, \myref{ARM_leaf_example2}, 
\myref{ARM_leaf_example3}, \myref{ARM_leaf_example4}, \myref{ARM_leaf_example5},
\myref{ARM_leaf_example6}, \myref{ARM_leaf_example7}, \myref{ARM_leaf_example10}.



\subsubsection{Przekazywanie argumentów funkcji}

Najpopularniejszy sposób na przekazywanie parametrów funkcji w x86 to \q{cdecl}:

\begin{lstlisting}[style=customasmx86]
push arg3
push arg2
push arg1
call f
add esp, 12 ; 4*3=12
\end{lstlisting}

Wywoływana funkcja pobiera swoje argumentu za pomocą wskaźnik stosu.

Stos, przed wykonaniem pierwszej instrukcji z \ttf{}, wygląda następująco:

\begin{center}
\begin{tabular}{ | l | l | }
\hline
ESP & adres powrotu \\
\hline
ESP+4 & \argument \#1, \MarkedInIDAAs{} \TT{arg\_0} \\
\hline
ESP+8 & \argument \#2, \MarkedInIDAAs{} \TT{arg\_4} \\
\hline
ESP+0xC & \argument \#3, \MarkedInIDAAs{} \TT{arg\_8} \\
\hline
\dots & \dots \\
\hline
\end{tabular}
\end{center}

Opis innych konwencji wywoływania funkcji znajduje się tutaj: ~(\myref{sec:callingconventions}).

\par Nawiasem mówiąc, funkcja wywoływana nie posiada informacji o liczbie przekazywanych do niej argumentów.
Funkcje w C o zmiennej liczbie argumentów (jak \printf) ustalają ich liczbę za pomocą  specyfikatorów formatu (rozpoczynających się od znaku \%).

Jeśli napiszemy:

\begin{lstlisting}
printf("%d %d %d", 1234);
\end{lstlisting}

\printf wypisze 1234, a następnie jeszcze dwie losowe\footnote{Tak na prawdę nie są one losowe, patrz: \myref{noise_in_stack}} liczby, który przypadkowo znalazły się na stosie obok.

\label{main_arguments}
\par
Dlatego nie ma znaczenia jak zapiszemy funkcję \main{}:\\
jak \main{}, \TT{main(int argc, char *argv[])}\\
lub \TT{main(int argc, char *argv[], char *envp[])}.

W rzeczywistości kod z \ac{CRT} wywołuje \main mniej więcej tak:
	
\begin{lstlisting}[style=customasmx86]
push envp
push argv
push argc
call main
...
\end{lstlisting}

Jeśli zadeklarujesz \main bez argumentów, one i tak będą na stosie, lecz nie zostaną wykorzystane.
Jeśli zadeklarujesz \main jako \TT{main(int argc, char *argv[])},
to będziesz mógł skorzystać z pierwszych dwóch argumentów, a trzeci będzie dla funkcji \q{niewidoczny}.
Co więcej, można nawet zadeklarować \TT{main(int argc)} i to również zadziała.

Inny, podobny, przykład: \ref{cdecl_DLL}.

\myparagraph{Alternatywne sposoby na przekazywanie argumentów}

Warto zauważyć, że nic nie zmusza programisty do przekazywanie argumentów przez stos. Nie ma takiego wymagania, można to robić to zupełnie inaczej, nie korzystając ze stosu.

Dość popularnym sposobem wśród początkujących jest przekazywanie argumentów przez zmienne globalne, na przykład:

\lstinputlisting[caption=Kod w asemblerze,style=customasmx86]{patterns/02_stack/global_args.asm}

Ta metoda posiada oczywistą wadę: funkcja \emph{do\_something()} nie może wywołać sama siebie przez rekurencję (lub za pomocą innej funkcji), gdyż musiałaby nadpisać własne argumenty.
To samo dotyczy zmiennych lokalnych, gdyby przechowywać je w zmiennych globalnych, to funkcja nie będzie mogła wywołać sama siebie.
Co więcej, nie jest to bezpieczne w środowisku wielowątkowym\footnote{W poprawnej implementacji
każdy wątek miałby własny stos lokalny, ze swoimi argumentami/zmiennymi.}.
Przechowywania danych na stosie wszystko upraszcza ---
stos może przechować tyle argumentów funkcji/zmiennych,
na ile pozwoli jego rozmiar.

W \InSqBrackets{\TAOCPvolI{}, 189} można przeczytać o jeszcze dziwniejszych metodach przekazywania argumentów, szczególnie wygodnych na IBM System/360.

\myindex{MS-DOS}
\myindex{x86!\Instructions!INT}

W MS-DOS istniała metoda przekazywania argumentów przez rejestry, na przykład ten fragment kodu na wiekowym 16-bitowym MS-DOS
wypisze \q{Hello, world!}:

\begin{lstlisting}[style=customasmx86]
mov  dx, msg      ; adres wiadomości
mov  ah, 9        ; 9 oznacza funkcję "wypisz łańcuch znaków"
int  21h          ; wywołanie systemowe ("syscall") DOS

mov  ah, 4ch      ; funkcja "zakończ program"
int  21h          ; wywołanie systemowe ("syscall") DOS

msg  db 'Hello, World!\$'
\end{lstlisting}

\myindex{fastcall}
Jest to całkiem podobne do metody \myref{fastcall}.
Przypomina to również sposoby korzystania z wywołań systemowych (ang. syscall) na systemach Linuks (\myref{linux_syscall}) i Windows.

\myindex{x86!\Flags!CF}
Jeżeli funkcja w MS-DOS zwraca wartość typu boolean (jeden bit, zwykle oznaczający wystąpienie błędu), to często wykorzystywana jest flaga \TT{CF}.

Na przykład:

\begin{lstlisting}[style=customasmx86]
mov ah, 3ch       ; "3c" oznacza "stwórz plik
lea dx, filename
mov cl, 1
int 21h
jc  error
mov file_handle, ax
...
error:
...
\end{lstlisting}

W razie wystąpienia błędu flaga \TT{CF} zostaje ustawiona. W przeciwnym razie uchwyt (ang. \emph{file handle}) stworzonego pliku zwracany jest przez rejestr \TT{AX}.

Ta metoda wciąż jest wykorzystywana przez programistów asemblera.
W kodach źródłowych Windows Research Kernel (który jest bardzo podobny do Windows 2003) możemy znaleźć coś takiego\\
(plik \emph{base/ntos/ke/i386/cpu.asm}):

\begin{lstlisting}[style=customasmx86]
        public  Get386Stepping
Get386Stepping  proc

        call    MultiplyTest            ; Perform multiplication test
        jnc     short G3s00             ; if nc, muttest is ok
        mov     ax, 0
        ret
G3s00:
        call    Check386B0              ; Check for B0 stepping
        jnc     short G3s05             ; if nc, it's B1/later
        mov     ax, 100h                ; It is B0/earlier stepping
        ret

G3s05:
        call    Check386D1              ; Check for D1 stepping
        jc      short G3s10             ; if c, it is NOT D1
        mov     ax, 301h                ; It is D1/later stepping
        ret

G3s10:
        mov     ax, 101h                ; assume it is B1 stepping
        ret

	...

MultiplyTest    proc

        xor     cx,cx                   ; 64K times is a nice round number
mlt00:  push    cx
        call    Multiply                ; does this chip's multiply work?
        pop     cx
        jc      short mltx              ; if c, No, exit
        loop    mlt00                   ; if nc, YEs, loop to try again
        clc
mltx:
        ret

MultiplyTest    endp
\end{lstlisting}




\subsubsection{Przechowywanie zmiennych lokalnych}

Funkcja może zaalokować miejsce na stosie dla własnych zmiennych lokalnych przez zmniejszenie
\glslink{stack pointer}{wskaźnika stosu}, w kierunku końca stosu (pamiętaj, że stos rośnie w dół, w kierunku niskich adresów!).

% I think here, "stack bottom" means the lowest address in the stack space,
% but the reader might also think it means towards the top of the stack space,
% like in a pop, so you might change "towards the stack bottom" to
% "towards the lowest address of the stack", or just take it out,
% since "decreasing" also suggests that.

Jest to bardzo szybkie, niezależnie od liczby zmiennych lokalnych.
Wiedz, że nie ma przymusu trzymania zmiennych lokalnych na stosie. Możesz je trzymać gdziekolwiek, ale tradycyjnie wykorzystuje się do tego stos.

\EN{\mysection{Returning Values: redux}

Again, when we know about function prologue and epilogue, let's recompile an example returning a value
(\ref{ret_val_func}, \ref{lst:ret_val_func}) using non-optimizing GCC:

\lstinputlisting[caption=\NonOptimizing GCC 8.2 x64 (\assemblyOutput),style=customasmx86]{patterns/017_ret_redux/1.s}

Effective instructions here are \INS{MOV} and \INS{RET}, others are -- prologue and epilogue.

}

\input{patterns/02_stack/05_SEH}
\ifdefined\ENGLISH
\subsubsection{Buffer overflow protection}

More about it here~(\myref{subsec:bufferoverflow}).
\fi

\ifdefined\RUSSIAN
\subsubsection{Защита от переполнений буфера}

Здесь больше об этом~(\myref{subsec:bufferoverflow}).
\fi

\ifdefined\BRAZILIAN
\subsubsection{Proteção contra estouro de buffer}

Mais sobre aqui~(\myref{subsec:bufferoverflow}).
\fi

\ifdefined\ITALIAN
\subsubsection{Protezione contro buffer overflow}

Maggiori informazioni qui~(\myref{subsec:bufferoverflow}).
\fi

\ifdefined\FRENCH
\subsubsection{Protection contre les débordements de tampon}

Lire à ce propos~(\myref{subsec:bufferoverflow}).
\fi


\ifdefined\POLISH
\subsubsection{Ochrona przed przepełnieniem bufora}

Więcej o tym tutaj~(\myref{subsec:bufferoverflow}).
\fi

\ifdefined\JAPANESE
\subsubsection{バッファオーバーフロー保護}

詳細はこちら~(\myref{subsec:bufferoverflow})
\fi


\subsubsection{Automatyczne zwalnianie danych na stosie}
Możliwym powodem przechowywania zmiennych lokalnych i rekordów SEH na stosie jest to, że kiedy funkcja zakończy działanie są one automatycznie zwalniane ze stosu używając tylko jednej instrukcji w celu przywrócenia poprzedniego stanu stosu (często jest to instrukcja ADD). Argumenty funkcji także są automatycznie zwalniane z pamięci pod koniec funkcji. Natomiast wszystko co jest przechowywane na stercie(\emph{heap}) trzeba zwalniać jawnie.

% subsections
\subsection{Struktura typowego stosu}

Struktura typowego stosu w środowisku 32-bitowym,
przed wykonaniem pierwszej instrukcji w funkcji, wygląda następująco:

\begin{center}
\begin{tabular}{ | l | l | }
\hline
\dots & \dots \\
\hline
ESP-0xC & \localVariable \#2, \MarkedInIDAAsFem{} \TT{var\_8} \\
\hline
ESP-8 & \localVariable \#1, \MarkedInIDAAsFem{} \TT{var\_4} \\
\hline
ESP-4 & \savedValueOf  \EBP \\
\hline
ESP & \ReturnAddress \\
\hline
ESP+4 & \argument \#1, \MarkedInIDAAs{} \TT{arg\_0} \\
\hline
ESP+8 & \argument \#2, \MarkedInIDAAs{} \TT{arg\_4} \\
\hline
ESP+0xC & \argument \#3, \MarkedInIDAAs{} \TT{arg\_8} \\
\hline
\dots & \dots \\
\hline
\end{tabular}
\end{center}



% I think this only applies to RISC architectures
% that don't have a POP instruction that only lets you read one value
% (ie. ARM and MIPS).
% In x86, the return address is saved before entering the function,
% and the function does not have the chance to save the frame pointer.
% Also, you should mention that this is how the stack looks like
% right after the function prologue,
% which some readers might think is the first instruction,
% but is needed to save the frame pointer.



\EN{\mysection{Returning Values: redux}

Again, when we know about function prologue and epilogue, let's recompile an example returning a value
(\ref{ret_val_func}, \ref{lst:ret_val_func}) using non-optimizing GCC:

\lstinputlisting[caption=\NonOptimizing GCC 8.2 x64 (\assemblyOutput),style=customasmx86]{patterns/017_ret_redux/1.s}

Effective instructions here are \INS{MOV} and \INS{RET}, others are -- prologue and epilogue.

}

\input{patterns/02_stack/exercises}

