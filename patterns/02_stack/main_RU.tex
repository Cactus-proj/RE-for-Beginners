\mysection{\Stack}
\label{sec:stack}
\myindex{\Stack}

Стек в компьютерных науках~--- это одна из наиболее фундаментальных структур данных
\footnote{\href{http://go.yurichev.com/17119}{wikipedia.org/wiki/Call\_stack}}.
\ac{AKA} \ac{LIFO}.

Технически это просто блок памяти в памяти процесса + регистр \ESP в x86 или \RSP в x64, либо \ac{SP} в ARM, который указывает где-то в пределах этого блока.

\myindex{ARM!\Instructions!PUSH}
\myindex{ARM!\Instructions!POP}
\myindex{x86!\Instructions!PUSH}
\myindex{x86!\Instructions!POP}
Часто используемые инструкции для работы со стеком~--- это \PUSH и \POP (в x86 и Thumb-режиме ARM). 
\PUSH уменьшает \ESP/\RSP/\ac{SP} на 4 в 32-битном режиме (или на 8 в 64-битном),
затем записывает по адресу, на который указывает \ESP/\RSP/\ac{SP}, содержимое своего единственного операнда.

\POP это обратная операция~--- сначала достает из \glslink{stack pointer}{указателя стека} значение и помещает его в операнд 
(который очень часто является регистром) и затем увеличивает указатель стека на 4 (или 8).

В самом начале \glslink{stack pointer}{регистр-указатель} указывает на конец стека.
Конец стека находится в начале блока памяти, выделенного под стек. Это странно, но это так.
\PUSH уменьшает \glslink{stack pointer}{регистр-указатель}, а \POP~--- увеличивает.

В процессоре ARM, тем не менее, есть поддержка стеков, растущих как в сторону уменьшения, так и в сторону увеличения.

\myindex{ARM!\Instructions!STMFD}
\myindex{ARM!\Instructions!LDMFD}
\myindex{ARM!\Instructions!STMED}
\myindex{ARM!\Instructions!LDMED}
\myindex{ARM!\Instructions!STMFA}
\myindex{ARM!\Instructions!LDMFA}
\myindex{ARM!\Instructions!STMEA}
\myindex{ARM!\Instructions!LDMEA}

Например, инструкции \ac{STMFD}/\ac{LDMFD}, \ac{STMED}/\ac{LDMED} предназначены для descending-стека (растет назад, начиная с высоких адресов в сторону низких).\\
Инструкции \ac{STMFA}/\ac{LDMFA}, \ac{STMEA}/\ac{LDMEA} предназначены для ascending-стека (растет вперед, начиная с низких адресов в сторону высоких).

% It might be worth mentioning that STMED and STMEA write first,
% and then move the pointer,
% and that LDMED and LDMEA move the pointer first, and then read.
% In other words, ARM not only lets the stack grow in a non-standard direction,
% but also in a non-standard order.
% Maybe this can be in the glossary, which would explain why E stands for "empty".

\subsection{Почему стек растет в обратную сторону?}
\label{stack_grow_backwards}

Интуитивно мы можем подумать, что, как и любая другая структура данных, стек мог бы расти вперед, т.е. в сторону увеличения адресов.

Причина, почему стек растет назад, видимо, историческая.
Когда компьютеры были большие и занимали целую комнату, было очень легко разделить сегмент на две части: для \glslink{heap}{кучи} и для стека.
Заранее было неизвестно, насколько большой может быть \glslink{heap}{куча} или стек, так что это решение было самым простым.

\input{patterns/02_stack/stack_and_heap}

В \RitchieThompsonUNIX можно прочитать:

\begin{framed}
\begin{quotation}
The user-core part of an image is divided into three logical segments. The program text segment begins at location 0 in the virtual address space. During execution, this segment is write-protected and a single copy of it is shared among all processes executing the same program. At the first 8K byte boundary above the program text segment in the virtual address space begins a nonshared, writable data segment, the size of which may be extended by a system call. Starting at the highest address in the virtual address space is a stack segment, which automatically grows downward as the hardware's stack pointer fluctuates.
\end{quotation}
\end{framed}

Это немного напоминает как некоторые студенты
пишут два конспекта в одной тетрадке:
первый конспект начинается обычным образом, второй пишется с конца, перевернув тетрадку.
Конспекты могут встретиться где-то посредине, в случае недостатка свободного места.

% I think if we want to expand on this analogy,
% one might remember that the line number increases as as you go down a page.
% So when you decrease the address when pushing to the stack, visually,
% the stack does grow upwards.
% Of course, the problem is that in most human languages,
% just as with computers,
% we write downwards, so this direction is what makes buffer overflows so messy.

\subsection{Для чего используется стек?}

% subsections
\subsubsection{Сохранение адреса возврата управления}

\myparagraph{x86}

\myindex{x86!\Instructions!CALL}
При вызове другой функции через \CALL сначала в стек записывается адрес, указывающий на место после 
инструкции \CALL, затем делается безусловный переход (почти как \TT{JMP}) на адрес, указанный в операнде.

\myindex{x86!\Instructions!PUSH}
\myindex{x86!\Instructions!JMP}
\CALL~--- это аналог пары инструкций \INS{PUSH address\_after\_call / JMP}.

\myindex{x86!\Instructions!RET}
\myindex{x86!\Instructions!POP}
\RET вытаскивает из стека значение и передает управление по этому адресу~--- 
это аналог пары инструкций \TT{POP tmp / JMP tmp}.

\myindex{\Stack!\MLStackOverflow}
\myindex{\Recursion}
Крайне легко устроить переполнение стека, запустив бесконечную рекурсию:

\begin{lstlisting}[style=customc]
void f()
{
	f();
};
\end{lstlisting}

MSVC 2008 предупреждает о проблеме:

\begin{lstlisting}
c:\tmp6>cl ss.cpp /Fass.asm
Microsoft (R) 32-bit C/C++ Optimizing Compiler Version 15.00.21022.08 for 80x86
Copyright (C) Microsoft Corporation.  All rights reserved.

ss.cpp
c:\tmp6\ss.cpp(4) : warning C4717: 'f' : recursive on all control paths, function will cause runtime stack overflow
\end{lstlisting}

\dots но, тем не менее, создает нужный код:

\lstinputlisting[style=customasmx86]{patterns/02_stack/1.asm}

\dots причем, если включить оптимизацию (\TT{\Ox}), то будет даже интереснее, без переполнения стека, 
но работать будет \emph{корректно}\footnote{здесь ирония}:

\lstinputlisting[style=customasmx86]{patterns/02_stack/2.asm}

GCC 4.4.1 генерирует точно такой же код в обоих случаях, хотя и не предупреждает о проблеме.

\myparagraph{ARM}

\myindex{ARM!\Registers!Link Register}
Программы для ARM также используют стек для сохранения \ac{RA}, куда нужно вернуться, но несколько иначе.
Как уже упоминалось в секции \q{\HelloWorldSectionName}~(\myref{sec:hw_ARM}),
\ac{RA} записывается в регистр \ac{LR} (\gls{link register}).
Но если есть необходимость вызывать какую-то другую функцию и использовать регистр \ac{LR} ещё
раз, его значение желательно сохранить.
\myindex{Function prologue}
\myindex{ARM!\Instructions!PUSH}
\myindex{ARM!\Instructions!POP}

Обычно это происходит в прологе функции, часто мы видим там инструкцию вроде \INS{PUSH \{R4-R7,LR\}}, а в эпилоге
\INS{POP \{R4-R7,PC\}}~--- так сохраняются регистры, которые будут использоваться в текущей функции, в том числе \ac{LR}.

\myindex{ARM!Leaf function}
Тем не менее, если некая функция не вызывает никаких более функций, в терминологии \ac{RISC} она называется
\emph{\gls{leaf function}}\footnote{\href{http://infocenter.arm.com/help/index.jsp?topic=/com.arm.doc.faqs/ka13785.html}{infocenter.arm.com/help/index.jsp?topic=/com.arm.doc.faqs/ka13785.html}}. 
Как следствие, \q{leaf}-функция не сохраняет регистр \ac{LR} (потому что не изменяет его).
А если эта функция небольшая, использует мало регистров, она может не использовать стек вообще.
Таким образом, в ARM возможен вызов небольших leaf-функций не используя стек.
Это может быть быстрее чем в старых x86, ведь внешняя память для стека не используется
\footnote{Когда-то, очень давно, на PDP-11 и VAX на инструкцию CALL (вызов других функций) могло тратиться
вплоть до 50\% времени (возможно из-за работы с памятью),
поэтому считалось, что много небольших функций это \glslink{anti-pattern}{анти-паттерн}
\InSqBrackets{\TAOUP Chapter 4, Part II}.}.
Либо это может быть полезным для тех ситуаций, когда память для стека ещё не выделена, либо недоступна,

Некоторые примеры таких функций:
\myref{ARM_leaf_example1}, \myref{ARM_leaf_example2}, 
\myref{ARM_leaf_example3}, \myref{ARM_leaf_example4}, \myref{ARM_leaf_example5},
\myref{ARM_leaf_example6}, \myref{ARM_leaf_example7}, \myref{ARM_leaf_example10}.


\subsubsection{Передача параметров функции}

Самый распространенный способ передачи параметров в x86 называется \q{cdecl}:

\begin{lstlisting}[style=customasmx86]
push arg3
push arg2
push arg1
call f
add esp, 12 ; 4*3=12
\end{lstlisting}

Вызываемая функция получает свои параметры также через указатель стека.

Следовательно, так расположены значения в стеке перед исполнением самой первой инструкции функции \ttf{}:

\begin{center}
\begin{tabular}{ | l | l | }
\hline
ESP & адрес возврата \\
\hline
ESP+4 & \argument \#1, \MarkedInIDAAs{} \TT{arg\_0} \\
\hline
ESP+8 & \argument \#2, \MarkedInIDAAs{} \TT{arg\_4} \\
\hline
ESP+0xC & \argument \#3, \MarkedInIDAAs{} \TT{arg\_8} \\
\hline
\dots & \dots \\
\hline
\end{tabular}
\end{center}

См. также в соответствующем разделе о других способах передачи аргументов через стек~(\myref{sec:callingconventions}).

\par Кстати, вызываемая функция не имеет информации о количестве переданных ей аргументов.
Функции Си с переменным количеством аргументов (как \printf) определяют их количество по спецификаторам строки формата (начинающиеся со знака \%).

Если написать что-то вроде:

\begin{lstlisting}
printf("%d %d %d", 1234);
\end{lstlisting}

\printf выведет 1234, затем ещё два случайных числа\footnote{В строгом смысле, они не случайны, скорее, непредсказуемы: \myref{noise_in_stack}}, которые волею случая оказались в стеке рядом.

\label{main_arguments}
\par
Вот почему не так уж и важно, как объявлять функцию \main{}:\\
как \main{}, \TT{main(int argc, char *argv[])}\\
либо \TT{main(int argc, char *argv[], char *envp[])}.

В реальности, \ac{CRT}-код вызывает \main примерно так:

\begin{lstlisting}[style=customasmx86]
push envp
push argv
push argc
call main
...
\end{lstlisting}

Если вы объявляете \main без аргументов, они, тем не менее, присутствуют в стеке, но не используются.
Если вы объявите \main как \TT{main(int argc, char *argv[])}, 
вы можете использовать два первых аргумента, а третий останется для вашей функции \q{невидимым}.
Более того, можно даже объявить \TT{main(int argc)}, и это будет работать.

% Еще один пример с этим связанный: \ref{cdecl_DLL}.

\myparagraph{Альтернативные способы передачи аргументов}

Важно отметить, что, в общем, никто не заставляет программистов передавать параметры именно через стек, это не является требованием к исполняемому коду.
Вы можете делать это совершенно иначе, не используя стек вообще.

В каком-то смысле, популярный метод среди начинающих использовать язык ассемблера,
это передавать аргументы в глобальных переменных, например:

\lstinputlisting[caption=Код на ассемблере,style=customasmx86]{patterns/02_stack/global_args.asm}

Но у этого метода есть очевидный недостаток: ф-ция \emph{do\_something()} не сможет вызвать саму себя рекурсивно (либо, через
какую-то стороннюю ф-цию),
потому что тогда придется затереть свои собственные аргументы.
Та же история с локальными переменными: если хранить их в глобальных переменных, ф-ция не сможет вызывать сама себя.
К тому же, этот метод не безопасный для мультитредовой среды\footnote{При корректной реализации,
каждый тред будет иметь свой собственный стек со своими аргументами/переменными.}.
Способ хранения подобной информации в стеке заметно всё упрощает ---
он может хранить столько аргументов ф-ций и/или значений вообще,
сколько в нем есть места.

В \InSqBrackets{\TAOCPvolI{}, 189} можно прочитать про еще более странные схемы передачи аргументов,
которые были очень удобны на IBM System/360.

\myindex{MS-DOS}
\myindex{x86!\Instructions!INT}

В MS-DOS был метод передачи аргументов через регистры, например, этот фрагмент кода для древней 16-битной MS-DOS
выводит ``Hello, world!'':

\begin{lstlisting}[style=customasmx86]
mov  dx, msg      ; адрес сообщения
mov  ah, 9        ; 9 означает ф-цию "вывод строки"
int  21h          ; DOS "syscall"

mov  ah, 4ch      ; ф-ция "закончить программу"
int  21h          ; DOS "syscall"

msg  db 'Hello, World!\$'
\end{lstlisting}

\myindex{fastcall}
Это очень похоже на метод \myref{fastcall}.
И еще на метод вызовов сисколлов в Linux (\myref{linux_syscall}) и Windows.

\myindex{x86!\Flags!CF}
Если ф-ция в MS-DOS возвращает булево значение (т.е., один бит, обычно сигнализирующий об ошибке),
часто использовался флаг \TT{CF}.

Например:

\begin{lstlisting}[style=customasmx86]
mov ah, 3ch       ; создать файл
lea dx, filename
mov cl, 1
int 21h
jc  error
mov file_handle, ax
...
error:
...
\end{lstlisting}

В случае ошибки, флаг \TT{CF} будет выставлен.
Иначе, хэндл только что созданного файла возвращается в \TT{AX}.

Этот метод до сих пор используется программистами на ассемблере.
В исходных кодах Windows Research Kernel (который очень похож на Windows 2003) мы можем найти такое\\
(файл \emph{base/ntos/ke/i386/cpu.asm}):

\begin{lstlisting}[style=customasmx86]
        public  Get386Stepping
Get386Stepping  proc

        call    MultiplyTest            ; Perform multiplication test
        jnc     short G3s00             ; if nc, muttest is ok
        mov     ax, 0
        ret
G3s00:
        call    Check386B0              ; Check for B0 stepping
        jnc     short G3s05             ; if nc, it's B1/later
        mov     ax, 100h                ; It is B0/earlier stepping
        ret

G3s05:
        call    Check386D1              ; Check for D1 stepping
        jc      short G3s10             ; if c, it is NOT D1
        mov     ax, 301h                ; It is D1/later stepping
        ret

G3s10:
        mov     ax, 101h                ; assume it is B1 stepping
        ret

	...

MultiplyTest    proc

        xor     cx,cx                   ; 64K times is a nice round number
mlt00:  push    cx
        call    Multiply                ; does this chip's multiply work?
        pop     cx
        jc      short mltx              ; if c, No, exit
        loop    mlt00                   ; if nc, YEs, loop to try again
        clc
mltx:
        ret

MultiplyTest    endp
\end{lstlisting}



\input{patterns/02_stack/03_local_vars_RU}
\EN{\mysection{Returning Values: redux}

Again, when we know about function prologue and epilogue, let's recompile an example returning a value
(\ref{ret_val_func}, \ref{lst:ret_val_func}) using non-optimizing GCC:

\lstinputlisting[caption=\NonOptimizing GCC 8.2 x64 (\assemblyOutput),style=customasmx86]{patterns/017_ret_redux/1.s}

Effective instructions here are \INS{MOV} and \INS{RET}, others are -- prologue and epilogue.

}

\input{patterns/02_stack/05_SEH}
\ifdefined\ENGLISH
\subsubsection{Buffer overflow protection}

More about it here~(\myref{subsec:bufferoverflow}).
\fi

\ifdefined\RUSSIAN
\subsubsection{Защита от переполнений буфера}

Здесь больше об этом~(\myref{subsec:bufferoverflow}).
\fi

\ifdefined\BRAZILIAN
\subsubsection{Proteção contra estouro de buffer}

Mais sobre aqui~(\myref{subsec:bufferoverflow}).
\fi

\ifdefined\ITALIAN
\subsubsection{Protezione contro buffer overflow}

Maggiori informazioni qui~(\myref{subsec:bufferoverflow}).
\fi

\ifdefined\FRENCH
\subsubsection{Protection contre les débordements de tampon}

Lire à ce propos~(\myref{subsec:bufferoverflow}).
\fi


\ifdefined\POLISH
\subsubsection{Ochrona przed przepełnieniem bufora}

Więcej o tym tutaj~(\myref{subsec:bufferoverflow}).
\fi

\ifdefined\JAPANESE
\subsubsection{バッファオーバーフロー保護}

詳細はこちら~(\myref{subsec:bufferoverflow})
\fi


\subsubsection{Автоматическое освобождение данных в стеке}

Возможно, причина хранения локальных переменных и SEH-записей в стеке в том, что после выхода из функции, всё эти данные освобождаются автоматически,
используя только одну инструкцию корректирования указателя стека (часто это \ADD).
Аргументы функций, можно сказать, тоже освобождаются автоматически в конце функции.
А всё что хранится в куче (\emph{heap}) нужно освобождать явно.

% subsections
\input{patterns/02_stack/07_layout_RU}
\EN{\mysection{Returning Values: redux}

Again, when we know about function prologue and epilogue, let's recompile an example returning a value
(\ref{ret_val_func}, \ref{lst:ret_val_func}) using non-optimizing GCC:

\lstinputlisting[caption=\NonOptimizing GCC 8.2 x64 (\assemblyOutput),style=customasmx86]{patterns/017_ret_redux/1.s}

Effective instructions here are \INS{MOV} and \INS{RET}, others are -- prologue and epilogue.

}

\input{patterns/02_stack/exercises}

