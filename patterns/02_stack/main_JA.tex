\mysection{\Stack}
\label{sec:stack}
\myindex{\Stack}

スタックは、コンピュータサイエンスにおける最も基本的なデータ構造の1つです。
\footnote{\href{http://en.wikipedia.org/wiki/Call_stack}{wikipedia.org/wiki/Call\_stack}}.
\ac{AKA} \ac{LIFO}.

技術的には、それは、プロセスメモリ内のメモリのブロックであり、x86またはx64の \ESP または \RSP レジスタ、またはARMの\ac{SP}レジスタをそのブロック内のポインタとして使用します。

\myindex{ARM!\Instructions!PUSH}
\myindex{ARM!\Instructions!POP}
\myindex{x86!\Instructions!PUSH}
\myindex{x86!\Instructions!POP}
最も頻繁に使用されるスタックアクセス命令は、 \PUSH と \POP (x86およびARM Thumbモードの両方)です。 
\PUSH は、32ビットモード(または64ビットモードでは8)で \ESP/\RSP/\ac{SP} 4を減算し、その単独オペランドの内容を \ESP/\RSP/\ac{SP}が指すメモリアドレスに書き込みます。

\POP は逆の操作です:\ac{SP}が指し示すメモリ位置からデータを取り出し、命令オペランド(しばしばレジスタ)にロードし、\gls{stack pointer}に4(または8)を追加します。

スタック割り当ての後、\gls{stack pointer}はスタックの一番下を指します。 \PUSH は\gls{stack pointer}を減らし、 \POP はそれを増やします。 
スタックの最下部は実際にスタックブロックに割り当てられたメモリの先頭にあります。 それは奇妙に見えますが、それはそうです。

ARMは降順スタックと昇順スタックの両方をサポートしています。

\myindex{ARM!\Instructions!STMFD}
\myindex{ARM!\Instructions!LDMFD}
\myindex{ARM!\Instructions!STMED}
\myindex{ARM!\Instructions!LDMED}
\myindex{ARM!\Instructions!STMFA}
\myindex{ARM!\Instructions!LDMFA}
\myindex{ARM!\Instructions!STMEA}
\myindex{ARM!\Instructions!LDMEA}

例えば、\ac{STMFD}/\ac{LDMFD}、\ac{STMED}/\ac{LDMED}命令は、降順のスタックを扱うことを意図しています(下位に向かって、高いアドレスから始まり、低いアドレスに進む)。 \ac{STMFA}/\ac{LDMFA}、\ac{STMEA}/\ac{LDMEA}命令は、昇順のスタックを扱うことを意図しています(上位アドレスから始まり、上位アドレスに向かって進みます)。

% It might be worth mentioning that STMED and STMEA write first,
% and then move the pointer,
% and that LDMED and LDMEA move the pointer first, and then read.
% In other words, ARM not only lets the stack grow in a non-standard direction,
% but also in a non-standard order.
% Maybe this can be in the glossary, which would explain why E stands for "empty".

\subsection{スタックはなぜ後方に進むのか}
\label{stack_grow_backwards}

直感的には、他のデータ構造と同様に、スタックが上方に、すなわちより高いアドレスに向かって成長すると考えるかもしれません。

スタックが後方に成長する理由はおそらく歴史的なものです。 
コンピュータが大きくて部屋全体を占有していた時代、メモリを2つの部分に分けるのは簡単でした。1つは \gls{heap} 用、もう1つはスタック用です。 
もちろん、プログラムの実行中に\gls{heap}とスタックがどれだけ大きくなるかは不明であったため、この解決策は最も簡単でした。

\input{patterns/02_stack/stack_and_heap}

\RitchieThompsonUNIX では、以下のように書かれています。

\begin{framed}
\begin{quotation}
画像のユーザコア部分は、3つの論理セグメントに分割される。 プログラムテキストセグメントは仮想アドレス空間の位置0で始まります。 実行中、このセグメントは書き込み保護されており、同じプログラムを実行しているすべてのプロセス間でこのセグメントが共有されます。 仮想アドレス空間のプログラムテキストセグメントの上の最初の8Kバイト境界では、共有されない書き込み可能なデータセグメントが開始されます。このデータセグメントのサイズはシステムコールによって拡張されます。 仮想アドレス空間の最上位アドレスから始まるスタックセグメントは、ハードウェアのスタックポインタが変動すると自動的に下に向かって成長します。
\end{quotation}
\end{framed}

これは、一部の学生が1つのノートブックを使用して2つの講義ノートを書く方法を思い出させます。
最初の講義のノートはいつものように書かれ、
2つ目のノートはノートブックの最後からそれを反転させて書き込まれます。 
空き領域がない場合に、ノートはその間のどこかで互いに会うことになります。

% I think if we want to expand on this analogy,
% one might remember that the line number increases as as you go down a page.
% So when you decrease the address when pushing to the stack, visually,
% the stack does grow upwards.
% Of course, the problem is that in most human languages,
% just as with computers,
% we write downwards, so this direction is what makes buffer overflows so messy.

\subsection{スタックは何に使用されるか}

% subsections
\subsubsection{関数のリターンアドレスを保存する}

\myparagraph{x86}

\myindex{x86!\Instructions!CALL}
\CALL 命令で別の関数を呼び出すと、 \CALL 命令の直後のポイントのアドレスがスタックに保存され、 
\CALL オペランドのアドレスへの無条件ジャンプが実行されます。

\myindex{x86!\Instructions!PUSH}
\myindex{x86!\Instructions!JMP}
\CALL 命令は、PUSHの\INS{PUSH address\_after\_call / JMP operand}命令対に相当する。

\RET はスタックから値を取り出し、ジャンプします。これは\TT{POP tmp / JMP tmp}命令の対に相当します。

\myindex{\Stack!\MLStackOverflow}
\myindex{\Recursion}
スタックのオーバーフローは簡単です。 永遠の再帰を実行するだけです:


\begin{lstlisting}[style=customc]
void f()
{
	f();
};
\end{lstlisting}

MSVC 2008が問題をレポートします:

\begin{lstlisting}
c:\tmp6>cl ss.cpp /Fass.asm
Microsoft (R) 32-bit C/C++ Optimizing Compiler Version 15.00.21022.08 for 80x86
Copyright (C) Microsoft Corporation.  All rights reserved.

ss.cpp
c:\tmp6\ss.cpp(4) : warning C4717: 'f' : recursive on all control paths, function will cause runtime stack overflow
\end{lstlisting}

\dots しかし、正しいコードを生成します。

\lstinputlisting[style=customasmx86]{patterns/02_stack/1.asm}

\dots また、コンパイラ最適化(\TT{\Ox}オプション)を有効にすると、最適化されたコードはスタックをオーバーフローせず、
代わりに\emph{正しく}\footnote{ここの皮肉}動作します。

\lstinputlisting[style=customasmx86]{patterns/02_stack/2.asm}

GCC 4.4.1はどちらの場合も問題の警告を出さずに同様のコードを生成します。

\myparagraph{ARM}

\myindex{ARM!\Registers!Link Register}
また、ARMプログラムはスタックを使用してリターンアドレスを保存しますが、別の方法でスタックを使用します。 
\q{\HelloWorldSectionName}~(\myref{sec:hw_ARM})で述べたように、\ac{RA}は\ac{LR}(\gls{link register})に保存されます。 
ただし、別の関数を呼び出してもう一度\ac{LR}レジスタを使用する必要がある場合は、その値を保存する必要があります。 
\myindex{Function prologue}
通常、関数プロローグに保存されます。

\myindex{ARM!\Instructions!PUSH}
\myindex{ARM!\Instructions!POP}
多くの場合、\INS{PUSH {R4-R7,LR}}のような命令が、エピローグで\INS{POP {R4-R7,PC}}とともに見られます。
したがって、関数で使用されるレジスタ値は、\ac{LR}を含めてスタックに保存されます。

\myindex{ARM!Leaf function}
それにもかかわらず、ある関数が他の関数を呼び出すことがなければ、\ac{RISC}の用語ではそれを
\emph{\gls{leaf function}}\footnote{\href{http://infocenter.arm.com/help/index.jsp?topic=/com.arm.doc.faqs/ka13785.html}{infocenter.arm.com/help/index.jsp?topic=/com.arm.doc.faqs/ka13785.html}}と呼びます。
その結果、リーフ関数は\ac{LR}レジスタを保存しません(\ac{LR}レジスタを変更しないため)。 
このような関数が小さく、少数のレジスタを使用する場合は、スタックをまったく使用しないことがあります。 
したがって、スタックを使用せずにリーフ関数を呼び出すことができます。
\footnote{いくつかの時間前、PDP-11とVAXでは、CALL命令(他の関数を呼び出す)は高価でした。 
実行時間の50%までが費やされる可能性があるため、小さな機能を多数持つことは\gls{anti-pattern} \InSqBrackets{\TAOUP Chapter 4, Part II}}
これは、外部RAMがスタックに使用されないため、古いx86マシンよりも高速になる可能性があります。これは、スタックのメモリがまだ割り当てられていない状況 または利用できません。

リーフ関数のいくつかの例:
\myref{ARM_leaf_example1}, \myref{ARM_leaf_example2}, 
\myref{ARM_leaf_example3}, \myref{ARM_leaf_example4}, \myref{ARM_leaf_example5},
\myref{ARM_leaf_example6}, \myref{ARM_leaf_example7}, \myref{ARM_leaf_example10}.

\subsubsection{関数の引数を渡す}

x86でパラメータを渡す最も一般的な方法は、\q{cdecl}です。

\begin{lstlisting}[style=customasmx86]
push arg3
push arg2
push arg1
call f
add esp, 12 ; 4*3=12
\end{lstlisting}

\gls{callee}関数はスタックポインタを介して引数を取得します。

したがって、 \ttf{} 関数の最初の命令が実行される前に、引数の値がスタックにどのように格納されているかがわかります。

\begin{center}
\begin{tabular}{ | l | l | }
\hline
ESP & return address \\
\hline
ESP+4 & \argument \#1, \MarkedInIDAAs{} \TT{arg\_0} \\
\hline
ESP+8 & \argument \#2, \MarkedInIDAAs{} \TT{arg\_4} \\
\hline
ESP+0xC & \argument \#3, \MarkedInIDAAs{} \TT{arg\_8} \\
\hline
\dots & \dots \\
\hline
\end{tabular}
\end{center}

他の呼び出し規約の詳細については、セクション(\myref{sec:callingconventions})も参照してください。

\par
ちなみに、\gls{callee}関数には、渡された引数の数に関する情報はありません。
( \printf のような)可変数の引数を持つC関数は、フォーマット文字列指定子(\%記号で始まる)を使ってその数を決定します。
% to be sync: C functions with a variable number of arguments (like \printf) can determine their number using format string specifiers (which begin with the \% symbol).

私たちが次のように書くとします。

\begin{lstlisting}
printf("%d %d %d", 1234);
\end{lstlisting}

\printf は1234を出力し、次にそのスタックの隣にある2つの乱数\footnote{厳密な意味でランダムではなく、むしろ予測不可能: \myref{noise_in_stack}}を出力します。

\label{main_arguments}
\par
だから、\main 関数を宣言する方法はあまり重要ではありません: \main
\TT{main(int argc, char *argv[])} または \TT{main(int argc, char *argv[], char *envp[])}のいずれかです。

実際、\ac{CRT}コードは \main を以下のように呼び出しています:
	
\begin{lstlisting}[style=customasmx86]
push envp
push argv
push argc
call main
...
\end{lstlisting}

引数なしで \main を \main として宣言すると、\main はスタックにまだ残っていますが使用されません。 
\main を\TT{main(int argc, char *argv[])}として宣言すると、
最初の2つの引数を使用することができ、3つ目の引数は関数の \q{不可視}のままになります。 
さらに、\TT{main(int argc)}を宣言することも可能です。これは動作します。

% TBT Another related example: \ref{cdecl_DLL}.

\myparagraph{引数を渡す別の方法}

プログラマがスタックを介して引数を渡すことは何も必要ではないことは注目に値する。
それは要件ではありません。 スタックをまったく使用せずに他の方法を実装することもできます。

アセンブリ言語初心者の間でやや普及している方法は、グローバル変数を介して引数を渡すことです

\lstinputlisting[caption=Assembly code,style=customasmx86]{patterns/02_stack/global_args.asm}

しかし、このメソッドには明白な欠点があります。\emph{do\_something()}関数は、独自の引数をzapする必要があるため、
再帰的に(または別の関数を介して)呼び出すことはできません。 
ローカル変数を使った同じ話:グローバル変数でそれらを保持すると、関数は自分自身を呼び出すことができませんでした。 
また、これはスレッドセーフ
\footnote{正しく実装され、各スレッドは独自の引数/変数を持つ独自のスタックを持ちます}
ではありません。このような情報をスタックに格納する方法は、これをより簡単にします。多くの関数の引数や値、スペースを確保できます。

\InSqBrackets{\TAOCPvolI{}, 189}は、IBM System/360上で特に便利な奇妙なスキームについても言及しています。

\myindex{MS-DOS}
\myindex{x86!\Instructions!INT}

MS-DOSには、レジスタを介してすべての関数引数を渡す方法がありました。
たとえば、古代16ビットMS-DOSの ``Hello, world!''コードのコードです。

\begin{lstlisting}[style=customasmx86]
mov  dx, msg      ; address of message
mov  ah, 9        ; 9 means "print string" function
int  21h          ; DOS "syscall"

mov  ah, 4ch      ; "terminate program" function
int  21h          ; DOS "syscall"

msg  db 'Hello, World!\$'
\end{lstlisting}

\myindex{fastcall}
これは、\myref{fastcall}のメソッドと非常によく似ています。 
また、Linuxのsyscalls((\myref{linux_syscall}))とWindowsを呼び出すのと非常によく似ています。

\myindex{x86!\Flags!CF}
MS-DOS関数がブール値(すなわち単一ビット、通常はエラー状態を示す)を返す場合、\TT{CF}フラグがしばしば使用されます。

例えば:

\begin{lstlisting}[style=customasmx86]
mov ah, 3ch       ; create file
lea dx, filename
mov cl, 1
int 21h
jc  error
mov file_handle, ax
...
error:
...
\end{lstlisting}

エラーの場合、\TT{CF}フラグが立てられます。 それ以外の場合は、新しく作成されたファイルのハンドルが\TT{AX}を介して返されます。

このメソッドは、アセンブリ言語プログラマによって引き続き使用されます。 
Windows Research Kernelのソースコード(Windows 2003と非常に似ています)では、次のようなものが見つかります

(ファイル \emph{base/ntos/ke/i386/cpu.asm})

\begin{lstlisting}[style=customasmx86]
        public  Get386Stepping
Get386Stepping  proc

        call    MultiplyTest            ; Perform multiplication test
        jnc     short G3s00             ; if nc, muttest is ok
        mov     ax, 0
        ret
G3s00:
        call    Check386B0              ; Check for B0 stepping
        jnc     short G3s05             ; if nc, it's B1/later
        mov     ax, 100h                ; It is B0/earlier stepping
        ret

G3s05:
        call    Check386D1              ; Check for D1 stepping
        jc      short G3s10             ; if c, it is NOT D1
        mov     ax, 301h                ; It is D1/later stepping
        ret

G3s10:
        mov     ax, 101h                ; assume it is B1 stepping
        ret

	...

MultiplyTest    proc

        xor     cx,cx                   ; 64K times is a nice round number
mlt00:  push    cx
        call    Multiply                ; does this chip's multiply work?
        pop     cx
        jc      short mltx              ; if c, No, exit
        loop    mlt00                   ; if nc, YEs, loop to try again
        clc
mltx:
        ret

MultiplyTest    endp
\end{lstlisting}


\input{patterns/02_stack/03_local_vars_JA}
\EN{\mysection{Returning Values: redux}

Again, when we know about function prologue and epilogue, let's recompile an example returning a value
(\ref{ret_val_func}, \ref{lst:ret_val_func}) using non-optimizing GCC:

\lstinputlisting[caption=\NonOptimizing GCC 8.2 x64 (\assemblyOutput),style=customasmx86]{patterns/017_ret_redux/1.s}

Effective instructions here are \INS{MOV} and \INS{RET}, others are -- prologue and epilogue.

}

\input{patterns/02_stack/05_SEH}
\ifdefined\ENGLISH
\subsubsection{Buffer overflow protection}

More about it here~(\myref{subsec:bufferoverflow}).
\fi

\ifdefined\RUSSIAN
\subsubsection{Защита от переполнений буфера}

Здесь больше об этом~(\myref{subsec:bufferoverflow}).
\fi

\ifdefined\BRAZILIAN
\subsubsection{Proteção contra estouro de buffer}

Mais sobre aqui~(\myref{subsec:bufferoverflow}).
\fi

\ifdefined\ITALIAN
\subsubsection{Protezione contro buffer overflow}

Maggiori informazioni qui~(\myref{subsec:bufferoverflow}).
\fi

\ifdefined\FRENCH
\subsubsection{Protection contre les débordements de tampon}

Lire à ce propos~(\myref{subsec:bufferoverflow}).
\fi


\ifdefined\POLISH
\subsubsection{Ochrona przed przepełnieniem bufora}

Więcej o tym tutaj~(\myref{subsec:bufferoverflow}).
\fi

\ifdefined\JAPANESE
\subsubsection{バッファオーバーフロー保護}

詳細はこちら~(\myref{subsec:bufferoverflow})
\fi


\subsubsection{スタック内のデータの自動解放}

おそらく、ローカル変数とSEHレコードをスタックに格納する理由は、スタックポインタを修正するための命令を1つだけ使用して(通常は \ADD です)、
関数が終了すると自動的に解放されるからです。 
関数の引数は、関数の終わりに自動的に割り当て解除されます。 
対照的に、ヒープに格納されているものはすべて明示的に割り当て解除する必要があります。

% subsections
\input{patterns/02_stack/07_layout_JA}
\EN{\mysection{Returning Values: redux}

Again, when we know about function prologue and epilogue, let's recompile an example returning a value
(\ref{ret_val_func}, \ref{lst:ret_val_func}) using non-optimizing GCC:

\lstinputlisting[caption=\NonOptimizing GCC 8.2 x64 (\assemblyOutput),style=customasmx86]{patterns/017_ret_redux/1.s}

Effective instructions here are \INS{MOV} and \INS{RET}, others are -- prologue and epilogue.

}

\input{patterns/02_stack/exercises}

