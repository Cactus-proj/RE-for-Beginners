\mysection{\Stack}
\label{sec:stack}
\myindex{\Stack}

La pile est une des structures de données les plus fondamentales en informatique
\footnote{\href{http://en.wikipedia.org/wiki/Call_stack}{wikipedia.org/wiki/Call\_stack}}.
\ac{AKA} \ac{LIFO}.

Techniquement, il s'agit d'un bloc de mémoire situé dans l'espace d'adressage
d'un processus et qui est utilisé par le registre \ESP en x86, \RSP en x64
ou par le registre \ac{SP} en ARM comme un pointeur dans ce bloc mémoire.

\myindex{ARM!\Instructions!PUSH}
\myindex{ARM!\Instructions!POP}
\myindex{x86!\Instructions!PUSH}
\myindex{x86!\Instructions!POP}
Les instructions d'accès à la pile sont \PUSH et \POP (en x86 ainsi qu'en ARM Thumb-mode).
\PUSH soustrait à \ESP/\RSP/\ac{SP} 4 en mode 32-bit (ou 8 en mode 64-bit) et écrit
ensuite le contenu de l'opérande associé à l'adresse mémoire pointée par \ESP/\RSP/\ac{SP}.

\POP est l'opération inverse: elle récupère la donnée depuis l'adresse mémoire pointée par \ac{SP},
l'écrit dans l'opérande associé (souvent un registre) puis ajoute 4 (ou 8) au \glslink{stack pointer}{pointeur de pile}.

Après une allocation sur la pile, le \glslink{stack pointer}{pointeur de pile} pointe sur le bas de la pile.
\PUSH décrémente le \glslink{stack pointer}{pointeur de pile} et \POP l'incrémente.

Le bas de la pile représente en réalité le début de la mémoire allouée pour
le bloc de pile. Cela semble étrange, mais c'est comme ça.

ARM supporte à la fois les piles ascendantes et descendantes.

\myindex{ARM!\Instructions!STMFD}
\myindex{ARM!\Instructions!LDMFD}
\myindex{ARM!\Instructions!STMED}
\myindex{ARM!\Instructions!LDMED}
\myindex{ARM!\Instructions!STMFA}
\myindex{ARM!\Instructions!LDMFA}
\myindex{ARM!\Instructions!STMEA}
\myindex{ARM!\Instructions!LDMEA}

Par exemple les instructions \ac{STMFD}/\ac{LDMFD}, \ac{STMED}/\ac{LDMED} sont utilisées pour gérer les piles
descendantes (qui grandissent vers le bas en commençant avec une adresse haute et évoluent vers une plus basse).

Les instructions \ac{STMFA}/\ac{LDMFA}, \ac{STMEA}/\ac{LDMEA} sont utilisées pour gérer les piles montantes
(qui grandissent vers les adresses hautes de l'espace d'adressage, en commençant
avec une adresse située en bas de l'espace d'adressage).

% It might be worth mentioning that STMED and STMEA write first,
% and then move the pointer,
% and that LDMED and LDMEA move the pointer first, and then read.
% In other words, ARM not only lets the stack grow in a non-standard direction,
% but also in a non-standard order.
% Maybe this can be in the glossary, which would explain why E stands for "empty".

\subsection{Pourquoi la pile grandit en descendant ?}
\label{stack_grow_backwards}

Intuitivement, on pourrait penser que la pile grandit vers le haut, i.e. vers des
adresses plus élevées, comme n'importe qu'elle autre structure de données.

La raison pour laquelle la pile grandit vers le bas est probablement historique.
Dans le passé, les ordinateurs étaient énormes et occupaient des pièces entières,
il était facile de diviser la mémoire en deux parties, une pour le \gls{heap} et
une pour la pile.
Évidemment, on ignorait quelle serait la taille du \gls{heap} et de la pile durant
l'exécution du programme, donc cette solution était la plus simple possible.

\input{patterns/02_stack/stack_and_heap}

Dans \RitchieThompsonUNIX on peut lire:

\begin{framed}
\begin{quotation}
The user-core part of an image is divided into three logical segments. The program text segment begins at location 0 in the virtual address space. During execution, this segment is write-protected and a single copy of it is shared among all processes executing the same program. At the first 8K byte boundary above the program text segment in the virtual address space begins a nonshared, writable data segment, the size of which may be extended by a system call. Starting at the highest address in the virtual address space is a pile segment, which automatically grows downward as the hardware's pile pointer fluctuates.
\end{quotation}
\end{framed}

Cela nous rappelle comment certains étudiants prennent des notes pour deux cours différents dans
un seul et même cahier en prenant un cours d'un côté du cahier, et l'autre cours de l'autre côté.
Les notes de cours finissent par se rencontrer à un moment dans le cahier quand il n'y a plus de place.

% I think if we want to expand on this analogy,
% one might remember that the line number increases as as you go down a page.
% So when you decrease the address when pushing to the stack, visually,
% the stack does grow upwards.
% Of course, the problem is that in most human languages,
% just as with computers,
% we write downwards, so this direction is what makes buffer overflows so messy.

\subsection{Quel est le rôle de la pile ?}

% subsections
\subsubsection{Sauvegarder l'adresse de retour de la fonction}

\myparagraph{x86}

\myindex{x86!\Instructions!CALL}
Lorsque l'on appelle une fonction avec une instruction \CALL, l'adresse du point
exactement après cette dernière est sauvegardée sur la pile et un saut inconditionnel
à l'adresse de l'opérande \CALL est exécuté.

\myindex{x86!\Instructions!PUSH}
\myindex{x86!\Instructions!JMP}
L'instruction \CALL est équivalente à la paire d'instructions\\
\INS{PUSH address\_after\_call / JMP operand}.

\myindex{x86!\Instructions!RET}
\myindex{x86!\Instructions!POP}
\RET va chercher une valeur sur la pile et y saute~---ce qui est équivalent à
la paire d'instructions \TT{POP tmp / JMP tmp}.

\myindex{\Stack!\MLStackOverflow}
\myindex{\Recursion}
Déborder de la pile est très facile. Il suffit de lancer une récursion éternelle:

\begin{lstlisting}[style=customc]
void f()
{
	f();
};
\end{lstlisting}

MSVC 2008 signale le problème:

\begin{lstlisting}
c:\tmp6>cl ss.cpp /Fass.asm
Microsoft (R) 32-bit C/C++ Optimizing Compiler Version 15.00.21022.08 for 80x86
Copyright (C) Microsoft Corporation.  All rights reserved.

ss.cpp
c:\tmp6\ss.cpp(4) : warning C4717: 'f' : recursive on all control paths, function will cause runtime stack overflow
\end{lstlisting}

\dots mais génère tout de même le code correspondant:

\lstinputlisting[style=customasmx86]{patterns/02_stack/1.asm}

\dots Si nous utilisons l'option d'optimisation du compilateur (option \TT{\Ox})
le code optimisé ne va pas déborder de la pile et au lieu de cela va fonctionner
\emph{correctemment}\footnote{ironique ici}:

\lstinputlisting[style=customasmx86]{patterns/02_stack/2.asm}

GCC 4.4.1 génère un code similaire dans les deux cas, sans, toutefois émettre
d'avertissement à propos de ce problème.

\myparagraph{ARM}

\myindex{ARM!\Registers!Link Register}
Les programmes ARM utilisent également la pile pour sauver les adresses de retour,
mais différemment.
Comme mentionné dans \q{\HelloWorldSectionName}~(\myref{sec:hw_ARM}),
\ac{RA} est sauvegardé dans \ac{LR} (\gls{link register}).
Si l'on a toutefois besoin d'appeler une autre fonction et d'utiliser le registre
\ac{LR} une fois de plus, sa valeur doit être sauvegardée.
\myindex{Function prologue}
Usuellement, cela se fait dans le prologue de la fonction.

\myindex{ARM!\Instructions!PUSH}
\myindex{ARM!\Instructions!POP}
Souvent, nous voyons des instructions comme \INS{PUSH {R4-R7,LR}} en même temps
que cette instruction dans l'épilogue \INS{POP {R4-R7,PC}}---ces registres qui
sont utilisés dans la fonction sont sauvegardés sur la pile, \ac{LR} inclus.

\myindex{ARM!Fonction leaf} % FIXME traduire avec feuille ?
Néanmoins, si une fonction n'appelle jamais d'autre fonction, dans la terminologie
\ac{RISC} elle est appelée \emph{\glslink{leaf function}{fonction leaf}}\footnote{\href{http://infocenter.arm.com/help/index.jsp?topic=/com.arm.doc.faqs/ka13785.html}{infocenter.arm.com/help/index.jsp?topic=/com.arm.doc.faqs/ka13785.html}}.
Ceci a comme conséquence que les fonctions leaf ne sauvegardent pas le registre
\ac{LR} (car elles ne le modifient pas).
Si une telle fonction est petite et utilise un petit nombre de registres, elle
peut ne pas utiliser du tout la pile.
Ainsi, il est possible d'appeler des fonctions leaf sans utiliser la pile.
Ce qui peut être plus rapide sur des vieilles machines x86 car la mémoire externe
n'est pas utilisée pour la pile
\footnote{Il y a quelque temps, sur PDP-11 et VAX, l'instruction CALL (appel d'autres fonctions) était coûteuse; jusqu'à 50\%
du temps d'exécution pouvait être passé à ça, il était donc considéré qu'avoir un grand nombre de petites fonctions était un \gls{anti-pattern} \InSqBrackets{\TAOUP Chapter 4, Part II}.}.
Cela peut être utile pour des situations où la mémoire pour la pile n'est pas
encore allouée ou disponible.

Quelques exemples de fonctions leaf:
\myref{ARM_leaf_example1}, \myref{ARM_leaf_example2},
\myref{ARM_leaf_example3}, \myref{ARM_leaf_example4}, \myref{ARM_leaf_example5},
\myref{ARM_leaf_example6}, \myref{ARM_leaf_example7}, \myref{ARM_leaf_example10}.


\subsubsection{Passage des arguments d'une fonction}

Le moyen le plus utilisé pour passer des arguments en x86 est appelé \q{cdecl}:

\begin{lstlisting}[style=customasmx86]
push arg3
push arg2
push arg1
call f
add esp, 12 ; 4*3=12
\end{lstlisting}

La fonction \glslink{callee}{appelée} reçoit ses arguments par la pile.

Voici donc comment sont stockés les arguments sur la pile avant l'exécution
de la première instruction de la fonction \ttf{}:

\begin{center}
\begin{tabular}{ | l | l | }
\hline
ESP & return address \\
\hline
ESP+4 & \argument \#1, \MarkedInIDAAs{} \TT{arg\_0} \\
\hline
ESP+8 & \argument \#2, \MarkedInIDAAs{} \TT{arg\_4} \\
\hline
ESP+0xC & \argument \#3, \MarkedInIDAAs{} \TT{arg\_8} \\
\hline
\dots & \dots \\
\hline
\end{tabular}
\end{center}

Pour plus d'information sur les conventions d'appel, voir cette section~(\myref{sec:callingconventions}).

\par
À propos, la fonction \glslink{callee}{appelée} n'a aucune d'information sur le
nombre d'arguments qui ont été passés.
Les fonctions C avec un nombre variable d'arguments (comme \printf) peuvent déterminer
leur nombre en utilisant les spécificateurs de la chaîne de format (qui commencent
pas le symbole \%).

Si nous écrivons quelque comme:

\begin{lstlisting}
printf("%d %d %d", 1234);
\end{lstlisting}

\printf va afficher 1234, et deux autres nombres aléatoires\footnote{Pas aléatoire
dans le sens strict du terme, mais plutôt imprévisibles: \myref{noise_in_stack}},
qui sont situés à côté dans la pile.

\label{main_arguments}
\par
C'est pourquoi la façon dont la fonction \main est déclarée n'est pas très importante:
comme \main, \\\TT{main(int argc, char *argv[])} ou \TT{main(int argc, char *argv[], char *envp[])}.

En fait, le code-\ac{CRT} appelle \main, schématiquement, de cette façon:
	
\begin{lstlisting}[style=customasmx86]
push envp
push argv
push argc
call main
...
\end{lstlisting}

Si vous déclarez \main comme \main sans argument, ils sont néanmoins toujours présents
sur la pile, mais ne sont pas utilisés.
Si vous déclarez \main as comme \TT{main(int argc, char *argv[])},
vous pourrez utiliser les deux premiers arguments, et le troisième restera \q{invisible}
pour votre fonction.
Il est même possible de déclarer \main comme \TT{main(int argc)}, cela fonctionnera.

Un autre exemple apparenté: \ref{cdecl_DLL}.

\myparagraph{Autres façons de passer les arguments}

Il est à noter que rien n'oblige les programmeurs à passer les arguments à travers
la pile. Ce n'est pas une exigence.
On peut implémenter n'importe quelle autre méthode sans utiliser du tout la pile.

Une méthode répandue chez les débutants en assembleur est de passer les arguments
par des variables globales, comme:

\lstinputlisting[caption=Code assembleur,style=customasmx86]{patterns/02_stack/global_args.asm}

Mais cette méthode a un inconvénient évident: la fonction \emph{do\_something()}
ne peut pas s'appeler elle-même récursivement (ou par une autre fonction),
car il faudrait écraser ses propres arguments.
La même histoire avec les variables locales: si vous les stockez dans des variables
globales, la fonction ne peut pas s'appeler elle-même.
Et ce n'est pas thread-safe
\footnote{Correctement implémenté, chaque thread aurait sa propre pile avec ses propres arguments/variables.}.
Une méthode qui stocke ces informations sur la pile rend cela plus facile---elle
peut contenir autant d'arguments de fonctions et/ou de valeurs, que la pile a d'espace.

\InSqBrackets{\TAOCPvolI{}, 189} mentionne un schéma encore plus étrange, particulièrement
pratique sur les IBM System/360.

\myindex{MS-DOS}
\myindex{x86!\Instructions!INT}

MS-DOS a une manière de passer tous les arguments de fonctions via des registres,
par exemple, c'est un morceau de code pour un ancien MS-DOS 16-bit qui affiche
``Hello, world!'':

\begin{lstlisting}[style=customasmx86]
mov  dx, msg      ; address of message
mov  ah, 9        ; 9 means "print string" function
int  21h          ; DOS "syscall"

mov  ah, 4ch      ; "terminate program" function
int  21h          ; DOS "syscall"

msg  db 'Hello, World!\$'
\end{lstlisting}

\myindex{fastcall}
C'est presque similaire à la méthode \myref{fastcall}.
Et c'est aussi très similaire aux appels systèmes sous Linux (\myref{linux_syscall}) et Windows.

\myindex{x86!\Flags!CF}
Si une fonction MS-DOS devait renvoyer une valeur booléenne (i.e., un simple bit,
souvent pour indiquer un état d'erreur), le flag \TT{CF} était souvent utilisé.

Par exemple:

\begin{lstlisting}[style=customasmx86]
mov ah, 3ch       ; create file
lea dx, filename
mov cl, 1
int 21h
jc  error
mov file_handle, ax
...
error:
...
\end{lstlisting}

En cas d'erreur, le flag \TT{CF} est mis. Sinon, le handle du fichier nouvellement
créé est retourné via \TT{AX}.

Cette méthode est encore utilisée par les programmeurs en langage d'assemblage.
Dans le code source de Windows Research Kernel (qui est très similaire à Windows
2003) nous pouvons trouver quelque chose comme ça (file \emph{base/ntos/ke/i386/cpu.asm}):

\begin{lstlisting}[style=customasmx86]
        public  Get386Stepping
Get386Stepping  proc

        call    MultiplyTest            ; Perform multiplication test
        jnc     short G3s00             ; if nc, muttest is ok
        mov     ax, 0
        ret
G3s00:
        call    Check386B0              ; Check for B0 stepping
        jnc     short G3s05             ; if nc, it's B1/later
        mov     ax, 100h                ; It is B0/earlier stepping
        ret

G3s05:
        call    Check386D1              ; Check for D1 stepping
        jc      short G3s10             ; if c, it is NOT D1
        mov     ax, 301h                ; It is D1/later stepping
        ret

G3s10:
        mov     ax, 101h                ; assume it is B1 stepping
        ret

	...

MultiplyTest    proc

        xor     cx,cx                   ; 64K times is a nice round number
mlt00:  push    cx
        call    Multiply                ; does this chip's multiply work?
        pop     cx
        jc      short mltx              ; if c, No, exit
        loop    mlt00                   ; if nc, YEs, loop to try again
        clc
mltx:
        ret

MultiplyTest    endp
\end{lstlisting}


\input{patterns/02_stack/03_local_vars_FR}
\EN{\mysection{Returning Values: redux}

Again, when we know about function prologue and epilogue, let's recompile an example returning a value
(\ref{ret_val_func}, \ref{lst:ret_val_func}) using non-optimizing GCC:

\lstinputlisting[caption=\NonOptimizing GCC 8.2 x64 (\assemblyOutput),style=customasmx86]{patterns/017_ret_redux/1.s}

Effective instructions here are \INS{MOV} and \INS{RET}, others are -- prologue and epilogue.

}

\input{patterns/02_stack/05_SEH}
\ifdefined\ENGLISH
\subsubsection{Buffer overflow protection}

More about it here~(\myref{subsec:bufferoverflow}).
\fi

\ifdefined\RUSSIAN
\subsubsection{Защита от переполнений буфера}

Здесь больше об этом~(\myref{subsec:bufferoverflow}).
\fi

\ifdefined\BRAZILIAN
\subsubsection{Proteção contra estouro de buffer}

Mais sobre aqui~(\myref{subsec:bufferoverflow}).
\fi

\ifdefined\ITALIAN
\subsubsection{Protezione contro buffer overflow}

Maggiori informazioni qui~(\myref{subsec:bufferoverflow}).
\fi

\ifdefined\FRENCH
\subsubsection{Protection contre les débordements de tampon}

Lire à ce propos~(\myref{subsec:bufferoverflow}).
\fi


\ifdefined\POLISH
\subsubsection{Ochrona przed przepełnieniem bufora}

Więcej o tym tutaj~(\myref{subsec:bufferoverflow}).
\fi

\ifdefined\JAPANESE
\subsubsection{バッファオーバーフロー保護}

詳細はこちら~(\myref{subsec:bufferoverflow})
\fi


\subsubsection{Dé-allocation automatique de données dans la pile}

Peut-être que la raison pour laquelle les variables locales et les enregistrements SEH sont stockés dans la
pile est qu'ils sont automatiquement libérés quand la fonction se termine en utilisant simplement une
instruction pour corriger la position du pointeur de pile (souvent \ADD).
Les arguments de fonction sont aussi désalloués automatiquement à la fin de la fonction.
À l'inverse, toutes les données allouées sur le \emph{heap} doivent être désallouées de façon explicite.

% subsections
\input{patterns/02_stack/07_layout_FR}
\EN{\mysection{Returning Values: redux}

Again, when we know about function prologue and epilogue, let's recompile an example returning a value
(\ref{ret_val_func}, \ref{lst:ret_val_func}) using non-optimizing GCC:

\lstinputlisting[caption=\NonOptimizing GCC 8.2 x64 (\assemblyOutput),style=customasmx86]{patterns/017_ret_redux/1.s}

Effective instructions here are \INS{MOV} and \INS{RET}, others are -- prologue and epilogue.

}

\input{patterns/02_stack/exercises}

