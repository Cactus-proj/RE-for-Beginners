\mysection{Impossible shutdown practical joke (Windows 7)}

\myindex{Windows 98}
\myindex{user32.dll}
I don't quite remember how I found the \verb|ExitWindowsEx()| function in Windows 98's (it was late 1990s) user32.dll file.
Probably, I just spotted its self-describing name.
And then I tried to \emph{block} it by patching its beginning by 0xC3 byte (\INS{RETN}).

The result was funny: Windows 98 cannot be shutted down anymore.
Had to press reset button.

These days I tried to do the same in Windows 7, that was created almost 10 years later and based on completely different Windows NT
base.
Still, \verb|ExitWindowsEx()| function present in user32.dll file and serves the same purpose.

First, I turned off \emph{Windows File Protection} by adding this to registry
(Windows would silently restore modified system files otherwise):

\myindex{Windows File Protection}
\begin{lstlisting}
Windows Registry Editor Version 5.00

[HKEY_LOCAL_MACHINE\SOFTWARE\Microsoft\Windows NT\CurrentVersion\Winlogon]
"SFCDisable"=dword:ffffff9d
\end{lstlisting}

\myindex{Hiew}
\myindex{IDA}
Then I renamed \verb|c:\windows\system32\user32.dll| to \verb|user32.dll.bak|.
I found \verb|ExitWindowsEx()| export entry using Hiew (IDA can help as well) and put 0xC3 byte here.
I restarted by Windows 7 and now it can't be shutted down.
"Restart" and "Logoff" buttons don't work anymore.

I don't know if it's funny today or not, but back then, in late 1990s, my friend took patched user32.dll file
on a floppy diskette and copied it to all the computers
(within his reach, that worked under Windows 98 (almost all))
at his university.
No Windows can be shutted down after and his computer science teacher was lurid.
(Hopefully he can forgive us if he is reading this right now.)

% A note to translators: the "lurid" word here is in this sense:
% "lighted or shining with an unnatural, fiery glow; wildly or garishly red:" ( https://www.dictionary.com/browse/lurid )

If you do this, backup everything.
The best idea is to run Windows under a virtual machine.

