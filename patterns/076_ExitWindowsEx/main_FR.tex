\mysection{Blague de l'arrêt impossible (Windows 7)}

\myindex{Windows 98}
\myindex{user32.dll}
Je ne me rappelle pas vraiment comment j'ai découvert la fonction \verb|ExitWindowsEx()|
dans le fichier user32.dll de Windows 98 (c'était à la fin des années 1990)
J'ai probablement juste remarqué le nom évocateur.
Et j'ai ensuite essayé de la \emph{bloquer} en modifiant son début par l'octet 0xC3 byte (\INS{RETN}).

Le résultat était rigolo: Windows 98 ne pouvait plus être arrêté.
Il fallait appuyer sur le bouton reset.

Ces jours, j'ai essayé de faire de même dans Windows 7, qui a été créé presque
10 ans après et qui est basé sur une base Windows NT complètement différente.
Quand même, la fonction \verb|ExitWindowsEx()| est présente dans le fichier
user32.dll et sert à la même chose.

Premièrement, j'ai arrêté \emph{Windows File Protection} en ajoutant ceci dans la
base de registres (sinon Windows restaurerai silencieusement les fichiers système
modifiés):

\myindex{Windows File Protection}
\begin{lstlisting}
Windows Registry Editor Version 5.00

[HKEY_LOCAL_MACHINE\SOFTWARE\Microsoft\Windows NT\CurrentVersion\Winlogon]
"SFCDisable"=dword:ffffff9d
\end{lstlisting}

\myindex{Hiew}
\myindex{IDA}
Puis, j'ai renommé \verb|c:\windows\system32\user32.dll| en \verb|user32.dll.bak|.
J'ai trouvé l' entrée de l'export \verb|ExitWindowsEx()| en utilisant Hiew (\IDA
peut aider de même) et y ai mis l'octet 0xC3.
J'ai redémarré Windows 7 et maintenant on ne peut plus l'arrêter.
Les boutons "Restart" et "Logoff" ne fonctionnent plus.

Je ne sais pas si c'est rigolo aujourd'hui ou pas, mais dans le passé, à la fin
des années 1990, mon ami a pris le fichier user32.dll patché sur une disquette
et l'a copié sur tous les ordinateurs (à portée de main, qui fonctionnaient sous
Windows 98 (presque tous)) de son université.
Plus aucun ordinateur sous Windows ne pouvait être arrêté après et son professeur
d'informatique était rouge.
(Espérons qu'il puisse nous pardonner aujourd'hui s'il lit ceci maintenant.)

Si vous faîtes ça, sauvegardez tout.
La meilleure idée est de lancer Windows dans une machine virtuelle.

