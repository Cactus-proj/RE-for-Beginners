\mysection{\FPUChapterName}
\label{sec:FPU}

\newcommand{\FNURLSTACK}{\footnote{\href{http://go.yurichev.com/17123}{wikipedia.org/wiki/Stack\_machine}}}
\newcommand{\FNURLFORTH}{\footnote{\href{http://go.yurichev.com/17124}{wikipedia.org/wiki/Forth\_(programming\_language)}}}
\newcommand{\FNURLIEEE}{\footnote{\href{http://go.yurichev.com/17125}{wikipedia.org/wiki/IEEE\_floating\_point}}}
\newcommand{\FNURLSP}{\footnote{\href{http://go.yurichev.com/17126}{wikipedia.org/wiki/Single-precision\_floating-point\_format}}}
\newcommand{\FNURLDP}{\footnote{\href{http://go.yurichev.com/17127}{wikipedia.org/wiki/Double-precision\_floating-point\_format}}}
\newcommand{\FNURLEP}{\footnote{\href{http://go.yurichev.com/17128}{wikipedia.org/wiki/Extended\_precision}}}

Le \ac{FPU} est un dispositif à l'intérieur du \ac{CPU}, spécialement conçu pour
traiter les nombres à virgules flottantes.

Il était appelé \q{coprocesseur} dans le passé et il était en dehors du \ac{CPU}.

\subsection{IEEE 754}

Un nombre au format IEEE 754 consiste en un \emph{signe}, un \emph{significande} (aussi
appelé \emph{fraction}) et un \emph{exposant}.

\subsection{x86}

Ça vaut la peine de jeter un \oe{}il sur les machines à base de piles\FNURLSTACK ou
d'apprendre les bases du langage Forth\FNURLFORTH, avant d'étudier le \ac{FPU} en
x86.

\myindex{Intel!80486}
\myindex{Intel!FPU}
Il est intéressant de savoir que dans le passé (avant le CPU 80486) le coprocesseur
était une puce séparée et n'était pas toujours préinstallé sur la carte mère. Il
était possible de l'acheter séparément et de l'installer\footnote{Par exemple, John
Carmack a utilisé des valeurs arithmétiques à virgule fixe
(\href{http://go.yurichev.com/17356}{wikipedia.org/wiki/Fixed-point\_arithmetic})
dans son jeu vidéo Doom, stockées dans des registres 32-bit \ac{GPR} (16 bit pour
la partie entière et 16 bit pour la partie fractionnaire), donc Doom pouvait fonctionner
sur des ordinateurs 32-bit sans FPU, i.e., 80386 et 80486 SX.}.

A partir du CPU 80486 DX, le \ac{FPU} est intégré dans le \ac{CPU}.

\myindex{x86!\Instructions!FWAIT}
L'instruction \INS{FWAIT} nous rappelle le fait qu'elle passe le \ac{CPU} dans un
état d'attente, jusqu'à ce que le \ac{FPU} ait fini son traitement.

Un autre rudiment est le fait que les opcodes d'instruction \ac{FPU} commencent
avec ce qui est appelé l'opcode-\q{d'échappement} (\GTT{D8..DF}), i.e., opcodes
passés à un coprocesseur séparé.

\myindex{IEEE 754}
\label{FPU_is_stack}

Le FPU a une pile capable de contenir 8 registres de 80-bit, et chaque registre peut
contenir un nombre au format IEEE 754\FNURLIEEE.

Ce sont \ST{0}..\ST{7}. Par concision, \IDA et \olly montrent \ST{0} comme \GTT{ST},
qui est représenté dans certains livres et manuels comme \q{Stack Top}.

\subsection{ARM, MIPS, x86/x64 SIMD}

En ARM et MIPS le FPU n'a pas de pile, mais un ensemble de registres, qui peuvent
être accédés aléatoirement, comme \ac{GPR}.

La même idéologie est utilisée dans l'extension SIMD des CPUs x86/x64.

\subsection{\CCpp}

\myindex{float}
\myindex{double}

Le standard des langages \CCpp offre au moins deux types de nombres à virgule flottante,
\Tfloat (\emph{simple-précision}\FNURLSP, 32 bits) \footnote{le format des nombres
à virgule flottante simple précision est aussi abordé dans la section \emph{\WorkingWithFloatAsWithStructSubSubSectionName}~(\myref{sec:floatasstruct})}
et \Tdouble  (\emph{double-précision}\FNURLDP, 64 bits).

Dans \InSqBrackets{\TAOCPvolII 246} nous pouvons trouver que \emph{simple-précision}
signifie que la valeur flottante peut être stockée dans un simple mot machine [32-bit],
\emph{double-précision} signifie qu'elle peut être stockée dans deux mots (64 bits).

\myindex{long double}

GCC supporte également le type \emph{long double} (\emph{précision étendue}\FNURLEP,
80 bit), que MSVC ne supporte pas.

Le type \Tfloat nécessite le même nombre de bits que le type \Tint dans les environnements
32-bit, mais la représentation du nombre est complètement différente.

\EN{\mysection{Returning Values: redux}

Again, when we know about function prologue and epilogue, let's recompile an example returning a value
(\ref{ret_val_func}, \ref{lst:ret_val_func}) using non-optimizing GCC:

\lstinputlisting[caption=\NonOptimizing GCC 8.2 x64 (\assemblyOutput),style=customasmx86]{patterns/017_ret_redux/1.s}

Effective instructions here are \INS{MOV} and \INS{RET}, others are -- prologue and epilogue.

}

\EN{\mysection{Returning Values: redux}

Again, when we know about function prologue and epilogue, let's recompile an example returning a value
(\ref{ret_val_func}, \ref{lst:ret_val_func}) using non-optimizing GCC:

\lstinputlisting[caption=\NonOptimizing GCC 8.2 x64 (\assemblyOutput),style=customasmx86]{patterns/017_ret_redux/1.s}

Effective instructions here are \INS{MOV} and \INS{RET}, others are -- prologue and epilogue.

}

\EN{\mysection{Returning Values: redux}

Again, when we know about function prologue and epilogue, let's recompile an example returning a value
(\ref{ret_val_func}, \ref{lst:ret_val_func}) using non-optimizing GCC:

\lstinputlisting[caption=\NonOptimizing GCC 8.2 x64 (\assemblyOutput),style=customasmx86]{patterns/017_ret_redux/1.s}

Effective instructions here are \INS{MOV} and \INS{RET}, others are -- prologue and epilogue.

}


\subsection{Quelques constantes}

Il est facile de trouver la représentation de certaines constantes pour des nombres
encodés au format IEEE 754 sur Wikipédia.
Il est intéressant de savoir que 0,0 en IEEE 754 est représenté par 32 bits à zéro
(pour la simple précision) ou 64 bits à zéro (pour la double).
Donc pour mettre une variable flottante à 0,0 dans un registre ou en mémoire, on
peut utiliser l'instruction \MOV ou \TT{XOR reg, reg}.
\myindex{\CStandardLibrary!memset()}
Ceci est utilisable pour les structures où des variables de types variés sont présentes.
Avec la fonction usuelle memset() il est possible de mettre toutes les variables
entières à 0, toutes les variables booléennes à \emph{false}, tous les pointeurs à
NULL, et toutes les variables flottantes (de n'importe quelle précision) à 0,0.

\subsection{Copie}

On peut tout d'abord penser qu'il faut utiliser les instructions \INS{FLD}/\INS{FST}
pour charger et stocker (et donc, copier) des valeurs IEEE 754.
Néanmoins, la même chose peut-être effectuée plus facilement avec l'instruction usuelle
\INS{MOV}, qui, bien sûr, copie les valeurs au niveau binaire.

\subsection{Pile, calculateurs et notation polonaise inverse}

\myindex{Notation polonaise inverse}

Maintenant nous comprenons pourquoi certains anciens calculateurs utilisent la notation
Polonaise inverse
\footnote{\href{http://go.yurichev.com/17354}{wikipedia.org/wiki/Reverse\_Polish\_notation}}.

Par exemple, pour additionner 12 et 34, on doit entrer 12, puis 34, et presser le
signe \q{plus}.

C'est parce que les anciens calculateurs programmable étaient simplement des implémentations
de machine à pile, et c'était bien plus simple que de manipuler des expressions
complexes avec parenthèses.

Un tel calculateur est encore présent dans de nombreuses distributions Unix: \emph{dc}.

\subsection{80 bits?}

\myindex{Punched card}

Représentation interne des nombres dans le FPU --- 80-bit.
Nombre étrange, car il n'est pas de la forme $2^n$.
Il y a une hypothèse que c'est probablement dû à des raisons historiques---le standard
IBM de carte perforée peut encoder 12 lignes de 80 bits.
La résolution en mode texte de $80\cdot 25$ était aussi très populaire dans le passé.

Il y a une autre explication sur Wikipédia: \url{https://en.wikipedia.org/wiki/Extended_precision}.

Si vous en savez plus, s'il vous plaît, envoyez-moi un email: \EMAIL{}.

\subsection{x64}

Sur la manière dont sont traités les nombres à virgules flottante en x86-64, lire
ici: \myref{floating_SIMD}.

% sections
\input{patterns/12_FPU/exercises}

