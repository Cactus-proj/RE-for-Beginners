\mysection{\FPUChapterName}
\label{sec:FPU}

Die \ac{FPU} is ein Gerät innerhalb der \ac{CPU}, welche speziell für den Umgang
mit Fließkommazahlen ausgelegt ist. 

In der Vergangenheit wurde die \ac{FPU} auch als \q{Koprozessor} bezeichnet und
sie befindet sich neben der \ac{CPU}. 

\subsection{IEEE 754}

Eine Zahl besteht gemäß IEEE 754 Format aus einem \emph{Vorzeichne}, einer
\emph{Mantisse} (auch \emph{Bruch} genannt) und einem \emph{Exponenten}. 

\subsection{x86}

Es lohnt sich einen Blick auf die Stackmaschine zu werfen oder die
Grundlagen der Sprache Forth zu erlernen, bevor man die \ac{FPU} in
x86 genauer untersucht. 

\myindex{Intel!80486}
\myindex{Intel!FPU}
Es ist interessant zu wissen, dass sich der Koprozessor in der Vergangenheit
(vor dem 80486 Chip) auf einem separaten Chip befand und nicht immer auf dem
Mainboard vorinstalliert war. Es war möglich, diesen separat zu kaufen und zu
installieren\footnote{John Carmack z.B. verwendete Fixpunktarithmetikwerte 
seinem Videospiel Doom, gespeichert in 32-bit \ac{GPR} Registern (16 Bit
für den Hauptteil und weitere 16 Bit für den gebrochenen Teil), sodass Doom
auf 32-Bit-Computern ohne FPU, d.h., 80386 und 80486 SX, lauffähig war.}.

Seit der 80486 DX CPU ist die \ac{FPU} in die \ac{CPU} integriert.

\myindex{x86!\Instructions!FWAIT}
Der Befehl \INS{FWAIT} erinnert uns an diese Tatsache--er lässt die \ac{CPU} in
einen Wartezustand wechseln, in dem sie verbleibt, bis die \ac{FPU} ihre Arbeit
beendet hat.

Ein anderes Überbleibsel ist die Tatsache, dass die Opcodes der \ac{FPU} Befehle
mit sogenannten \q{escape}-Opcodes (\GTT{D8..DF}, d.h. Opcodes, die an einen
separaten Koprozessor übergeben werden) beginnen.

\myindex{IEEE 754}
\label{FPU_is_stack}
Die FPU besitzt einen Stack, auf dem 8 80-bit-Register Platz finden, wobei jedes
Register eine Zahl im IEEE 754 Format aufnehmen kann.

Es gibt \ST{0}..\ST{7}. Verkürzend stellen \IDA und \olly \ST{0}, welches in
manchen (Hand-)büchern als \q{Stack Top} dargestellt wird, als \GTT{ST} dar.

\subsection{ARM, MIPS, x86/x64 SIMD}
In ARM und MIPS ist die FPU kein Stack, sondern ein Registersatz.
% TBT

Das gleiche Paradigma wird in den SIMD-Erweiterungen von x86/x64 CPUs verwendet.

\subsection{\CCpp}

\myindex{float}
\myindex{double}
Die Standardsparche \CCpp bietet zumindest two unterschiedliche
Fließkommezahlentypen, \Tfloat (\emph{einfache Genauigkeit}, 32 Bit)
\footnote{Fließkommazahlen mit einfacher Genauigkeit werden auch im Abschnitt
\emph{\WorkingWithFloatAsWithStructSubSubSectionName}~(\myref{sec:floatasstruct})
behandelt} und \Tdouble (\emph{doppelte Genauigkeit}, 64 Bit).

In \InSqBrackets{\TAOCPvolII 246} können wir nachlesen, dass \emph{einfache
Genauigkeit} bedeutet, dass die Fließkommazahl in ein einzelnes (32-Bit)-Wort
hineinpasst, \emph{doppelte Genauigkeit} bedeutet dagegen, dass die Zahl in zwei
Worten (64 Bit) abgelegt werden kann.

\myindex{long double}
GCC unterstützt im Gegensatz zu MSVC ebenfalls den \emph{long double} Typ
(\emph{erweiterte Genauigkeit} 80 Bit).

Der \Tfloat Typ benötigt dieselbe Anzahl an Bits wie der \Tint Typ in
32-Bit-Umgebungen, aber die Darstellung der Zahlen ist komplett verschieden
voneinander.

\EN{\mysection{Returning Values: redux}

Again, when we know about function prologue and epilogue, let's recompile an example returning a value
(\ref{ret_val_func}, \ref{lst:ret_val_func}) using non-optimizing GCC:

\lstinputlisting[caption=\NonOptimizing GCC 8.2 x64 (\assemblyOutput),style=customasmx86]{patterns/017_ret_redux/1.s}

Effective instructions here are \INS{MOV} and \INS{RET}, others are -- prologue and epilogue.

}

\EN{\mysection{Returning Values: redux}

Again, when we know about function prologue and epilogue, let's recompile an example returning a value
(\ref{ret_val_func}, \ref{lst:ret_val_func}) using non-optimizing GCC:

\lstinputlisting[caption=\NonOptimizing GCC 8.2 x64 (\assemblyOutput),style=customasmx86]{patterns/017_ret_redux/1.s}

Effective instructions here are \INS{MOV} and \INS{RET}, others are -- prologue and epilogue.

}

\EN{\mysection{Returning Values: redux}

Again, when we know about function prologue and epilogue, let's recompile an example returning a value
(\ref{ret_val_func}, \ref{lst:ret_val_func}) using non-optimizing GCC:

\lstinputlisting[caption=\NonOptimizing GCC 8.2 x64 (\assemblyOutput),style=customasmx86]{patterns/017_ret_redux/1.s}

Effective instructions here are \INS{MOV} and \INS{RET}, others are -- prologue and epilogue.

}


\subsection{Einige Konstanten}
Es ist leicht, in Wikipedia die Darstellungen einiger Konstanten nach dem IEEE
754 Standard nachzulesen. Es ist interessant zu wissen, dass 0.0 nach IEEE 754
als 32 Nullbits (in einfacher Genauigkeit) oder 64 Nullbits (in doppelter
Genauigkeit) dargestellt wird.
Wenn also eine Fließkommavariable im Register oder Speicher auf 0.0 gesetzt
werden soll, kann der Befehl \MOV oder \emph{XOR reg, reg} verwendet werden.
\myindex{\CStandardLibrary!memset()}
Dies ist geeignet für Strukturen, in denen viele Variable unterschiedlichster
Datentypen vorhanden sind. Mit der gewöhnlichen memset() Funktion kann man alle
Integervariablen auf 0, alle Booleschen Variablen auf \emph{false}, alle Pointer
auf NULL und alle Fließkommavariablen (beliebiger Genauigkeit) auf 0.0 setzen.

\subsection{Kopieren}
Man könnte fälschlicherweise annehmen, dass die Befehle \INS{FLD}/\INS{FST}
verwendet werden müssen, um IEEE 754 Werte zu laden oder zu speichern (also
auch zu kopieren). Nichtsdestotrotz kann dies einfacher durch den Befehl
\INS{MOV} erreicht werden, welcher, logischerweise, Werte bitweise kopiert.

\subsection{Stack, Taschenrechner und umgekehrte polnische Notation}

\myindex{Reverse Polish notation}
Jetzt können wir einsehen, warum manche alten Taschenrechner die umgekehrte
polnische Notation verwenden.

Für die Addition von 12 und 34 muss beispielsweise zuerst 12, dann 34 und dann
das \q{plus}-Zeichen eingegeben werden. 

Dies liegt daran, dass alte Taschenrechner als einfache Stackmaschinen
implementiert waren und es auf diese Weise wesentlich einfacher war, mit
komplexen geklammerten Ausdrücke umzugehen.

% TBT
% Such a calculator still present in many Unix distributions: \emph{dc}.

\subsection{x64}
Mehr dazu wie Fließkommazahlen in x86-64 verarbeitet werden unter:
\myref{floating_SIMD}

% sections
\input{patterns/12_FPU/exercises}
