\myparagraph{MSVC}

Compile it in MSVC 2010:

\lstinputlisting[caption=MSVC 2010: \ttf{},style=customasmx86]{patterns/12_FPU/1_simple/MSVC_EN.asm}

\FLD takes 8 bytes from stack and loads the number into the \ST{0} register, automatically converting 
it into the internal 80-bit format (\emph{extended precision}).

\myindex{x86!\Instructions!FDIV}

\FDIV divides the value in \ST{0} by the number stored at address \\
\GTT{\_\_real@40091eb851eb851f}~---the value 3.14 is encoded there. 
The assembly syntax doesn't support floating point numbers, so 
what we see here is the hexadecimal representation of 3.14 in 64-bit IEEE 754 format.

After the execution of \FDIV \ST{0} holds the \gls{quotient}.

\myindex{x86!\Instructions!FDIVP}

By the way, there is also the \FDIVP instruction, which divides \ST{1} by \ST{0}, 
popping both these values from stack and then pushing the result. 
If you know the Forth language,
you can quickly understand that this is a stack machine.

The subsequent \FLD instruction pushes the value of $b$ into the stack.

After that, the quotient is placed in \ST{1}, and \ST{0} has the value of $b$.

\myindex{x86!\Instructions!FMUL}

The next \FMUL instruction does multiplication: $b$ from \ST{0} is multiplied by value at\\
\GTT{\_\_real@4010666666666666} (the number 4.1 is there) and leaves the result in the \ST{0} register.

\myindex{x86!\Instructions!FADDP}

The last \FADDP instruction adds the two values at top of stack, storing the result in \ST{1} 
and then popping the value of \ST{0}, thereby leaving the result at the top of the stack, in \ST{0}.

The function must return its result in the \ST{0} register, 
so there are no any other instructions except the function epilogue after \FADDP.

\clearpage
\subsubsection{MSVC: x86 + \olly}

Let's try to hack our program in \olly, forcing it to think \scanf always works without error.
When an address of a local variable is passed into \scanf,
the variable initially contains some random garbage, in this case \TT{0x6E494714}:

\begin{figure}[H]
\centering
\myincludegraphics{patterns/04_scanf/3_checking_retval/olly_1.png}
\caption{\olly: passing variable address into \scanf}
\label{fig:scanf_ex3_olly_1}
\end{figure}

\clearpage
While \scanf executes, in the console we enter something that is definitely not a number, like \q{asdasd}.
\scanf finishes with 0 in \EAX, which indicates that an error has occurred.

We can also check the local variable in the stack and note that it has not changed.
Indeed, what would \scanf write there?
It simply did nothing except returning zero.

Let's try to \q{hack} our program.
Right-click on \EAX, 
Among the options there is \q{Set to 1}.
This is what we need.

We now have 1 in \EAX, so the following check is to be executed as intended, 
and \printf will print the value of the variable in the stack.

When we run the program (F9) we can see the following in the console window:

\lstinputlisting[caption=console window]{patterns/04_scanf/3_checking_retval/console.txt}

Indeed, 1850296084 is a decimal representation of the number in the stack (\TT{0x6E494714})!

