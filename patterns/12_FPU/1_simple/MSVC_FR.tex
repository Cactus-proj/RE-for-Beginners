\myparagraph{MSVC}

Compilons-le avec MSVC 2010:

\lstinputlisting[caption=MSVC 2010: \ttf{},style=customasmx86]{patterns/12_FPU/1_simple/MSVC_FR.asm}

\FLD prend 8 octets depuis la pile et charge le nombre dans le registre \ST{0}, en
le convertissant automatiquement dans le format interne sur 80-bit (\emph{précision
étendue}):

\myindex{x86!\Instructions!FDIV}

\FDIV divise la valeur dans \ST{0} par le nombre stocké à l'adresse \\
\GTT{\_\_real@40091eb851eb851f}~---la valeur 3.14 est encodée ici.
La syntaxe assembleur ne supporte pas les nombres à virgule flottante, donc ce que
l'on voit ici est la représentation hexadécimale de 3.14 au format 64-bit IEEE 754.

Après l'exécution de \FDIV, \ST{0} contient le \gls{quotient}.

\myindex{x86!\Instructions!FDIVP}

À propos, il y a aussi l'instruction \FDIVP, qui divise \ST{1} par \ST{0}, prenant
ces deux valeurs dans la pile et poussant le résultant.
Si vous connaissez le langage Forth, vous pouvez comprendre rapidement
que ceci est une machine à pile.

L'instruction \FLD subséquente pousse la valeur de $b$ sur la pile.

Après cela, le quotient est placé dans \ST{1}, et \ST{0} a la valeur de $b$.

\myindex{x86!\Instructions!FMUL}

L'instruction suivante effectue la multiplication: $b$ de \ST{0} est multiplié par
la valeur en\\
\GTT{\_\_real@4010666666666666} (le nombre 4.1 est là) et met le résultat
dans le registre \ST{0}.

\myindex{x86!\Instructions!FADDP}

La dernière instruction \FADDP ajoute les deux valeurs au sommet de la pile, stockant
le résultat dans \ST{1} et supprimant la valeur de \ST{0}, laissant ainsi le résultat
au sommet de la pile, dans \ST{0}.

La fonction doit renvoyer son résultat dans le registre \ST{0}, donc il n'y a aucune
autre instruction après \FADDP, excepté l'épilogue de la fonction.

\clearpage
\subsubsection{MSVC: x86 + \olly}

Essayons de hacker notre programme dans \olly, pour le forcer à penser que \scanf
fonctionne toujours sans erreur.
Lorsque l'adresse d'une variable locale est passée à \scanf, la variable contient
initiallement toujours des restes de données aléatoires, dans ce cas \TT{0x6E494714}:

\begin{figure}[H]
\centering
\myincludegraphics{patterns/04_scanf/3_checking_retval/olly_1.png}
\caption{\olly: passer l'adresse de la variable à \scanf}
\label{fig:scanf_ex3_olly_1}
\end{figure}

\clearpage
Lorsque \scanf s'exécute dans la console, entrons quelque chose qui n'est pas du
tout un nombre, comme \q{asdasd}.
\scanf termine avec 0 dans \EAX, ce qui indique qu'une erreur s'est produite.

Nous pouvons vérifier la variable locale dans le pile et noter qu'elle n'a pas changé.
En effet, qu'aurait écrit \scanf ici?
Elle n'a simplement rien fait à part renvoyer zéro.

Essayons de \q{hacker} notre programme.
Clique-droit sur \EAX,
parmi les options il y a \q{Set to 1} (mettre à 1).
C'est ce dont nous avons besoin.

Nous avons maintenant 1 dans \EAX, donc la vérification suivante va s'exécuter comme
souhaiter et \printf va afficher la valeur de la variable dans la pile.

Lorsque nous lançons le programme (F9) nous pouvons voir ceci dans la fenêtre
de la console:

\lstinputlisting[caption=fenêtre console]{patterns/04_scanf/3_checking_retval/console.txt}

En effet, 1850296084 est la représentation en décimal du nombre dans la pile (\TT{0x6E494714})!

