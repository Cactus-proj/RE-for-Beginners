\subsubsection{MIPS}

MIPS unterstütz mehrere Koprozessoren (bis zu 4); der nullte\footnote{TBT} in ein spezeiller
Kontroll-Koprozessor und der erste Koprozessor ist die FPU.

Genau wie in ARM ist der MIPS Koprozessor keine Stackmaschine, sondern hat 32
32-bit-Register (\$F0-\$F31):
\myref{MIPS_FPU_registers}.

Muss man mit 64-bit \Tdouble Werten arbeiten, wird ein Paar 32-bit F-Register
hierfür verwendet.

\lstinputlisting[caption=\Optimizing GCC 4.4.5 (IDA)]{patterns/12_FPU/1_simple/MIPS_O3_IDA_DE.lst}

Die neuen Befehl sind im Einzelnen:

\myindex{MIPS!\Instructions!LWC1}
\myindex{MIPS!\Instructions!DIV.D}
\myindex{MIPS!\Instructions!MUL.D}
\myindex{MIPS!\Instructions!ADD.D}
\begin{itemize}

\item \INS{LWC1} lädt ein 32-bit-Wort in ein Register des ersten Koprozessors
(daher \q{1} im Namen des Befehls).
\myindex{MIPS!\Pseudoinstructions!L.D}

Ein Parr \INS{LWC1} Befehle kann zu einem \INS{L.D} Pseudobefehl zusammengefasst
werden.

\item \INS{DIV.D}, \INS{MUL.D}, \INS{ADD.D} führen Division, Multiplikation bzw.
Addition aus (das \q{D.} im Suffix steht für doppelte Genauigkeit, \q{S.}
bedeutet entsprechend einfache Genauigkeit).

\end{itemize}

\myindex{MIPS!\Instructions!LUI}
\myindex{\CompilerAnomaly}
\label{MIPS_FPU_LUI}
Es gibt ein verrückte Anomalie im Compiler: die \INS{LUI} Befehle, die wir mit
einem Fragezeichen versehen haben. 
Es ist sehr schwer zu verstehen, warum ein Teil einer
64-bit-Konstante vom Typ \Tdouble in das \$V0 Register geladen wird. 
Dieser Befehl hat keine Auswirkungen. 

% TODO did you try checking out compiler source code?
Sollte jemand mehr zu dieser Anomalie wissen, bittet der Autor um eine
Mail\footnote{\EMAILS}.

