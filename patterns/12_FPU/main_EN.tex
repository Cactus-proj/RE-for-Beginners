\mysection{\FPUChapterName}
\label{sec:FPU}

The \ac{FPU} is a device within the main \ac{CPU}, specially designed to deal with floating point numbers.

It was called \q{coprocessor} in the past and it stays somewhat aside of the main \ac{CPU}.

\subsection{IEEE 754}

A number in the IEEE 754 format consists of a \emph{sign}, a \emph{significand} (also called \emph{fraction}) and an \emph{exponent}.

\subsection{x86}

It is worth looking into stack machines or learning the basics of the Forth language,
before studying the \ac{FPU} in x86.

\myindex{Intel!80486}
\myindex{Intel!FPU}
It is interesting to know that in the past (before the 80486 CPU) the coprocessor was a separate chip 
and it was not always pre-installed on the motherboard. It was possible to buy it separately and install it
\footnote{For example, John Carmack used fixed-point arithmetic 
values in his Doom video game, stored in 
32-bit \ac{GPR} registers (16 bit for integral part and another 16 bit for fractional part), so Doom
could work on 32-bit computers without FPU, i.e., 80386 and 80486 SX.}.

Starting with the 80486 DX CPU, the \ac{FPU} is integrated in the \ac{CPU}.

\myindex{x86!\Instructions!FWAIT}
The \INS{FWAIT} instruction reminds us of that fact---it switches the \ac{CPU} to a waiting state, so it can wait until the \ac{FPU} has finished with its work.

Another rudiment is the fact that the \ac{FPU} instruction 
opcodes start with the so called \q{escape}-opcodes (\GTT{D8..DF}), i.e., 
opcodes passed to a separate coprocessor.

\myindex{IEEE 754}
\label{FPU_is_stack}

The FPU has a stack capable to holding 8 80-bit registers, and each register can hold a number 
in the IEEE 754 format.

They are \ST{0}..\ST{7}. For brevity, \IDA and \olly show \ST{0} as \GTT{ST}, 
which is represented in some textbooks and manuals as \q{Stack Top}.

\subsection{ARM, MIPS, x86/x64 SIMD}

In ARM and MIPS the FPU is not a stack, but a set of registers, which can be accessed randomly, like \ac{GPR}.

The same ideology is used in the SIMD extensions of x86/x64 CPUs.

\subsection{\CCpp}

\myindex{float}
\myindex{double}

The standard \CCpp languages offer at least two floating number types, \Tfloat (\emph{single-precision}, 32 bits)
\footnote{the single precision floating point number format is also addressed in 
the \emph{\WorkingWithFloatAsWithStructSubSubSectionName}~(\myref{sec:floatasstruct}) section}
and \Tdouble (\emph{double-precision}, 64 bits).

In \InSqBrackets{\TAOCPvolII 246} we can find the \emph{single-precision} means that the floating point value can be placed into a single
[32-bit] machine word, \emph{double-precision} means it can be stored in two words (64 bits).

\myindex{long double}

GCC also supports the \emph{long double} type (\emph{extended precision}, 80 bit), which MSVC doesn't.

The \Tfloat type requires the same number of bits as the \Tint type in 32-bit environments, 
but the number representation is completely different.

\EN{\mysection{Returning Values: redux}

Again, when we know about function prologue and epilogue, let's recompile an example returning a value
(\ref{ret_val_func}, \ref{lst:ret_val_func}) using non-optimizing GCC:

\lstinputlisting[caption=\NonOptimizing GCC 8.2 x64 (\assemblyOutput),style=customasmx86]{patterns/017_ret_redux/1.s}

Effective instructions here are \INS{MOV} and \INS{RET}, others are -- prologue and epilogue.

}

\EN{\mysection{Returning Values: redux}

Again, when we know about function prologue and epilogue, let's recompile an example returning a value
(\ref{ret_val_func}, \ref{lst:ret_val_func}) using non-optimizing GCC:

\lstinputlisting[caption=\NonOptimizing GCC 8.2 x64 (\assemblyOutput),style=customasmx86]{patterns/017_ret_redux/1.s}

Effective instructions here are \INS{MOV} and \INS{RET}, others are -- prologue and epilogue.

}

\EN{\mysection{Returning Values: redux}

Again, when we know about function prologue and epilogue, let's recompile an example returning a value
(\ref{ret_val_func}, \ref{lst:ret_val_func}) using non-optimizing GCC:

\lstinputlisting[caption=\NonOptimizing GCC 8.2 x64 (\assemblyOutput),style=customasmx86]{patterns/017_ret_redux/1.s}

Effective instructions here are \INS{MOV} and \INS{RET}, others are -- prologue and epilogue.

}


\subsection{Some constants}

It's easy to find representations of some constants in Wikipedia for IEEE 754 encoded numbers.
It's interesting to know that 0.0 in IEEE 754 is represented as 32 zero bits (for single precision) or 64 zero bits
(for double).
So in order to set a floating point variable to 0.0 in register or memory, one can use \MOV or \TT{XOR reg, reg} instruction.
\myindex{\CStandardLibrary!memset()}
This is suitable for structures where many variables present of various data types.
With usual memset() function one can set all integer variables to 0, all boolean variables to \emph{false}, all pointers
to NULL, and all floating point variables (of any precision) to 0.0.

\subsection{Copying}

One may think inertially that \INS{FLD}/\INS{FST} instructions must be used to load and store (and hence, copy) IEEE 754 values.
Nevertheless, same can be achieved easier by usual \INS{MOV} instruction, which, of course, copies values bitwisely.

\subsection{Stack, calculators and reverse Polish notation}

\myindex{Reverse Polish notation}

Now we understand why some old programmable calculators use reverse Polish notation.

For example, for addition of 12 and 34 one has to enter 12, then 34, then press \q{plus} sign.

It's because old calculators were just stack machine implementations, and this was much simpler
than to handle complex parenthesized expressions.

Such a calculator still present in many Unix distributions: \emph{dc}.

\subsection{80 bits?}

\myindex{Punched card}
Internal numbers representation in FPU --- 80-bit.
Strange number, because the number not in $2^n$ form.
There is a hypothesis that this is probably due to historical reasons---the standard IBM puched card can encode 12 rows of 80 bits.
$80\cdot 25$ text mode resolution was also popular in past.

Wikipedia has another explanation: \url{https://en.wikipedia.org/wiki/Extended_precision}.

If you know better, please a drop email to the author: \EMAIL{}.

\subsection{x64}

On how floating point numbers are processed in x86-64, read more here: \myref{floating_SIMD}.

% sections
\input{patterns/12_FPU/exercises}
