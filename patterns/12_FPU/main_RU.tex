\mysection{\FPUChapterName}
\label{sec:FPU}

\ac{FPU}~--- блок в процессоре работающий с числами с плавающей запятой.

Раньше он назывался \q{сопроцессором} и он стоит немного в стороне от \ac{CPU}.

\subsection{IEEE 754}

Число с плавающей точкой в формате IEEE 754 состоит из \emph{знака}, \emph{мантиссы}\footnote{\emph{significand} или \emph{fraction} 
в англоязычной литературе} и \emph{экспоненты}.

\subsection{x86}

Перед изучением \ac{FPU} в x86 полезно ознакомиться с тем как работают стековые машины
или ознакомиться с основами языка Forth.

\myindex{Intel!80486}
\myindex{Intel!FPU}
Интересен факт, что в свое время (до 80486) сопроцессор был отдельным чипом на материнской плате, 
и вследствие его высокой цены, он не всегда присутствовал. Его можно было докупить и установить отдельно
\footnote{Например, Джон Кармак использовал в своей игре Doom числа с фиксированной запятой, хранящиеся
в обычных 32-битных \ac{GPR} (16 бит на целую часть и 16 на дробную),
чтобы Doom работал на 32-битных компьютерах без FPU, т.е. 80386 и 80486 SX.}.
Начиная с 80486 DX в состав процессора всегда входит FPU.

\myindex{x86!\Instructions!FWAIT}
Этот факт может напоминать такой рудимент как наличие инструкции \INS{FWAIT}, которая заставляет
\ac{CPU} ожидать, пока \ac{FPU} закончит работу.
Другой рудимент это тот факт, что опкоды \ac{FPU}-инструкций начинаются с т.н. \q{escape}-опкодов 
(\GTT{D8..DF}) как опкоды, передающиеся в отдельный сопроцессор.

\myindex{IEEE 754}
\label{FPU_is_stack}
FPU имеет стек из восьми 80-битных регистров: \ST{0}..\ST{7}.
Для краткости, \IDA и \olly отображают \ST{0} как \GTT{ST},
что в некоторых учебниках и документациях означает \q{Stack Top} (\q{вершина стека}).
Каждый регистр может содержать число в формате IEEE 754.

\subsection{ARM, MIPS, x86/x64 SIMD}

В ARM и MIPS FPU это не стек, а просто набор регистров, к которым можно обращаться произвольно, как к \ac{GPR}.

Такая же идеология применяется в расширениях SIMD в процессорах x86/x64.

\subsection{\CCpp}

\myindex{float}
\myindex{double}
В стандартных \CCpp имеются два типа для работы с числами с плавающей запятой: 
\Tfloat (\emph{число одинарной точности}, 32 бита)
\footnote{Формат представления чисел с плавающей точкой одинарной точности затрагивается в разделе 
\emph{\WorkingWithFloatAsWithStructSubSubSectionName}~(\myref{sec:floatasstruct}).}
и \Tdouble (\emph{число двойной точности}, 64 бита).

В \InSqBrackets{\TAOCPvolII 246} мы можем найти что \emph{single-precision} означает, что значение с плавающей точкой может быть
помещено в одно [32-битное] машинное слово, а \emph{doulbe-precision} означает, что оно размещено в двух словах (64 бита).

\myindex{long double}
GCC также поддерживает тип \emph{long double} (\emph{extended precision}, 80 бит), но MSVC~--- нет.

Несмотря на то, что \Tfloat занимает столько же места, сколько и \Tint на 32-битной архитектуре, 
представление чисел, разумеется, совершенно другое.

\EN{\mysection{Returning Values: redux}

Again, when we know about function prologue and epilogue, let's recompile an example returning a value
(\ref{ret_val_func}, \ref{lst:ret_val_func}) using non-optimizing GCC:

\lstinputlisting[caption=\NonOptimizing GCC 8.2 x64 (\assemblyOutput),style=customasmx86]{patterns/017_ret_redux/1.s}

Effective instructions here are \INS{MOV} and \INS{RET}, others are -- prologue and epilogue.

}

\EN{\mysection{Returning Values: redux}

Again, when we know about function prologue and epilogue, let's recompile an example returning a value
(\ref{ret_val_func}, \ref{lst:ret_val_func}) using non-optimizing GCC:

\lstinputlisting[caption=\NonOptimizing GCC 8.2 x64 (\assemblyOutput),style=customasmx86]{patterns/017_ret_redux/1.s}

Effective instructions here are \INS{MOV} and \INS{RET}, others are -- prologue and epilogue.

}

\EN{\mysection{Returning Values: redux}

Again, when we know about function prologue and epilogue, let's recompile an example returning a value
(\ref{ret_val_func}, \ref{lst:ret_val_func}) using non-optimizing GCC:

\lstinputlisting[caption=\NonOptimizing GCC 8.2 x64 (\assemblyOutput),style=customasmx86]{patterns/017_ret_redux/1.s}

Effective instructions here are \INS{MOV} and \INS{RET}, others are -- prologue and epilogue.

}


\subsection{Некоторые константы}

В Wikipedia легко найти представление некоторых констант в IEEE 754.
Любопытно узнать, что 0.0 в IEEE 754 представляется как 32 нулевых бита (для одинарной точности) или 64 нулевых бита
(для двойной).
Так что, для записи числа 0.0 в переменную в памяти или регистр, можно пользоваться инструкцией \MOV, или \TT{XOR reg, reg}.
\myindex{\CStandardLibrary!memset()}
Это тем может быть удобно, что если в структуре есть много переменных разных типов, то обычной ф-ций memset()
можно установить все целочисленные переменные в 0, все булевы переменные в \emph{false}, все указатели в NULL,
и все переменные с плавающей точкой (любой точности) в 0.0.

\subsection{Копирование}

По инерции можно подумать, что для загрузки и сохранения (и, следовательно, копирования) чисел в формате
IEEE 754 нужно использовать пару инструкций \INS{FLD}/\INS{FST}.
Тем не менее, этого куда легче достичь используя обычную инструкцию \INS{MOV},
которая, конечно же, просто копирует значения побитово.

\subsection{Стек, калькуляторы и обратная польская запись}

\myindex{Обратная польская запись}
Теперь понятно, почему некоторые старые программируемые калькуляторы используют обратную польскую запись.

Например для сложения 12 и 34 нужно было набрать 12, потом 34, потом нажать знак \q{плюс}.

Это потому что старые калькуляторы просто реализовали стековую машину и это было куда проще, чем обрабатывать сложные выражения со скобками.

Подобный калькулятор все еще присутствует во многих Unix-дистрибутивах: \emph{dc}.

\subsection{80 бит?}

\myindex{Перфокарты}
Внутреннее представление чисел с FPU --- 80-битное.
Странное число, потому как не является числом вида $2^n$.
Имеется гипотеза, что причина, возможно, историческая --- стандартные IBM-овские перфокарты могли кодировать 12 строк по 80 бит.
Раньше было также популярно текстовое разрешение $80 \cdot 25$.

В Wikipedia есть еще одно объяснение: \url{https://en.wikipedia.org/wiki/Extended_precision}.

Если вы знаете более точную причину, просьба сообщить автору: \EMAIL{}.

\subsection{x64}

О том, как происходит работа с числами с плавающей запятой в x86-64, читайте здесь: \myref{floating_SIMD}.

% sections
\input{patterns/12_FPU/exercises}

