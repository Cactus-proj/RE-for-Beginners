\subsection{レジスタへの定数のロード}
\label{ARM_big_constants}

\subsubsection{32ビットARM}
\label{ARM_big_constants_loading}

すでにご存じのとおり、すべての命令の長さはARMモードでは4バイト、Thumbモードでは2バイトです。

それでは、1つの命令でエンコードできない場合、どうすれば32ビット値をレジスタにロードできるでしょうか。

やってみましょう。

\begin{lstlisting}[style=customc]
unsigned int f()
{
	return 0x12345678;
};
\end{lstlisting}

\begin{lstlisting}[caption=GCC 4.6.3 -O3 \ARMMode,style=customasmARM]
f:
        ldr     r0, .L2
        bx      lr
.L2:
        .word   305419896 ; 0x12345678
\end{lstlisting}

したがって、\TT{0x12345678}の値はメモリに保存され、必要に応じてロードされます。

しかし、追加のメモリアクセスを取り除くことは可能です。

\begin{lstlisting}[caption=GCC 4.6.3 -O3 -march{=}armv7-a (\ARMMode),style=customasmARM]
movw    r0, #22136      ; 0x5678
movt    r0, #4660       ; 0x1234
bx      lr
\end{lstlisting}

値は部分的にレジスタにロードされ、最初に下位部分(\INS{MOVW}を使用)、
次に上位部分(\INS{MOVT}を使用)の順にロードされます。

これは、32ビット値をレジスタにロードするためにARMモードでは2命令が必要であることを意味します。

実際のコードには定数がそれほど多くないため(0と1を除く)、実際のところ問題にはなりません。

2命令のバージョンは1命令のバージョンより遅いということでしょうか?

間違いなくそうです。 ほとんどの場合、最新のARMプロセッサはそのようなシーケンスを検出して高速に実行できます。

一方、 \IDA はコード内のそのようなパターンを検出し、この関数を次のように逆アセンブルします。

\begin{lstlisting}[style=customasmARM]
MOV    R0, 0x12345678
BX     LR
\end{lstlisting}

\subsubsection{ARM64}

\begin{lstlisting}[style=customc]
uint64_t f()
{
	return 0x12345678ABCDEF01;
};
\end{lstlisting}

\begin{lstlisting}[caption=GCC 4.9.1 -O3,style=customasmARM]
mov	x0, 61185   ; 0xef01
movk	x0, 0xabcd, lsl 16
movk	x0, 0x5678, lsl 32
movk	x0, 0x1234, lsl 48
ret
\end{lstlisting}

\myindex{ARM!\Instructions!MOVK}
\TT{MOVK}は\q{MOV Keep}を表し、つまり、残りのビットには触れずに、レジスタに16ビット値を書き込みます。
\myindex{ARM!Optional operators!LSL}
\TT{LSL}接尾辞は、各ステップで値を16、32、48ビット左にシフトします。シフトはロード前に行われます。

これは、64ビット値をレジスタにロードするために4つの命令が必要であることを意味します。

\myparagraph{浮動小数点数をレジスタに格納}

命令を1つだけ使用して浮動小数点数をDレジスタに格納することは可能です。

例えば:

\begin{lstlisting}[style=customc]
double a()
{
	return 1.5;
};
\end{lstlisting}

\begin{lstlisting}[caption=GCC 4.9.1 -O3 + objdump,style=customasmARM]
0000000000000000 <a>:
   0:   1e6f1000        fmov    d0, #1.500000000000000000e+000
   4:   d65f03c0        ret
\end{lstlisting}

$1.5$ という数は実際、32ビット命令でエンコードされます。
しかし、どうやって?
\myindex{ARM!\Instructions!FMOV}

ARM64では、いくつかの浮動小数点数をエンコードするために8ビットが \TT{FMOV} 命令にはあります。

アルゴリズムは \ARMSixFourRefURL では \TT{VFPExpandImm()}と呼ばれます。
\myindex{minifloat}
\emph{minifloat}\footnote{\href{http://en.wikipedia.org/wiki/Minifloat}{wikipedia}}とも呼ばれます。

さまざまな値を試すことができます。コンパイラは $30.0$ と $31.0$ をエンコードできますが、
IEEE 754形式で8バイトをこの番号に割り当てる必要があるため、 $32.0$ をエンコードすることはできません。

\begin{lstlisting}[style=customc]
double a()
{
	return 32;
};
\end{lstlisting}

\begin{lstlisting}[caption=GCC 4.9.1 -O3,style=customasmARM]
a:
	ldr	d0, .LC0
	ret
.LC0:
	.word	0
	.word	1077936128
\end{lstlisting}
