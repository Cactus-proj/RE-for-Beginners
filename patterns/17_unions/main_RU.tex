\mysection{Объединения (union)}

\emph{union} в \CCpp используется в основном для интерпретации переменной (или блока памяти) одного типа как переменной другого типа.

% sections
\EN{\mysection{Returning Values: redux}

Again, when we know about function prologue and epilogue, let's recompile an example returning a value
(\ref{ret_val_func}, \ref{lst:ret_val_func}) using non-optimizing GCC:

\lstinputlisting[caption=\NonOptimizing GCC 8.2 x64 (\assemblyOutput),style=customasmx86]{patterns/017_ret_redux/1.s}

Effective instructions here are \INS{MOV} and \INS{RET}, others are -- prologue and epilogue.

}

\EN{\mysection{Returning Values: redux}

Again, when we know about function prologue and epilogue, let's recompile an example returning a value
(\ref{ret_val_func}, \ref{lst:ret_val_func}) using non-optimizing GCC:

\lstinputlisting[caption=\NonOptimizing GCC 8.2 x64 (\assemblyOutput),style=customasmx86]{patterns/017_ret_redux/1.s}

Effective instructions here are \INS{MOV} and \INS{RET}, others are -- prologue and epilogue.

}

\input{patterns/17_unions/FSCALE_RU}

\subsection{Быстрое вычисление квадратного корня}

Вот где еще можно на практике применить трактовку типа \Tfloat как целочисленного, это быстрое вычисление квадратного корня.

\begin{lstlisting}[caption=Исходный код взят из Wikipedia: \url{https://en.wikipedia.org/wiki/Methods_of_computing_square_roots},style=customc]
/* Assumes that float is in the IEEE 754 single precision floating point format
 * and that int is 32 bits. */
float sqrt_approx(float z)
{
    int val_int = *(int*)&z; /* Same bits, but as an int */
    /*
     * To justify the following code, prove that
     *
     * ((((val_int / 2^m) - b) / 2) + b) * 2^m = ((val_int - 2^m) / 2) + ((b + 1) / 2) * 2^m)
     *
     * where
     *
     * b = exponent bias
     * m = number of mantissa bits
     *
     * .
     */
 
    val_int -= 1 << 23; /* Subtract 2^m. */
    val_int >>= 1; /* Divide by 2. */
    val_int += 1 << 29; /* Add ((b + 1) / 2) * 2^m. */
 
    return *(float*)&val_int; /* Interpret again as float */
}
\end{lstlisting}

В качестве упражнения, вы можете попробовать скомпилировать эту функцию и разобраться, как она работает. \\
\\
Имеется также известный алгоритм быстрого вычисления $\frac{1}{\sqrt{x}}$.
\myindex{Quake III Arena}
Алгоритм стал известным, вероятно потому, что был применен в Quake III Arena.

Описание алгоритма есть в Wikipedia: \url{http://go.yurichev.com/17361}.
