\mysection{Unions}

Les \emph{unions} en \CCpp sont utilisées principalement pour interpréter une variable
(ou un bloc de mémoire) d'un type de données comme une variable d'un autre type de données.

% sections
\EN{\mysection{Returning Values: redux}

Again, when we know about function prologue and epilogue, let's recompile an example returning a value
(\ref{ret_val_func}, \ref{lst:ret_val_func}) using non-optimizing GCC:

\lstinputlisting[caption=\NonOptimizing GCC 8.2 x64 (\assemblyOutput),style=customasmx86]{patterns/017_ret_redux/1.s}

Effective instructions here are \INS{MOV} and \INS{RET}, others are -- prologue and epilogue.

}

\EN{\mysection{Returning Values: redux}

Again, when we know about function prologue and epilogue, let's recompile an example returning a value
(\ref{ret_val_func}, \ref{lst:ret_val_func}) using non-optimizing GCC:

\lstinputlisting[caption=\NonOptimizing GCC 8.2 x64 (\assemblyOutput),style=customasmx86]{patterns/017_ret_redux/1.s}

Effective instructions here are \INS{MOV} and \INS{RET}, others are -- prologue and epilogue.

}

\input{patterns/17_unions/FSCALE_FR}

\subsection{Calcul rapide de racine carré}

Un autre algorithme connu où un \Tfloat est interprété comme un entier est celui
de calcul rapide de racine carrée.

\begin{lstlisting}[caption=Le code source provient de Wikipédia: \url{https://en.wikipedia.org/wiki/Methods_of_computing_square_roots},style=customc]
/* Assumes that float is in the IEEE 754 single precision floating point format
 * and that int is 32 bits. */
float sqrt_approx(float z)
{
    int val_int = *(int*)&z; /* Same bits, but as an int */
    /*
     * To justify the following code, prove that
     *
     * ((((val_int / 2^m) - b) / 2) + b) * 2^m = ((val_int - 2^m) / 2) + ((b + 1) / 2) * 2^m)
     *
     * where
     *
     * b = exponent bias
     * m = number of mantissa bits
     *
     * .
     */
 
    val_int -= 1 << 23; /* Subtract 2^m. */
    val_int >>= 1; /* Divide by 2. */
    val_int += 1 << 29; /* Add ((b + 1) / 2) * 2^m. */
 
    return *(float*)&val_int; /* Interpret again as float */
}
\end{lstlisting}

À titre d'exercice, vous pouvez essayez de compiler cette fonction et de comprendre
comme elle fonctionne.\\
\\
C'est un algorithme connu de calcul rapide de $\frac{1}{\sqrt{x}}$.
\myindex{Quake III Arena}
L'algorithme devînt connu, supposément, car il a été utilisé dans Quake III Arena.

La description de l'algorithme peut être trouvée sur Wikipédia: \url{http://en.wikipedia.org/wiki/Fast_inverse_square_root}.

