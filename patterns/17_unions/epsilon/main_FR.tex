\subsection{Calcul de l'epsilon de la machine}

L'epsilon de la machine est la plus petite valeur avec laquelle le \ac{FPU} peut
travailler.
Plus il y a de bits alloués pour représenter un nombre en virgule flottante, plus
l'epsilon est petit,
C'est $2^{-23} = 1.19e-07$ pour les \Tfloat et $2^{-52} = 2.22e-16$ pour les \Tdouble.
Voir aussi: \href{https://en.wikipedia.org/wiki/Arithmetic_underflow}{l'article de Wikipédia}.%
% TODO recheck values
% I get
% 1.19209e-07
% 2.22045e-16

Il intéressant de voir comme il est facile de calculer l'epsilon de la machine:

\lstinputlisting[style=customc]{patterns/17_unions/epsilon/float.c}

Ce que l'on fait ici est simplement de traiter la partie fractionnaire du nombre
au format IEEE 754 comme un entier et de lui ajouter 1.
Le nombre flottant en résultant est égal à $starting\_value+machine\_epsilon$, donc
il suffit de soustraire $starting\_value$ (en utilisant l'arithmétique flottante)
pour mesurer ce que la différence d'un bit représente dans un nombre flottant simple
précision(\Tfloat).
L' \emph{union} permet ici d'accéder au nombre IEEE 754 comme à un entier normal.
Lui ajouter 1 ajoute en fait 1 au \emph{significande} du nombre, toutefois, inutile
de dire, un débordement est possible, qui ajouterait 1 à l'exposant.

\subsubsection{x86}

\lstinputlisting[caption=\Optimizing MSVC 2010,style=customasmx86]{patterns/17_unions/epsilon/float_MSVC_2010_Ox_FR.asm}

La seconde instruction \INS{FST} est redondante: il n'est pas nécessaire de stocker
la valeur en entrée à la même place (le compilateur a décidé d'allouer la variable
$v$ à la même place dans la pile locale que l'argument en entrée).
Puis elle est incrémentée avec \INS{INC}, puisque c'est une variable entière normale.
Ensuite elle est chargée dans le FPU comme un nombre IEEE 754 32-bit, \INS{FSUBR}
fait le reste du travail et la valeur résultante est stockée dans \TT{ST0}.
La dernière paire d'instructions \INS{FSTP}/\INS{FLD} est redondante, mais le compilateur
n'a pas optimisé le code.

\subsubsection{ARM64}

Étendons notre exemple à 64-bit:

\lstinputlisting[label=machine_epsilon_double_c,style=customc]{patterns/17_unions/epsilon/double.c}

ARM64 n'a pas d'instruction qui peut ajouter un nombre a un D-registre FPU, donc
la valeur en entrée (qui provient du registre x64 \TT{D0}) est d'abord copiée dans
le \ac{GPR}, incrémentée, copiée dans le registre FPU \TT{D1}, et puis la soustraction
est faite.

\lstinputlisting[caption=GCC 4.9 ARM64 \Optimizing,style=customasmARM]{patterns/17_unions/epsilon/double_GCC49_ARM64_O3_FR.s}

Voir aussi cet exemple compilé pour x64 avec instructions SIMD: \myref{machine_epsilon_x64_and_SIMD}.

\subsubsection{MIPS}

\myindex{MIPS!\Instructions!MTC1}

Il y a ici la nouvelle instruction \INS{MTC1} (\q{Move To Coprocessor 1}), elle transfère
simplement des données vers les registres du FPU.

\lstinputlisting[caption=GCC 4.4.5 \Optimizing (IDA),style=customasmMIPS]{patterns/17_unions/epsilon/MIPS_O3_IDA.lst}

\subsubsection{\Conclusion}

Il est difficile de dire si quelqu'un pourrait avoir besoin de cette astuce dans
du code réel, mais comme cela a été mentionné plusieurs fois dans ce livre, cet exemple
est utile pour expliquer le format IEEE 754 et les \emph{union}s en \CCpp.
