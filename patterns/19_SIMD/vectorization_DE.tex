\subsection{Vektorisierung}

\newcommand{\URLVEC}{\href{http://go.yurichev.com/17080}{Wikipedia: vectorization}}
Vektorisierung\footnote{\URLVEC} tritt auf, wenn man beispielsweise eine Schleife hat, die mehrere Arrays als Input und
ein Array als Output hat. Der Rumpf der Schleife nimmt Werte aus den Eingabearrays, vearbeitet sie und legt das Ergebnis
im Outputarray ab.
Vektorisierung wird hier verwendet um mehrere Elemente gleichzeitig zu verarbeiten.

\myindex{Cray}
Vektorisierung ist keine besonders neue Idee: der Autor hat sie bereits auf dem Cray Y-MP Supercomputer aus dem Jahre
1988 ausgemacht, als er mit dessen kleinerem Brude Cray Y-MP EL spielte.

Vectorization is not very fresh technology: the author of this textbook saw it at least on the Cray Y-MP 
supercomputer line from 1988 when he played with its \q{lite} version Cray Y-MP EL
\footnote{Er befindet sich im Museum für Supercomputer: \url{http://go.yurichev.com/17081}}.

% FIXME! add assembly listing!
Ein Beispiel:

\begin{lstlisting}[style=customc]
for (i = 0; i < 1024; i++)
{
    C[i] = A[i]*B[i];
}
\end{lstlisting}

Dieses Codefragment nimmt Elemente aus A und B, multipliziert sie und speichert das Ergebnis in C.

\myindex{x86!\Instructions!PLMULLD}
\myindex{x86!\Instructions!PLMULHW}
\newcommand{\PMULLD}{\emph{PMULLD} (\emph{Multiply Packed Signed Dword Integers and Store Low Result})}
\newcommand{\PMULHW}{\TT{PMULHW} (\emph{Multiply Packed Signed Integers and Store High Result})}
Wenn jedes Arrayelement ein 32-Bit \Tint ist, ist es möglich 4 Elemente aus A in ein 128-Bit-XMM-Register zu laden, 4
weitere Elemente aus B in ein anderes, und dann durch die Befehle \PMULLD{} und \PMULHW{} 4 64-Bit-Produkte auf einmal
zu erzeugen.

Dadurch wir der Rumpf der Schleife nur $\frac{1024}{4}=256$ Mal ausgeführt. Das ist nur ein Viertel der normalen Anzahl
an Durchläufen und geht natürlich schneller.

\newcommand{\URLINTELVEC}{\href{http://go.yurichev.com/17082}{Auszug: Effektive automatische Vektorisierung}}

\subsubsection{Beispiel zur Addition}

\myindex{Intel C++}
Manche Compiler können Vektorsierung in einfachen Fällen automatisch anwenden, zum Beispiel Intel C++\footnote{Mehr zur
automatischen Vektorsierung in Intel C++ unter:\URLINTELVEC}.

Hier ist eine kleine Funktion:

\begin{lstlisting}[style=customc]
int f (int sz, int *ar1, int *ar2, int *ar3)
{
	for (int i=0; i<sz; i++)
		ar3[i]=ar1[i]+ar2[i];

	return 0;
};
\end{lstlisting}

\myparagraph{Intel C++}

Kompilieren wir das Programm mit Intel C++ 11.1.051 win32:

\begin{verbatim}
icl intel.cpp /QaxSSE2 /Faintel.asm /Ox
\end{verbatim}

Wir erhalten (in \IDA) folgenden Code:

\lstinputlisting[style=customasmx86]{patterns/19_SIMD/18_1_DE.asm}

Die SSE2-Befehle sind:
\myindex{x86!\Instructions!MOVDQA}
\myindex{x86!\Instructions!MOVDQU}
\myindex{x86!\Instructions!PADDD}
\begin{itemize}
\item
\MOVDQU (\emph{Move Unaligned Double Quadword})---lädt 16 Byte aus dem Speicher in ein XMM-Register.

\item
\PADDD (\emph{Add Packed Integers}) addiert 4 Paare von 32-Bit-Zahlen und speichert das Ergebnis im ersten Operanden.
Hier wird übrigens bei einem Überlauf keine Exception geworfen und auch keine Flags gesetzt, aber es werden dann nur die
niederen 23 Bit des Ergebnisses gespeichert.
Wenn einer der Operanden von \PADDD die Speicheradresse eines Wertes ist, muss sich die Adresse auf einer 16-Byte-Grenze
befinden. Ist dies nicht der Fall, wird eine Exception ausgelöst.

\item
\MOVDQA (\emph{Move Aligned Double Quadword}) entspricht \MOVDQU, verlangt aber, dass die Adresse des Wertes im Speicher
auf einer 16-Byte-Grenze liegt. Ist dies nicht der Fall, wird eine Exception ausgelöst. \MOVDQA ist schneller als
\MOVDQU, hat aber die genannte zusätzliche Vorbedingung.

\end{itemize}
Diese SSE2-Befehle werden nur dann ausgeführt, wenn auf mehr als 4 Paaren gearbeitet werden muss und sich der Pointer
\TT{ar3} auf einer 16-Byte-Grenze befindet.

Wenn sich \TT{ar2} ebenfalls auf einer 16-Byte-Grenze befindet, wird das folgende Codefragment ebenfalls ausgeführt:

\begin{lstlisting}[style=customasmx86]
movdqu  xmm0, xmmword ptr [ebx+edi*4] ; ar1+i*4
paddd   xmm0, xmmword ptr [esi+edi*4] ; ar2+i*4
movdqa  xmmword ptr [eax+edi*4], xmm0 ; ar3+i*4
\end{lstlisting}
Ansonsten wird der Wert aus \TT{ar2} mit \MOVDQU nach \XMM{0} geladen. \MOVDQU benötigt keinen angeordneten Pointer,
arbeitet aber möglicherweise langsamer:

\lstinputlisting[style=customasmx86]{patterns/19_SIMD/18_1_excerpt_DE.asm}

In allen anderen Fällen wird nicht-SSE2-Code ausgeführt.

\myparagraph{GCC}

\newcommand{\URLGCCVEC}{\url{http://go.yurichev.com/17083}}

GCC kann in einfachen Fällen ebenfalls Vektorisierung durchführen\footnote{Mehr zur Unterstützung von Vektorisierung in
GCC:
\URLGCCVEC}, wenn die Option \Othree verwendet und SSE2 Unterstützung aktiviert ist: \TT{-msse2}.

Wie erhalten den folgenden Code (GCC 4.4.1):

\lstinputlisting[style=customasmx86]{patterns/19_SIMD/18_2_gcc_O3.asm}

Fast der gleiche wie Intel C++, aber nicht ganz so kleinschrittig.

\subsubsection{Beispiel zum Kopieren von Speicherblöcken}
\label{vec_memcpy}
Betrachten wir erneut das Beispiel zum einfachen \TT{memcpy}(\myref{loop_memcpy}):

\lstinputlisting[style=customc]{memcpy.c}

Der optimierende GCC 4.9.1 erzeugt folgenden Code:

\lstinputlisting[caption=\Optimizing GCC 4.9.1 x64,style=customasmx86]{patterns/19_SIMD/memcpy_GCC49_x64_O3_DE.s}
