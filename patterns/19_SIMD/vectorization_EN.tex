\subsection{Vectorization}

\newcommand{\URLVEC}{\href{http://en.wikipedia.org/wiki/Vectorization_(computer_science)}{Wikipedia: vectorization}}

Vectorization\footnote{\URLVEC} is when, for example, you have a loop taking couple of arrays for input and producing one array.
The loop body takes values from the input arrays, does something and puts the result into the output array.
%It is important that there is only a single operation applied to each element.
Vectorization is to process several elements simultaneously.

\myindex{Cray}
Vectorization is not very fresh technology: the author of this textbook saw it at least on the Cray Y-MP 
supercomputer line from 1988 when he played with its \q{lite} version Cray Y-MP EL
\footnote{Remotely. It is installed in the museum of supercomputers: \url{http://www.cray-cyber.org}}.

% FIXME! add assembly listing!
For example:

\begin{lstlisting}[style=customc]
for (i = 0; i < 1024; i++)
{
    C[i] = A[i]*B[i];
}
\end{lstlisting}

This fragment of code takes elements from A and B, multiplies them and saves the result into C.

\myindex{x86!\Instructions!PLMULLD}
\myindex{x86!\Instructions!PLMULHW}
\newcommand{\PMULLD}{\emph{PMULLD} (\emph{Multiply Packed Signed Dword Integers and Store Low Result})}
\newcommand{\PMULHW}{\TT{PMULHW} (\emph{Multiply Packed Signed Integers and Store High Result})}

If each array element we have is 32-bit \Tint, then it is possible to load 4 elements from A into a 128-bit 
XMM-register, from B to another XMM-registers, and by executing \PMULLD{} and \PMULHW{}, 
it is possible to get 4 64-bit \glspl{product} at once.

Thus, loop body execution count is $1024/4$ instead of 1024, that is 4 times less and, of course, faster.

\newcommand{\URLINTELVEC}{\href{http://www.intel.com/intelpress/sum_vmmx.htm}{Excerpt: Effective Automatic Vectorization}}

\subsubsection{Addition example}

\myindex{Intel C++}

Some compilers can do vectorization automatically in simple cases, 
e.g., Intel C++\footnote{More about Intel C++ automatic vectorization: \URLINTELVEC}.

Here is tiny function:

\begin{lstlisting}[style=customc]
int f (int sz, int *ar1, int *ar2, int *ar3)
{
	for (int i=0; i<sz; i++)
		ar3[i]=ar1[i]+ar2[i];

	return 0;
};
\end{lstlisting}

\myparagraph{Intel C++}

Let's compile it with Intel C++ 11.1.051 win32:

\begin{verbatim}
icl intel.cpp /QaxSSE2 /Faintel.asm /Ox
\end{verbatim}

We got (in \IDA):

\lstinputlisting[style=customasmx86]{patterns/19_SIMD/18_1_EN.asm}

The SSE2-related instructions are:
\myindex{x86!\Instructions!MOVDQA}
\myindex{x86!\Instructions!MOVDQU}
\myindex{x86!\Instructions!PADDD}
\begin{itemize}
\item
\MOVDQU (\emph{Move Unaligned Double Quadword})---just loads 16 bytes from memory into a XMM-register.

\item
\PADDD (\emph{Add Packed Integers})---adds 4 pairs of 32-bit numbers and leaves the result in the first operand.
By the way, no exception is raised in case of overflow and no flags are to be set, 
just the low 32 bits of the result are to be stored.
If one of \PADDD's operands is the address of a value in memory,
then the address must be aligned on a 16-byte boundary. 
If it is not aligned, an exception will be triggered.

\item
\MOVDQA (\emph{Move Aligned Double Quadword})
is the same as \MOVDQU, but requires the address of the value in memory to be aligned on a 16-bit boundary.
If it is not aligned, exception will be raised.
\MOVDQA works faster than \MOVDQU, but requires aforesaid.

\end{itemize}

So, these SSE2-instructions are to be executed only in case there are more than 4 pairs to work on
and the pointer \TT{ar3} is aligned on a 16-byte boundary.

Also, if \TT{ar2} is aligned on a 16-byte boundary as well, 
this fragment of code is to be executed:

\begin{lstlisting}[style=customasmx86]
movdqu  xmm0, xmmword ptr [ebx+edi*4] ; ar1+i*4
paddd   xmm0, xmmword ptr [esi+edi*4] ; ar2+i*4
movdqa  xmmword ptr [eax+edi*4], xmm0 ; ar3+i*4
\end{lstlisting}

Otherwise, the value from \TT{ar2} is to be loaded into \XMM{0} using \MOVDQU,
which does not require aligned pointer, but may work slower:

\lstinputlisting[style=customasmx86]{patterns/19_SIMD/18_1_excerpt_EN.asm}

In all other cases, non-SSE2 code is to be executed.

\myparagraph{GCC}

\newcommand{\URLGCCVEC}{\url{http://gcc.gnu.org/projects/tree-ssa/vectorization.html}}

GCC may also vectorize in simple cases\footnote{More about GCC vectorization support: \URLGCCVEC},
if the \Othree option is used and SSE2 support is turned on: \TT{-msse2}.

What we get (GCC 4.4.1):

\lstinputlisting[style=customasmx86]{patterns/19_SIMD/18_2_gcc_O3.asm}

Almost the same, however, not as meticulously as Intel C++.

\subsubsection{Memory copy example}
\label{vec_memcpy}

Let's revisit the simple memcpy() example
(\myref{loop_memcpy}):

\lstinputlisting[style=customc]{memcpy.c}

And that's what optimizations GCC 4.9.1 did:

\lstinputlisting[caption=\Optimizing GCC 4.9.1 x64,style=customasmx86]{patterns/19_SIMD/memcpy_GCC49_x64_O3_EN.s}
