\mysection{Sauts conditionnels}
\label{sec:Jcc}
\myindex{\CLanguageElements!if}

% sections
\subsection{\RU{Простой пример}\EN{Simple example}\DE{einfaches Beispiel}%
\FR{Exemple simple}\IT{Esempio semplice}\JA{シンプルな例}
}

\lstinputlisting[style=customc]{patterns/07_jcc/simple/ex.c}

% subsections
\EN{\subsubsection{x86}
\myindex{Windows!Win32}

Win32 API example:

\begin{lstlisting}[style=customc]
	HANDLE fh;

	fh=CreateFile ("file", GENERIC_WRITE | GENERIC_READ, FILE_SHARE_READ, NULL, OPEN_ALWAYS, FILE_ATTRIBUTE_NORMAL, NULL);
\end{lstlisting}

We get (MSVC 2010):

\begin{lstlisting}[caption=MSVC 2010,style=customasmx86]
	push	0
	push	128		; 00000080H
	push	4
	push	0
	push	1
	push	-1073741824	; c0000000H
	push	OFFSET $SG78813
	call	DWORD PTR __imp__CreateFileA@28
	mov	DWORD PTR _fh$[ebp], eax
\end{lstlisting}

Let's take a look in WinNT.h:

\begin{lstlisting}[caption=WinNT.h,style=customc]
#define GENERIC_READ                     (0x80000000L)
#define GENERIC_WRITE                    (0x40000000L)
#define GENERIC_EXECUTE                  (0x20000000L)
#define GENERIC_ALL                      (0x10000000L)
\end{lstlisting}

Everything is clear,
GENERIC\_READ | GENERIC\_WRITE = 0x80000000 | 0x40000000 = 0xC0000000,
and that value is used as the second argument for the \TT{CreateFile()}\footnote{\href{http://msdn.microsoft.com/en-us/library/aa363858(VS.85).aspx}{msdn.microsoft.com/en-us/library/aa363858(VS.85).aspx}}function.

How would \TT{CreateFile()} check these flags?
\myindex{Windows!KERNEL32.DLL}

If we look in KERNEL32.DLL in Windows XP SP3 x86, we'll find this fragment of code in \TT{CreateFileW}:

\begin{lstlisting}[caption=KERNEL32.DLL (Windows XP SP3 x86),style=customasmx86]
.text:7C83D429     test    byte ptr [ebp+dwDesiredAccess+3], 40h
.text:7C83D42D     mov     [ebp+var_8], 1
.text:7C83D434     jz      short loc_7C83D417
.text:7C83D436     jmp     loc_7C810817
\end{lstlisting}

\myindex{x86!\Instructions!TEST}

Here we see the \TEST instruction, however it doesn't take the whole second argument,\\
but only the most significant byte (\TT{ebp+dwDesiredAccess+3}) and checks it for flag \TT{0x40}
(which implies the \TT{GENERIC\_WRITE} flag here).
\myindex{x86!\Instructions!AND}

\TEST is basically the same instruction as \AND, but without saving the result
(recall the fact \CMP is merely the same as \SUB, but without saving the result~(\myref{CMPandSUB})).

The logic of this code fragment is as follows:

\begin{lstlisting}[style=customc]
if ((dwDesiredAccess&0x40000000) == 0) goto loc_7C83D417
\end{lstlisting}

\myindex{x86!\Instructions!AND}
\myindex{x86!\Registers!ZF}

If \AND instruction leaves this bit, the \ZF flag is to be cleared and the 
\JZ conditional jump is not to be triggered.
The conditional jump is triggered only if the \TT{0x40000000} bit is absent in \TT{dwDesiredAccess} variable~---then the result of \AND is 0,
\ZF is to be set and the conditional jump is to be triggered.

Let's try GCC 4.4.1 and Linux:

\begin{lstlisting}[style=customc]
#include <stdio.h>
#include <fcntl.h>

void main()
{
	int handle;

	handle=open ("file", O_RDWR | O_CREAT);
};
\end{lstlisting}

We get:

\lstinputlisting[caption=GCC 4.4.1,style=customasmx86]{patterns/14_bitfields/1_check/check.asm}

\myindex{Linux!libc.so.6}
\myindex{syscall}

If we take a look in the \TT{open()} function in the \TT{libc.so.6} library, it is only a syscall:

\lstinputlisting[caption=open() (libc.so.6),style=customasmx86]{patterns/14_bitfields/1_check/tmp1.asm}

So, the bit fields for \TT{open()} are apparently checked somewhere in the Linux kernel.

Of course, it is easy to download both Glibc and the Linux kernel source code, 
but we are interested in understanding the matter without it.

So, as of Linux 2.6, when the \TT{sys\_open} syscall is called, control eventually passes to \TT{do\_sys\_open},
and from there---to the \TT{do\_filp\_open()} function (it's located in the kernel source tree in \TT{fs/namei.c}).

\newcommand{\URLREGPARM}{\href{http://www.ohse.de/uwe/articles/gcc-attributes.html\#func-regparm}{ohse.de/uwe/articles/gcc-attributes.html\#func-regparm}}

\myindex{fastcall}
\label{regparm}
N.B.  Aside from passing arguments via the stack,
there is also a method of passing some of them
via registers. This is also called fastcall~(\myref{fastcall}).
This works faster since CPU does not need to access the stack in memory to read argument values.
GCC has the option \emph{regparm}\footnote{\URLREGPARM},
through which it's possible to set the number of arguments that can be passed via registers.

\newcommand{\URLKERNELNEWB}{\href{http://kernelnewbies.org/Linux_2_6_20\#head-042c62f290834eb1fe0a1942bbf5bb9a4accbc8f}{kernelnewbies.org/Linux\_2\_6\_20\#head-042c62f290834eb1fe0a1942bbf5bb9a4accbc8f}}
\newcommand{\CALLINGHFILE}{arch/x86/include/asm/calling.h}

The Linux 2.6 kernel is compiled with \TT{-mregparm=3} option~\footnote{\URLKERNELNEWB}
\footnote{See also \TT{\CALLINGHFILE} file in kernel tree}.

What this means to us is that the first 3 arguments are to be passed via registers \EAX, \EDX and \ECX, 
and the rest via the stack. 
Of course, if the number of arguments is less than 3, only part of registers set is to be used.

So, let's download Linux Kernel 2.6.31, compile it in Ubuntu: \TT{make vmlinux}, open it in \IDA, 
and find the \TT{do\_filp\_open()} function. At the beginning, we see (the comments are mine):

\lstinputlisting[caption=do\_filp\_open() (linux kernel 2.6.31),style=customasmx86]{patterns/14_bitfields/1_check/check2_EN.asm}

GCC saves the values of the first 3 arguments in the local stack. 
If that wasn't done, the compiler would not touch these registers, 
and that would be too tight environment for the compiler's \gls{register allocator}.

Let's find this fragment of code:

\lstinputlisting[caption=do\_filp\_open() (linux kernel 2.6.31),style=customasmx86]{patterns/14_bitfields/1_check/check3.asm}

\TT{0x40}---is what the \TT{O\_CREAT} macro equals to.
\TT{open\_flag} gets checked for the presence of the \TT{0x40} bit, and if this bit is 1, 
the next \JNZ instruction is triggered.
}
\RU{\subsubsection{x86: 3 целочисленных аргумента}

\myparagraph{MSVC}

Компилируем при помощи MSVC 2010 Express, и в итоге получим:

\begin{lstlisting}[style=customasmx86]
$SG3830	DB	'a=%d; b=%d; c=%d', 00H

...

	push	3
	push	2
	push	1
	push	OFFSET $SG3830
	call	_printf
	add	esp, 16
\end{lstlisting}

Всё почти то же, за исключением того, что теперь видно, что аргументы для \printf заталкиваются в стек в обратном порядке: самый первый аргумент заталкивается последним.

Кстати, вспомним, что переменные типа \Tint в 32-битной системе, как известно, имеют ширину 32 бита, это 4 байта.

Итак, у нас всего 4 аргумента. $4*4 = 16$~--- именно 16 байт занимают в стеке указатель на строку плюс ещё 3 числа типа \Tint.

\myindex{x86!\Instructions!ADD}
\myindex{x86!\Registers!ESP}
\myindex{cdecl}
Когда при помощи инструкции \INS{ADD ESP, X} корректируется \glslink{stack pointer}{указатель стека} \ESP 
после вызова какой-либо функции, зачастую можно сделать вывод о том, сколько аргументов 
у вызываемой функции было, разделив X на 4.

Конечно, это относится только к cdecl-методу передачи аргументов через стек, и только для 32-битной среды.

См. также в соответствующем разделе о способах передачи аргументов через стек ~(\myref{sec:callingconventions}).

Иногда бывает так, что подряд идут несколько вызовов разных функций, но стек корректируется только один раз, после последнего вызова:

\begin{lstlisting}[style=customasmx86]
push a1
push a2
call ...
...
push a1
call ...
...
push a1
push a2
push a3
call ...
add esp, 24
\end{lstlisting}

Вот пример из реальной жизни:

\lstinputlisting[caption=x86,style=customasmx86]{patterns/03_printf/x86/add_example_RU.lst}

\clearpage
\myparagraph{MSVC и \olly}
\myindex{\olly}

Попробуем этот же пример в \olly.
Это один из наиболее популярных win32-отладчиков пользовательского режима.
Мы можем компилировать наш пример в MSVC 2012 
с опцией \GTT{/MD} что означает линковать с библиотекой \GTT{MSVCR*.DLL},
чтобы импортируемые функции были хорошо видны в отладчике.

Затем загружаем исполняемый файл в \olly.
Самая первая точка останова в \GTT{ntdll.dll}, нажмите F9 (запустить).
Вторая точка останова в \ac{CRT}-коде.
Теперь мы должны найти функцию \main.

Найдите этот код, прокрутив окно кода до самого верха (MSVC располагает функцию \main в самом начале секции кода): 

\begin{figure}[H]
\centering
\myincludegraphics{patterns/03_printf/x86/olly3_1.png}
\caption{\olly: самое начало функции \main}
\label{fig:printf3_olly_1}
\end{figure}

Кликните на инструкции \INS{PUSH EBP}, нажмите F2 (установка точки останова) и нажмите F9 (запустить).
Нам нужно произвести все эти манипуляции, чтобы пропустить \ac{CRT}-код, потому что нам он пока
не интересен.

\clearpage
Нажмите F8 (\stepover) 6 раз, т.е. пропустить 6 инструкций:

\begin{figure}[H]
\centering
\myincludegraphics{patterns/03_printf/x86/olly3_2.png}
\caption{\olly: перед исполнением \printf}
\label{fig:printf3_olly_2}
\end{figure}

Теперь \ac{PC} указывает на инструкцию \INS{CALL printf}.
\olly, как и другие отладчики, подсвечивает регистры со значениями, которые изменились.
Поэтому каждый раз когда мы нажимаем F8, \EIP изменяется и его значение подсвечивается красным.
\ESP также меняется, потому что значения заталкиваются в стек.\\
\\
Где находятся эти значения в стеке?
Посмотрите на правое нижнее окно в отладчике:

\begin{figure}[H]
\centering
\input{patterns/03_printf/x86/incl_olly3_stack}
\caption{\olly: стек с сохраненными значениями (красная рамка добавлена в графическом редакторе)}
\end{figure}

Здесь видно 3 столбца: адрес в стеке, значение в стеке и ещё дополнительный комментарий
от \olly. 
\olly может находить указатели на ASCII-строки в стеке, так что он показывает здесь \printf{}-строку.

Можно кликнуть правой кнопкой мыши на строке формата, кликнуть на \q{Follow in dump}
и строка формата появится в окне слева внизу, где всегда виден какой-либо участок памяти.
Эти значения в памяти можно редактировать.
Можно изменить саму строку формата, и тогда результат работы нашего примера будет другой.
В данном случае пользы от этого немного, но для упражнения это полезно,
чтобы начать чувствовать как тут всё работает.

\clearpage
Нажмите F8 (\stepover).

В консоли мы видим вывод:

\lstinputlisting{patterns/03_printf/x86/console.txt}

Посмотрим как изменились регистры и состояние стека: 

\begin{figure}[H]
\centering
\myincludegraphics{patterns/03_printf/x86/olly3_3.png}
\caption{\olly после исполнения \printf}
\label{fig:printf3_olly_3}
\end{figure}

Регистр \EAX теперь содержит \GTT{0xD} (13).
Всё верно: \printf возвращает количество выведенных символов.
Значение \EIP изменилось. Действительно, теперь здесь адрес инструкции после \INS{CALL printf}.
Значения регистров \ECX и \EDX также изменились.
Очевидно, внутренности функции \printf используют их для каких-то своих нужд.

Очень важно то, что значение \ESP не изменилось. И аргументы-значения в стеке также!
Мы ясно видим здесь и строку формата и соответствующие ей 3 значения, они всё ещё здесь.
Действительно, по соглашению вызовов \emph{cdecl}, вызываемая функция не возвращает \ESP назад.
Это должна делать вызывающая функция (\gls{caller}).

\clearpage
Нажмите F8 снова, чтобы исполнилась инструкция \INS{ADD ESP, 10}:

\begin{figure}[H]
\centering
\myincludegraphics{patterns/03_printf/x86/olly3_4.png}
\caption{\olly: после исполнения инструкции \INS{ADD ESP, 10}}
\label{fig:printf3_olly_4}
\end{figure}

\ESP изменился, но значения всё ещё в стеке!
Конечно, никому не нужно заполнять эти значения нулями или что-то в этом роде.
Всё что выше указателя стека (\ac{SP}) 
это \emph{шум} или \emph{\garbage{}} и не имеет особой ценности.
Было бы очень затратно по времени очищать ненужные элементы стека, к тому же, никому это и не нужно.

\myparagraph{GCC}

Скомпилируем то же самое в Linux при помощи GCC 4.4.1 и посмотрим на результат в \IDA:

\lstinputlisting[style=customasmx86]{patterns/03_printf/x86/x86_1.asm}

Можно сказать, что этот короткий код, созданный GCC, отличается от кода MSVC только способом помещения 
значений в стек.
Здесь GCC снова работает со стеком напрямую без \PUSH/\POP.

\myparagraph{GCC и GDB}
\myindex{GDB}

Попробуем также этот пример и в \ac{GDB} в Linux.

\GTT{-g} означает генерировать отладочную информацию в выходном исполняемом файле.

\begin{lstlisting}
$ gcc 1.c -g -o 1
\end{lstlisting}

\begin{lstlisting}
$ gdb 1
GNU gdb (GDB) 7.6.1-ubuntu
...
Reading symbols from /home/dennis/polygon/1...done.
\end{lstlisting}

\begin{lstlisting}[caption=установим точку останова на \printf]
(gdb) b printf
Breakpoint 1 at 0x80482f0
\end{lstlisting}

Запукаем.
У нас нет исходного кода функции, поэтому \ac{GDB} не может его показать.

\begin{lstlisting}
(gdb) run
Starting program: /home/dennis/polygon/1 

Breakpoint 1, __printf (format=0x80484f0 "a=%d; b=%d; c=%d") at printf.c:29
29	printf.c: No such file or directory.
\end{lstlisting}

Выдать 10 элементов стека. Левый столбец~--- это адрес в стеке.

\begin{lstlisting}
(gdb) x/10w $esp
0xbffff11c:	0x0804844a	0x080484f0	0x00000001	0x00000002
0xbffff12c:	0x00000003	0x08048460	0x00000000	0x00000000
0xbffff13c:	0xb7e29905	0x00000001
\end{lstlisting}

Самый первый элемент это \ac{RA} (\GTT{0x0804844a}).
Мы можем удостовериться в этом, дизассемблируя память по этому адресу:

\begin{lstlisting}[label=NOP_as_XCHG_example,style=customasmx86]
(gdb) x/5i 0x0804844a
   0x804844a <main+45>:	mov    $0x0,%eax
   0x804844f <main+50>:	leave  
   0x8048450 <main+51>:	ret    
   0x8048451:	xchg   %ax,%ax
   0x8048453:	xchg   %ax,%ax
\end{lstlisting}

Две инструкции \INS{XCHG} это холостые инструкции, аналогичные \ac{NOP}.

Второй элемент (\GTT{0x080484f0}) это адрес строки формата:

\begin{lstlisting}
(gdb) x/s 0x080484f0
0x80484f0:	"a=%d; b=%d; c=%d"
\end{lstlisting}

Остальные 3 элемента (1, 2, 3) это аргументы функции \printf.
Остальные элементы это может быть и мусор в стеке, но могут быть и значения
от других функций, их локальные переменные, итд.
Пока что мы можем игнорировать их.

Исполняем \q{finish}. 
Это значит исполнять все инструкции до самого конца функции. 
В данном случае это означает исполнять до завершения \printf.

\begin{lstlisting}
(gdb) finish
Run till exit from #0  __printf (format=0x80484f0 "a=%d; b=%d; c=%d") at printf.c:29
main () at 1.c:6
6		return 0;
Value returned is $2 = 13
\end{lstlisting}

\ac{GDB} показывает, что вернула \printf в \EAX (13).
Это, так же как и в примере с \olly, количество напечатанных символов.

А ещё мы видим \q{return 0;} и что это выражение находится в файле \GTT{1.c} в строке 6.
Действительно, файл \GTT{1.c} лежит в текущем директории и \ac{GDB} находит там эту строку.
Как \ac{GDB} знает, какая строка Си-кода сейчас исполняется?
Компилятор, генерируя отладочную информацию, также сохраняет информацию о соответствии строк в исходном коде и адресов инструкций.
GDB это всё-таки отладчик уровня исходных текстов.

Посмотрим регистры.
13 в \EAX:

\begin{lstlisting}
(gdb) info registers
eax            0xd	13
ecx            0x0	0
edx            0x0	0
ebx            0xb7fc0000	-1208221696
esp            0xbffff120	0xbffff120
ebp            0xbffff138	0xbffff138
esi            0x0	0
edi            0x0	0
eip            0x804844a	0x804844a <main+45>
...
\end{lstlisting}

Попробуем дизассемблировать текущие инструкции.
Стрелка указывает на инструкцию, которая будет исполнена следующей.

\begin{lstlisting}[style=customasmx86]
(gdb) disas
Dump of assembler code for function main:
   0x0804841d <+0>:	push   %ebp
   0x0804841e <+1>:	mov    %esp,%ebp
   0x08048420 <+3>:	and    $0xfffffff0,%esp
   0x08048423 <+6>:	sub    $0x10,%esp
   0x08048426 <+9>:	movl   $0x3,0xc(%esp)
   0x0804842e <+17>:	movl   $0x2,0x8(%esp)
   0x08048436 <+25>:	movl   $0x1,0x4(%esp)
   0x0804843e <+33>:	movl   $0x80484f0,(%esp)
   0x08048445 <+40>:	call   0x80482f0 <printf@plt>
=> 0x0804844a <+45>:	mov    $0x0,%eax
   0x0804844f <+50>:	leave  
   0x08048450 <+51>:	ret    
End of assembler dump.
\end{lstlisting}

По умолчанию \ac{GDB} показывает дизассемблированный листинг в формате AT\&T.
Но можно также переключиться в формат Intel:

\begin{lstlisting}[style=customasmx86]
(gdb) set disassembly-flavor intel
(gdb) disas
Dump of assembler code for function main:
   0x0804841d <+0>:	push   ebp
   0x0804841e <+1>:	mov    ebp,esp
   0x08048420 <+3>:	and    esp,0xfffffff0
   0x08048423 <+6>:	sub    esp,0x10
   0x08048426 <+9>:	mov    DWORD PTR [esp+0xc],0x3
   0x0804842e <+17>:	mov    DWORD PTR [esp+0x8],0x2
   0x08048436 <+25>:	mov    DWORD PTR [esp+0x4],0x1
   0x0804843e <+33>:	mov    DWORD PTR [esp],0x80484f0
   0x08048445 <+40>:	call   0x80482f0 <printf@plt>
=> 0x0804844a <+45>:	mov    eax,0x0
   0x0804844f <+50>:	leave  
   0x08048450 <+51>:	ret    
End of assembler dump.
\end{lstlisting}

Исполняем следующую строку \CCpp{}-кода.
\ac{GDB} покажет закрывающуюся скобку, означая, что это конец блока в функции.

\begin{lstlisting}
(gdb) step
7	};
\end{lstlisting}

Посмотрим регистры после исполнения инструкции \INS{MOV EAX, 0}.
\EAX здесь уже действительно ноль.

\begin{lstlisting}
(gdb) info registers
eax            0x0	0
ecx            0x0	0
edx            0x0	0
ebx            0xb7fc0000	-1208221696
esp            0xbffff120	0xbffff120
ebp            0xbffff138	0xbffff138
esi            0x0	0
edi            0x0	0
eip            0x804844f	0x804844f <main+50>
...
\end{lstlisting}
}
\DE{\subsubsection{x86}

\myindex{x86!\Instructions!LOOP}
Es gibt einen speziellen \LOOP Befehl im x86 Befehlssatz, der den Wert des
Registers \ECX prüft und falls dieser ungleich 0 ist, dekrementiert und danach
die den control flow wieder an das Label des \LOOP Operanden übergibt.
Vermutlich ist dieser Befehl nicht allzu geläufig und es gibt keine modernen
Compiler, welche ihn automatisch erzeugen. Wenn man also diesen Befehl irgendwo
im Code entdeckt, dann ist es äußerst wahrscheinlich, dass es sich um ein
handgeschriebenes Stück Assemblercode handelt.

\par In \CCpp werden Schleifen normalerweise mittels \TT{for()}-, \TT{while()}- oder
\TT{do/while()}-Ausdrücken erzeugt.

Starten wir mit \TT{for()}.

\myindex{\CLanguageElements!for}

Dieser Ausdruck definiert eine Schleifeninitialisierung (setzt den Zähler auf
einen Startwert), definiert eine Schleifenbedingung (ist der Zähler größer als
ein Grenzwert?), legt fest, was in jedem Durchlauf
(\glslink{increment}{Inkrement}/\glslink{decrement}{Dekrement}) geschieht und umschließt einen
Schleifenkörper.

\lstinputlisting[style=customc]{patterns/09_loops/simple/loops_1_DE.c}

Der erzeugte Code besteht ebenfalls aus vier Teilen.

Beginnen wir mit einem einfachen Beispiel:

\lstinputlisting[label=loops_src,style=customc]{patterns/09_loops/simple/loops_2.c}

Ergebnis (MSVC 2010):

\lstinputlisting[caption=MSVC 2010,style=customasmx86]{patterns/09_loops/simple/1_MSVC_DE.asm}

Hier gibt es nichts Besonderes zu sehen.

GCC 4.4.1 erzeugt einen fast identischen Code mit nur einen kleinen Unterschied:

\lstinputlisting[caption=GCC 4.4.1,style=customasmx86]{patterns/09_loops/simple/1_GCC_DE.asm}

Schauen wir uns nun an, was wir erhalten, wenn wir die Optimierung aktivieren
(\TT{\Ox}):

\lstinputlisting[caption=\Optimizing MSVC,style=customasmx86]{patterns/09_loops/simple/1_MSVC_Ox.asm}

Was hier passiert ist, dass der Speicherplatz für die $i$ Variable nicht mehr
auf dem lokalen Stack bereitgestellt wird, sondern das extra ein Register, \ESI,
hierfür verwendet wird. Dies ist bei derartig kleinen Funktionen möglich, wenn
nicht zu viele lokalen Variablen existieren.

Wichtig ist, dass die \ttf Funktion den Wert im Register \ESI nicht verändern
darf. Unser Compiler ist sich dieser Sache hier sicher. 
Und falls der Compiler entscheidet, das \ESI auch innerhalb der Funktion \ttf zu
verwenden, würde der Wert des Registers im Funktionsprolog gesichert und im
Funktionsepilog wiederhergestellt werden; fast genauso wie im folgenden Listing.
Man beachte das \TT{PUSH ESI/POP ESI} bei Funktionsbeginn und -ende. 
 
Probieren wir aus, was GCC 4.4.1 mit maximaler Optimierung (\Othree option)
liefert:

\lstinputlisting[caption=\Optimizing GCC 4.4.1,style=customasmx86]{patterns/09_loops/simple/1_GCC_O3.asm}

\myindex{Loop unwinding}

Aha, GCC hat unsere Schleife unrolled (d.h. ausgerollt).

\Gls{loop unwinding} hat Vorteile in Fällen, in denen es nicht viele
Schleifendurchläufe gibt und Ausführungszeit durch das Weglassen der
Befehle für die Kontrollstrukturen der Schleife gewonnen werden kann. 
Andererseits ist der erzeugte Code natürlich deutlich länger.

Große Schleifen zu \textit{unrollen} ist heutzutage nicht empfehlenswert, denn
größere Funktionen erfordern einen größeren Cache-Fußabdruck.%
%
\footnote{Ein hervorragender Artikel zum Thema: \DrepperMemory.
Für weitere Empfehlungen von Intel zum Unrolling siehe hier: 
\InSqBrackets{\IntelOptimization 3.4.1.7}.}.

Gut, nun wollen wir den Höchstwert der Variable $i$ auf 100 setzen und
kompilieren erneut. GCC liefert:

\lstinputlisting[caption=GCC,style=customasmx86]{patterns/09_loops/simple/2_GCC_DE.asm}

Das Ergebnis ist sehr ähnlich dem, das MSVC 2010 mit Optimierung (\Ox) erzeugt,
mit der Ausnahme, dass das \EBX Register für die Variable $i$ verwendet wird.

GCC ist sicher, dass das Register innerhalb der \ttf Funktion nicht verändert
wird und sollte dies doch der Fall sein, dass es im Funktionsprolog gesichert
und im Funktionsepilog wiederhergestellt werden wird, genau wie hier in der
\main Funktion.

\clearpage
\subsubsection{x86: \olly}
\myindex{\olly}

Wir kompilieren unser Beispiel in MSVC 2010 mit den Optionen \Ox und \Obzero und
laden es in \olly.

Es scheint, dass \olly in der Lage ist, einfache Schleifen zu erkennen und in
eckigen Klammern darzustellen, um die Übersichtlichkeit zu erhöhen:

\begin{figure}[H]
\centering
\myincludegraphics{patterns/09_loops/simple/olly1.png}
\caption{\olly: \main Einstieg}
\label{fig:loops_olly_1}
\end{figure}

Verfolgen mit (F8~--- \stepover) zeigt \ESI 
\glslink{increment}{incrementing}.
Hier ist zum Beispiel, $ESI=i=6$:

\begin{figure}[H]
\centering
\myincludegraphics{patterns/09_loops/simple/olly2.png}
\caption{\olly: Schleifenkörper wird gerade ausgeführt für $i=6$}
\label{fig:loops_olly_2}
\end{figure}

9 ist der letzte Wert in der Schleife
Deshalb triggert \JL nach der \glslink{increment}{inkrement}-Anweisung nicht und die Funktion
wird beendet:

\begin{figure}[H]
\centering
\myincludegraphics{patterns/09_loops/simple/olly3.png}
\caption{\olly: $ESI=10$, Ende der Schleife}
\label{fig:loops_olly_3}
\end{figure}

\subsubsection{x86: tracer}
\myindex{tracer}

Wie wir bemerken ist es nicht sonderlich komfortabel, Werte im Debugger manuell
nachzuverfolgen. Aus diesem Grund probieren wir \tracer aus.

Wir öffnen das kompilierte Beispiel in \IDA, finden die Adresse mit dem Befehl
\INS{PUSH ESI} (das einzige Argument an \ttf übergebend), welche hier
\TT{0x401026} ist und aktivieren den \tracer:

\begin{lstlisting}
tracer.exe -l:loops_2.exe bpx=loops_2.exe!0x00401026
\end{lstlisting}

\TT{BPX} setzt einen Breakpoint an der Adresse und der Tracer zeigt uns den
momentanen Status der Register an. In \TT{tracer.log} sehen wir das Folgende:

\lstinputlisting{patterns/09_loops/simple/tracer.log}

Wir sehen wie der Wert des \ESI Registers sich schrittweise von 2 zu 9
verändert. 

Mehr noch, der \tracer kann alle Registerwerte für alle Adressen innerhalb der
Funktion zusammensammeln. Dies wird hier mit \emph{trace} (dt. Nachverfolgung)
bezeichnet. Jeder Befehl wird verfolgt, alle interessanten Registerwerte werden
aufgezeichnet.

Danach die ein \IDA .idc-script erzeugt, das Kommentare hinzufügt. Wir haben
also herausgefunden, dass die Adresse der \main Funktion \TT{0x00401020} ist und
wir führen nun das Folgende aus:

\begin{lstlisting}
tracer.exe -l:loops_2.exe bpf=loops_2.exe!0x00401020,trace:cc
\end{lstlisting}

\TT{BPF} setzt einen Breakpoint auf eine Funktion.

Als Ergebnis erahlten wir die Skripte \TT{loops\_2.exe.idc} und
\TT{loops\_2.exe\_clear.idc}.

\clearpage
Wir laden \TT{loops\_2.exe.idc} in \IDA und erhalten:

\begin{figure}[H]
\centering
\myincludegraphics{patterns/09_loops/simple/IDA_tracer_cc.png}
\caption{\IDA mit geladenem .idc-script}
\label{fig:loops_IDA_tracer}
\end{figure}

Wir sehen, dass der Wert von \ESI zu Beginn der Schleife zwischen 2 und 9 und
nach dem Inkrement zwischen 3 und 0xA (10) liegt. Wir sehen auch, dass die
Funktion \main mit dem Rückgabewert 0 in \EAX terminiert.

\tracer erzeugt ebenfalls die Datei \TT{loops\_2.exe.txt}, welche Informationen
darüber enthält, welcher Befehl wie oft ausgeführt wurde, sowie zugehörige
Registerwerte:

\lstinputlisting[caption=loops\_2.exe.txt]{patterns/09_loops/simple/loops_2.exe.txt}
\myindex{\GrepUsage}
An dieser Stelle können wir grep verwenden.

}
\FR{\subsubsection{MSVC: x86}

Voici ce que nous obtenons dans la sortie assembleur (MSVC 2010):

\lstinputlisting[style=customasmx86]{patterns/04_scanf/3_checking_retval/ex3_MSVC_x86.asm}

\myindex{x86!\Registers!EAX}
La fonction \glslink{caller}{appelante} (\main) a besoin du résultat de la fonction
\glslink{callee}{appelée}, donc la fonction \glslink{callee}{appelée} le renvoie
dans la registre \EAX.

\myindex{x86!\Instructions!CMP}
Nous le vérifions avec l'aide de l'instruction \TT{CMP EAX, 1} (\emph{CoMPare}).
En d'autres mots, nous comparons la valeur dans le registre \EAX avec 1.

\myindex{x86!\Instructions!JNE}
Une instruction de saut conditionnelle \JNE suit l'instruction \CMP. \JNE signifie
\emph{Jump if Not Equal} (saut si non égal).

Donc, si la valeur dans le registre \EAX n'est pas égale à 1, le \ac{CPU} va poursuivre
l'exécution à l'adresse mentionnée dans l'opérande \JNE, dans notre cas \TT{\$LN2@main}.
Passer le contrôle à cette adresse résulte en l'exécution par le \ac{CPU} de
\printf avec l'argument \TT{What you entered? Huh?}.
Mais si tout est bon, le saut conditionnel n'est pas pris, et un autre appel à \printf
est exécuté, avec deux arguments:\\
\TT{'You entered \%d...'} et la valeur de \TT{x}.

\myindex{x86!\Instructions!XOR}
\myindex{\CLanguageElements!return}
Puisque dans ce cas le second \printf n'a pas été exécuté, il y a un \JMP qui le précède (saut inconditionnel).
Il passe le contrôle au point après le second \printf et juste avant l'instruction \TT{XOR EAX, EAX}, qui implémente \TT{return 0}.

% FIXME internal \ref{} to x86 flags instead of wikipedia
\myindex{x86!\Registers!\Flags}
Donc, on peut dire que comparer une valeur avec une autre est \emph{usuellement} implémenté
par la paire d'instructions \CMP/\Jcc, où \emph{cc} est un \emph{code de condition}.
\CMP compare deux valeurs et met les flags\footnote{flags x86, voir aussi: \href{http://en.wikipedia.org/wiki/FLAGS_register_(computing)}{Wikipédia}.}
du processeur.
\Jcc vérifie ces flags et décide de passer LE Contrôle à l'adresse spécifiée ou non.

\myindex{x86!\Instructions!CMP}
\myindex{x86!\Instructions!SUB}
\myindex{x86!\Instructions!JNE}
\myindex{x86!\Registers!ZF}
\label{CMPandSUB} 
Cela peut sembler paradoxal, mais l'instruction \CMP est en fait un \SUB (soustraction).
Toutes les instructions arithmétiques mettent les flags du processeur, pas seulement \CMP.
Si nous comparons 1 et 1, $1-1$ donne 0 donc le flag \ZF va être mis (signifiant
que le dernier résultat est 0).
Dans aucune autre circonstance \ZF ne sera mis, sauf si les opérandes sont égaux.
\JNE vérifie seulement le flag \ZF et saute seulement si il n'est pas mis. \JNE
est un synonyme pour \JNZ (\emph{Jump if Not Zero} (saut si non zéro)).
L'assembleur génère le même opcode pour les instructions \JNE et \JNZ.
Donc, l'instruction \CMP peut être remplacée par une instruction \SUB et presque
tout ira bien, à la différence que \SUB altère la valeur du premier opérande.
\CMP est un \emph{SUB sans sauver le résultat, mais modifiant les flags}.

\subsubsection{MSVC: x86: IDA}

\myindex{IDA}
C'est le moment de lancer \IDA et d'essayer de faire quelque chose avec.
À propos, pour les débutants, c'est une bonne idée d'utiliser l'option \TT{/MD}
de MSVC, qui signifie que toutes les fonctions standards ne vont pas être liées
avec le fichier exécutable, mais vont à la place être importées depuis le fichier
\TT{MSVCR*.DLL}.
Ainsi il est plus facile de voir quelles fonctions standards sont utilisées et où.

En analysant du code dans \IDA, il est très utile de laisser des notes pour soi-même
(et les autres).
En la circonstance, analysons cet exemple, nous voyons que \TT{JNZ} sera déclenché
en cas d'erreur.
Donc il est possible de déplacer le curseur sur le label, de presser \q{n} et de
lui donner le nom \q{error}.
Créons un autre label---dans \q{exit}.
Voici mon résultat:

\lstinputlisting[style=customasmx86]{patterns/04_scanf/3_checking_retval/ex3.lst}

Maintenant, il est légèrement plus facile de comprendre le code.
Toutefois, ce n'est pas une bonne idée de commenter chaque instruction.

% FIXME draw button?
Vous pouvez aussi cacher (replier) des parties d'une fonction dans \IDA.
Pour faire cela, marquez le bloc, puis appuyez sur Ctrl-\q{--} sur le pavé numérique et
entrez le texte qui doit être affiché à la place.

Cachons deux blocs et donnons leurs un nom:

\lstinputlisting[style=customasmx86]{patterns/04_scanf/3_checking_retval/ex3_2.lst}

% FIXME draw button?
Pour étendre les parties de code précédemment cachées. utilisez Ctrl-\q{+} sur le
pavé numérique.

\clearpage
En appuyant sur \q{space}, nous voyons comment \IDA représente une fonction sous
forme de graphe:

\begin{figure}[H]
\centering
\myincludegraphics{patterns/04_scanf/3_checking_retval/IDA.png}
\caption{IDA en mode graphe}
\label{fig:ex3_IDA_1}
\end{figure}

Il y a deux flèches après chaque saut conditionnel: une verte et une rouge.
La flèche verte pointe vers le bloc qui sera exécuté si le saut est déclenché,
et la rouge sinon.

\clearpage
Il est possible de replier des n\œu{}ds dans ce mode et de leurs donner aussi un nom (\q{group nodes}).
Essayons avec 3 blocs:

\begin{figure}[H]
\centering
\myincludegraphics{patterns/04_scanf/3_checking_retval/IDA2.png}
\caption{IDA en mode graphe avec 3 nœuds repliés}
\label{fig:ex3_IDA_2}
\end{figure}

C'est très pratique.
On peut dire qu'une part importante du travail des rétro-ingénieurs (et de tout
autre chercheur également) est de réduire la quantité d'information avec laquelle
travailler.

\input{patterns/04_scanf/3_checking_retval/olly_FR.tex}

\clearpage
\subsubsection{MSVC: x86 + Hiew}
\myindex{Hiew}

Cela peut également être utilisé comme un exemple simple de modification de fichier
exécutable.
Nous pouvons essayer de modifier l'exécutable de telle sorte que le programme va
toujours afficher notre entrée, quelle qui'elle soit.

En supposant que l'exécutable est compilé avec la bibliothèque externe \TT{MSVCR*.DLL}
(i.e., avec l'option \TT{/MD}) \footnote{c'est aussi appelé \q{dynamic linking}},
nous voyons la fonction \main au début de la section \TT{.text}.
Ouvrons l'exécutable dans Hiew et cherchons le début de la section \TT{.text} (Enter,
F8, F6, Enter, Enter).

Nous pouvons voir cela:

\begin{figure}[H]
\centering
\myincludegraphics{patterns/04_scanf/3_checking_retval/hiew_1.png}
\caption{Hiew: fonction \main}
\label{fig:scanf_ex3_hiew_1}
\end{figure}

Hiew trouve les chaîne \ac{ASCIIZ} et les affiche, comme il le fait avec le nom
des fonctions importées.

\clearpage
Déplacez le curseur à l'adresse \TT{.00401027} (où se trouve l'instruction \TT{JNZ},
que l'on doit sauter), appuyez sur F3, et ensuite tapez \q{9090} (qui signifie deux
\ac{NOP}s):

\begin{figure}[H]
\centering
\myincludegraphics{patterns/04_scanf/3_checking_retval/hiew_2.png}
\caption{Hiew: remplacement de \TT{JNZ} par deux \ac{NOP}s}
\label{fig:scanf_ex3_hiew_2}
\end{figure}

Appuyez sur F9 (update). Maintenant, l'exécutable est sauvé sur le disque. Il va
se comporter comme nous le voulions.

Deux \ac{NOP}s ne constitue probablement pas l'approche la plus esthétique.
Une autre façon de modifier cette instruction est d'écrire simplement 0 dans le
second octet de l'opcode ((\gls{jump offset}), donc ce \TT{JNZ} va toujours sauter
à l'instruction suivante.

Nous pouvons également faire le contraire: remplacer le premier octet avec \TT{EB}
sans modifier le second octet (\gls{jump offset}).
Nous obtiendrions un saut inconditionnel qui est toujours déclenché.
Dans ce cas le message d'erreur sera affiché à chaque fois, peu importe l'entrée.

}
\JA{\subsubsection{x86}

\myparagraph{\NonOptimizing MSVC}

結果 (MSVC 2010):

\lstinputlisting[caption=MSVC 2010,style=customasmx86]{patterns/08_switch/1_few/few_msvc.asm}

実際、switch()でいくつかのcaseを持つ私たちの関数は、この構造に似ています。

\lstinputlisting[label=switch_few_ifelse,style=customc]{patterns/08_switch/1_few/few_analogue.c}

\myindex{\CLanguageElements!switch}
\myindex{\CLanguageElements!if}

いくつかのcaseでswitch()を使用する場合、ソースコード内の実際のswitch()か、
単にif文の組であるかどうかを確認することは不可能です。
\myindex{\SyntacticSugar}

これはswitch()が多段にネストされたif文との糖衣構文のようなものであることを意味します。

コンパイラが入力変数 $a$ を一時的なローカル変数\TT{tv64}に移動することを除いて、
生成されたコードには特に新しいことはありません。
\footnote{スタック内のローカル変数には接頭辞\TT{tv}が付きます。MSVCが内部変数として使用するために命名しています。}

これをGCC 4.4.1でコンパイルすると、最大限の最適化(\Othree option)を有効にしても
ほぼ同じ結果になります。

\myparagraph{\Optimizing MSVC}

% TODO separate various kinds of \TT
% idea: enclose command lines in a specific environment, like \cmdline{} 
% assembly instructions in \asm{} (now both \TT and \q{} are used),
% variables in,  like \var{}
% messages (string constants) in something else, like \strconst
% to separate them all. Now they all use \TT, which is not best
% \INS{} for all instructions including operands? --DY
では、MSVC(\Ox)の最適化を有効にしましょう:\TT{cl 1.c /Fa1.asm /Ox}

\label{JMP_instead_of_RET}
\lstinputlisting[caption=MSVC,style=customasmx86]{patterns/08_switch/1_few/few_msvc_Ox.asm}

ここで、汚いハックを見ることができます。

\myindex{x86!\Instructions!JZ}
\myindex{x86!\Instructions!JE}
\myindex{x86!\Instructions!SUB}

最初に、 $a$ の値を \EAX に置き、0を引きます。 EAXの値が0かどうかを確認するために行われますが、
そうであれば、 \ZF フラグがセットされます(例えば、0からの減算は0)
最初の条件ジャンプ \JE (\emph{Jump if Equal} またはあ同義語 \JZ~---\emph{Jump if Zero})は実行され、
制御フローは\TT{\$LN4@f}ラベルに渡されます。ここでは、 \TT{'zero'}メッセージが出力されます。
最初のジャンプが実行されない場合は、入力値から1が減算され、結果が0の場合、対応するジャンプが実行されます。

また、ジャンプが全く実行されない場合、制御フローは文字列引数\TT{'something unknown'}を \printf に渡します。

\label{jump_to_last_printf}
\myindex{\Stack}

次に、文字列ポインタが $a$ 変数に置かれ、 \printf が \CALL ではなく \JMP を介して呼び出されます。 
簡単に説明するとこうなります:
\gls{caller} は値をスタックにプッシュし、 \CALL 経由で関数を呼び出します。
\CALL 自体は戻りアドレス(\ac{RA})をスタックにプッシュし、関数アドレスへの無条件ジャンプを行います。
スタックポインタを移動させる命令が含まれていないため、任意の実行時点での関数は、次のスタックレイアウトを持ちます。

\begin{itemize}
\item\ESP---points to \ac{RA}
\item\TT{ESP+4}---points to the $a$ variable 
\end{itemize}

反対に、\printf をここで呼び出さなければならないときは、文字列を指し示す必要がある最初の\printf 引数を除いて、
全く同じスタックレイアウトが必要です。それが私たちのコードがすることです。

ファンクションの最初の引数を文字列のアドレスに置き換え、
関数 \ttf を直接呼び出しずに直接 \printf を呼び出すかのように、 \printf にジャンプします。
\printf は文字列を \gls{stdout} に出力し、 \RET 命令を実行します。スタックから\ac{RA}を取り出し、
制御フローは \ttf ではなく \ttf 関数の終りをバイパスして、 \ttf の \gls{caller} です。

\myindex{\CStandardLibrary!longjmp()}
\newcommand{\URLSJ}{\href{http://en.wikipedia.org/wiki/Setjmp.h}{wikipedia}}

% TODO \myref{}
\printf はすべての場合に  \ttf 関数の終わりで右に呼ばれるので、これはすべて可能です。
ある意味では、\TT{longjmp()}\footnote{\URLSJ}関数に似ています。
そしてもちろん、それはスピードのためにすべて行われます。

ARMコンパイラと同様のケースは、\q{\PrintfSeveralArgumentsSectionName}セクションに記載されています。
こちら:~(\myref{ARM_B_to_printf})

\input{patterns/08_switch/1_few/olly_JA.tex}

}
\IT{\subsubsection{x86}

\myindex{x86!\Instructions!LOOP}

Nell' instruction set x86, c'è una speciale istruzione di \LOOP per controllare il valore nel registro \ECX e
se non è 0, \gls{decrement} \ECX
e passa il controllo del flusso alla label nell' operando di \LOOP. 
Probabilmente questa istruzione non è molto conveniente, e non ci sono moderni compilatori che la inseriscono automaticamente.
Di conseguenza, se la vedete da qualche parte, probabilmente quella parte di codice assembly è stata scritta a mano.

\par

In \CCpp i cicli sono solitamente costruiti usando le istruzioni \TT{for()}, \TT{while()} o \TT{do/while()}.

Iniziamo con \TT{for()}.
\myindex{\CLanguageElements!for}

Questa istruzione definisce l'inizializzazione del ciclo (imposta un contatore di cicli ad un valore iniziale), 
la condizione di ciclo (il contatore è maggiore di un valore limite?), cosa viene eseguito ad ogni iterazione (\gls{increment}/\gls{decrement} il contatore)
e ovviamente il corpo del ciclo.

\lstinputlisting[style=customc]{patterns/09_loops/simple/loops_1_IT.c}

Anche il codice generato è composto da quattro parti.

Iniziamo con un semplice esempio:

\lstinputlisting[label=loops_src,style=customc]{patterns/09_loops/simple/loops_2.c}

Il risultato (MSVC 2010):

\lstinputlisting[caption=MSVC 2010,style=customasmx86]{patterns/09_loops/simple/1_MSVC_IT.asm}

Come possiamo vedere, nulla di speciale.

GCC 4.4.1 emette quasi lo stesso codice, con una sottile differenza:

\lstinputlisting[caption=GCC 4.4.1,style=customasmx86]{patterns/09_loops/simple/1_GCC_IT.asm}

Ora vediamo cosa ottieniamo con l'ottimizzazione impostata su (\TT{\Ox}):

\lstinputlisting[caption=\Optimizing MSVC,style=customasmx86]{patterns/09_loops/simple/1_MSVC_Ox.asm}

Quello che abbiamo è che lo spazio per la veriabile $i$ non è più allocato nello stack,
ma viene utilizzato un registro, \ESI.
Questo è possibile nelle piccole funzioni dove non ci somo molte variabili locali.

Una cosa importante è che la funzione \ttf non deve cambiare il valore in \ESI.
Il nostro compilatore c'è lo assicura. 
E se il compilatore decide di usare il registro \ESI anche nella funzione \ttf, il suo valore viene salvato durante il prologo della funzione e ripristinato durante l'epilogo della funzione,
similmente al nostro esempio: notare \TT{PUSH ESI/POP ESI}
all'inizio e fine della funzione.

Proviamo GCC 4.4.1 con la massima ottimizzazione impostata (opzione \Othree):

\lstinputlisting[caption=\Optimizing GCC 4.4.1,style=customasmx86]{patterns/09_loops/simple/1_GCC_O3.asm}

\myindex{Loop unwinding}

Huh, GCC ha appena "srotolato" il nostro ciclo.

\Gls{loop unwinding} è vantaggioso nel caso in cui non ci siano molte iterazioni, perchè possiamo ridurre il tempo di esecuzione rimuovenfo tutte le istruzioni di supporto ai cicli. 
Dall' altro lato, il codice risultante è ovviamente maggiore.

Srotolare grossi cicli non è raccomandato al giorno d'oggi, perchè grosse funzioni possono richiedere un ingombro della cache maggiore%
%
\footnote{Un ottimo articolo a riguardo: \DrepperMemory.
Qui ci sono altre raccomandazioni da Intel riguardo lo srotolamento dei cicli: 
\InSqBrackets{\IntelOptimization 3.4.1.7}.}.

OK, aumentiamo a 100 il massimo valore della variabile $i$ e proviamo nuovamente. GCC fa:

\lstinputlisting[caption=GCC,style=customasmx86]{patterns/09_loops/simple/2_GCC_IT.asm}

E' abbastanza simile a quello che produce MSVC 2010 con ottimizzazione (\Ox), 
con l'eccezione che il registro \EBX è allocato per la variabile $i$.

GCC è sicuro che questo registro non verrà modificato nella funzione \ttf 
e nel caso, verrà salvato durante il prologo della funzione e verrà ripristinato durante l'epilogo, 
proprio come in questo caso nella funzione \main.

\clearpage
\subsubsection{x86: \olly}
\myindex{\olly}

Compiliamo il nostro esempio con MSVC 2010 con le opzioni \Ox e \Obzero,
carichiamolo poi in \olly.

Sembrerebbe che \olly sia in grado di rilevare dei semplici cicli e ce li mostra tra parentesi quadre, per convenienza:

\begin{figure}[H]
\centering
\myincludegraphics{patterns/09_loops/simple/olly1.png}
\caption{\olly: inizio \main}
\label{fig:loops_olly_1}
\end{figure}

Tracciando (F8~--- \stepover) vediamo \ESI 
\glslink{increment}{incrementing}.
Qui, per esempio, $ESI=i=6$:

\begin{figure}[H]
\centering
\myincludegraphics{patterns/09_loops/simple/olly2.png}
\caption{\olly: corpo del ciclo appena eseguito con $i=6$}
\label{fig:loops_olly_2}
\end{figure}

9 è l'ultimo valore del ciclo.
Motivo per il quale, \JL non si attiva dopo \gls{increment} e la funzione concluderà:

\begin{figure}[H]
\centering
\myincludegraphics{patterns/09_loops/simple/olly3.png}
\caption{\olly: $ESI=10$, fine ciclo}
\label{fig:loops_olly_3}
\end{figure}

\subsubsection{x86: tracer}
\myindex{tracer}

Come possimo vedere, non è molto comodo tracciare manualmente nel debugger.
Questa è la ragione per cui proveremo ad usare \tracer.

Apriamo l'esempio compilato in \IDA, cerchiamo l'indirizzo dell' istruzione \INS{PUSH ESI}
(che passa l'unico argomento a \ttf), che è \TT{0x401026} in questo caso ed eseguiamo il \tracer:

\begin{lstlisting}
tracer.exe -l:loops_2.exe bpx=loops_2.exe!0x00401026
\end{lstlisting}

\TT{BPX} imposta solamente un breakpoint all' indirizzo e \tracer stamperà poi lo stato dei registri.

Questo è ciò che vediamo in  \TT{tracer.log}:

\lstinputlisting{patterns/09_loops/simple/tracer.log}

Vediamo il valore del registro \ESI, cambiare da 2 a 9.

Oltre a ciò, il \tracer può collezionare i valori del registro a tutti gli indirizzi all' interno della funzione.
Questo si chiama \emph{trace}.
Ogni istruzione viene tracciata, tutti i valori interessanti del registro vengono collezionati.

Dopodichè, viene generato un \IDA .idc-script, che aggiunge i commenti.
Quindi, abbiamo appreso in \IDA che l' indirizzo della funzione \main è \TT{0x00401020}, quindi eseguiamo:

\begin{lstlisting}
tracer.exe -l:loops_2.exe bpf=loops_2.exe!0x00401020,trace:cc
\end{lstlisting}

\TT{BPF} sta per "imposta breakpoint alla funzione".

Come risultato, otteniamo gli script \TT{loops\_2.exe.idc} e \TT{loops\_2.exe\_clear.idc}.

\clearpage
Carichiamo \TT{loops\_2.exe.idc} in \IDA e osserviamo:

\begin{figure}[H]
\centering
\myincludegraphics{patterns/09_loops/simple/IDA_tracer_cc.png}
\caption{\IDA con .idc-script caricato}
\label{fig:loops_IDA_tracer}
\end{figure}

Notiamo che \ESI può assumere valori da 2 a 9 all' inizio del corpo del ciclo,
ma da 3 a 0xA (10) dopo l'incremento.
Notiamo inoltre che il \main termina con 0 in \EAX.

\tracer genera inoltre \TT{loops\_2.exe.txt}, 
che contiene informazioni riguardo a quante volte ogni istruzione è stata eseguita e i valori del registro:

\lstinputlisting[caption=loops\_2.exe.txt]{patterns/09_loops/simple/loops_2.exe.txt}
\myindex{\GrepUsage}
Possiamo usare grep qui.

}

\EN{\mysection{Returning Values: redux}

Again, when we know about function prologue and epilogue, let's recompile an example returning a value
(\ref{ret_val_func}, \ref{lst:ret_val_func}) using non-optimizing GCC:

\lstinputlisting[caption=\NonOptimizing GCC 8.2 x64 (\assemblyOutput),style=customasmx86]{patterns/017_ret_redux/1.s}

Effective instructions here are \INS{MOV} and \INS{RET}, others are -- prologue and epilogue.

}

\subsubsection{MIPS}

\lstinputlisting[caption=\Optimizing GCC 4.4.5 (IDA),style=customasmMIPS]{patterns/04_scanf/3_checking_retval/MIPS_O3_IDA.lst}

\myindex{MIPS!\Instructions!BEQ}

\EN{\subsection{MIPS}

\lstinputlisting[caption=\Optimizing GCC 4.4.5,style=customasmMIPS]{patterns/05_passing_arguments/MIPS_O3_IDA_EN.lst}

The first four function arguments are passed in four registers prefixed by A-.

\myindex{MIPS!\Instructions!MULT}

There are two special registers in MIPS: HI and LO which are filled with the 64-bit result of the multiplication during the execution of the \TT{MULT} instruction.
\myindex{MIPS!\Instructions!MFLO}
\myindex{MIPS!\Instructions!MFHI}

These registers are accessible only by using the \TT{MFLO} and \TT{MFHI} instructions.
\TT{MFLO} here takes the low-part of the multiplication result and stores it into \$V0.
So the high 32-bit part of the multiplication result is dropped (the HI register content is not used).
Indeed: we work with 32-bit \Tint data types here.

\myindex{MIPS!\Instructions!ADDU}

Finally, \TT{ADDU} (\q{Add Unsigned}) adds the value of the third argument to the result.

\myindex{MIPS!\Instructions!ADD}
\myindex{MIPS!\Instructions!ADDU}
\myindex{Ada}
\myindex{Integer overflow}

There are two different addition instructions in MIPS: \TT{ADD} and \TT{ADDU}.
The difference between them is not related to signedness, but to exceptions. \TT{ADD} can raise an exception on overflow, which is sometimes useful\footnote{\url{http://blog.regehr.org/archives/1154}} and supported in Ada \ac{PL}, for instance.
\TT{ADDU} does not raise exceptions on overflow.

Since \CCpp does not support this, in our example we see \TT{ADDU} instead of \TT{ADD}.

The 32-bit result is left in \$V0.

\myindex{MIPS!\Instructions!JAL}
\myindex{MIPS!\Instructions!JALR}

There is a new instruction for us in \main: \TT{JAL} (\q{Jump and Link}). 

The difference between \INS{JAL} and \INS{JALR} is that a relative offset is encoded in the first instruction, 
while \INS{JALR} jumps to the absolute address stored in a register (\q{Jump and Link Register}).

Both \ttf and \main functions are located in the same object file, so the relative address of \ttf 
is known and fixed.
}
\RU{\subsection{MIPS}

\lstinputlisting[caption=\Optimizing GCC 4.4.5,style=customasmMIPS]{patterns/05_passing_arguments/MIPS_O3_IDA_RU.lst}

Первые 4 аргумента функции передаются в четырех регистрах с префиксами A-.

\myindex{MIPS!\Instructions!MULT}
В MIPS есть два специальных регистра: HI и LO, которые выставляются в 64-битный результат умножения
во время исполнения инструкции \TT{MULT}.

\myindex{MIPS!\Instructions!MFLO}
\myindex{MIPS!\Instructions!MFHI}
К регистрам можно обращаться только используя инструкции \TT{MFLO} и \TT{MFHI}.
Здесь \TT{MFLO} берет младшую часть результата умножения и записывает в \$V0.
Так что старшая 32-битная часть результата игнорируется (содержимое регистра HI не используется).
Действительно, мы ведь работаем с 32-битным типом \Tint.


\myindex{MIPS!\Instructions!ADDU}
И наконец, \TT{ADDU} (\q{Add Unsigned}~--- добавить беззнаковое) прибавляет значение третьего аргумента к результату.

\myindex{MIPS!\Instructions!ADD}
\myindex{MIPS!\Instructions!ADDU}
\myindex{Ada}
\myindex{Integer overflow}
В MIPS есть две разных инструкции сложения: \TT{ADD} и \TT{ADDU}.
На самом деле, дело не в знаковых числах, а в исключениях: \TT{ADD} может вызвать исключение
во время переполнения. Это иногда полезно\footnote{\url{http://blog.regehr.org/archives/1154}} и поддерживается,
например, в \ac{PL} Ada.

\TT{ADDU} не вызывает исключения во время переполнения.
А так как \CCpp не поддерживает всё это, мы видим здесь \TT{ADDU} вместо \TT{ADD}.

32-битный результат оставляется в \$V0.

\myindex{MIPS!\Instructions!JAL}
\myindex{MIPS!\Instructions!JALR}
В \main есть новая для нас инструкция: \TT{JAL} (\q{Jump and Link}). 
Разница между \INS{JAL} и \INS{JALR} в том, что относительное смещение кодируется в первой инструкции,
а \INS{JALR} переходит по абсолютному адресу, записанному в регистр (\q{Jump and Link Register}).

Обе функции \ttf и \main расположены в одном объектном файле, так что относительный адрес \ttf известен и фиксирован.

}
\IT{\scanf restituisce il risultato del suo lavoro nel registro \$V0. Ciò viene controllato all'indirizzo 0x004006E4
confrontando il valore in \$V0 con quello in \$V1 (1 era stato memorizzato in \$V1 precedentemente, a 0x004006DC).
\INS{BEQ} sta per \q{Branch Equal}.
Se i due valori sono uguali (cioè \scanf è terminata con successo), l'esecuzione salta all'indirizzo 0x0040070C.

}
\JA{\subsubsection{MIPS}

MIPSはいくつかのコプロセッサ(最大4個)をサポートすることができます。
そのうちの0番目\footnote{0から始まる}は特別な制御コプロセッサであり、最初のコプロセッサはFPUです。

ARMと同様に、MIPSコプロセッサはスタックマシンではなく、32個の32ビットレジスタ(\$F0-\$F31)を持ちます。
\myref{MIPS_FPU_registers}.

64ビットの \Tdouble 値を扱う必要がある場合、32ビットのFレジスタのペアが使用されます。

\lstinputlisting[caption=\Optimizing GCC 4.4.5 (IDA),style=customasmMIPS]{patterns/12_FPU/1_simple/MIPS_O3_IDA_JA.lst}

新しい命令は以下です。

\myindex{MIPS!\Instructions!LWC1}
\myindex{MIPS!\Instructions!DIV.D}
\myindex{MIPS!\Instructions!MUL.D}
\myindex{MIPS!\Instructions!ADD.D}
\begin{itemize}

\item \INS{LWC1}は32ビットワードを第1コプロセッサのレジスタにロードします(命令名は\q{1})。
\myindex{MIPS!\Pseudoinstructions!L.D}

一対の\INS{LWC1}命令を組み合わせて\INS{L.D}疑似命令にすることができます。

\item \INS{DIV.D}、 \INS{MUL.D}、 \INS{ADD.D}はそれぞれ除算、乗算、加算を行います
(接尾辞の\q{.D}は倍精度、\q{.S}は単精度を表します)

\end{itemize}

\myindex{MIPS!\Instructions!LUI}
\myindex{\CompilerAnomaly}
\label{MIPS_FPU_LUI}

また、奇妙なコンパイラの例外があります\INS{LUI}命令に疑問符がついています。 
\$V0 レジスタに64ビット定数の \Tdouble 型の一部をロードする理由を理解することは難しいです。 
これらの命令は何の効果もありません。 
% TODO did you try checking out compiler source code?
これについて何か知っているなら、著者に電子メール\footnote{\EMAILS}を送ってください。
}
\FR{\scanf renvoie le résultat de son traitement dans le registre \$V0. Il est testé à l'adresse 0x004006E4
en comparant la valeur dans \$V0 avec celle dans \$V1 (1 a été stocké dans \$V1 plus tôt, en 0x004006DC).
\INS{BEQ} signifie \q{Branch Equal} (branchement si égal).
Si les deux valeurs sont égales (i.e., succès), l'exécution saute à l'adresse 0x0040070C.
}



\EN{\mysection{Returning Values: redux}

Again, when we know about function prologue and epilogue, let's recompile an example returning a value
(\ref{ret_val_func}, \ref{lst:ret_val_func}) using non-optimizing GCC:

\lstinputlisting[caption=\NonOptimizing GCC 8.2 x64 (\assemblyOutput),style=customasmx86]{patterns/017_ret_redux/1.s}

Effective instructions here are \INS{MOV} and \INS{RET}, others are -- prologue and epilogue.

}

\EN{\mysection{Returning Values: redux}

Again, when we know about function prologue and epilogue, let's recompile an example returning a value
(\ref{ret_val_func}, \ref{lst:ret_val_func}) using non-optimizing GCC:

\lstinputlisting[caption=\NonOptimizing GCC 8.2 x64 (\assemblyOutput),style=customasmx86]{patterns/017_ret_redux/1.s}

Effective instructions here are \INS{MOV} and \INS{RET}, others are -- prologue and epilogue.

}

\EN{\mysection{Task manager practical joke (Windows Vista)}
\myindex{Windows!Windows Vista}

Let's see if it's possible to hack Task Manager slightly so it would detect more \ac{CPU} cores.

\myindex{Windows!NTAPI}

Let us first think, how does the Task Manager know the number of cores?

There is the \TT{GetSystemInfo()} win32 function present in win32 userspace which can tell us this.
But it's not imported in \TT{taskmgr.exe}.

There is, however, another one in \gls{NTAPI}, \TT{NtQuerySystemInformation()}, 
which is used in \TT{taskmgr.exe} in several places.

To get the number of cores, one has to call this function with the \TT{SystemBasicInformation} constant
as a first argument (which is zero
\footnote{\href{http://msdn.microsoft.com/en-us/library/windows/desktop/ms724509(v=vs.85).aspx}{MSDN}}).

The second argument has to point to the buffer which is getting all the information.

So we have to find all calls to the \\
\TT{NtQuerySystemInformation(0, ?, ?, ?)} function.
Let's open \TT{taskmgr.exe} in IDA. 
\myindex{Windows!PDB}

What is always good about Microsoft executables is that IDA can download the corresponding \gls{PDB} 
file for this executable and show all function names.

It is visible that Task Manager is written in \Cpp and some of the function names and classes are really 
speaking for themselves.
There are classes CAdapter, CNetPage, CPerfPage, CProcInfo, CProcPage, CSvcPage, 
CTaskPage, CUserPage.

Apparently, each class corresponds to each tab in Task Manager.

Let's visit each call and add comment with the value which is passed as the first function argument.
We will write \q{not zero} at some places, because the value there was clearly not zero, 
but something really different (more about this in the second part of this chapter).

And we are looking for zero passed as argument, after all.

\begin{figure}[H]
\centering
\myincludegraphics{examples/taskmgr/IDA_xrefs.png}
\caption{IDA: cross references to NtQuerySystemInformation()}
\end{figure}

Yes, the names are really speaking for themselves.

When we closely investigate each place where\\
\TT{NtQuerySystemInformation(0, ?, ?, ?)} is called,
we quickly find what we need in the \TT{InitPerfInfo()} function:

\lstinputlisting[caption=taskmgr.exe (Windows Vista),style=customasmx86]{examples/taskmgr/taskmgr.lst}

\TT{g\_cProcessors} is a global variable, and this name has been assigned by 
IDA according to the \gls{PDB} loaded from Microsoft's symbol server.

The byte is taken from \TT{var\_C20}. 
And \TT{var\_C58} is passed to\\
\TT{NtQuerySystemInformation()} 
as a pointer to the receiving buffer.
The difference between 0xC20 and 0xC58 is 0x38 (56).

Let's take a look at format of the return structure, which we can find in MSDN:

\begin{lstlisting}[style=customc]
typedef struct _SYSTEM_BASIC_INFORMATION {
    BYTE Reserved1[24];
    PVOID Reserved2[4];
    CCHAR NumberOfProcessors;
} SYSTEM_BASIC_INFORMATION;
\end{lstlisting}

This is a x64 system, so each PVOID takes 8 bytes.

All \emph{reserved} fields in the structure take $24+4*8=56$ bytes.

Oh yes, this implies that \TT{var\_C20} is the local stack is exactly the
\TT{NumberOfProcessors} field of the \TT{SYSTEM\_BASIC\_INFORMATION} structure.

Let's check our guess.
Copy \TT{taskmgr.exe} from \TT{C:\textbackslash{}Windows\textbackslash{}System32} 
to some other folder 
(so the \emph{Windows Resource Protection} 
will not try to restore the patched \TT{taskmgr.exe}).

Let's open it in Hiew and find the place:

\begin{figure}[H]
\centering
\myincludegraphics{examples/taskmgr/hiew2.png}
\caption{Hiew: find the place to be patched}
\end{figure}

Let's replace the \TT{MOVZX} instruction with ours.
Let's pretend we've got 64 CPU cores.

Add one additional \ac{NOP} (because our instruction is shorter than the original one):

\begin{figure}[H]
\centering
\myincludegraphics{examples/taskmgr/hiew1.png}
\caption{Hiew: patch it}
\end{figure}

And it works!
Of course, the data in the graphs is not correct.

At times, Task Manager even shows an overall CPU load of more than 100\%.

\begin{figure}[H]
\centering
\myincludegraphics{examples/taskmgr/taskmgr_64cpu_crop.png}
\caption{Fooled Windows Task Manager}
\end{figure}

The biggest number Task Manager does not crash with is 64.

Apparently, Task Manager in Windows Vista was not tested on computers with a large number of cores.

So there are probably some static data structure(s) inside it limited to 64 cores.

\subsection{Using LEA to load values}
\label{TaskMgr_LEA}

Sometimes, \TT{LEA} is used in \TT{taskmgr.exe} instead of \TT{MOV} to set the first argument of \\
\TT{NtQuerySystemInformation()}:

\lstinputlisting[caption=taskmgr.exe (Windows Vista),style=customasmx86]{examples/taskmgr/taskmgr2.lst}

\myindex{x86!\Instructions!LEA}

Perhaps \ac{MSVC} did so because machine code of \INS{LEA} is shorter than \INS{MOV REG, 5} (would be 5 instead of 4).

\INS{LEA} with offset in $-128..127$ range (offset will occupy 1 byte in opcode) with 32-bit registers is even shorter (for lack of REX prefix)---3 bytes.

Another example of such thing is: \myref{using_MOV_and_pack_of_LEA_to_load_values}.
}
\RU{\subsubsection{std::string}
\myindex{\Cpp!STL!std::string}
\label{std_string}

\myparagraph{Как устроена структура}

Многие строковые библиотеки \InSqBrackets{\CNotes 2.2} обеспечивают структуру содержащую ссылку 
на буфер собственно со строкой, переменную всегда содержащую длину строки 
(что очень удобно для массы функций \InSqBrackets{\CNotes 2.2.1}) и переменную содержащую текущий размер буфера.

Строка в буфере обыкновенно оканчивается нулем: это для того чтобы указатель на буфер можно было
передавать в функции требующие на вход обычную сишную \ac{ASCIIZ}-строку.

Стандарт \Cpp не описывает, как именно нужно реализовывать std::string,
но, как правило, они реализованы как описано выше, с небольшими дополнениями.

Строки в \Cpp это не класс (как, например, QString в Qt), а темплейт (basic\_string), 
это сделано для того чтобы поддерживать 
строки содержащие разного типа символы: как минимум \Tchar и \emph{wchar\_t}.

Так что, std::string это класс с базовым типом \Tchar.

А std::wstring это класс с базовым типом \emph{wchar\_t}.

\mysubparagraph{MSVC}

В реализации MSVC, вместо ссылки на буфер может содержаться сам буфер (если строка короче 16-и символов).

Это означает, что каждая короткая строка будет занимать в памяти по крайней мере $16 + 4 + 4 = 24$ 
байт для 32-битной среды либо $16 + 8 + 8 = 32$ 
байта в 64-битной, а если строка длиннее 16-и символов, то прибавьте еще длину самой строки.

\lstinputlisting[caption=пример для MSVC,style=customc]{\CURPATH/STL/string/MSVC_RU.cpp}

Собственно, из этого исходника почти всё ясно.

Несколько замечаний:

Если строка короче 16-и символов, 
то отдельный буфер для строки в \glslink{heap}{куче} выделяться не будет.

Это удобно потому что на практике, основная часть строк действительно короткие.
Вероятно, разработчики в Microsoft выбрали размер в 16 символов как разумный баланс.

Теперь очень важный момент в конце функции main(): мы не пользуемся методом c\_str(), тем не менее,
если это скомпилировать и запустить, то обе строки появятся в консоли!

Работает это вот почему.

В первом случае строка короче 16-и символов и в начале объекта std::string (его можно рассматривать
просто как структуру) расположен буфер с этой строкой.
\printf трактует указатель как указатель на массив символов оканчивающийся нулем и поэтому всё работает.

Вывод второй строки (длиннее 16-и символов) даже еще опаснее: это вообще типичная программистская ошибка 
(или опечатка), забыть дописать c\_str().
Это работает потому что в это время в начале структуры расположен указатель на буфер.
Это может надолго остаться незамеченным: до тех пока там не появится строка 
короче 16-и символов, тогда процесс упадет.

\mysubparagraph{GCC}

В реализации GCC в структуре есть еще одна переменная --- reference count.

Интересно, что указатель на экземпляр класса std::string в GCC указывает не на начало самой структуры, 
а на указатель на буфера.
В libstdc++-v3\textbackslash{}include\textbackslash{}bits\textbackslash{}basic\_string.h 
мы можем прочитать что это сделано для удобства отладки:

\begin{lstlisting}
   *  The reason you want _M_data pointing to the character %array and
   *  not the _Rep is so that the debugger can see the string
   *  contents. (Probably we should add a non-inline member to get
   *  the _Rep for the debugger to use, so users can check the actual
   *  string length.)
\end{lstlisting}

\href{http://gcc.gnu.org/onlinedocs/libstdc++/libstdc++-html-USERS-4.4/a01068.html}{исходный код basic\_string.h}

В нашем примере мы учитываем это:

\lstinputlisting[caption=пример для GCC,style=customc]{\CURPATH/STL/string/GCC_RU.cpp}

Нужны еще небольшие хаки чтобы сымитировать типичную ошибку, которую мы уже видели выше, из-за
более ужесточенной проверки типов в GCC, тем не менее, printf() работает и здесь без c\_str().

\myparagraph{Чуть более сложный пример}

\lstinputlisting[style=customc]{\CURPATH/STL/string/3.cpp}

\lstinputlisting[caption=MSVC 2012,style=customasmx86]{\CURPATH/STL/string/3_MSVC_RU.asm}

Собственно, компилятор не конструирует строки статически: да в общем-то и как
это возможно, если буфер с ней нужно хранить в \glslink{heap}{куче}?

Вместо этого в сегменте данных хранятся обычные \ac{ASCIIZ}-строки, а позже, во время выполнения, 
при помощи метода \q{assign}, конструируются строки s1 и s2
.
При помощи \TT{operator+}, создается строка s3.

Обратите внимание на то что вызов метода c\_str() отсутствует,
потому что его код достаточно короткий и компилятор вставил его прямо здесь:
если строка короче 16-и байт, то в регистре EAX остается указатель на буфер,
а если длиннее, то из этого же места достается адрес на буфер расположенный в \glslink{heap}{куче}.

Далее следуют вызовы трех деструкторов, причем, они вызываются только если строка длиннее 16-и байт:
тогда нужно освободить буфера в \glslink{heap}{куче}.
В противном случае, так как все три объекта std::string хранятся в стеке,
они освобождаются автоматически после выхода из функции.

Следовательно, работа с короткими строками более быстрая из-за м\'{е}ньшего обращения к \glslink{heap}{куче}.

Код на GCC даже проще (из-за того, что в GCC, как мы уже видели, не реализована возможность хранить короткую
строку прямо в структуре):

% TODO1 comment each function meaning
\lstinputlisting[caption=GCC 4.8.1,style=customasmx86]{\CURPATH/STL/string/3_GCC_RU.s}

Можно заметить, что в деструкторы передается не указатель на объект,
а указатель на место за 12 байт (или 3 слова) перед ним, то есть, на настоящее начало структуры.

\myparagraph{std::string как глобальная переменная}
\label{sec:std_string_as_global_variable}

Опытные программисты на \Cpp знают, что глобальные переменные \ac{STL}-типов вполне можно объявлять.

Да, действительно:

\lstinputlisting[style=customc]{\CURPATH/STL/string/5.cpp}

Но как и где будет вызываться конструктор \TT{std::string}?

На самом деле, эта переменная будет инициализирована даже перед началом \main.

\lstinputlisting[caption=MSVC 2012: здесь конструируется глобальная переменная{,} а также регистрируется её деструктор,style=customasmx86]{\CURPATH/STL/string/5_MSVC_p2.asm}

\lstinputlisting[caption=MSVC 2012: здесь глобальная переменная используется в \main,style=customasmx86]{\CURPATH/STL/string/5_MSVC_p1.asm}

\lstinputlisting[caption=MSVC 2012: эта функция-деструктор вызывается перед выходом,style=customasmx86]{\CURPATH/STL/string/5_MSVC_p3.asm}

\myindex{\CStandardLibrary!atexit()}
В реальности, из \ac{CRT}, еще до вызова main(), вызывается специальная функция,
в которой перечислены все конструкторы подобных переменных.
Более того: при помощи atexit() регистрируется функция, которая будет вызвана в конце работы программы:
в этой функции компилятор собирает вызовы деструкторов всех подобных глобальных переменных.

GCC работает похожим образом:

\lstinputlisting[caption=GCC 4.8.1,style=customasmx86]{\CURPATH/STL/string/5_GCC.s}

Но он не выделяет отдельной функции в которой будут собраны деструкторы: 
каждый деструктор передается в atexit() по одному.

% TODO а если глобальная STL-переменная в другом модуле? надо проверить.

}
\DE{\subsection{Minimale und maximale Werte berechnen}

\subsubsection{32-bit}

\lstinputlisting[style=customc]{patterns/07_jcc/minmax/minmax.c}

\lstinputlisting[caption=\NonOptimizing MSVC 2013,style=customasmx86]{patterns/07_jcc/minmax/minmax_MSVC_2013_DE.asm}

\myindex{x86!\Instructions!Jcc}
Diese beiden Funktionen unterscheiden sich nur hinsichtliche der bedingten Sprungbefehle:
\INS{JGE} (\q{Jump if Greater or Equal}) wird in der ersten verwendet
und \INS{JLE} (\q{Jump if Less or Equal}) in der zweiten.

\myindex{\CompilerAnomaly}
\label{MSVC_double_JMP_anomaly}
Hier gibt es jeweils einen unnötigen \JMP Befehl pro Funtion, den MSVC wahrscheinlich fehlerhafterweise dort belassen
hat.

\myparagraph{Verzweigungslos}

ARM im Thumb mode erinnert uns an den x86 Code:

\lstinputlisting[caption=\OptimizingKeilVI (\ThumbMode),style=customasmARM]{patterns/07_jcc/minmax/minmax_Keil_Thumb_O3_DE.s}

\myindex{ARM!\Instructions!Bcc}
Die Funktionen unterscheiden sich in den Verzweigebefehlen: \INS{BGT} und \INS{BLT}.
Es ist möglich im ARM mode conditional codes zu verwenden, sodass der Code kürzer ist.

\myindex{ARM!\Instructions!MOVcc}
\INS{MOVcc} wird nur ausgeführt, wenn die Bedingung erfüllt (d.h. wahr) ist:

\lstinputlisting[caption=\OptimizingKeilVI (\ARMMode),style=customasmARM]{patterns/07_jcc/minmax/minmax_Keil_ARM_O3_DE.s}

\myindex{x86!\Instructions!CMOVcc}
\Optimizing GCC 4.8.1 und der optimierende MSVC 2013 können den \INS{CMOVcc} Befehl verwenden, der analog zu
\INS{MOVcc} in ARM funktioniert:

\lstinputlisting[caption=\Optimizing MSVC 2013,style=customasmx86]{patterns/07_jcc/minmax/minmax_GCC481_O3_DE.s}

\subsubsection{64-bit}

\lstinputlisting[style=customc]{patterns/07_jcc/minmax/minmax64.c}
Hier findet ein unnötiges Verschieben von Variablen statt, aber der Code ist verständlich:

\lstinputlisting[caption=\NonOptimizing GCC 4.9.1 ARM64,style=customasmARM]{patterns/07_jcc/minmax/minmax64_GCC_49_ARM64_O0.s}

\myparagraph{Verzweigungslos}
Die Funtionsargumente müssen nicht vom Stack geladen werden, da sie sich bereits in den Registern befinden:

\lstinputlisting[caption=\Optimizing GCC 4.9.1 x64,style=customasmx86]{patterns/07_jcc/minmax/minmax64_GCC_49_x64_O3_DE.s}

MSVC 2013 tut beinahe das gleiche:

\myindex{ARM!\Instructions!CSEL}

ARM64 verfügt über den \INS{CSEL} Befehl, der genau wie \INS{MOVcc} in ARM oder \INS{CMOVcc} in x86 arbeitet; er hat
lediglich einen anderen Namen:
\q{Conditional SELect}.

\lstinputlisting[caption=\Optimizing GCC 4.9.1 ARM64,style=customasmARM]{patterns/07_jcc/minmax/minmax64_GCC_49_ARM64_O3_DE.s}

\subsubsection{MIPS}

Leider ist GCC 4.4.5 für MIPS nicht so gut:

\lstinputlisting[caption=\Optimizing GCC 4.4.5 (IDA),style=customasmMIPS]{patterns/07_jcc/minmax/minmax_MIPS_O3_IDA_DE.lst} 
Vergessen Sie nicht die \emph{branch delay slots}: der erste \INS{MOVE} wird \emph{vor} \INS{BEQZ} ausgeführt, der zweite
\INS{MOVE} wird nur dann ausgeführt, wenn die Verzweigung nicht genommen wird.


}
\FR{\subsection{Table \TT{X\$KSMLRU} dans \oracle}
\myindex{\oracle}

Il y a une mention d'une table spéciale dans la note \emph{Diagnosing and Resolving
Error ORA-04031 on the Shared Pool or Other Memory Pools [Video] [ID 146599.1]}:

\begin{framed}
\begin{quotation}
Il y a une table fixée appelée X\$KSMLRU qui suit les différentes allocations dans
le pool partagé qui force les autres objets du pool partagé à vieillir. Cette table
fixée peut être utilisée pour identifier ce qui cause une grosse allocation.

Si plusieurs objets sont supprimés périodiquement du pool partagé, alors ceci va poser des problèmes de temps de réponse
et va probablement provoquer des problèmes de contention du verrou de cache de bibliothèque lorsque
les objets seront rechargés dans le pool partagé.

Une chose inhabituelle à propos de la table fixée X\$KSMLRU est que le contenu de la table fixée est écrasé à chaque fois que
quelqu'uni effectue requête dans la table fixée. Ceci est fait puisque la table fixée ne contient que l'allocation la plus large
qui s'est produite. Les valeurs sont réinitialisées après avoir été sélectionnées, de sorte que les allocations importantes
suivantes puissent ête inscrites, même si elles ne sont pas aussi laarges que celles qui se sont produites précédemment.
À cause de cette réinitialisation, la sortie produite par la sélection de cette table doit être soigneusement conservée
puisqu'elle ne peut plus être récupérée après que la requête a été faite.
\end{quotation}
\end{framed}

Toutefois, comme on peut le vérifier facilement, le contenu de cette table est effacé à chaque fois qu'on l'interroge.
Pouvons-nous trouver pourquoi?
Retournons aux tables que nous connaissons déjà: \TT{kqftab} et \TT{kqftap} qui
sont générées avec l'aide d'\oracletables, qui a toutes les informations concernant les table X\$-.
Nous pouvons voir ici que la fonction \TT{ksmlrs()} est appelée pour préparer les éléments de cette table:

\begin{lstlisting}[caption=Résultat de \OracleTablesName]
kqftab_element.name: [X$KSMLRU] ?: [ksmlr] 0x4 0x64 0x11 0xc 0xffffc0bb 0x5
kqftap_param.name=[ADDR] ?: 0x917 0x0 0x0 0x0 0x4 0x0 0x0
kqftap_param.name=[INDX] ?: 0xb02 0x0 0x0 0x0 0x4 0x0 0x0
kqftap_param.name=[INST_ID] ?: 0xb02 0x0 0x0 0x0 0x4 0x0 0x0
kqftap_param.name=[KSMLRIDX] ?: 0xb02 0x0 0x0 0x0 0x4 0x0 0x0
kqftap_param.name=[KSMLRDUR] ?: 0xb02 0x0 0x0 0x0 0x4 0x4 0x0
kqftap_param.name=[KSMLRSHRPOOL] ?: 0xb02 0x0 0x0 0x0 0x4 0x8 0x0
kqftap_param.name=[KSMLRCOM] ?: 0x501 0x0 0x0 0x0 0x14 0xc 0x0
kqftap_param.name=[KSMLRSIZ] ?: 0x2 0x0 0x0 0x0 0x4 0x20 0x0
kqftap_param.name=[KSMLRNUM] ?: 0x2 0x0 0x0 0x0 0x4 0x24 0x0
kqftap_param.name=[KSMLRHON] ?: 0x501 0x0 0x0 0x0 0x20 0x28 0x0
kqftap_param.name=[KSMLROHV] ?: 0xb02 0x0 0x0 0x0 0x4 0x48 0x0
kqftap_param.name=[KSMLRSES] ?: 0x17 0x0 0x0 0x0 0x4 0x4c 0x0
kqftap_param.name=[KSMLRADU] ?: 0x2 0x0 0x0 0x0 0x4 0x50 0x0
kqftap_param.name=[KSMLRNID] ?: 0x2 0x0 0x0 0x0 0x4 0x54 0x0
kqftap_param.name=[KSMLRNSD] ?: 0x2 0x0 0x0 0x0 0x4 0x58 0x0
kqftap_param.name=[KSMLRNCD] ?: 0x2 0x0 0x0 0x0 0x4 0x5c 0x0
kqftap_param.name=[KSMLRNED] ?: 0x2 0x0 0x0 0x0 0x4 0x60 0x0
kqftap_element.fn1=ksmlrs
kqftap_element.fn2=NULL
\end{lstlisting}

\myindex{tracer}
En effet, avec l'aide de \tracer, il est facile de voir que cette fonction est appelée
à chaque fois que nous interrogeons la table \TT{X\$KSMLRU}.

\myindex{\CStandardLibrary!memset()}
Ici nous voyons une référence aux fonctions \TT{ksmsplu\_sp()} et \TT{ksmsplu\_jp()},
chacune d'elles appelle \TT{ksmsplu()} à la fin.
À la fin de la fonction \TT{ksmsplu()} nous voyons un appel à \TT{memset()}:

\begin{lstlisting}[caption=ksm.o,style=customasmx86]
...

.text:00434C50 loc_434C50:    ; DATA XREF: .rdata:off\_5E50EA8
.text:00434C50         mov     edx, [ebp-4]
.text:00434C53         mov     [eax], esi
.text:00434C55         mov     esi, [edi]
.text:00434C57         mov     [eax+4], esi
.text:00434C5A         mov     [edi], eax
.text:00434C5C         add     edx, 1
.text:00434C5F         mov     [ebp-4], edx
.text:00434C62         jnz     loc_434B7D
.text:00434C68         mov     ecx, [ebp+14h]
.text:00434C6B         mov     ebx, [ebp-10h]
.text:00434C6E         mov     esi, [ebp-0Ch]
.text:00434C71         mov     edi, [ebp-8]
.text:00434C74         lea     eax, [ecx+8Ch]
.text:00434C7A         push    370h            ; Size
.text:00434C7F         push    0               ; Val
.text:00434C81         push    eax             ; Dst
.text:00434C82         call    __intel_fast_memset
.text:00434C87         add     esp, 0Ch
.text:00434C8A         mov     esp, ebp
.text:00434C8C         pop     ebp
.text:00434C8D         retn
.text:00434C8D _ksmsplu  endp
\end{lstlisting}

Des constructions comme \TT{memset (block, 0, size)} sont souvent utilisées pour
mettre à zéro un bloc de mémoire.
Que se passe-t-il si nous prenons le risque de bloquer l'appel à \TT{memset (block, 0, size)}
et regardons ce qui se produit?

\myindex{tracer}

Lançons \tracer avec les options suivantes: mettre un point d'arrêt en \TT{0x434C7A}
(le point où les arguments sont passés à \TT{memset()}), afin que \tracer mette le compteur de programme \TT{EIP} au point
où les arguments passés à \TT{memset()} sont éffacés (en \TT{0x434C8A}).
On peut dire que nous simulons juste un saut inconditionnel de l'adresse \TT{0x434C7A} à \TT{0x434C8A}.

\begin{lstlisting}
tracer -a:oracle.exe bpx=oracle.exe!0x00434C7A,set(eip,0x00434C8A)
\end{lstlisting}

(Important: toutes ces adresses sont valides seulement pour la version win32 de \oracle 11.2)

En effet, nous pouvons maintenant interroger la table \TT{X\$KSMLRU} autant de fois
que nous voulons et elle n'est plus du tout effacée!

% \sout{Do not try this at home ("MythBusters")}
Au cas où, n'essayez pas ceci sur vos serveurs de production.

Ce n'est probablement pas un comportement très utile ou souhaité, mais comme une
expérience pour déterminer l'emplacement d'un bout de code dont nous avons besoin,
ça remplit parfaitement notre besoin!

}
\JA{\subsection{文字列へのポインタの配列}
\label{array_of_pointers_to_strings}

ここでは、ポインタの配列の例を示します。

\lstinputlisting[caption=Get month name,label=get_month1,style=customc]{patterns/13_arrays/45_month_1D/month1_JA.c}

\subsubsection{x64}

\lstinputlisting[caption=\Optimizing MSVC 2013 x64,style=customasmx86]{patterns/13_arrays/45_month_1D/month1_MSVC_2013_x64_Ox.asm}

コードはとても単純です。

\begin{itemize}

\item
\myindex{x86!\Instructions!MOVSXD}

最初の\INS{MOVSXD}命令は、 \ECX ( $month$ 引数が渡される)から32ビットの値を
符号拡張付きの \RAX ( $month$ 引数は \Tint 型なので)にコピーします。

符号拡張の理由は、この32ビット値が他の64ビット値との計算に使用されるためです。

したがって、64ビット142に昇格させる必要があります。%
\footnote{やや奇妙ですが、負の配列インデックスはここで $month$ として渡すことができます
(負の配列インデックスは後で説明します:\ref{negative_array_indices})。 % FIXME should be \myref{} here, but varioref package complains...
これが起こると、 \Tint 型の負の入力値が正しく符号拡張され、
テーブルの前の対応する要素が選択されます。 符号拡張なしでは正しく動作しません。}

\item
次にポインタテーブルのアドレスが \RCX にロードされます。

\item
最後に、入力値($month$)に8を掛けてアドレスに加算します。 
確かに:私たちは64ビット環境にあり、すべてのアドレス(またはポインタ)は正確に64ビット(または8バイト)の
記憶域を必要とします。 
したがって、各テーブル要素は8バイト幅です。 
それで、なぜ特定の要素 $month*8$ をスキップする必要があるのでしょうか。
これが \MOV が行うことです。 
さらに、この命令はこのアドレスの要素もロードします。 
1の場合、要素は\q{February}などを含む文字列へのポインタになります。

\end{itemize}

\Optimizing GCC 4.9はもっとよく仕事をこなします。
\footnote{GCCアセンブラ出力が排除するのに十分なほど整っていないので、\q{0+}がリストに残っていました。 
それは\emph{変位}であり、ここではゼロです。}

\begin{lstlisting}[caption=\Optimizing GCC 4.9 x64,style=customasmx86]
	movsx	rdi, edi
	mov	rax, QWORD PTR month1[0+rdi*8]
	ret
\end{lstlisting}

\myparagraph{32ビットMSVC}

32ビットMSVCコンパイラでもコンパイルしてみましょう。

\lstinputlisting[caption=\Optimizing MSVC 2013 x86,style=customasmx86]{patterns/13_arrays/45_month_1D/month1_MSVC_2013_x86_Ox.asm}

入力値は64ビットに拡張する必要がないので、そのまま使われます。

そして4倍されます。テーブル要素が32ビット(または4バイト)幅だからです。

% FIXME1 move to another file
\subsubsection{32ビット ARM}

\myparagraph{ARMモードでのARM}

\lstinputlisting[caption=\OptimizingKeilVI (\ARMMode),style=customasmARM]{patterns/13_arrays/45_month_1D/month1_Keil_ARM_O3.s}

% TODO Fix R1s

テーブルのアドレスはR1にロードされます。
\myindex{ARM!\Instructions!LDR}

残りのすべては \LDR 命令1つだけを使って行われます。

入力値 $month$ は2ビット左シフトします(4倍するのと同じです)。それから
R1に加えらえます(テーブルのアドレスの場所)。そしてテーブル要素はこのアドレスからロードされます。

32ビットテーブル要素はテーブルからR0にロードされます。

\myparagraph{ThumbモードでのARM}

コードはほとんど同じですが、より密度が低いです。 \LSL サフィックスは \LDR 命令では特定できないからです。

\begin{lstlisting}[style=customasmARM]
get_month1 PROC
        LSLS     r0,r0,#2
        LDR      r1,|L0.64|
        LDR      r0,[r1,r0]
        BX       lr
        ENDP
\end{lstlisting}

\subsubsection{ARM64}

\lstinputlisting[caption=\Optimizing GCC 4.9 ARM64,style=customasmARM]{patterns/13_arrays/45_month_1D/month1_GCC49_ARM64_O3.s}

\myindex{ARM!\Instructions!ADRP/ADD pair}

テーブルのアドレスは \ADRP/\ADD 命令の組を使ってX1にロードされます。

それから付随する要素 \LDR を使って選ばれて、W0を取ります(入力引数 $month$ の場所のレジスタ)。
左に3ビットシフトします(8倍するのと同じです)。
符号拡張し(\q{sxtw}サフィックスが暗示しています)、X0に加算します。
それから64ビット値がテーブルからX0にロードされます。

\subsubsection{MIPS}

\lstinputlisting[caption=\Optimizing GCC 4.4.5 (IDA),style=customasmMIPS]{patterns/13_arrays/45_month_1D/MIPS_O3_IDA_JA.lst}

\subsubsection{配列オーバーフロー}

関数は0~11の範囲の値を受け付けますが、12は通すでしょうか?
テーブルにはその場所の要素はありません。

なので関数はそこにたまたまある値をロードしてリターンします。

すぐ後で、他の関数がこのアドレスからテキスト文字列を取得しようとしてクラッシュするかもしれません。

例をwin64用としてMSVCでコンパイルして、テーブルの後にリンカーが何を配置したのかを \IDA で見てみましょう。

\lstinputlisting[caption=IDAでの実行可能ファイル,style=customasmx86]{patterns/13_arrays/45_month_1D/MSVC2012_win64_1.lst}

月の名前がそのあとに来ています。

プログラムは小さいので、データセグメントにパックされるデータは多くありません。
だから単に次の名前が来ています。
しかし注意すべきはリンカーが配置するように決定するのは\emph{どんなものも}ありえます。

だからもし12が関数に渡されたら?
13番目の要素がリターンされます。

CPUがそこにあるバイトを64ビットの値としてどのように扱うかをみてみましょう。

\lstinputlisting[caption=IDAでの実行可能ファイル,style=customasmx86]{patterns/13_arrays/45_month_1D/MSVC2012_win64_2.lst}

0x797261756E614Aです。

すぐ後で、他の関数(おそらく文字列を扱う関数)がこのアドレスでバイトを読み込もうとすると、
C言語の文字列を期待します。

十中八九、クラッシュします。この値は有効なアドレスのようには見えないからです。

\myparagraph{配列オーバーフロー保護}

\epigraph{失敗する可能性のあるものは、失敗する。}{マーフィーの法則}

あなたの関数を使用するプログラマはみな11より大きな値を引数として渡さないと
期待するのはちょっとナイーブです。

問題をできるだけ早く報告し停止することを意味する\q{fail early and fail loudly}
または\q{早く失敗する}という哲学があります。

\myindex{\CStandardLibrary!assert()}

そのような方法の1つに \CCpp のassertionがあります。

不正な値が通ってきたら、失敗するようにプログラムを変更できます。

\lstinputlisting[caption=assert()を追加,style=customc]{patterns/13_arrays/45_month_1D/month1_assert.c}

アサーションマクロは関数の開始時に妥当な値かチェックし、式が偽の場合に失敗します。

\lstinputlisting[caption=\Optimizing MSVC 2013 x64,style=customasmx86]{patterns/13_arrays/45_month_1D/MSVC2013_x64_Ox_checked.asm}

実際、assert() は関数ではなくマクロです。条件をチェックし、
行数とファイル名を他の関数に渡してユーザに情報を報告します。

ファイル名と条件の両方がUTF-16でエンコードされています。
行数も渡されます(29です)。

このメカニズムはおそらくすべてのコンパイラで同じです。
GCCはこのようにします。

\lstinputlisting[caption=\Optimizing GCC 4.9 x64,style=customasmx86]{patterns/13_arrays/45_month_1D/GCC491_x64_O3_checked.s}

GCCのマクロは利便性のために関数名も渡します。

何事もただではできませんが、サニタイズチェックもこれと同様です。

それはプログラムを遅くしますが、特にassert()マクロが小さなタイムクリティカルな関数で使用されると遅くなります。

なのでMSVCでは、例えばデバッグビルドではチェックを残し、リリースビルドでは取り除いたりします。
 
マイクロソフト\gls{Windows NT}カーネルは\q{チェックされた}と\q{フリー}ビルドです。
\footnote{\href{http://msdn.microsoft.com/en-us/library/windows/hardware/ff543450(v=vs.85).aspx}{msdn.microsoft.com/en-us/library/windows/hardware/ff543450(v=vs.85).aspx}}.

最初のものは妥当性チェック(\q{チェックされた}なので)があり、もう一つはチェックしていません(チェックが\q{フリー}なので)。

もちろん、 \q{チェックされた}カーネルはこれらのチェックのために遅く動作するので、通常はデバッグセッションでのみ使用されます。

% FIXME: ARM? MIPS?

\subsubsection{特定の文字へのアクセス}

文字列へのポインタの配列はこのようにアクセスできます。

\lstinputlisting[style=customc]{patterns/13_arrays/45_month_1D/month2_JA.c}

\dots \emph{month[3]}式は\emph{const char*}型をもつので、
5番目の文字列はこのアドレスに4バイトを足した式から取得します。

さて、\emph{main()}関数に渡された引数リストは同じデータ型を持ちます。

\lstinputlisting[style=customc]{patterns/13_arrays/45_month_1D/argv_JA.c}

似た構文ですが、2次元配列とは異なることを理解することが非常に重要です。
これについては後で検討します。

もう1つの重要なことに注意してください。アドレス指定される文字列は、各文字が\ac{ASCII}や拡張\ac{ASCII}のように1バイトを占めるシステムで
エンコードされなければなりません。 
UTF-8はここでは動作しません。
}
\IT{\subsection{Ottenere i valori massimo e minimo}

\subsubsection{32-bit}

\lstinputlisting[style=customc]{patterns/07_jcc/minmax/minmax.c}

\lstinputlisting[caption=\NonOptimizing MSVC 2013,style=customasmx86]{patterns/07_jcc/minmax/minmax_MSVC_2013_IT.asm}

\myindex{x86!\Instructions!Jcc}

Queste due funzioni differiscono solo per l'istruzione di salto condizionale: 
\INS{JGE} (\q{Jump if Greater or Equal}) è usata nella prima
e \INS{JLE} (\q{Jump if Less or Equal}) nella seconda.

\myindex{\CompilerAnomaly}
\label{MSVC_double_JMP_anomaly}

In ciascuna funzione c'è un'istruzione \JMP non necessaria, che MSVC ha probabilmente lasciato per sbaglio.

\myparagraph{Branchless}

ARM in modalità Thumb ci ricorda molto codice x86:

\lstinputlisting[caption=\OptimizingKeilVI (\ThumbMode),style=customasmARM]{patterns/07_jcc/minmax/minmax_Keil_Thumb_O3_IT.s}

\myindex{ARM!\Instructions!Bcc}

Le funzioni differiscono per le istruzioni di branching: \INS{BGT} e \INS{BLT}.
Essendo possibile usare suffissi condizionali in modalità ARM, il codice è più conciso.

\myindex{ARM!\Instructions!MOVcc}
\INS{MOVcc} viene eseguita se la condizione è soddisfatta:

\lstinputlisting[caption=\OptimizingKeilVI (\ARMMode),style=customasmARM]{patterns/07_jcc/minmax/minmax_Keil_ARM_O3_IT.s}

\myindex{x86!\Instructions!CMOVcc}
\Optimizing, GCC 4.8.1 e MSVC 2013 possono usare l'istruzione \INS{CMOVcc}, che è analoga a \INS{MOVcc} in ARM:

\lstinputlisting[caption=\Optimizing MSVC 2013,style=customasmx86]{patterns/07_jcc/minmax/minmax_GCC481_O3_IT.s}

\subsubsection{64-bit}

\lstinputlisting[style=customc]{patterns/07_jcc/minmax/minmax64.c}

C'è un po' di spostamento di valori non necessario, ma il codice è comprensibile:

\lstinputlisting[caption=\NonOptimizing GCC 4.9.1 ARM64,style=customasmARM]{patterns/07_jcc/minmax/minmax64_GCC_49_ARM64_O0.s}

\myparagraph{Branchless}

Non è necessario caricare gli argomenti della funzione dallo stack poiché sono già nei registri:

\lstinputlisting[caption=\Optimizing GCC 4.9.1 x64,style=customasmx86]{patterns/07_jcc/minmax/minmax64_GCC_49_x64_O3_IT.s}

MSVC 2013 fa pressoché lo stesso.

\myindex{ARM!\Instructions!CSEL}

ARM64 ha l'istruzione \INS{CSEL}, che funziona esattamente come \INS{MOVcc} in ARM o \INS{CMOVcc} in x86, cambia soltanto il nome:
\q{Conditional SELect}.

\lstinputlisting[caption=\Optimizing GCC 4.9.1 ARM64,style=customasmARM]{patterns/07_jcc/minmax/minmax64_GCC_49_ARM64_O3_IT.s}

\subsubsection{MIPS}

Sfortunatamente GCC 4.4.5 per MIPS non è altrettanto bravo:

\lstinputlisting[caption=\Optimizing GCC 4.4.5 (IDA),style=customasmMIPS]{patterns/07_jcc/minmax/minmax_MIPS_O3_IDA_IT.lst}

Non ci dimentichiamo dei \emph{branch delay slot}: la prima \INS{MOVE} è eseguita \emph{prima} di \INS{BEQZ}, 
la seconda \INS{MOVE} viene eseguita solo se il branch non è stato seguito.

}


\subsection{\Conclusion{}}

\subsubsection{x86}

Voici le squelette générique d'un saut conditionnel:

\begin{lstlisting}[caption=x86,style=customasmx86]
CMP registre, registre/valeur
Jcc true ; cc=condition code, code de condition
false:
... le code qui sera exécuté si le résultat de la comparaison est faux (false) ...
JMP exit 
true:
... le code qui sera exécuté si le résultat de la comparaison est vrai (true) ...
exit:
\end{lstlisting}

\subsubsection{ARM}

\begin{lstlisting}[caption=ARM,style=customasmARM]
CMP registre, registre/valeur
Bcc true ; cc=condition code
false:
... le code qui sera exécuté si le résultat de la comparaison est faux (false) ...
JMP exit 
true:
... le code qui sera exécuté si le résultat de la comparaison est vrai (true) ...
exit:
\end{lstlisting}

\subsubsection{MIPS}

\begin{lstlisting}[caption=Teste si égal à zéro (Branch if EQual Zero),style=customasmMIPS]
BEQZ REG, label
...
\end{lstlisting}

\begin{lstlisting}[caption=Teste si plus petit que zéro (Branch if Less Than Zero) en utilisant une pseudo instruction,style=customasmMIPS]
BLTZ REG, label
...
\end{lstlisting}

\begin{lstlisting}[caption=Teste si les valeurs sont égales (Branch if EQual),style=customasmMIPS]
BEQ REG1, REG2, label
...
\end{lstlisting}

\begin{lstlisting}[caption=Teste si les valeurs ne sont pas égales (Branch if Not Equal),style=customasmMIPS]
BNE REG1, REG2, label
...
\end{lstlisting}

\begin{lstlisting}[caption=Teste si REG2 est plus petit que REG3 (signé),style=customasmMIPS]
SLT REG1, REG2, REG3
BEQ REG1, label
...
\end{lstlisting}

\begin{lstlisting}[caption=Teste si REG2 est plus petit que REG3 (non signé),style=customasmMIPS]
SLTU REG1, REG2, REG3
BEQ REG1, label
...
\end{lstlisting}

\subsubsection{Sans branchement}

\myindex{ARM!\Instructions!MOVcc}
\myindex{x86!\Instructions!CMOVcc}
\myindex{ARM!\Instructions!CSEL}
Si le corps d'instruction conditionnelle est très petit, l'instruction de déplacement
conditionnel peut être utilisée:
\INS{MOVcc} en ARM (en mode ARM), \INS{CSEL} en ARM64, \INS{CMOVcc} en x86.

\myparagraph{ARM}

Il est possible d'utiliser les suffixes conditionnels pour certaines instructions
ARM:

\begin{lstlisting}[caption=ARM (\ARMMode),style=customasmARM]
CMP registre, registre/valeur
instr1_cc ; cette instruction sera exécutée si le code conditionnel est vrai (true)
instr2_cc ; cette autre instruction sera exécutée si cet autre code conditionnel est vrai (true)
... etc.
\end{lstlisting}

Bien sûr, il n'y a pas de limite au nombre d'instructions avec un suffixe de code
conditionnel, tant que les flags du CPU ne sont pas modifiés par l'une d'entre elles.
% FIXME: list of such instructions or \myref{} to it

\myindex{ARM!\Instructions!IT}

Le mode Thumb possède l'instruction \INS{IT}, permettant d'ajouter le suffixe conditionnel
pour les quatre instructions suivantes.
Lire à ce propos: \myref{ARM_Thumb_IT}.

\begin{lstlisting}[caption=ARM (\ThumbMode),style=customasmARM]
CMP registre, registre/valeur
ITEEE EQ ; met ces suffixes: if-then-else-else-else
instr1   ; instruction exécutée si la condition est vraie
instr2   ; instruction exécutée si la condition est fausse
instr3   ; instruction exécutée si la condition est fausse
instr4   ; instruction exécutée si la condition est fausse
\end{lstlisting}

\subsection{\Exercise}

(ARM64) Essayez de récrire le code pour \lstref{cond_ARM64} en supprimant toutes
les instructions de saut conditionnel et en utilisant l'instruction \TT{CSEL}.

