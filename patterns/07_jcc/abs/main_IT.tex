\subsection{Calcolo del valore assoluto}
\label{sec:abs}

Una funzione semplice:

\lstinputlisting[style=customc]{abs.c}

\subsubsection{\Optimizing MSVC}

Il codice solitamente generato è questo:

\lstinputlisting[caption=\Optimizing MSVC 2012 x64,style=customasmx86]{patterns/07_jcc/abs/abs_MSVC2012_Ox_x64_IT.asm}

GCC 4.9 fa più o meno lo stesso.

\subsubsection{\OptimizingKeilVI: \ThumbMode}

\lstinputlisting[caption=\OptimizingKeilVI: \ThumbMode,style=customasmARM]{patterns/07_jcc/abs/abs_Keil_thumb_O3_IT.s}

\myindex{ARM!\Instructions!RSB}

In ARM manca l'istruzione di negazione, quindi il compilatore Keil usa l'istruzione \q{Reverse Subtract}, che semplicemente sottrae gli operandi in modo inverso.

\subsubsection{\OptimizingKeilVI: \ARMMode}

In modalità ARM è possibile aggiungere codice condizionale ad alcune istruzioni, e questo è ciò che fa il compilatore keil:

\lstinputlisting[caption=\OptimizingKeilVI: \ARMMode,style=customasmARM]{patterns/07_jcc/abs/abs_Keil_ARM_O3_IT.s}

Adesso non ci sono più jump condizionali, e ciò è bene: \myref{branch_predictors}.

\subsubsection{\NonOptimizing GCC 4.9 (ARM64)}

\myindex{ARM!\Instructions!XOR}

ARM64 ha un'istruzione \INS{NEG} per la negazione:

\lstinputlisting[caption=\Optimizing GCC 4.9 (ARM64),style=customasmARM]{patterns/07_jcc/abs/abs_GCC49_ARM64_O0_IT.s}

\subsubsection{MIPS}

\lstinputlisting[caption=\Optimizing GCC 4.4.5 (IDA),style=customasmMIPS]{patterns/07_jcc/abs/MIPS_O3_IDA_IT.lst}

\myindex{MIPS!\Instructions!BLTZ}
Qui vediamo una nuova istruzione: \INS{BLTZ} (\q{Branch if Less Than Zero}).
\myindex{MIPS!\Instructions!SUBU}
\myindex{MIPS!\Pseudoinstructions!NEGU}

C'è anche la pseudoistruzione \INS{NEGU} , che semplicemente fa la sottrazione da zero.
Il suffisso \q{U} suffix in entrambe \INS{SUBU} e \INS{NEGU} implica che non verrà sollevata nessuna eccezione in caso di integer overflow.

\subsubsection{Branchless version?}

C'è anche una versione senza diramazioni (branchless) di questo codice. Ne riparleremo più avanti: \myref{chap:branchless_abs}.
