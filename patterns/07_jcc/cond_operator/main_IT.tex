\subsection{Operatore ternario}
\label{chap:cond}

L'operatore ternario in \CCpp ha la seguente sintassi:

\begin{lstlisting}
espressione ? espressione : espressione
\end{lstlisting}

Ecco un semplice esempio:

\lstinputlisting[style=customc]{patterns/07_jcc/cond_operator/cond.c}

\subsubsection{x86}

I vecchi compilatori e quelli non ottimizzanti generano codice assembly come se fosse stata usata una coppia \TT{if/else}:

\lstinputlisting[caption=\NonOptimizing MSVC 2008,style=customasmx86]{patterns/07_jcc/cond_operator/MSVC2008_IT.asm}

\lstinputlisting[caption=\Optimizing MSVC 2008,style=customasmx86]{patterns/07_jcc/cond_operator/MSVC2008_Ox_IT.asm}

I nuovi compilatori sono più concisi:

\lstinputlisting[caption=\Optimizing MSVC 2012 x64,style=customasmx86]{patterns/07_jcc/cond_operator/MSVC2012_Ox_x64_IT.asm}

\myindex{x86!\Instructions!CMOVcc}
\Optimizing GCC 4.8 per x86 usa anche l'istruzione \TT{CMOVcc}, mentre la versione non-optimizing usa jump condizionali.

\subsubsection{ARM}

\myindex{x86!\Instructions!ADRcc}
Anche l'ottimizzazione Keil per ARM mode usa le istruzioni condizionali \TT{ADRcc}:

\lstinputlisting[label=cond_Keil_ARM_O3,caption=\OptimizingKeilVI (\ARMMode),style=customasmARM]{patterns/07_jcc/cond_operator/Keil_ARM_O3_IT.s}

Senza alcun intervento manuale, le due istruzioni \TT{ADREQ} e \TT{ADRNE} non possono essere eseguite nella stessa istanza.

\Optimizing Keil per Thumb mode deve usare i jump condizionali, in quanto non esistono istruzioni di caricamento che supportano i flag condizionali:

\lstinputlisting[caption=\OptimizingKeilVI (\ThumbMode),style=customasmARM]{patterns/07_jcc/cond_operator/Keil_thumb_O3_IT.s}

\subsubsection{ARM64}

L' \Optimizing GCC (Linaro) 4.9 per ARM64 usa anch'esso i jump condizionali:

\lstinputlisting[label=cond_ARM64,caption=\Optimizing GCC (Linaro) 4.9,style=customasmARM]{patterns/07_jcc/cond_operator/ARM64_GCC_O3_IT.s}

Ciò avviene perchè ARM64 non ha una semplice istruzione di caricamento con flag condizionali, come \TT{ADRcc} in ARM mode a 32-bit o \INS{CMOVcc} in x86.

\myindex{ARM!\Instructions!CSEL}
Tuttavia ha l'istruzione \q{Conditional SELect} (\TT{CSEL})\InSqBrackets{\ARMSixFourRef p390, C5.5},
ma GCC 4.9 non sembra essere abbastanza intelligente da usarla in un simile pezzo di codice.

\subsubsection{MIPS}

Sfortunatamente, anche GCC 4.4.5 per MIPS non è molto intelligente:

\lstinputlisting[caption=\Optimizing GCC 4.4.5 (\assemblyOutput),style=customasmMIPS]{patterns/07_jcc/cond_operator/MIPS_O3_IT.s}

\subsubsection{Riscriviamolo come un \TT{if/else}}

\lstinputlisting[style=customc]{patterns/07_jcc/cond_operator/cond2.c}

\myindex{x86!\Instructions!CMOVcc}

E' interessante notare che GCC 4.8 ottimizzante per x86 è stato in grado di usare \TT{CMOVcc} in questo caso:

\lstinputlisting[caption=\Optimizing GCC 4.8,style=customasmx86]{patterns/07_jcc/cond_operator/cond2_GCC_O3_IT.s}

Keil ottimizzante in ARM mode genera codice identico a \lstref{cond_Keil_ARM_O3}.

Ma MSVC 2012 ottimizzante non è (ancora) così in gamba.

% Do not translate, this is macro:
\subsubsection{\Conclusion{}}

Perchè i compilatori ottimizzanti cercano di sbarazzarsi dei jump condizionali? Leggi qui: \myref{branch_predictors}.
