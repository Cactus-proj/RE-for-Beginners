\subsection{Ottenere i valori massimo e minimo}

\subsubsection{32-bit}

\lstinputlisting[style=customc]{patterns/07_jcc/minmax/minmax.c}

\lstinputlisting[caption=\NonOptimizing MSVC 2013,style=customasmx86]{patterns/07_jcc/minmax/minmax_MSVC_2013_IT.asm}

\myindex{x86!\Instructions!Jcc}

Queste due funzioni differiscono solo per l'istruzione di salto condizionale: 
\INS{JGE} (\q{Jump if Greater or Equal}) è usata nella prima
e \INS{JLE} (\q{Jump if Less or Equal}) nella seconda.

\myindex{\CompilerAnomaly}
\label{MSVC_double_JMP_anomaly}

In ciascuna funzione c'è un'istruzione \JMP non necessaria, che MSVC ha probabilmente lasciato per sbaglio.

\myparagraph{Branchless}

ARM in modalità Thumb ci ricorda molto codice x86:

\lstinputlisting[caption=\OptimizingKeilVI (\ThumbMode),style=customasmARM]{patterns/07_jcc/minmax/minmax_Keil_Thumb_O3_IT.s}

\myindex{ARM!\Instructions!Bcc}

Le funzioni differiscono per le istruzioni di branching: \INS{BGT} e \INS{BLT}.
Essendo possibile usare suffissi condizionali in modalità ARM, il codice è più conciso.

\myindex{ARM!\Instructions!MOVcc}
\INS{MOVcc} viene eseguita se la condizione è soddisfatta:

\lstinputlisting[caption=\OptimizingKeilVI (\ARMMode),style=customasmARM]{patterns/07_jcc/minmax/minmax_Keil_ARM_O3_IT.s}

\myindex{x86!\Instructions!CMOVcc}
\Optimizing, GCC 4.8.1 e MSVC 2013 possono usare l'istruzione \INS{CMOVcc}, che è analoga a \INS{MOVcc} in ARM:

\lstinputlisting[caption=\Optimizing MSVC 2013,style=customasmx86]{patterns/07_jcc/minmax/minmax_GCC481_O3_IT.s}

\subsubsection{64-bit}

\lstinputlisting[style=customc]{patterns/07_jcc/minmax/minmax64.c}

C'è un po' di spostamento di valori non necessario, ma il codice è comprensibile:

\lstinputlisting[caption=\NonOptimizing GCC 4.9.1 ARM64,style=customasmARM]{patterns/07_jcc/minmax/minmax64_GCC_49_ARM64_O0.s}

\myparagraph{Branchless}

Non è necessario caricare gli argomenti della funzione dallo stack poiché sono già nei registri:

\lstinputlisting[caption=\Optimizing GCC 4.9.1 x64,style=customasmx86]{patterns/07_jcc/minmax/minmax64_GCC_49_x64_O3_IT.s}

MSVC 2013 fa pressoché lo stesso.

\myindex{ARM!\Instructions!CSEL}

ARM64 ha l'istruzione \INS{CSEL}, che funziona esattamente come \INS{MOVcc} in ARM o \INS{CMOVcc} in x86, cambia soltanto il nome:
\q{Conditional SELect}.

\lstinputlisting[caption=\Optimizing GCC 4.9.1 ARM64,style=customasmARM]{patterns/07_jcc/minmax/minmax64_GCC_49_ARM64_O3_IT.s}

\subsubsection{MIPS}

Sfortunatamente GCC 4.4.5 per MIPS non è altrettanto bravo:

\lstinputlisting[caption=\Optimizing GCC 4.4.5 (IDA),style=customasmMIPS]{patterns/07_jcc/minmax/minmax_MIPS_O3_IDA_IT.lst}

Non ci dimentichiamo dei \emph{branch delay slot}: la prima \INS{MOVE} è eseguita \emph{prima} di \INS{BEQZ}, 
la seconda \INS{MOVE} viene eseguita solo se il branch non è stato seguito.

