% FIXME1 divide this file into separate ones...
\mysection{SIMDを使用した浮動小数点数の取り扱い}

\label{floating_SIMD}
\myindex{IEEE 754}
\myindex{SIMD}
\myindex{SSE}
\myindex{SSE2}

もちろん、\ac{SIMD}拡張機能が追加されたとき、\ac{FPU}はx86互換プロセッサに残っていました。

\ac{SIMD}拡張(SSE2)は、浮動小数点数を扱うためのより簡単な方法を提供します。

数値のフォーマットは変わりません(IEEE 754)。

\myindex{x86-64}
そのため、現代のコンパイラ(x86-64用に生成されたものも含む)
は通常、FPUの代わりに\ac{SIMD}命令を使用します。

彼らと一緒に動作する方が簡単なので、それは良いニュースだと言えます。

ここではFPUセクションの例を再利用します:\myref{sec:FPU}

\subsection{単純な例}

\lstinputlisting[style=customc]{patterns/12_FPU/1_simple/simple.c}

\subsubsection{x64}

\lstinputlisting[caption=\Optimizing MSVC 2012 x64,style=customasmx86]{patterns/205_floating_SIMD/simple_MSVC_2012_x64_Ox.asm}

入力浮動小数点値は\XMM{0}-\XMM{3}レジスタに渡され、
残りはすべてスタックを介して渡されます。
\footnote{\href{http://msdn.microsoft.com/en-us/library/zthk2dkh.aspx}{MSDN: Parameter Passing}}.

$a$ は\XMM{0}に渡され、$b$は\XMM{1}を介して渡されます。

XMMレジスタは128ビットです(SIMDに関するセクションからわかるように:\myref{SIMD_x86})。
しかし \Tdouble 値は64ビットですので、下位半分のレジスタだけが使用されます。

\myindex{x86!\Instructions!DIVSD}
\TT{DIVSD}は\q{Divide Scalar Double-Precision Floating-Point Values}
を表すSSE命令です。
これは、
オペランドの下半分に格納された \Tdouble 型の値を別の値で除算するだけです。

定数は、コンパイラによってIEEE 754形式でエンコードされています。

\myindex{x86!\Instructions!MULSD}
\myindex{x86!\Instructions!ADDSD}
\TT{MULSD}と\TT{ADDSD}はまったく同じように機能しますが、乗算と加算を行います。

関数が \Tdouble 型で実行された結果は、\XMM{0}レジスタに残ります。\\
\\
これが、最適化されていないMSVCの仕組みです。

\lstinputlisting[caption=MSVC 2012 x64,style=customasmx86]{patterns/205_floating_SIMD/simple_MSVC_2012_x64.asm}

\myindex{Shadow space}
少し冗長です。 
入力引数は\q{シャドースペース}(\myref{shadow_space})に保存されますが、
それらの下位レジスタのみが半分になります。つまり、 \Tdouble 型の64ビット値だけです。 
GCCは同じコードを生成します。

\subsubsection{x86}

この例もx86用にコンパイルしましょう。 x86用に生成されているという事実にもかかわらず、MSVC 2012はSSE2命令を使用します。

\lstinputlisting[caption=\NonOptimizing MSVC 2012 x86,style=customasmx86]{patterns/205_floating_SIMD/simple_MSVC_2012_x86.asm}

\lstinputlisting[caption=\Optimizing MSVC 2012 x86,style=customasmx86]{patterns/205_floating_SIMD/simple_MSVC_2012_x86_Ox.asm}

これはほぼ同じコードですが、呼び出し規約に関していくつかの違いがあります。
1)引数はXMMレジスタではなくスタックに渡されます(FPUの例(\myref{sec:FPU})のように)。 
2)関数の結果が\ST{0}に返されます。 そのためには、
XMMレジスタの1つから\ST{0}に(ローカル変数\TT{tv}を介して)コピーします。

\clearpage
最適化された例を \olly で試してみましょう

\begin{figure}[H]
\centering
\myincludegraphics{patterns/205_floating_SIMD/simple_olly1.png}
\caption{\olly: \TT{MOVSD} は $a$ の値を\XMM{1}にロード}
\label{fig:FPU_SIMD_simple_olly1}
\end{figure}

\clearpage
\begin{figure}[H]
\centering
\myincludegraphics{patterns/205_floating_SIMD/simple_olly2.png}
\caption{\olly: \TT{DIVSD} \gls{quotient} を計算し、
結果を\XMM{1}に保存する}
\label{fig:FPU_SIMD_simple_olly2}
\end{figure}

\clearpage
\begin{figure}[H]
\centering
\myincludegraphics{patterns/205_floating_SIMD/simple_olly3.png}
\caption{\olly: \TT{MULSD} calculated \gls{product}を計算し、
\XMM{0}に保存する}
\label{fig:FPU_SIMD_simple_olly3}
\end{figure}

\clearpage
\begin{figure}[H]
\centering
\myincludegraphics{patterns/205_floating_SIMD/simple_olly4.png}
\caption{\olly: \TT{ADDSD} は値を\XMM{0}と\XMM{1}に追加する}
\label{fig:FPU_SIMD_simple_olly4}
\end{figure}

\clearpage
\begin{figure}[H]
\centering
\myincludegraphics{patterns/205_floating_SIMD/simple_olly5.png}
\caption{\olly: \FLD は関数の結果を\ST{0}に残す}
\label{fig:FPU_SIMD_simple_olly5}
\end{figure}

\olly はXMMレジスタを \Tdouble 型の数のペアとして示していますが、
使用されているのはその低位の部分だけです。

SSE2命令(接尾辞\TT{-SD})が現在実行されているため、明らかに、
\olly はそれらをその形式で表示します。

しかし、もちろん、レジスタフォーマットを切り替えて、その内容を4つの \Tfloat{} 数
またはちょうど16バイトとして表示することは可能です。

\clearpage
\subsection{引数を介して浮動小数点数を渡す}

\lstinputlisting[style=customc]{patterns/12_FPU/2_passing_floats/pow.c}

それらは\XMM{0}-\XMM{3}レジスタの
下位半分に渡されます。

\lstinputlisting[caption=\Optimizing MSVC 2012 x64,style=customasmx86]{patterns/205_floating_SIMD/pow_MSVC_2012_x64_Ox.asm}

\myindex{x86!\Instructions!MOVSD}
\myindex{x86!\Instructions!MOVSDX}
IntelおよびAMDのマニュアルには\TT{MOVSDX}命令がなく(\myref{x86_manuals})、単に\TT{MOVSD}と呼ばれています。 
そのため、x86では同じ名前を共有する2つの命令があります(他のものについては \myref{REP_MOVSx} を参照)。
どうやら、Microsoftの開発者たちはこの混乱を取り除きたかったので、\TT{MOVSDX}に改名しました。 
XMMレジスタの下半分に値をロードするだけです。

\TT{pow()}は\XMM{0}と\XMM{1}から引数を取り、結果を\XMM{0}に返します。 
その後 \printf のために \RDX に移動されます。 
なぜでしょうか?
おそらく \
printf が可変引数関数だからでしょう。

\lstinputlisting[caption=\Optimizing GCC 4.4.6 x64,style=customasmx86]{patterns/205_floating_SIMD/pow_GCC446_x64_O3_JA.s}

GCCはより明確な出力を生成します。 
\printf の値は\XMM{0}に渡されます。 
ところで、 \printf のために \EAX に1が書かれている場合は、
標準で要求されているように \SysVABI 、1つの引数がベクトルレジスタに渡されることを意味します。

\subsection{Comparison example}

\lstinputlisting[style=customc]{patterns/12_FPU/3_comparison/d_max.c}

\subsubsection{x64}

\lstinputlisting[caption=\Optimizing MSVC 2012 x64,style=customasmx86]{patterns/205_floating_SIMD/d_max_MSVC_2012_x64_Ox.asm}

\Optimizing MSVCはとても理解しやすいコードを生成します。

\myindex{x86!\Instructions!COMISD}
\TT{COMISD} は \q{Compare Scalar Ordered Double-Precision Floating-Point 
Values and Set EFLAGS}です。 本質的に、命令が示すそのもののことをします。\\
\\
\NonOptimizing MSVC はもっと冗長なコードを生成します。
しかし、これもそんなに理解するのが難しくないです。

\lstinputlisting[caption=MSVC 2012 x64,style=customasmx86]{patterns/205_floating_SIMD/d_max_MSVC_2012_x64.asm}

\myindex{x86!\Instructions!MAXSD}
しかしながら、GCC 4.4.6は
もっと最適化して\TT{MAXSD}(\q{Return Maximum Scalar 
Double-Precision Floating-Point Value})命令を使用します。
これは単に最大値を選択します!

\lstinputlisting[caption=\Optimizing GCC 4.4.6 x64,style=customasmx86]{patterns/205_floating_SIMD/d_max_GCC446_x64_O3.s}

\clearpage
\subsubsection{x86}

この例をMSVC 2012の最適化オプションをONにしてコンパイルしてみましょう。

\lstinputlisting[caption=\Optimizing MSVC 2012 x86,style=customasmx86]{patterns/205_floating_SIMD/d_max_MSVC_2012_x86_Ox.asm}

ほとんど同じですが、 $a$ と $b$ の値はスタックからとられて、関数の結果は
\ST{0}に残ります。

\olly でこの例をロードした場合、
\TT{COMISD}命令が値を比較して \CF と \PF フラグをどうやってセット/クリアするのかみることができます。

\begin{figure}[H]
\centering
\myincludegraphics{patterns/205_floating_SIMD/d_max_olly.png}
\caption{\olly: \TT{COMISD} は \CF と \PF フラグを変更}
\label{fig:FPU_SIMD_d_max_olly}
\end{figure}

\subsection{計算機イプシロンを計算する: x64 と SIMD}
\label{machine_epsilon_x64_and_SIMD}

\q{計算機イプシロンを計算する}の例を \Tdouble 型で再訪しましょう:\lstref{machine_epsilon_double_c}

x64でコンパイルします。

\lstinputlisting[caption=\Optimizing MSVC 2012 x64,style=customasmx86]{patterns/205_floating_SIMD/epsilon_double_MSVC_2012_x64_Ox.asm}

128ビットXMMレジスタに値を1加える方法がないので、メモリ上に配置しなければなりません。

しかしながら、\INS{ADDSD}命令(\IT{Add Scalar Double-Precision Floating-Point Values})があります。
高位64ビットを無視しつつ、これは低位64ビットのXMMレジスタに値を加えることができます。
しかし、MSVC 2012はおそらくそれほど良くないです。
\footnote{練習として、ローカルスタックの使用を取り除く
ためにこのコードを書き直してもいいかもしれません。}

それでも、値はXMMレジスタに再ロードされ、減算が行われます。
\INS{SUBSD}は\q{Subtract Scalar Double-Precision Floating-Point Values}です。
つまり、128ビットXMMレジスタの下位64ビット部に対して動作します。
結果はXMM0レジスタに返されます。

\subsection{文字列へのポインタの配列}
\label{array_of_pointers_to_strings}

ここでは、ポインタの配列の例を示します。

\lstinputlisting[caption=Get month name,label=get_month1,style=customc]{patterns/13_arrays/45_month_1D/month1_JA.c}

\subsubsection{x64}

\lstinputlisting[caption=\Optimizing MSVC 2013 x64,style=customasmx86]{patterns/13_arrays/45_month_1D/month1_MSVC_2013_x64_Ox.asm}

コードはとても単純です。

\begin{itemize}

\item
\myindex{x86!\Instructions!MOVSXD}

最初の\INS{MOVSXD}命令は、 \ECX ( $month$ 引数が渡される)から32ビットの値を
符号拡張付きの \RAX ( $month$ 引数は \Tint 型なので)にコピーします。

符号拡張の理由は、この32ビット値が他の64ビット値との計算に使用されるためです。

したがって、64ビット142に昇格させる必要があります。%
\footnote{やや奇妙ですが、負の配列インデックスはここで $month$ として渡すことができます
(負の配列インデックスは後で説明します:\ref{negative_array_indices})。 % FIXME should be \myref{} here, but varioref package complains...
これが起こると、 \Tint 型の負の入力値が正しく符号拡張され、
テーブルの前の対応する要素が選択されます。 符号拡張なしでは正しく動作しません。}

\item
次にポインタテーブルのアドレスが \RCX にロードされます。

\item
最後に、入力値($month$)に8を掛けてアドレスに加算します。 
確かに:私たちは64ビット環境にあり、すべてのアドレス(またはポインタ)は正確に64ビット(または8バイト)の
記憶域を必要とします。 
したがって、各テーブル要素は8バイト幅です。 
それで、なぜ特定の要素 $month*8$ をスキップする必要があるのでしょうか。
これが \MOV が行うことです。 
さらに、この命令はこのアドレスの要素もロードします。 
1の場合、要素は\q{February}などを含む文字列へのポインタになります。

\end{itemize}

\Optimizing GCC 4.9はもっとよく仕事をこなします。
\footnote{GCCアセンブラ出力が排除するのに十分なほど整っていないので、\q{0+}がリストに残っていました。 
それは\emph{変位}であり、ここではゼロです。}

\begin{lstlisting}[caption=\Optimizing GCC 4.9 x64,style=customasmx86]
	movsx	rdi, edi
	mov	rax, QWORD PTR month1[0+rdi*8]
	ret
\end{lstlisting}

\myparagraph{32ビットMSVC}

32ビットMSVCコンパイラでもコンパイルしてみましょう。

\lstinputlisting[caption=\Optimizing MSVC 2013 x86,style=customasmx86]{patterns/13_arrays/45_month_1D/month1_MSVC_2013_x86_Ox.asm}

入力値は64ビットに拡張する必要がないので、そのまま使われます。

そして4倍されます。テーブル要素が32ビット(または4バイト)幅だからです。

% FIXME1 move to another file
\subsubsection{32ビット ARM}

\myparagraph{ARMモードでのARM}

\lstinputlisting[caption=\OptimizingKeilVI (\ARMMode),style=customasmARM]{patterns/13_arrays/45_month_1D/month1_Keil_ARM_O3.s}

% TODO Fix R1s

テーブルのアドレスはR1にロードされます。
\myindex{ARM!\Instructions!LDR}

残りのすべては \LDR 命令1つだけを使って行われます。

入力値 $month$ は2ビット左シフトします(4倍するのと同じです)。それから
R1に加えらえます(テーブルのアドレスの場所)。そしてテーブル要素はこのアドレスからロードされます。

32ビットテーブル要素はテーブルからR0にロードされます。

\myparagraph{ThumbモードでのARM}

コードはほとんど同じですが、より密度が低いです。 \LSL サフィックスは \LDR 命令では特定できないからです。

\begin{lstlisting}[style=customasmARM]
get_month1 PROC
        LSLS     r0,r0,#2
        LDR      r1,|L0.64|
        LDR      r0,[r1,r0]
        BX       lr
        ENDP
\end{lstlisting}

\subsubsection{ARM64}

\lstinputlisting[caption=\Optimizing GCC 4.9 ARM64,style=customasmARM]{patterns/13_arrays/45_month_1D/month1_GCC49_ARM64_O3.s}

\myindex{ARM!\Instructions!ADRP/ADD pair}

テーブルのアドレスは \ADRP/\ADD 命令の組を使ってX1にロードされます。

それから付随する要素 \LDR を使って選ばれて、W0を取ります(入力引数 $month$ の場所のレジスタ)。
左に3ビットシフトします(8倍するのと同じです)。
符号拡張し(\q{sxtw}サフィックスが暗示しています)、X0に加算します。
それから64ビット値がテーブルからX0にロードされます。

\subsubsection{MIPS}

\lstinputlisting[caption=\Optimizing GCC 4.4.5 (IDA),style=customasmMIPS]{patterns/13_arrays/45_month_1D/MIPS_O3_IDA_JA.lst}

\subsubsection{配列オーバーフロー}

関数は0~11の範囲の値を受け付けますが、12は通すでしょうか?
テーブルにはその場所の要素はありません。

なので関数はそこにたまたまある値をロードしてリターンします。

すぐ後で、他の関数がこのアドレスからテキスト文字列を取得しようとしてクラッシュするかもしれません。

例をwin64用としてMSVCでコンパイルして、テーブルの後にリンカーが何を配置したのかを \IDA で見てみましょう。

\lstinputlisting[caption=IDAでの実行可能ファイル,style=customasmx86]{patterns/13_arrays/45_month_1D/MSVC2012_win64_1.lst}

月の名前がそのあとに来ています。

プログラムは小さいので、データセグメントにパックされるデータは多くありません。
だから単に次の名前が来ています。
しかし注意すべきはリンカーが配置するように決定するのは\emph{どんなものも}ありえます。

だからもし12が関数に渡されたら?
13番目の要素がリターンされます。

CPUがそこにあるバイトを64ビットの値としてどのように扱うかをみてみましょう。

\lstinputlisting[caption=IDAでの実行可能ファイル,style=customasmx86]{patterns/13_arrays/45_month_1D/MSVC2012_win64_2.lst}

0x797261756E614Aです。

すぐ後で、他の関数(おそらく文字列を扱う関数)がこのアドレスでバイトを読み込もうとすると、
C言語の文字列を期待します。

十中八九、クラッシュします。この値は有効なアドレスのようには見えないからです。

\myparagraph{配列オーバーフロー保護}

\epigraph{失敗する可能性のあるものは、失敗する。}{マーフィーの法則}

あなたの関数を使用するプログラマはみな11より大きな値を引数として渡さないと
期待するのはちょっとナイーブです。

問題をできるだけ早く報告し停止することを意味する\q{fail early and fail loudly}
または\q{早く失敗する}という哲学があります。

\myindex{\CStandardLibrary!assert()}

そのような方法の1つに \CCpp のassertionがあります。

不正な値が通ってきたら、失敗するようにプログラムを変更できます。

\lstinputlisting[caption=assert()を追加,style=customc]{patterns/13_arrays/45_month_1D/month1_assert.c}

アサーションマクロは関数の開始時に妥当な値かチェックし、式が偽の場合に失敗します。

\lstinputlisting[caption=\Optimizing MSVC 2013 x64,style=customasmx86]{patterns/13_arrays/45_month_1D/MSVC2013_x64_Ox_checked.asm}

実際、assert() は関数ではなくマクロです。条件をチェックし、
行数とファイル名を他の関数に渡してユーザに情報を報告します。

ファイル名と条件の両方がUTF-16でエンコードされています。
行数も渡されます(29です)。

このメカニズムはおそらくすべてのコンパイラで同じです。
GCCはこのようにします。

\lstinputlisting[caption=\Optimizing GCC 4.9 x64,style=customasmx86]{patterns/13_arrays/45_month_1D/GCC491_x64_O3_checked.s}

GCCのマクロは利便性のために関数名も渡します。

何事もただではできませんが、サニタイズチェックもこれと同様です。

それはプログラムを遅くしますが、特にassert()マクロが小さなタイムクリティカルな関数で使用されると遅くなります。

なのでMSVCでは、例えばデバッグビルドではチェックを残し、リリースビルドでは取り除いたりします。
 
マイクロソフト\gls{Windows NT}カーネルは\q{チェックされた}と\q{フリー}ビルドです。
\footnote{\href{http://msdn.microsoft.com/en-us/library/windows/hardware/ff543450(v=vs.85).aspx}{msdn.microsoft.com/en-us/library/windows/hardware/ff543450(v=vs.85).aspx}}.

最初のものは妥当性チェック(\q{チェックされた}なので)があり、もう一つはチェックしていません(チェックが\q{フリー}なので)。

もちろん、 \q{チェックされた}カーネルはこれらのチェックのために遅く動作するので、通常はデバッグセッションでのみ使用されます。

% FIXME: ARM? MIPS?

\subsubsection{特定の文字へのアクセス}

文字列へのポインタの配列はこのようにアクセスできます。

\lstinputlisting[style=customc]{patterns/13_arrays/45_month_1D/month2_JA.c}

\dots \emph{month[3]}式は\emph{const char*}型をもつので、
5番目の文字列はこのアドレスに4バイトを足した式から取得します。

さて、\emph{main()}関数に渡された引数リストは同じデータ型を持ちます。

\lstinputlisting[style=customc]{patterns/13_arrays/45_month_1D/argv_JA.c}

似た構文ですが、2次元配列とは異なることを理解することが非常に重要です。
これについては後で検討します。

もう1つの重要なことに注意してください。アドレス指定される文字列は、各文字が\ac{ASCII}や拡張\ac{ASCII}のように1バイトを占めるシステムで
エンコードされなければなりません。 
UTF-8はここでは動作しません。


\subsection{概要}

ここに示すすべての例では、数値をIEEE 754形式で格納するために、
XMMレジスタの下半分だけが使用されています。

本質的に、\TT{-SD}(\q{Scalar Double-Precision})
の接頭語がついた命令はすべて、IEEE 754形式の浮動小数点数として動作します。
そして、XMMレジスタの下位64ビット半分に格納されます。

そしてそれは、おそらくSIMD拡張が過去のFPU拡張よりも混沌としていない
方法で進化したために、FPUよりも簡単です。
スタックレジスタモデルは使用されません。

\myindex{x86!\Instructions!ADDSS}
\myindex{x86!\Instructions!MOVSS}
\myindex{x86!\Instructions!COMISS}
% TODO1: do this!
\Tdouble を \Tfloat に置き換えようとした場合

% FIXME1 ... but their -SS versions
これらの例では、同じ命令が使用されますが、\TT{-SS}(\q{Scalar Single-Precision})接頭語が
つきます。例えば、\TT{MOVSS}、\TT{COMISS}、\TT{ADDSS}などです。

\q{Scalar} 
は、SIMDレジスタに複数の値ではなく1つの値しか含まれていないことを意味します。

レジスタ内の複数の値を同時に扱う命令は、それらの名前に\q{パック}されています。

言うまでもなく、SSE2命令は64ビットのIEEE 754形式の数( \Tdouble )で機能しますが、
FPUの浮動小数点数の内部表現は80ビットの数値です。

したがって、FPUは丸め誤差を少なくすることができ、その結果、FPUはより正確な計算結果を
得られます。
