\mysection{The method}

When the author of this book first started learning C and, later, \Cpp, he used to write small pieces of code, compile them,
and then look at the assembly language output. This made it very easy for him to understand what was going on in the code that he had written.
\footnote{In fact, he still does this when he can't understand what a particular bit of code does.
Recent example from 2019: \verb|p += p+(i&1)+2;| from the ``SAT0W'' SAT-solver by D.Knuth.}.
He did this so many times that the relationship between the \CCpp code and what the compiler produced was imprinted deeply in his mind.
It's now easy for him to imagine instantly a rough outline of a C code's appearance and function.
Perhaps this technique could be helpful for others.

%There are a lot of examples for both x86/x64 and ARM.
%Those who already familiar with one of architectures, may freely skim over pages.

By the way, there is a great website where you can do the same, with various compilers, instead of installing them on your box.
You can use it as well: \url{https://godbolt.org/}.

\section*{\Exercises}

When the author of this book studied assembly language, he also often compiled small C functions and then rewrote
them gradually to assembly, trying to make their code as short as possible.
This probably is not worth doing in real-world scenarios today,
because it's hard to compete with the latest compilers in terms of efficiency. It is, however, a very good way to gain a better understanding of assembly.
Feel free, therefore, to take any assembly code from this book and try to make it shorter.
However, don't forget to test what you have written.

% rewrote to show that debug\release and optimisations levels are orthogonal concepts.
\section*{Optimization levels and debug information}

Source code can be compiled by different compilers with various optimization levels.
A typical compiler has about three such levels, where level zero means that optimization is completely disabled.
Optimization can also be targeted towards code size or code speed.
A non-optimizing compiler is faster and produces more understandable (albeit verbose) code,
whereas an optimizing compiler is slower and tries to produce code that runs faster (but is not necessarily more compact).
In addition to optimization levels, a compiler can include some debug information in the resulting file,
producing code that is easy to debug.
One of the important features of the ´debug' code is that it might contain links
between each line of the source code and its respective machine code address.
Optimizing compilers, on the other hand, tend to produce output where entire lines of source code
can be optimized away and thus not even be present in the resulting machine code.
Reverse engineers can encounter either version, simply because some developers turn on the compiler's optimization flags and others do not.
Because of this, we'll try to work on examples of both debug and release versions of the code featured in this book, wherever possible.

Sometimes some pretty ancient compilers are used in this book, in order to get the shortest (or simplest) possible code snippet.
