\subsection{多次元配列}

内部的には、多次元配列は本質的には一次元の配列と同じです。

コンピュータメモリは一次元なので、メモリは一次元配列です。
便宜上、多次元配列は一次元として表現可能です。

例えば、3x4の配列の要素が12のセルの1次元配列にどのように配置されるかを示します。

% TODO FIXME not clear. First, horizontal would be better. Second, why two columns?
% I'd first show 3x4 with numbered elements (e.g. 32-bit ints) in colored lines,
% then linear with the same numbered elements (and colored blocks)
% then linear with addresses (offsets) - assuming let say 32-bit ints.
\begin{table}[H]
\centering
\begin{tabular}{ | l | l | }
\hline
Offset in memory & array element \\
\hline
0 & [0][0] \\
\hline
1 & [0][1] \\
\hline
2 & [0][2] \\
\hline
3 & [0][3] \\
\hline
4 & [1][0] \\
\hline
5 & [1][1] \\
\hline
6 & [1][2] \\
\hline
7 & [1][3] \\
\hline
8 & [2][0] \\
\hline
9 & [2][1] \\
\hline
10 & [2][2] \\
\hline
11 & [2][3] \\
\hline
\end{tabular}
\caption{1次元配列としてメモリ上で表現される2次元配列}
\end{table}

3*4配列の各セルがメモリ上でどう配置されるかを示します。

% TODO coordinates. TikZ?
\begin{table}[H]
\centering
\begin{tabular}{ | l | l | l | l | }
\hline                        
0 & 1 & 2 & 3 \\
\hline  
4 & 5 & 6 & 7 \\
\hline  
8 & 9 & 10 & 11 \\
\hline  
\end{tabular}
\caption{2次元配列の各セルのメモリアドレス}
\end{table}

\myindex{row-major order}

したがって、必要な要素のアドレスを計算するには、まず最初のインデックスに
4(配列の幅)を掛けてから2番目のインデックスを追加します。
これは\emph{行優先順位}と呼ばれ、配列と行列表現のこの方法は、少なくとも \CCpp とPythonで使用されます。
単純な英単語の\emph{行優先順位}は、\q{最初に、最初の行の要素を書き、次に2番目の行 \dots 
最後に最後の行の要素を書き込む}という意味です。

\myindex{column-major order}
\myindex{Fortran}
表現のもう1つの方法は、\emph{列優先順位}(配列の添字は逆順で使用されます)と呼ばれ、
少なくともFortran、MATLAB、およびRで使用されます。
\emph{列優先順位}は、単純な英語では、\q{最初に、最初の列の要素を書き込み、次に2番目の列を \dots
最後に最後の列の要素を書き込む}となります。

どの方法が良いでしょうか?

一般に、パフォーマンスとキャッシュメモリの観点からは、
データ編成のための最良の方法は、要素が順次アクセスされる方法です。

したがって、関数が行ごとにデータにアクセスする場合は、\emph{行優先順位}が優れていて、逆もまた同様です。

% subsubsections
\input{patterns/13_arrays/5_multidimensional/2D_JA}
\input{patterns/13_arrays/5_multidimensional/2D_as_1D_JA}
\input{patterns/13_arrays/5_multidimensional/3D_JA}
%\input{patterns/13_arrays/5_multidimensional/dimensions_JA}

\subsubsection{More examples}

コンピュータ画面は2D配列として表現されますが、ビデオバッファは1次元配列です。
これについてはこちらで:\myref{Mandelbrot_demo}

本書での他の例としてはマインスイーパーゲームがあります。そのフィールドは2次元配列です:\myref{minesweeper_winxp}
