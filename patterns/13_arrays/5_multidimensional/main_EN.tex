\subsection{Multidimensional arrays}

Internally, a multidimensional array is essentially the same thing as a linear array.

Since the computer memory is linear, it is an one-dimensional array.
For convenience, this multi-dimensional array can be easily represented as one-dimensional.

For example, this is how the elements of the 3x4 array are placed in one-dimensional array of 12 cells:

% TODO FIXME not clear. First, horizontal would be better. Second, why two columns?
% I'd first show 3x4 with numbered elements (e.g. 32-bit ints) in colored lines,
% then linear with the same numbered elements (and colored blocks)
% then linear with addresses (offsets) - assuming let say 32-bit ints.
\begin{table}[H]
\centering
\begin{tabular}{ | l | l | }
\hline
Offset in memory & array element \\
\hline
0 & [0][0] \\
\hline
1 & [0][1] \\
\hline
2 & [0][2] \\
\hline
3 & [0][3] \\
\hline
4 & [1][0] \\
\hline
5 & [1][1] \\
\hline
6 & [1][2] \\
\hline
7 & [1][3] \\
\hline
8 & [2][0] \\
\hline
9 & [2][1] \\
\hline
10 & [2][2] \\
\hline
11 & [2][3] \\
\hline
\end{tabular}
\caption{Two-dimensional array represented in memory as one-dimensional}
\end{table}

Here is how each cell of 3*4 array are placed in memory:

% TODO coordinates. TikZ?
\begin{table}[H]
\centering
\begin{tabular}{ | l | l | l | l | }
\hline                        
0 & 1 & 2 & 3 \\
\hline  
4 & 5 & 6 & 7 \\
\hline  
8 & 9 & 10 & 11 \\
\hline  
\end{tabular}
\caption{Memory addresses of each cell of two-dimensional array}
\end{table}

\myindex{row-major order}

So, in order to calculate the address of the element we need, we first multiply the first index by
4 (array width) and then add the second index.
That's called \emph{row-major order}, 
and this method of array and matrix representation is used in at least \CCpp and Python. 
The term \emph{row-major order} 
in plain English language means: \q{first, write the elements of the first row, then the second row \dots 
and finally the elements of the last row}.

\myindex{column-major order}
\myindex{Fortran}
Another method for representation is called \emph{column-major order} (the array indices are used in reverse order) 
and it is used at least in Fortran, MATLAB and R. 
\emph{column-major order} term in plain English language means: \q{first, write the elements of the first column, then the second column \dots
and finally the elements of the last column}.

Which method is better?

In general, in terms of performance and cache memory, 
the best scheme for data organization is the one,
in which the elements are accessed sequentially.

So if your function accesses data per row, \emph{row-major order} is better, and vice versa.

% subsubsections
\input{patterns/13_arrays/5_multidimensional/2D_EN}
\input{patterns/13_arrays/5_multidimensional/2D_as_1D_EN}
\input{patterns/13_arrays/5_multidimensional/3D_EN}
\input{patterns/13_arrays/5_multidimensional/dimensions_EN}

\subsubsection{More examples}

The computer screen is represented as a 2D array, but the video-buffer is a linear 1D array. 
We talk about it here: \myref{Mandelbrot_demo}.

Another example in this book is Minesweeper game: it's field is also two-dimensional array: \myref{minesweeper_winxp}.

