\subsection{Tableaux multidimensionnels}

En interne, un tableau multidimensionnel est pratiquement la même chose qu'un tableau
linéaire.

Puisque la mémoire d'un ordinateur est linéaire, c'est un tableau uni-dimensionnel.
Par commodité, ce tableau multidimensionnel peut facilement être représenté comme
un uni-dimensionnel.

Par exemple, voici comment les éléments du tableau 3*4 sont placés dans un tableau
uni-dimensionnel de 12 éléments:

% TODO FIXME not clear. First, horizontal would be better. Second, why two columns?
% I'd first show 3x4 with numbered elements (e.g. 32-bit ints) in colored lines,
% then linear with the same numbered elements (and colored blocks)
% then linear with addresses (offsets) - assuming let say 32-bit ints.
\begin{table}[H]
\centering
\begin{tabular}{ | l | l | }
\hline
Offset en mémoire & élément du tableau \\
\hline
0 & [0][0] \\
\hline
1 & [0][1] \\
\hline
2 & [0][2] \\
\hline
3 & [0][3] \\
\hline
4 & [1][0] \\
\hline
5 & [1][1] \\
\hline
6 & [1][2] \\
\hline
7 & [1][3] \\
\hline
8 & [2][0] \\
\hline
9 & [2][1] \\
\hline
10 & [2][2] \\
\hline
11 & [2][3] \\
\hline
\end{tabular}
\caption{Tableau en deux dimensions représenté en mémoire en une dimension}
\end{table}

Voici comment chacun des éléments du tableau 3*4 sont placés en mémoire:

% TODO coordinates. TikZ?
\begin{table}[H]
\centering
\begin{tabular}{ | l | l | l | l | }
\hline                        
0 & 1 & 2 & 3 \\
\hline  
4 & 5 & 6 & 7 \\
\hline  
8 & 9 & 10 & 11 \\
\hline  
\end{tabular}
\caption{Adresse mémoire de chaque élément d'un tableau à deux dimensions}
\end{table}

\myindex{row-major order}

Donc, afin de calculer l'adresse de l'élément voulu, nous devons d'abord multiplier
le premier index par 4 (largeur du tableau) et puis ajouter le second index.
Ceci est appelé \emph{row-major order} (ordre ligne d'abord),
et c'est la méthode de représentation des tableaux et des matrices au moins en \CCpp
et Python.
Le terme \emph{row-major order} est de l'anglais signifiant: \q{ d'abord, écrire les
éléments de la première ligne, puis ceux de la seconde ligne \dots et enfin les éléments
de la dernière ligne}.

\myindex{column-major order}
\myindex{Fortran}
Une autre méthode de représentation est appelée \emph{column-major order} (ordre colonne
d'abord) (les indices du tableau sont utilisés dans l'ordre inverse) et est utilisé
au moins en ForTran, MATLAB et R.
Le terme \emph{column-major order} est de l'anglais signifiant: \q{ d'abord, écrire les
éléments de la première colonne, puis ceux de la seconde colonne \dots et enfin les
éléments de la dernière colonne}.

Quelle méthode est la meilleure?

En général, en termes de performance et de mémoire cache, le meilleur schéma pour
l'organisation des données est celui dans lequel les éléments sont accédés séquentiellement.

Donc si votre fonction accède les données par ligne, \emph{row-major order} est meilleur,
et vice-versa.

% subsubsections
\input{patterns/13_arrays/5_multidimensional/2D_FR}
\input{patterns/13_arrays/5_multidimensional/2D_as_1D_FR}
\input{patterns/13_arrays/5_multidimensional/3D_FR}
%\subsubsection{Obtenir la dimension d'un tableau multidimensionnel}
\myindex{Hex-Rays}

Toute fonction de traitement de chaîne, à laquelle un tableau de caractère lui est passée,
ne peut pas en déduire la taille de ce tableau en entrée.

Par exemple:

\lstinputlisting[style=customc]{patterns/13_arrays/5_multidimensional/dimensions2.c}

\dots si compilé (par n'importe quel compilateur) et ensuite décompilé par Hex-Rays:

\lstinputlisting[style=customc]{patterns/13_arrays/5_multidimensional/dimensions2_hexrays.c}

Il n'y a pas moyen de trouver la taille de la première dimension.
Si la valeur $x$ passée est trop grosse, un dépassement de tampon peut se produire,
un élément d'un endroit aléatoire en mémoire sera lu.

Et un tableau 3D:

\lstinputlisting[style=customc]{patterns/13_arrays/5_multidimensional/dimensions3.c}

Hex-Rays:

\lstinputlisting[style=customc]{patterns/13_arrays/5_multidimensional/dimensions3_hexrays.c}

À nouveau, seules deux des 3 dimensions peuvent être déduites.


\subsubsection{Plus d'exemples}

L'écran de l'ordinateur est représenté comme un tableau 2D, mais le buffer vidéo
est un tableau linéaire 1D.
Nous en parlons ici: \myref{Mandelbrot_demo}.

Un autre exemple dans ce livre est le jeu Minesweeper: son champ est aussi un tableau
à deux dimensions: \myref{minesweeper_winxp}.

