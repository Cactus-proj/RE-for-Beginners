\subsection{Многомерные массивы}

Внутри многомерный массив выглядит так же как и линейный.

Ведь память компьютера линейная, это одномерный массив.
Но для удобства этот одномерный массив легко представить как многомерный.

К примеру, вот как элементы массива 3x4 расположены в одномерном массиве из 12 ячеек:

% TODO FIXME not clear. First, horizontal would be better. Second, why two columns?
% I'd first show 3x4 with numbered elements (e.g. 32-bit ints) in colored lines,
% then linear with the same numbered elements (and colored blocks)
% then linear with addresses (offsets) - assuming let say 32-bit ints.
\begin{table}[H]
\centering
\begin{tabular}{ | l | l | }
\hline
Смещение в памяти & элемент массива \\
\hline
0 & [0][0] \\
\hline
1 & [0][1] \\
\hline
2 & [0][2] \\
\hline
3 & [0][3] \\
\hline
4 & [1][0] \\
\hline
5 & [1][1] \\
\hline
6 & [1][2] \\
\hline
7 & [1][3] \\
\hline
8 & [2][0] \\
\hline
9 & [2][1] \\
\hline
10 & [2][2] \\
\hline
11 & [2][3] \\
\hline
\end{tabular}
\caption{Двухмерный массив представляется в памяти как одномерный}
\end{table}

Вот по каким адресам в памяти располагается каждая ячейка двухмерного массива 3*4:

\begin{table}[H]
\centering
\begin{tabular}{ | l | l | l | l | }
\hline                        
0 & 1 & 2 & 3 \\
\hline  
4 & 5 & 6 & 7 \\
\hline  
8 & 9 & 10 & 11 \\
\hline  
\end{tabular}
\caption{Адреса в памяти каждой ячейки двухмерного массива}
\end{table}

\myindex{row-major order}
Чтобы вычислить адрес нужного элемента, сначала умножаем первый индекс (строку) на 4 (ширину массива), 
затем прибавляем второй индекс (столбец).

Это называется \emph{row-major order}, 
и такой способ представления массивов и матриц используется по крайней мере в \CCpp и Python. 
Термин \emph{row-major order} означает по-русски примерно следующее: \q{сначала записываем элементы первой строки, затем второй,~\dots~и~элементы последней 
строки в самом конце}.

\myindex{column-major order}
\myindex{Фортран}
Другой способ представления называется \emph{column-major order} (индексы массива используются в обратном порядке) 
и это используется по крайней мере в Фортране, MATLAB и R. 
Термин \emph{column-major order} означает по-русски
следующее: \q{сначала записываем элементы первого столбца, затем второго,~\dots~и~элементы последнего столбца
в самом конце}.

Какой из способов лучше?
В терминах производительности и кэш-памяти, лучший метод организации данных это тот,
при котором к данным обращаются последовательно.

Так что если ваша функция обращается к данным построчно, то \emph{row-major order} лучше,
и наоборот.

% subsubsections
\input{patterns/13_arrays/5_multidimensional/2D_RU}
\input{patterns/13_arrays/5_multidimensional/2D_as_1D_RU}
\input{patterns/13_arrays/5_multidimensional/3D_RU}
\input{patterns/13_arrays/5_multidimensional/dimensions_RU}

\subsubsection{Ещё примеры}

Компьютерный экран представляет собой двумерный массив, но видеобуфер это линейный
одномерный массив. 
Мы рассматриваем это здесь: \myref{Mandelbrot_demo}.

Еще один пример в этой книге это игра ``Сапер'': её поле это тоже двухмерный массив: \myref{minesweeper_winxp}.

