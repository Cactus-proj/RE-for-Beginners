\subsection{Buffer overflow protection methods}
\label{subsec:BO_protection}

There are several methods to protect against this scourge, regardless of the \CCpp programmers' negligence.
MSVC has options like\footnote{compiler-side buffer overflow protection methods:
\href{http://en.wikipedia.org/wiki/Buffer_overflow_protection}{wikipedia.org/wiki/Buffer\_overflow\_protection}}:

\begin{lstlisting}
 /RTCs Stack Frame runtime checking
 /GZ Enable stack checks (/RTCs)
\end{lstlisting}

\myindex{x86!\Instructions!RET}
\myindex{Function prologue}
\myindex{Security cookie}

One of the methods is to write a random value between the local variables in stack at function prologue 
and to check it in function epilogue before the function exits.
If value is not the same, do not execute the last instruction \RET, but stop (or hang).
The process will halt, but that is much better than a remote attack to your host.
    
\newcommand{\CANARYURL}{\href{http://en.wikipedia.org/wiki/Domestic_Canary\#Miner.27s_canary}{wikipedia.org/wiki/Domestic\_canary\#Miner.27s\_canary}}

\myindex{Canary}

This random value is called a \q{canary} sometimes, it is related to the miners' canary\footnote{\CANARYURL},
they were used by miners in the past days in order to detect poisonous gases quickly.

Canaries are very sensitive to mine gases, they become very agitated in case of danger, or even die.

If we compile our very simple array example~(\myref{arrays_simple}) in \ac{MSVC}
with RTC1 and RTCs option,\\
you can see a call to \TT{@\_RTC\_CheckStackVars@8} a function at the end of the function that checks if the \q{canary} is correct.

Let's see how GCC handles this. 
Let's take an \TT{alloca()}~(\myref{alloca}) example:

\lstinputlisting[style=customc]{patterns/02_stack/04_alloca/2_1.c}

By default, without any additional options, GCC 4.7.3 inserts a \q{canary} check into the code:

\lstinputlisting[caption=GCC 4.7.3,style=customasmx86]{patterns/13_arrays/3_BO_protection/gcc_canary_EN.asm}

\myindex{x86!\Registers!GS}
The random value is located in \TT{gs:20}. 
It gets written on the stack and then at the end of the function
the value in the stack is compared with the correct \q{canary} in \TT{gs:20}. 
If the values are not equal, the 
\TT{\_\_stack\_chk\_fail} 
function is called and we can see in the console something like that (Ubuntu 13.04 x86):

\begin{lstlisting}
*** buffer overflow detected ***: ./2_1 terminated
======= Backtrace: =========
/lib/i386-linux-gnu/libc.so.6(__fortify_fail+0x63)[0xb7699bc3]
/lib/i386-linux-gnu/libc.so.6(+0x10593a)[0xb769893a]
/lib/i386-linux-gnu/libc.so.6(+0x105008)[0xb7698008]
/lib/i386-linux-gnu/libc.so.6(_IO_default_xsputn+0x8c)[0xb7606e5c]
/lib/i386-linux-gnu/libc.so.6(_IO_vfprintf+0x165)[0xb75d7a45]
/lib/i386-linux-gnu/libc.so.6(__vsprintf_chk+0xc9)[0xb76980d9]
/lib/i386-linux-gnu/libc.so.6(__sprintf_chk+0x2f)[0xb7697fef]
./2_1[0x8048404]
/lib/i386-linux-gnu/libc.so.6(__libc_start_main+0xf5)[0xb75ac935]
======= Memory map: ========
08048000-08049000 r-xp 00000000 08:01 2097586    /home/dennis/2_1
08049000-0804a000 r--p 00000000 08:01 2097586    /home/dennis/2_1
0804a000-0804b000 rw-p 00001000 08:01 2097586    /home/dennis/2_1
094d1000-094f2000 rw-p 00000000 00:00 0          [heap]
b7560000-b757b000 r-xp 00000000 08:01 1048602    /lib/i386-linux-gnu/libgcc_s.so.1
b757b000-b757c000 r--p 0001a000 08:01 1048602    /lib/i386-linux-gnu/libgcc_s.so.1
b757c000-b757d000 rw-p 0001b000 08:01 1048602    /lib/i386-linux-gnu/libgcc_s.so.1
b7592000-b7593000 rw-p 00000000 00:00 0
b7593000-b7740000 r-xp 00000000 08:01 1050781    /lib/i386-linux-gnu/libc-2.17.so
b7740000-b7742000 r--p 001ad000 08:01 1050781    /lib/i386-linux-gnu/libc-2.17.so
b7742000-b7743000 rw-p 001af000 08:01 1050781    /lib/i386-linux-gnu/libc-2.17.so
b7743000-b7746000 rw-p 00000000 00:00 0
b775a000-b775d000 rw-p 00000000 00:00 0
b775d000-b775e000 r-xp 00000000 00:00 0          [vdso]
b775e000-b777e000 r-xp 00000000 08:01 1050794    /lib/i386-linux-gnu/ld-2.17.so
b777e000-b777f000 r--p 0001f000 08:01 1050794    /lib/i386-linux-gnu/ld-2.17.so
b777f000-b7780000 rw-p 00020000 08:01 1050794    /lib/i386-linux-gnu/ld-2.17.so
bff35000-bff56000 rw-p 00000000 00:00 0          [stack]
Aborted (core dumped)
\end{lstlisting}

\myindex{MS-DOS}
gs is the so-called segment register. These registers were used widely in MS-DOS and DOS-extenders
times.
Today, its function is different.
\myindex{TLS}
\myindex{Windows!TIB}

To say it briefly, the \TT{gs} register in Linux always points to the
\ac{TLS}~(\myref{TLS})---some information specific to thread is stored there.
By the way, in win32 the \TT{fs} register plays the same role, pointing to
\ac{TIB} \footnote{\href{https://en.wikipedia.org/wiki/Win32_Thread_Information_Block}{wikipedia.org/wiki/Win32\_Thread\_Information\_Block}}. 

More information can be found in the Linux kernel source code (at least in 3.11 version),\\
in \emph{arch/x86/include/asm/stackprotector.h} this variable is described in the comments.

\input{patterns/13_arrays/3_BO_protection/ARM_EN.tex}

