\subsubsection{Écrire hors des bornes du tableau}

Ok, nous avons lu quelques valeurs de la pile \emph{illégalement}, mais que se passe-t-il
si nous essayons d'écrire quelque chose?

Voici ce que nous avons:

\lstinputlisting[style=customc]{patterns/13_arrays/2_BO/w.c}

\myparagraph{MSVC}

Et ce que nous obtenons:

\lstinputlisting[caption=MSVC 2008 \NonOptimizing,style=customasmx86]{patterns/13_arrays/2_BO/w_FR.asm}

Le programme compilé plante après le lancement. Pas de miracle. Voyons exactement
où il plante.

\clearpage
\myindex{\olly}

Chargeons le dans \olly, et traçons le jusqu'à ce que les 30 éléments du tableau
soient écrits:

\begin{figure}[H]
\centering
\myincludegraphics{patterns/13_arrays/2_BO/olly_w1.png}
\caption{\olly: après avoir restauré la valeur de EBP}
\label{fig:array_BO_olly_w1}
\end{figure}

\clearpage
Exécutons pas à pas jusqu'à la fin de la fonction:

\begin{figure}[H]
\centering
\myincludegraphics{patterns/13_arrays/2_BO/olly_w2.png}
\caption{\olly: 
\TT{EIP} a été restauré, mais \olly ne peut pas désassembler en 0x15}
\label{fig:array_BO_olly_w2}
\end{figure}

Maintenant, gardez vos yeux sur les registres.

\EIP contient maintenant 0x15. Ce n'est pas une adresse légale pour du code---au
moins pour du code win32!
Nous sommes arrivés ici contre notre volonté.
Il est aussi intéressant de voir que le registre \EBP contient 0x14, \ECX et \EDX
contiennent 0x1D.

Étudions un peu plus la structure de la pile.

Après que le contrôle du flux a été passé à \TT{\main}, la valeur du registre \EBP
a été sauvée sur la pile.
Puis, 84 octets ont été alloués pour le tableau et la variable $i$.
C'est \TT{(20+1)*sizeof(int)}.
\ESP pointe maintenant sur la variable \TT{\_i} dans la pile locale et après l'exécution
du \TT{PUSH quelquechose} suivant, \emph{quelquechose} apparaît à côté de \TT{\_i}.

C'est la structure de la pile pendant que le contrôle est dans \main:

\begin{center}
\begin{tabular}{ | l | l | }
\hline
  \TT{ESP}    & 4 octets alloués pour la variable $i$ \\
\hline
  \TT{ESP+4}  & 80 octets alloués pour le tableau \TT{a[20]} \\
\hline
  \TT{ESP+84} & valeur sauvegardée de \EBP \\
\hline
  \TT{ESP+88} & adresse de retour \\
\hline
\end{tabular}
\end{center}

L'expression \TT{a[19]=quelquechose} écrit le dernier \Tint dans des bornes du tableau
(dans les limites jusqu'ici!)

L'expression \TT{a[20]=quelquechose} écrit \emph{quelquechose} à l'endroit où la valeur
sauvegardée de \EBP se trouve.

S'il vous plaît, regardez l'état du registre lors du plantage. Dans notre cas,
20 a été écrit dans le 20ème élément.
À la fin de la fonction, l'épilogue restaure la valeur d'origine de \EBP.
(20 en décimal est \TT{0x14} en hexadécimal).
Ensuite \RET est exécuté, qui est équivalent à l'instruction \TT{POP EIP}.

L'instruction \RET prend la valeur de retour sur la pile (c'est l'adresse dans \ac{CRT}),
qui a appelé \main), et 21 est stocké ici (\TT{0x15} en hexadécimal).
% TODO: clarifier trap
Le CPU trape à l'adresse \TT{0x15}, mais il n'y a pas de code exécutable ici, donc
une exception est levée.

\myindex{\BufferOverflow}

Bienvenu! Ça s'appelle un \emph{buffer overflow (débordement de tampon)}\footnote{\href{http://en.wikipedia.org/wiki/Stack_buffer_overflow}{Wikipédia}}.

Remplacez la tableau de \Tint avec une chaîne (\Tchar array), créez délibérément
une longue chaîne et passez-là au programme, à la fonction, qui ne teste pas la longueur
de la chaîne et la copie dans un petit buffer et vous serez capable de faire pointer
le programme à une adresse où il devra sauter.
C'est pas aussi simple dans la réalité, mais c'est comme cela que ça a apparu.
L'article classique à propos de ça: \AlephOne.

\myparagraph{GCC}

Essayons le même code avec GCC 4.4.1. Nous obtenons:

\lstinputlisting[style=customasmx86]{patterns/13_arrays/2_BO/w_gcc.asm}

Lancer ce programme sous Linux donnera: \TT{Segmentation fault}.

\myindex{GDB}

Si nous le lançons dans le débogueur GDB, nous obtenons ceci:

\begin{lstlisting}
(gdb) r
Starting program: /home/dennis/RE/1

Program received signal SIGSEGV, Segmentation fault.
0x00000016 in ?? ()
(gdb) info registers
eax            0x0	0
ecx            0xd2f96388	-755407992
edx            0x1d	29
ebx            0x26eff4	2551796
esp            0xbffff4b0	0xbffff4b0
ebp            0x15	0x15
esi            0x0	0
edi            0x0	0
eip            0x16	0x16
eflags         0x10202	[ IF RF ]
cs             0x73	115
ss             0x7b	123
ds             0x7b	123
es             0x7b	123
fs             0x0	0
gs             0x33	51
(gdb)
\end{lstlisting}

Les valeurs des registres sont légèrement différentes de l'exemple win32, puisque
la structure de la pile est également légèrement différente.
