\subsection{Набор строк как двухмерный массив}

Снова вернемся к примеру, который возвращает название месяца: \lstref{get_month1}.
Как видно, нужна как минимум одна операция загрузки из памяти для подготовки указателя на строку,
состоящую из имени месяца.

Возможно ли избавиться от операции загрузки из памяти?

Да, если представить список строк как двумерный массив:

\lstinputlisting[style=customc]{patterns/13_arrays/55_month_2D/month2_RU.c}

Вот что получаем:

\lstinputlisting[caption=\Optimizing MSVC 2013 x64,style=customasmx86]{patterns/13_arrays/55_month_2D/MSVC2013_x64_Ox_RU.asm}

Здесь нет обращений к памяти вообще.
Эта функция только вычисляет место, где находится первый символ названия месяца:
 
$pointer\_to\_the\_table + month * 10$.
Там также две инструкции \LEA, которые работают как несколько инструкций \MUL и \MOV.

Ширина массива~--- 10 байт. 
Действительно, самая длинная строка это \q{September} (9 байт) плюс оконечивающий ноль, получается 10 байт.

Остальные названия месяцев дополнены нулевыми байтами, чтобы они занимали столько же места (10 байт).

Таким образом, наша функция и работает быстрее, потому что все строки начинаются с тех адресов, 
которые легко вычислить.

\Optimizing GCC 4.9 может ещё короче:

\begin{lstlisting}[caption=\Optimizing GCC 4.9 x64,style=customasmx86]
	movsx	rdi, edi
	lea	rax, [rdi+rdi*4]
	lea	rax, month2[rax+rax]
	ret
\end{lstlisting}

\LEA здесь также используется для умножения на 10.

Неоптимизирующие компиляторы делают умножение по-разному.

\lstinputlisting[caption=\NonOptimizing GCC 4.9 x64,style=customasmx86]{patterns/13_arrays/55_month_2D/x64_GCC49_O0_RU.asm}

\NonOptimizing MSVC просто использует инструкцию \IMUL:
\myindex{x86!\Instructions!IMUL}

\lstinputlisting[caption=\NonOptimizing MSVC 2013 x64,style=customasmx86]{patterns/13_arrays/55_month_2D/MSVC2013_x64_RU.asm}

\myindex{\CompilerAnomaly}
\label{MSVC2013_anomaly}
Но вот что странно: зачем добавлять умножение на ноль и добавлять ноль к конечному результату?

Это выглядит как странность кодегенератора компилятора, который не был покрыт тестами
компилятора. Но так или иначе, итоговый код работает корректно.
% класс! ЩИТО? 
% что это блядь значит...
Мы сознательно рассматриваем такие фрагменты кода, чтобы читатель понимал, что иногда не нужно
ломать себе голову над подобными артефактами компиляторов.

\subsubsection{32-bit ARM}

\Optimizing Keil для режима Thumb использует инструкцию умножения \INS{MULS}:

\lstinputlisting[caption=\OptimizingKeilVI (\ThumbMode),style=customasmARM]{patterns/13_arrays/55_month_2D/Keil_O3_thumb_RU.asm}

\Optimizing Keil для режима ARM использует операции сложения и сдвига:

\lstinputlisting[caption=\OptimizingKeilVI (\ARMMode),style=customasmARM]{patterns/13_arrays/55_month_2D/Keil_O3_ARM_RU.asm}

\subsubsection{ARM64}

\lstinputlisting[caption=\Optimizing GCC 4.9 ARM64,style=customasmARM]{patterns/13_arrays/55_month_2D/GCC49_ARM64_RU.asm}

\myindex{ARM!\Instructions!SXTW}
\myindex{ARM!\Instructions!ADRP/ADD pair}
\INS{SXTW} используется для знакового расширения и расширения
входного 32-битного значения в 64-битное и сохранения его в X0.

Пара \ADRP/\ADD используется для загрузки адреса таблицы.

У инструкции \ADD также есть суффикс \LSL, что помогает с умножением.

\subsubsection{MIPS}
\lstinputlisting[caption=\Optimizing GCC 4.4.5 (IDA),style=customasmMIPS]{patterns/13_arrays/55_month_2D/MIPS_O3_IDA_RU.lst}

\subsubsection{\Conclusion{}}

Это немного олд-скульная техника для хранения текстовых строк.
Много такого можно найти в \oracle, например.
Трудно сказать, стоит ли оно того на современных компьютерах.
Так или иначе, это был хороший пример массивов, поэтому он был добавлен в эту книгу.

