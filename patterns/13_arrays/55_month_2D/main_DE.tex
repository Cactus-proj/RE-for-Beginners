\subsection{Strings als zweidimensionales Array}
Betrachten wir erneut die Funktion, die den Namen eines Monats zurückgibt: \lstref{get_month1}.

Wie man sieht wird mindestens eine Befehl benötigt, der einen Wert aus dem Speicher lädt, um den Pointer auf den String,
der den Monatsnamen enthält, vorzubereiten.

Ist es möglich diesen Speicherzugriff loszuwerden?

Die Antwort ist: ja, wenn man die Liste aus String als zweidimensionales Array darstellt:

\lstinputlisting[style=customc]{patterns/13_arrays/55_month_2D/month2_DE.c}

Hier ist was wir erhalten:

\lstinputlisting[caption=\Optimizing MSVC 2013 x64,style=customasmx86]{patterns/13_arrays/55_month_2D/MSVC2013_x64_Ox_DE.asm}

Es gibt überhaupt keine Speicherzugriffe.
Alles, was diese Funktion tut, ist einen Pointer zu berechnen, der auf den ersten Buchstaben des Monats
zeigt:$pointer\_to\_the\_table + month \cdot 10$.

Es gibt auch zwei \LEA Befehle, die wie mehrere \MUL und \MOV Befehle funktionieren.

Die Breite des Arrays beträgt 10 Byte.

Der längste String im Beispiel---\q{September}---hat eine Länge von 9 Byte zuzüglich einer terminierenden Null, also
insgesamt 10 Byte.

Die übrigen Monatsnamen werden mit Zerobytes aufgefüllt, sodass alle denselben Speicherplatz (10 Byte) benötigen.

Dadurch arbeitet unsere Funktion noch schneller, denn die Startadresse jedes Strings kann so einfach berechnet werden.

\Optimizing GCC 4.9 kann sogar noch kürzeren Code erzeugen:

\begin{lstlisting}[caption=\Optimizing GCC 4.9 x64,style=customasmx86]
	movsx	rdi, edi
	lea	rax, [rdi+rdi*4]
	lea	rax, month2[rax+rax]
	ret
\end{lstlisting}

\LEA wird hier auch für die Multiplikation mit 10 verwendet.

Nicht optimierende Compiler führen die Multiplikation anders durch.

\lstinputlisting[caption=\NonOptimizing GCC 4.9 x64,style=customasmx86]{patterns/13_arrays/55_month_2D/x64_GCC49_O0_DE.asm}

\NonOptimizing MSVC verwendet nur den \IMUL Befehl:
\myindex{x86!\Instructions!IMUL}

\lstinputlisting[caption=\NonOptimizing MSVC 2013 x64,style=customasmx86]{patterns/13_arrays/55_month_2D/MSVC2013_x64_DE.asm}

\myindex{\CompilerAnomaly}
\label{MSVC2013_anomaly}
Eine Sache hier ist seltsam: warum wird die Multiplikation mit null und die Addition von null zum Endergebnis
hinzugefügt?

Dies sieht wie ein Fehler im Codegenerator des Compilers aus, der nicht durch die Tests des Compilers abgefangen wurde.
(Trotzdem funktioniert der erzeugte Code korrekt.)

% класс!
%
Wir betrachten solche Codes ganz bewußt, damit der Leser sich klarmacht, dass man sich über solche Merkwürdigkeiten und
Artefakte des Compilers nicht allzu sehr wundern soll.

\subsubsection{32-bit ARM}

\Optimizing Keil im Thumb mode verwendet zur Multiplikation den Befehl \INS{MULS}:

\lstinputlisting[caption=\OptimizingKeilVI (\ThumbMode),style=customasmARM]{patterns/13_arrays/55_month_2D/Keil_O3_thumb_DE.asm}

\Optimizing Keil für ARM mode verwendet Additions- und Schiebebefehle:

\lstinputlisting[caption=\OptimizingKeilVI (\ARMMode),style=customasmARM]{patterns/13_arrays/55_month_2D/Keil_O3_ARM_DE.asm}

\subsubsection{ARM64}

\lstinputlisting[caption=\Optimizing GCC 4.9 ARM64,style=customasmARM]{patterns/13_arrays/55_month_2D/GCC49_ARM64_DE.asm}

\myindex{ARM!\Instructions!SXTW}
\myindex{ARM!\Instructions!ADRP/ADD pair}
\INS{SXTW} wird für Vorzeichenerweiterung und Übertragung von 32-Bit-Werten in 64-Bit-Werte und das Speichern in X0
verwendet.

Das \ADRP/\ADD Paar wird für das Laden der Adresse der Tabelle verwendet.

Der \ADD Befehl trägt auch den \LSL Suffix, der bei der Multiplikation hilft.

\subsubsection{MIPS}
\lstinputlisting[caption=\Optimizing GCC 4.4.5 (IDA),style=customasmMIPS]{patterns/13_arrays/55_month_2D/MIPS_O3_IDA_EN.lst}

\subsubsection{\Conclusion{}}
Das Gezeigte ist eine etwas altmodische Technik um Textstrings zu speichern.
Man findet diesen Ansatz beispielsweise oft in \oracle.
Es ist schwer zu sagen, ob es sich für moderne Computer lohnt.
Nichtsdestotrotz ist es ein gutes Beispiel für Arrays und hat daher seine Berechtigung in diesem Buch.

