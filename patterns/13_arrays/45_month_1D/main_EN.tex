\subsection{Array of pointers to strings}
\label{array_of_pointers_to_strings}

Here is an example for an array of pointers.

\lstinputlisting[caption=Get month name,label=get_month1,style=customc]{patterns/13_arrays/45_month_1D/month1_EN.c}

\subsubsection{x64}

\lstinputlisting[caption=\Optimizing MSVC 2013 x64,style=customasmx86]{patterns/13_arrays/45_month_1D/month1_MSVC_2013_x64_Ox.asm}

The code is very simple:

\begin{itemize}

\item
\myindex{x86!\Instructions!MOVSXD}

The first \INS{MOVSXD} instruction copies a 32-bit value from \ECX (where $month$ argument is passed) 
to \RAX with sign-extension (because the $month$ argument is of type \Tint).

The reason for the sign extension is that this 32-bit value is to be used in calculations
with other 64-bit values.

Hence, it has to be promoted to 64-bit%
\footnote{It is somewhat weird, but negative array index could be passed here as $month$
(negative array indices will have been explained later: \myref{negative_array_indices}).
And if this happens, the negative input value of \Tint type is sign-extended correctly 
and the corresponding element before table is picked. 
It is not going to work correctly without sign-extension.}.

\item
Then the address of the pointer table is loaded into \RCX.

\item
Finally, the input value ($month$) is multiplied by 8 and added to the address.
Indeed: we are in a 64-bit environment and all address (or pointers) require exactly 64 bits (or 8 bytes) 
for storage.
Hence, each table element is 8 bytes wide.
And that's why to pick a specific element, $month*8$ bytes has to be skipped from the start.
That's what \MOV does.
In addition, this instruction also loads the element at this address.
For 1, an element would be a pointer to a string that contains \q{February}, etc.

\end{itemize}

\Optimizing GCC 4.9 can do the job even better
\footnote{\q{0+} was left in the listing because GCC assembler output is not tidy enough to eliminate it.
It's \emph{displacement}, and it's zero here.}:

\begin{lstlisting}[caption=\Optimizing GCC 4.9 x64,style=customasmx86]
	movsx	rdi, edi
	mov	rax, QWORD PTR month1[0+rdi*8]
	ret
\end{lstlisting}

\myparagraph{32-bit MSVC}

Let's also compile it in the 32-bit MSVC compiler:

\lstinputlisting[caption=\Optimizing MSVC 2013 x86,style=customasmx86]{patterns/13_arrays/45_month_1D/month1_MSVC_2013_x86_Ox.asm}

The input value does not need to be extended to 64-bit value, so it is used as is.

And it's multiplied by 4, because the table elements are 32-bit (or 4 bytes) wide.

% FIXME1 move to another file
\subsubsection{32-bit ARM}

\myparagraph{ARM in ARM mode}

\lstinputlisting[caption=\OptimizingKeilVI (\ARMMode),style=customasmARM]{patterns/13_arrays/45_month_1D/month1_Keil_ARM_O3.s}

% TODO Fix R1s

The address of the table is loaded in R1.
\myindex{ARM!\Instructions!LDR}

All the rest is done using just one \LDR instruction.

Then input value $month$ is shifted left by 2 (which is the same as multiplying by 4), then added
to R1 (where the address of the table is) and then a table element is loaded from this address.

The 32-bit table element is loaded into R0 from the table.

\myparagraph{ARM in Thumb mode}

The code is mostly the same, but less dense, because the \LSL suffix cannot be specified in the \LDR instruction here:

\begin{lstlisting}[style=customasmARM]
get_month1 PROC
        LSLS     r0,r0,#2
        LDR      r1,|L0.64|
        LDR      r0,[r1,r0]
        BX       lr
        ENDP
\end{lstlisting}

\subsubsection{ARM64}

\lstinputlisting[caption=\Optimizing GCC 4.9 ARM64,style=customasmARM]{patterns/13_arrays/45_month_1D/month1_GCC49_ARM64_O3.s}

\myindex{ARM!\Instructions!ADRP/ADD pair}

The address of the table is loaded in X1 using \ADRP/\ADD pair.

Then corresponding element is picked using just one \LDR, which takes W0 
(the register where input argument $month$ is), shifts it 3 bits to the left (which is the same as multiplying by 8), 
sign-extends it (this is what \q{sxtw} suffix implies) and adds to X0.
Then the 64-bit value is loaded from the table into X0.

\subsubsection{MIPS}

\lstinputlisting[caption=\Optimizing GCC 4.4.5 (IDA),style=customasmMIPS]{patterns/13_arrays/45_month_1D/MIPS_O3_IDA_EN.lst}

\subsubsection{Array overflow}

Our function accepts values in the range of 0..11, but what if 12 is passed?
There is no element in table at this place.

So the function will load some value which happens to be there, and return it.

Soon after, some other function can try to get a text string from this address and may crash.

Let's compile the example in MSVC for win64 and open it in \IDA to see what the linker has placed after the table:

\lstinputlisting[caption=Executable file in IDA,style=customasmx86]{patterns/13_arrays/45_month_1D/MSVC2012_win64_1.lst}

Month names are came right after.

Our program is tiny, so there isn't much data to pack in the data segment, 
so it just the month names.
But it has to be noted that there might be really \emph{anything} that linker has decided to put by chance.

So what if 12 is passed to the function?
The 13th element will be returned.

Let's see how the CPU treats the bytes there as a 64-bit value:

\lstinputlisting[caption=Executable file in IDA,style=customasmx86]{patterns/13_arrays/45_month_1D/MSVC2012_win64_2.lst}

And this is 0x797261756E614A.

Soon after, some other function (presumably, one that processes strings) may try to read bytes at 
this address, expecting a C-string there.

Most likely it is about to crash, because this value doesn't look like a valid address.

\myparagraph{Array overflow protection}

\epigraph{If something can go wrong, it will}{Murphy's Law}

It's a bit naïve to expect that every programmer who use your function or library will never pass
an argument larger than 11.

There exists the philosophy that says \q{fail early and fail loudly} or \q{fail-fast}, 
which teaches to report problems as early as possible and halt.
\myindex{\CStandardLibrary!assert()}

One such method in \CCpp is assertions.

We can modify our program to fail if an incorrect value is passed:

\lstinputlisting[caption=assert() added,style=customc]{patterns/13_arrays/45_month_1D/month1_assert.c}

The assertion macro checks for valid values at every function start and fails if the expression is false.

\lstinputlisting[caption=\Optimizing MSVC 2013 x64,style=customasmx86]{patterns/13_arrays/45_month_1D/MSVC2013_x64_Ox_checked.asm}

In fact, assert() is not a function, but macro. It checks for a condition, then passes also the line number and file
name to another function which reports this information to the user.

Here we see that both file name and condition are encoded in UTF-16.
The line number is also passed (it's 29).

This mechanism is probably the same in all compilers.
Here is what GCC does:

\lstinputlisting[caption=\Optimizing GCC 4.9 x64,style=customasmx86]{patterns/13_arrays/45_month_1D/GCC491_x64_O3_checked.s}

So the macro in GCC also passes the function name for convenience.

Nothing is really free, and this is true for the sanitizing checks as well.

They make your program slower, especially if the assert() macros used in small time-critical functions.

So MSVC, for example, leaves the checks in debug builds, but in release builds they all disappear.
 
Microsoft \gls{Windows NT} kernels come in \q{checked} and \q{free} builds
\footnote{\href{http://msdn.microsoft.com/en-us/library/windows/hardware/ff543450(v=vs.85).aspx}{msdn.microsoft.com/en-us/library/windows/hardware/ff543450(v=vs.85).aspx}}.

The first has validation checks (hence, \q{checked}), the second one doesn't (hence, \q{free} of checks).

Of course, \q{checked} kernel works slower because of all these checks, so it is usually used only in debug sessions.

% FIXME: ARM? MIPS?

\subsubsection{Accessing specific character}

An array of pointers to strings can be accessed like this:

\lstinputlisting[style=customc]{patterns/13_arrays/45_month_1D/month2_EN.c}

\dots since \emph{month[3]} expression has a \emph{const char*} type.
And then, 5th character is taken from that expression by adding 4 bytes to its address.

By the way, arguments list passed to \emph{main()} function has the same data type:

\lstinputlisting[style=customc]{patterns/13_arrays/45_month_1D/argv_EN.c}

It's very important to understand, that, despite similar syntax,
this is different from two-dimensional arrays, which we will consider later.

Another important thing to notice: strings to be addressed must be encoded in a system, where each character occupies single
byte, like \ac{ASCII} and extended \ac{ASCII}.
UTF-8 wouldn't work here.

