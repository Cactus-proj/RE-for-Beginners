\mysection{Fonction presque vide}
\label{Boolector}
\myindex{Boolector}
\myindex{x86!\Instructions!JMP}

Ceci est un morceau de code réel que j'ai trouvé dans Boolector\footnote{\url{https://boolector.github.io/}}:

\lstinputlisting[style=customc]{patterns/025_almost_empty/boolectormain.c}

Pourquoi quelqu'un ferait-il comme ça?
Je ne sais pas mais mon hypothèse est que \verb|boolector_main()| peut être compilée
dans une sorte de DLL ou bibliothèque dynamique, et appelée depuis une suite de test.
Certainement qu'une suite de test peut préparer les variables \verb|argc/argv| comme
le ferait \ac{CRT}.

Il est intéressant de voir comment c'est compilé:

\lstinputlisting[caption=GCC 8.2 x64 \NonOptimizing (\assemblyOutput),style=customasmx86]{patterns/025_almost_empty/boolectormain_O0.s}

Ceci est OK, le prologue (non optimisé) déplace inutilement deux arguments,
\INS{CALL}, épilogue, \INS{RET}.
Mais regardons la version optimisée:

\lstinputlisting[caption=GCC 8.2 x64 \Optimizing (\assemblyOutput),style=customasmx86]{patterns/025_almost_empty/boolectormain_O3.s}

Aussi simple que ça: la pile et les registres ne sont pas touchés et \verb|boolector_main()|
a le même ensemble d'arguments.
Donc, tout ce que nous avons à faire est de passer l'exécution à une autre adresse.

Ceci est proche d'une \glslink{thunk function}{fonction thunk}.

Nous verons queelque chose de plus avancé plus tard: \myref{ARM_B_to_printf}, \myref{JMP_instead_of_RET}.
