\mysection{Почти пустая ф-ция}
\label{Boolector}
\myindex{Boolector}
\myindex{x86!\Instructions!JMP}

Это фрагмент реального кода найденного в Boolector\footnote{\url{https://boolector.github.io/}}:

\lstinputlisting[style=customc]{patterns/025_almost_empty/boolectormain.c}

Зачем кому-то это делать?
Не ясно, но можно предположить что \verb|boolector_main()| может быть скомпилирована в виде DLL или динамической библиотеки,
и может вызываться во время тестов.
Конечно, тесты могут подготовить переменные \verb|argc/argv| так же, как это сделал бы \ac{CRT}.

Интересно, как это компилируется:

\lstinputlisting[caption=\NonOptimizing GCC 8.2 x64 (\assemblyOutput),style=customasmx86]{patterns/025_almost_empty/boolectormain_O0.s}

Здесь у нас: пролог, ненужная (не соптимизированная) перетасовка двух аргументов, \INS{CALL}, эпилог, \INS{RET}.
Посмотрим на оптимизированную версию:

\lstinputlisting[caption=\Optimizing GCC 8.2 x64 (\assemblyOutput),style=customasmx86]{patterns/025_almost_empty/boolectormain_O3.s}

Вот так просто: стек/регистры остаются как есть, а ф-ция \verb|boolector_main()| имеет тот же набор аргументов.
Так что всё что нужно, это просто передать управление по другому адресу.

Это близко к \gls{thunk function}.

Кое-что посложнее мы увидим позже: \myref{ARM_B_to_printf}, \myref{JMP_instead_of_RET}.

