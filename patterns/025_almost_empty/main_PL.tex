\mysection{Funkcja niemal pusta}
\label{Boolector}
\myindex{Boolector}
\myindex{x86!\Instructions!JMP}

Poniższy fragment kodu znalazłem w projekcie Boolector\footnote{\url{https://boolector.github.io/}}:

\lstinputlisting[style=customc]{patterns/025_almost_empty/boolectormain.c}

Dlaczego ktoś miałby tak robić? Nie wiem, ale przypuszczam, że \verb|boolector_main()| może być kompilowane do biblioteki dynamicznej (jak np. DLL) i wywoływane w testach. Kod testowy również może przygotować argumenty argc/argv, tak jak to robi \ac{CRT}.

Ciekawy jest wynik kompilacji:

\lstinputlisting[caption=\NonOptimizing GCC 8.2 x64 (\assemblyOutput),style=customasmx86]{patterns/025_almost_empty/boolectormain_O0.s}

Jest to jak najbardziej poprawny kod, mamy: prolog, niepotrzebne (niezoptymalizowane) przetasowanie dwóch argumentów, \INS{CALL}, epilog i \INS{RET}.

Zobaczmy na efekt kompilacji GCC z włączoną optymalizacją:

\lstinputlisting[caption=\Optimizing GCC 8.2 x64 (\assemblyOutput),style=customasmx86]{patterns/025_almost_empty/boolectormain_O3.s}

Bardzo prosty kod ~--- rejestr i stos zostały nienaruszone, gdyż \verb|boolector_main()| ma taki sam zestaw argumentów. Jedyne co należało zrobić, to przekazać sterowanie pod inny adres.

Przypomina to \glslink{thunk function}{thunk funkcje}.

Później zobaczymy nieco bardziej zaawanzowane przykłady: \myref{ARM_B_to_printf}, \myref{JMP_instead_of_RET}.

