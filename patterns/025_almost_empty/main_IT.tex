\mysection{Una funzione quasi vuota}
\label{Boolector}
\myindex{Boolector}
\myindex{x86!\Instructions!JMP}

Questo è un vero pezzo di codice che ho trovato in  Boolector\footnote{\url{https://boolector.github.io/}}:

\lstinputlisting[style=customc]{patterns/025_almost_empty/boolectormain.c}

Perchè qualcuno dovrebbe farlo?
Non lo so, ma la mia ipotesi migliore è che \verb|boolector_main()| potrebbe essere compilata in qualche sorta di DLL o libreria dinamica,
ed essere chiamata da una suite di test.
Sicuramente, una suite di test può preparare variabili argc/argv come farebbe \ac{CRT}.

E' abbastanza interessante come viene compilato:

\lstinputlisting[caption=\NonOptimizing GCC 8.2 x64 (\assemblyOutput),style=customasmx86]{patterns/025_almost_empty/boolectormain_O0.s}

Questo è OK, prologo, non necessario (non ottimizzato) scambio dei due argomenti \INS{CALL}, epilogo, \INS{RET}.
Ma guardiamo la versione ottimizzata:

\lstinputlisting[caption=\Optimizing GCC 8.2 x64 (\assemblyOutput),style=customasmx86]{patterns/025_almost_empty/boolectormain_O3.s}

Così semplice che: stack/registri non vengono toccati e \verb|boolector_main()| ha gli stessi argomenti settati.
Quindi tutto ciò che dobbiamo fare è passare l'esecuzione ad un altro indirizzo.

Questo è molto simile alla \gls{thunk function}.

Vedremo qualcosa di più avanzato dopo: \myref{ARM_B_to_printf}, \myref{JMP_instead_of_RET}.

