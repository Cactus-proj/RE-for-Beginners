\mysection{Almost empty function}
\label{Boolector}
\myindex{Boolector}
\myindex{x86!\Instructions!JMP}

This is a real piece of code I found in Boolector\footnote{\url{https://boolector.github.io/}}:

\lstinputlisting[style=customc]{patterns/025_almost_empty/boolectormain.c}

Why would anyone do so?
I don't know, but my best guess is that \verb|boolector_main()| may be compiled in some kind of DLL or dynamic library,
and be called from a test suite.
Surely, a test suite can prepare argc/argv variables as \ac{CRT} would do it.

Interestingly enough, how this compiles:

\lstinputlisting[style=customasmx86]{patterns/025_almost_empty/boolectormain.s}

As simple as that: stack/registers are untouched and \verb|boolector_main()| has the same arguments set.
So all we need to do is pass execution to another address.

This is close to \gls{thunk function}.

We will see something more advanced later: \ref{ARM_B_to_printf}, \ref{JMP_instead_of_RET}.

