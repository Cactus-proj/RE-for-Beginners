\subsection{\StructurePackingSectionName}
\label{structure_packing}

1つ重要なことは、構造内のフィールドのパッキングです.

簡単な例を考えてみましょう:

\lstinputlisting[style=customc]{patterns/15_structs/4_packing/packing.c}

見てきたように、2つの \Tchar フィールド(それぞれ1バイト)と2つの \Tint(それぞれ4バイト)があります。

% subsections:
\subsubsection{x86}

このようにコンパイルされます。

\lstinputlisting[caption=MSVC 2012 /GS- /Ob0,label=src:struct_packing_4,numbers=left,style=customasmx86]{patterns/15_structs/4_packing/packing_JA.asm}

構造全体を渡しますが、実際には、構造体は
一時的な領域にコピーされて、(スタック内の領域は10行目に割り当てられ、
次に4つのフィールドはすべて1つずつ、12行目から19行目にコピーされます)
そのポインタ(アドレス)が渡されます。

\ttf{} 関数が構造体を変更するかどうかわからないため、
構造体がコピーされます。 
それが変更された場合、 \main の構造体はそのままでいなければなりません。

私たちは \CCpp ポインタを使うことができました。結果のコードはほぼ同じですが、
コピーは行いません。

次に見るように、各フィールドのアドレスは4バイトの境界に揃えられています。 
だからこそ、各 \Tchar が( \Tint のように)4バイトを占めるのです。なぜでしょうか? 
CPUが整列したアドレスでメモリにアクセスし、メモリからデータをキャッシュする方が簡単であるためです。

しかし、あまり経済的ではありません。

オプション(\TT{/Zp1})(nバイト境界で構造体をパックする \emph{/Zp[n]})で
コンパイルしてみましょう。

\lstinputlisting[caption=MSVC 2012 /GS- /Zp1,label=src:struct_packing_1,numbers=left,style=customasmx86]{patterns/15_structs/4_packing/packing_msvc_Zp1_JA.asm}

Now the structure takes only 10 bytes and each \Tchar value takes 1 byte. What does it give to us?
Size economy. And as drawback~---the CPU accessing these fields slower than it could.

構造体は10バイトしかなく、各 \Tchar 値は1バイト必要です。それは私たちに何を与えるのですか?
サイズ経済。そして欠点として、CPUはこれらのフィールドにアクセスするのが遅くなります。

\label{short_struct_copying_using_MOV}

構造体も \main にコピーされます。フィールド単位ではなく、3つの \MOV ペアを使用して直接10バイトをコピーします。
なぜ4ではないのでしょうか?

コンパイラは、3つの \MOV ペアを使用して10バイトをコピーする方が、2つの32ビットワードと
4つの \MOV ペアを使用して2バイトをコピーするよりも優れていると判断しました。

ちなみに、\TT{memcpy()}関数を呼び出す代わりに \MOV を使用するようなコピーの実装は、
\TT{memcpy()}の呼び出しよりも速いため、広く使用されています。
\myref{copying_short_blocks}

簡単に推測できるように、構造体が多くのソースファイルとオブジェクトファイルで使用されている場合、
構造体パッキングについてはすべて同じ規則でコンパイルする必要があります。

各構造体フィールドの配置方法を設定するMSVC \TT{/Zp}オプションの他に、
\TT{\#pragma pack}コンパイラオプションもあります。このオプションはソースコード内で直接定義できます。 
MSVC\FNURLMSDNZP と GCC\FNURLGCCPC{} の両方で利用できます。

16ビットのフィールドで構成される\TT{SYSTEMTIME}構造体に戻りましょう。
私たちのコンパイラは、1バイト境界でパックすることをどうやって知っていますか?

\TT{WinNT.h}ファイルはこれを持っています:

\begin{lstlisting}[caption=WinNT.h,style=customc]
#include "pshpack1.h"
\end{lstlisting}

そしてこれを。

\begin{lstlisting}[caption=WinNT.h,style=customc]
#include "pshpack4.h"                   // 4バイトパッキングがデフォルト
\end{lstlisting}

PshPack1.h ファイルはこのようになっています。

\lstinputlisting[caption=PshPack1.h,style=customc]{patterns/15_structs/4_packing/tmp1.c}

コンパイラは \TT{\#pragma pack} の後で定義される構造体をパックする方法を知らせます。

\clearpage
\subsubsection{MSVC: x86 + \olly}

\olly でプログラムをハックしようとして、 \scanf が常にエラーなく動作するようにしましょう。 
ローカル変数のアドレスが \scanf に渡されると、
変数には最初にいくつかのランダムなガベージが含まれます。この場合、\TT{0x6E494714}です。

\begin{figure}[H]
\centering
\myincludegraphics{patterns/04_scanf/3_checking_retval/olly_1.png}
\caption{\olly: passing variable address into \scanf}
\label{fig:scanf_ex3_olly_1}
\end{figure}

\clearpage
\scanf が実行されている間、コンソールでは、 \q{asdasd}のように、数字ではないものを入力します。 
\scanf は、エラーが発生したことを示す \EAX が0で終了します。

また、スタック内のローカル変数をチェックし、変更されていないことに注意してください。 
実際、 \scanf は何を書いていますか? 
ゼロを返す以外は何もしませんでした。

私たちのプログラムを\q{ハックする}ようにしましょう。 
\EAX を右クリックし、
オプションの中に\q{Set to 1}があります。 
これが必要なものです。

\EAX には1があるので、以下のチェックを意図どおりに実行し、
\printf は変数の値をスタックに出力します。 

プログラム(F9)を実行すると、コンソールウィンドウで次のように表示されます。

\lstinputlisting[caption=console window]{patterns/04_scanf/3_checking_retval/console.txt}

実際、1850296084はスタック(\TT{0x6E494714})の数値を10進表現したものです!


\input{patterns/15_structs/4_packing/ARM_JA}
\subsubsection{MIPS}

MIPSはいくつかのコプロセッサ(最大4個)をサポートすることができます。
そのうちの0番目\footnote{0から始まる}は特別な制御コプロセッサであり、最初のコプロセッサはFPUです。

ARMと同様に、MIPSコプロセッサはスタックマシンではなく、32個の32ビットレジスタ(\$F0-\$F31)を持ちます。
\myref{MIPS_FPU_registers}.

64ビットの \Tdouble 値を扱う必要がある場合、32ビットのFレジスタのペアが使用されます。

\lstinputlisting[caption=\Optimizing GCC 4.4.5 (IDA),style=customasmMIPS]{patterns/12_FPU/1_simple/MIPS_O3_IDA_JA.lst}

新しい命令は以下です。

\myindex{MIPS!\Instructions!LWC1}
\myindex{MIPS!\Instructions!DIV.D}
\myindex{MIPS!\Instructions!MUL.D}
\myindex{MIPS!\Instructions!ADD.D}
\begin{itemize}

\item \INS{LWC1}は32ビットワードを第1コプロセッサのレジスタにロードします(命令名は\q{1})。
\myindex{MIPS!\Pseudoinstructions!L.D}

一対の\INS{LWC1}命令を組み合わせて\INS{L.D}疑似命令にすることができます。

\item \INS{DIV.D}、 \INS{MUL.D}、 \INS{ADD.D}はそれぞれ除算、乗算、加算を行います
(接尾辞の\q{.D}は倍精度、\q{.S}は単精度を表します)

\end{itemize}

\myindex{MIPS!\Instructions!LUI}
\myindex{\CompilerAnomaly}
\label{MIPS_FPU_LUI}

また、奇妙なコンパイラの例外があります\INS{LUI}命令に疑問符がついています。 
\$V0 レジスタに64ビット定数の \Tdouble 型の一部をロードする理由を理解することは難しいです。 
これらの命令は何の効果もありません。 
% TODO did you try checking out compiler source code?
これについて何か知っているなら、著者に電子メール\footnote{\EMAILS}を送ってください。


\subsubsection{もう一言}

関数の引数として構造体を渡すのは(構造体へのポインタを渡すのではなく)構造体のフィールドを
1つ1つ渡すのと同じです。

構造体のフィールドがデフォルトでパックされる場合、f()関数は以下のように書き換える可能です。

\begin{lstlisting}[style=customc]
void f(char a, int b, char c, int d)
{
    printf ("a=%d; b=%d; c=%d; d=%d\n", a, b, c, d);
};
\end{lstlisting}

そして同じコードになります。
