\subsubsection{x86}

Das Beispiel kompiliert zu folgendem Code:

\lstinputlisting[caption=MSVC 2012 /GS-
/Ob0,label=src:struct_packing_4,numbers=left,style=customasmx86]{patterns/15_structs/4_packing/packing_DE.asm}
Wir übergeben das struct als Ganzes, aber im Code können wir sehen, dass das struct in ein temporäres struct kopiert
wird (ein Platz hierfür wird in Zeile 10 auf dem Stack reserviert), und dass dann alle 4 Felder einzeln (in den Zeilen
12\ldots 19) kopiert werden und anschließend ihr Pointer (Adresse) übergeben wird.

Das struct wird kopiert, da nicht bekannt ist, ob die Funktion \ttf{} das struct verändern wird oder nicht.
Wenn es verändert wird, muss das struct in \main auf dem vorherigen Stand bleiben.

Wir könnten \CCpp Pointer verwenden und der erzeugte Code wäre fast der gleiche nur ohne das Kopieren.

Wie wir sehen können, wird die Adresse von jedem Feld auf einer 4 Byte Grenze angeordnet. Das ist der Grund dafür, dass
jeder \Tchar hier 4 Byte belegt (wie ein \Tint). Es ist für die CPU einfacher auf Speicher an entsprechend angeordneten
Adressen zuzugreifen und Daten von dort im Cache zwischenzuspeichern.

Trotzdem ist dieses Vorgehen nicht besonders ökonomisch.

Komplieren wir mit der Option (\TT{/Zp1})(\emph{/Zp[n] pack structures on n-byte boundary}).

\lstinputlisting[caption=MSVC 2012 /GS-
/Zp1,label=src:struct_packing_1,numbers=left,style=customasmx86]{patterns/15_structs/4_packing/packing_msvc_Zp1_DE.asm}
Das struct benötigt nun lediglich 10 Byte und jeder \Tchar Wert genau 1 Byte. Welchen Vorteil bringt uns das? 
In erster Linie spart man Platz. Ein Nachteil hieran~---die CPU greift hier auf die Felder langsamer zu als es möglich
wäre.

\label{short_struct_copying_using_MOV}
Das struct wird auch in \main kopiert. Nicht Feld für Feld, sondern alle 10 Byte direkt durch Verwendung von drei Paaren
von \MOV Befehlen. Warum aber nicht 4 Paare?

Der Compiler hat entschieden, dass es besser ist, die 10 Byte mit 3 \MOV Befehlspaaren zu kopieren als zwei 32-Bit-Worte
und dann zwei Byte mit insgesamt 4 \MOV Paaren.

Solch eine Implementierung, die zum Kopieren \MOV anstelle eines Aufrufs von \TT{memcpy()} verwendet, ist sehr
gebräuchlich, da es schneller ist als ein Funktionsaufruf von \TT{memcpy()}---zumindest für kurze Blöcke:
\myref{copying_short_blocks}.
Man kann sich leicht überlegen, dass, wenn ein struct in vielen Quellcode und Objektdateien verwendet wird, alle diese
mit der gleichen Konvention bezüglich des Packens in structs kompiliert werden müssen.

Neben der Option \TT{Zp} in MSVC, die die Anordnung der Felder von structs festlegt, gibt es auch die
Compileroption \TT{\#pragma pack}, die direkt im Quellcode definiert werden kann.
Sie ist sowohl in MSVC\FNURLMSDNZP als auch in GCC\FNURLGCCPC{} verfügbar.

Gehen wir zurück zum \TT{SYSTEMTIME} struct, das aus 16-Bit-Feldern besteht.
Wie kann unser Compiler wissen, dass diese in 1-Byte-Anordnung gepackt werden müssen?

Die Datei \TT{WinNT.h} enthält dies:

\begin{lstlisting}[caption=WinNT.h,style=customc]
#include "pshpack1.h"
\end{lstlisting}

Und dies:

\begin{lstlisting}[caption=WinNT.h,style=customc]
#include "pshpack4.h"                   // 4 byte packing is the default
\end{lstlisting}

Die Datei PshPack1.h sieht wie folgt aus:

\lstinputlisting[caption=PshPack1.h,style=customc]{patterns/15_structs/4_packing/tmp1.c}

Dies sagt dem Compiler wie die structs, die hinter \TT{\#pragma pack} definiert werden, gepackt werden müssen.

\clearpage
\subsubsection{MSVC: x86 + \olly}
Laden wir unser Programm in \olly und zwingen es dazu zu glauben, dass \scanf stets ohne Fehler arbeitet.
Wenn die Adresse einer lokalen Variablen an \scanf übergeben wird, enthält die Variable zu Beginn einen zufälligen Wert,
in diesem Fall \TT{0x6E494714}:

\begin{figure}[H]
\centering
\myincludegraphics{patterns/04_scanf/3_checking_retval/olly_1.png}
\caption{\olly: Adresse der Variablen an \scanf übergeben}
\label{fig:scanf_ex3_olly_1}
\end{figure}

\clearpage
Während \scanf ausgeführt wird, geben wir in der Konsole etwas ein, das definitiv keine Zahl ist, z.B. \q{asdasd}.
\scanf beendet sich mit 0 in \EAX, was anzeigt, dass ein Fehler aufgetreten ist.

Wir können auch die lokale Variable auf dem Stack überprüfen und stellen fest, dass sie sich nicht verändert hat.
Was könnte \scanf hier auch hineinschreiben? Die Funktion hat nichts getan außer 0 zurückzugeben.

Versuchen wir unser Programm zu modifizieren, d.i. zu \q{hacken}.
Rechtsklick auf \EAX, in den Optionen finden wir \q{Set to 1}. Das ist was wir brauchen.

Wir haben jetzt 1 in \EAX, sodass die folgende Überprüfung wie gewünscht ausgeführt wird und \printf den Wert der
Variablen auf dem Stack ausgibt.

Wenn wir das Programm laufen lassen (F9), sehen wir das Folgende im Konsolenfenster:

\lstinputlisting[caption=console window]{patterns/04_scanf/3_checking_retval/console.txt}

Und tatsächlich ist 1850296084 die dezimale Darstellung der Zahl auf dem Stack (\TT{0x6E494714})!

