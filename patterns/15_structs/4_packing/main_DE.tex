\subsection{\StructurePackingSectionName}
\label{structure_packing}
Ein wichtiges Thema ist das Packen von Feldern in structs.

Betrachten wir ein einfaches Beispiel:

\lstinputlisting[style=customc]{patterns/15_structs/4_packing/packing.c}
Wie wir sehen haben wir zwei \Tchar Felder (jedes ist exakt ein Byte groß) und zwei weitere vom Typ \Tint (zu je 4
Byte).

% subsections:
\subsubsection{x86}

Das Beispiel kompiliert zu folgendem Code:

\lstinputlisting[caption=MSVC 2012 /GS-
/Ob0,label=src:struct_packing_4,numbers=left,style=customasmx86]{patterns/15_structs/4_packing/packing_DE.asm}
Wir übergeben das struct als Ganzes, aber im Code können wir sehen, dass das struct in ein temporäres struct kopiert
wird (ein Platz hierfür wird in Zeile 10 auf dem Stack reserviert), und dass dann alle 4 Felder einzeln (in den Zeilen
12\ldots 19) kopiert werden und anschließend ihr Pointer (Adresse) übergeben wird.

Das struct wird kopiert, da nicht bekannt ist, ob die Funktion \ttf{} das struct verändern wird oder nicht.
Wenn es verändert wird, muss das struct in \main auf dem vorherigen Stand bleiben.

Wir könnten \CCpp Pointer verwenden und der erzeugte Code wäre fast der gleiche nur ohne das Kopieren.

Wie wir sehen können, wird die Adresse von jedem Feld auf einer 4 Byte Grenze angeordnet. Das ist der Grund dafür, dass
jeder \Tchar hier 4 Byte belegt (wie ein \Tint). Es ist für die CPU einfacher auf Speicher an entsprechend angeordneten
Adressen zuzugreifen und Daten von dort im Cache zwischenzuspeichern.

Trotzdem ist dieses Vorgehen nicht besonders ökonomisch.

Komplieren wir mit der Option (\TT{/Zp1})(\emph{/Zp[n] pack structures on n-byte boundary}).

\lstinputlisting[caption=MSVC 2012 /GS-
/Zp1,label=src:struct_packing_1,numbers=left,style=customasmx86]{patterns/15_structs/4_packing/packing_msvc_Zp1_DE.asm}
Das struct benötigt nun lediglich 10 Byte und jeder \Tchar Wert genau 1 Byte. Welchen Vorteil bringt uns das? 
In erster Linie spart man Platz. Ein Nachteil hieran~---die CPU greift hier auf die Felder langsamer zu als es möglich
wäre.

\label{short_struct_copying_using_MOV}
Das struct wird auch in \main kopiert. Nicht Feld für Feld, sondern alle 10 Byte direkt durch Verwendung von drei Paaren
von \MOV Befehlen. Warum aber nicht 4 Paare?

Der Compiler hat entschieden, dass es besser ist, die 10 Byte mit 3 \MOV Befehlspaaren zu kopieren als zwei 32-Bit-Worte
und dann zwei Byte mit insgesamt 4 \MOV Paaren.

Solch eine Implementierung, die zum Kopieren \MOV anstelle eines Aufrufs von \TT{memcpy()} verwendet, ist sehr
gebräuchlich, da es schneller ist als ein Funktionsaufruf von \TT{memcpy()}---zumindest für kurze Blöcke:
\myref{copying_short_blocks}.
Man kann sich leicht überlegen, dass, wenn ein struct in vielen Quellcode und Objektdateien verwendet wird, alle diese
mit der gleichen Konvention bezüglich des Packens in structs kompiliert werden müssen.

Neben der Option \TT{Zp} in MSVC, die die Anordnung der Felder von structs festlegt, gibt es auch die
Compileroption \TT{\#pragma pack}, die direkt im Quellcode definiert werden kann.
Sie ist sowohl in MSVC\FNURLMSDNZP als auch in GCC\FNURLGCCPC{} verfügbar.

Gehen wir zurück zum \TT{SYSTEMTIME} struct, das aus 16-Bit-Feldern besteht.
Wie kann unser Compiler wissen, dass diese in 1-Byte-Anordnung gepackt werden müssen?

Die Datei \TT{WinNT.h} enthält dies:

\begin{lstlisting}[caption=WinNT.h,style=customc]
#include "pshpack1.h"
\end{lstlisting}

Und dies:

\begin{lstlisting}[caption=WinNT.h,style=customc]
#include "pshpack4.h"                   // 4 byte packing is the default
\end{lstlisting}

Die Datei PshPack1.h sieht wie folgt aus:

\lstinputlisting[caption=PshPack1.h,style=customc]{patterns/15_structs/4_packing/tmp1.c}

Dies sagt dem Compiler wie die structs, die hinter \TT{\#pragma pack} definiert werden, gepackt werden müssen.

\clearpage
\subsubsection{MSVC: x86 + \olly}
Laden wir unser Programm in \olly und zwingen es dazu zu glauben, dass \scanf stets ohne Fehler arbeitet.
Wenn die Adresse einer lokalen Variablen an \scanf übergeben wird, enthält die Variable zu Beginn einen zufälligen Wert,
in diesem Fall \TT{0x6E494714}:

\begin{figure}[H]
\centering
\myincludegraphics{patterns/04_scanf/3_checking_retval/olly_1.png}
\caption{\olly: Adresse der Variablen an \scanf übergeben}
\label{fig:scanf_ex3_olly_1}
\end{figure}

\clearpage
Während \scanf ausgeführt wird, geben wir in der Konsole etwas ein, das definitiv keine Zahl ist, z.B. \q{asdasd}.
\scanf beendet sich mit 0 in \EAX, was anzeigt, dass ein Fehler aufgetreten ist.

Wir können auch die lokale Variable auf dem Stack überprüfen und stellen fest, dass sie sich nicht verändert hat.
Was könnte \scanf hier auch hineinschreiben? Die Funktion hat nichts getan außer 0 zurückzugeben.

Versuchen wir unser Programm zu modifizieren, d.i. zu \q{hacken}.
Rechtsklick auf \EAX, in den Optionen finden wir \q{Set to 1}. Das ist was wir brauchen.

Wir haben jetzt 1 in \EAX, sodass die folgende Überprüfung wie gewünscht ausgeführt wird und \printf den Wert der
Variablen auf dem Stack ausgibt.

Wenn wir das Programm laufen lassen (F9), sehen wir das Folgende im Konsolenfenster:

\lstinputlisting[caption=console window]{patterns/04_scanf/3_checking_retval/console.txt}

Und tatsächlich ist 1850296084 die dezimale Darstellung der Zahl auf dem Stack (\TT{0x6E494714})!


\subsubsection{ARM + \OptimizingXcodeIV (\ARMMode)}

\lstinputlisting[caption=\OptimizingXcodeIV (\ARMMode),label=ARM_leaf_example4,style=customasmARM]{patterns/14_bitfields/4_popcnt/ARM_Xcode_O3_DE.lst}

\myindex{ARM!\Instructions!TST}
\TST entspricht dem Befehl \TEST in x86.

\myindex{ARM!Optional operators!LSL}
\myindex{ARM!Optional operators!LSR}
\myindex{ARM!Optional operators!ASR}
\myindex{ARM!Optional operators!ROR}
\myindex{ARM!Optional operators!RRX}
\myindex{ARM!\Instructions!MOV}
\myindex{ARM!\Instructions!TST}
\myindex{ARM!\Instructions!CMP}
\myindex{ARM!\Instructions!ADD}
\myindex{ARM!\Instructions!SUB}
\myindex{ARM!\Instructions!RSB}
Wie bereits in~(\myref{shifts_in_ARM_mode}) besprochen gibt es zwei verschiedene
Schiebebefehle im ARM mode.
Zusätzlich gibt es aber noch die Suffixe
LSL (\emph{Logical Shift Left}), 
LSR (\emph{Logical Shift Right}), 
ASR (\emph{Arithmetic Shift Right}), 
ROR (\emph{Rotate Right}) und
RRX (\emph{Rotate Right with Extend}), die an Befehle wie \MOV, \TST,
\CMP, \ADD, \SUB, \RSB\footnote{\DataProcessingInstructionsFootNote} angehängt
werden können.

Diese Suffixe legen fest, wie und um wie viele Bits der zweite Operand
verschoben werden soll.

\myindex{ARM!\Instructions!TST}
\myindex{ARM!Optional operators!LSL}
Dadurch entspricht der Befehl \TT{\q{TST R1, R2,LSL R3}} hier 
$R1 \land (R2 \ll R3)$.

\subsubsection{ARM + \OptimizingXcodeIV (\ThumbTwoMode)}

\myindex{ARM!\Instructions!LSL.W}
\myindex{ARM!\Instructions!LSL}
Fast das gleiche, aber hier werden zwei \INS{LSL.W}/\TST Befehle anstelle eines
einzelnen \TST verwendet, da es im Thumb mode nicht möglich ist, den Suffix \LSL
direkt in \TST zu definieren.

\begin{lstlisting}[label=ARM_leaf_example5,style=customasmARM]
                MOV             R1, R0
                MOVS            R0, #0
                MOV.W           R9, #1
                MOVS            R3, #0
loc_2F7A
                LSL.W           R2, R9, R3
                TST             R2, R1
                ADD.W           R3, R3, #1
                IT NE
                ADDNE           R0, #1
                CMP             R3, #32
                BNE             loc_2F7A
                BX              LR
\end{lstlisting}

\subsubsection{ARM64 + \Optimizing GCC 4.9}
Betrachten wir ein 64-Bit-Beispiel, das wir bereits
kennen:\myref{popcnt_x64_example}.

\lstinputlisting[caption=\Optimizing GCC (Linaro) 4.8,style=customasmARM]{patterns/14_bitfields/4_popcnt/ARM64_GCC_O3_DE.s}
Das Ergebnis ist ähnlich dem was GCC für x64 erzeugt:\myref{shifts64_GCC_O3}.

\myindex{ARM!\Instructions!CSEL}
Der Befehl \CSEL steht für \q{Conditional SELect}.
Er wählt eine von zwei Variablen abhängig von den durch \TST gesetzen Flags aus
und kopiert deren Wert nach \RegW{2}, wo die Variable \q{rt} gespeichert wird.

\subsubsection{ARM64 + \NonOptimizing GCC 4.9}
Wieder werden wir hier mit dem bereits bekannten 64-Bit-Beispiel arbeiten:
\myref{popcnt_x64_example}.
Der Code ist umfangreicher als gewöhnlich.

\lstinputlisting[caption=\NonOptimizing GCC (Linaro) 4.8,style=customasmARM]{patterns/14_bitfields/4_popcnt/ARM64_GCC_O0_DE.s}


\subsection{MIPS}

\lstinputlisting[caption=\Optimizing GCC 4.4.5,style=customasmMIPS]{patterns/05_passing_arguments/MIPS_O3_IDA_EN.lst}

Die ersten vier Funktoins Argumente werden in vier Register übergeben die das A- Präfix haben.

\myindex{MIPS!\Instructions!MULT}

Es gibt zwei spezial Register in MIPS: HI und LO die das 64-Bit Multiplikation Ergebnis der
Ausführung der \TT{MULT} Instruktion enthalten.

\myindex{MIPS!\Instructions!MFLO}
\myindex{MIPS!\Instructions!MFHI}

Auf diese Register sind nur zugreifbar durch die \TT{MFLO} und die \TT{MFHI} Instruktionen.
\TT{MFLO} enthält hier die niedrigen Bits aus dem Multiplikations Ergebnis und speichert diese in \$V0.
Also wird der höhere Wert des 32-Bit Teils der multiplikation einfach verworfen ( der HI Register in halt 
wird nicht verwendet ) .
In der Tat: Wir operieren hier auf 32-Bit \Tint Daten Typen.

\myindex{MIPS!\Instructions!ADDU}

Zum Schluss addiert \TT{ADDU} (\q{Add Unsigned}) den Wert des dritten Argumentes zum Ergebnis.

\myindex{MIPS!\Instructions!ADD}
\myindex{MIPS!\Instructions!ADDU}
\myindex{Ada}
\myindex{Integer overflow}

Es gibt zwei unterschiedliche Addition Instruktionen auf der MIPS Plattform: \TT{ADD} und \TT{ADDU}.
Der unterschied zwischen den beiden Instruktionen bezieht sich nicht auf das Vorzeichen (+/-) sondern
auf die exceptions. \TT{ADD} kann eine exception werfen bei einem overflow, was manchmal nützlich\footnote{\url{http://blog.regehr.org/archives/1154}} sein kann und wird auch bei Ada \ac{PL} unterstützt, zum Beispiel:

\TT{ADDU} wirft keine exception bei einem overflow.

Da \CCpp keine Unterstützung hierfür bietet, sehen wir in unserem Beispiel \TT{ADDU} statt \TT{ADD}.


Das 32-Bit Ergebnis bleibt übrig in \$V0.

\myindex{MIPS!\Instructions!JAL}
\myindex{MIPS!\Instructions!JALR}

In \main existiert nun eine neue Instruktion, die interessant für uns ist: \TT{JAL} {\q{Jump an Link}).

Der unterschied zwischen \INS{JAL} und \INS{JALR} ist das in der ersten Instruktion ein relatives offset
hart codiert ist, während \INS{JALR} zur absoluten Adresse gespeichert in einem Register springt (\q{Jump und Link Register}).

Beide \ttf und die  \main Funktionen liegen innerhalb der gleichen Objekt Datei, also ist die 
relative Adresse von \ttf bekannt und fix.



\subsubsection{Eine Sache noch}
Ein struct als Funktionsargument zu übergeben (anstelle eines Pointers auf ein struct) ist das gleiche wie alle Felder
des structs einzeln zu übergeben.

Wenn die Felder im struct standardmäßig gepackt werden, kann die Funktion f() wie folgt neu geschrieben werden:

\begin{lstlisting}[style=customc]
void f(char a, int b, char c, int d)
{
    printf ("a=%d; b=%d; c=%d; d=%d\n", a, b, c, d);
};
\end{lstlisting}
Das führt schlussendlich zum gleichen Code.
