\subsection{MSVC: SYSTEMTIME Beispiel}
\label{sec:SYSTEMTIME}

\newcommand{\FNSYSTEMTIME}{\footnote{\href{http://msdn.microsoft.com/en-us/library/ms724950(VS.85).aspx}{MSDN: SYSTEMTIME structure}}}
Betrachten wir das SYSTEMTIME\FNSYSTEMTIME{} struct in win32, das die Systemzeit beschreibt.

Das struct ist folgendermaßen definiert:

\begin{lstlisting}[caption=WinBase.h,style=customc]
typedef struct _SYSTEMTIME {
  WORD wYear;
  WORD wMonth;
  WORD wDayOfWeek;
  WORD wDay;
  WORD wHour;
  WORD wMinute;
  WORD wSecond;
  WORD wMilliseconds;
} SYSTEMTIME, *PSYSTEMTIME;
\end{lstlisting}
Schreiben wir eine C-Funktion, um die aktuelle Zeit auszugeben:

\lstinputlisting[style=customc]{patterns/15_structs/1_systemtime/systemtime.c}

Wir erhalten das Folgende (MSVC 2010):

\lstinputlisting[caption=MSVC 2010 /GS-,style=customasmx86]{patterns/15_structs/1_systemtime/systemtime.asm}
Für dieses struct werden 16 Byte im lokalen Stack reserviert~---das entspricht genau \TT{sizeof(WORD)*8} (es gibt 8
WORD Variablen in diesem struct).

\newcommand{\FNMSDNGST}{\footnote{\href{http://msdn.microsoft.com/en-us/library/ms724390(VS.85).aspx}{MSDN: GetSystemTime function}}}
Man beachte, dass dieses struct mit dem \TT{wYear} Feld beginnt.
Man kann sagen, dass ein Pointer auf das SYSTEMTIME struct an die Funktion \TT{GetSystemTime()}\FNSYSTEMTIME übergeben
wird, aber man könnte auch sagen, dass ein Pointer auf das Feld \TT{wYear} übergeben wird, denn dabei handelt es sich um
dasselbe!
\TT{GetSystemTime()} schreibt das aktuelle Jahr in den WORD Pointer, verschiebt um 2 Byte, schreibt den aktuellen Monat,
usw. usf. 

\clearpage
\subsubsection{MSVC: x86 + \olly}
Laden wir unser Programm in \olly und zwingen es dazu zu glauben, dass \scanf stets ohne Fehler arbeitet.
Wenn die Adresse einer lokalen Variablen an \scanf übergeben wird, enthält die Variable zu Beginn einen zufälligen Wert,
in diesem Fall \TT{0x6E494714}:

\begin{figure}[H]
\centering
\myincludegraphics{patterns/04_scanf/3_checking_retval/olly_1.png}
\caption{\olly: Adresse der Variablen an \scanf übergeben}
\label{fig:scanf_ex3_olly_1}
\end{figure}

\clearpage
Während \scanf ausgeführt wird, geben wir in der Konsole etwas ein, das definitiv keine Zahl ist, z.B. \q{asdasd}.
\scanf beendet sich mit 0 in \EAX, was anzeigt, dass ein Fehler aufgetreten ist.

Wir können auch die lokale Variable auf dem Stack überprüfen und stellen fest, dass sie sich nicht verändert hat.
Was könnte \scanf hier auch hineinschreiben? Die Funktion hat nichts getan außer 0 zurückzugeben.

Versuchen wir unser Programm zu modifizieren, d.i. zu \q{hacken}.
Rechtsklick auf \EAX, in den Optionen finden wir \q{Set to 1}. Das ist was wir brauchen.

Wir haben jetzt 1 in \EAX, sodass die folgende Überprüfung wie gewünscht ausgeführt wird und \printf den Wert der
Variablen auf dem Stack ausgibt.

Wenn wir das Programm laufen lassen (F9), sehen wir das Folgende im Konsolenfenster:

\lstinputlisting[caption=console window]{patterns/04_scanf/3_checking_retval/console.txt}

Und tatsächlich ist 1850296084 die dezimale Darstellung der Zahl auf dem Stack (\TT{0x6E494714})!


\subsubsection{Ein struct durch ein Array ersetzen}
Die Tatsache, dass die Felder eines structs Variablen sind, die nebeneinander angeordnet sind, kann leicht durch
folgendes Beispiel belegt werden. Wir erinnern uns an die Beschreibung des \TT{SYSTEMTIME} structs und schreiben unser
Beispiel wie folgt um: 

\lstinputlisting[style=customc]{patterns/15_structs/1_systemtime/systemtime2.c}
Der Compiler meckert ein wenig:

\begin{lstlisting}
systemtime2.c(7) : warning C4133: 'function' : incompatible types - from 'WORD [8]' to 'LPSYSTEMTIME'
\end{lstlisting}

Trotzdem erzeugt er den folgenden Code:

\lstinputlisting[caption=\NonOptimizing MSVC 2010,style=customasmx86]{patterns/15_structs/1_systemtime/systemtime2.asm}
Dieses Programm funktioniert genau wie das erste!

Sehr interessant ist, dass dieser Assemblercode nicht vom entsprechenden Code des ersten Beispiels unterschieden werden
kann.

Beim bloßen Ansehen des Codes kann man also nicht feststellen, ob ein struct oder ein Array deklariert wurde.

Trotzdem würde man letzteres nicht annehmen, da es sehr ungebräuchlich ist.

DIe Felder des structs können durch den Entwickler ausgetauscht oder verändert werden, etc.

Wir untersuchen dieses Beispiel nicht in \olly, da es mit dem Beispiel mit dem struct identisch ist.

