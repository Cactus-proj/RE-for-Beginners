\subsection{MSVC: SYSTEMTIME example}
\label{sec:SYSTEMTIME}

\newcommand{\FNSYSTEMTIME}{\footnote{\href{http://msdn.microsoft.com/en-us/library/ms724950(VS.85).aspx}{MSDN: SYSTEMTIME structure}}}

Let's take the SYSTEMTIME\FNSYSTEMTIME{} win32 structure that describes time.

This is how it's defined:

\begin{lstlisting}[caption=WinBase.h,style=customc]
typedef struct _SYSTEMTIME {
  WORD wYear;
  WORD wMonth;
  WORD wDayOfWeek;
  WORD wDay;
  WORD wHour;
  WORD wMinute;
  WORD wSecond;
  WORD wMilliseconds;
} SYSTEMTIME, *PSYSTEMTIME;
\end{lstlisting}

Let's write a C function to get the current time:

\lstinputlisting[style=customc]{patterns/15_structs/1_systemtime/systemtime.c}

We get (MSVC 2010):

\lstinputlisting[caption=MSVC 2010 /GS-,style=customasmx86]{patterns/15_structs/1_systemtime/systemtime.asm}

16 bytes are allocated for this structure in the local stack~---that is exactly \TT{sizeof(WORD)*8}
(there are 8 WORD variables in the structure).

\newcommand{\FNMSDNGST}{\footnote{\href{http://msdn.microsoft.com/en-us/library/ms724390(VS.85).aspx}{MSDN: GetSystemTime function}}}

Pay attention to the fact that the structure begins with the \TT{wYear} field.
It can be said that a pointer to the SYSTEMTIME structure is passed to the \TT{GetSystemTime()}\FNSYSTEMTIME,
but it is also can be said that a pointer to the \TT{wYear} field is passed, and that is the same!
\TT{GetSystemTime()} writes the current year to the WORD pointer pointing to, then shifts 2 bytes
ahead, writes current month, etc., etc.

\clearpage
\subsubsection{MSVC: x86 + \olly}

Let's try to hack our program in \olly, forcing it to think \scanf always works without error.
When an address of a local variable is passed into \scanf,
the variable initially contains some random garbage, in this case \TT{0x6E494714}:

\begin{figure}[H]
\centering
\myincludegraphics{patterns/04_scanf/3_checking_retval/olly_1.png}
\caption{\olly: passing variable address into \scanf}
\label{fig:scanf_ex3_olly_1}
\end{figure}

\clearpage
While \scanf executes, in the console we enter something that is definitely not a number, like \q{asdasd}.
\scanf finishes with 0 in \EAX, which indicates that an error has occurred.

We can also check the local variable in the stack and note that it has not changed.
Indeed, what would \scanf write there?
It simply did nothing except returning zero.

Let's try to \q{hack} our program.
Right-click on \EAX, 
Among the options there is \q{Set to 1}.
This is what we need.

We now have 1 in \EAX, so the following check is to be executed as intended, 
and \printf will print the value of the variable in the stack.

When we run the program (F9) we can see the following in the console window:

\lstinputlisting[caption=console window]{patterns/04_scanf/3_checking_retval/console.txt}

Indeed, 1850296084 is a decimal representation of the number in the stack (\TT{0x6E494714})!


\subsubsection{Replacing the structure with array}

The fact that the structure fields are just variables located side-by-side, can be easily demonstrated by doing the following.
Keeping in mind the \TT{SYSTEMTIME} structure description, it's possible to rewrite this simple example like this:

\lstinputlisting[style=customc]{patterns/15_structs/1_systemtime/systemtime2.c}

The compiler grumbles a bit:

\begin{lstlisting}
systemtime2.c(7) : warning C4133: 'function' : incompatible types - from 'WORD [8]' to 'LPSYSTEMTIME'
\end{lstlisting}

But nevertheless, it produces this code:

\lstinputlisting[caption=\NonOptimizing MSVC 2010,style=customasmx86]{patterns/15_structs/1_systemtime/systemtime2.asm}

And it works just as the same!

It is very interesting that the
result in assembly form cannot be distinguished from the result of the previous compilation.

So by looking at this code, one cannot say for sure if there was a structure declared, or an array. 

Nevertheless, no sane person would do it, as it is not convenient. 

Also the structure fields may be changed by developers, swapped, etc.

We will not study this example in \olly, because it will be just the same as in the case with the structure.

