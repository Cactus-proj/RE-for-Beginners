\subsection{MSVC: exemple SYSTEMTIME}
\label{sec:SYSTEMTIME}

\newcommand{\FNSYSTEMTIME}{\footnote{\href{http://msdn.microsoft.com/en-us/library/ms724950(VS.85).aspx}{MSDN: SYSTEMTIME structure}}}

Considérons la structure win32 SYSTEMTIME\FNSYSTEMTIME{} qui décrit un instant dans le temps. Voici
comment elle est définie:

\begin{lstlisting}[caption=WinBase.h,style=customc]
typedef struct _SYSTEMTIME {
  WORD wYear;
  WORD wMonth;
  WORD wDayOfWeek;
  WORD wDay;
  WORD wHour;
  WORD wMinute;
  WORD wSecond;
  WORD wMilliseconds;
} SYSTEMTIME, *PSYSTEMTIME;
\end{lstlisting}

Écrivons une fonction C pour récupérer l'instant qu'il est:

\lstinputlisting[style=customc]{patterns/15_structs/1_systemtime/systemtime.c}

Le résultat de la compilation avec MSVC 2010 donne:

\lstinputlisting[caption=MSVC 2010 /GS-,style=customasmx86]{patterns/15_structs/1_systemtime/systemtime.asm}

16 octets sont réservés sur la pile pour cette structure, ce qui correspond exactement à
\TT{sizeof(WORD)*8}. La structure comprend effectivement 8 variables d'un WORD chacun.

\newcommand{\FNMSDNGST}{\footnote{\href{http://msdn.microsoft.com/en-us/library/ms724390(VS.85).aspx}{MSDN: GetSystemTime function}}}

Faites attention au fait que le premier membre de la structure est le champ \TT{wYear}.
On peut donc considérer que la fonction \TT{GetSystemTime()}\FNSYSTEMTIME reçoit comme argument
un pointeur sur la structure SYSTEMTIME, ou bien qu'elle reçoit un pointeur sur le champ \TT{wYear}.
Et en fait c'est exactement la même chose!
\TT{GetSystemTime()} écrit l'année courante dans à l'adresse du WORD qu'il a reçu, avance de 2
octets, écrit le mois courant et ainsi de suite.

\clearpage
\subsubsection{MSVC: x86 + \olly}

Essayons de hacker notre programme dans \olly, pour le forcer à penser que \scanf
fonctionne toujours sans erreur.
Lorsque l'adresse d'une variable locale est passée à \scanf, la variable contient
initiallement toujours des restes de données aléatoires, dans ce cas \TT{0x6E494714}:

\begin{figure}[H]
\centering
\myincludegraphics{patterns/04_scanf/3_checking_retval/olly_1.png}
\caption{\olly: passer l'adresse de la variable à \scanf}
\label{fig:scanf_ex3_olly_1}
\end{figure}

\clearpage
Lorsque \scanf s'exécute dans la console, entrons quelque chose qui n'est pas du
tout un nombre, comme \q{asdasd}.
\scanf termine avec 0 dans \EAX, ce qui indique qu'une erreur s'est produite.

Nous pouvons vérifier la variable locale dans le pile et noter qu'elle n'a pas changé.
En effet, qu'aurait écrit \scanf ici?
Elle n'a simplement rien fait à part renvoyer zéro.

Essayons de \q{hacker} notre programme.
Clique-droit sur \EAX,
parmi les options il y a \q{Set to 1} (mettre à 1).
C'est ce dont nous avons besoin.

Nous avons maintenant 1 dans \EAX, donc la vérification suivante va s'exécuter comme
souhaiter et \printf va afficher la valeur de la variable dans la pile.

Lorsque nous lançons le programme (F9) nous pouvons voir ceci dans la fenêtre
de la console:

\lstinputlisting[caption=fenêtre console]{patterns/04_scanf/3_checking_retval/console.txt}

En effet, 1850296084 est la représentation en décimal du nombre dans la pile (\TT{0x6E494714})!


\subsubsection{Remplacer la structure par un tableau}

Le fait que les champs d'une structure ne sont que des variables situées côte-à-côte peut être
aisément démontré de la manière suivante.
Tout en conservant à l'esprit la description de la structure \TT{SYSTEMTIME}, il est possible de
réécrire cet exemple simple de la manière suivante:

\lstinputlisting[style=customc]{patterns/15_structs/1_systemtime/systemtime2.c}

Le compilateur ronchonne certes un peu:

\begin{lstlisting}
systemtime2.c(7) : warning C4133: 'function' : incompatible types - from 'WORD [8]' to 'LPSYSTEMTIME'
\end{lstlisting}

Mais, il consent quand même à produire le code suivant:

\lstinputlisting[caption=\NonOptimizing MSVC 2010,style=customasmx86]{patterns/15_structs/1_systemtime/systemtime2.asm}

Qui fonctionne à l'identique du précédent!

Il est extrêmement intéressant de constater que le code assembleur produit est impossible à
distinguer de celui produit par la compilation précédente.

Et ainsi celui qui observe ce code assembleur est incapable de décider avec certitude si une
structure ou un tableau était déclaré dans le code source en C.

Cela étant, aucun esprit sain ne s'amuserait à déclarer un tableau ici. Car il faut aussi compter
avec la possibilité que la structure soit modifiée par les développeurs, que les champs soient
triés dans un autre ordre ...

Nous n'étudierons pas cet exemple avec \olly, car les résultats seraient identiques à ceux que nous
avons observé en utilisant la structure.

