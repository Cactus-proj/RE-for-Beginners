\subsubsection{CPUID Beispiel}
Die Sprache \CCpp erlaubt die Definition der exakten Anzahl von Bits für jedes Feld in einem struct.
Das ist sehr nützlich, wenn man Speicherplatz sparen muss.
Zum Beispiel genügt ein Bit für eine \Tbool Variable.
Natürlich ist dieses Vorgehen nicht angebracht, wenn die Geschwindigkeit wichtig ist.

% FIXME!
% another use of this is to parse binary protocols/packets, for example
% the definition of struct iphdr in include/linux/ip.h

\newcommand{\FNCPUID}{\footnote{\href{http://en.wikipedia.org/wiki/CPUID}{wikipedia}}}

\myindex{x86!\Instructions!CPUID}
\label{cpuid}
Betrachten wir das Beispiel des \CPUID\FNCPUID Befehls.
Dieser Befehl liefert Informationen über die aktuelle CPU und ihre Eigenschaften.

Wenn \EAX vor der Ausführung des Befehls auf 1 gesetzt ist, liefert \CPUID diese Informationen gepackt in das \EAX
Register zurück:

\begin{center}
\begin{tabular}{ | l | l | }
\hline
3:0 (4 bits)& Schrittweite \\
7:4 (4 bits) & Modell \\
11:8 (4 bits) & Familie \\
13:12 (2 bits) & Prozessortype \\
19:16 (4 bits) & Erweitertes Modell \\
27:20 (8 bits) & Erweiterte Familie \\
\hline
\end{tabular}
\end{center}

\newcommand{\FNGCCAS}{\footnote{\href{http://www.ibiblio.org/gferg/ldp/GCC-Inline-Assembly-HOWTO.html}
{Mehr zum internen GCC Assembler}}}
MSVC 2010 verfügt über ein \CPUID Makro, aber GCC 4.4.1 nicht.
Erstellen wir also für uns eine solche Funktion in GCC, indem wir den built-in Assembler\FNGCCAS verwenden.

\lstinputlisting[style=customc]{patterns/15_structs/6_bitfields/cpuid/CPUID.c}
Nachdem \CPUID die Register \EAX/\EBX/\ECX/\EDX befüllt hat, werden deren Inhalte in das Array \TT{b[]} geschrieben.
Danach haben wir einen Pointer auf das \TT{CPUID\_1\_EAX} struct und zeigen auf den Wert in \EAX aus dem Array \TT{b[]}.

Mit anderen Worten: wir behandeln einen 32-Bit \Tint wie ein struct.
Danach lesen wir spezifische Bits aus dem struct.

\myparagraph{MSVC}
Kompilieren wir das Beispiel in MSVC 2008 mit der Option \Ox:

\lstinputlisting[caption=\Optimizing MSVC 2008,style=customasmx86]{patterns/15_structs/6_bitfields/cpuid/CPUID_msvc_Ox.asm}

\myindex{x86!\Instructions!SHR}
Der Befehl \TT{SHR} verschiebt den Wert in \EAX um die Anzahl der Bits die überprungen werden müssen, d.h. wir
ignorieren einige Bits am rechten Rand.

\myindex{x86!\Instructions!AND}
Der Befehl \AND löscht die nicht benötigten Bits am linken Rand bzw. belässt nur die Bits in \EAX, die wir auch
benötigen.

\clearpage
\subsubsection{MSVC: x86 + \olly}
Laden wir unser Programm in \olly und zwingen es dazu zu glauben, dass \scanf stets ohne Fehler arbeitet.
Wenn die Adresse einer lokalen Variablen an \scanf übergeben wird, enthält die Variable zu Beginn einen zufälligen Wert,
in diesem Fall \TT{0x6E494714}:

\begin{figure}[H]
\centering
\myincludegraphics{patterns/04_scanf/3_checking_retval/olly_1.png}
\caption{\olly: Adresse der Variablen an \scanf übergeben}
\label{fig:scanf_ex3_olly_1}
\end{figure}

\clearpage
Während \scanf ausgeführt wird, geben wir in der Konsole etwas ein, das definitiv keine Zahl ist, z.B. \q{asdasd}.
\scanf beendet sich mit 0 in \EAX, was anzeigt, dass ein Fehler aufgetreten ist.

Wir können auch die lokale Variable auf dem Stack überprüfen und stellen fest, dass sie sich nicht verändert hat.
Was könnte \scanf hier auch hineinschreiben? Die Funktion hat nichts getan außer 0 zurückzugeben.

Versuchen wir unser Programm zu modifizieren, d.i. zu \q{hacken}.
Rechtsklick auf \EAX, in den Optionen finden wir \q{Set to 1}. Das ist was wir brauchen.

Wir haben jetzt 1 in \EAX, sodass die folgende Überprüfung wie gewünscht ausgeführt wird und \printf den Wert der
Variablen auf dem Stack ausgibt.

Wenn wir das Programm laufen lassen (F9), sehen wir das Folgende im Konsolenfenster:

\lstinputlisting[caption=console window]{patterns/04_scanf/3_checking_retval/console.txt}

Und tatsächlich ist 1850296084 die dezimale Darstellung der Zahl auf dem Stack (\TT{0x6E494714})!


\myparagraph{GCC}
Versuchen wir es mit GCC 4.4.1 mit der Option \Othree

\lstinputlisting[caption=\Optimizing GCC 4.4.1,style=customasmx86]{patterns/15_structs/6_bitfields/cpuid/CPUID_gcc_O3.asm}
Fast das gleiche. Das einzig Bemerkenswerte ist, dass GCC die Berechnung von \TT{extended\_model\_id} und
\TT{extended\_family\_id} in einem Block kombiniert, anstatt sie vor jedem Aufruf von \printf getrennt zu berechnen.
