\subsubsection{CPUIDの例}

\CCpp 言語では、各構造体フィールドの正確なビット数を定義できます。 
メモリ空間を節約する必要がある場合に非常に便利です。 
たとえば、 \Tbool 変数には1ビットで十分です。 
しかし、スピードが重要なら合理的ではありません。
% FIXME!
% another use of this is to parse binary protocols/packets, for example
% the definition of struct iphdr in include/linux/ip.h

\newcommand{\FNCPUID}{\footnote{\href{http://en.wikipedia.org/wiki/CPUID}{wikipedia}}}

\myindex{x86!\Instructions!CPUID}
\label{cpuid}

\CPUID\FNCPUID 命令の例を考えてみましょう。
この命令は、現在のCPUとその機能に関する情報を返します。

命令が実行される前に \EAX が1に設定されている場合、
\CPUID は \EAX レジスタに情報がパックされて返ります。

\begin{center}
\begin{tabular}{ | l | l | }
\hline
3:0 (4 bits)& ステッピング \\
7:4 (4 bits) & モデル \\
11:8 (4 bits) & ファミリーy \\
13:12 (2 bits) & プロセッサタイプ \\
19:16 (4 bits) & 拡張モデル \\
27:20 (8 bits) & 拡張ファミリー \\
\hline
\end{tabular}
\end{center}

\newcommand{\FNGCCAS}{\footnote{\href{http://www.ibiblio.org/gferg/ldp/GCC-Inline-Assembly-HOWTO.html}
{GCCアセンブラ内部の詳細}}}

MSVC 2010には \CPUID マクロがありますが、GCC 4.4.1にはありません。 
ですから組み込みアセンブラ \FNGCCAS の助けを借りてGCCのためにこの機能を自分自身で作ってみましょう。

\lstinputlisting[style=customc]{patterns/15_structs/6_bitfields/cpuid/CPUID.c}

\CPUID が \EAX/\EBX/\ECX/\EDX を満たすと、これらのレジスタは\TT{b[]}配列に書き込まれます。 
次に、\TT{CPUID\_1\_EAX}構造体へのポインタを持ち、それを\TT{b[]}配列から \EAX の値に向けます。

つまり、32ビットの \Tint 値を構造体として扱います。 
次に、構造体から特定のビットを読み込みます。

\myparagraph{MSVC}

\Ox オプションを付けてMSVC 2008でコンパイルしてみましょう。

\lstinputlisting[caption=\Optimizing MSVC 2008,style=customasmx86]{patterns/15_structs/6_bitfields/cpuid/CPUID_msvc_Ox.asm}

\myindex{x86!\Instructions!SHR}

\TT{SHR}命令は、 \EAX 内の値を、\emph{スキップ}しなければならないビット数だけシフトします。
例えば、\emph{右側の}ビットを無視します。

\myindex{x86!\Instructions!AND}

\AND 命令は、\emph{左側の}不要ビットをクリアします。言い換えれば、
必要な \EAX レジスタのビットだけを残します。

\clearpage
\subsubsection{MSVC: x86 + \olly}

\olly でプログラムをハックしようとして、 \scanf が常にエラーなく動作するようにしましょう。 
ローカル変数のアドレスが \scanf に渡されると、
変数には最初にいくつかのランダムなガベージが含まれます。この場合、\TT{0x6E494714}です。

\begin{figure}[H]
\centering
\myincludegraphics{patterns/04_scanf/3_checking_retval/olly_1.png}
\caption{\olly: passing variable address into \scanf}
\label{fig:scanf_ex3_olly_1}
\end{figure}

\clearpage
\scanf が実行されている間、コンソールでは、 \q{asdasd}のように、数字ではないものを入力します。 
\scanf は、エラーが発生したことを示す \EAX が0で終了します。

また、スタック内のローカル変数をチェックし、変更されていないことに注意してください。 
実際、 \scanf は何を書いていますか? 
ゼロを返す以外は何もしませんでした。

私たちのプログラムを\q{ハックする}ようにしましょう。 
\EAX を右クリックし、
オプションの中に\q{Set to 1}があります。 
これが必要なものです。

\EAX には1があるので、以下のチェックを意図どおりに実行し、
\printf は変数の値をスタックに出力します。 

プログラム(F9)を実行すると、コンソールウィンドウで次のように表示されます。

\lstinputlisting[caption=console window]{patterns/04_scanf/3_checking_retval/console.txt}

実際、1850296084はスタック(\TT{0x6E494714})の数値を10進表現したものです!


\myparagraph{GCC}

\Othree オプション付きのGCC 4.4.1を試してみましょう。

\lstinputlisting[caption=\Optimizing GCC 4.4.1,style=customasmx86]{patterns/15_structs/6_bitfields/cpuid/CPUID_gcc_O3.asm}

ほとんど同じです。 
唯一注目すべきは、GCCは、 \printf の各呼び出しの前に個別に計算するのではなく、
\TT{extended\_model\_id}と\TT{extended\_family\_id}の計算を
どういうわけか1つのブロックに組み合わせることです。
