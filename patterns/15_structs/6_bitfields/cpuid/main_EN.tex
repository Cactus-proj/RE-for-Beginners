\subsubsection{CPUID example}

The \CCpp language allows to define the exact number of bits for each structure field.
It is very useful if one needs to save memory space. 
For example, one bit is enough for a \Tbool variable.
But of course, it is not rational if speed is important.
% FIXME!
% another use of this is to parse binary protocols/packets, for example
% the definition of struct iphdr in include/linux/ip.h

\newcommand{\FNCPUID}{\footnote{\href{http://en.wikipedia.org/wiki/CPUID}{wikipedia}}}

\myindex{x86!\Instructions!CPUID}
\label{cpuid}

Let's consider the \CPUID\FNCPUID instruction example.
This instruction returns information about the current CPU and its features.

If the \EAX is set to 1 before the instruction's execution, 
\CPUID returning this information packed into the \EAX register:

\begin{center}
\begin{tabular}{ | l | l | }
\hline
3:0 (4 bits)& Stepping \\
7:4 (4 bits) & Model \\
11:8 (4 bits) & Family \\
13:12 (2 bits) & Processor Type \\
19:16 (4 bits) & Extended Model \\
27:20 (8 bits) & Extended Family \\
\hline
\end{tabular}
\end{center}

\newcommand{\FNGCCAS}{\footnote{\href{http://www.ibiblio.org/gferg/ldp/GCC-Inline-Assembly-HOWTO.html}
{More about internal GCC assembler}}}

MSVC 2010 has \CPUID macro, but GCC 4.4.1 does not.
So let's make this function by ourselves for GCC with the help of its built-in assembler\FNGCCAS.

\lstinputlisting[style=customc]{patterns/15_structs/6_bitfields/cpuid/CPUID.c}

After \CPUID fills \EAX/\EBX/\ECX/\EDX, these registers are to be written in the \TT{b[]} array.
Then, we have a pointer to the \TT{CPUID\_1\_EAX} structure and we point it to the value in \EAX from the \TT{b[]} array.

In other words, we treat a 32-bit \Tint value as a structure.
Then we read specific bits from the structure.

\myparagraph{MSVC}

Let's compile it in MSVC 2008 with \Ox option:

\lstinputlisting[caption=\Optimizing MSVC 2008,style=customasmx86]{patterns/15_structs/6_bitfields/cpuid/CPUID_msvc_Ox.asm}

\myindex{x86!\Instructions!SHR}

The \TT{SHR} instruction shifting the value in \EAX by the number of bits that must be
\emph{skipped}, e.g., we ignore some bits \emph{at the right side}.

\myindex{x86!\Instructions!AND}

The \AND instruction clears the unneeded bits \emph{on the left}, or, in other words, 
leaves only those bits in the \EAX register we need.

\clearpage
\subsubsection{MSVC: x86 + \olly}

Let's try to hack our program in \olly, forcing it to think \scanf always works without error.
When an address of a local variable is passed into \scanf,
the variable initially contains some random garbage, in this case \TT{0x6E494714}:

\begin{figure}[H]
\centering
\myincludegraphics{patterns/04_scanf/3_checking_retval/olly_1.png}
\caption{\olly: passing variable address into \scanf}
\label{fig:scanf_ex3_olly_1}
\end{figure}

\clearpage
While \scanf executes, in the console we enter something that is definitely not a number, like \q{asdasd}.
\scanf finishes with 0 in \EAX, which indicates that an error has occurred.

We can also check the local variable in the stack and note that it has not changed.
Indeed, what would \scanf write there?
It simply did nothing except returning zero.

Let's try to \q{hack} our program.
Right-click on \EAX, 
Among the options there is \q{Set to 1}.
This is what we need.

We now have 1 in \EAX, so the following check is to be executed as intended, 
and \printf will print the value of the variable in the stack.

When we run the program (F9) we can see the following in the console window:

\lstinputlisting[caption=console window]{patterns/04_scanf/3_checking_retval/console.txt}

Indeed, 1850296084 is a decimal representation of the number in the stack (\TT{0x6E494714})!


\myparagraph{GCC}

Let's try GCC 4.4.1 with \Othree option.

\lstinputlisting[caption=\Optimizing GCC 4.4.1,style=customasmx86]{patterns/15_structs/6_bitfields/cpuid/CPUID_gcc_O3.asm}

Almost the same.
The only thing worth noting is that GCC somehow combines the calculation of\\
\TT{extended\_model\_id} and \TT{extended\_family\_id} into one block,
instead of calculating them separately before each \printf call.
