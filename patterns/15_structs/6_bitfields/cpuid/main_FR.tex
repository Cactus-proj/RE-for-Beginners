\subsubsection{Exemple CPUID}

Le langage \CCpp permet de définir précisément le nombre de bits occupés par chaque champ d'une structure. 
Ceci est très utile lorsque l'on cherche à économise de la place. Par exemple, chaque bit permet de 
représenter une variable \Tbool. Bien entendu, c'est au détriment de la vitesse d'exécution.
% FIXME!
% another use of this is to parse binary protocols/packets, for example
% the definition of struct iphdr in include/linux/ip.h

\newcommand{\FNCPUID}{\footnote{\href{http://en.wikipedia.org/wiki/CPUID}{Wikipédia}}}

\myindex{x86!\Instructions!CPUID}
\label{cpuid}

Prenons par exemple l'instruction \CPUID\FNCPUID. Elle retourne des informations au sujet de la CPU qui 
exécute le programme et de ses capacités.

Si le registre \EAX est positionné à la valeur 1 avant d'invoquer cette instruction, \CPUID va retourné 
les informations suivantes dans le registre \EAX:

\begin{center}
\begin{tabular}{ | l | l | }
\hline
3:0 (4 bits)& Stepping \\
7:4 (4 bits) & Modèle \\
11:8 (4 bits) & Famille \\
13:12 (2 bits) & Type de processeur \\
19:16 (4 bits) & Sous-modèle \\
27:20 (8 bits) & Sous-famille \\
\hline
\end{tabular}
\end{center}

\newcommand{\FNGCCAS}{\footnote{\href{http://www.ibiblio.org/gferg/ldp/GCC-Inline-Assembly-HOWTO.html}
{Complément sur le fonctionnement interne de l'assembleur GCC}}}

MSVC 2010 fourni une macro \CPUID, qui est absente de GCC 4.4.1. Tentons donc de rédiger nous même cette 
fonction pour une utilisation dans GCC grâce à l'assembleur\FNGCCAS intégré à ce compilateur.

\lstinputlisting[style=customc]{patterns/15_structs/6_bitfields/cpuid/CPUID.c}

Après que l'instruction \CPUID ait rempli les registres \EAX/\EBX/\ECX/\EDX, ceux-ci doivent être recopiés 
dans le tableau \TT{b[]}. Nous affectons dont le pointeur de structure \TT{CPUID\_1\_EAX} pour qu'il 
contienne l'adresse du tableau \TT{b[]}.

En d'autres termes, nous traitons une valeur \Tint comme une structure, puis nous lisons des bits spécifiques 
de la structure.

\myparagraph{MSVC}

Compilons notre exemple avec MSVC 2008 en utilisant l'option \Ox:

\lstinputlisting[caption=\Optimizing MSVC 2008,style=customasmx86]{patterns/15_structs/6_bitfields/cpuid/CPUID_msvc_Ox.asm}

\myindex{x86!\Instructions!SHR}

L'instruction \TT{SHR} va décaler la valeur du registre \EAX d'un certain nombre de bits qui vont être 
abandonnées. Nous ignorons donc certains des bits de la partie droite.

\myindex{x86!\Instructions!AND}

L'instruction \AND "efface" les bits inutiles sur la gauche, ou en d'autres termes, ne laisse dans le 
registre \EAX que les bits qui nous intéressent.

\clearpage
\subsubsection{MSVC: x86 + \olly}

Essayons de hacker notre programme dans \olly, pour le forcer à penser que \scanf
fonctionne toujours sans erreur.
Lorsque l'adresse d'une variable locale est passée à \scanf, la variable contient
initiallement toujours des restes de données aléatoires, dans ce cas \TT{0x6E494714}:

\begin{figure}[H]
\centering
\myincludegraphics{patterns/04_scanf/3_checking_retval/olly_1.png}
\caption{\olly: passer l'adresse de la variable à \scanf}
\label{fig:scanf_ex3_olly_1}
\end{figure}

\clearpage
Lorsque \scanf s'exécute dans la console, entrons quelque chose qui n'est pas du
tout un nombre, comme \q{asdasd}.
\scanf termine avec 0 dans \EAX, ce qui indique qu'une erreur s'est produite.

Nous pouvons vérifier la variable locale dans le pile et noter qu'elle n'a pas changé.
En effet, qu'aurait écrit \scanf ici?
Elle n'a simplement rien fait à part renvoyer zéro.

Essayons de \q{hacker} notre programme.
Clique-droit sur \EAX,
parmi les options il y a \q{Set to 1} (mettre à 1).
C'est ce dont nous avons besoin.

Nous avons maintenant 1 dans \EAX, donc la vérification suivante va s'exécuter comme
souhaiter et \printf va afficher la valeur de la variable dans la pile.

Lorsque nous lançons le programme (F9) nous pouvons voir ceci dans la fenêtre
de la console:

\lstinputlisting[caption=fenêtre console]{patterns/04_scanf/3_checking_retval/console.txt}

En effet, 1850296084 est la représentation en décimal du nombre dans la pile (\TT{0x6E494714})!


\myparagraph{GCC}

Essayons maintenant une compilation avec GCC 4.4.1 en utilisant l'option \Othree.

\lstinputlisting[caption=\Optimizing GCC 4.4.1,style=customasmx86]{patterns/15_structs/6_bitfields/cpuid/CPUID_gcc_O3.asm}

Le résultat est quasiment identique.
Le seul élément notable est que GCC combine en quelques sortes le calcul de \TT{extended\_model\_id} et 
\TT{extended\_family\_id} en un seul bloc au lieu de les calculer séparément avant chaque appel à \printf.
