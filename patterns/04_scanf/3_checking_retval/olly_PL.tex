\clearpage
\subsubsection{MSVC: x86 + \olly}

Spróbujmy zhackować nasz program w \olly, zmuszając go, by uznał, że funkcja \scanf wykonała się bez błędów.
Kiedy adres zmiennej lokalnej jest przekazywany do \scanf,
zmienna początkowo zawiera przypadkową wartość, w tym wypadku \TT{0x6E494714}:

\begin{figure}[H]
\centering
\myincludegraphics{patterns/04_scanf/3_checking_retval/olly_1.png}
\caption{\olly: przekazywanie adresu zmiennej do \scanf}
\label{fig:scanf_ex3_olly_1}
\end{figure}

\clearpage
Kiedy wykonywana jest funkcja \scanf , w konsoli wpiszmy coś, co z pewnością nie jest liczbą, na przykład \q{asdasd}.
\scanf kończy działanie z 0 w \EAX, co wskazuje na wystąpienie błędu:

\begin{figure}[H]
\centering
\myincludegraphics{patterns/04_scanf/3_checking_retval/olly_2.png}
\caption{\olly: \scanf zwraca błąd}
\label{fig:scanf_ex3_olly_2}
\end{figure}

Możemy sprawdzić wartość zmiennej lokalnej na stosie i zauważyć, że się ona nie zmieniła.
W rzeczy samej, dlaczego funkcja \scanf miałaby cokolwiek tam zapisać?
Jej wykonanie nie spowodowało nic, poza zwróceniem zera.

Spróbujmy \q{zhackować} nasz program.
Kliknij prawym przyciskiem na \EAX,
wśród opcji znajduje się \q{Set to 1} (\emph{ustaw na 1}).
To jest to, czego szukamy.

Mamy teraz 1 w \EAX, a więc kolejne sprawdzenie powinno się wykonać zgodnie z oczekiwaniami i
\printf powinna wyświetlić wartość zmiennej ze stosu.

Po wznowieniu wykonania programu (F9) widzimy następujący efekt w oknie konsoli:

\lstinputlisting[caption=console window]{patterns/04_scanf/3_checking_retval/console.txt}

1850296084 to postać dziesiętna liczby, którą widzieliśmy na stosie (\TT{0x6E494714})!
