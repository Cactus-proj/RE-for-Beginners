\subsection{scanf()}

Come già detto in precedenza, usare \scanf oggi è un pò antiquato.
Se proprio dobbiamo, è necessario almeno controllare se \scanf termina correttamente senza errori.

\lstinputlisting[style=customc]{patterns/04_scanf/3_checking_retval/ex3.c}

Per standard, la funzione \scanf\footnote{scanf, wscanf: \href{http://msdn.microsoft.com/en-us/library/9y6s16x1(VS.71).aspx}{MSDN}} restituisce il numero di campi che è riuscita a leggere con successo.
Nel nostro caso, se tutto va bene e l'utente inserisce un numero, \scanf restituisce 1, oppure 0 (o \ac{EOF}) in caso di errore. 

Aggiungiamo un po' di codice C per controllare che \scanf restituisca un valore e stampi un messaggio in caso di errore.

Funziona come ci si aspetta:

\begin{lstlisting}
C:\...>ex3.exe
Enter X:
123
You entered 123...

C:\...>ex3.exe
Enter X:
ouch
What you entered? Huh?
\end{lstlisting}

% subsections
\EN{\subsubsection{MSVC: x86}

Here is what we get in the assembly output (MSVC 2010):

\lstinputlisting[style=customasmx86]{patterns/04_scanf/3_checking_retval/ex3_MSVC_x86.asm}

\myindex{x86!\Registers!EAX}
The \gls{caller} function (\main) needs the \gls{callee} function (\scanf) result, 
so the \gls{callee} returns it in the \EAX register.

\myindex{x86!\Instructions!CMP}
We check it with the help of the instruction \TT{CMP EAX, 1} (\emph{CoMPare}). In other words, we compare the value in the \EAX register with 1.

\myindex{x86!\Instructions!JNE}
A \JNE conditional jump follows the \CMP instruction. \JNE stands for \emph{Jump if Not Equal}.

So, if the value in the \EAX register is not equal to 1, the \ac{CPU} will pass the execution to the address mentioned in the \JNE operand, in our case \TT{\$LN2@main}.
Passing the control to this address results in the \ac{CPU} executing \printf with the argument \TT{What you entered? Huh?}.
But if everything is fine, the conditional jump is not be taken, and another \printf call is to be executed, with two arguments:\\
\TT{'You entered \%d...'} and the value of \TT{x}.

\myindex{x86!\Instructions!XOR}
\myindex{\CLanguageElements!return}
Since in this case the second \printf has not to be executed, there is a \JMP preceding it (unconditional jump). 
It passes the control to the point after the second \printf and just before the \TT{XOR EAX, EAX} instruction, which implements \TT{return 0}.

% FIXME internal \ref{} to x86 flags instead of wikipedia
\myindex{x86!\Registers!\Flags}
So, it could be said that comparing a value with another is \emph{usually} implemented by \CMP/\Jcc instruction pair, where \emph{cc} is \emph{condition code}.
\CMP compares two values and sets processor flags\footnote{x86 flags, see also: \href{http://en.wikipedia.org/wiki/FLAGS_register_(computing)}{wikipedia}.}.
\Jcc checks those flags and decides to either pass the control to the specified address or not.

\myindex{x86!\Instructions!CMP}
\myindex{x86!\Instructions!SUB}
\myindex{x86!\Instructions!JNE}
\myindex{x86!\Registers!ZF}
\label{CMPandSUB}
This could sound paradoxical, but the \CMP instruction is in fact \SUB (subtract).
All arithmetic instructions set processor flags, not just \CMP.
If we compare 1 and 1, $1-1$ is 0 so the \ZF flag would be set (meaning that the last result is 0).
In no other circumstances \ZF can be set, except when the operands are equal.
\JNE checks only the \ZF flag and jumps only if it is not set.  \JNE is in fact a synonym for \JNZ (\emph{Jump if Not Zero}).
Assembler translates both \JNE and \JNZ instructions into the same opcode.
So, the \CMP instruction can be replaced with a \SUB instruction and almost everything will be fine, with the difference that \SUB alters the value of the first operand.
\CMP is \emph{SUB without saving the result, but affecting flags}.

\subsubsection{MSVC: x86: IDA}

\myindex{IDA}
It is time to run \IDA and try to do something in it.
By the way, for beginners it is good idea to use \TT{/MD} option in MSVC, which means that all these
standard functions are not be linked with the executable file, 
but are to be imported from the \TT{MSVCR*.DLL} file instead.
Thus it will be easier to see which standard function are used and where.

While analyzing code in \IDA, it is very helpful to leave notes for oneself (and others).
In instance, analyzing this example, 
we see that \TT{JNZ} is to be triggered in case of an error.
So it is possible to move the cursor to the label, press \q{n} and rename it to \q{error}.
Create another label---into \q{exit}.
Here is my result:

\lstinputlisting[style=customasmx86]{patterns/04_scanf/3_checking_retval/ex3.lst}

Now it is slightly easier to understand the code.
However, it is not a good idea to comment on every instruction.

% FIXME draw button?
You could also hide(collapse) parts of a function in \IDA.
To do that mark the block, then press \q{--} on the numerical pad and enter the text to be displayed instead.

Let's hide two blocks and give them names:

\lstinputlisting[style=customasmx86]{patterns/04_scanf/3_checking_retval/ex3_2.lst}

% FIXME draw button?
To expand previously collapsed parts of the code, use \q{+} on the numerical pad.

\clearpage
By pressing \q{space}, we can see how \IDA represents a function as a graph:

\begin{figure}[H]
\centering
\myincludegraphics{patterns/04_scanf/3_checking_retval/IDA.png}
\caption{Graph mode in IDA}
\label{fig:ex3_IDA_1}
\end{figure}

There are two arrows after each conditional jump: green and red.
The green arrow points to the block which executes if the jump is triggered, and red if otherwise.

\clearpage
It is possible to fold nodes in this mode and give them names as well (\q{group nodes}).
Let's do it for 3 blocks:

\begin{figure}[H]
\centering
\myincludegraphics{patterns/04_scanf/3_checking_retval/IDA2.png}
\caption{Graph mode in IDA with 3 nodes folded}
\label{fig:ex3_IDA_2}
\end{figure}

That is very useful.
It could be said that a very important part of the reverse engineers' job (and any other researcher as well) is to reduce the amount of information they deal with.

\clearpage
\subsubsection{MSVC: x86 + \olly}

Let's try to hack our program in \olly, forcing it to think \scanf always works without error.
When an address of a local variable is passed into \scanf,
the variable initially contains some random garbage, in this case \TT{0x6E494714}:

\begin{figure}[H]
\centering
\myincludegraphics{patterns/04_scanf/3_checking_retval/olly_1.png}
\caption{\olly: passing variable address into \scanf}
\label{fig:scanf_ex3_olly_1}
\end{figure}

\clearpage
While \scanf executes, in the console we enter something that is definitely not a number, like \q{asdasd}.
\scanf finishes with 0 in \EAX, which indicates that an error has occurred.

We can also check the local variable in the stack and note that it has not changed.
Indeed, what would \scanf write there?
It simply did nothing except returning zero.

Let's try to \q{hack} our program.
Right-click on \EAX, 
Among the options there is \q{Set to 1}.
This is what we need.

We now have 1 in \EAX, so the following check is to be executed as intended, 
and \printf will print the value of the variable in the stack.

When we run the program (F9) we can see the following in the console window:

\lstinputlisting[caption=console window]{patterns/04_scanf/3_checking_retval/console.txt}

Indeed, 1850296084 is a decimal representation of the number in the stack (\TT{0x6E494714})!


\clearpage
\subsubsection{MSVC: x86 + Hiew}
\myindex{Hiew}

This can also be used as a simple example of executable file patching.
We may try to patch the executable so the program would always print the input, no matter what we enter.

Assuming that the executable is compiled against external \TT{MSVCR*.DLL} (i.e., with \TT{/MD} option)
\footnote{that's what also called \q{dynamic linking}}, 
we see the \main function at the beginning of the \TT{.text} section.
Let's open the executable in Hiew and find the beginning of the \TT{.text} section (Enter, F8, F6, Enter, Enter).

We can see this:

\begin{figure}[H]
\centering
\myincludegraphics{patterns/04_scanf/3_checking_retval/hiew_1.png}
\caption{Hiew: \main function}
\label{fig:scanf_ex3_hiew_1}
\end{figure}

Hiew finds \ac{ASCIIZ} strings and displays them, as it does with the imported functions' names.

\clearpage
Move the cursor to address \TT{.00401027} (where the \TT{JNZ} instruction, we have to bypass, is located), press F3, and then type \q{9090} (meaning two \ac{NOP}s):

\begin{figure}[H]
\centering
\myincludegraphics{patterns/04_scanf/3_checking_retval/hiew_2.png}
\caption{Hiew: replacing \TT{JNZ} with two \ac{NOP}s}
\label{fig:scanf_ex3_hiew_2}
\end{figure}

Then press F9 (update). Now the executable is saved to the disk. It will behave as we wanted.

Two \ac{NOP}s are probably not the most \ae{}sthetic approach.
Another way to patch this instruction is to write just 0 to the second opcode byte (\gls{jump offset}), 
so that \TT{JNZ} will always jump to the next instruction.

We could also do the opposite: replace first byte with \TT{EB} while not touching the second byte (\gls{jump offset}).
We would get an unconditional jump that is always triggered.
In this case the error message would be printed every time, no matter the input.

}
\RU{\subsubsection{MSVC: x86}

Вот что выходит на ассемблере (MSVC 2010):

\lstinputlisting[style=customasmx86]{patterns/04_scanf/3_checking_retval/ex3_MSVC_x86.asm}

\myindex{x86!\Registers!EAX}
Для того чтобы вызывающая функция имела доступ к результату вызываемой функции, 
вызываемая функция (в нашем случае \scanf) оставляет это значение в регистре \EAX.

\myindex{x86!\Instructions!CMP}
Мы проверяем его инструкцией \TT{CMP EAX, 1} (\emph{CoMPare}), то есть сравниваем значение в \EAX с 1.

\myindex{x86!\Instructions!JNE}
Следующий за инструкцией \CMP: условный переход \JNE. Это означает \emph{Jump if Not Equal}, то есть условный переход \emph{если не равно}.

Итак, если \EAX не равен 1, то \JNE заставит \ac{CPU} перейти по адресу указанном в операнде \JNE, у нас это \TT{\$LN2@main}.
Передав управление по этому адресу, \ac{CPU} начнет исполнять вызов \printf с аргументом \TT{What you entered? Huh?}.
Но если всё нормально, перехода не случится и исполнится другой \printf с двумя аргументами:\\
\TT{'You entered \%d...'} и значением переменной \TT{x}.

\myindex{x86!\Instructions!XOR}
\myindex{\CLanguageElements!return}
Для того чтобы после этого вызова не исполнился сразу второй вызов \printf, 
после него есть инструкция \JMP, безусловный переход, который отправит процессор на место 
после второго \printf и перед инструкцией \TT{XOR EAX, EAX}, которая реализует \TT{return 0}.

% FIXME internal \ref{} to x86 flags instead of wikipedia
\myindex{x86!\Registers!\Flags}
Итак, можно сказать, что в подавляющих случаях сравнение какой-либо переменной с чем-то другим происходит при помощи пары инструкций \CMP и \Jcc, где \emph{cc} это \emph{condition code}.
\CMP сравнивает два значения и выставляет  флаги процессора\footnote{См. также о флагах x86-процессора: \href{http://en.wikipedia.org/wiki/FLAGS_register_(computing)}{wikipedia}.}.
\Jcc проверяет нужные ему флаги и выполняет переход по указанному адресу (или не выполняет).

\myindex{x86!\Instructions!CMP}
\myindex{x86!\Instructions!SUB}
\myindex{x86!\Instructions!JNE}
\myindex{x86!\Registers!ZF}
\label{CMPandSUB}
Но на самом деле, как это не парадоксально поначалу звучит, \CMP это почти то же самое что и инструкция \SUB, которая отнимает числа одно от другого.
Все арифметические инструкции также выставляют флаги в соответствии с результатом, не только \CMP.
Если мы сравним 1 и 1, от единицы отнимется единица, получится 0, и выставится флаг \ZF (\emph{zero flag}), означающий, что последний полученный результат был 0.
Ни при каких других значениях \EAX, флаг \ZF не может быть выставлен, кроме тех, когда операнды равны друг другу.
Инструкция \JNE проверяет только флаг \ZF, и совершает переход только если флаг не поднят. Фактически, \JNE это синоним инструкции \JNZ (\emph{Jump if Not Zero}).
Ассемблер транслирует обе инструкции в один и тот же опкод.
Таким образом, можно \CMP заменить на \SUB и всё будет работать также, но разница в том, что \SUB всё-таки испортит значение в первом операнде.
\CMP это \emph{SUB без сохранения результата, но изменяющая флаги}.

\subsubsection{MSVC: x86: IDA}

\myindex{IDA}
Наверное, уже пора делать первые попытки анализа кода в \IDA.
Кстати, начинающим полезно компилировать в MSVC с ключом \TT{/MD}, что означает, что все эти стандартные
функции не будут скомпонованы с исполняемым файлом, а будут импортироваться из файла \TT{MSVCR*.DLL}.
Так будет легче увидеть, где какая стандартная функция используется.

Анализируя код в \IDA, очень полезно делать пометки для себя (и других).
Например, разбирая этот пример, мы сразу видим, что \TT{JNZ} срабатывает в случае ошибки.
Можно навести курсор на эту метку, нажать \q{n} и переименовать метку в \q{error}.
Ещё одну метку --- в \q{exit}.
Вот как у меня получилось в итоге:

\lstinputlisting[style=customasmx86]{patterns/04_scanf/3_checking_retval/ex3.lst}

Так понимать код становится чуть легче.
Впрочем, меру нужно знать во всем и комментировать каждую инструкцию не стоит.

% FIXME draw button?
В \IDA также можно скрывать части функций: нужно выделить скрываемую часть, нажать \q{--} на цифровой клавиатуре и ввести текст.

Скроем две части и придумаем им названия:

\lstinputlisting[style=customasmx86]{patterns/04_scanf/3_checking_retval/ex3_2.lst}

% FIXME draw button?
Раскрывать скрытые части функций можно при помощи \q{+} на цифровой клавиатуре.

\clearpage
Нажав \q{пробел}, мы увидим, как \IDA может представить функцию в виде графа:

\begin{figure}[H]
\centering
\myincludegraphics{patterns/04_scanf/3_checking_retval/IDA.png}
\caption{Отображение функции в IDA в виде графа}
\label{fig:ex3_IDA_1}
\end{figure}

После каждого условного перехода видны две стрелки: зеленая и красная.
Зеленая ведет к тому блоку, который исполнится если переход сработает, 
а красная~--- если не сработает.

\clearpage
В этом режиме также можно сворачивать узлы и давать им названия (\q{group nodes}).
Сделаем это для трех блоков:

\begin{figure}[H]
\centering
\myincludegraphics{patterns/04_scanf/3_checking_retval/IDA2.png}
\caption{Отображение в IDA в виде графа с тремя свернутыми блоками}
\label{fig:ex3_IDA_2}
\end{figure}

Всё это очень полезно делать.
Вообще, очень важная часть работы реверсера (да и любого исследователя) состоит в том, чтобы уменьшать количество имеющейся информации.

\clearpage
\subsubsection{MSVC: x86 + \olly}

Попробуем в \olly немного хакнуть программу и сделать вид, что \scanf срабатывает всегда без ошибок.
Когда в \scanf передается адрес локальной переменной, изначально в этой переменной
находится некий мусор. В данном случае это \TT{0x6E494714}:

\begin{figure}[H]
\centering
\myincludegraphics{patterns/04_scanf/3_checking_retval/olly_1.png}
\caption{\olly: передача адреса переменной в \scanf}
\label{fig:scanf_ex3_olly_1}
\end{figure}

\clearpage
Когда \scanf запускается, вводим в консоли что-то непохожее на число, например \q{asdasd}.
\scanf заканчивается с 0 в \EAX, что означает, что произошла ошибка.

Вместе с этим мы можем посмотреть на локальную переменную в стеке~--- она не изменилась.
Действительно, ведь что туда записала бы функция \scanf?
Она не делала ничего кроме возвращения нуля.
Попробуем ещё немного \q{хакнуть} нашу программу.
Щелкнем правой кнопкой на \EAX, там, в числе опций, будет также \q{Set to 1}.
Это нам и нужно.

В \EAX теперь 1, последующая проверка пройдет как надо, и \printf выведет значение переменной из стека.

Запускаем (F9) и видим в консоли следующее:

\lstinputlisting[caption=консоль]{patterns/04_scanf/3_checking_retval/console.txt}

Действительно, 1850296084 это десятичное представление числа в стеке (\TT{0x6E494714})!


\clearpage
\subsubsection{MSVC: x86 + Hiew}
\myindex{Hiew}

Это ещё может быть и простым примером исправления исполняемого файла.
Мы можем попробовать исправить его таким образом, что программа всегда будет выводить числа, вне зависимости от ввода.

Исполняемый файл скомпилирован с импортированием функций из
\TT{MSVCR*.DLL} (т.е. с опцией \TT{/MD})\footnote{то, что ещё называют \q{dynamic linking}}, 
поэтому мы можем отыскать функцию \main в самом начале секции \TT{.text}.
Откроем исполняемый файл в Hiew, найдем самое начало секции \TT{.text} (Enter, F8, F6, Enter, Enter).

Мы увидим следующее:

\begin{figure}[H]
\centering
\myincludegraphics{patterns/04_scanf/3_checking_retval/hiew_1.png}
\caption{Hiew: функция \main}
\label{fig:scanf_ex3_hiew_1}
\end{figure}

Hiew находит \ac{ASCIIZ}-строки и показывает их, также как и имена импортируемых функций.

\clearpage
Переведите курсор на адрес \TT{.00401027} (с инструкцией \TT{JNZ}, которую мы хотим заблокировать), нажмите F3, затем наберите \q{9090} (что означает два \ac{NOP}-а):

\begin{figure}[H]
\centering
\myincludegraphics{patterns/04_scanf/3_checking_retval/hiew_2.png}
\caption{Hiew: замена \TT{JNZ} на два \ac{NOP}-а}
\label{fig:scanf_ex3_hiew_2}
\end{figure}

Затем F9 (update). Теперь исполняемый файл записан на диск. Он будет вести себя так, как нам надо.

Два \ac{NOP}-а, возможно, не так эстетично, как могло бы быть.
Другой способ изменить инструкцию это записать 0 во второй байт опкода (смещение перехода),
так что \TT{JNZ} всегда будет переходить на следующую инструкцию.

Можно изменить и наоборот: первый байт заменить на \TT{EB}, второй байт (смещение перехода) не трогать.
Получится всегда срабатывающий безусловный переход.
Теперь сообщение об ошибке будет выдаваться всегда, даже если мы ввели число.

}
\PTBR{\subsubsection{MSVC: x86}

Aqui está o a saída em assembly (MSVC 2010):

\lstinputlisting[style=customasmx86]{patterns/04_scanf/3_checking_retval/ex3_MSVC_x86.asm}

\myindex{x86!\Registers!EAX}
A função que chamou (\main) precisa do resultado da função chamada (\scanf),
então a função chamada retorna esse valor no registrador \EAX.

\myindex{x86!\Instructions!CMP}
Nós verificamos com a ajuda da instrução \TT{CMP EAX, 1} (\emph{CoMParar}). Em outras palavras, comparamos o valor em \EAX com 1.

\myindex{x86!\Instructions!JNE}
O jump condicional \JNE está logo depois da instrução \CMP. \JNE significa \emph{Jump if Not Equal} ou seja, ela desvia se o valor não for igual ao comparado.

Então, se o valor em \EAX não é 1, a \ac{CPU} vai passar a execução para o endereço contido no operando de \JNE, no nosso caso \TT{\$LN2@main}.
Passando a execução para esse endereço resulta na \ac{CPU} executando \printf com o argumento \TT{What you entered? Huh?}.
Mas se tudo estiver correto, o jump condicional não será efetuado e outra chamada do \printf é executada, com dois argumentos: \TT{`You entered \%d...'} e o valor de \TT{x}.

\myindex{x86!\Instructions!XOR}
\myindex{\CLanguageElements!return}
Como nesse caso o segundo \printf() não tem que ser executado, tem um \JMP precedendo ele (jump incondicional).
Ele passa a execução para o ponto depois do segundo \printf e logo antes de \TT{XOR EAX, EAX}, que implementa \TT{return 0}.

% FIXME internal \ref{} to x86 flags instead of wikipedia
\myindex{x86!\Registers!\Flags}
Então, podemos dizer que comparar um valor com outro é geralmente realizado através do par de instruções \CMP/\Jcc, onde \emph{cc} é código condicional.
\CMP compara dois valores e altera os registros da \ac{CPU} (flags)\footnote{\ac{TBT}: x86 flags, see also: \href{http://en.wikipedia.org/wiki/FLAGS_register_(computing)}{wikipedia}.}.
\Jcc checa esses registro e decide passar a execução para o endereço específico contido no operando ou não.

\myindex{x86!\Instructions!CMP}
\myindex{x86!\Instructions!SUB}
\myindex{x86!\Instructions!JNE}
\myindex{x86!\Registers!ZF}
\label{CMPandSUB}
Isso pode parecer meio paradoxal, mas a instrução \CMP é na verdade \SUB (subtrair).
Todo o conjunto de instruções aritiméticas alteram os registros da \ac{CPU}, não só \CMP.
Se compararmos 1 e 1, $1-1$ é 0 então \ZF (zero flag) será acionado (significando que o último resultado foi zero).
Em nenhuma outra circunstância \ZF pode ser acionado, exceto quando os operandos forem iguais.
\JNE verifica somente o ZF e desvia só não estiver acionado.
\JNE é na verdade um sinônimo para \JNZ (jump se não zero).
\JNE e \JNZ são traduzidos no mesmo código de operação.
Então, a instrução CMP pode ser substituida com a instrução \SUB e quase tudo estará certo, com a diferença de que \SUB altera o valor do primeiro operando.
\CMP é \SUB sem salvar o resultado, mas afetando os registros da \ac{CPU}.

\subsubsection{MSVC: x86: IDA}

\PTBRph{}

% TODO translate: \input{patterns/04_scanf/3_checking_retval/olly_PTBR.tex}

\clearpage
\subsubsection{MSVC: x86 + Hiew}
\myindex{Hiew}

Esse exemplo também pode ser usado como uma maneira simples de exemplificar o patch de arquivos executáveis.
Nós podemos tentar rearranjar o executável de forma que o programa sempre imprima a saída, não importando o que inserirmos.

Assumindo que o executavel está compilado com a opção \TT{/MD}\footnote{isso também é chamada ``linkagem dinâmica''}
(\TT{MSVCR*.DLL}), nós vemos a função main no começo da seção \TT{.text}.
Vamos abrir o executável no Hiew e procurar o começo da seção \TT{.text} (Enter, F8, F6, Enter, Enter).

Nós chegamos a isso:

\begin{figure}[H]
\centering
\myincludegraphics{patterns/04_scanf/3_checking_retval/hiew_1.png}
\caption{\PTBRph{}}
\label{fig:scanf_ex3_hiew_1}
\end{figure}

Hiew encontra strings em \ac{ASCIIZ} e as exibe, como faz com os nomes de funções importadas.

\clearpage
Mova o cursor para o endereço \TT{.00401027} (onde a instrução \TT{JNZ}, que temos de evitar, está localizada), aperte F3 e então digite \q{9090} (que significa dois \ac{NOP}s):

\begin{figure}[H]
\centering
\myincludegraphics{patterns/04_scanf/3_checking_retval/hiew_2.png}
\caption{PTBRph{}}
\label{fig:scanf_ex3_hiew_2}
\end{figure}

Então aperte F9 (atualizar). Agora o executável está salvo no disco. Ele executará da maneira que nós desejávamos.

Duas instruções \ac{NOP} não é a abordagem mais estética.
Outra maneira de rearranjar essa instrução é somente escrever um 0 no operando da instrução jump,
então \INS{JNZ} só avançará para a próxima instrução.
% to be sync: Another way to patch this instruction is to write just 0 to the second byte of opcode (\gls{jump offset}), so that \TT{JNZ} will always jump to the next instruction.

Nós poderíamos também ter feito o oposto: mudado o primeiro byte com \TT{EB} e deixa o segundo byte como está.
Nós teriamos um jump incondicional que é sempre deviado.
Nesse caso, a mensagem de erro seria mostrada todas as vezes, não importando a entrada.

}
\IT{\subsubsection{MSVC: x86}

Questo è l'output assembly ottenuto con MSVC 2010:

\lstinputlisting[style=customasmx86]{patterns/04_scanf/3_checking_retval/ex3_MSVC_x86.asm}

\myindex{x86!\Registers!EAX}
La funzione chiamante (\gls{caller}) \main necessita di ottenere il risultato della funzione chiamata (\gls{callee}), 
e pertanto quest'ultima lo restituisce nel registro \EAX register.

\myindex{x86!\Instructions!CMP}
Il controllo viene eseguito con l'aiuto dell'istruzione \TT{CMP EAX, 1} (\emph{CoMPare}). In altre parole, confrontiamo il valore nel registro \EAX con 1.

\myindex{x86!\Instructions!JNE}
U jump condizionale \JNE segue l'istruzione \CMP. \JNE sta per \emph{Jump if Not Equal}.

Quindi, se il valore nel registro \EAX non è uguale a 1, la \ac{CPU} passerà l'esecuzione all'indirizzo specificato nell'operando di \JNE, nel nostro caso \TT{\$LN2@main}.
Passare il controllo a questo indirizzo risulta nel fatto che la \ac{CPU} eseguirà la funzione \printf con l'argomento \TT{What you entered? Huh?}.
Ma se tutto va bene, il salto condizionale non viene effettuato, e viene eseguita un'altra chiamata a \printf con due argomenti: \TT{'You entered \%d...'} e il valore di \TT{x}.

\myindex{x86!\Instructions!XOR}
\myindex{\CLanguageElements!return}
Poichè in questo caso la seconda \printf non deve essere eseguita, c'è un jump non condizionale (unconditional jump) \JMP che la precede. 
Questo passa il controllo al punto dopo la seconda \printf e prima dell'istruzione \TT{XOR EAX, EAX}, che implementa \TT{return 0}.

% FIXME internal \ref{} to x86 flags instead of wikipedia
\myindex{x86!\Registers!\Flags}
Possiamo quindi dire che il confronto di valori è \emph{solitamente} implementato con una coppia di istruzioni \CMP/\Jcc, dove \emph{cc} è un \emph{condition code}.
\CMP confronta due valori e imposta i flag del processore \footnote{x86 flags, vedere anche: \href{http://en.wikipedia.org/wiki/FLAGS_register_(computing)}{wikipedia}.}.
\Jcc controlla questi flag e decide se passare o meno il controllo all'indirizzo specificato.

\myindex{x86!\Instructions!CMP}
\myindex{x86!\Instructions!SUB}
\myindex{x86!\Instructions!JNE}
\myindex{x86!\Registers!ZF}
\label{CMPandSUB}
Può sembrare un paradosso, ma l'istruzione \CMP è in effetti una \SUB (subtract).
Tutte le istruzioni aritmetiche settano i flag del processore, non solo \CMP.
Se confrontiamo 1 e 1, $1-1$ è 0 e quindi il flag \ZF sarebbe impostato a 1 (significando che l'ultimo risultato era 0).
In nessun'altra circostanza il flag \ZF può essere impostato, eccetto il caso in cui gli operandi sono uguali.
\JNE controlla soltanto il flag \ZF e salta se e solo se il flag non è settato.  \JNE è infatti un sinonimo di \JNZ (\emph{Jump if Not Zero}).
L'assembler traduce entrambe le istruzioni \JNE e \JNZ nello stesso opcode.
Quindi l'istruzione \CMP può essere sostituita dall'istruzione \SUB e quasi tutto funzionerà, con la differenza che \SUB altera il valore del primo operando.
\CMP è uguale a \emph{SUB senza salvare il risultato, ma settando i flag}.

\subsubsection{MSVC: x86: IDA}

\myindex{IDA}
E' arrivato il momento di avviare \IDA. A proposito, per i principianti è buona norma usare l'opzione \TT{/MD} in MSVC, che significa che tutte le funzioni
standard non saranno linkate dentro il file eseguibile, ma importate dal file \TT{MSVCR*.DLL}.
In questo modo sarà più facile vedere quali funzioni standard sono usate, e dove.

Quando si analizza il codice con \IDA, è sempre molto utile lasciare note per se stessi (e per gli altri, nel caso in cui si lavori in gruppo).
Per esempio, analizzando questo esempio, notiamo che 
\TT{JNZ} sarà innescato in caso di errore.
E' possibile muovere il cursore fino alla label, premere \q{n} e rinominarla in \q{errore}.
Creare un'altra label ---in \q{exit}.
Ecco il mio risultato:

\lstinputlisting[style=customasmx86]{patterns/04_scanf/3_checking_retval/ex3.lst}

Adesso è leggermente più facile capire il codice.
Non è comunque una buona idea commentare ogni istruzione!

% FIXME draw button?
Si possono anche nascondere (collapse) parti di una funzione in \IDA.
Per farlo, selezionare il blocco e premere Ctrl-\q{--} sul tastierino numerico, inserendo il testo da visualizzare al posto del blocco di codice.

Nascondiamo due blocchi e diamogli un nome:

\lstinputlisting[style=customasmx86]{patterns/04_scanf/3_checking_retval/ex3_2.lst}

% FIXME draw button?
Per espandere dei blocchi nascosti, premere Ctrl-\q{+} sul tastierino numerico.

\clearpage
Premendo \q{spazio}, possiamo vedere come \IDA rappresenta una funzione in forma di grafo:

\begin{figure}[H]
\centering
\myincludegraphics{patterns/04_scanf/3_checking_retval/IDA.png}
\caption{Graph mode in IDA}
\label{fig:ex3_IDA_1}
\end{figure}

Ci sono due frecce dopo ogni jump condizionale: verde e rossa.
La freccia verde punta al blocco che viene eseguito se il jump è innescato, la rossa nel caso opposto.

\clearpage
Anche in questa modalità è possibile "chiudere" i nodi e dargli un'etichetta (\q{group nodes}).
Facciamolo per 3 blocchi:

\begin{figure}[H]
\centering
\myincludegraphics{patterns/04_scanf/3_checking_retval/IDA2.png}
\caption{Graph mode in IDA con 3 nodi "chiusi"}
\label{fig:ex3_IDA_2}
\end{figure}

Come si può vedere questa funzione è molto utile.
Si può dire che una buona parte del lavoro di un reverse engineer (così come di altri tipi di ricercatori) è rappresentata dalla riduzione della quantità di informazioni da trattare.

\clearpage
\subsubsection{MSVC: x86 + \olly}

Proviamo ad hackerare il nostro programma in \olly, forzandolo a pensare che \scanf funzioni sempre senza errori.
Quando l'indirizzo di una variabile locale è passato a \scanf, la variabile inizialmente contiene un valore random inutile, in questo caso \TT{0x6E494714}:

\begin{figure}[H]
\centering
\myincludegraphics{patterns/04_scanf/3_checking_retval/olly_1.png}
\caption{\olly: passaggio dell'indirizzo della variabile a \scanf}
\label{fig:scanf_ex3_olly_1}
\end{figure}

\clearpage
Quando \scanf viene eseguita, immettiamo nella console qualcosa di diverso da un numero, come \q{asdasd}.
\scanf finisce con 0 in \EAX, indicante che un errore si è verificato:

\begin{figure}[H]
\centering
\myincludegraphics{patterns/04_scanf/3_checking_retval/olly_2.png}
\caption{\olly: \scanf ritorno errore}
\label{fig:scanf_ex3_olly_2}
\end{figure}

Possiamo anche controllare la variabile locale nello stack e notare che non è stata modificata.
Infatti cosa avrebbe potuto scrivere \scanf in essa? Non ha fatto niente oltre che restituire zero. 


Proviamo ad \q{hackerare} il nostro programma.
Click destro su \EAX, 
Tra le opzioni vediamo \q{Set to 1}.
Esattamente ciò che ci serve.

Adesso abbiamo 1 in \EAX, il controllo successivo sta per essere eseguito come previsto,
e \printf stamperà il valore della variabile nello stack.

Quando avviamo il programma (F9) vediamo il seguente output nella finestra della console:

\lstinputlisting[caption=console window]{patterns/04_scanf/3_checking_retval/console.txt}

1850296084 è infatti la rappresentazione decimale del numero nello stack (\TT{0x6E494714})!


\clearpage
\subsubsection{MSVC: x86 + Hiew}
\myindex{Hiew}

Quanto detto può essere anche usato come semplice esempio di patching di un eseguibile.
Possiamo provare a modificare l'eseguibile in modo che il programma stampi sempre l'input, a prescindere da cosa si inserisce.

Assumendo che l'eseguibile sia compilato rispetto \TT{MSVCR*.DLL} esterna (ovvero con l'opzione \TT{/MD})
\footnote{detta anche \q{dynamic linking}}, 
vediamo la funzione \main all'inizio della sezione \TT{.text}.
Apriamo l'eseguibile con Hiew e troviamo l'inizio della sezione \TT{.text} (Enter, F8, F6, Enter, Enter).

Vedremo questo:

\begin{figure}[H]
\centering
\myincludegraphics{patterns/04_scanf/3_checking_retval/hiew_1.png}
\caption{Hiew: funzione \main}
\label{fig:scanf_ex3_hiew_1}
\end{figure}

Hiew trova le stringhe \ac{ASCIIZ} e le visualizza, così come i nomi delle funzioni importate.

\clearpage
Spostiamo il cursore all'indirizzo \TT{.00401027} (dove si trova l'istruzione \TT{JNZ} che vogliamo bypassare), premiamo F3, e scriviamo \q{9090} (cioe' due \ac{NOP}):

\begin{figure}[H]
\centering
\myincludegraphics{patterns/04_scanf/3_checking_retval/hiew_2.png}
\caption{Hiew: sostituzione di \TT{JNZ} con due \ac{NOP}}
\label{fig:scanf_ex3_hiew_2}
\end{figure}

Premiamo quindi F9 (update). L'eseguibile viene quindi salvato su disco, e si comporterà come vogliamo.

Utilizzare due \ac{NOP} non rappresenta l'approccio esteticamente migliore.
Un altro modo di patchare questa istruzione è scrivere 0 al secondo byte dell'opcode (\gls{jump offset}), 
in modo che \TT{JNZ} salti sempre alla prossima istruzione.
% to be sync: Another way to patch this instruction is to write just 0 to the second byte of opcode (\gls{jump offset}), so that \TT{JNZ} will always jump to the next instruction.

Potremmo anche fare l'opposto: sostituire il primo byte con \TT{EB} senza toccare il secondo byte (\gls{jump offset}).
Otterremmo un jump non condizionale che è sempre eseguito.
In questo caso il messaggio di errore sarebbe stampato sempre, a prescindere dall'input.

}
\FR{\subsubsection{MSVC: x86}

Voici ce que nous obtenons dans la sortie assembleur (MSVC 2010):

\lstinputlisting[style=customasmx86]{patterns/04_scanf/3_checking_retval/ex3_MSVC_x86.asm}

\myindex{x86!\Registers!EAX}
La fonction \glslink{caller}{appelante} (\main) a besoin du résultat de la fonction
\glslink{callee}{appelée}, donc la fonction \glslink{callee}{appelée} le renvoie
dans la registre \EAX.

\myindex{x86!\Instructions!CMP}
Nous le vérifions avec l'aide de l'instruction \TT{CMP EAX, 1} (\emph{CoMPare}).
En d'autres mots, nous comparons la valeur dans le registre \EAX avec 1.

\myindex{x86!\Instructions!JNE}
Une instruction de saut conditionnelle \JNE suit l'instruction \CMP. \JNE signifie
\emph{Jump if Not Equal} (saut si non égal).

Donc, si la valeur dans le registre \EAX n'est pas égale à 1, le \ac{CPU} va poursuivre
l'exécution à l'adresse mentionnée dans l'opérande \JNE, dans notre cas \TT{\$LN2@main}.
Passer le contrôle à cette adresse résulte en l'exécution par le \ac{CPU} de
\printf avec l'argument \TT{What you entered? Huh?}.
Mais si tout est bon, le saut conditionnel n'est pas pris, et un autre appel à \printf
est exécuté, avec deux arguments:\\
\TT{'You entered \%d...'} et la valeur de \TT{x}.

\myindex{x86!\Instructions!XOR}
\myindex{\CLanguageElements!return}
Puisque dans ce cas le second \printf n'a pas été exécuté, il y a un \JMP qui le précède (saut inconditionnel).
Il passe le contrôle au point après le second \printf et juste avant l'instruction \TT{XOR EAX, EAX}, qui implémente \TT{return 0}.

% FIXME internal \ref{} to x86 flags instead of wikipedia
\myindex{x86!\Registers!\Flags}
Donc, on peut dire que comparer une valeur avec une autre est \emph{usuellement} implémenté
par la paire d'instructions \CMP/\Jcc, où \emph{cc} est un \emph{code de condition}.
\CMP compare deux valeurs et met les flags\footnote{flags x86, voir aussi: \href{http://en.wikipedia.org/wiki/FLAGS_register_(computing)}{Wikipédia}.}
du processeur.
\Jcc vérifie ces flags et décide de passer LE Contrôle à l'adresse spécifiée ou non.

\myindex{x86!\Instructions!CMP}
\myindex{x86!\Instructions!SUB}
\myindex{x86!\Instructions!JNE}
\myindex{x86!\Registers!ZF}
\label{CMPandSUB} 
Cela peut sembler paradoxal, mais l'instruction \CMP est en fait un \SUB (soustraction).
Toutes les instructions arithmétiques mettent les flags du processeur, pas seulement \CMP.
Si nous comparons 1 et 1, $1-1$ donne 0 donc le flag \ZF va être mis (signifiant
que le dernier résultat est 0).
Dans aucune autre circonstance \ZF ne sera mis, sauf si les opérandes sont égaux.
\JNE vérifie seulement le flag \ZF et saute seulement si il n'est pas mis. \JNE
est un synonyme pour \JNZ (\emph{Jump if Not Zero} (saut si non zéro)).
L'assembleur génère le même opcode pour les instructions \JNE et \JNZ.
Donc, l'instruction \CMP peut être remplacée par une instruction \SUB et presque
tout ira bien, à la différence que \SUB altère la valeur du premier opérande.
\CMP est un \emph{SUB sans sauver le résultat, mais modifiant les flags}.

\subsubsection{MSVC: x86: IDA}

\myindex{IDA}
C'est le moment de lancer \IDA et d'essayer de faire quelque chose avec.
À propos, pour les débutants, c'est une bonne idée d'utiliser l'option \TT{/MD}
de MSVC, qui signifie que toutes les fonctions standards ne vont pas être liées
avec le fichier exécutable, mais vont à la place être importées depuis le fichier
\TT{MSVCR*.DLL}.
Ainsi il est plus facile de voir quelles fonctions standards sont utilisées et où.

En analysant du code dans \IDA, il est très utile de laisser des notes pour soi-même
(et les autres).
En la circonstance, analysons cet exemple, nous voyons que \TT{JNZ} sera déclenché
en cas d'erreur.
Donc il est possible de déplacer le curseur sur le label, de presser \q{n} et de
lui donner le nom \q{error}.
Créons un autre label---dans \q{exit}.
Voici mon résultat:

\lstinputlisting[style=customasmx86]{patterns/04_scanf/3_checking_retval/ex3.lst}

Maintenant, il est légèrement plus facile de comprendre le code.
Toutefois, ce n'est pas une bonne idée de commenter chaque instruction.

% FIXME draw button?
Vous pouvez aussi cacher (replier) des parties d'une fonction dans \IDA.
Pour faire cela, marquez le bloc, puis appuyez sur Ctrl-\q{--} sur le pavé numérique et
entrez le texte qui doit être affiché à la place.

Cachons deux blocs et donnons leurs un nom:

\lstinputlisting[style=customasmx86]{patterns/04_scanf/3_checking_retval/ex3_2.lst}

% FIXME draw button?
Pour étendre les parties de code précédemment cachées. utilisez Ctrl-\q{+} sur le
pavé numérique.

\clearpage
En appuyant sur \q{space}, nous voyons comment \IDA représente une fonction sous
forme de graphe:

\begin{figure}[H]
\centering
\myincludegraphics{patterns/04_scanf/3_checking_retval/IDA.png}
\caption{IDA en mode graphe}
\label{fig:ex3_IDA_1}
\end{figure}

Il y a deux flèches après chaque saut conditionnel: une verte et une rouge.
La flèche verte pointe vers le bloc qui sera exécuté si le saut est déclenché,
et la rouge sinon.

\clearpage
Il est possible de replier des n\œu{}ds dans ce mode et de leurs donner aussi un nom (\q{group nodes}).
Essayons avec 3 blocs:

\begin{figure}[H]
\centering
\myincludegraphics{patterns/04_scanf/3_checking_retval/IDA2.png}
\caption{IDA en mode graphe avec 3 nœuds repliés}
\label{fig:ex3_IDA_2}
\end{figure}

C'est très pratique.
On peut dire qu'une part importante du travail des rétro-ingénieurs (et de tout
autre chercheur également) est de réduire la quantité d'information avec laquelle
travailler.

\clearpage
\subsubsection{MSVC: x86 + \olly}

Essayons de hacker notre programme dans \olly, pour le forcer à penser que \scanf
fonctionne toujours sans erreur.
Lorsque l'adresse d'une variable locale est passée à \scanf, la variable contient
initiallement toujours des restes de données aléatoires, dans ce cas \TT{0x6E494714}:

\begin{figure}[H]
\centering
\myincludegraphics{patterns/04_scanf/3_checking_retval/olly_1.png}
\caption{\olly: passer l'adresse de la variable à \scanf}
\label{fig:scanf_ex3_olly_1}
\end{figure}

\clearpage
Lorsque \scanf s'exécute dans la console, entrons quelque chose qui n'est pas du
tout un nombre, comme \q{asdasd}.
\scanf termine avec 0 dans \EAX, ce qui indique qu'une erreur s'est produite.

Nous pouvons vérifier la variable locale dans le pile et noter qu'elle n'a pas changé.
En effet, qu'aurait écrit \scanf ici?
Elle n'a simplement rien fait à part renvoyer zéro.

Essayons de \q{hacker} notre programme.
Clique-droit sur \EAX,
parmi les options il y a \q{Set to 1} (mettre à 1).
C'est ce dont nous avons besoin.

Nous avons maintenant 1 dans \EAX, donc la vérification suivante va s'exécuter comme
souhaiter et \printf va afficher la valeur de la variable dans la pile.

Lorsque nous lançons le programme (F9) nous pouvons voir ceci dans la fenêtre
de la console:

\lstinputlisting[caption=fenêtre console]{patterns/04_scanf/3_checking_retval/console.txt}

En effet, 1850296084 est la représentation en décimal du nombre dans la pile (\TT{0x6E494714})!


\clearpage
\subsubsection{MSVC: x86 + Hiew}
\myindex{Hiew}

Cela peut également être utilisé comme un exemple simple de modification de fichier
exécutable.
Nous pouvons essayer de modifier l'exécutable de telle sorte que le programme va
toujours afficher notre entrée, quelle qui'elle soit.

En supposant que l'exécutable est compilé avec la bibliothèque externe \TT{MSVCR*.DLL}
(i.e., avec l'option \TT{/MD}) \footnote{c'est aussi appelé \q{dynamic linking}},
nous voyons la fonction \main au début de la section \TT{.text}.
Ouvrons l'exécutable dans Hiew et cherchons le début de la section \TT{.text} (Enter,
F8, F6, Enter, Enter).

Nous pouvons voir cela:

\begin{figure}[H]
\centering
\myincludegraphics{patterns/04_scanf/3_checking_retval/hiew_1.png}
\caption{Hiew: fonction \main}
\label{fig:scanf_ex3_hiew_1}
\end{figure}

Hiew trouve les chaîne \ac{ASCIIZ} et les affiche, comme il le fait avec le nom
des fonctions importées.

\clearpage
Déplacez le curseur à l'adresse \TT{.00401027} (où se trouve l'instruction \TT{JNZ},
que l'on doit sauter), appuyez sur F3, et ensuite tapez \q{9090} (qui signifie deux
\ac{NOP}s):

\begin{figure}[H]
\centering
\myincludegraphics{patterns/04_scanf/3_checking_retval/hiew_2.png}
\caption{Hiew: remplacement de \TT{JNZ} par deux \ac{NOP}s}
\label{fig:scanf_ex3_hiew_2}
\end{figure}

Appuyez sur F9 (update). Maintenant, l'exécutable est sauvé sur le disque. Il va
se comporter comme nous le voulions.

Deux \ac{NOP}s ne constitue probablement pas l'approche la plus esthétique.
Une autre façon de modifier cette instruction est d'écrire simplement 0 dans le
second octet de l'opcode ((\gls{jump offset}), donc ce \TT{JNZ} va toujours sauter
à l'instruction suivante.

Nous pouvons également faire le contraire: remplacer le premier octet avec \TT{EB}
sans modifier le second octet (\gls{jump offset}).
Nous obtiendrions un saut inconditionnel qui est toujours déclenché.
Dans ce cas le message d'erreur sera affiché à chaque fois, peu importe l'entrée.

}
\DE{\subsubsection{MSVC: x86}
Wir erhalten den folgenden Assembleroutput (MSVC 2010):

\lstinputlisting[style=customasmx86]{patterns/04_scanf/3_checking_retval/ex3_MSVC_x86.asm}

\myindex{x86!\Registers!EAX}
Die aufrufende Funktion (\main) benötigt das Ergebnis der aufgerufenen Funktion (\scanf), weshalb diese es über das
Register \EAX zurückgibt.

\myindex{x86!\Instructions!CMP}
Wir prüfen mithilfe des Befehls \TT{CMP EAX, 1} (\emph{CoMPare}). Mit anderen Worten, wir vergleichen den Wert in \EAX mit
1.

\myindex{x86!\Instructions!JNE}
Ein bedingter \JNE Sprung folgt auf den \CMP Befehl. \JNE steht für \emph{Jump if Not Equal}.

Wenn also der Wert in \EAX ungleich 1 ist, wird die \ac{CPU} die Ausführung an der Stelle fortsetzen, die im Operanden
von \JNE steht, in unserem Fall \TT{\$LN2@main}.
Den Control Flow an diese Adresse zu übergeben hat zur Folge, dass die Funktion \printf mit dem Argument \TT{What you
entered? Huh?} aufgerufen wird.
Wenn aber alles funktioniert und der bedingte Sprung nicht ausgeführt wird, wird ein anderer Aufruf von \printf mit zwei
Argumenten ausgeführt:\\
\TT{'You entered \%d...'} und dem Wert von \TT{x}.


\myindex{x86!\Instructions!XOR}
\myindex{\CLanguageElements!return}
Da in diesem Fall das zweite \printf nicht ausgeführt werden darf, befindet sich davor ein \JMP (unbedingter Sprung).
Dieser gibt den Control Flow ab an den Punkt nach dem zweiten \printf, genau vor dem \TT{XOR EAX EAX} Befehl, welcher
die Rückgabe von 0 implementiert.

% FIXME internal \ref{} to x86 flags instead of wikipedia
\myindex{x86!\Registers!\Flags}
Man kann also festhalten, dass der Vergleich von zwei Werten gewöhnlich durch ein \CMP/\Jcc Befehlspaar implementiert
wird, wobei \emph{cc} für \emph{condition code}, also Sprungbedingung, steht. 
\CMP vergleicht zwei Werte und setzt die Flags des Prozessors\footnote{zu x86 Flags, siehe auch:
\href{http://en.wikipedia.org/wiki/FLAGS_register_(computing)}{wikipedia}.}.
\Jcc prüft diese Flags und entscheidet entweder den Control Flow an die angegebene Adresse zu übergeben oder nicht.

\myindex{x86!\Instructions!CMP}
\myindex{x86!\Instructions!SUB}
\myindex{x86!\Instructions!JNE}
\myindex{x86!\Registers!ZF}
\label{CMPandSUB}
Es klingt möglicherweise paradox, aber der \CMP Befehl ist tatsächlich ein \SUB (subtract).
Alle arithmetischen Befehle setzen die Flags des Prozessors, nicht nur \CMP.
Wenn wir 1 und 1 vergleichen, ist $1-1=0$ und daher wird das \ZF Flag gesetzt (gleichbedeutend damit, dass das Ergebnis
der letzten Berechnung 0 ergeben hat).
\ZF kann nur durch diesen Umstand gesetzt werden, nämlich, dass zwei Operanden gleich sind.
\JNE prüft nur das \ZF Flag und springt nur, wenn dieses nicht gesetzt ist. \JNE ist daher ein Synonym für \JNZ
(\emph{Jump if Not Zero}).
Der Assembler übersetzt \JNE und \JNZ in den gleichen Opcode.
Der \CMP Befehl kann also durch ein \SUB ersetzt werden, aber mit dem Unterschied, dass \SUB den Wert des ersten
Operanden verändert. \CMP bedeutet also \emph{SUB ohne Speichern des Ergebnisses, aber mit Setzen der Flags}.

\subsubsection{MSVC: x86: IDA}

\myindex{IDA}
Es ist an der Zeit \IDA auszuprobieren und etwas damit zu machen.
Für Anfänger ist es übrigens eine gute Idee, die \TT{/MD} Option in MSVC zu verwenden, da diese bewirkt, dass alle
Standardfunktionen nicht mit der ausführbaren Datei verlinkt werden, sondern aus der Datei \TT{MSVCR*.DLL} importiert
werden. Dadurch ist es einfacher zu erkennen, welche Standardfunktionen verwendet werden und wo dies geschieht.

Bei der Codeanalyse in \IDA ist es hilfreich Notizen für sich selbst (und andere) zu hinterlassen.
Bei der Analyse dieses Beispiels sehen wir, dass \TT{JNZ} im Falle eines Fehlers ausgeführt wird.
Es ist nun möglich den Cursor auf das Label zu setzen, \q{n} zu drücken und es in \q{error} umzubenennen.
Wir erstellen noch ein Label---in \q{exit}.
Hier ist das Ergebnis:

\lstinputlisting[style=customasmx86]{patterns/04_scanf/3_checking_retval/ex3.lst}
So ist es etwas einfacher den Code zu verstehen. Natürlich ist es aber auch keine gute Idee, jeden Befehl zu
kommentieren.

% FIXME draw button?
Man kann Teile einer Funktion in \IDA auch einklappen. Um dies zu tun, markiert man den Block und drückt dann Ctrl-\q{--} auf
dem Zahlenblock der Tastatur und gibt den stattdessen anzuzeigenden Text ein.

Verstecken wir zwei Blöcke und geben ihnen Namen:

\lstinputlisting[style=customasmx86]{patterns/04_scanf/3_checking_retval/ex3_2.lst}

% FIXME draw button?
Um eingeklappte Teile des Code wieder auszuklappen, verwendet man Ctrl-\q{+} auf dem Zahlenblock der Tastatur.

\clearpage
Durch Drücken von der \q{Leertaste} sehen wir, wie \IDA die Funktion als Graph darstellt:

\begin{figure}[H]
\centering
\myincludegraphics{patterns/04_scanf/3_checking_retval/IDA.png}
\caption{Graph Modus in IDA}
\label{fig:ex3_IDA_1}
\end{figure}
Es gibt hinter jedem bedingten Sprung zwei Pfeile: einen grünen und einen roten.
Der grüne Pfeil zeigt auf den Codeblock der ausgeführt wird, wenn der Sprung ausgeführt wird und der rote den Codeblock,
der ausgeführt wird, falls nicht gesprungen wird.

\clearpage
In diesem Modus ist es möglich, Knoten einzuklappen und ihnen auch Namen zu geben (\q{group nodes}).
Wir probieren das mit 3 Blöcken aus:

\begin{figure}[H]
\centering
\myincludegraphics{patterns/04_scanf/3_checking_retval/IDA2.png}
\caption{Graph Modus in IDA mit 3 eingeklappten Knoten}
\label{fig:ex3_IDA_2}
\end{figure}

Das ist sehr nützlich.
Man kann sagen, dass ein großer Teil der Arbeit eines Reverse Engineers (und eines jeden anderen Forsches) darin
besteht, die Menge der zur Verfügung stehenden Informationen zu reduzieren.

\clearpage
\subsubsection{MSVC: x86 + \olly}
Laden wir unser Programm in \olly und zwingen es dazu zu glauben, dass \scanf stets ohne Fehler arbeitet.
Wenn die Adresse einer lokalen Variablen an \scanf übergeben wird, enthält die Variable zu Beginn einen zufälligen Wert,
in diesem Fall \TT{0x6E494714}:

\begin{figure}[H]
\centering
\myincludegraphics{patterns/04_scanf/3_checking_retval/olly_1.png}
\caption{\olly: Adresse der Variablen an \scanf übergeben}
\label{fig:scanf_ex3_olly_1}
\end{figure}

\clearpage
Während \scanf ausgeführt wird, geben wir in der Konsole etwas ein, das definitiv keine Zahl ist, z.B. \q{asdasd}.
\scanf beendet sich mit 0 in \EAX, was anzeigt, dass ein Fehler aufgetreten ist.

Wir können auch die lokale Variable auf dem Stack überprüfen und stellen fest, dass sie sich nicht verändert hat.
Was könnte \scanf hier auch hineinschreiben? Die Funktion hat nichts getan außer 0 zurückzugeben.

Versuchen wir unser Programm zu modifizieren, d.i. zu \q{hacken}.
Rechtsklick auf \EAX, in den Optionen finden wir \q{Set to 1}. Das ist was wir brauchen.

Wir haben jetzt 1 in \EAX, sodass die folgende Überprüfung wie gewünscht ausgeführt wird und \printf den Wert der
Variablen auf dem Stack ausgibt.

Wenn wir das Programm laufen lassen (F9), sehen wir das Folgende im Konsolenfenster:

\lstinputlisting[caption=console window]{patterns/04_scanf/3_checking_retval/console.txt}

Und tatsächlich ist 1850296084 die dezimale Darstellung der Zahl auf dem Stack (\TT{0x6E494714})!


\clearpage
\subsubsection{MSVC: x86 + Hiew}
\myindex{Hiew}
Unser Programm kann auch als einfaches Beispel für das Patchen einer Executable dienen.
Wir könnten versuchen, die Executable so zu patchen, dass das Programm unabhängig vom Input diesen stets auszugeben.

Angenommen, dass die Executable mit externer \TT{MSVCR*.DLL} (d.h. mit der Option \TT{MD}) kompiliert
wurde\footnote{dieser Vorgang wird auch \q{dynamisches Verlinken genannt}}, finden wir die Funktion \main am Anfang des
\TT{.text} Segments. 
Öffnen wir die Executable in Hiew und schauen uns den Anfang des \TT{.text} Segments an (Enter, F8, F6, Enter, Enter).

Wir sehen das Folgende:

\begin{figure}[H]
\centering
\myincludegraphics{patterns/04_scanf/3_checking_retval/hiew_1.png}
\caption{Hiew: \main Funktion}
\label{fig:scanf_ex3_hiew_1}
\end{figure}

Hiew erkennt \ac{ASCIIZ} Strings und die Namen importierter Funktionen und zeigt diese an.

\clearpage
Setzen wir den Cursor auf die Adresse \TT{.00401027}, an der sich der \TT{JNZ} Befehl, den wir umgehen müssen, befindet,
drücken F3 und fügen dann \q{9090} (zwei \ac{NOPS}s) ein. 

\begin{figure}[H]
\centering
\myincludegraphics{patterns/04_scanf/3_checking_retval/hiew_2.png}
\caption{Hiew: ersetzen von \TT{JNZ} durch zwei \ac{NOP}s}
\label{fig:scanf_ex3_hiew_2}
\end{figure}
Wir drücken F9 (update). Die Executable wird gespeichert und verhält sich wie gewünscht.

Zwei \ac{NOP}s sind wahrscheinlich nicht der ästhetischte Ansatz. Ein anderer Weg die Executable zu patchen besteht
darin, das zweite Byte des Opcodes (den \gls{jump offset}) auf 0 zu setzen, sodass \TT{JNZ} immer zum nächsten Befehl
springt.
% to be sync: Another way to patch this instruction is to write just 0 to the second byte of opcode (\gls{jump offset}), so that \TT{JNZ} will always jump to the next instruction.

Wir könnten auch das Gegenteil tun: das erste Byte durch \TT{EB} ersetzen und das zweite (\gls{jump offset})
unangetastet lassen. Wir würde einen unbedingten Sprung erhalten, der stets ausgeführt wird.
In diesem Fall würde unabhängig vom Input stets die Fehlermeldung ausgegeben.
}
\JA{\subsubsection{MSVC: x86}

アセンブリ出力(MSVC 2010)の内容は次のとおりです。

\lstinputlisting[style=customasmx86]{patterns/04_scanf/3_checking_retval/ex3_MSVC_x86.asm}

\myindex{x86!\Registers!EAX}
\gls{caller} 関数( \main )は \gls{callee} 関数( \scanf )の結果を必要とするため、
呼び出し先は \EAX レジスタに返します。

\myindex{x86!\Instructions!CMP}
我々は、\TT{CMP EAX, 1} (\emph{CoMPare})の指示によりそれをチェックします。つまり、 \EAX レジスタの値と1を比較します。

\myindex{x86!\Instructions!JNE}
\JNE 条件ジャンプが \CMP 命令の後に続きます。  \JNE は\emph{Jump if Not Equal}の略です。

したがって、 \EAX レジスタの値が1に等しくない場合、\ac{CPU}は \JNE オペランドに記述されているアドレス(この場合は\TT{\$LN2@main})に実行を渡します。
このアドレスに制御を渡すと、\ac{CPU}は \printf を引数\TT{What you entered? Huh?} で実行します 。
しかし、すべてがうまくいけば、条件付きジャンプは取られず、別の \printf 呼び出しが\TT{'You entered \%d...'}と\TT{x}の値を引数にとって実行されます。

\myindex{x86!\Instructions!XOR}
\myindex{\CLanguageElements!return}
この場合、2番目の \printf は実行されないため、その前に \JMP があります(無条件ジャンプ)。
2番目の \printf の後、\TT{戻り値0}を実装する\TT{XOR EAX, EAX}命令の直前に制御を渡します。

% FIXME internal \ref{} to x86 flags instead of wikipedia
\myindex{x86!\Registers!\Flags}
したがって、ある値を別の値と比較することは、\TT{通常}、 \CMP/\Jcc 命令ペアによって実装されると言えます\emph{cc}は\emph{条件コード}です。 
\CMP は2つの値を比較し、プロセッサフラグ\footnote{x86フラグは以下を参照: \href{http://en.wikipedia.org/wiki/FLAGS_register_(computing)}{wikipedia}}を設定します。 
\Jcc はこれらのフラグをチェックし、指定されたアドレスに制御を渡すかどうかを決定します。

\myindex{x86!\Instructions!CMP}
\myindex{x86!\Instructions!SUB}
\myindex{x86!\Instructions!JNE}
\myindex{x86!\Registers!ZF}
\label{CMPandSUB}
これは逆説的に聞こえるかもしれませんが、 \CMP 命令は実際には \SUB (減算)です。
すべての算術命令は、 \CMP だけでなくプロセッサフラグを設定します。 1と1を比較し、$1-1$が0であるため、ZFフラグが設定されます(最後の結果が0であることを意味します)。
オペランドが等しい場合を除いて、 \ZF は設定できません。  \JNE は \ZF フラグのみをチェックし、設定されていない場合にジャンプします。 
JNEは実際にはJNZ(\emph{Jump if Not Zero})の同義語です。アセンブラは、JNE命令とJNZ命令の両方を同じオペコードに変換します。
したがって、 \CMP 命令は \SUB 命令で置き換えることができ、 \SUB が最初のオペランドの値を変更するという違いを除けば、ほとんどすべてが問題ありません。 
\CMP \emph{は結果を保存しない \SUB ですが、フラグに影響}します。

\subsubsection{MSVC: x86: IDA}

\myindex{IDA}
IDAを実行してIDAを実行しようとします。 
ところで、初心者の方は、MSVCで\TT{/MD}オプションを使用することをお勧めします。つまり、
これらの標準関数はすべて実行可能ファイルにリンクされず、
代わりに\TT{MSVCR*.DLL}ファイルからインポートされます。 
したがって、どの標準関数が使用され、どこでどこが使用されているのかが分かりやすくなります。

\IDA のコードを分析する際には、自分自身(と他者)のためにノートを残すことが非常に役に立ちます。 
例えば、この例を分析すると、
エラーが発生した場合に \JNZ がトリガーされることがわかります。 
カーソルをラベルに移動して\q{n}を押し、\q{エラー}に名前を変更することができます。 
別のラベルを作成し、\q{終了}にします。 
以下が私の環境での結果です。

\lstinputlisting[style=customasmx86]{patterns/04_scanf/3_checking_retval/ex3.lst}

これで、コードを少し理解しやすくなりました。 
しかし、すべての命令についてコメントするのは良い考えではありません。

% FIXME draw button?
また、 \IDA の関数の一部を隠すこともできます。 
ブロックをマークするには、 \q{--} を数値パッドに入力し、代わりに表示するテキストを入力します。

2つのブロックを隠して名前を付けましょう。

\lstinputlisting[style=customasmx86]{patterns/04_scanf/3_checking_retval/ex3_2.lst}

% FIXME draw button?
以前に折りたたまれた部分を展開するには、数値パッドで\q{+}を使用します。

\clearpage
\q{スペース}を押すと、 \IDA が関数をグラフとして表示するのを見ることができます。

\begin{figure}[H]
\centering
\myincludegraphics{patterns/04_scanf/3_checking_retval/IDA.png}
\caption{Graph mode in IDA}
\label{fig:ex3_IDA_1}
\end{figure}

各条件ジャンプの後、緑と赤の2つの矢印があります。 
緑の矢印は、ジャンプがトリガされた場合に実行されるブロックを指し、そうでない場合は赤を指します。

\clearpage
このモードでノードを折りたたみ、名前を付けることもできます([q{グループノード})。
3つのブロックでやってみましょう。

\begin{figure}[H]
\centering
\myincludegraphics{patterns/04_scanf/3_checking_retval/IDA2.png}
\caption{Graph mode in IDA with 3 nodes folded}
\label{fig:ex3_IDA_2}
\end{figure}

それは非常に便利です。
リバースエンジニアの仕事(および他の研究者の仕事)の非常に重要な部分は、彼らが扱う情報の量を減らすことであると言えます。

\clearpage
\subsubsection{MSVC: x86 + \olly}

\olly でプログラムをハックしようとして、 \scanf が常にエラーなく動作するようにしましょう。 
ローカル変数のアドレスが \scanf に渡されると、
変数には最初にいくつかのランダムなガベージが含まれます。この場合、\TT{0x6E494714}です。

\begin{figure}[H]
\centering
\myincludegraphics{patterns/04_scanf/3_checking_retval/olly_1.png}
\caption{\olly: passing variable address into \scanf}
\label{fig:scanf_ex3_olly_1}
\end{figure}

\clearpage
\scanf が実行されている間、コンソールでは、 \q{asdasd}のように、数字ではないものを入力します。 
\scanf は、エラーが発生したことを示す \EAX が0で終了します。

また、スタック内のローカル変数をチェックし、変更されていないことに注意してください。 
実際、 \scanf は何を書いていますか? 
ゼロを返す以外は何もしませんでした。

私たちのプログラムを\q{ハックする}ようにしましょう。 
\EAX を右クリックし、
オプションの中に\q{Set to 1}があります。 
これが必要なものです。

\EAX には1があるので、以下のチェックを意図どおりに実行し、
\printf は変数の値をスタックに出力します。 

プログラム(F9)を実行すると、コンソールウィンドウで次のように表示されます。

\lstinputlisting[caption=console window]{patterns/04_scanf/3_checking_retval/console.txt}

実際、1850296084はスタック(\TT{0x6E494714})の数値を10進表現したものです!


\clearpage
\subsubsection{MSVC: x86 + Hiew}
\myindex{Hiew}

これは、実行可能ファイルのパッチ適用の簡単な例としても使用できます。 
実行可能ファイルにパッチを適用して、入力内容にかかわらずプログラムが常に入力を出力するようにすることがあります。

実行可能ファイルが外部の\TT{MSVCR*.DLL}(つまり\TT{/MD}オプション付き)
\footnote{\q{ダイナミックリンク}とも呼ばれる}
に対してコンパイルされていると仮定すると、\TT{.text}セクションの先頭に \main 関数があります。 
Hiewで実行可能ファイルを開き、\TT{.text}セクションの先頭を見つけましょう(Enter、F8、F6、Enter、Enter)。 

以下のように見えます。

\begin{figure}[H]
\centering
\myincludegraphics{patterns/04_scanf/3_checking_retval/hiew_1.png}
\caption{Hiew: \main function}
\label{fig:scanf_ex3_hiew_1}
\end{figure}

Hiewは\ac{ASCIIZ}文字列を検索し、インポートされた関数の名前と同様に表示します。

\clearpage
カーソルを\TT{.00401027}番地(ここでバイパスする \JNZ 命令がある場所)に移動し、F3を押し、\q{9090}(2つの\ac{NOP}を意味する)と入力します。

\begin{figure}[H]
\centering
\myincludegraphics{patterns/04_scanf/3_checking_retval/hiew_2.png}
\caption{Hiew: replacing \TT{JNZ} with two \ac{NOP}s}
\label{fig:scanf_ex3_hiew_2}
\end{figure}

その後、F9(更新)を押します。 これで、実行可能ファイルがディスクに保存されます。 私たちが望むように動作します。

2つの\ac{NOP}はおそらく最も美しいアプローチではありません。 
この命令をパッチする別の方法は、第2オペコードバイトに0を書き込むことであり( \gls{jump offset} )、 
\JNZ は常に次の命令にジャンプします。
% to be sync: Another way to patch this instruction is to write just 0 to the second byte of opcode (\gls{jump offset}), so that \TT{JNZ} will always jump to the next instruction.

また、最初のバイトを\TT{EB}で置き換え、2番目のバイト( \gls{jump offset} )には触れないでください。 
私たちは常に無条件のジャンプを得るでしょう。 
この場合、エラーメッセージは入力に関係なく毎回表示されます。
}
\PL{\subsubsection{MSVC: x86}

Poniżej wynik kompilacji pod MSVC 2010:

\lstinputlisting[style=customasmx86]{patterns/04_scanf/3_checking_retval/ex3_MSVC_x86.asm}

\myindex{x86!\Registers!EAX}
\glslink{caller}{Funkcja wywołująca} (\main) potrzebuje rezultatu zwróconego przez \glslink{callee}{funkcję wywoływaną} (\scanf),
więc \glslink{callee}{funkcja wywoływana} zwraca go za pomocą rejestru \EAX.

\myindex{x86!\Instructions!CMP}
Rezultat sprawdzamy za pomocą instrukcji \TT{CMP EAX, 1} (\emph{CoMPare}) ~--- porównujemy wartość w rejestrze \EAX z liczbą 1.

\myindex{x86!\Instructions!JNE}
Instrukcja \JNE to skok warunkowy, następujący po \CMP. \JNE oznacza \emph{Jump if Not Equal}.

Jeśli wartość w \EAX jest różna od 1, \ac{CPU} przekaże sterowanie pod adres z operandu instrukcji \JNE, w naszym przypadku jest to \TT{\$LN2@main}.
Przekazanie sterowania pod ten adres oznacza wykonanie funkcji \printf z argumentem \TT{What you entered? Huh?}.
Jeśli natomiast \scanf zakończyła się sukcesem i wartość w EAX jest równa 1, skok warunkowy nie zostanie wykonany i kolejno zostanie wywołana funkcja \printf, z dwoma argumentami:\\
\TT{'You entered \%d...'} i wartością \TT{x}.

\myindex{x86!\Instructions!XOR}
\myindex{\CLanguageElements!return}
W tym drugin przypadku - gdy \scanf zakończyła się poprawnie - nie ma potrzeby wykonywać drugiego wywołania funkcji \printf, stąd przed wywołaniem znajduje się instrukcja \JMP (skok bezwarunkowy).
Instrukcja przekazuje sterowanie w miejsce za drugim wywołaniem \printf, ale przed instrukcją \TT{XOR EAX, EAX}, która realizuje \TT{return 0}.

% FIXME internal \ref{} to x86 flags instead of wikipedia
\myindex{x86!\Registers!\Flags}
Można powiedzieć, że porówywanie dwóch wartości jest zwykle realizowane przez parę instrukcji \CMP/\Jcc, gdzie \emph{cc} oznacza \emph{condition code} (\emph{kod warunku}).
\CMP porównuje dwie wartości i ustawia flagę procesora \footnote{rejestr FLAGS, więcej o tym przeczytasz pod adresem: \href{http://en.wikipedia.org/wiki/FLAGS_register_(computing)}{wikipedia}.}.
\Jcc sprawdza te flagi i decyduje czy przekazać sterowanie pod podany adres.

\myindex{x86!\Instructions!CMP}
\myindex{x86!\Instructions!SUB}
\myindex{x86!\Instructions!JNE}
\myindex{x86!\Registers!ZF}
\label{CMPandSUB}
Zabrzmi to paradoksalnie, ale instrukcja \CMP to tak na prawdę \SUB (subtract - \emph{odejmij}).
Nie tylko \CMP, ale wszystkie instrukcje arytmetyczne modyfikują flagi procesora.
Jeśli porównamy 1 z 1, $1-1$ daje 0, więc flaga \ZF zostanie ustawiona.
W żadnym innym przypadku flaga \ZF nie zostanie ustawiona, poza tym gdy operandy są sobie równe.
\JNE sprawdza tylko flagę \ZF i wykonuje skok, jeśli nie jest ustawiona. \JNE jest synonimem \JNZ (\emph{Jump if Not Zero}).
Asembler tłumaczy zarówno \JNE jak i \JNZ na ten sam kod operacji (opcode).
Instrukcja \CMP może być zastąpiona przez \SUB i prawie wszystko powinno działać poprawnie, poza tym, że \SUB zmieni wartość pierwszego operandu na wynik operacji odejmowania.
\CMP to \emph{SUB bez zapisywania wyniku operacji, ale ze zmianą flag}.

\subsubsection{MSVC: x86: IDA}

\myindex{IDA}
Nadszedł czas na uruchomienie programu \IDA i pokazanie jego możliwości.
Przy okazji, początkującym pomoże ustawienie opcji \TT{/MD} w MSVC, co spowoduje, że wszystkie
funkcji biblioteki standardowej nie będą statycznie zlinkowane do pliku wykonywalnego,
ale zostaną zaimportowane z \TT{MSVCR*.DLL} podczas wykonania.
Dzięki temu łatwiej będzie zobaczyć, które funkcje z biblioteki standardowej zostały użyte i gdzie.

Podczas analizy kodu w programie \IDA warto dla siebie (i innych) robić notatki.
W tym przypadku widzimy, że skok \TT{JNZ} wykona się w przypadku błędu.
Można przesunąć kursor do etykiety, nacisnąć \q{n} i zmienić nazwę na \q{error}.
Zmienimy również nazwę kolejnej etykiety na \q{exit}.

Poniżej listing po zmianiach nazw:

\lstinputlisting[style=customasmx86]{patterns/04_scanf/3_checking_retval/ex3.lst}

Te drobne modyfikacje ułatwiły zrozumienie kodu, jednak nie warto przesadzać i komentować każdej instrukcji.

% FIXME draw button?
W \IDA nożesz również ukryć (zwinąć) kod wybranej funkcji.
Zaznacz blok kodu, wciśnij Ctrl-\q{--} na klawiaturze numerycznej i wpisz tekst, który ma
zostać wyświetlony zamiast kodu.

Ukryjmy dwa bloki kodu i nadajmy im nazwy:

\lstinputlisting[style=customasmx86]{patterns/04_scanf/3_checking_retval/ex3_2.lst}

% FIXME draw button?
By rozwinąć poprzednio zwinięte fragmenty, użyj Ctrl-\q{+} na klawiaturze numerycznej.

\clearpage
Po naciśnięciu \q{spacji} zobaczymy reprezentację funkcji w postaci grafu.

\begin{figure}[H]
\centering
\myincludegraphics{patterns/04_scanf/3_checking_retval/IDA.png}
\caption{Tryb grafu w IDA}
\label{fig:ex3_IDA_1}
\end{figure}

Z każdego skoku warunkowego wychodzą dwie strzałki: zielona i czerwona.
Zielona wskazuje blok, który się wykona w przypadku wykonania skoku, a czerwona - blok, który się wykona, gdy do skoku nie dojdzie.

\clearpage
W tym trybie można zwinąć węzły i nadać nazwę tak stworzonej \q{grupie węzłów}.
Zróbmy to dla 3 bloków:

\begin{figure}[H]
\centering
\myincludegraphics{patterns/04_scanf/3_checking_retval/IDA2.png}
\caption{Tryb grafu w IDA przy 3 zwiniętych węzłach}
\label{fig:ex3_IDA_2}
\end{figure}

Jest to dość użyteczne.
Można powiedzieć, że istotną częścią pracy osoby zajmującej się inżynierią wsteczną (a także każdego innego badacza) jest ograniczenie ilości informacji.

\clearpage
\subsubsection{MSVC: x86 + \olly}

Spróbujmy zhackować nasz program w \olly, zmuszając go, by uznał, że funkcja \scanf wykonała się bez błędów.
Kiedy adres zmiennej lokalnej jest przekazywany do \scanf,
zmienna początkowo zawiera przypadkową wartość, w tym wypadku \TT{0x6E494714}:

\begin{figure}[H]
\centering
\myincludegraphics{patterns/04_scanf/3_checking_retval/olly_1.png}
\caption{\olly: przekazywanie adresu zmiennej do \scanf}
\label{fig:scanf_ex3_olly_1}
\end{figure}

\clearpage
Kiedy wykonywana jest funkcja \scanf , w konsoli wpiszmy coś, co z pewnością nie jest liczbą, na przykład \q{asdasd}.
\scanf kończy działanie z 0 w \EAX, co wskazuje na wystąpienie błędu:

\begin{figure}[H]
\centering
\myincludegraphics{patterns/04_scanf/3_checking_retval/olly_2.png}
\caption{\olly: \scanf zwraca błąd}
\label{fig:scanf_ex3_olly_2}
\end{figure}

Możemy sprawdzić wartość zmiennej lokalnej na stosie i zauważyć, że się ona nie zmieniła.
W rzeczy samej, dlaczego funkcja \scanf miałaby cokolwiek tam zapisać?
Jej wykonanie nie spowodowało nic, poza zwróceniem zera.

Spróbujmy \q{zhackować} nasz program.
Kliknij prawym przyciskiem na \EAX,
wśród opcji znajduje się \q{Set to 1} (\emph{ustaw na 1}).
To jest to, czego szukamy.

Mamy teraz 1 w \EAX, a więc kolejne sprawdzenie powinno się wykonać zgodnie z oczekiwaniami i
\printf powinna wyświetlić wartość zmiennej ze stosu.

Po wznowieniu wykonania programu (F9) widzimy następujący efekt w oknie konsoli:

\lstinputlisting[caption=console window]{patterns/04_scanf/3_checking_retval/console.txt}

1850296084 to postać dziesiętna liczby, którą widzieliśmy na stosie (\TT{0x6E494714})!


\clearpage
\subsubsection{MSVC: x86 + Hiew}
\myindex{Hiew}

Na tym przykładzie pokażemy proste \emph{poprawianie} plików wykonalnych.
Tak zmodyfikujmy program, by zawsze wypisał wejście wprowadzony przez użytkownika, niezależnie od jego treści.

Zakładając, że plik wykonywalny jest linkowany dynamicznie z \TT{MSVCR*.DLL} (kompilacja z opcją \TT{/MD}),
zobaczymy funkcję \main na początku sekcji \TT{.text}.
Otwórzmy plik w Hiew i znajdźmy początek sekcji \TT{.text} (Enter, F8, F6, Enter, Enter).

Widzimy:

\begin{figure}[H]
\centering
\myincludegraphics{patterns/04_scanf/3_checking_retval/hiew_1.png}
\caption{Hiew: funkcja \main}
\label{fig:scanf_ex3_hiew_1}
\end{figure}

Hiew znajduje łańuchy znaków \ac{ASCIIZ} i je wyświetla, tak samo dzieje się również z nazwami zaimportowanych funkcji.

\clearpage
Przesuń kursor do adresu \TT{.00401027} (znajduje się tam instrukcja \TT{JNZ}, którą musimy ominąć), naciśnij F3 i wpisz \q{9090} (co oznacza dwie instrukcje \ac{NOP}):

\begin{figure}[H]
\centering
\myincludegraphics{patterns/04_scanf/3_checking_retval/hiew_2.png}
\caption{Hiew: zastąpienie \TT{JNZ} przez dwie instrukcje \ac{NOP}}
\label{fig:scanf_ex3_hiew_2}
\end{figure}

Następnie naciśnij F9 (update). Plik wykonywalny został zapisany na dysk i będzie się zachowywał zgodnie z naszymi oczekiwaniami.

Dwie instrukcje \ac{NOP} nie są najbardziej eleganckim rozwiązaniem.
Innym sposobem byłoby poprawienie instrukcji przez zapisanie 0 do drugiego bajtu kodu operacji (\glslink{jump offset}{przesunięcie skoku}),
by \TT{JNZ} zawsze skakała do kolejnej instrukcji.

Można też program zmodyfikować w drugą stronę: zastąpić pierwszy bajt przez \TT{EB}, nie zmieniając drugiego bajtu (\glslink{jump offset}{przesunięcie skoku}).
Otrzymamy wtedy skok bezwarunkowy, który zawsze będzie zachodził, przez co za każdym razem dostaniemy wiadomość o błędzie.
}

\EN{\subsubsection{x64}
\label{subsec:popcnt}

Let's modify the example slightly to extend it to 64-bit:

\lstinputlisting[label=popcnt_x64_example,style=customc]{patterns/14_bitfields/4_popcnt/shifts64.c}

\myparagraph{\NonOptimizing GCC 4.8.2}

So far so easy.

\lstinputlisting[caption=\NonOptimizing GCC 4.8.2,style=customasmx86]{patterns/14_bitfields/4_popcnt/shifts64_GCC_O0_EN.s}

\myparagraph{\Optimizing GCC 4.8.2}

\lstinputlisting[caption=\Optimizing GCC 4.8.2,numbers=left,label=shifts64_GCC_O3,style=customasmx86]{patterns/14_bitfields/4_popcnt/shifts64_GCC_O3_EN.s}

This code is terser, but has a quirk.

In all examples that we see so far, we were incrementing the \q{rt} value after comparing a specific bit,
but the code here increments \q{rt} before (line 6), writing the new value into register \EDX .
Thus, if the last bit is 1, the \CMOVNE\footnote{Conditional MOVe if Not Equal} instruction
(which is a synonym for \CMOVNZ\footnote{Conditional MOVe if Not Zero}) \emph{commits} 
the new value of \q{rt}
by moving \EDX (\q{proposed rt value}) into \EAX (\q{current rt} to be returned at the end).

Hence, the incrementing is performed at each step of loop, i.e., 64 times, without any relation to the input value.

The advantage of this code is that it contain only one conditional jump (at the end of the loop) instead of 
two jumps (skipping the \q{rt} value increment and at the end of loop).
And that might work faster on the modern CPUs with branch predictors: \myref{branch_predictors}.

\label{FATRET}
\myindex{x86!\Instructions!FATRET}
The last instruction is \INS{REP RET} (opcode \TT{F3 C3}) 
which is also called \INS{FATRET} by MSVC.
This is somewhat optimized version of \RET, 
which is recommended by AMD to be placed at the end of function, if \RET goes right after conditional jump: 
\InSqBrackets{\AMDOptimization p.15}
\footnote{More information on it: \url{http://repzret.org/p/repzret/}}.

\myparagraph{\Optimizing MSVC 2010}

\lstinputlisting[caption=\Optimizing MSVC 2010,style=customasmx86]{patterns/14_bitfields/4_popcnt/MSVC_2010_x64_Ox_EN.asm}

\myindex{x86!\Instructions!ROL}
Here the \ROL instruction is used instead of 
\SHL, which is in fact \q{rotate left} 
instead of \q{shift left},
but in this example it works just as \TT{SHL}.

You can read more about the rotate instruction here: \myref{ROL_ROR}.

\Reg{8} here is counting from 64 to 0.
It's just like an inverted $i$.

Here is a table of some registers during the execution:

\begin{center}
\begin{tabular}{ | l | l | }
\hline
\HeaderColor RDX & \HeaderColor R8 \\
\hline
0x0000000000000001 & 64 \\
\hline
0x0000000000000002 & 63 \\
\hline
0x0000000000000004 & 62 \\
\hline
0x0000000000000008 & 61 \\
\hline
... & ... \\
\hline
0x4000000000000000 & 2 \\
\hline
0x8000000000000000 & 1 \\
\hline
\end{tabular}
\end{center}

\myindex{x86!\Instructions!FATRET}
At the end we see the \INS{FATRET} instruction, which was explained here: \myref{FATRET}.

\myparagraph{\Optimizing MSVC 2012}

\lstinputlisting[caption=\Optimizing MSVC 2012,style=customasmx86]{patterns/14_bitfields/4_popcnt/MSVC_2012_x64_Ox_EN.asm}

\myindex{\CompilerAnomaly}
\label{MSVC2012_anomaly}
\Optimizing MSVC 2012 does almost the same job as 
optimizing MSVC 2010, but somehow, it generates two identical loop bodies and the loop count is now 32 instead of 64.

To be honest, it's not possible to say why. Some optimization trick? Maybe it's better for the loop body to be slightly 
longer?

Anyway, such code is relevant here to show that sometimes the compiler output may be really weird and 
illogical, but perfectly working.

}
\RU{\subsection{x64}

\myindex{x86-64}
В x86-64 всё немного иначе, здесь аргументы функции (4 или 6) передаются через регистры, 
а \gls{callee} из читает их из регистров, а не из стека.

\subsubsection{MSVC}

\Optimizing MSVC:

\lstinputlisting[caption=\Optimizing MSVC 2012 x64,style=customasmx86]{patterns/05_passing_arguments/x64_MSVC_Ox_RU.asm}

Как видно, очень компактная функция \ttf берет аргументы прямо из регистров.

Инструкция \LEA используется здесь для сложения чисел. 
Должно быть компилятор посчитал, что это будет эффективнее использования \TT{ADD}.

\myindex{x86!\Instructions!LEA}
В самой \main{} \LEA{} также используется для подготовки первого и третьего аргумента: должно быть,
компилятор решил, что \LEA{} будет работать здесь быстрее, чем загрузка значения в регистр при помощи \MOV.

Попробуем посмотреть вывод неоптимизирующего MSVC:

\lstinputlisting[caption=MSVC 2012 x64,style=customasmx86]{patterns/05_passing_arguments/x64_MSVC_IDA_RU.asm}

Немного путанее: все 3 аргумента из регистров зачем-то сохраняются в стеке.

\myindex{Shadow space}
\label{shadow_space}
Это называется \q{shadow space} \footnote{\href{http://msdn.microsoft.com/en-us/library/zthk2dkh(v=vs.80).aspx}{MSDN}}: 
каждая функция в Win64 может (хотя и не обязана) сохранять значения 4-х регистров там.

Это делается по крайней мере из-за двух причин: 
1) в большой функции отвести целый регистр (а тем более 4 регистра) для входного аргумента 
слишком расточительно, так что к нему будет обращение через стек;

2) отладчик всегда знает, где найти аргументы функции в момент останова
\footnote{\href{http://msdn.microsoft.com/en-us/library/ew5tede7(v=VS.90).aspx}{MSDN}}.

Так что, какие-то большие функции могут сохранять входные аргументы в \q{shadows space} 
для использования в будущем, а небольшие функции, как наша, могут этого и не делать.

Место в стеке для \q{shadow space} выделяет именно \gls{caller}.

\subsubsection{GCC}

\Optimizing GCC также делает понятный код:

\lstinputlisting[caption=\Optimizing GCC 4.4.6 x64,style=customasmx86]{patterns/05_passing_arguments/x64_GCC_O3_RU.s}

\NonOptimizing GCC:

\lstinputlisting[caption=GCC 4.4.6 x64,style=customasmx86]{patterns/05_passing_arguments/x64_GCC_RU.s}

\myindex{Shadow space}
В соглашении о вызовах System V *NIX (\SysVABI) нет \q{shadow space}, но \gls{callee} тоже иногда
должен сохранять где-то аргументы, потому что, опять же, регистров может и не хватить на все действия.
Что мы здесь и видим.

\subsubsection{GCC: uint64\_t вместо int}

Наш пример работал с 32-битным \Tint, поэтому использовались 32-битные части регистров с префиксом \TT{E-}.

Его можно немного переделать, чтобы он заработал с 64-битными значениями:

\lstinputlisting[style=customc]{patterns/05_passing_arguments/ex64.c}

\lstinputlisting[caption=\Optimizing GCC 4.4.6 x64,style=customasmx86]{patterns/05_passing_arguments/ex64_GCC_O3_IDA_RU.asm}

Собствено, всё то же самое, только используются регистры \emph{целиком}, с префиксом \TT{R-}.

}
\FR{\subsubsection{x64: 8 arguments entier}

\myindex{x86-64}
\label{example_printf8_x64}
Pour voir comment les autres arguments sont passés par la pile, changeons encore
notre exemple en augmentant le nombre d'arguments à 9 (chaîne de format de
\printf + 8 variables \Tint):

\lstinputlisting[style=customc]{patterns/03_printf/2.c}

\myparagraph{MSVC}

Comme il a déjà été mentionné, les 4 premiers arguments sont passés par les registres
\RCX, \RDX, \Reg{8}, \Reg{9} sous Win64, tandis les autres le sont---par la pile.
C'est exactement de que l'on voit ici.
Toutefois, l'instruction \MOV est utilisée ici à la place de \PUSH, donc les valeurs
sont stockées sur la pile d'une manière simple.

\lstinputlisting[caption=MSVC 2012 x64,style=customasmx86]{patterns/03_printf/x86/2_MSVC_x64_FR.asm}

Le lecteur observateur pourrait demander pourquoi 8 octets sont alloués sur la
pile pour les valeurs \Tint, alors que 4 suffisent?
Oui, il faut se rappeler: 8 octets sont alloués pour tout type de données plus
petit que 64 bits.
Ceci est instauré pour des raisons de commodités: cela rend facile le calcul
de l'adresse de n'importe quel argument.
En outre, ils sont tous situés à des adresses mémoires alignées.
Il en est de même dans les environnements 32-bit: 4 octets sont réservés pour tout
types de données.

% also for local variables?

\myparagraph{GCC}

Le tableau est similaire pour les OS x86-64 *NIX, excepté que les 6 premiers arguments
sont passés par les registres \RDI, \RSI, \RDX, \RCX, \Reg{8}, \Reg{9}.
Tout les autres---par la pile.
GCC génère du code stockant le pointeur de chaîne dans \EDI au lieu de \RDI{}---nous
l'avons noté précédemment:
\myref{hw_EDI_instead_of_RDI}.

Nous avions également noté que le registre \EAX a été vidé avant l'appel à
\printf: \myref{SysVABI_input_EAX}.

\lstinputlisting[caption=GCC 4.4.6 x64 \Optimizing,style=customasmx86]{patterns/03_printf/x86/2_GCC_x64_FR.s}

\myparagraph{GCC + GDB}
\myindex{GDB}

Essayons cet exemple dans \ac{GDB}.

\begin{lstlisting}
$ gcc -g 2.c -o 2
\end{lstlisting}

\begin{lstlisting}
$ gdb 2
GNU gdb (GDB) 7.6.1-ubuntu
...
Reading symbols from /home/dennis/polygon/2...done.
\end{lstlisting}

\begin{lstlisting}[caption=mettons le point d'arrêt à \printf{,} et lançons]
(gdb) b printf
Breakpoint 1 at 0x400410
(gdb) run
Starting program: /home/dennis/polygon/2 

Breakpoint 1, __printf (format=0x400628 "a=%d; b=%d; c=%d; d=%d; e=%d; f=%d; g=%d; h=%d\n") at printf.c:29
29	printf.c: No such file or directory.
\end{lstlisting}

Les registres \RSI/\RDX/\RCX/\Reg{8}/\Reg{9} ont les valeurs attendues.
\RIP contient l'adresse de la toute première instruction de la fonction \printf.

\begin{lstlisting}
(gdb) info registers
rax            0x0	0
rbx            0x0	0
rcx            0x3	3
rdx            0x2	2
rsi            0x1	1
rdi            0x400628	4195880
rbp            0x7fffffffdf60	0x7fffffffdf60
rsp            0x7fffffffdf38	0x7fffffffdf38
r8             0x4	4
r9             0x5	5
r10            0x7fffffffdce0	140737488346336
r11            0x7ffff7a65f60	140737348263776
r12            0x400440	4195392
r13            0x7fffffffe040	140737488347200
r14            0x0	0
r15            0x0	0
rip            0x7ffff7a65f60	0x7ffff7a65f60 <__printf>
...
\end{lstlisting}

\begin{lstlisting}[caption=inspectons la chaîne de format]
(gdb) x/s $rdi
0x400628:	"a=%d; b=%d; c=%d; d=%d; e=%d; f=%d; g=%d; h=%d\n"
\end{lstlisting}

Affichons la pile avec la commande x/g cette fois---\emph{g} est l'unité pour \emph{giant words}, i.e., mots de 64-bit.

\begin{lstlisting}
(gdb) x/10g $rsp
0x7fffffffdf38:	0x0000000000400576	0x0000000000000006
0x7fffffffdf48:	0x0000000000000007	0x00007fff00000008
0x7fffffffdf58:	0x0000000000000000	0x0000000000000000
0x7fffffffdf68:	0x00007ffff7a33de5	0x0000000000000000
0x7fffffffdf78:	0x00007fffffffe048	0x0000000100000000
\end{lstlisting}

Le tout premier élément de la pile, comme dans le cas précédent, est la \ac{RA}.
3 valeurs sont aussi passées par la pile: 6, 7, 8.
Nous voyons également que 8 est passé avec les 32-bits de poids fort non
effacés: \GTT{0x00007fff00000008}.
C'est en ordre, car les valeurs sont d'un type \Tint, qui est 32-bit.
Donc, la partie haute du registre ou l'élément de la pile peuvent contenir des
\q{restes de données aléatoires}.

Si vous regardez où le contrôle reviendra après l'exécution de \printf,
\ac{GDB} affiche la fonction \main en entier:

\begin{lstlisting}[style=customasmx86]
(gdb) set disassembly-flavor intel
(gdb) disas 0x0000000000400576
Dump of assembler code for function main:
   0x000000000040052d <+0>:	push   rbp
   0x000000000040052e <+1>:	mov    rbp,rsp
   0x0000000000400531 <+4>:	sub    rsp,0x20
   0x0000000000400535 <+8>:	mov    DWORD PTR [rsp+0x10],0x8
   0x000000000040053d <+16>:	mov    DWORD PTR [rsp+0x8],0x7
   0x0000000000400545 <+24>:	mov    DWORD PTR [rsp],0x6
   0x000000000040054c <+31>:	mov    r9d,0x5
   0x0000000000400552 <+37>:	mov    r8d,0x4
   0x0000000000400558 <+43>:	mov    ecx,0x3
   0x000000000040055d <+48>:	mov    edx,0x2
   0x0000000000400562 <+53>:	mov    esi,0x1
   0x0000000000400567 <+58>:	mov    edi,0x400628
   0x000000000040056c <+63>:	mov    eax,0x0
   0x0000000000400571 <+68>:	call   0x400410 <printf@plt>
   0x0000000000400576 <+73>:	mov    eax,0x0
   0x000000000040057b <+78>:	leave  
   0x000000000040057c <+79>:	ret    
End of assembler dump.
\end{lstlisting}

Laissons se terminer l'exécution de \printf, exécutez l'instruction mettant \EAX
à zéro, et notez que le registre \EAX à une valeur d'exactement zéro.
\RIP pointe maintenant sur l'instruction \INS{LEAVE}, i.e, la pénultième de la
fonction \main.

\begin{lstlisting}
(gdb) finish
Run till exit from #0  __printf (format=0x400628 "a=%d; b=%d; c=%d; d=%d; e=%d; f=%d; g=%d; h=%d\n") at printf.c:29
a=1; b=2; c=3; d=4; e=5; f=6; g=7; h=8
main () at 2.c:6
6		return 0;
Value returned is $1 = 39
(gdb) next
7	};
(gdb) info registers
rax            0x0	0
rbx            0x0	0
rcx            0x26	38
rdx            0x7ffff7dd59f0	140737351866864
rsi            0x7fffffd9	2147483609
rdi            0x0	0
rbp            0x7fffffffdf60	0x7fffffffdf60
rsp            0x7fffffffdf40	0x7fffffffdf40
r8             0x7ffff7dd26a0	140737351853728
r9             0x7ffff7a60134	140737348239668
r10            0x7fffffffd5b0	140737488344496
r11            0x7ffff7a95900	140737348458752
r12            0x400440	4195392
r13            0x7fffffffe040	140737488347200
r14            0x0	0
r15            0x0	0
rip            0x40057b	0x40057b <main+78>
...
\end{lstlisting}
}
\PTBR{\subsubsection{x64: 8 argumentos}
% to be sync: \subsubsection{x64: 8 integer arguments}

\myindex{x86-64}
\label{example_printf8_x64}
Para ver como outros argumentos são passados pela pilha,
vamos mudar nosso exemplo novamente aumentando o numero de argumentos para 9 (\printf + 8 variáveis \Tint):

\lstinputlisting[style=customc]{patterns/03_printf/2.c}

\myparagraph{MSVC}

Como mencionado anteriormente, os primeiros 4 argumentos tem de ser passados pelos registradores \RCX, \RDX, \Reg{8}, \Reg{9} no Win64, enquanto o resto pela pilha.
Isso é exatamente o que veremos aqui.
Entretanto, a instrução \MOV, ao invés de \PUSH, é usada para preparar a pilha, portanto os valores são armazenados de uma maneira direta.

% TODO translate
\lstinputlisting[caption=MSVC 2012 x64,style=customasmx86]{patterns/03_printf/x86/2_MSVC_x64_EN.asm}

Um leitor observativo pode se indagar por que são alocados 8 bytes para valores int quando 4 já é suficiente?
Sim, mas lembre-se: 8 bytes são alocados para qualquer tipo de informação menor do que 64 bits.
Isso é estabelecido com o objetivo de ser conveniente: é mais facil calcular o endereço de um argumento arbitrário.
Além do mais, eles são alocados em endereços de memórias alinhados. Da mesma maneira no 32-bits: 4 bytes são reservados para todos os tipos de informação.

% also for local variables?

\PTBRph{}

}
\IT{\subsubsection{MSVC: x64}

\myindex{x86-64}

Poichè qui lavoriamo con variabili di tipo \Tint{}, che sono sempre a 32-bit in x86-64, vediamo che viene usata la parte a 32-bit dei registri (con il prefisso \TT{E-}).
Lavorando invece con i puntatori, sono usate la parti a 64-bit dei registri (con il prefisso \TT{R-}).

% TODO translate
\lstinputlisting[caption=MSVC 2012 x64,style=customasmx86]{patterns/04_scanf/3_checking_retval/ex3_MSVC_x64_EN.asm}

}
\JA{\subsection{x64}

\myindex{x86-64}

この話はx86-64では少し違っています。関数の引数(最初の4つまたは最初の6つ)
はレジスタに渡されます。つまり、\gls{callee}はレジスタからレジスタを読み込みます。

\subsubsection{MSVC}

\Optimizing MSVC:

\lstinputlisting[caption=\Optimizing MSVC 2012 x64,style=customasmx86]{patterns/05_passing_arguments/x64_MSVC_Ox_JA.asm}

見てわかるように、コンパクトな関数 \ttf はすべての引数をレジスタから取ります。

ここでの \LEA 命令は加算に使用され、
明らかにコンパイラは \TT{ADD} よりも速いと考えました。
\myindex{x86!\Instructions!LEA}

\LEA は、第1および第3の \ttf 引数を準備するために \main 関数でも使用されます。
コンパイラは、 \MOV 命令を使用してレジスタに値をロードする通常の方法よりも速く動作すると判断する必要があります。

非最適化MSVCの出力を見てみましょう。

\lstinputlisting[caption=MSVC 2012 x64,style=customasmx86]{patterns/05_passing_arguments/x64_MSVC_IDA_JA.asm}

レジスタからの3つの引数は何らかの理由でスタックに保存されるため、ややこしいことになっています。

\myindex{Shadow space}
\label{shadow_space}
これは "シャドウスペース"と呼ばれます。
\footnote{\href{http://msdn.microsoft.com/en-us/library/zthk2dkh(v=vs.80).aspx}{MSDN}}
すべてのWin64は、そこにある4つのレジスタ値をすべて保存することができます(必須ではありません)。
これは2つの理由で行われます。
1)入力引数にレジスタ全体(または4つのレジスタ)を割り当てるのはあまりにも贅沢なので、スタック経由でアクセスされます。 
2)デバッガはブレークで関数の引数をどこに見つけるか常に認識しています。
\footnote{\href{http://msdn.microsoft.com/en-us/library/ew5tede7(v=VS.90).aspx}{MSDN}}

だから、大規模な関数の中には、実行中にそれらを使用したい場合、入力引数を \q{シャドウスペース}に保存することができますが、
私たちのような小さな関数ではそうでないかもしれません。

スタックに\q{シャドウスペース}を割り当てるのは\gls{caller}の責任です。

\subsubsection{GCC}

\Optimizing GCCはまあまあわかりやすいコードを生成します。

\lstinputlisting[caption=\Optimizing GCC 4.4.6 x64,style=customasmx86]{patterns/05_passing_arguments/x64_GCC_O3_JA.s}

\NonOptimizing GCC:

\lstinputlisting[caption=GCC 4.4.6 x64,style=customasmx86]{patterns/05_passing_arguments/x64_GCC_JA.s}

\myindex{Shadow space}

System V *NIX (\SysVABI)には\q{シャドースペース}の要件はありませんが、 \gls{callee}は
レジスタが不足している場合には引数をどこかに保存します。

\subsubsection{GCC: intの代わりのuint64\_t}

私たちの例は32ビットintで動作するため、32ビットのレジスタが使用されています(\TT{E-}が前に付いています)。 

64ビット値を使用するためには少し変更する必要があります。

\lstinputlisting[style=customc]{patterns/05_passing_arguments/ex64.c}

\lstinputlisting[caption=\Optimizing GCC 4.4.6 x64,style=customasmx86]{patterns/05_passing_arguments/ex64_GCC_O3_IDA_JA.asm}

コードは同じですが、今回は\emph{フルサイズ}のレジスタ(\TT{R-}が前に付いています)が使用されています。
}
\PL{\subsubsection{MSVC: x64}

\myindex{x86-64}

Pracujemy ze zmiennymi typu \Tint{}, które na x86-64 wciaż będą 32-bitowe, stąd w kodzie zobaczymy wykorzystanie 32-bitowych części rejestrów (z prefiksem \TT{E-}).
Jednak przy pracy ze wskaźnikami będą używane 64-bitowe rejestry, z prefiksem \TT{R-}.

\lstinputlisting[caption=MSVC 2012 x64,style=customasmx86]{patterns/04_scanf/3_checking_retval/ex3_MSVC_x64_PL.asm}

}


\EN{\input{patterns/04_scanf/3_checking_retval/ARM_EN}}
\RU{\subsubsection{ARM}

\myparagraph{\NonOptimizingKeilVI (\ARMMode)}

\lstinputlisting[style=customasmARM]{patterns/13_arrays/1_simple/simple_Keil_ARM_O0_RU.asm}

Тип \Tint требует 32 бита для хранения (или 4 байта),

так что для хранения 20 переменных типа \Tint, нужно 80 (\TT{0x50}) байт.

Поэтому инструкция \INS{SUB SP, SP, \#0x50} 
в прологе функции выделяет в локальном стеке под массив именно столько места.

И в первом и во втором цикле итератор цикла \var{i} будет постоянно находиться в регистре \Reg{4}.

\myindex{ARM!Optional operators!LSL}
Число, которое нужно записать в массив, вычисляется так: $i*2$, и это эквивалентно 
сдвигу на 1 бит влево,\\
так что инструкция \INS{MOV R0, R4,LSL\#1} делает это.

\myindex{ARM!\Instructions!STR}
\INS{STR R0, [SP,R4,LSL\#2]} записывает содержимое \Reg{0} в массив.
Указатель на элемент массива вычисляется так: \ac{SP} указывает на начало массива, \Reg{4} это $i$.

Так что сдвигаем $i$ на 2 бита влево, что эквивалентно умножению на 4 
(ведь каждый элемент массива занимает 4 байта) и прибавляем это к адресу начала массива.

\myindex{ARM!\Instructions!LDR}
Во втором цикле используется обратная инструкция\\
\INS{LDR R2, [SP,R4,LSL\#2]}.
Она загружает из массива нужное значение и указатель на него вычисляется точно так же.

\myparagraph{\OptimizingKeilVI (\ThumbMode)}

\lstinputlisting[style=customasmARM]{patterns/13_arrays/1_simple/simple_Keil_thumb_O3_RU.asm}

Код для Thumb очень похожий.
\myindex{ARM!\Instructions!LSLS}
В Thumb имеются отдельные инструкции для битовых сдвигов (как \TT{LSLS}), 
вычисляющие и число для записи в массив и адрес каждого элемента массива.

Компилятор почему-то выделил в локальном стеке немного больше места, 
однако последние 4 байта не используются.

\myparagraph{\NonOptimizing GCC 4.9.1 (ARM64)}

\lstinputlisting[caption=\NonOptimizing GCC 4.9.1 (ARM64),style=customasmARM]{patterns/13_arrays/1_simple/ARM64_GCC491_O0_RU.s}

}
\IT{\subsubsection{ARM}

\myparagraph{ARM: \OptimizingKeilVI (\ThumbMode)}

\lstinputlisting[caption=\OptimizingKeilVI (\ThumbMode),style=customasmARM]{patterns/04_scanf/3_checking_retval/ex3_ARM_Keil_thumb_O3.asm}

\myindex{ARM!\Instructions!CMP}
\myindex{ARM!\Instructions!BEQ}

Le due nuove istruzioni qui sono \CMP e \ac{BEQ}.

\CMP è analoga all'istruzione omonima in x86, sottrae uno degli argomenti dall'altro e aggiorna il conditional flags (se necessario).
% TODO: в мануале ARM $op1 + NOT(op2) + 1$ вместо вычитания

\myindex{ARM!\Registers!Z}
\myindex{x86!\Instructions!JZ}
\ac{BEQ} salta ad un altro indirizzo se gli operandi sono uguali, o se il risultato dell'ultima operazione era 0, oppure ancora se il flag Z è 1.
Si comporta come \JZ in x86.

Tutto il resto è semplice: il flusso di esecuzione si divide in due rami, e successivamente i due rami convergono al punto in cui 0 viene scritto in 
\Reg{0} come valore di ritorno di una funzione, infine la funzione termina.

\myparagraph{ARM64}

\lstinputlisting[caption=\NonOptimizing GCC 4.9.1 ARM64,numbers=left,style=customasmARM]{patterns/04_scanf/3_checking_retval/ARM64_GCC491_O0_IT.s}

\myindex{ARM!\Instructions!CMP}
\myindex{ARM!\Instructions!Bcc}
Il flusso di codice in questo caso si divide con l'uso della coppia di istruzioni \INS{CMP}/\INS{BNE} (Branch if Not Equal).

}
\FR{\input{patterns/04_scanf/3_checking_retval/ARM_FR}}
\JA{\input{patterns/04_scanf/3_checking_retval/ARM_JA}}
\DE{\subsubsection{ARM}

\myparagraph{ARM: \OptimizingKeilVI (\ThumbMode)}

\lstinputlisting[caption=\OptimizingKeilVI (\ThumbMode),style=customasmARM]{patterns/04_scanf/3_checking_retval/ex3_ARM_Keil_thumb_O3.asm}

\myindex{ARM!\Instructions!CMP}
\myindex{ARM!\Instructions!BEQ}
Die neuen Befehle hier sind \CMP und \ac{BEQ}.
\CMP verhält sich analog zum x86 Befehl gleichen Namens, er zieht ein Argument vom anderen ab und aktualisiert die
Flags, falls nötig.
% TODO: в мануале ARM $op1 + NOT(op2) + 1$ вместо вычитания

\myindex{ARM!\Registers!Z}
\myindex{x86!\Instructions!JZ}
\ac{BEQ} springt zu einer anderen Adresse, falls die beiden Operanden gleich waren oder das Ergebnis der letzten
Berechnung 0 war oder das Zero Flag auf 1 gesetzt ist. Der Befehl verhält sich wie \JZ in x86.

Der Rest ist einfach: der Ausführung verläuft in zwei Zweigen, dann vereinen sich die Zweige an der Stelle wieder, an
der 0 als Rückgabewert der Funktion in \Reg{0} geschrieben wird, und der Funktionsablauf endet. 

\myparagraph{ARM64}

\lstinputlisting[caption=\NonOptimizing GCC 4.9.1 ARM64,numbers=left,style=customasmARM]{patterns/04_scanf/3_checking_retval/ARM64_GCC491_O0_DE.s}

\myindex{ARM!\Instructions!CMP}
\myindex{ARM!\Instructions!Bcc}
Der Kontrollfluss wird in diesem Fall mithilfe von \INS{CMP}/\INS{BNE} (Branch if Not Equal) aufgespalten.

}

\subsubsection{MIPS}

\lstinputlisting[caption=\Optimizing GCC 4.4.5 (IDA),style=customasmMIPS]{patterns/04_scanf/3_checking_retval/MIPS_O3_IDA.lst}

\myindex{MIPS!\Instructions!BEQ}

\EN{\subsection{MIPS}

\lstinputlisting[caption=\Optimizing GCC 4.4.5,style=customasmMIPS]{patterns/05_passing_arguments/MIPS_O3_IDA_EN.lst}

The first four function arguments are passed in four registers prefixed by A-.

\myindex{MIPS!\Instructions!MULT}

There are two special registers in MIPS: HI and LO which are filled with the 64-bit result of the multiplication during the execution of the \TT{MULT} instruction.
\myindex{MIPS!\Instructions!MFLO}
\myindex{MIPS!\Instructions!MFHI}

These registers are accessible only by using the \TT{MFLO} and \TT{MFHI} instructions.
\TT{MFLO} here takes the low-part of the multiplication result and stores it into \$V0.
So the high 32-bit part of the multiplication result is dropped (the HI register content is not used).
Indeed: we work with 32-bit \Tint data types here.

\myindex{MIPS!\Instructions!ADDU}

Finally, \TT{ADDU} (\q{Add Unsigned}) adds the value of the third argument to the result.

\myindex{MIPS!\Instructions!ADD}
\myindex{MIPS!\Instructions!ADDU}
\myindex{Ada}
\myindex{Integer overflow}

There are two different addition instructions in MIPS: \TT{ADD} and \TT{ADDU}.
The difference between them is not related to signedness, but to exceptions. \TT{ADD} can raise an exception on overflow, which is sometimes useful\footnote{\url{http://blog.regehr.org/archives/1154}} and supported in Ada \ac{PL}, for instance.
\TT{ADDU} does not raise exceptions on overflow.

Since \CCpp does not support this, in our example we see \TT{ADDU} instead of \TT{ADD}.

The 32-bit result is left in \$V0.

\myindex{MIPS!\Instructions!JAL}
\myindex{MIPS!\Instructions!JALR}

There is a new instruction for us in \main: \TT{JAL} (\q{Jump and Link}). 

The difference between \INS{JAL} and \INS{JALR} is that a relative offset is encoded in the first instruction, 
while \INS{JALR} jumps to the absolute address stored in a register (\q{Jump and Link Register}).

Both \ttf and \main functions are located in the same object file, so the relative address of \ttf 
is known and fixed.
}
\RU{\subsection{MIPS}

\lstinputlisting[caption=\Optimizing GCC 4.4.5,style=customasmMIPS]{patterns/05_passing_arguments/MIPS_O3_IDA_RU.lst}

Первые 4 аргумента функции передаются в четырех регистрах с префиксами A-.

\myindex{MIPS!\Instructions!MULT}
В MIPS есть два специальных регистра: HI и LO, которые выставляются в 64-битный результат умножения
во время исполнения инструкции \TT{MULT}.

\myindex{MIPS!\Instructions!MFLO}
\myindex{MIPS!\Instructions!MFHI}
К регистрам можно обращаться только используя инструкции \TT{MFLO} и \TT{MFHI}.
Здесь \TT{MFLO} берет младшую часть результата умножения и записывает в \$V0.
Так что старшая 32-битная часть результата игнорируется (содержимое регистра HI не используется).
Действительно, мы ведь работаем с 32-битным типом \Tint.


\myindex{MIPS!\Instructions!ADDU}
И наконец, \TT{ADDU} (\q{Add Unsigned}~--- добавить беззнаковое) прибавляет значение третьего аргумента к результату.

\myindex{MIPS!\Instructions!ADD}
\myindex{MIPS!\Instructions!ADDU}
\myindex{Ada}
\myindex{Integer overflow}
В MIPS есть две разных инструкции сложения: \TT{ADD} и \TT{ADDU}.
На самом деле, дело не в знаковых числах, а в исключениях: \TT{ADD} может вызвать исключение
во время переполнения. Это иногда полезно\footnote{\url{http://blog.regehr.org/archives/1154}} и поддерживается,
например, в \ac{PL} Ada.

\TT{ADDU} не вызывает исключения во время переполнения.
А так как \CCpp не поддерживает всё это, мы видим здесь \TT{ADDU} вместо \TT{ADD}.

32-битный результат оставляется в \$V0.

\myindex{MIPS!\Instructions!JAL}
\myindex{MIPS!\Instructions!JALR}
В \main есть новая для нас инструкция: \TT{JAL} (\q{Jump and Link}). 
Разница между \INS{JAL} и \INS{JALR} в том, что относительное смещение кодируется в первой инструкции,
а \INS{JALR} переходит по абсолютному адресу, записанному в регистр (\q{Jump and Link Register}).

Обе функции \ttf и \main расположены в одном объектном файле, так что относительный адрес \ttf известен и фиксирован.

}
\IT{\scanf restituisce il risultato del suo lavoro nel registro \$V0. Ciò viene controllato all'indirizzo 0x004006E4
confrontando il valore in \$V0 con quello in \$V1 (1 era stato memorizzato in \$V1 precedentemente, a 0x004006DC).
\INS{BEQ} sta per \q{Branch Equal}.
Se i due valori sono uguali (cioè \scanf è terminata con successo), l'esecuzione salta all'indirizzo 0x0040070C.

}
\JA{\subsubsection{MIPS}

MIPSはいくつかのコプロセッサ(最大4個)をサポートすることができます。
そのうちの0番目\footnote{0から始まる}は特別な制御コプロセッサであり、最初のコプロセッサはFPUです。

ARMと同様に、MIPSコプロセッサはスタックマシンではなく、32個の32ビットレジスタ(\$F0-\$F31)を持ちます。
\myref{MIPS_FPU_registers}.

64ビットの \Tdouble 値を扱う必要がある場合、32ビットのFレジスタのペアが使用されます。

\lstinputlisting[caption=\Optimizing GCC 4.4.5 (IDA),style=customasmMIPS]{patterns/12_FPU/1_simple/MIPS_O3_IDA_JA.lst}

新しい命令は以下です。

\myindex{MIPS!\Instructions!LWC1}
\myindex{MIPS!\Instructions!DIV.D}
\myindex{MIPS!\Instructions!MUL.D}
\myindex{MIPS!\Instructions!ADD.D}
\begin{itemize}

\item \INS{LWC1}は32ビットワードを第1コプロセッサのレジスタにロードします(命令名は\q{1})。
\myindex{MIPS!\Pseudoinstructions!L.D}

一対の\INS{LWC1}命令を組み合わせて\INS{L.D}疑似命令にすることができます。

\item \INS{DIV.D}、 \INS{MUL.D}、 \INS{ADD.D}はそれぞれ除算、乗算、加算を行います
(接尾辞の\q{.D}は倍精度、\q{.S}は単精度を表します)

\end{itemize}

\myindex{MIPS!\Instructions!LUI}
\myindex{\CompilerAnomaly}
\label{MIPS_FPU_LUI}

また、奇妙なコンパイラの例外があります\INS{LUI}命令に疑問符がついています。 
\$V0 レジスタに64ビット定数の \Tdouble 型の一部をロードする理由を理解することは難しいです。 
これらの命令は何の効果もありません。 
% TODO did you try checking out compiler source code?
これについて何か知っているなら、著者に電子メール\footnote{\EMAILS}を送ってください。
}
\FR{\scanf renvoie le résultat de son traitement dans le registre \$V0. Il est testé à l'adresse 0x004006E4
en comparant la valeur dans \$V0 avec celle dans \$V1 (1 a été stocké dans \$V1 plus tôt, en 0x004006DC).
\INS{BEQ} signifie \q{Branch Equal} (branchement si égal).
Si les deux valeurs sont égales (i.e., succès), l'exécution saute à l'adresse 0x0040070C.
}



\subsubsection{\Exercise}

\myindex{x86!\Instructions!Jcc}
\myindex{ARM!\Instructions!Bcc}
Come possiamo vedere, le istruzioni \INS{JNE}/\INS{JNZ} possono essere scambiate con \INS{JE}/\INS{JZ} e viceversa.
(lo stesso vale per \INS{BNE} e \INS{BEQ}).
Ma se ciò avviene i blocchi base devono anch'essi essere scambiati. Provate a farlo in qualche esempio.

