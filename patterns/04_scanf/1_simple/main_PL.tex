\subsection{Prosty przykład}

\lstinputlisting[style=customc]{patterns/04_scanf/1_simple/ex1.c}

Używanie \scanf do interakcji z użytkownikiem nie jest dobrym pomysłem w dzisiejszych czasach, jednak mimo wszystko funkcja ta jest dobrym przykładem użycia wskaźnika na zmienną typu  \Tint.

\subsubsection{O wskaźnikach}
\myindex{\CLanguageElements!\Pointers}

Wskaźniki są jednym z podstawowych pojęć informatycznych. Często przekazywanie dużej tablicy, struktury lub obiektu do innej funkcji jest pamięciożerne, podczas gdy przekazanie samego adresu jest znacznie tańsze.

Kiedy chcesz wypisać tekst w konsoli, najprościej będzie wskazać jego adres w pamięci.

W dodatku, jeśli wywoływana funkcja potrzebuje zmodyfikować cokolwiek w dużej tablicy lub strukturze danych przekazanej jako parametr a następnie ją zwrócić, kopiowanie tylu danych byłoby prawie absurdalne. Dlatego najprościej będzie przekazać adres tej tablicy/struktury do wywoływanej funkcji i wtedy zmodyfikować to, co wymaga modyfikacji.

Wskaźnik w \CCpp---jest adresem pewnego miejsca w pamięci.

\myindex{x86-64}
W x86 adresy są reprezentowane przy pomocy 32-bitowych liczb (czyli 4 bajtowych), a w x86-64 jako liczby 64-bitowe (czyli 8 bajtowe). Przy okazji jest to powód dlaczego niektórych ludzi oburza przeskok na x86-64---wszystkie wskaźniki w architekturze x64 wymagają dwa razy więcej miejsca, włączając pamięć cache, która jest bardzo "kosztowna".

% TODO ... а делать разные версии memcpy для разных типов - абсурд
\myindex{\CStandardLibrary!memcpy()}
Można pracować jedynie z nietypowanymi wskaźnikami, wymaga to jednak nieco wysiłku, np. użycia funkcji z biblioteki standardowej C \TT{memcpy()}, która kopiuje blok z jednego miejsca w pamięci do drugiego. \TT{memcpy()} jako argumenty przyjmuje 2 wskaźniki typu \TT{void*}, co umożliwia kopiowanie dowolnych typów danych. Typy danych nie są istotne, znaczenie mają tylko rozmiary bloków pamięci.

Wskaźniki są także często używane kiedy funkcja potrzebuje zwrócić więcej niż jedną wartość (wrócę do tego później
~(\myref{label_pointers})
).

Funkcja \emph{scanf()} jest takim przypadkiem. Poza tym, że funkcja \scanf zwraca liczbę wczytanych wartości, to musi jeszcze je jakoś przekazać.

W \CCpp typ wskaźnika jest potrzebny tylko do sprawdzania typów podczas kompilacji.

W skompilowanym kodzie nie ma żadnej informacji jakiego typu są wskaźniki.
% TODO это сильно затрудняет декомпиляцию

\EN{\subsubsection{x86}
\myindex{Windows!Win32}

Win32 API example:

\begin{lstlisting}[style=customc]
	HANDLE fh;

	fh=CreateFile ("file", GENERIC_WRITE | GENERIC_READ, FILE_SHARE_READ, NULL, OPEN_ALWAYS, FILE_ATTRIBUTE_NORMAL, NULL);
\end{lstlisting}

We get (MSVC 2010):

\begin{lstlisting}[caption=MSVC 2010,style=customasmx86]
	push	0
	push	128		; 00000080H
	push	4
	push	0
	push	1
	push	-1073741824	; c0000000H
	push	OFFSET $SG78813
	call	DWORD PTR __imp__CreateFileA@28
	mov	DWORD PTR _fh$[ebp], eax
\end{lstlisting}

Let's take a look in WinNT.h:

\begin{lstlisting}[caption=WinNT.h,style=customc]
#define GENERIC_READ                     (0x80000000L)
#define GENERIC_WRITE                    (0x40000000L)
#define GENERIC_EXECUTE                  (0x20000000L)
#define GENERIC_ALL                      (0x10000000L)
\end{lstlisting}

Everything is clear,
GENERIC\_READ | GENERIC\_WRITE = 0x80000000 | 0x40000000 = 0xC0000000,
and that value is used as the second argument for the \TT{CreateFile()}\footnote{\href{http://msdn.microsoft.com/en-us/library/aa363858(VS.85).aspx}{msdn.microsoft.com/en-us/library/aa363858(VS.85).aspx}}function.

How would \TT{CreateFile()} check these flags?
\myindex{Windows!KERNEL32.DLL}

If we look in KERNEL32.DLL in Windows XP SP3 x86, we'll find this fragment of code in \TT{CreateFileW}:

\begin{lstlisting}[caption=KERNEL32.DLL (Windows XP SP3 x86),style=customasmx86]
.text:7C83D429     test    byte ptr [ebp+dwDesiredAccess+3], 40h
.text:7C83D42D     mov     [ebp+var_8], 1
.text:7C83D434     jz      short loc_7C83D417
.text:7C83D436     jmp     loc_7C810817
\end{lstlisting}

\myindex{x86!\Instructions!TEST}

Here we see the \TEST instruction, however it doesn't take the whole second argument,\\
but only the most significant byte (\TT{ebp+dwDesiredAccess+3}) and checks it for flag \TT{0x40}
(which implies the \TT{GENERIC\_WRITE} flag here).
\myindex{x86!\Instructions!AND}

\TEST is basically the same instruction as \AND, but without saving the result
(recall the fact \CMP is merely the same as \SUB, but without saving the result~(\myref{CMPandSUB})).

The logic of this code fragment is as follows:

\begin{lstlisting}[style=customc]
if ((dwDesiredAccess&0x40000000) == 0) goto loc_7C83D417
\end{lstlisting}

\myindex{x86!\Instructions!AND}
\myindex{x86!\Registers!ZF}

If \AND instruction leaves this bit, the \ZF flag is to be cleared and the 
\JZ conditional jump is not to be triggered.
The conditional jump is triggered only if the \TT{0x40000000} bit is absent in \TT{dwDesiredAccess} variable~---then the result of \AND is 0,
\ZF is to be set and the conditional jump is to be triggered.

Let's try GCC 4.4.1 and Linux:

\begin{lstlisting}[style=customc]
#include <stdio.h>
#include <fcntl.h>

void main()
{
	int handle;

	handle=open ("file", O_RDWR | O_CREAT);
};
\end{lstlisting}

We get:

\lstinputlisting[caption=GCC 4.4.1,style=customasmx86]{patterns/14_bitfields/1_check/check.asm}

\myindex{Linux!libc.so.6}
\myindex{syscall}

If we take a look in the \TT{open()} function in the \TT{libc.so.6} library, it is only a syscall:

\lstinputlisting[caption=open() (libc.so.6),style=customasmx86]{patterns/14_bitfields/1_check/tmp1.asm}

So, the bit fields for \TT{open()} are apparently checked somewhere in the Linux kernel.

Of course, it is easy to download both Glibc and the Linux kernel source code, 
but we are interested in understanding the matter without it.

So, as of Linux 2.6, when the \TT{sys\_open} syscall is called, control eventually passes to \TT{do\_sys\_open},
and from there---to the \TT{do\_filp\_open()} function (it's located in the kernel source tree in \TT{fs/namei.c}).

\newcommand{\URLREGPARM}{\href{http://www.ohse.de/uwe/articles/gcc-attributes.html\#func-regparm}{ohse.de/uwe/articles/gcc-attributes.html\#func-regparm}}

\myindex{fastcall}
\label{regparm}
N.B.  Aside from passing arguments via the stack,
there is also a method of passing some of them
via registers. This is also called fastcall~(\myref{fastcall}).
This works faster since CPU does not need to access the stack in memory to read argument values.
GCC has the option \emph{regparm}\footnote{\URLREGPARM},
through which it's possible to set the number of arguments that can be passed via registers.

\newcommand{\URLKERNELNEWB}{\href{http://kernelnewbies.org/Linux_2_6_20\#head-042c62f290834eb1fe0a1942bbf5bb9a4accbc8f}{kernelnewbies.org/Linux\_2\_6\_20\#head-042c62f290834eb1fe0a1942bbf5bb9a4accbc8f}}
\newcommand{\CALLINGHFILE}{arch/x86/include/asm/calling.h}

The Linux 2.6 kernel is compiled with \TT{-mregparm=3} option~\footnote{\URLKERNELNEWB}
\footnote{See also \TT{\CALLINGHFILE} file in kernel tree}.

What this means to us is that the first 3 arguments are to be passed via registers \EAX, \EDX and \ECX, 
and the rest via the stack. 
Of course, if the number of arguments is less than 3, only part of registers set is to be used.

So, let's download Linux Kernel 2.6.31, compile it in Ubuntu: \TT{make vmlinux}, open it in \IDA, 
and find the \TT{do\_filp\_open()} function. At the beginning, we see (the comments are mine):

\lstinputlisting[caption=do\_filp\_open() (linux kernel 2.6.31),style=customasmx86]{patterns/14_bitfields/1_check/check2_EN.asm}

GCC saves the values of the first 3 arguments in the local stack. 
If that wasn't done, the compiler would not touch these registers, 
and that would be too tight environment for the compiler's \gls{register allocator}.

Let's find this fragment of code:

\lstinputlisting[caption=do\_filp\_open() (linux kernel 2.6.31),style=customasmx86]{patterns/14_bitfields/1_check/check3.asm}

\TT{0x40}---is what the \TT{O\_CREAT} macro equals to.
\TT{open\_flag} gets checked for the presence of the \TT{0x40} bit, and if this bit is 1, 
the next \JNZ instruction is triggered.
}
\RU{\subsubsection{x86: 3 целочисленных аргумента}

\myparagraph{MSVC}

Компилируем при помощи MSVC 2010 Express, и в итоге получим:

\begin{lstlisting}[style=customasmx86]
$SG3830	DB	'a=%d; b=%d; c=%d', 00H

...

	push	3
	push	2
	push	1
	push	OFFSET $SG3830
	call	_printf
	add	esp, 16
\end{lstlisting}

Всё почти то же, за исключением того, что теперь видно, что аргументы для \printf заталкиваются в стек в обратном порядке: самый первый аргумент заталкивается последним.

Кстати, вспомним, что переменные типа \Tint в 32-битной системе, как известно, имеют ширину 32 бита, это 4 байта.

Итак, у нас всего 4 аргумента. $4*4 = 16$~--- именно 16 байт занимают в стеке указатель на строку плюс ещё 3 числа типа \Tint.

\myindex{x86!\Instructions!ADD}
\myindex{x86!\Registers!ESP}
\myindex{cdecl}
Когда при помощи инструкции \INS{ADD ESP, X} корректируется \glslink{stack pointer}{указатель стека} \ESP 
после вызова какой-либо функции, зачастую можно сделать вывод о том, сколько аргументов 
у вызываемой функции было, разделив X на 4.

Конечно, это относится только к cdecl-методу передачи аргументов через стек, и только для 32-битной среды.

См. также в соответствующем разделе о способах передачи аргументов через стек ~(\myref{sec:callingconventions}).

Иногда бывает так, что подряд идут несколько вызовов разных функций, но стек корректируется только один раз, после последнего вызова:

\begin{lstlisting}[style=customasmx86]
push a1
push a2
call ...
...
push a1
call ...
...
push a1
push a2
push a3
call ...
add esp, 24
\end{lstlisting}

Вот пример из реальной жизни:

\lstinputlisting[caption=x86,style=customasmx86]{patterns/03_printf/x86/add_example_RU.lst}

\clearpage
\myparagraph{MSVC и \olly}
\myindex{\olly}

Попробуем этот же пример в \olly.
Это один из наиболее популярных win32-отладчиков пользовательского режима.
Мы можем компилировать наш пример в MSVC 2012 
с опцией \GTT{/MD} что означает линковать с библиотекой \GTT{MSVCR*.DLL},
чтобы импортируемые функции были хорошо видны в отладчике.

Затем загружаем исполняемый файл в \olly.
Самая первая точка останова в \GTT{ntdll.dll}, нажмите F9 (запустить).
Вторая точка останова в \ac{CRT}-коде.
Теперь мы должны найти функцию \main.

Найдите этот код, прокрутив окно кода до самого верха (MSVC располагает функцию \main в самом начале секции кода): 

\begin{figure}[H]
\centering
\myincludegraphics{patterns/03_printf/x86/olly3_1.png}
\caption{\olly: самое начало функции \main}
\label{fig:printf3_olly_1}
\end{figure}

Кликните на инструкции \INS{PUSH EBP}, нажмите F2 (установка точки останова) и нажмите F9 (запустить).
Нам нужно произвести все эти манипуляции, чтобы пропустить \ac{CRT}-код, потому что нам он пока
не интересен.

\clearpage
Нажмите F8 (\stepover) 6 раз, т.е. пропустить 6 инструкций:

\begin{figure}[H]
\centering
\myincludegraphics{patterns/03_printf/x86/olly3_2.png}
\caption{\olly: перед исполнением \printf}
\label{fig:printf3_olly_2}
\end{figure}

Теперь \ac{PC} указывает на инструкцию \INS{CALL printf}.
\olly, как и другие отладчики, подсвечивает регистры со значениями, которые изменились.
Поэтому каждый раз когда мы нажимаем F8, \EIP изменяется и его значение подсвечивается красным.
\ESP также меняется, потому что значения заталкиваются в стек.\\
\\
Где находятся эти значения в стеке?
Посмотрите на правое нижнее окно в отладчике:

\begin{figure}[H]
\centering
\input{patterns/03_printf/x86/incl_olly3_stack}
\caption{\olly: стек с сохраненными значениями (красная рамка добавлена в графическом редакторе)}
\end{figure}

Здесь видно 3 столбца: адрес в стеке, значение в стеке и ещё дополнительный комментарий
от \olly. 
\olly может находить указатели на ASCII-строки в стеке, так что он показывает здесь \printf{}-строку.

Можно кликнуть правой кнопкой мыши на строке формата, кликнуть на \q{Follow in dump}
и строка формата появится в окне слева внизу, где всегда виден какой-либо участок памяти.
Эти значения в памяти можно редактировать.
Можно изменить саму строку формата, и тогда результат работы нашего примера будет другой.
В данном случае пользы от этого немного, но для упражнения это полезно,
чтобы начать чувствовать как тут всё работает.

\clearpage
Нажмите F8 (\stepover).

В консоли мы видим вывод:

\lstinputlisting{patterns/03_printf/x86/console.txt}

Посмотрим как изменились регистры и состояние стека: 

\begin{figure}[H]
\centering
\myincludegraphics{patterns/03_printf/x86/olly3_3.png}
\caption{\olly после исполнения \printf}
\label{fig:printf3_olly_3}
\end{figure}

Регистр \EAX теперь содержит \GTT{0xD} (13).
Всё верно: \printf возвращает количество выведенных символов.
Значение \EIP изменилось. Действительно, теперь здесь адрес инструкции после \INS{CALL printf}.
Значения регистров \ECX и \EDX также изменились.
Очевидно, внутренности функции \printf используют их для каких-то своих нужд.

Очень важно то, что значение \ESP не изменилось. И аргументы-значения в стеке также!
Мы ясно видим здесь и строку формата и соответствующие ей 3 значения, они всё ещё здесь.
Действительно, по соглашению вызовов \emph{cdecl}, вызываемая функция не возвращает \ESP назад.
Это должна делать вызывающая функция (\gls{caller}).

\clearpage
Нажмите F8 снова, чтобы исполнилась инструкция \INS{ADD ESP, 10}:

\begin{figure}[H]
\centering
\myincludegraphics{patterns/03_printf/x86/olly3_4.png}
\caption{\olly: после исполнения инструкции \INS{ADD ESP, 10}}
\label{fig:printf3_olly_4}
\end{figure}

\ESP изменился, но значения всё ещё в стеке!
Конечно, никому не нужно заполнять эти значения нулями или что-то в этом роде.
Всё что выше указателя стека (\ac{SP}) 
это \emph{шум} или \emph{\garbage{}} и не имеет особой ценности.
Было бы очень затратно по времени очищать ненужные элементы стека, к тому же, никому это и не нужно.

\myparagraph{GCC}

Скомпилируем то же самое в Linux при помощи GCC 4.4.1 и посмотрим на результат в \IDA:

\lstinputlisting[style=customasmx86]{patterns/03_printf/x86/x86_1.asm}

Можно сказать, что этот короткий код, созданный GCC, отличается от кода MSVC только способом помещения 
значений в стек.
Здесь GCC снова работает со стеком напрямую без \PUSH/\POP.

\myparagraph{GCC и GDB}
\myindex{GDB}

Попробуем также этот пример и в \ac{GDB} в Linux.

\GTT{-g} означает генерировать отладочную информацию в выходном исполняемом файле.

\begin{lstlisting}
$ gcc 1.c -g -o 1
\end{lstlisting}

\begin{lstlisting}
$ gdb 1
GNU gdb (GDB) 7.6.1-ubuntu
...
Reading symbols from /home/dennis/polygon/1...done.
\end{lstlisting}

\begin{lstlisting}[caption=установим точку останова на \printf]
(gdb) b printf
Breakpoint 1 at 0x80482f0
\end{lstlisting}

Запукаем.
У нас нет исходного кода функции, поэтому \ac{GDB} не может его показать.

\begin{lstlisting}
(gdb) run
Starting program: /home/dennis/polygon/1 

Breakpoint 1, __printf (format=0x80484f0 "a=%d; b=%d; c=%d") at printf.c:29
29	printf.c: No such file or directory.
\end{lstlisting}

Выдать 10 элементов стека. Левый столбец~--- это адрес в стеке.

\begin{lstlisting}
(gdb) x/10w $esp
0xbffff11c:	0x0804844a	0x080484f0	0x00000001	0x00000002
0xbffff12c:	0x00000003	0x08048460	0x00000000	0x00000000
0xbffff13c:	0xb7e29905	0x00000001
\end{lstlisting}

Самый первый элемент это \ac{RA} (\GTT{0x0804844a}).
Мы можем удостовериться в этом, дизассемблируя память по этому адресу:

\begin{lstlisting}[label=NOP_as_XCHG_example,style=customasmx86]
(gdb) x/5i 0x0804844a
   0x804844a <main+45>:	mov    $0x0,%eax
   0x804844f <main+50>:	leave  
   0x8048450 <main+51>:	ret    
   0x8048451:	xchg   %ax,%ax
   0x8048453:	xchg   %ax,%ax
\end{lstlisting}

Две инструкции \INS{XCHG} это холостые инструкции, аналогичные \ac{NOP}.

Второй элемент (\GTT{0x080484f0}) это адрес строки формата:

\begin{lstlisting}
(gdb) x/s 0x080484f0
0x80484f0:	"a=%d; b=%d; c=%d"
\end{lstlisting}

Остальные 3 элемента (1, 2, 3) это аргументы функции \printf.
Остальные элементы это может быть и мусор в стеке, но могут быть и значения
от других функций, их локальные переменные, итд.
Пока что мы можем игнорировать их.

Исполняем \q{finish}. 
Это значит исполнять все инструкции до самого конца функции. 
В данном случае это означает исполнять до завершения \printf.

\begin{lstlisting}
(gdb) finish
Run till exit from #0  __printf (format=0x80484f0 "a=%d; b=%d; c=%d") at printf.c:29
main () at 1.c:6
6		return 0;
Value returned is $2 = 13
\end{lstlisting}

\ac{GDB} показывает, что вернула \printf в \EAX (13).
Это, так же как и в примере с \olly, количество напечатанных символов.

А ещё мы видим \q{return 0;} и что это выражение находится в файле \GTT{1.c} в строке 6.
Действительно, файл \GTT{1.c} лежит в текущем директории и \ac{GDB} находит там эту строку.
Как \ac{GDB} знает, какая строка Си-кода сейчас исполняется?
Компилятор, генерируя отладочную информацию, также сохраняет информацию о соответствии строк в исходном коде и адресов инструкций.
GDB это всё-таки отладчик уровня исходных текстов.

Посмотрим регистры.
13 в \EAX:

\begin{lstlisting}
(gdb) info registers
eax            0xd	13
ecx            0x0	0
edx            0x0	0
ebx            0xb7fc0000	-1208221696
esp            0xbffff120	0xbffff120
ebp            0xbffff138	0xbffff138
esi            0x0	0
edi            0x0	0
eip            0x804844a	0x804844a <main+45>
...
\end{lstlisting}

Попробуем дизассемблировать текущие инструкции.
Стрелка указывает на инструкцию, которая будет исполнена следующей.

\begin{lstlisting}[style=customasmx86]
(gdb) disas
Dump of assembler code for function main:
   0x0804841d <+0>:	push   %ebp
   0x0804841e <+1>:	mov    %esp,%ebp
   0x08048420 <+3>:	and    $0xfffffff0,%esp
   0x08048423 <+6>:	sub    $0x10,%esp
   0x08048426 <+9>:	movl   $0x3,0xc(%esp)
   0x0804842e <+17>:	movl   $0x2,0x8(%esp)
   0x08048436 <+25>:	movl   $0x1,0x4(%esp)
   0x0804843e <+33>:	movl   $0x80484f0,(%esp)
   0x08048445 <+40>:	call   0x80482f0 <printf@plt>
=> 0x0804844a <+45>:	mov    $0x0,%eax
   0x0804844f <+50>:	leave  
   0x08048450 <+51>:	ret    
End of assembler dump.
\end{lstlisting}

По умолчанию \ac{GDB} показывает дизассемблированный листинг в формате AT\&T.
Но можно также переключиться в формат Intel:

\begin{lstlisting}[style=customasmx86]
(gdb) set disassembly-flavor intel
(gdb) disas
Dump of assembler code for function main:
   0x0804841d <+0>:	push   ebp
   0x0804841e <+1>:	mov    ebp,esp
   0x08048420 <+3>:	and    esp,0xfffffff0
   0x08048423 <+6>:	sub    esp,0x10
   0x08048426 <+9>:	mov    DWORD PTR [esp+0xc],0x3
   0x0804842e <+17>:	mov    DWORD PTR [esp+0x8],0x2
   0x08048436 <+25>:	mov    DWORD PTR [esp+0x4],0x1
   0x0804843e <+33>:	mov    DWORD PTR [esp],0x80484f0
   0x08048445 <+40>:	call   0x80482f0 <printf@plt>
=> 0x0804844a <+45>:	mov    eax,0x0
   0x0804844f <+50>:	leave  
   0x08048450 <+51>:	ret    
End of assembler dump.
\end{lstlisting}

Исполняем следующую строку \CCpp{}-кода.
\ac{GDB} покажет закрывающуюся скобку, означая, что это конец блока в функции.

\begin{lstlisting}
(gdb) step
7	};
\end{lstlisting}

Посмотрим регистры после исполнения инструкции \INS{MOV EAX, 0}.
\EAX здесь уже действительно ноль.

\begin{lstlisting}
(gdb) info registers
eax            0x0	0
ecx            0x0	0
edx            0x0	0
ebx            0xb7fc0000	-1208221696
esp            0xbffff120	0xbffff120
ebp            0xbffff138	0xbffff138
esi            0x0	0
edi            0x0	0
eip            0x804844f	0x804844f <main+50>
...
\end{lstlisting}
}
\DE{\subsubsection{x86}

\myindex{x86!\Instructions!LOOP}
Es gibt einen speziellen \LOOP Befehl im x86 Befehlssatz, der den Wert des
Registers \ECX prüft und falls dieser ungleich 0 ist, dekrementiert und danach
die den control flow wieder an das Label des \LOOP Operanden übergibt.
Vermutlich ist dieser Befehl nicht allzu geläufig und es gibt keine modernen
Compiler, welche ihn automatisch erzeugen. Wenn man also diesen Befehl irgendwo
im Code entdeckt, dann ist es äußerst wahrscheinlich, dass es sich um ein
handgeschriebenes Stück Assemblercode handelt.

\par In \CCpp werden Schleifen normalerweise mittels \TT{for()}-, \TT{while()}- oder
\TT{do/while()}-Ausdrücken erzeugt.

Starten wir mit \TT{for()}.

\myindex{\CLanguageElements!for}

Dieser Ausdruck definiert eine Schleifeninitialisierung (setzt den Zähler auf
einen Startwert), definiert eine Schleifenbedingung (ist der Zähler größer als
ein Grenzwert?), legt fest, was in jedem Durchlauf
(\glslink{increment}{Inkrement}/\glslink{decrement}{Dekrement}) geschieht und umschließt einen
Schleifenkörper.

\lstinputlisting[style=customc]{patterns/09_loops/simple/loops_1_DE.c}

Der erzeugte Code besteht ebenfalls aus vier Teilen.

Beginnen wir mit einem einfachen Beispiel:

\lstinputlisting[label=loops_src,style=customc]{patterns/09_loops/simple/loops_2.c}

Ergebnis (MSVC 2010):

\lstinputlisting[caption=MSVC 2010,style=customasmx86]{patterns/09_loops/simple/1_MSVC_DE.asm}

Hier gibt es nichts Besonderes zu sehen.

GCC 4.4.1 erzeugt einen fast identischen Code mit nur einen kleinen Unterschied:

\lstinputlisting[caption=GCC 4.4.1,style=customasmx86]{patterns/09_loops/simple/1_GCC_DE.asm}

Schauen wir uns nun an, was wir erhalten, wenn wir die Optimierung aktivieren
(\TT{\Ox}):

\lstinputlisting[caption=\Optimizing MSVC,style=customasmx86]{patterns/09_loops/simple/1_MSVC_Ox.asm}

Was hier passiert ist, dass der Speicherplatz für die $i$ Variable nicht mehr
auf dem lokalen Stack bereitgestellt wird, sondern das extra ein Register, \ESI,
hierfür verwendet wird. Dies ist bei derartig kleinen Funktionen möglich, wenn
nicht zu viele lokalen Variablen existieren.

Wichtig ist, dass die \ttf Funktion den Wert im Register \ESI nicht verändern
darf. Unser Compiler ist sich dieser Sache hier sicher. 
Und falls der Compiler entscheidet, das \ESI auch innerhalb der Funktion \ttf zu
verwenden, würde der Wert des Registers im Funktionsprolog gesichert und im
Funktionsepilog wiederhergestellt werden; fast genauso wie im folgenden Listing.
Man beachte das \TT{PUSH ESI/POP ESI} bei Funktionsbeginn und -ende. 
 
Probieren wir aus, was GCC 4.4.1 mit maximaler Optimierung (\Othree option)
liefert:

\lstinputlisting[caption=\Optimizing GCC 4.4.1,style=customasmx86]{patterns/09_loops/simple/1_GCC_O3.asm}

\myindex{Loop unwinding}

Aha, GCC hat unsere Schleife unrolled (d.h. ausgerollt).

\Gls{loop unwinding} hat Vorteile in Fällen, in denen es nicht viele
Schleifendurchläufe gibt und Ausführungszeit durch das Weglassen der
Befehle für die Kontrollstrukturen der Schleife gewonnen werden kann. 
Andererseits ist der erzeugte Code natürlich deutlich länger.

Große Schleifen zu \textit{unrollen} ist heutzutage nicht empfehlenswert, denn
größere Funktionen erfordern einen größeren Cache-Fußabdruck.%
%
\footnote{Ein hervorragender Artikel zum Thema: \DrepperMemory.
Für weitere Empfehlungen von Intel zum Unrolling siehe hier: 
\InSqBrackets{\IntelOptimization 3.4.1.7}.}.

Gut, nun wollen wir den Höchstwert der Variable $i$ auf 100 setzen und
kompilieren erneut. GCC liefert:

\lstinputlisting[caption=GCC,style=customasmx86]{patterns/09_loops/simple/2_GCC_DE.asm}

Das Ergebnis ist sehr ähnlich dem, das MSVC 2010 mit Optimierung (\Ox) erzeugt,
mit der Ausnahme, dass das \EBX Register für die Variable $i$ verwendet wird.

GCC ist sicher, dass das Register innerhalb der \ttf Funktion nicht verändert
wird und sollte dies doch der Fall sein, dass es im Funktionsprolog gesichert
und im Funktionsepilog wiederhergestellt werden wird, genau wie hier in der
\main Funktion.

\clearpage
\subsubsection{x86: \olly}
\myindex{\olly}

Wir kompilieren unser Beispiel in MSVC 2010 mit den Optionen \Ox und \Obzero und
laden es in \olly.

Es scheint, dass \olly in der Lage ist, einfache Schleifen zu erkennen und in
eckigen Klammern darzustellen, um die Übersichtlichkeit zu erhöhen:

\begin{figure}[H]
\centering
\myincludegraphics{patterns/09_loops/simple/olly1.png}
\caption{\olly: \main Einstieg}
\label{fig:loops_olly_1}
\end{figure}

Verfolgen mit (F8~--- \stepover) zeigt \ESI 
\glslink{increment}{incrementing}.
Hier ist zum Beispiel, $ESI=i=6$:

\begin{figure}[H]
\centering
\myincludegraphics{patterns/09_loops/simple/olly2.png}
\caption{\olly: Schleifenkörper wird gerade ausgeführt für $i=6$}
\label{fig:loops_olly_2}
\end{figure}

9 ist der letzte Wert in der Schleife
Deshalb triggert \JL nach der \glslink{increment}{inkrement}-Anweisung nicht und die Funktion
wird beendet:

\begin{figure}[H]
\centering
\myincludegraphics{patterns/09_loops/simple/olly3.png}
\caption{\olly: $ESI=10$, Ende der Schleife}
\label{fig:loops_olly_3}
\end{figure}

\subsubsection{x86: tracer}
\myindex{tracer}

Wie wir bemerken ist es nicht sonderlich komfortabel, Werte im Debugger manuell
nachzuverfolgen. Aus diesem Grund probieren wir \tracer aus.

Wir öffnen das kompilierte Beispiel in \IDA, finden die Adresse mit dem Befehl
\INS{PUSH ESI} (das einzige Argument an \ttf übergebend), welche hier
\TT{0x401026} ist und aktivieren den \tracer:

\begin{lstlisting}
tracer.exe -l:loops_2.exe bpx=loops_2.exe!0x00401026
\end{lstlisting}

\TT{BPX} setzt einen Breakpoint an der Adresse und der Tracer zeigt uns den
momentanen Status der Register an. In \TT{tracer.log} sehen wir das Folgende:

\lstinputlisting{patterns/09_loops/simple/tracer.log}

Wir sehen wie der Wert des \ESI Registers sich schrittweise von 2 zu 9
verändert. 

Mehr noch, der \tracer kann alle Registerwerte für alle Adressen innerhalb der
Funktion zusammensammeln. Dies wird hier mit \emph{trace} (dt. Nachverfolgung)
bezeichnet. Jeder Befehl wird verfolgt, alle interessanten Registerwerte werden
aufgezeichnet.

Danach die ein \IDA .idc-script erzeugt, das Kommentare hinzufügt. Wir haben
also herausgefunden, dass die Adresse der \main Funktion \TT{0x00401020} ist und
wir führen nun das Folgende aus:

\begin{lstlisting}
tracer.exe -l:loops_2.exe bpf=loops_2.exe!0x00401020,trace:cc
\end{lstlisting}

\TT{BPF} setzt einen Breakpoint auf eine Funktion.

Als Ergebnis erahlten wir die Skripte \TT{loops\_2.exe.idc} und
\TT{loops\_2.exe\_clear.idc}.

\clearpage
Wir laden \TT{loops\_2.exe.idc} in \IDA und erhalten:

\begin{figure}[H]
\centering
\myincludegraphics{patterns/09_loops/simple/IDA_tracer_cc.png}
\caption{\IDA mit geladenem .idc-script}
\label{fig:loops_IDA_tracer}
\end{figure}

Wir sehen, dass der Wert von \ESI zu Beginn der Schleife zwischen 2 und 9 und
nach dem Inkrement zwischen 3 und 0xA (10) liegt. Wir sehen auch, dass die
Funktion \main mit dem Rückgabewert 0 in \EAX terminiert.

\tracer erzeugt ebenfalls die Datei \TT{loops\_2.exe.txt}, welche Informationen
darüber enthält, welcher Befehl wie oft ausgeführt wurde, sowie zugehörige
Registerwerte:

\lstinputlisting[caption=loops\_2.exe.txt]{patterns/09_loops/simple/loops_2.exe.txt}
\myindex{\GrepUsage}
An dieser Stelle können wir grep verwenden.

}
\FR{\subsubsection{MSVC: x86}

Voici ce que nous obtenons dans la sortie assembleur (MSVC 2010):

\lstinputlisting[style=customasmx86]{patterns/04_scanf/3_checking_retval/ex3_MSVC_x86.asm}

\myindex{x86!\Registers!EAX}
La fonction \glslink{caller}{appelante} (\main) a besoin du résultat de la fonction
\glslink{callee}{appelée}, donc la fonction \glslink{callee}{appelée} le renvoie
dans la registre \EAX.

\myindex{x86!\Instructions!CMP}
Nous le vérifions avec l'aide de l'instruction \TT{CMP EAX, 1} (\emph{CoMPare}).
En d'autres mots, nous comparons la valeur dans le registre \EAX avec 1.

\myindex{x86!\Instructions!JNE}
Une instruction de saut conditionnelle \JNE suit l'instruction \CMP. \JNE signifie
\emph{Jump if Not Equal} (saut si non égal).

Donc, si la valeur dans le registre \EAX n'est pas égale à 1, le \ac{CPU} va poursuivre
l'exécution à l'adresse mentionnée dans l'opérande \JNE, dans notre cas \TT{\$LN2@main}.
Passer le contrôle à cette adresse résulte en l'exécution par le \ac{CPU} de
\printf avec l'argument \TT{What you entered? Huh?}.
Mais si tout est bon, le saut conditionnel n'est pas pris, et un autre appel à \printf
est exécuté, avec deux arguments:\\
\TT{'You entered \%d...'} et la valeur de \TT{x}.

\myindex{x86!\Instructions!XOR}
\myindex{\CLanguageElements!return}
Puisque dans ce cas le second \printf n'a pas été exécuté, il y a un \JMP qui le précède (saut inconditionnel).
Il passe le contrôle au point après le second \printf et juste avant l'instruction \TT{XOR EAX, EAX}, qui implémente \TT{return 0}.

% FIXME internal \ref{} to x86 flags instead of wikipedia
\myindex{x86!\Registers!\Flags}
Donc, on peut dire que comparer une valeur avec une autre est \emph{usuellement} implémenté
par la paire d'instructions \CMP/\Jcc, où \emph{cc} est un \emph{code de condition}.
\CMP compare deux valeurs et met les flags\footnote{flags x86, voir aussi: \href{http://en.wikipedia.org/wiki/FLAGS_register_(computing)}{Wikipédia}.}
du processeur.
\Jcc vérifie ces flags et décide de passer LE Contrôle à l'adresse spécifiée ou non.

\myindex{x86!\Instructions!CMP}
\myindex{x86!\Instructions!SUB}
\myindex{x86!\Instructions!JNE}
\myindex{x86!\Registers!ZF}
\label{CMPandSUB} 
Cela peut sembler paradoxal, mais l'instruction \CMP est en fait un \SUB (soustraction).
Toutes les instructions arithmétiques mettent les flags du processeur, pas seulement \CMP.
Si nous comparons 1 et 1, $1-1$ donne 0 donc le flag \ZF va être mis (signifiant
que le dernier résultat est 0).
Dans aucune autre circonstance \ZF ne sera mis, sauf si les opérandes sont égaux.
\JNE vérifie seulement le flag \ZF et saute seulement si il n'est pas mis. \JNE
est un synonyme pour \JNZ (\emph{Jump if Not Zero} (saut si non zéro)).
L'assembleur génère le même opcode pour les instructions \JNE et \JNZ.
Donc, l'instruction \CMP peut être remplacée par une instruction \SUB et presque
tout ira bien, à la différence que \SUB altère la valeur du premier opérande.
\CMP est un \emph{SUB sans sauver le résultat, mais modifiant les flags}.

\subsubsection{MSVC: x86: IDA}

\myindex{IDA}
C'est le moment de lancer \IDA et d'essayer de faire quelque chose avec.
À propos, pour les débutants, c'est une bonne idée d'utiliser l'option \TT{/MD}
de MSVC, qui signifie que toutes les fonctions standards ne vont pas être liées
avec le fichier exécutable, mais vont à la place être importées depuis le fichier
\TT{MSVCR*.DLL}.
Ainsi il est plus facile de voir quelles fonctions standards sont utilisées et où.

En analysant du code dans \IDA, il est très utile de laisser des notes pour soi-même
(et les autres).
En la circonstance, analysons cet exemple, nous voyons que \TT{JNZ} sera déclenché
en cas d'erreur.
Donc il est possible de déplacer le curseur sur le label, de presser \q{n} et de
lui donner le nom \q{error}.
Créons un autre label---dans \q{exit}.
Voici mon résultat:

\lstinputlisting[style=customasmx86]{patterns/04_scanf/3_checking_retval/ex3.lst}

Maintenant, il est légèrement plus facile de comprendre le code.
Toutefois, ce n'est pas une bonne idée de commenter chaque instruction.

% FIXME draw button?
Vous pouvez aussi cacher (replier) des parties d'une fonction dans \IDA.
Pour faire cela, marquez le bloc, puis appuyez sur Ctrl-\q{--} sur le pavé numérique et
entrez le texte qui doit être affiché à la place.

Cachons deux blocs et donnons leurs un nom:

\lstinputlisting[style=customasmx86]{patterns/04_scanf/3_checking_retval/ex3_2.lst}

% FIXME draw button?
Pour étendre les parties de code précédemment cachées. utilisez Ctrl-\q{+} sur le
pavé numérique.

\clearpage
En appuyant sur \q{space}, nous voyons comment \IDA représente une fonction sous
forme de graphe:

\begin{figure}[H]
\centering
\myincludegraphics{patterns/04_scanf/3_checking_retval/IDA.png}
\caption{IDA en mode graphe}
\label{fig:ex3_IDA_1}
\end{figure}

Il y a deux flèches après chaque saut conditionnel: une verte et une rouge.
La flèche verte pointe vers le bloc qui sera exécuté si le saut est déclenché,
et la rouge sinon.

\clearpage
Il est possible de replier des n\œu{}ds dans ce mode et de leurs donner aussi un nom (\q{group nodes}).
Essayons avec 3 blocs:

\begin{figure}[H]
\centering
\myincludegraphics{patterns/04_scanf/3_checking_retval/IDA2.png}
\caption{IDA en mode graphe avec 3 nœuds repliés}
\label{fig:ex3_IDA_2}
\end{figure}

C'est très pratique.
On peut dire qu'une part importante du travail des rétro-ingénieurs (et de tout
autre chercheur également) est de réduire la quantité d'information avec laquelle
travailler.

\clearpage
\subsubsection{MSVC: x86 + \olly}

Essayons de hacker notre programme dans \olly, pour le forcer à penser que \scanf
fonctionne toujours sans erreur.
Lorsque l'adresse d'une variable locale est passée à \scanf, la variable contient
initiallement toujours des restes de données aléatoires, dans ce cas \TT{0x6E494714}:

\begin{figure}[H]
\centering
\myincludegraphics{patterns/04_scanf/3_checking_retval/olly_1.png}
\caption{\olly: passer l'adresse de la variable à \scanf}
\label{fig:scanf_ex3_olly_1}
\end{figure}

\clearpage
Lorsque \scanf s'exécute dans la console, entrons quelque chose qui n'est pas du
tout un nombre, comme \q{asdasd}.
\scanf termine avec 0 dans \EAX, ce qui indique qu'une erreur s'est produite.

Nous pouvons vérifier la variable locale dans le pile et noter qu'elle n'a pas changé.
En effet, qu'aurait écrit \scanf ici?
Elle n'a simplement rien fait à part renvoyer zéro.

Essayons de \q{hacker} notre programme.
Clique-droit sur \EAX,
parmi les options il y a \q{Set to 1} (mettre à 1).
C'est ce dont nous avons besoin.

Nous avons maintenant 1 dans \EAX, donc la vérification suivante va s'exécuter comme
souhaiter et \printf va afficher la valeur de la variable dans la pile.

Lorsque nous lançons le programme (F9) nous pouvons voir ceci dans la fenêtre
de la console:

\lstinputlisting[caption=fenêtre console]{patterns/04_scanf/3_checking_retval/console.txt}

En effet, 1850296084 est la représentation en décimal du nombre dans la pile (\TT{0x6E494714})!


\clearpage
\subsubsection{MSVC: x86 + Hiew}
\myindex{Hiew}

Cela peut également être utilisé comme un exemple simple de modification de fichier
exécutable.
Nous pouvons essayer de modifier l'exécutable de telle sorte que le programme va
toujours afficher notre entrée, quelle qui'elle soit.

En supposant que l'exécutable est compilé avec la bibliothèque externe \TT{MSVCR*.DLL}
(i.e., avec l'option \TT{/MD}) \footnote{c'est aussi appelé \q{dynamic linking}},
nous voyons la fonction \main au début de la section \TT{.text}.
Ouvrons l'exécutable dans Hiew et cherchons le début de la section \TT{.text} (Enter,
F8, F6, Enter, Enter).

Nous pouvons voir cela:

\begin{figure}[H]
\centering
\myincludegraphics{patterns/04_scanf/3_checking_retval/hiew_1.png}
\caption{Hiew: fonction \main}
\label{fig:scanf_ex3_hiew_1}
\end{figure}

Hiew trouve les chaîne \ac{ASCIIZ} et les affiche, comme il le fait avec le nom
des fonctions importées.

\clearpage
Déplacez le curseur à l'adresse \TT{.00401027} (où se trouve l'instruction \TT{JNZ},
que l'on doit sauter), appuyez sur F3, et ensuite tapez \q{9090} (qui signifie deux
\ac{NOP}s):

\begin{figure}[H]
\centering
\myincludegraphics{patterns/04_scanf/3_checking_retval/hiew_2.png}
\caption{Hiew: remplacement de \TT{JNZ} par deux \ac{NOP}s}
\label{fig:scanf_ex3_hiew_2}
\end{figure}

Appuyez sur F9 (update). Maintenant, l'exécutable est sauvé sur le disque. Il va
se comporter comme nous le voulions.

Deux \ac{NOP}s ne constitue probablement pas l'approche la plus esthétique.
Une autre façon de modifier cette instruction est d'écrire simplement 0 dans le
second octet de l'opcode ((\gls{jump offset}), donc ce \TT{JNZ} va toujours sauter
à l'instruction suivante.

Nous pouvons également faire le contraire: remplacer le premier octet avec \TT{EB}
sans modifier le second octet (\gls{jump offset}).
Nous obtiendrions un saut inconditionnel qui est toujours déclenché.
Dans ce cas le message d'erreur sera affiché à chaque fois, peu importe l'entrée.

}
\JA{\subsubsection{x86}

\myparagraph{\NonOptimizing MSVC}

結果 (MSVC 2010):

\lstinputlisting[caption=MSVC 2010,style=customasmx86]{patterns/08_switch/1_few/few_msvc.asm}

実際、switch()でいくつかのcaseを持つ私たちの関数は、この構造に似ています。

\lstinputlisting[label=switch_few_ifelse,style=customc]{patterns/08_switch/1_few/few_analogue.c}

\myindex{\CLanguageElements!switch}
\myindex{\CLanguageElements!if}

いくつかのcaseでswitch()を使用する場合、ソースコード内の実際のswitch()か、
単にif文の組であるかどうかを確認することは不可能です。
\myindex{\SyntacticSugar}

これはswitch()が多段にネストされたif文との糖衣構文のようなものであることを意味します。

コンパイラが入力変数 $a$ を一時的なローカル変数\TT{tv64}に移動することを除いて、
生成されたコードには特に新しいことはありません。
\footnote{スタック内のローカル変数には接頭辞\TT{tv}が付きます。MSVCが内部変数として使用するために命名しています。}

これをGCC 4.4.1でコンパイルすると、最大限の最適化(\Othree option)を有効にしても
ほぼ同じ結果になります。

\myparagraph{\Optimizing MSVC}

% TODO separate various kinds of \TT
% idea: enclose command lines in a specific environment, like \cmdline{} 
% assembly instructions in \asm{} (now both \TT and \q{} are used),
% variables in,  like \var{}
% messages (string constants) in something else, like \strconst
% to separate them all. Now they all use \TT, which is not best
% \INS{} for all instructions including operands? --DY
では、MSVC(\Ox)の最適化を有効にしましょう:\TT{cl 1.c /Fa1.asm /Ox}

\label{JMP_instead_of_RET}
\lstinputlisting[caption=MSVC,style=customasmx86]{patterns/08_switch/1_few/few_msvc_Ox.asm}

ここで、汚いハックを見ることができます。

\myindex{x86!\Instructions!JZ}
\myindex{x86!\Instructions!JE}
\myindex{x86!\Instructions!SUB}

最初に、 $a$ の値を \EAX に置き、0を引きます。 EAXの値が0かどうかを確認するために行われますが、
そうであれば、 \ZF フラグがセットされます(例えば、0からの減算は0)
最初の条件ジャンプ \JE (\emph{Jump if Equal} またはあ同義語 \JZ~---\emph{Jump if Zero})は実行され、
制御フローは\TT{\$LN4@f}ラベルに渡されます。ここでは、 \TT{'zero'}メッセージが出力されます。
最初のジャンプが実行されない場合は、入力値から1が減算され、結果が0の場合、対応するジャンプが実行されます。

また、ジャンプが全く実行されない場合、制御フローは文字列引数\TT{'something unknown'}を \printf に渡します。

\label{jump_to_last_printf}
\myindex{\Stack}

次に、文字列ポインタが $a$ 変数に置かれ、 \printf が \CALL ではなく \JMP を介して呼び出されます。 
簡単に説明するとこうなります:
\gls{caller} は値をスタックにプッシュし、 \CALL 経由で関数を呼び出します。
\CALL 自体は戻りアドレス(\ac{RA})をスタックにプッシュし、関数アドレスへの無条件ジャンプを行います。
スタックポインタを移動させる命令が含まれていないため、任意の実行時点での関数は、次のスタックレイアウトを持ちます。

\begin{itemize}
\item\ESP---points to \ac{RA}
\item\TT{ESP+4}---points to the $a$ variable 
\end{itemize}

反対に、\printf をここで呼び出さなければならないときは、文字列を指し示す必要がある最初の\printf 引数を除いて、
全く同じスタックレイアウトが必要です。それが私たちのコードがすることです。

ファンクションの最初の引数を文字列のアドレスに置き換え、
関数 \ttf を直接呼び出しずに直接 \printf を呼び出すかのように、 \printf にジャンプします。
\printf は文字列を \gls{stdout} に出力し、 \RET 命令を実行します。スタックから\ac{RA}を取り出し、
制御フローは \ttf ではなく \ttf 関数の終りをバイパスして、 \ttf の \gls{caller} です。

\myindex{\CStandardLibrary!longjmp()}
\newcommand{\URLSJ}{\href{http://en.wikipedia.org/wiki/Setjmp.h}{wikipedia}}

% TODO \myref{}
\printf はすべての場合に  \ttf 関数の終わりで右に呼ばれるので、これはすべて可能です。
ある意味では、\TT{longjmp()}\footnote{\URLSJ}関数に似ています。
そしてもちろん、それはスピードのためにすべて行われます。

ARMコンパイラと同様のケースは、\q{\PrintfSeveralArgumentsSectionName}セクションに記載されています。
こちら:~(\myref{ARM_B_to_printf})

\clearpage
\subsubsection{MSVC: x86 + \olly}

\olly でプログラムをハックしようとして、 \scanf が常にエラーなく動作するようにしましょう。 
ローカル変数のアドレスが \scanf に渡されると、
変数には最初にいくつかのランダムなガベージが含まれます。この場合、\TT{0x6E494714}です。

\begin{figure}[H]
\centering
\myincludegraphics{patterns/04_scanf/3_checking_retval/olly_1.png}
\caption{\olly: passing variable address into \scanf}
\label{fig:scanf_ex3_olly_1}
\end{figure}

\clearpage
\scanf が実行されている間、コンソールでは、 \q{asdasd}のように、数字ではないものを入力します。 
\scanf は、エラーが発生したことを示す \EAX が0で終了します。

また、スタック内のローカル変数をチェックし、変更されていないことに注意してください。 
実際、 \scanf は何を書いていますか? 
ゼロを返す以外は何もしませんでした。

私たちのプログラムを\q{ハックする}ようにしましょう。 
\EAX を右クリックし、
オプションの中に\q{Set to 1}があります。 
これが必要なものです。

\EAX には1があるので、以下のチェックを意図どおりに実行し、
\printf は変数の値をスタックに出力します。 

プログラム(F9)を実行すると、コンソールウィンドウで次のように表示されます。

\lstinputlisting[caption=console window]{patterns/04_scanf/3_checking_retval/console.txt}

実際、1850296084はスタック(\TT{0x6E494714})の数値を10進表現したものです!


}
\IT{\subsubsection{x86}

\myindex{x86!\Instructions!LOOP}

Nell' instruction set x86, c'è una speciale istruzione di \LOOP per controllare il valore nel registro \ECX e
se non è 0, \gls{decrement} \ECX
e passa il controllo del flusso alla label nell' operando di \LOOP. 
Probabilmente questa istruzione non è molto conveniente, e non ci sono moderni compilatori che la inseriscono automaticamente.
Di conseguenza, se la vedete da qualche parte, probabilmente quella parte di codice assembly è stata scritta a mano.

\par

In \CCpp i cicli sono solitamente costruiti usando le istruzioni \TT{for()}, \TT{while()} o \TT{do/while()}.

Iniziamo con \TT{for()}.
\myindex{\CLanguageElements!for}

Questa istruzione definisce l'inizializzazione del ciclo (imposta un contatore di cicli ad un valore iniziale), 
la condizione di ciclo (il contatore è maggiore di un valore limite?), cosa viene eseguito ad ogni iterazione (\gls{increment}/\gls{decrement} il contatore)
e ovviamente il corpo del ciclo.

\lstinputlisting[style=customc]{patterns/09_loops/simple/loops_1_IT.c}

Anche il codice generato è composto da quattro parti.

Iniziamo con un semplice esempio:

\lstinputlisting[label=loops_src,style=customc]{patterns/09_loops/simple/loops_2.c}

Il risultato (MSVC 2010):

\lstinputlisting[caption=MSVC 2010,style=customasmx86]{patterns/09_loops/simple/1_MSVC_IT.asm}

Come possiamo vedere, nulla di speciale.

GCC 4.4.1 emette quasi lo stesso codice, con una sottile differenza:

\lstinputlisting[caption=GCC 4.4.1,style=customasmx86]{patterns/09_loops/simple/1_GCC_IT.asm}

Ora vediamo cosa ottieniamo con l'ottimizzazione impostata su (\TT{\Ox}):

\lstinputlisting[caption=\Optimizing MSVC,style=customasmx86]{patterns/09_loops/simple/1_MSVC_Ox.asm}

Quello che abbiamo è che lo spazio per la veriabile $i$ non è più allocato nello stack,
ma viene utilizzato un registro, \ESI.
Questo è possibile nelle piccole funzioni dove non ci somo molte variabili locali.

Una cosa importante è che la funzione \ttf non deve cambiare il valore in \ESI.
Il nostro compilatore c'è lo assicura. 
E se il compilatore decide di usare il registro \ESI anche nella funzione \ttf, il suo valore viene salvato durante il prologo della funzione e ripristinato durante l'epilogo della funzione,
similmente al nostro esempio: notare \TT{PUSH ESI/POP ESI}
all'inizio e fine della funzione.

Proviamo GCC 4.4.1 con la massima ottimizzazione impostata (opzione \Othree):

\lstinputlisting[caption=\Optimizing GCC 4.4.1,style=customasmx86]{patterns/09_loops/simple/1_GCC_O3.asm}

\myindex{Loop unwinding}

Huh, GCC ha appena "srotolato" il nostro ciclo.

\Gls{loop unwinding} è vantaggioso nel caso in cui non ci siano molte iterazioni, perchè possiamo ridurre il tempo di esecuzione rimuovenfo tutte le istruzioni di supporto ai cicli. 
Dall' altro lato, il codice risultante è ovviamente maggiore.

Srotolare grossi cicli non è raccomandato al giorno d'oggi, perchè grosse funzioni possono richiedere un ingombro della cache maggiore%
%
\footnote{Un ottimo articolo a riguardo: \DrepperMemory.
Qui ci sono altre raccomandazioni da Intel riguardo lo srotolamento dei cicli: 
\InSqBrackets{\IntelOptimization 3.4.1.7}.}.

OK, aumentiamo a 100 il massimo valore della variabile $i$ e proviamo nuovamente. GCC fa:

\lstinputlisting[caption=GCC,style=customasmx86]{patterns/09_loops/simple/2_GCC_IT.asm}

E' abbastanza simile a quello che produce MSVC 2010 con ottimizzazione (\Ox), 
con l'eccezione che il registro \EBX è allocato per la variabile $i$.

GCC è sicuro che questo registro non verrà modificato nella funzione \ttf 
e nel caso, verrà salvato durante il prologo della funzione e verrà ripristinato durante l'epilogo, 
proprio come in questo caso nella funzione \main.

\clearpage
\subsubsection{x86: \olly}
\myindex{\olly}

Compiliamo il nostro esempio con MSVC 2010 con le opzioni \Ox e \Obzero,
carichiamolo poi in \olly.

Sembrerebbe che \olly sia in grado di rilevare dei semplici cicli e ce li mostra tra parentesi quadre, per convenienza:

\begin{figure}[H]
\centering
\myincludegraphics{patterns/09_loops/simple/olly1.png}
\caption{\olly: inizio \main}
\label{fig:loops_olly_1}
\end{figure}

Tracciando (F8~--- \stepover) vediamo \ESI 
\glslink{increment}{incrementing}.
Qui, per esempio, $ESI=i=6$:

\begin{figure}[H]
\centering
\myincludegraphics{patterns/09_loops/simple/olly2.png}
\caption{\olly: corpo del ciclo appena eseguito con $i=6$}
\label{fig:loops_olly_2}
\end{figure}

9 è l'ultimo valore del ciclo.
Motivo per il quale, \JL non si attiva dopo \gls{increment} e la funzione concluderà:

\begin{figure}[H]
\centering
\myincludegraphics{patterns/09_loops/simple/olly3.png}
\caption{\olly: $ESI=10$, fine ciclo}
\label{fig:loops_olly_3}
\end{figure}

\subsubsection{x86: tracer}
\myindex{tracer}

Come possimo vedere, non è molto comodo tracciare manualmente nel debugger.
Questa è la ragione per cui proveremo ad usare \tracer.

Apriamo l'esempio compilato in \IDA, cerchiamo l'indirizzo dell' istruzione \INS{PUSH ESI}
(che passa l'unico argomento a \ttf), che è \TT{0x401026} in questo caso ed eseguiamo il \tracer:

\begin{lstlisting}
tracer.exe -l:loops_2.exe bpx=loops_2.exe!0x00401026
\end{lstlisting}

\TT{BPX} imposta solamente un breakpoint all' indirizzo e \tracer stamperà poi lo stato dei registri.

Questo è ciò che vediamo in  \TT{tracer.log}:

\lstinputlisting{patterns/09_loops/simple/tracer.log}

Vediamo il valore del registro \ESI, cambiare da 2 a 9.

Oltre a ciò, il \tracer può collezionare i valori del registro a tutti gli indirizzi all' interno della funzione.
Questo si chiama \emph{trace}.
Ogni istruzione viene tracciata, tutti i valori interessanti del registro vengono collezionati.

Dopodichè, viene generato un \IDA .idc-script, che aggiunge i commenti.
Quindi, abbiamo appreso in \IDA che l' indirizzo della funzione \main è \TT{0x00401020}, quindi eseguiamo:

\begin{lstlisting}
tracer.exe -l:loops_2.exe bpf=loops_2.exe!0x00401020,trace:cc
\end{lstlisting}

\TT{BPF} sta per "imposta breakpoint alla funzione".

Come risultato, otteniamo gli script \TT{loops\_2.exe.idc} e \TT{loops\_2.exe\_clear.idc}.

\clearpage
Carichiamo \TT{loops\_2.exe.idc} in \IDA e osserviamo:

\begin{figure}[H]
\centering
\myincludegraphics{patterns/09_loops/simple/IDA_tracer_cc.png}
\caption{\IDA con .idc-script caricato}
\label{fig:loops_IDA_tracer}
\end{figure}

Notiamo che \ESI può assumere valori da 2 a 9 all' inizio del corpo del ciclo,
ma da 3 a 0xA (10) dopo l'incremento.
Notiamo inoltre che il \main termina con 0 in \EAX.

\tracer genera inoltre \TT{loops\_2.exe.txt}, 
che contiene informazioni riguardo a quante volte ogni istruzione è stata eseguita e i valori del registro:

\lstinputlisting[caption=loops\_2.exe.txt]{patterns/09_loops/simple/loops_2.exe.txt}
\myindex{\GrepUsage}
Possiamo usare grep qui.

}

\EN{\subsubsection{x64}
\label{subsec:popcnt}

Let's modify the example slightly to extend it to 64-bit:

\lstinputlisting[label=popcnt_x64_example,style=customc]{patterns/14_bitfields/4_popcnt/shifts64.c}

\myparagraph{\NonOptimizing GCC 4.8.2}

So far so easy.

\lstinputlisting[caption=\NonOptimizing GCC 4.8.2,style=customasmx86]{patterns/14_bitfields/4_popcnt/shifts64_GCC_O0_EN.s}

\myparagraph{\Optimizing GCC 4.8.2}

\lstinputlisting[caption=\Optimizing GCC 4.8.2,numbers=left,label=shifts64_GCC_O3,style=customasmx86]{patterns/14_bitfields/4_popcnt/shifts64_GCC_O3_EN.s}

This code is terser, but has a quirk.

In all examples that we see so far, we were incrementing the \q{rt} value after comparing a specific bit,
but the code here increments \q{rt} before (line 6), writing the new value into register \EDX .
Thus, if the last bit is 1, the \CMOVNE\footnote{Conditional MOVe if Not Equal} instruction
(which is a synonym for \CMOVNZ\footnote{Conditional MOVe if Not Zero}) \emph{commits} 
the new value of \q{rt}
by moving \EDX (\q{proposed rt value}) into \EAX (\q{current rt} to be returned at the end).

Hence, the incrementing is performed at each step of loop, i.e., 64 times, without any relation to the input value.

The advantage of this code is that it contain only one conditional jump (at the end of the loop) instead of 
two jumps (skipping the \q{rt} value increment and at the end of loop).
And that might work faster on the modern CPUs with branch predictors: \myref{branch_predictors}.

\label{FATRET}
\myindex{x86!\Instructions!FATRET}
The last instruction is \INS{REP RET} (opcode \TT{F3 C3}) 
which is also called \INS{FATRET} by MSVC.
This is somewhat optimized version of \RET, 
which is recommended by AMD to be placed at the end of function, if \RET goes right after conditional jump: 
\InSqBrackets{\AMDOptimization p.15}
\footnote{More information on it: \url{http://repzret.org/p/repzret/}}.

\myparagraph{\Optimizing MSVC 2010}

\lstinputlisting[caption=\Optimizing MSVC 2010,style=customasmx86]{patterns/14_bitfields/4_popcnt/MSVC_2010_x64_Ox_EN.asm}

\myindex{x86!\Instructions!ROL}
Here the \ROL instruction is used instead of 
\SHL, which is in fact \q{rotate left} 
instead of \q{shift left},
but in this example it works just as \TT{SHL}.

You can read more about the rotate instruction here: \myref{ROL_ROR}.

\Reg{8} here is counting from 64 to 0.
It's just like an inverted $i$.

Here is a table of some registers during the execution:

\begin{center}
\begin{tabular}{ | l | l | }
\hline
\HeaderColor RDX & \HeaderColor R8 \\
\hline
0x0000000000000001 & 64 \\
\hline
0x0000000000000002 & 63 \\
\hline
0x0000000000000004 & 62 \\
\hline
0x0000000000000008 & 61 \\
\hline
... & ... \\
\hline
0x4000000000000000 & 2 \\
\hline
0x8000000000000000 & 1 \\
\hline
\end{tabular}
\end{center}

\myindex{x86!\Instructions!FATRET}
At the end we see the \INS{FATRET} instruction, which was explained here: \myref{FATRET}.

\myparagraph{\Optimizing MSVC 2012}

\lstinputlisting[caption=\Optimizing MSVC 2012,style=customasmx86]{patterns/14_bitfields/4_popcnt/MSVC_2012_x64_Ox_EN.asm}

\myindex{\CompilerAnomaly}
\label{MSVC2012_anomaly}
\Optimizing MSVC 2012 does almost the same job as 
optimizing MSVC 2010, but somehow, it generates two identical loop bodies and the loop count is now 32 instead of 64.

To be honest, it's not possible to say why. Some optimization trick? Maybe it's better for the loop body to be slightly 
longer?

Anyway, such code is relevant here to show that sometimes the compiler output may be really weird and 
illogical, but perfectly working.

}
\RU{\subsection{x64}

\myindex{x86-64}
В x86-64 всё немного иначе, здесь аргументы функции (4 или 6) передаются через регистры, 
а \gls{callee} из читает их из регистров, а не из стека.

\subsubsection{MSVC}

\Optimizing MSVC:

\lstinputlisting[caption=\Optimizing MSVC 2012 x64,style=customasmx86]{patterns/05_passing_arguments/x64_MSVC_Ox_RU.asm}

Как видно, очень компактная функция \ttf берет аргументы прямо из регистров.

Инструкция \LEA используется здесь для сложения чисел. 
Должно быть компилятор посчитал, что это будет эффективнее использования \TT{ADD}.

\myindex{x86!\Instructions!LEA}
В самой \main{} \LEA{} также используется для подготовки первого и третьего аргумента: должно быть,
компилятор решил, что \LEA{} будет работать здесь быстрее, чем загрузка значения в регистр при помощи \MOV.

Попробуем посмотреть вывод неоптимизирующего MSVC:

\lstinputlisting[caption=MSVC 2012 x64,style=customasmx86]{patterns/05_passing_arguments/x64_MSVC_IDA_RU.asm}

Немного путанее: все 3 аргумента из регистров зачем-то сохраняются в стеке.

\myindex{Shadow space}
\label{shadow_space}
Это называется \q{shadow space} \footnote{\href{http://msdn.microsoft.com/en-us/library/zthk2dkh(v=vs.80).aspx}{MSDN}}: 
каждая функция в Win64 может (хотя и не обязана) сохранять значения 4-х регистров там.

Это делается по крайней мере из-за двух причин: 
1) в большой функции отвести целый регистр (а тем более 4 регистра) для входного аргумента 
слишком расточительно, так что к нему будет обращение через стек;

2) отладчик всегда знает, где найти аргументы функции в момент останова
\footnote{\href{http://msdn.microsoft.com/en-us/library/ew5tede7(v=VS.90).aspx}{MSDN}}.

Так что, какие-то большие функции могут сохранять входные аргументы в \q{shadows space} 
для использования в будущем, а небольшие функции, как наша, могут этого и не делать.

Место в стеке для \q{shadow space} выделяет именно \gls{caller}.

\subsubsection{GCC}

\Optimizing GCC также делает понятный код:

\lstinputlisting[caption=\Optimizing GCC 4.4.6 x64,style=customasmx86]{patterns/05_passing_arguments/x64_GCC_O3_RU.s}

\NonOptimizing GCC:

\lstinputlisting[caption=GCC 4.4.6 x64,style=customasmx86]{patterns/05_passing_arguments/x64_GCC_RU.s}

\myindex{Shadow space}
В соглашении о вызовах System V *NIX (\SysVABI) нет \q{shadow space}, но \gls{callee} тоже иногда
должен сохранять где-то аргументы, потому что, опять же, регистров может и не хватить на все действия.
Что мы здесь и видим.

\subsubsection{GCC: uint64\_t вместо int}

Наш пример работал с 32-битным \Tint, поэтому использовались 32-битные части регистров с префиксом \TT{E-}.

Его можно немного переделать, чтобы он заработал с 64-битными значениями:

\lstinputlisting[style=customc]{patterns/05_passing_arguments/ex64.c}

\lstinputlisting[caption=\Optimizing GCC 4.4.6 x64,style=customasmx86]{patterns/05_passing_arguments/ex64_GCC_O3_IDA_RU.asm}

Собствено, всё то же самое, только используются регистры \emph{целиком}, с префиксом \TT{R-}.

}
\PTBR{\subsubsection{x64: 8 argumentos}
% to be sync: \subsubsection{x64: 8 integer arguments}

\myindex{x86-64}
\label{example_printf8_x64}
Para ver como outros argumentos são passados pela pilha,
vamos mudar nosso exemplo novamente aumentando o numero de argumentos para 9 (\printf + 8 variáveis \Tint):

\lstinputlisting[style=customc]{patterns/03_printf/2.c}

\myparagraph{MSVC}

Como mencionado anteriormente, os primeiros 4 argumentos tem de ser passados pelos registradores \RCX, \RDX, \Reg{8}, \Reg{9} no Win64, enquanto o resto pela pilha.
Isso é exatamente o que veremos aqui.
Entretanto, a instrução \MOV, ao invés de \PUSH, é usada para preparar a pilha, portanto os valores são armazenados de uma maneira direta.

% TODO translate
\lstinputlisting[caption=MSVC 2012 x64,style=customasmx86]{patterns/03_printf/x86/2_MSVC_x64_EN.asm}

Um leitor observativo pode se indagar por que são alocados 8 bytes para valores int quando 4 já é suficiente?
Sim, mas lembre-se: 8 bytes são alocados para qualquer tipo de informação menor do que 64 bits.
Isso é estabelecido com o objetivo de ser conveniente: é mais facil calcular o endereço de um argumento arbitrário.
Além do mais, eles são alocados em endereços de memórias alinhados. Da mesma maneira no 32-bits: 4 bytes são reservados para todos os tipos de informação.

% also for local variables?

\PTBRph{}

}
\IT{\subsubsection{x64}

\myindex{x86-64}
La situazione è simile, con l'unica differenza che, per il passaggio degli argomenti, i registri sono usati al posto dello stack.

\myparagraph{MSVC}

\lstinputlisting[caption=MSVC 2012 x64,style=customasmx86]{patterns/04_scanf/1_simple/ex1_MSVC_x64_EN.asm}

\myparagraph{GCC}

\lstinputlisting[caption=\Optimizing GCC 4.4.6 x64,style=customasmx86]{patterns/04_scanf/1_simple/ex1_GCC_x64_EN.s}

}
\DE{\subsubsection{x64}
\label{subsec:popcnt}
Verändern wir das Beispiel ein wenig um es auf 64 Bit zu erweitern:

\lstinputlisting[label=popcnt_x64_example,style=customc]{patterns/14_bitfields/4_popcnt/shifts64.c}

\myparagraph{\NonOptimizing GCC 4.8.2}

So weit, so einfach.

\lstinputlisting[caption=\NonOptimizing GCC 4.8.2,style=customasmx86]{patterns/14_bitfields/4_popcnt/shifts64_GCC_O0_DE.s}

\myparagraph{\Optimizing GCC 4.8.2}

\lstinputlisting[caption=\Optimizing GCC 4.8.2,numbers=left,label=shifts64_GCC_O3,style=customasmx86]{patterns/14_bitfields/4_popcnt/shifts64_GCC_O3_DE.s}
Dieser Code ist kürzer, birgt aber eine Besonderheit.

In allen bisher betrachteten Beispieln haben wir den Wert von \q{rt} nach dem
Vergleich mit einem speziellen Bit erhöht, aber dieser Code erhöht \q{rt} vorher
(Zeile 6) und schreibt den neuen Wert in das Register \EDX.
Dadurch überträgt der Befehl \CMOVNE\footnote{Conditional MOVe if Not Equal}
(der ein Synonym für \CMOVNZ\footnote{Conditional MOVe if Not Zero} ist) den
neuen Wert von \q{rt} durch Verschieben des Wertes in \EDX (vorgeschlagener
Wert von \q{rt}) nach \EAX (\q{aktueller Wert von rt}). Der in \EAX befindliche
Wert wird schließlich zurückgegeben.

Deshalb wird die Erhöhung des Zählers in jedem Durchlauf der Schleife
durchgeführt, d.h. 64 mal, ohne dass eine Abhängigkeit vom Eingabewert
besteht.

Der Vorteil dieses Code ist, dass er nur einen bedingten Sprung enthält (am
Ende der Schleife) anstatt zwei Sprüngen (Überspringen des Erhöhens von \q{rt}
und Ende der Schleife). 
Der Code ist somit auf modernen CPUs mit Branch Pedictors möglicherweise
schneller:\myref{branch_predictors}.

\label{FATRET}
\myindex{x86!\Instructions!FATRET}
Der letzte Befehl hier ist \INS{REP RET} (Opcode \TT{F3 C3}), der von MSVC auch
\INS{FATRET} genannt wird.
Hierbei handelt es sich um eine optimierte Version von \RET, die von ARM
bevorzugt am Ende der Funktion verwendet wird, wenn \RET direkt nach einem
bedingten Sprung folg:.
\InSqBrackets{\AMDOptimization p.15}
\footnote{Mehr Informationen dazu: \url{http://repzret.org/p/repzret/}}.

\myparagraph{\Optimizing MSVC 2010}

\lstinputlisting[caption=\Optimizing MSVC 2010,style=customasmx86]{patterns/14_bitfields/4_popcnt/MSVC_2010_x64_Ox_DE.asm}

\myindex{x86!\Instructions!ROL}
Hier wird der Befehl \ROL anstelle von \SHL verwendet, welches einer
\q{Linksrotation} anstatt einer \q{Linksverschiebung} entspricht.
In diesem Beispiel entspricht \ROL einem \TT{SHL}.

Für mehr Informationen zu Rotationsbefehlen siehe: \myref{ROL_ROR}. 

\Reg{8} zählt hier von 64 auf 0 herunter. 
Dies entspricht dem invertierten $i$.

Hier ist eine Tabelle einiger Register während der Ausführung des Programms:

\begin{center}
\begin{tabular}{ | l | l | }
\hline
\HeaderColor RDX & \HeaderColor R8 \\
\hline
0x0000000000000001 & 64 \\
\hline
0x0000000000000002 & 63 \\
\hline
0x0000000000000004 & 62 \\
\hline
0x0000000000000008 & 61 \\
\hline
... & ... \\
\hline
0x4000000000000000 & 2 \\
\hline
0x8000000000000000 & 1 \\
\hline
\end{tabular}
\end{center}

\myindex{x86!\Instructions!FATRET}
Am Ende finden wir den Befehl \INS{FATRET}, der hier schon erklärt
wurde:\myref{FATRET}.

\myparagraph{\Optimizing MSVC 2012}

\lstinputlisting[caption=\Optimizing MSVC 2012,style=customasmx86]{patterns/14_bitfields/4_popcnt/MSVC_2012_x64_Ox_DE.asm}

\myindex{\CompilerAnomaly}
\label{MSVC2012_anomaly}
Der optimierende MSVC 2012 erzeugt fast den gleichen Code wie MSVC 2012,
generiert aber aus irgendeinem Grund zwei identischen Rümpfe für die Schleifen
und die Schleife zählt nun bis 32 anstatt 64.

Ehrlich gesagt, kann man nicht genau erklären warum. Es könnte sich um einen
Optimierungstrick handeln. Vielleicht ist es für den Rumpf der Schleife besser
ein wenig länger zu sein.

Trotzdem ist solcher Code relevant um zu zeigen, dass der Output des Compilers
manchmal sehr merkwürdig und unlogisch sein kann und dennoch tadellos
funktioniert.
}
\FR{\subsubsection{x64: 8 arguments entier}

\myindex{x86-64}
\label{example_printf8_x64}
Pour voir comment les autres arguments sont passés par la pile, changeons encore
notre exemple en augmentant le nombre d'arguments à 9 (chaîne de format de
\printf + 8 variables \Tint):

\lstinputlisting[style=customc]{patterns/03_printf/2.c}

\myparagraph{MSVC}

Comme il a déjà été mentionné, les 4 premiers arguments sont passés par les registres
\RCX, \RDX, \Reg{8}, \Reg{9} sous Win64, tandis les autres le sont---par la pile.
C'est exactement de que l'on voit ici.
Toutefois, l'instruction \MOV est utilisée ici à la place de \PUSH, donc les valeurs
sont stockées sur la pile d'une manière simple.

\lstinputlisting[caption=MSVC 2012 x64,style=customasmx86]{patterns/03_printf/x86/2_MSVC_x64_FR.asm}

Le lecteur observateur pourrait demander pourquoi 8 octets sont alloués sur la
pile pour les valeurs \Tint, alors que 4 suffisent?
Oui, il faut se rappeler: 8 octets sont alloués pour tout type de données plus
petit que 64 bits.
Ceci est instauré pour des raisons de commodités: cela rend facile le calcul
de l'adresse de n'importe quel argument.
En outre, ils sont tous situés à des adresses mémoires alignées.
Il en est de même dans les environnements 32-bit: 4 octets sont réservés pour tout
types de données.

% also for local variables?

\myparagraph{GCC}

Le tableau est similaire pour les OS x86-64 *NIX, excepté que les 6 premiers arguments
sont passés par les registres \RDI, \RSI, \RDX, \RCX, \Reg{8}, \Reg{9}.
Tout les autres---par la pile.
GCC génère du code stockant le pointeur de chaîne dans \EDI au lieu de \RDI{}---nous
l'avons noté précédemment:
\myref{hw_EDI_instead_of_RDI}.

Nous avions également noté que le registre \EAX a été vidé avant l'appel à
\printf: \myref{SysVABI_input_EAX}.

\lstinputlisting[caption=GCC 4.4.6 x64 \Optimizing,style=customasmx86]{patterns/03_printf/x86/2_GCC_x64_FR.s}

\myparagraph{GCC + GDB}
\myindex{GDB}

Essayons cet exemple dans \ac{GDB}.

\begin{lstlisting}
$ gcc -g 2.c -o 2
\end{lstlisting}

\begin{lstlisting}
$ gdb 2
GNU gdb (GDB) 7.6.1-ubuntu
...
Reading symbols from /home/dennis/polygon/2...done.
\end{lstlisting}

\begin{lstlisting}[caption=mettons le point d'arrêt à \printf{,} et lançons]
(gdb) b printf
Breakpoint 1 at 0x400410
(gdb) run
Starting program: /home/dennis/polygon/2 

Breakpoint 1, __printf (format=0x400628 "a=%d; b=%d; c=%d; d=%d; e=%d; f=%d; g=%d; h=%d\n") at printf.c:29
29	printf.c: No such file or directory.
\end{lstlisting}

Les registres \RSI/\RDX/\RCX/\Reg{8}/\Reg{9} ont les valeurs attendues.
\RIP contient l'adresse de la toute première instruction de la fonction \printf.

\begin{lstlisting}
(gdb) info registers
rax            0x0	0
rbx            0x0	0
rcx            0x3	3
rdx            0x2	2
rsi            0x1	1
rdi            0x400628	4195880
rbp            0x7fffffffdf60	0x7fffffffdf60
rsp            0x7fffffffdf38	0x7fffffffdf38
r8             0x4	4
r9             0x5	5
r10            0x7fffffffdce0	140737488346336
r11            0x7ffff7a65f60	140737348263776
r12            0x400440	4195392
r13            0x7fffffffe040	140737488347200
r14            0x0	0
r15            0x0	0
rip            0x7ffff7a65f60	0x7ffff7a65f60 <__printf>
...
\end{lstlisting}

\begin{lstlisting}[caption=inspectons la chaîne de format]
(gdb) x/s $rdi
0x400628:	"a=%d; b=%d; c=%d; d=%d; e=%d; f=%d; g=%d; h=%d\n"
\end{lstlisting}

Affichons la pile avec la commande x/g cette fois---\emph{g} est l'unité pour \emph{giant words}, i.e., mots de 64-bit.

\begin{lstlisting}
(gdb) x/10g $rsp
0x7fffffffdf38:	0x0000000000400576	0x0000000000000006
0x7fffffffdf48:	0x0000000000000007	0x00007fff00000008
0x7fffffffdf58:	0x0000000000000000	0x0000000000000000
0x7fffffffdf68:	0x00007ffff7a33de5	0x0000000000000000
0x7fffffffdf78:	0x00007fffffffe048	0x0000000100000000
\end{lstlisting}

Le tout premier élément de la pile, comme dans le cas précédent, est la \ac{RA}.
3 valeurs sont aussi passées par la pile: 6, 7, 8.
Nous voyons également que 8 est passé avec les 32-bits de poids fort non
effacés: \GTT{0x00007fff00000008}.
C'est en ordre, car les valeurs sont d'un type \Tint, qui est 32-bit.
Donc, la partie haute du registre ou l'élément de la pile peuvent contenir des
\q{restes de données aléatoires}.

Si vous regardez où le contrôle reviendra après l'exécution de \printf,
\ac{GDB} affiche la fonction \main en entier:

\begin{lstlisting}[style=customasmx86]
(gdb) set disassembly-flavor intel
(gdb) disas 0x0000000000400576
Dump of assembler code for function main:
   0x000000000040052d <+0>:	push   rbp
   0x000000000040052e <+1>:	mov    rbp,rsp
   0x0000000000400531 <+4>:	sub    rsp,0x20
   0x0000000000400535 <+8>:	mov    DWORD PTR [rsp+0x10],0x8
   0x000000000040053d <+16>:	mov    DWORD PTR [rsp+0x8],0x7
   0x0000000000400545 <+24>:	mov    DWORD PTR [rsp],0x6
   0x000000000040054c <+31>:	mov    r9d,0x5
   0x0000000000400552 <+37>:	mov    r8d,0x4
   0x0000000000400558 <+43>:	mov    ecx,0x3
   0x000000000040055d <+48>:	mov    edx,0x2
   0x0000000000400562 <+53>:	mov    esi,0x1
   0x0000000000400567 <+58>:	mov    edi,0x400628
   0x000000000040056c <+63>:	mov    eax,0x0
   0x0000000000400571 <+68>:	call   0x400410 <printf@plt>
   0x0000000000400576 <+73>:	mov    eax,0x0
   0x000000000040057b <+78>:	leave  
   0x000000000040057c <+79>:	ret    
End of assembler dump.
\end{lstlisting}

Laissons se terminer l'exécution de \printf, exécutez l'instruction mettant \EAX
à zéro, et notez que le registre \EAX à une valeur d'exactement zéro.
\RIP pointe maintenant sur l'instruction \INS{LEAVE}, i.e, la pénultième de la
fonction \main.

\begin{lstlisting}
(gdb) finish
Run till exit from #0  __printf (format=0x400628 "a=%d; b=%d; c=%d; d=%d; e=%d; f=%d; g=%d; h=%d\n") at printf.c:29
a=1; b=2; c=3; d=4; e=5; f=6; g=7; h=8
main () at 2.c:6
6		return 0;
Value returned is $1 = 39
(gdb) next
7	};
(gdb) info registers
rax            0x0	0
rbx            0x0	0
rcx            0x26	38
rdx            0x7ffff7dd59f0	140737351866864
rsi            0x7fffffd9	2147483609
rdi            0x0	0
rbp            0x7fffffffdf60	0x7fffffffdf60
rsp            0x7fffffffdf40	0x7fffffffdf40
r8             0x7ffff7dd26a0	140737351853728
r9             0x7ffff7a60134	140737348239668
r10            0x7fffffffd5b0	140737488344496
r11            0x7ffff7a95900	140737348458752
r12            0x400440	4195392
r13            0x7fffffffe040	140737488347200
r14            0x0	0
r15            0x0	0
rip            0x40057b	0x40057b <main+78>
...
\end{lstlisting}
}
\JA{\subsubsection{x64}

\myindex{x86-64}
ここの画像は、スタックではなくレジスタが引数の受け渡しに使用されるという違いと似ています。

\myparagraph{MSVC}

\lstinputlisting[caption=MSVC 2012 x64,style=customasmx86]{patterns/04_scanf/1_simple/ex1_MSVC_x64_JA.asm}

\myparagraph{GCC}

\lstinputlisting[caption=\Optimizing GCC 4.4.6 x64,style=customasmx86]{patterns/04_scanf/1_simple/ex1_GCC_x64_JA.s}

}
\PL{\subsubsection{x64}

\myindex{x86-64}
Sposób wykonania programy będzie niemal taki sam, z tą różnicą, że argumenty tym razem będą będą przekazywane przez rejestry, a nie przez stos.

\myparagraph{MSVC}

\lstinputlisting[caption=MSVC 2012 x64,style=customasmx86]{patterns/04_scanf/1_simple/ex1_MSVC_x64_PL.asm}

\myparagraph{GCC}

\lstinputlisting[caption=\Optimizing GCC 4.4.6 x64,style=customasmx86]{patterns/04_scanf/1_simple/ex1_GCC_x64_PL.s}

}

\EN{\input{patterns/04_scanf/1_simple/ARM_EN}}
\RU{\subsubsection{ARM}

\myparagraph{\NonOptimizingKeilVI (\ARMMode)}

\lstinputlisting[style=customasmARM]{patterns/13_arrays/1_simple/simple_Keil_ARM_O0_RU.asm}

Тип \Tint требует 32 бита для хранения (или 4 байта),

так что для хранения 20 переменных типа \Tint, нужно 80 (\TT{0x50}) байт.

Поэтому инструкция \INS{SUB SP, SP, \#0x50} 
в прологе функции выделяет в локальном стеке под массив именно столько места.

И в первом и во втором цикле итератор цикла \var{i} будет постоянно находиться в регистре \Reg{4}.

\myindex{ARM!Optional operators!LSL}
Число, которое нужно записать в массив, вычисляется так: $i*2$, и это эквивалентно 
сдвигу на 1 бит влево,\\
так что инструкция \INS{MOV R0, R4,LSL\#1} делает это.

\myindex{ARM!\Instructions!STR}
\INS{STR R0, [SP,R4,LSL\#2]} записывает содержимое \Reg{0} в массив.
Указатель на элемент массива вычисляется так: \ac{SP} указывает на начало массива, \Reg{4} это $i$.

Так что сдвигаем $i$ на 2 бита влево, что эквивалентно умножению на 4 
(ведь каждый элемент массива занимает 4 байта) и прибавляем это к адресу начала массива.

\myindex{ARM!\Instructions!LDR}
Во втором цикле используется обратная инструкция\\
\INS{LDR R2, [SP,R4,LSL\#2]}.
Она загружает из массива нужное значение и указатель на него вычисляется точно так же.

\myparagraph{\OptimizingKeilVI (\ThumbMode)}

\lstinputlisting[style=customasmARM]{patterns/13_arrays/1_simple/simple_Keil_thumb_O3_RU.asm}

Код для Thumb очень похожий.
\myindex{ARM!\Instructions!LSLS}
В Thumb имеются отдельные инструкции для битовых сдвигов (как \TT{LSLS}), 
вычисляющие и число для записи в массив и адрес каждого элемента массива.

Компилятор почему-то выделил в локальном стеке немного больше места, 
однако последние 4 байта не используются.

\myparagraph{\NonOptimizing GCC 4.9.1 (ARM64)}

\lstinputlisting[caption=\NonOptimizing GCC 4.9.1 (ARM64),style=customasmARM]{patterns/13_arrays/1_simple/ARM64_GCC491_O0_RU.s}

}
\IT{\subsubsection{ARM}

\myparagraph{\OptimizingKeilVI (\ThumbMode)}

\lstinputlisting[style=customasmARM]{patterns/04_scanf/1_simple/ARM_IDA.lst}

\myindex{\CLanguageElements!\Pointers}

Affinchè \scanf possa leggere l'input, necessita di un parametro ---puntatore ad un \Tint.
\Tint è 32-bit, quindi servono 4 byte per memorizzarlo da qualche parte in memoria, e entra perfettamente in un registro a 32-bit.
\myindex{IDA!var\_?}
Uno spazio per la variabile locale \GTT{x} è allocato nello stack e \IDA
lo ha chiamato \emph{var\_8}. Non è comunque necessario allocarlo in questo modo poichè \ac{SP} (\gls{stack pointer}) punta già a quella posizione e può essere usato direttamente.

Successivamente il valore di \ac{SP} è copiato nel registro \Reg{1} e sono passati, insieme alla format-string, a \scanf.

Le istruzioni \INS{PUSH/POP} si comportano diversamente in ARM rispetto a x86 (è il contrario). Sono sinonimi delle istruzioni \INS{STM/STMDB/LDM/LDMIA}.
E l'istruzione \INS{PUSH} innanzitutto scrive un valore nello stack, \emph{e poi} sottrae 4 allo \ac{SP}.
\INS{POP} innanzitutto aggiunge 4 allo \ac{SP}, \emph{e poi} legge un valore dallo stack
Quindi, dopo \INS{PUSH}, lo \ac{SP} punta ad uno spazio inutilizzato nello stackn stack.
E' usato da \scanf, e da \printf dopo.

\INS{LDMIA} significa \emph{Load Multiple Registers Increment address After each transfer}.
\INS{STMDB} significa \emph{Store Multiple Registers Decrement address Before each transfer}.

\myindex{ARM!\Instructions!LDR}
Questo valore, con l'aiuto dell'istruzione \INS{LDR} , viene poi spostato dallo stack al registro \Reg{1} per essere passato a \printf.

\myparagraph{ARM64}

\lstinputlisting[caption=\NonOptimizing GCC 4.9.1 ARM64,numbers=left,style=customasmARM]{patterns/04_scanf/1_simple/ARM64_GCC491_O0_IT.s}

Ci sono 32 byte allocati per lo stack frame, che è più' grande del necessario. Forse a causa di meccanismi di allineamento della memoria?
La parte più interessante è quella in cui trova spazio per la variabile $x$ nello stack frame (riga 22).
Perchè 28? Il compilatore ha in qualche modo deciso di piazzare questa variabile alla fine dello stack frame anzichè all'inizio.
L'indirizzo è passato a \scanf, che memorizzerà il valore immesso dall'utente nella memoria a quell'indirizzo.Si tratta di un valore a 32-bit di tipo \Tint.Il valore è recuperato successivamente a riga 27 e passato a \printf.

}
\DE{\subsubsection{ARM + \OptimizingXcodeIV (\ARMMode)}

\lstinputlisting[caption=\OptimizingXcodeIV (\ARMMode),label=ARM_leaf_example4,style=customasmARM]{patterns/14_bitfields/4_popcnt/ARM_Xcode_O3_DE.lst}

\myindex{ARM!\Instructions!TST}
\TST entspricht dem Befehl \TEST in x86.

\myindex{ARM!Optional operators!LSL}
\myindex{ARM!Optional operators!LSR}
\myindex{ARM!Optional operators!ASR}
\myindex{ARM!Optional operators!ROR}
\myindex{ARM!Optional operators!RRX}
\myindex{ARM!\Instructions!MOV}
\myindex{ARM!\Instructions!TST}
\myindex{ARM!\Instructions!CMP}
\myindex{ARM!\Instructions!ADD}
\myindex{ARM!\Instructions!SUB}
\myindex{ARM!\Instructions!RSB}
Wie bereits in~(\myref{shifts_in_ARM_mode}) besprochen gibt es zwei verschiedene
Schiebebefehle im ARM mode.
Zusätzlich gibt es aber noch die Suffixe
LSL (\emph{Logical Shift Left}), 
LSR (\emph{Logical Shift Right}), 
ASR (\emph{Arithmetic Shift Right}), 
ROR (\emph{Rotate Right}) und
RRX (\emph{Rotate Right with Extend}), die an Befehle wie \MOV, \TST,
\CMP, \ADD, \SUB, \RSB\footnote{\DataProcessingInstructionsFootNote} angehängt
werden können.

Diese Suffixe legen fest, wie und um wie viele Bits der zweite Operand
verschoben werden soll.

\myindex{ARM!\Instructions!TST}
\myindex{ARM!Optional operators!LSL}
Dadurch entspricht der Befehl \TT{\q{TST R1, R2,LSL R3}} hier 
$R1 \land (R2 \ll R3)$.

\subsubsection{ARM + \OptimizingXcodeIV (\ThumbTwoMode)}

\myindex{ARM!\Instructions!LSL.W}
\myindex{ARM!\Instructions!LSL}
Fast das gleiche, aber hier werden zwei \INS{LSL.W}/\TST Befehle anstelle eines
einzelnen \TST verwendet, da es im Thumb mode nicht möglich ist, den Suffix \LSL
direkt in \TST zu definieren.

\begin{lstlisting}[label=ARM_leaf_example5,style=customasmARM]
                MOV             R1, R0
                MOVS            R0, #0
                MOV.W           R9, #1
                MOVS            R3, #0
loc_2F7A
                LSL.W           R2, R9, R3
                TST             R2, R1
                ADD.W           R3, R3, #1
                IT NE
                ADDNE           R0, #1
                CMP             R3, #32
                BNE             loc_2F7A
                BX              LR
\end{lstlisting}

\subsubsection{ARM64 + \Optimizing GCC 4.9}
Betrachten wir ein 64-Bit-Beispiel, das wir bereits
kennen:\myref{popcnt_x64_example}.

\lstinputlisting[caption=\Optimizing GCC (Linaro) 4.8,style=customasmARM]{patterns/14_bitfields/4_popcnt/ARM64_GCC_O3_DE.s}
Das Ergebnis ist ähnlich dem was GCC für x64 erzeugt:\myref{shifts64_GCC_O3}.

\myindex{ARM!\Instructions!CSEL}
Der Befehl \CSEL steht für \q{Conditional SELect}.
Er wählt eine von zwei Variablen abhängig von den durch \TST gesetzen Flags aus
und kopiert deren Wert nach \RegW{2}, wo die Variable \q{rt} gespeichert wird.

\subsubsection{ARM64 + \NonOptimizing GCC 4.9}
Wieder werden wir hier mit dem bereits bekannten 64-Bit-Beispiel arbeiten:
\myref{popcnt_x64_example}.
Der Code ist umfangreicher als gewöhnlich.

\lstinputlisting[caption=\NonOptimizing GCC (Linaro) 4.8,style=customasmARM]{patterns/14_bitfields/4_popcnt/ARM64_GCC_O0_DE.s}

}
\FR{\input{patterns/04_scanf/1_simple/ARM_FR}}
\JA{\input{patterns/04_scanf/1_simple/ARM_JA}}
\PL{\subsubsection{ARM}

\myparagraph{\OptimizingKeilVI (\ThumbMode)}

\lstinputlisting[style=customasmARM]{patterns/04_scanf/1_simple/ARM_IDA.lst}

\myindex{\CLanguageElements!\Pointers}

Funkcja \scanf potrzebuje argumentu--- wskaźnika na \Tint, by mogła zapisać wycztaną wartość.
\Tint jest 32-bitowy i zmieści się idealnie do 32-bitowego rejestru.
\myindex{IDA!var\_?}
Miejsce na zmienną lokalną \GTT{x} jest zaalokowane na stosie, \IDA
oznaczyła przesunięcie względem \ac{SP} makrem \emph{var\_8}. Można by się bez niego obyć, gdyż \ac{SP} (\glslink{stack pointer}{wskaźnik stosu}) już pokazuje na to miejsce i mógły być użyty bezpośrednio.

Wartość z \ac{SP} jest kopiowany do rejestru \Reg{1} i razem z łańcuchem znaków formatu przekazywana jako argumenty do funkcji \scanf.

Instrukcja \INS{PUSH/POP} zachowuje się inaczej niż na x86 (odwrotnie).
Są synonimami instrukcji \\ \INS{STM/STMDB/LDM/LDMIA}.
\INS{PUSH} najpierw zapisuje wartość na stosie, \emph{a następnie} zmniejsza \ac{SP} o 4.
\INS{POP} najpierw dodaje 4 do \ac{SP}, \emph{a następnie} wczytuje wartość ze stosu.
Stąd po wykonaniu \INS{PUSH}, \ac{SP} pokazuje na nieużywane miejsce na stosie.
Zostanie ono wykorzystane przez \scanf, a następnie \printf do zapisania i wczytania zmiennej lokalnej.

\INS{LDMIA} oznacza \emph{Load Multiple Registers Increment address After each transfer}.
\INS{STMDB} oznacza \emph{Store Multiple Registers Decrement address Before each transfer}.

\myindex{ARM!\Instructions!LDR}
Później, za pomocą instrukcji \INS{LDR}, wartość zmiennej lokalnej jest wczytywana ze stosu do rejestru \Reg{1}, by następnie zostać przekazana do funkcji \printf.

\myparagraph{ARM64}

\lstinputlisting[caption=\NonOptimizing GCC 4.9.1 ARM64,numbers=left,style=customasmARM]{patterns/04_scanf/1_simple/ARM64_GCC491_O0_PL.s}

Na ramkę stosu zaalokowano 32 bajty, a więc więcej niż to konieczne. Być może jest to efekty wyrównywania pamięci?
Najciekawszym fragmetem jest szukanie położenia zmiennej $x$ w obrębie ramki stosu (linia 22).
Dlaczego 28? Z jakiegoś powodu kompilator zdecydował umieścić zmienną na końcu ramki stosu, a nie na początku.
Adres jest przekazywany do funkcji \scanf, która umieszcza pod tym adresem wartość wpisaną przez użytkownika.
Jest to 32-bitowa wartość typu \Tint.
Wartość jest pobierana w linii 27 a następnie przekazywana do funkcji \printf.

}

\EN{\mysection{Task manager practical joke (Windows Vista)}
\myindex{Windows!Windows Vista}

Let's see if it's possible to hack Task Manager slightly so it would detect more \ac{CPU} cores.

\myindex{Windows!NTAPI}

Let us first think, how does the Task Manager know the number of cores?

There is the \TT{GetSystemInfo()} win32 function present in win32 userspace which can tell us this.
But it's not imported in \TT{taskmgr.exe}.

There is, however, another one in \gls{NTAPI}, \TT{NtQuerySystemInformation()}, 
which is used in \TT{taskmgr.exe} in several places.

To get the number of cores, one has to call this function with the \TT{SystemBasicInformation} constant
as a first argument (which is zero
\footnote{\href{http://msdn.microsoft.com/en-us/library/windows/desktop/ms724509(v=vs.85).aspx}{MSDN}}).

The second argument has to point to the buffer which is getting all the information.

So we have to find all calls to the \\
\TT{NtQuerySystemInformation(0, ?, ?, ?)} function.
Let's open \TT{taskmgr.exe} in IDA. 
\myindex{Windows!PDB}

What is always good about Microsoft executables is that IDA can download the corresponding \gls{PDB} 
file for this executable and show all function names.

It is visible that Task Manager is written in \Cpp and some of the function names and classes are really 
speaking for themselves.
There are classes CAdapter, CNetPage, CPerfPage, CProcInfo, CProcPage, CSvcPage, 
CTaskPage, CUserPage.

Apparently, each class corresponds to each tab in Task Manager.

Let's visit each call and add comment with the value which is passed as the first function argument.
We will write \q{not zero} at some places, because the value there was clearly not zero, 
but something really different (more about this in the second part of this chapter).

And we are looking for zero passed as argument, after all.

\begin{figure}[H]
\centering
\myincludegraphics{examples/taskmgr/IDA_xrefs.png}
\caption{IDA: cross references to NtQuerySystemInformation()}
\end{figure}

Yes, the names are really speaking for themselves.

When we closely investigate each place where\\
\TT{NtQuerySystemInformation(0, ?, ?, ?)} is called,
we quickly find what we need in the \TT{InitPerfInfo()} function:

\lstinputlisting[caption=taskmgr.exe (Windows Vista),style=customasmx86]{examples/taskmgr/taskmgr.lst}

\TT{g\_cProcessors} is a global variable, and this name has been assigned by 
IDA according to the \gls{PDB} loaded from Microsoft's symbol server.

The byte is taken from \TT{var\_C20}. 
And \TT{var\_C58} is passed to\\
\TT{NtQuerySystemInformation()} 
as a pointer to the receiving buffer.
The difference between 0xC20 and 0xC58 is 0x38 (56).

Let's take a look at format of the return structure, which we can find in MSDN:

\begin{lstlisting}[style=customc]
typedef struct _SYSTEM_BASIC_INFORMATION {
    BYTE Reserved1[24];
    PVOID Reserved2[4];
    CCHAR NumberOfProcessors;
} SYSTEM_BASIC_INFORMATION;
\end{lstlisting}

This is a x64 system, so each PVOID takes 8 bytes.

All \emph{reserved} fields in the structure take $24+4*8=56$ bytes.

Oh yes, this implies that \TT{var\_C20} is the local stack is exactly the
\TT{NumberOfProcessors} field of the \TT{SYSTEM\_BASIC\_INFORMATION} structure.

Let's check our guess.
Copy \TT{taskmgr.exe} from \TT{C:\textbackslash{}Windows\textbackslash{}System32} 
to some other folder 
(so the \emph{Windows Resource Protection} 
will not try to restore the patched \TT{taskmgr.exe}).

Let's open it in Hiew and find the place:

\begin{figure}[H]
\centering
\myincludegraphics{examples/taskmgr/hiew2.png}
\caption{Hiew: find the place to be patched}
\end{figure}

Let's replace the \TT{MOVZX} instruction with ours.
Let's pretend we've got 64 CPU cores.

Add one additional \ac{NOP} (because our instruction is shorter than the original one):

\begin{figure}[H]
\centering
\myincludegraphics{examples/taskmgr/hiew1.png}
\caption{Hiew: patch it}
\end{figure}

And it works!
Of course, the data in the graphs is not correct.

At times, Task Manager even shows an overall CPU load of more than 100\%.

\begin{figure}[H]
\centering
\myincludegraphics{examples/taskmgr/taskmgr_64cpu_crop.png}
\caption{Fooled Windows Task Manager}
\end{figure}

The biggest number Task Manager does not crash with is 64.

Apparently, Task Manager in Windows Vista was not tested on computers with a large number of cores.

So there are probably some static data structure(s) inside it limited to 64 cores.

\subsection{Using LEA to load values}
\label{TaskMgr_LEA}

Sometimes, \TT{LEA} is used in \TT{taskmgr.exe} instead of \TT{MOV} to set the first argument of \\
\TT{NtQuerySystemInformation()}:

\lstinputlisting[caption=taskmgr.exe (Windows Vista),style=customasmx86]{examples/taskmgr/taskmgr2.lst}

\myindex{x86!\Instructions!LEA}

Perhaps \ac{MSVC} did so because machine code of \INS{LEA} is shorter than \INS{MOV REG, 5} (would be 5 instead of 4).

\INS{LEA} with offset in $-128..127$ range (offset will occupy 1 byte in opcode) with 32-bit registers is even shorter (for lack of REX prefix)---3 bytes.

Another example of such thing is: \myref{using_MOV_and_pack_of_LEA_to_load_values}.
}
\RU{\subsubsection{std::string}
\myindex{\Cpp!STL!std::string}
\label{std_string}

\myparagraph{Как устроена структура}

Многие строковые библиотеки \InSqBrackets{\CNotes 2.2} обеспечивают структуру содержащую ссылку 
на буфер собственно со строкой, переменную всегда содержащую длину строки 
(что очень удобно для массы функций \InSqBrackets{\CNotes 2.2.1}) и переменную содержащую текущий размер буфера.

Строка в буфере обыкновенно оканчивается нулем: это для того чтобы указатель на буфер можно было
передавать в функции требующие на вход обычную сишную \ac{ASCIIZ}-строку.

Стандарт \Cpp не описывает, как именно нужно реализовывать std::string,
но, как правило, они реализованы как описано выше, с небольшими дополнениями.

Строки в \Cpp это не класс (как, например, QString в Qt), а темплейт (basic\_string), 
это сделано для того чтобы поддерживать 
строки содержащие разного типа символы: как минимум \Tchar и \emph{wchar\_t}.

Так что, std::string это класс с базовым типом \Tchar.

А std::wstring это класс с базовым типом \emph{wchar\_t}.

\mysubparagraph{MSVC}

В реализации MSVC, вместо ссылки на буфер может содержаться сам буфер (если строка короче 16-и символов).

Это означает, что каждая короткая строка будет занимать в памяти по крайней мере $16 + 4 + 4 = 24$ 
байт для 32-битной среды либо $16 + 8 + 8 = 32$ 
байта в 64-битной, а если строка длиннее 16-и символов, то прибавьте еще длину самой строки.

\lstinputlisting[caption=пример для MSVC,style=customc]{\CURPATH/STL/string/MSVC_RU.cpp}

Собственно, из этого исходника почти всё ясно.

Несколько замечаний:

Если строка короче 16-и символов, 
то отдельный буфер для строки в \glslink{heap}{куче} выделяться не будет.

Это удобно потому что на практике, основная часть строк действительно короткие.
Вероятно, разработчики в Microsoft выбрали размер в 16 символов как разумный баланс.

Теперь очень важный момент в конце функции main(): мы не пользуемся методом c\_str(), тем не менее,
если это скомпилировать и запустить, то обе строки появятся в консоли!

Работает это вот почему.

В первом случае строка короче 16-и символов и в начале объекта std::string (его можно рассматривать
просто как структуру) расположен буфер с этой строкой.
\printf трактует указатель как указатель на массив символов оканчивающийся нулем и поэтому всё работает.

Вывод второй строки (длиннее 16-и символов) даже еще опаснее: это вообще типичная программистская ошибка 
(или опечатка), забыть дописать c\_str().
Это работает потому что в это время в начале структуры расположен указатель на буфер.
Это может надолго остаться незамеченным: до тех пока там не появится строка 
короче 16-и символов, тогда процесс упадет.

\mysubparagraph{GCC}

В реализации GCC в структуре есть еще одна переменная --- reference count.

Интересно, что указатель на экземпляр класса std::string в GCC указывает не на начало самой структуры, 
а на указатель на буфера.
В libstdc++-v3\textbackslash{}include\textbackslash{}bits\textbackslash{}basic\_string.h 
мы можем прочитать что это сделано для удобства отладки:

\begin{lstlisting}
   *  The reason you want _M_data pointing to the character %array and
   *  not the _Rep is so that the debugger can see the string
   *  contents. (Probably we should add a non-inline member to get
   *  the _Rep for the debugger to use, so users can check the actual
   *  string length.)
\end{lstlisting}

\href{http://gcc.gnu.org/onlinedocs/libstdc++/libstdc++-html-USERS-4.4/a01068.html}{исходный код basic\_string.h}

В нашем примере мы учитываем это:

\lstinputlisting[caption=пример для GCC,style=customc]{\CURPATH/STL/string/GCC_RU.cpp}

Нужны еще небольшие хаки чтобы сымитировать типичную ошибку, которую мы уже видели выше, из-за
более ужесточенной проверки типов в GCC, тем не менее, printf() работает и здесь без c\_str().

\myparagraph{Чуть более сложный пример}

\lstinputlisting[style=customc]{\CURPATH/STL/string/3.cpp}

\lstinputlisting[caption=MSVC 2012,style=customasmx86]{\CURPATH/STL/string/3_MSVC_RU.asm}

Собственно, компилятор не конструирует строки статически: да в общем-то и как
это возможно, если буфер с ней нужно хранить в \glslink{heap}{куче}?

Вместо этого в сегменте данных хранятся обычные \ac{ASCIIZ}-строки, а позже, во время выполнения, 
при помощи метода \q{assign}, конструируются строки s1 и s2
.
При помощи \TT{operator+}, создается строка s3.

Обратите внимание на то что вызов метода c\_str() отсутствует,
потому что его код достаточно короткий и компилятор вставил его прямо здесь:
если строка короче 16-и байт, то в регистре EAX остается указатель на буфер,
а если длиннее, то из этого же места достается адрес на буфер расположенный в \glslink{heap}{куче}.

Далее следуют вызовы трех деструкторов, причем, они вызываются только если строка длиннее 16-и байт:
тогда нужно освободить буфера в \glslink{heap}{куче}.
В противном случае, так как все три объекта std::string хранятся в стеке,
они освобождаются автоматически после выхода из функции.

Следовательно, работа с короткими строками более быстрая из-за м\'{е}ньшего обращения к \glslink{heap}{куче}.

Код на GCC даже проще (из-за того, что в GCC, как мы уже видели, не реализована возможность хранить короткую
строку прямо в структуре):

% TODO1 comment each function meaning
\lstinputlisting[caption=GCC 4.8.1,style=customasmx86]{\CURPATH/STL/string/3_GCC_RU.s}

Можно заметить, что в деструкторы передается не указатель на объект,
а указатель на место за 12 байт (или 3 слова) перед ним, то есть, на настоящее начало структуры.

\myparagraph{std::string как глобальная переменная}
\label{sec:std_string_as_global_variable}

Опытные программисты на \Cpp знают, что глобальные переменные \ac{STL}-типов вполне можно объявлять.

Да, действительно:

\lstinputlisting[style=customc]{\CURPATH/STL/string/5.cpp}

Но как и где будет вызываться конструктор \TT{std::string}?

На самом деле, эта переменная будет инициализирована даже перед началом \main.

\lstinputlisting[caption=MSVC 2012: здесь конструируется глобальная переменная{,} а также регистрируется её деструктор,style=customasmx86]{\CURPATH/STL/string/5_MSVC_p2.asm}

\lstinputlisting[caption=MSVC 2012: здесь глобальная переменная используется в \main,style=customasmx86]{\CURPATH/STL/string/5_MSVC_p1.asm}

\lstinputlisting[caption=MSVC 2012: эта функция-деструктор вызывается перед выходом,style=customasmx86]{\CURPATH/STL/string/5_MSVC_p3.asm}

\myindex{\CStandardLibrary!atexit()}
В реальности, из \ac{CRT}, еще до вызова main(), вызывается специальная функция,
в которой перечислены все конструкторы подобных переменных.
Более того: при помощи atexit() регистрируется функция, которая будет вызвана в конце работы программы:
в этой функции компилятор собирает вызовы деструкторов всех подобных глобальных переменных.

GCC работает похожим образом:

\lstinputlisting[caption=GCC 4.8.1,style=customasmx86]{\CURPATH/STL/string/5_GCC.s}

Но он не выделяет отдельной функции в которой будут собраны деструкторы: 
каждый деструктор передается в atexit() по одному.

% TODO а если глобальная STL-переменная в другом модуле? надо проверить.

}
\IT{\subsubsection{MIPS}

Nello stack locale viene allocato spazio per la variabile $x$ , a cui viene fatto riferimento come $\$sp+24$.
\myindex{MIPS!\Instructions!LW}

Il suo indirizzo è passato a \scanf, il valore immesso dall'utente è caricato usand l'istruzione \INS{LW} (\q{Load Word}) ed è infine passato a \printf.

\lstinputlisting[caption=\Optimizing GCC 4.4.5 (\assemblyOutput),style=customasmMIPS]{patterns/04_scanf/1_simple/MIPS/ex1.O3_IT.s}

IDA mostra il layout dello stack nel modo seguente:

\lstinputlisting[caption=\Optimizing GCC 4.4.5 (IDA),style=customasmMIPS]{patterns/04_scanf/1_simple/MIPS/ex1.O3.IDA_IT.lst}

% TODO non-optimized version?
}
\DE{\subsection{Text strings}

\subsubsection{\CCpp}

\label{C_strings}

Die normalen C-strings sind NULL-Terminiert (\ac{ASCIIZ}-strings).

Der Grund warum C Stringformatierung so ist wie sie ist (NULL-Terminiert) scheint ein Historischer zu sein.
In [Dennis M. Ritchie, \emph{The Evolution of the Unix Time-sharing System}, (1979)] kann man nach lesen:

\begin{framed}
\begin{quotation}
Ein kleiner Unterschied war das die I/O Einheit ein ``word'' war, nicht ein Byte, weil die PDP-7 eine word-adressierte
Maschine war. In der Praxis bedeutete das lediglich das alle Programme die mit Zeichen Streams arbeiteten, das NULL 
Zeichen ignorieren mussten, weil die NULL benutzt wurde um eine Datei bis zu einer Graden Zahl an Bytes auf zu f\"ullen.

\end{quotation}
\end{framed}

\myindex{Hiew}

In Hiew oder FAR Manager sehen diese Strings so aus:

\begin{lstlisting}[style=customc]
int main()
{
	printf ("Hello, world!\n");
};
\end{lstlisting}

\begin{figure}[H]
\centering
\includegraphics[width=0.6\textwidth]{digging_into_code/strings/C-string.png}
\caption{Hiew}
\end{figure}

% FIXME видно \n в конце, потом пробел

\subsubsection{Borland Delphi}
\myindex{Pascal}
\myindex{Borland Delphi}

Dem String in Passcal und Borland Delphi h\"angt eine 8 oder 32-Bit Zeichenkette an. 

Zum Beispiel:

\begin{lstlisting}[caption=Delphi,style=customasmx86]
CODE:00518AC8                 dd 19h
CODE:00518ACC aLoading___Plea db 'Loading... , please wait.',0

...

CODE:00518AFC                 dd 10h
CODE:00518B00 aPreparingRun__ db 'Preparing run...',0
\end{lstlisting}

\subsubsection{Unicode}

\myindex{Unicode}

Oft, ist das was Unicode genannt wird einfach eine Methode um Strings zu codieren, bei denen jedes Zeichen 2 Byte oder 
16 Bits verbraucht. Das ist ein h\a"ufiger Terminologischer Fehler. Unicode ist ein Standard bei dem eine Nummer 
zu einem der vielen Schreibsysteme der Welt zugeordnet wird, aber es beschreibt nicht die codierungs Methode. 

\myindex{UTF-8}
\myindex{UTF-16LE}

Die bekannteste Methode zu Codieren ist: UTF-8 ( ist weit verteilt im Internet und auf *NIX Systemen) und UTF-16LE ( wird bei Windows benutzt). 

\myparagraph{UTF-8}

\myindex{UTF-8}
UTF-8 ist eine der erfolgreichsten Methoden um Zeichen zu codieren.
Alle Latein Zeichen werden codiert so wie in ASCII, und alle Symbole nach der
ASCII Tabelle wurden codiert mit zus\"atzlichen Bytes. 0 wird codiert als davor,
also arbeiten alle Standard C String Funktionen mit UTF-8 Strings wie mit jedem anderen String auch.

Lasst uns anschauen wie die Symbole in verschiedenen anderen Sprachen nach UTF-8 Codiert werden und 
wie man sie als FAR aussehen lassen kann, durch das benutzen der codepage 437.

\footnote{Beispiel und \"Ubersetzung k\o"nnen von hier bezogen werden:  
\url{http://www.columbia.edu/~fdc/utf8/}}:

\begin{figure}[H]
\centering
\includegraphics[width=0.6\textwidth]{digging_into_code/strings/multilang_sampler.png}
\end{figure}

% FIXME: cut it
\begin{figure}[H]
\centering
\myincludegraphics{digging_into_code/strings/multilang_sampler_UTF8.png}
\caption{FAR: UTF-8}
\end{figure}

Wie man hier sehen kann, der Englische String sieht genauso aus wie sein Gegenst\"uck in ASCII.

Die Ungarische Sprache benutzt Latein Symbole plus ein paar Symbole mit diacritic Markierungen.

Diese Symbole werden mit mehreren Bytes codiert, diese wurden rot unterstrichen.
Das gleiche gilt f\"ur die Isl\"andischen und Polnischen Sprachen.

Es gibt auch das \q{Euro} W\"ahrungs Symbol im Standard, das Symbol wurde mit 3 Bytes Codiert.

Der Rest der Schreibsysteme hat keinen Bezug zu Latein.

Zumindest in Russisch, Arabisch, Hebr\"aisch und Hindu k\"onnen wir wiederkehrende Bytes erkennen und das ist nicht mal \"uberraschend:
Alle Zeichen eines Schreibsystems werden normalerweise in der selben Unicode Tabelle angelegt, also f\"angt ihr code mit den 
immer gleichen nummern an. % <--- Wird anders \"ubersetzt.

Zu Anfang, noch vor dem \q{How much?} String sehen wir 3 Bytes, die tats\"achlich das \ac{BOM} darstellen.
Das \ac{BOM} definiert das Codierungssystem das benutzt werden soll.

\myparagraph{UTF-16LE}

\myindex{UTF-16LE}
\myindex{Windows!Win32}
Viele win32 Funktionen in Windows haben die Suffixe \TT{-A} und \TT{-W}. 
Der erste Typ Funktionen arbeitet mit normalen Strings, der andere Typ mit 
UTF-16LE Strings (\emph{wide}). 

Im zweiten Fall, wird jedes Symbol normal als 16-Bit Wert des Typs \emph{short} gespeichert.

Die Latein Symbole in UFT-16 Strings sehen in Hiew oder FAR aus als w\"aren sie mit Null Bytes verschachtelt:

\begin{lstlisting}[style=customc]
int wmain()
{
	wprintf (L"Hello, world!\n");
};
\end{lstlisting}

\begin{figure}[H]
\centering
\includegraphics[width=0.6\textwidth]{digging_into_code/strings/UTF16-string.png}
\caption{Hiew}
\end{figure}

Wir k\o"nnen das oft auch in gls{Windows NT} System Dateien sehen:

\begin{figure}[H]
\centering
\includegraphics[width=0.6\textwidth]{digging_into_code/strings/ntoskrnl_UTF16.png}
\caption{Hiew}
\end{figure}

\myindex{IDA}
Strings mit Zeichen die exakt 2 Bytes verbrauchen werden \q{Unicode} in \IDA genannt:

\begin{lstlisting}[style=customasmx86]
.data:0040E000 aHelloWorld:
.data:0040E000                 unicode 0, <Hello, world!>
.data:0040E000                 dw 0Ah, 0
\end{lstlisting}

Hier sieht man wie Russische Sprache in UTF-16LE Codiert wird:

\begin{figure}[H]
\centering
\includegraphics[width=0.6\textwidth]{digging_into_code/strings/russian_UTF16.png}
\caption{Hiew: UTF-16LE}
\end{figure}

Was man leicht sehen kann ist das die Symbole durchzogen sind von den Diamant Zeichen (das im ASCII code mit 4 codiert wird).
Tats\"achlich, findet man die Kyrillischen Symbole in der vierten Unicode Tabelle.
Deswegen, alle Kyrillischen Symbole in UTF-16LE findet man im Bereich \TT{0x400-0x4FF}.

Lass uns noch mal zu dem Beispiel gehen mit dem String der in verschiedenen Sprachen geschrieben ist.
Hier sieht man wie der String in UTF-16LE aussieht. 

% FIXME: cut it
\begin{figure}[H]
\centering
\myincludegraphics{digging_into_code/strings/multilang_sampler_UTF16.png}
\caption{FAR: UTF-16LE}
\end{figure}

Hier k\"onnen wir auch das \ac{BOM} am Anfang sehen. 
Alle Latein Zeichen enthalten Null Bytes.

Manche Zeichen mit unterschiedlichen Markierungen (Ungarisch und Isl\"andisch) wurden rot unterstrichen.

% subsection:
\input{digging_into_code/strings/base64_DE}

}
\FR{\subsection{Table \TT{X\$KSMLRU} dans \oracle}
\myindex{\oracle}

Il y a une mention d'une table spéciale dans la note \emph{Diagnosing and Resolving
Error ORA-04031 on the Shared Pool or Other Memory Pools [Video] [ID 146599.1]}:

\begin{framed}
\begin{quotation}
Il y a une table fixée appelée X\$KSMLRU qui suit les différentes allocations dans
le pool partagé qui force les autres objets du pool partagé à vieillir. Cette table
fixée peut être utilisée pour identifier ce qui cause une grosse allocation.

Si plusieurs objets sont supprimés périodiquement du pool partagé, alors ceci va poser des problèmes de temps de réponse
et va probablement provoquer des problèmes de contention du verrou de cache de bibliothèque lorsque
les objets seront rechargés dans le pool partagé.

Une chose inhabituelle à propos de la table fixée X\$KSMLRU est que le contenu de la table fixée est écrasé à chaque fois que
quelqu'uni effectue requête dans la table fixée. Ceci est fait puisque la table fixée ne contient que l'allocation la plus large
qui s'est produite. Les valeurs sont réinitialisées après avoir été sélectionnées, de sorte que les allocations importantes
suivantes puissent ête inscrites, même si elles ne sont pas aussi laarges que celles qui se sont produites précédemment.
À cause de cette réinitialisation, la sortie produite par la sélection de cette table doit être soigneusement conservée
puisqu'elle ne peut plus être récupérée après que la requête a été faite.
\end{quotation}
\end{framed}

Toutefois, comme on peut le vérifier facilement, le contenu de cette table est effacé à chaque fois qu'on l'interroge.
Pouvons-nous trouver pourquoi?
Retournons aux tables que nous connaissons déjà: \TT{kqftab} et \TT{kqftap} qui
sont générées avec l'aide d'\oracletables, qui a toutes les informations concernant les table X\$-.
Nous pouvons voir ici que la fonction \TT{ksmlrs()} est appelée pour préparer les éléments de cette table:

\begin{lstlisting}[caption=Résultat de \OracleTablesName]
kqftab_element.name: [X$KSMLRU] ?: [ksmlr] 0x4 0x64 0x11 0xc 0xffffc0bb 0x5
kqftap_param.name=[ADDR] ?: 0x917 0x0 0x0 0x0 0x4 0x0 0x0
kqftap_param.name=[INDX] ?: 0xb02 0x0 0x0 0x0 0x4 0x0 0x0
kqftap_param.name=[INST_ID] ?: 0xb02 0x0 0x0 0x0 0x4 0x0 0x0
kqftap_param.name=[KSMLRIDX] ?: 0xb02 0x0 0x0 0x0 0x4 0x0 0x0
kqftap_param.name=[KSMLRDUR] ?: 0xb02 0x0 0x0 0x0 0x4 0x4 0x0
kqftap_param.name=[KSMLRSHRPOOL] ?: 0xb02 0x0 0x0 0x0 0x4 0x8 0x0
kqftap_param.name=[KSMLRCOM] ?: 0x501 0x0 0x0 0x0 0x14 0xc 0x0
kqftap_param.name=[KSMLRSIZ] ?: 0x2 0x0 0x0 0x0 0x4 0x20 0x0
kqftap_param.name=[KSMLRNUM] ?: 0x2 0x0 0x0 0x0 0x4 0x24 0x0
kqftap_param.name=[KSMLRHON] ?: 0x501 0x0 0x0 0x0 0x20 0x28 0x0
kqftap_param.name=[KSMLROHV] ?: 0xb02 0x0 0x0 0x0 0x4 0x48 0x0
kqftap_param.name=[KSMLRSES] ?: 0x17 0x0 0x0 0x0 0x4 0x4c 0x0
kqftap_param.name=[KSMLRADU] ?: 0x2 0x0 0x0 0x0 0x4 0x50 0x0
kqftap_param.name=[KSMLRNID] ?: 0x2 0x0 0x0 0x0 0x4 0x54 0x0
kqftap_param.name=[KSMLRNSD] ?: 0x2 0x0 0x0 0x0 0x4 0x58 0x0
kqftap_param.name=[KSMLRNCD] ?: 0x2 0x0 0x0 0x0 0x4 0x5c 0x0
kqftap_param.name=[KSMLRNED] ?: 0x2 0x0 0x0 0x0 0x4 0x60 0x0
kqftap_element.fn1=ksmlrs
kqftap_element.fn2=NULL
\end{lstlisting}

\myindex{tracer}
En effet, avec l'aide de \tracer, il est facile de voir que cette fonction est appelée
à chaque fois que nous interrogeons la table \TT{X\$KSMLRU}.

\myindex{\CStandardLibrary!memset()}
Ici nous voyons une référence aux fonctions \TT{ksmsplu\_sp()} et \TT{ksmsplu\_jp()},
chacune d'elles appelle \TT{ksmsplu()} à la fin.
À la fin de la fonction \TT{ksmsplu()} nous voyons un appel à \TT{memset()}:

\begin{lstlisting}[caption=ksm.o,style=customasmx86]
...

.text:00434C50 loc_434C50:    ; DATA XREF: .rdata:off\_5E50EA8
.text:00434C50         mov     edx, [ebp-4]
.text:00434C53         mov     [eax], esi
.text:00434C55         mov     esi, [edi]
.text:00434C57         mov     [eax+4], esi
.text:00434C5A         mov     [edi], eax
.text:00434C5C         add     edx, 1
.text:00434C5F         mov     [ebp-4], edx
.text:00434C62         jnz     loc_434B7D
.text:00434C68         mov     ecx, [ebp+14h]
.text:00434C6B         mov     ebx, [ebp-10h]
.text:00434C6E         mov     esi, [ebp-0Ch]
.text:00434C71         mov     edi, [ebp-8]
.text:00434C74         lea     eax, [ecx+8Ch]
.text:00434C7A         push    370h            ; Size
.text:00434C7F         push    0               ; Val
.text:00434C81         push    eax             ; Dst
.text:00434C82         call    __intel_fast_memset
.text:00434C87         add     esp, 0Ch
.text:00434C8A         mov     esp, ebp
.text:00434C8C         pop     ebp
.text:00434C8D         retn
.text:00434C8D _ksmsplu  endp
\end{lstlisting}

Des constructions comme \TT{memset (block, 0, size)} sont souvent utilisées pour
mettre à zéro un bloc de mémoire.
Que se passe-t-il si nous prenons le risque de bloquer l'appel à \TT{memset (block, 0, size)}
et regardons ce qui se produit?

\myindex{tracer}

Lançons \tracer avec les options suivantes: mettre un point d'arrêt en \TT{0x434C7A}
(le point où les arguments sont passés à \TT{memset()}), afin que \tracer mette le compteur de programme \TT{EIP} au point
où les arguments passés à \TT{memset()} sont éffacés (en \TT{0x434C8A}).
On peut dire que nous simulons juste un saut inconditionnel de l'adresse \TT{0x434C7A} à \TT{0x434C8A}.

\begin{lstlisting}
tracer -a:oracle.exe bpx=oracle.exe!0x00434C7A,set(eip,0x00434C8A)
\end{lstlisting}

(Important: toutes ces adresses sont valides seulement pour la version win32 de \oracle 11.2)

En effet, nous pouvons maintenant interroger la table \TT{X\$KSMLRU} autant de fois
que nous voulons et elle n'est plus du tout effacée!

% \sout{Do not try this at home ("MythBusters")}
Au cas où, n'essayez pas ceci sur vos serveurs de production.

Ce n'est probablement pas un comportement très utile ou souhaité, mais comme une
expérience pour déterminer l'emplacement d'un bout de code dont nous avons besoin,
ça remplit parfaitement notre besoin!

}
\JA{\subsection{文字列へのポインタの配列}
\label{array_of_pointers_to_strings}

ここでは、ポインタの配列の例を示します。

\lstinputlisting[caption=Get month name,label=get_month1,style=customc]{patterns/13_arrays/45_month_1D/month1_JA.c}

\subsubsection{x64}

\lstinputlisting[caption=\Optimizing MSVC 2013 x64,style=customasmx86]{patterns/13_arrays/45_month_1D/month1_MSVC_2013_x64_Ox.asm}

コードはとても単純です。

\begin{itemize}

\item
\myindex{x86!\Instructions!MOVSXD}

最初の\INS{MOVSXD}命令は、 \ECX ( $month$ 引数が渡される)から32ビットの値を
符号拡張付きの \RAX ( $month$ 引数は \Tint 型なので)にコピーします。

符号拡張の理由は、この32ビット値が他の64ビット値との計算に使用されるためです。

したがって、64ビット142に昇格させる必要があります。%
\footnote{やや奇妙ですが、負の配列インデックスはここで $month$ として渡すことができます
(負の配列インデックスは後で説明します:\ref{negative_array_indices})。 % FIXME should be \myref{} here, but varioref package complains...
これが起こると、 \Tint 型の負の入力値が正しく符号拡張され、
テーブルの前の対応する要素が選択されます。 符号拡張なしでは正しく動作しません。}

\item
次にポインタテーブルのアドレスが \RCX にロードされます。

\item
最後に、入力値($month$)に8を掛けてアドレスに加算します。 
確かに:私たちは64ビット環境にあり、すべてのアドレス(またはポインタ)は正確に64ビット(または8バイト)の
記憶域を必要とします。 
したがって、各テーブル要素は8バイト幅です。 
それで、なぜ特定の要素 $month*8$ をスキップする必要があるのでしょうか。
これが \MOV が行うことです。 
さらに、この命令はこのアドレスの要素もロードします。 
1の場合、要素は\q{February}などを含む文字列へのポインタになります。

\end{itemize}

\Optimizing GCC 4.9はもっとよく仕事をこなします。
\footnote{GCCアセンブラ出力が排除するのに十分なほど整っていないので、\q{0+}がリストに残っていました。 
それは\emph{変位}であり、ここではゼロです。}

\begin{lstlisting}[caption=\Optimizing GCC 4.9 x64,style=customasmx86]
	movsx	rdi, edi
	mov	rax, QWORD PTR month1[0+rdi*8]
	ret
\end{lstlisting}

\myparagraph{32ビットMSVC}

32ビットMSVCコンパイラでもコンパイルしてみましょう。

\lstinputlisting[caption=\Optimizing MSVC 2013 x86,style=customasmx86]{patterns/13_arrays/45_month_1D/month1_MSVC_2013_x86_Ox.asm}

入力値は64ビットに拡張する必要がないので、そのまま使われます。

そして4倍されます。テーブル要素が32ビット(または4バイト)幅だからです。

% FIXME1 move to another file
\subsubsection{32ビット ARM}

\myparagraph{ARMモードでのARM}

\lstinputlisting[caption=\OptimizingKeilVI (\ARMMode),style=customasmARM]{patterns/13_arrays/45_month_1D/month1_Keil_ARM_O3.s}

% TODO Fix R1s

テーブルのアドレスはR1にロードされます。
\myindex{ARM!\Instructions!LDR}

残りのすべては \LDR 命令1つだけを使って行われます。

入力値 $month$ は2ビット左シフトします(4倍するのと同じです)。それから
R1に加えらえます(テーブルのアドレスの場所)。そしてテーブル要素はこのアドレスからロードされます。

32ビットテーブル要素はテーブルからR0にロードされます。

\myparagraph{ThumbモードでのARM}

コードはほとんど同じですが、より密度が低いです。 \LSL サフィックスは \LDR 命令では特定できないからです。

\begin{lstlisting}[style=customasmARM]
get_month1 PROC
        LSLS     r0,r0,#2
        LDR      r1,|L0.64|
        LDR      r0,[r1,r0]
        BX       lr
        ENDP
\end{lstlisting}

\subsubsection{ARM64}

\lstinputlisting[caption=\Optimizing GCC 4.9 ARM64,style=customasmARM]{patterns/13_arrays/45_month_1D/month1_GCC49_ARM64_O3.s}

\myindex{ARM!\Instructions!ADRP/ADD pair}

テーブルのアドレスは \ADRP/\ADD 命令の組を使ってX1にロードされます。

それから付随する要素 \LDR を使って選ばれて、W0を取ります(入力引数 $month$ の場所のレジスタ)。
左に3ビットシフトします(8倍するのと同じです)。
符号拡張し(\q{sxtw}サフィックスが暗示しています)、X0に加算します。
それから64ビット値がテーブルからX0にロードされます。

\subsubsection{MIPS}

\lstinputlisting[caption=\Optimizing GCC 4.4.5 (IDA),style=customasmMIPS]{patterns/13_arrays/45_month_1D/MIPS_O3_IDA_JA.lst}

\subsubsection{配列オーバーフロー}

関数は0~11の範囲の値を受け付けますが、12は通すでしょうか?
テーブルにはその場所の要素はありません。

なので関数はそこにたまたまある値をロードしてリターンします。

すぐ後で、他の関数がこのアドレスからテキスト文字列を取得しようとしてクラッシュするかもしれません。

例をwin64用としてMSVCでコンパイルして、テーブルの後にリンカーが何を配置したのかを \IDA で見てみましょう。

\lstinputlisting[caption=IDAでの実行可能ファイル,style=customasmx86]{patterns/13_arrays/45_month_1D/MSVC2012_win64_1.lst}

月の名前がそのあとに来ています。

プログラムは小さいので、データセグメントにパックされるデータは多くありません。
だから単に次の名前が来ています。
しかし注意すべきはリンカーが配置するように決定するのは\emph{どんなものも}ありえます。

だからもし12が関数に渡されたら?
13番目の要素がリターンされます。

CPUがそこにあるバイトを64ビットの値としてどのように扱うかをみてみましょう。

\lstinputlisting[caption=IDAでの実行可能ファイル,style=customasmx86]{patterns/13_arrays/45_month_1D/MSVC2012_win64_2.lst}

0x797261756E614Aです。

すぐ後で、他の関数(おそらく文字列を扱う関数)がこのアドレスでバイトを読み込もうとすると、
C言語の文字列を期待します。

十中八九、クラッシュします。この値は有効なアドレスのようには見えないからです。

\myparagraph{配列オーバーフロー保護}

\epigraph{失敗する可能性のあるものは、失敗する。}{マーフィーの法則}

あなたの関数を使用するプログラマはみな11より大きな値を引数として渡さないと
期待するのはちょっとナイーブです。

問題をできるだけ早く報告し停止することを意味する\q{fail early and fail loudly}
または\q{早く失敗する}という哲学があります。

\myindex{\CStandardLibrary!assert()}

そのような方法の1つに \CCpp のassertionがあります。

不正な値が通ってきたら、失敗するようにプログラムを変更できます。

\lstinputlisting[caption=assert()を追加,style=customc]{patterns/13_arrays/45_month_1D/month1_assert.c}

アサーションマクロは関数の開始時に妥当な値かチェックし、式が偽の場合に失敗します。

\lstinputlisting[caption=\Optimizing MSVC 2013 x64,style=customasmx86]{patterns/13_arrays/45_month_1D/MSVC2013_x64_Ox_checked.asm}

実際、assert() は関数ではなくマクロです。条件をチェックし、
行数とファイル名を他の関数に渡してユーザに情報を報告します。

ファイル名と条件の両方がUTF-16でエンコードされています。
行数も渡されます(29です)。

このメカニズムはおそらくすべてのコンパイラで同じです。
GCCはこのようにします。

\lstinputlisting[caption=\Optimizing GCC 4.9 x64,style=customasmx86]{patterns/13_arrays/45_month_1D/GCC491_x64_O3_checked.s}

GCCのマクロは利便性のために関数名も渡します。

何事もただではできませんが、サニタイズチェックもこれと同様です。

それはプログラムを遅くしますが、特にassert()マクロが小さなタイムクリティカルな関数で使用されると遅くなります。

なのでMSVCでは、例えばデバッグビルドではチェックを残し、リリースビルドでは取り除いたりします。
 
マイクロソフト\gls{Windows NT}カーネルは\q{チェックされた}と\q{フリー}ビルドです。
\footnote{\href{http://msdn.microsoft.com/en-us/library/windows/hardware/ff543450(v=vs.85).aspx}{msdn.microsoft.com/en-us/library/windows/hardware/ff543450(v=vs.85).aspx}}.

最初のものは妥当性チェック(\q{チェックされた}なので)があり、もう一つはチェックしていません(チェックが\q{フリー}なので)。

もちろん、 \q{チェックされた}カーネルはこれらのチェックのために遅く動作するので、通常はデバッグセッションでのみ使用されます。

% FIXME: ARM? MIPS?

\subsubsection{特定の文字へのアクセス}

文字列へのポインタの配列はこのようにアクセスできます。

\lstinputlisting[style=customc]{patterns/13_arrays/45_month_1D/month2_JA.c}

\dots \emph{month[3]}式は\emph{const char*}型をもつので、
5番目の文字列はこのアドレスに4バイトを足した式から取得します。

さて、\emph{main()}関数に渡された引数リストは同じデータ型を持ちます。

\lstinputlisting[style=customc]{patterns/13_arrays/45_month_1D/argv_JA.c}

似た構文ですが、2次元配列とは異なることを理解することが非常に重要です。
これについては後で検討します。

もう1つの重要なことに注意してください。アドレス指定される文字列は、各文字が\ac{ASCII}や拡張\ac{ASCII}のように1バイトを占めるシステムで
エンコードされなければなりません。 
UTF-8はここでは動作しません。
}
\PL{\subsubsection{MIPS}

Na stosie lokalnym zaalokowano miejsce dla zmiennej $x$, odwoływać będziemy się do niej przez $\$sp+24$.
\myindex{MIPS!\Instructions!LW}

Adres zmiennej przekazywany jest do funkcji \scanf. Wartość wpisana przez użytkownika i odczytana za pomocą \scanf jest następnie wczytywana za pomocą instrukcji \INS{LW} (\q{Load Word}) i przekazywana do \printf.

\lstinputlisting[caption=\Optimizing GCC 4.4.5 (\assemblyOutput),style=customasmMIPS]{patterns/04_scanf/1_simple/MIPS/ex1.O3_PL.s}

\IDA wyświetla układ stosu następująco:

\lstinputlisting[caption=\Optimizing GCC 4.4.5 (IDA),style=customasmMIPS]{patterns/04_scanf/1_simple/MIPS/ex1.O3.IDA_PL.lst}

% TODO non-optimized version?
}

