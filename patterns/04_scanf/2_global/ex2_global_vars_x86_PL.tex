\subsubsection{MSVC: x86}

\lstinputlisting[style=customasmx86]{patterns/04_scanf/2_global/ex2_MSVC.asm}

W tym przypadku zmienna \TT{x} jest zdefiniowana w segmencie \TT{\_DATA} i nie następuje alokacja pamięci na stosie lokalnym. Dostęp do zmiennej jest bezpośredni, z pominięciem stosu.
Niezainicjalizowane zmienne globalne nie zajmują miejsca w pliku wykonywalnym (po co alokować wyzerowaną pamięć?), dopiero w momencie odwołania pod adres zmiennej, \ac{OS} alokuje blok pamięci, wypełniony zerami.

Przypiszmy teraz wprost wartość do zmiennej:

\lstinputlisting[style=customc]{patterns/04_scanf/2_global/default_value_PL.c}

Otrzymamy:

\begin{lstlisting}[style=customasmx86]
_DATA	SEGMENT
_x	DD	0aH

...
\end{lstlisting}

Widać przypisaną wartość \TT{0xA} typu DWORD  (DD oznacza DWORD, czyli 32 bity).

Jeśli otworzysz skompilowany plik .exe w programie \IDA, zobaczysz zmienną \emph{x} na początku segmentu \TT{\_DATA}, a zaraz za nią ciąg znaków.

Gdybyś otworzył w programie \IDA skompilowany plik .exe z poprzedniego przykładu (gdy \emph{x} była niezainicjalizowana), zobaczyłbyś podobny wynik:

\lstinputlisting[caption=\IDA,style=customasmx86]{patterns/04_scanf/2_global/IDA.lst}

\label{BSSClearedByCStd}
Zmienna \TT{\_x} razem z innymi zmiennymi, które nie muszą być zainicjalizowane, jest oznaczona za pomocą \TT{?}.
To powoduje, że po załadowaniu pliku wykonalnego do pamięci, miejsce na te zmienne jest zaalokowane i wypełnione zerami \InSqBrackets{\CNineNineStd 6.7.8p10}.
W samym pliku wykonywalnym niezainicjalizowane zmienne nie zajmują żadnego miejsca. Ma to swoje zalety, np. przy dużych tablicach.

\EN{\clearpage
\subsubsection{MSVC: x86 + \olly}

Let's try to hack our program in \olly, forcing it to think \scanf always works without error.
When an address of a local variable is passed into \scanf,
the variable initially contains some random garbage, in this case \TT{0x6E494714}:

\begin{figure}[H]
\centering
\myincludegraphics{patterns/04_scanf/3_checking_retval/olly_1.png}
\caption{\olly: passing variable address into \scanf}
\label{fig:scanf_ex3_olly_1}
\end{figure}

\clearpage
While \scanf executes, in the console we enter something that is definitely not a number, like \q{asdasd}.
\scanf finishes with 0 in \EAX, which indicates that an error has occurred.

We can also check the local variable in the stack and note that it has not changed.
Indeed, what would \scanf write there?
It simply did nothing except returning zero.

Let's try to \q{hack} our program.
Right-click on \EAX, 
Among the options there is \q{Set to 1}.
This is what we need.

We now have 1 in \EAX, so the following check is to be executed as intended, 
and \printf will print the value of the variable in the stack.

When we run the program (F9) we can see the following in the console window:

\lstinputlisting[caption=console window]{patterns/04_scanf/3_checking_retval/console.txt}

Indeed, 1850296084 is a decimal representation of the number in the stack (\TT{0x6E494714})!
}
\RU{\clearpage
\mysubparagraph{Первый пример с \olly: a=1,2 и b=3,4}
\myindex{\olly}

Загружаем пример в \olly:

\begin{figure}[H]
\centering
\myincludegraphics{patterns/12_FPU/3_comparison/x86/MSVC/olly1_1.png}
\caption{\olly: первая \FLD исполнилась}
\label{fig:FPU_comparison_case1_olly1}
\end{figure}

Текущие параметры функции: $a=1,2$ и $b=3,4$ 
(их видно в стеке: 2 пары 32-битных значений).
$b$ (3,4) уже загружено в \ST{0}.
Сейчас будет исполняться \FCOMP. 
\olly показывает второй аргумент для \FCOMP, который сейчас находится в стеке.

\clearpage
\FCOMP отработал:

\begin{figure}[H]
\centering
\myincludegraphics{patterns/12_FPU/3_comparison/x86/MSVC/olly1_2.png}
\caption{\olly: \FCOMP исполнилась}
\label{fig:FPU_comparison_case1_olly2}
\end{figure}

Мы видим состояния condition-флагов \ac{FPU}: 
все нули.
Вытолкнутое значение отображается как \ST{7}. Почему это так, объяснялось ранее%
: 
\myref{FPU_is_rather_circular_buffer}.

\clearpage
\FNSTSW сработал:
\begin{figure}[H]
\centering
\myincludegraphics{patterns/12_FPU/3_comparison/x86/MSVC/olly1_3.png}
\caption{\olly: \FNSTSW исполнилась}
\label{fig:FPU_comparison_case1_olly3}
\end{figure}

Видно, что регистр \GTT{AX} содержит нули. Действительно, ведь все condition-флаги тоже содержали нули.

(\olly дизассемблирует команду \FNSTSW как \INS{FSTSW}~---%
 это синоним).

\clearpage
\TEST сработал:

\begin{figure}[H]
\centering
\myincludegraphics{patterns/12_FPU/3_comparison/x86/MSVC/olly1_4.png}
\caption{\olly: \TEST исполнилась}
\label{fig:FPU_comparison_case1_olly4}
\end{figure}

Флаг \GTT{PF} равен единице.
Всё верно: количество выставленных бит в 0~--- это 0, а 0~--- это четное число.

\olly дизассемблирует \INS{JP} как \ac{JPE}~--- это синонимы.
И она сейчас сработает.

\clearpage
\ac{JPE} сработала, \FLD загрузила в \ST{0} значение $b$ (3,4)%
:

\begin{figure}[H]
\centering
\myincludegraphics{patterns/12_FPU/3_comparison/x86/MSVC/olly1_5.png}
\caption{\olly: вторая \FLD исполнилась}
\label{fig:FPU_comparison_case1_olly5}
\end{figure}

Функция заканчивает свою работу.

\clearpage
\mysubparagraph{Второй пример с \olly: a=5,6 и b=-4}

Загружаем пример в \olly:

\begin{figure}[H]
\centering
\myincludegraphics{patterns/12_FPU/3_comparison/x86/MSVC/olly2_1.png}
\caption{\olly: первая \FLD исполнилась}
\label{fig:FPU_comparison_case2_olly1}
\end{figure}

Текущие параметры функции: $a=5,6$ и $b=-4$.
$b$ (-4) уже загружено в \ST{0}.
Сейчас будет исполняться \FCOMP. 
\olly показывает второй аргумент \FCOMP, который сейчас находится в стеке.


\clearpage
\FCOMP отработал:

\begin{figure}[H]
\centering
\myincludegraphics{patterns/12_FPU/3_comparison/x86/MSVC/olly2_2.png}
\caption{\olly: \FCOMP исполнилась}
\label{fig:FPU_comparison_case2_olly2}
\end{figure}

Мы видим значения condition-флагов \ac{FPU}: все нули, кроме \Czero.


\clearpage
\FNSTSW сработал:

\begin{figure}[H]
\centering
\myincludegraphics{patterns/12_FPU/3_comparison/x86/MSVC/olly2_3.png}
\caption{\olly: \FNSTSW исполнилась}
\label{fig:FPU_comparison_case2_olly3}
\end{figure}

Видно, что регистр \GTT{AX} содержит \GTT{0x100}: флаг \Czero стал на место 8-го бита.


\clearpage
\TEST сработал:

\begin{figure}[H]
\centering
\myincludegraphics{patterns/12_FPU/3_comparison/x86/MSVC/olly2_4.png}
\caption{\olly: \TEST исполнилась}
\label{fig:FPU_comparison_case2_olly4}
\end{figure}

Флаг \GTT{PF} равен нулю.
Всё верно: 
количество единичных бит в \GTT{0x100}~--- 1, а 1~--- нечетное число.

\ac{JPE} сейчас не сработает.

\clearpage
\ac{JPE} не сработала,  \FLD 
загрузила в \ST{0} значение $a$ (5,6)%
:

\begin{figure}[H]
\centering
\myincludegraphics{patterns/12_FPU/3_comparison/x86/MSVC/olly2_5.png}
\caption{\olly: вторая \FLD исполнилась}
\label{fig:FPU_comparison_case2_olly5}
\end{figure}

Функция заканчивает свою работу.
}
\IT{\clearpage
\subsubsection{MSVC: x86 + \olly}
\myindex{\olly}

Il quadro qui è ancora più semplice:

\begin{figure}[H]
\centering
\myincludegraphics{patterns/04_scanf/2_global/ex2_olly_1.png}
\caption{\olly: dopo l'esecuzione di \scanf}
\label{fig:scanf_ex2_olly_1}
\end{figure}

La variabile è collocata nel data segment.
Dopo che l'istruzione \PUSH (che fa il push dell'indirizzo di $x$) viene eseguita, 
l'indirizzo appare nella finestra dello stack. Facciamo click destro su quella riga e selezioniamo \q{Follow in dump}.
La variabile apparirà nella finestra di memoria a sinistra.
Dopo aver inserito il valore 123 in console, 
\TT{0x7B} apparirà nella finestra della memoria (vedere regioni evidenziate nello screenshot).

Ma perchè il primo byte è \TT{7B}?
A rigor di logica, dovremmo trovare \TT{00 00 00 7B}.
La causa per cui troviamo invece \TT{7B} è detta \gls{endianness}, e x86 usa la convenzione \emph{little-endian}.
Ciò significa che il byte piu basso è scritto per primo, e quello più alto per ultimo.
Maggiori informazioni sono disponibili nella sezione: \myref{sec:endianness}.
Tornando all'esempio, il valore a 32-bit è caricato da questo indirizzo di memoria in \EAX e passato a \printf.

L'indirizzo in memoria di $x$ è \TT{0x00C53394}.

\clearpage
\label{olly_memory_map_example}

In \olly possiamo osservare la mappa di memoria di un processo  (process memory map, Alt-M)
e notare che questo indirizzo è dentro il segmento PE \TT{.data} del nostro programma:

\begin{figure}[H]
\centering
\myincludegraphics{patterns/04_scanf/2_global/ex2_olly_2.png}
\caption{\olly: process memory map}
\label{fig:scanf_ex2_olly_2}
\end{figure}

}
\DE{\clearpage
\subsubsection{MSVC: x86 + \olly}
Laden wir unser Programm in \olly und zwingen es dazu zu glauben, dass \scanf stets ohne Fehler arbeitet.
Wenn die Adresse einer lokalen Variablen an \scanf übergeben wird, enthält die Variable zu Beginn einen zufälligen Wert,
in diesem Fall \TT{0x6E494714}:

\begin{figure}[H]
\centering
\myincludegraphics{patterns/04_scanf/3_checking_retval/olly_1.png}
\caption{\olly: Adresse der Variablen an \scanf übergeben}
\label{fig:scanf_ex3_olly_1}
\end{figure}

\clearpage
Während \scanf ausgeführt wird, geben wir in der Konsole etwas ein, das definitiv keine Zahl ist, z.B. \q{asdasd}.
\scanf beendet sich mit 0 in \EAX, was anzeigt, dass ein Fehler aufgetreten ist.

Wir können auch die lokale Variable auf dem Stack überprüfen und stellen fest, dass sie sich nicht verändert hat.
Was könnte \scanf hier auch hineinschreiben? Die Funktion hat nichts getan außer 0 zurückzugeben.

Versuchen wir unser Programm zu modifizieren, d.i. zu \q{hacken}.
Rechtsklick auf \EAX, in den Optionen finden wir \q{Set to 1}. Das ist was wir brauchen.

Wir haben jetzt 1 in \EAX, sodass die folgende Überprüfung wie gewünscht ausgeführt wird und \printf den Wert der
Variablen auf dem Stack ausgibt.

Wenn wir das Programm laufen lassen (F9), sehen wir das Folgende im Konsolenfenster:

\lstinputlisting[caption=console window]{patterns/04_scanf/3_checking_retval/console.txt}

Und tatsächlich ist 1850296084 die dezimale Darstellung der Zahl auf dem Stack (\TT{0x6E494714})!
}
\FR{\clearpage
\subsubsection{MSVC: x86 + \olly}

Essayons de hacker notre programme dans \olly, pour le forcer à penser que \scanf
fonctionne toujours sans erreur.
Lorsque l'adresse d'une variable locale est passée à \scanf, la variable contient
initiallement toujours des restes de données aléatoires, dans ce cas \TT{0x6E494714}:

\begin{figure}[H]
\centering
\myincludegraphics{patterns/04_scanf/3_checking_retval/olly_1.png}
\caption{\olly: passer l'adresse de la variable à \scanf}
\label{fig:scanf_ex3_olly_1}
\end{figure}

\clearpage
Lorsque \scanf s'exécute dans la console, entrons quelque chose qui n'est pas du
tout un nombre, comme \q{asdasd}.
\scanf termine avec 0 dans \EAX, ce qui indique qu'une erreur s'est produite.

Nous pouvons vérifier la variable locale dans le pile et noter qu'elle n'a pas changé.
En effet, qu'aurait écrit \scanf ici?
Elle n'a simplement rien fait à part renvoyer zéro.

Essayons de \q{hacker} notre programme.
Clique-droit sur \EAX,
parmi les options il y a \q{Set to 1} (mettre à 1).
C'est ce dont nous avons besoin.

Nous avons maintenant 1 dans \EAX, donc la vérification suivante va s'exécuter comme
souhaiter et \printf va afficher la valeur de la variable dans la pile.

Lorsque nous lançons le programme (F9) nous pouvons voir ceci dans la fenêtre
de la console:

\lstinputlisting[caption=fenêtre console]{patterns/04_scanf/3_checking_retval/console.txt}

En effet, 1850296084 est la représentation en décimal du nombre dans la pile (\TT{0x6E494714})!
}
\JA{\clearpage
\subsubsection{MSVC: x86 + \olly}

\olly でプログラムをハックしようとして、 \scanf が常にエラーなく動作するようにしましょう。 
ローカル変数のアドレスが \scanf に渡されると、
変数には最初にいくつかのランダムなガベージが含まれます。この場合、\TT{0x6E494714}です。

\begin{figure}[H]
\centering
\myincludegraphics{patterns/04_scanf/3_checking_retval/olly_1.png}
\caption{\olly: passing variable address into \scanf}
\label{fig:scanf_ex3_olly_1}
\end{figure}

\clearpage
\scanf が実行されている間、コンソールでは、 \q{asdasd}のように、数字ではないものを入力します。 
\scanf は、エラーが発生したことを示す \EAX が0で終了します。

また、スタック内のローカル変数をチェックし、変更されていないことに注意してください。 
実際、 \scanf は何を書いていますか? 
ゼロを返す以外は何もしませんでした。

私たちのプログラムを\q{ハックする}ようにしましょう。 
\EAX を右クリックし、
オプションの中に\q{Set to 1}があります。 
これが必要なものです。

\EAX には1があるので、以下のチェックを意図どおりに実行し、
\printf は変数の値をスタックに出力します。 

プログラム(F9)を実行すると、コンソールウィンドウで次のように表示されます。

\lstinputlisting[caption=console window]{patterns/04_scanf/3_checking_retval/console.txt}

実際、1850296084はスタック(\TT{0x6E494714})の数値を10進表現したものです!
}
\PL{\clearpage
\subsubsection{MSVC: x86 + \olly}
\myindex{\olly}

Teraz jest nawet prościej:

\begin{figure}[H]
\centering
\myincludegraphics{patterns/04_scanf/2_global/ex2_olly_1.png}
\caption{\olly: po wykonaniu \scanf}
\label{fig:scanf_ex2_olly_1}
\end{figure}

Zmienna jest umieszczona w segmencie danych.
Po wykonaniu instrukcji \PUSH (odłożenie adresu $x$),
adres pojawia się w oknie stosu. Kliknij prawym przyciskiem na ten wiersz i wybierz \q{Follow in dump}.
Zmienna pojawi się w oknie pamięci, po lewej stroniej.
Po wprowadzeniu 123 w konsoli,
\TT{0x7B} pojawi się w oknie pamięci (patrz miejsca zaznaczane na zrzucie ekranu).

Ale dlaczego pierwszy bajt to \TT{7B}?
Przecież logicznie rzecz biorąc, powinno być \TT{00 00 00 7B}!
Przyczyną jest tzw. \glslink{endianness}{kolejność bajtów (ang. \emph{endianess})}, a x86 używa kolejności od najmniej znaczącego bajtu (tzw. \emph{cienkokońcowość}, ang. \emph{little-endian}).
Więcej o tym przeczytasz tutaj:: \myref{sec:endianness}.
Wracając do przykładu, wartość 32-bitowa jest ładowana z adresu w pamięci do \EAX a następnie przekazywana do funkcji \printf.

$x$ znajduje się pod adresem \TT{0x00C53394}.

\clearpage
W \olly można sprawdzić mapę pamięci procesu (Alt-M).
Widzimy, że ten adres znajduje się w segmencie PE \TT{.data} programu.:

\label{olly_memory_map_example}
\begin{figure}[H]
\centering
\myincludegraphics{patterns/04_scanf/2_global/ex2_olly_2.png}
\caption{\olly: mapa pamięci procesu}
\label{fig:scanf_ex2_olly_2}
\end{figure}

}


\subsubsection{GCC: x86}

\myindex{ELF}
Wynik kompilacji na Linuksie jest prawi taki sam, różnicą jest to, że niezainicjalizowana zmienna jest umieszczona w segmencie \TT{\_bss}.
W plikach \ac{ELF} ten segment ma następujące atrybuty:

\begin{lstlisting}
; Segment type: Uninitialized
; Segment permissions: Read/Write
\end{lstlisting}

Jeśli jednak nadasz zmiennej jakość wartość, np. 10,
zostanie umieszczona w segmencie \TT{\_data}, który z kolei ma atrybuty:

\begin{lstlisting}
; Segment type: Pure data
; Segment permissions: Read/Write
\end{lstlisting}

\subsubsection{MSVC: x64}

\lstinputlisting[caption=MSVC 2012 x64,style=customasmx86]{patterns/04_scanf/2_global/ex2_MSVC_x64_PL.asm}

Kod jest niemal taki sam jak na x86.
Zauważ, że adres zmiennej $x$ jest przekazany do funkcji \TT{scanf()} za pomocą instrukcji \LEA,
ale wartość zmiennej jest przekazywana do drugiego wywołania \printf za pomocą instrukcji \MOV.
\TT{DWORD PTR}---to część kodu w asemblerze (bez związku z kodem maszynowym),
pokazująca, że zmienna jest 32-bitowa i instrukcja \MOV musi być odpowiednio zakodowana (opcode stosowny do rozmiaru).

