\subsubsection{MIPS}

\myparagraph{Niezainicjalizowane zmienne globalne}

$x$ jest teraz zmienną globalną.
Skompilujmy program do pliku wykonywalnego, zamiast obiektowego, i otwórzmy w programie \IDA.
IDA wyświetla zmienną $x$ w sekcji \TT{.sbss} pliku ELF (pamiętasz \q{Global Pointer}? \myref{MIPS_GP}),
gdyż zmienna nie jest zainicjalizowana na starcie.

\lstinputlisting[caption=\Optimizing GCC 4.4.5 (IDA),style=customasmMIPS]{patterns/04_scanf/2_global/MIPS/O3_IDA_PL.lst}

IDA nie wyświetliła wszystkich informacji, więc spójrzmy na listing wygenerowany za pomocą objdump oraz komentarze:

\lstinputlisting[caption=\Optimizing GCC 4.4.5 (objdump),numbers=left,style=customasmMIPS]{patterns/04_scanf/2_global/MIPS/O3_objdump_PL.txt}

Teraz widać, że adres zmiennej $x$ jest wczytywany z 64KiB bufora danych za pomocą GP i ujemnego przesunięcia (linia 18).
Co więcej, adresy trzech kolejnych zewnętrznych funkcji (\puts, \scanf, \printf) również wczytywane są z tego globalnego bufora, za pomocą GP (linia 9, 16 i 26).
GP wskazuje na środek bufora, a takie przesunięcie świadczy o tym, że adresy trzech funkcji oraz zmiennej $x$ są przechowywane gdzieś na jego początku.
Ma to sens, ponieważ nasz przykład jest bardzo prosty.

\myindex{MIPS!\Pseudoinstructions!MOVE}
\myindex{MIPS!\Pseudoinstructions!NOP}

Warto zauważyć, że funkcja kończy się dwiema instrukcjami \ac{NOP} (\TT{MOVE \$AT,\$AT} --- puste instrukcje), by wyrównać początek kolejnej funkcji do granicy 16 bajtów.

\myparagraph{Zainicjalizowane zmienne globalne}

Zmieńmy nasz przykład, nadając zmiennej $x$ wartość:

\lstinputlisting[style=customc]{patterns/04_scanf/2_global/default_value_PL.c}

Teraz IDA pokazuje, że zmienna $x$ jest wczytywana z sekcji .data:

\lstinputlisting[caption=\Optimizing GCC 4.4.5 (IDA),style=customasmMIPS]{patterns/04_scanf/2_global/MIPS/O3_IDA_init_PL.lst}

Dlaczego nie z .sdata? Być może wpływają na to pewne opcje GCC?

W każdym razie, $x$ jest w .data, a jest to pamięć współdzielona. Możemy zobaczyć jak taka pamięć jest wykorzystywana.

\myindex{MIPS!\Instructions!LUI}
\myindex{MIPS!\Instructions!ADDIU}

Adres zmiennej jest konstruowany za pomocą pary instrukcji.

W naszym przykładzie są to \INS{LUI} (\q{Load Upper Immediate}) i \INS{ADDIU} (\q{Add Immediate Unsigned Word}).

Poniżej listing wygenerowany za pomocą objdump, do bardziej szczegółowej analizy:

\lstinputlisting[caption=\Optimizing GCC 4.4.5 (objdump),style=customasmMIPS]{patterns/04_scanf/2_global/MIPS/O3_objdump_init_PL.txt}

\myindex{MIPS!\Instructions!LUI}
\myindex{MIPS!\Instructions!ADDIU}
\myindex{MIPS!\Instructions!LW}

Widać, że po zbudowaniu adresu za pomocą \INS{LUI} i \INS{ADDIU} starsze bity wciąż znajdują się w
rejestrze \$S0. Można teraz zakodować przesunięcie w instrukcji \INS{LW} (\q{Load Word}),
by za pomocą tylko tej jednej instukcji załadować zmienną z pamięci i przekazać do funkcji \printf.

Wiemy już, że rejestry przechowujące tymczasowe dane mają prefiks T-. W przykładzie widzimy również takie, które rozpoczynają się od S- ~---
są to rejestry, których zawartość musi zostać zachowana, jeśli funkcja planuje ich użyć (np. muszą być gdzieś zapisana, a później przywrócone).
Dzięki temu, gdy wywołujemy daną funkcję, to po jej zakończeniu i powrocie sterowania dane w tych rejestrach będą takie same, jak przed jej wywołaniem.
% FIXME:
% This needs to be clarified a bit, e.g. "the registers need to be preserved if a function is called and it wants to use them

Dlatego wartość \$S0 została ustawiona w adresie 0x4006cc, a następnie ponownie użyta pod adresem 0x4006e8, po wywołaniu funkcji \scanf.
Funkcja \scanf nie zmieniła tej wartości.

% TODO non-optimized example?
