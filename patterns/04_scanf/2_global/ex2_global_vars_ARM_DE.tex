\subsubsection{ARM: \OptimizingKeilVI (\ThumbMode)}

\lstinputlisting[caption=IDA,style=customasmARM]{patterns/04_scanf/2_global/ARM.lst}
Nun ist \TT{x} eine globale Variable und aus diesem Grund in einem anderen Segment, nämlich dem Datensegment
(\emph{.data}) angelegt. Man könnte sich fragen, warum die Textstrings sich im Codesegment (\emph{.text}) befindet und
\TT{x} nicht. Das liegt daran, dass \TT{x} eine Variable ist und sich damit per definitionem ihr Wert ändern kann; mehr
noch, er kann sich möglicherweise oft ändern. Textstrings dagegen sind Konstanten, können also nicht geändert werden,
und befinden sich deshalb im \emph{.text} Segment.
\myindex{\RAM}
\myindex{\ROM}
Das Codesegment kann manchmal in einem \ac{ROM} Chip liegen (erinnern wir uns, dass wir es hier mit eingebetteter
Mikroelektronik zu tun haben und Speicherknappheit ein häufiges Problem ist) und änderbare Variablen~---im \ac{RAM}.

Es ist nicht besonders ökonomisch konstante Variablen im RAM zu speichern, wenn ein ROM verfügbar ist.

Mehr noch, Konstanten im RAM müssen initialisiert werden, denn nach dem Einschalten enthält der RAM naturgemäß zufällige
Informationen.

\myindex{Linker}
Im Folgenden sehen wir einen Pointer auf die Variable \TT{x} (\TT{off\_2C}) im Codesegment und bemerken, dass alle
Operationen auf der Variable über den Pointer laufen.

Dies liegt daran, dass die Variable \TT{x} auch an einem weit entfernten Codefragment liegen könnte, weshalb ihre
Adresse irgendwo in der Nähe des Codes gespeichert werden muss.

\myindex{ARM!\Instructions!LDR}
Der Befehl \INS{LDR} im Thumb Mode kann nur Variablen in einer Reichweite von 1020 Bytes von seiner Speicherstelle
adressieren und im ARM Mode~---beträgt die Spannweite der Variablen $\pm{}4095$ Bytes.

Also muss die Adresse der Variable \TT{x} sich in der Nähe befinden, denn es gibt keine Garantie, dass der Linker in der
Lage ist, die Variable in der Nähe des Codes zu platzieren, sie könnte sich sogar auf einem externen Memory Chip
befinden.

\myindex{\CLanguageElements!const}
\myindex{\ROM}
Eine weitere Sache: wenn eine Variable as \emph{konstant} deklariert wird, legt der Keil Compiler sie im \TT{.constdata}
Segment ab.

Möglicherweise könnte der Linker dieses Segment anschließend gemeinsam mit dem Codesegment im ROM ablegen.

\subsubsection{ARM64}

\lstinputlisting[caption=\NonOptimizing GCC 4.9.1 ARM64,numbers=left,style=customasmARM]{patterns/04_scanf/2_global/ARM64_GCC491_O0_DE.s}

\myindex{ARM!\Instructions!ADRP/ADD pair}
In diesem Fall ist die Variable $x$ als global deklariert und ihre Adresse wird mithilfe des Befehlspaares
\INS{ADRP}/\INS{ADD} berechnet (Zeilen 21 und 25).

