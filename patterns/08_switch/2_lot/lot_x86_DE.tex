\subsubsection{x86}

\myparagraph{\NonOptimizing MSVC}

Wir erhalten (MSVC 2010):

\lstinputlisting[caption=MSVC 2010,style=customasmx86]{patterns/08_switch/2_lot/lot_msvc_DE.asm}

\myindex{jumptable}
Was wir hier sehen ist eine Ansammlung von Aufrufen von \printf mit diversen Argumenten.
Alle haben nicht Adressen im Speicher des Prozesses, sondern auch interne symbolische Labels, die ihnen vom Compiler
zugewiesen werden.
Alle diese Labels werden auch in der internen Tabelle \TT{\$LN11@f} aufgeführt. 

Zu Beginn der Funktion wird der Control Flow an das Label \TT{\$LN1@f} abgegeben, wenn $a$ größer ist als 4. An diesem
Label wird \printf mit dem Argument \TT{'something unknown'} aufgerufen.

Wenn aber der Wert von $a$ kleiner gleich 4 ist, dann wird dieser mit 4 multipliziert und mit der Tabellenadresse
\TT{\$LN11@f} addiert. Auf diese Weise wird die Adresse innerhalb der Tabelle konstruiert und zeigt genau auf das
gewünschte Element. Nehmen wir zum Beispiel an, dass $a$ gleich 2 ist. $2\cdot 4=8$ (alle Tabellenelemente sind
Adressen in einem 32-Bit-Prozess und haben daher eine Breite von 4 Bytes).
Die Adresse an der Stelle \TT{\$LN11@f} + 8 ist das Tabellenelement, an dem das Label \TT{\$LN4@f} gespeichert ist.
\JMP holt die Adresse \TT{\$LN4@f} aus der Tabelle und springt dorthin.

Diese Tabelle wird manchmal \emph{Jumptable} oder \emph{Verzweigungstabelle} genannt\footnote{Die ganze Methode wurde
in früheren Versionen von Fortran \emph{berechnetes GOTO} genannt:
\href{http://en.wikipedia.org/wiki/Branch_table}{wikipedia}.
Heutzutage zwar nicht mehr relevant, aber welch ein Ausdruck!}.

Dann wird das zugehörige \printf mit dem Argument \TT{'two'} aufgerufen.\\
Der Befehl TT{jmp DWORD PTR \$LN11@f[ecx*4]} bedeutet dabei \emph{springe zum an dieser Stelle gespeicherten
DWORD}\TT{\$LN11@f + ecx * 4}.

\TT{npad} (\myref{sec:npad}) ist ein Assemblermakro, dass das nächste Label so angeordnet, dass es an einer 4 Byte
(oder 16 Bit) Adressgrenze gespeichert wird. Das ist für den Prozessor sehr praktisch, da er die 32-Bit-Werte aus dem
Speicher durch den Speicherbus, den Cache, etc. in effektiverer Weise laden kann.

\clearpage
\subsubsection{MSVC: x86 + \olly}
Laden wir unser Programm in \olly und zwingen es dazu zu glauben, dass \scanf stets ohne Fehler arbeitet.
Wenn die Adresse einer lokalen Variablen an \scanf übergeben wird, enthält die Variable zu Beginn einen zufälligen Wert,
in diesem Fall \TT{0x6E494714}:

\begin{figure}[H]
\centering
\myincludegraphics{patterns/04_scanf/3_checking_retval/olly_1.png}
\caption{\olly: Adresse der Variablen an \scanf übergeben}
\label{fig:scanf_ex3_olly_1}
\end{figure}

\clearpage
Während \scanf ausgeführt wird, geben wir in der Konsole etwas ein, das definitiv keine Zahl ist, z.B. \q{asdasd}.
\scanf beendet sich mit 0 in \EAX, was anzeigt, dass ein Fehler aufgetreten ist.

Wir können auch die lokale Variable auf dem Stack überprüfen und stellen fest, dass sie sich nicht verändert hat.
Was könnte \scanf hier auch hineinschreiben? Die Funktion hat nichts getan außer 0 zurückzugeben.

Versuchen wir unser Programm zu modifizieren, d.i. zu \q{hacken}.
Rechtsklick auf \EAX, in den Optionen finden wir \q{Set to 1}. Das ist was wir brauchen.

Wir haben jetzt 1 in \EAX, sodass die folgende Überprüfung wie gewünscht ausgeführt wird und \printf den Wert der
Variablen auf dem Stack ausgibt.

Wenn wir das Programm laufen lassen (F9), sehen wir das Folgende im Konsolenfenster:

\lstinputlisting[caption=console window]{patterns/04_scanf/3_checking_retval/console.txt}

Und tatsächlich ist 1850296084 die dezimale Darstellung der Zahl auf dem Stack (\TT{0x6E494714})!


\myparagraph{\NonOptimizing GCC}
\label{switch_lot_GCC}

Schauen wir was GCC 4.4.1 erzeugt:

\lstinputlisting[caption=GCC 4.4.1,style=customasmx86]{patterns/08_switch/2_lot/lot_gcc.asm}

\myindex{x86!\Registers!JMP}
Es ist bis auf eine Nuance das gleiche: das Argument \TT{arg\_0} wird mit 4 multipliziert durch eine Verschiebung von 2
Bits nach links (dies entspricht einer Multiplikation mit 4)~(\myref{SHR}).
Dann wird die Adresse des Labels vom \TT{off\_804855C} genommen, die in \EAX gespeichert wird, und dann wird mit
\TT{JMP EAX} der eigentliche Sprung durchgeführt.


