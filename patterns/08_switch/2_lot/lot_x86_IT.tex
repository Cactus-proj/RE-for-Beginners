\subsubsection{x86}

\myparagraph{\NonOptimizing MSVC}

Con MSVC 2010 otteniamo:

\lstinputlisting[caption=MSVC 2010,style=customasmx86]{patterns/08_switch/2_lot/lot_msvc_IT.asm}

\myindex{jumptable}

Vediamo una serie di chiamate a \printf con vari argomenti. Hanno tutte non solo indirizzi nella memoria del processo, ma anche etichette
simboliche assegnate dal compilatore. Queste label sono anche menzionate nalle tabella interna \TT{\$LN11@f}.

All'inizio della funzione, se $a$ è maggiore di 4, il controllo del flusso è passato alla label 
\TT{\$LN1@f}, dove viene chiamata \printf con argomento \TT{'something unknown'}.

Se invece il valore di $a$ è minore o uguale a 4, viene moltiplicato per 4 e sommato all'indirizzo della tabella \TT{\$LN11@f}. 
In questo modo vengono costruiti gli indirizzi della tabella, facendo puntare esattamente all'elemento giusto per ogni caso,

Poniamo ad esempio che $a$ sia uguale a 2. $2*4 = 8$ (tutti gli elementi della tabella sono indirizzi in un processo a 32-bit, perciò tutti gli elementi sono larghi 4 byte).
L'indirizzo della tabella \TT{\$LN11@f} + 8 corrisponde all'elemento della tabella in cui è memorizzata la label \TT{\$LN4@f}.
L'istruzione \JMP recupera quindi l'indirizzo di \TT{\$LN4@f} dalla tabella e salta.

Questa tabella è talvolta detta \emph{jumptable} o \emph{branch table}\footnote{L'intero metodo una volta era noto come 
\emph{computed GOTO} nelle prime versioni di Fortran:
\href{http://go.yurichev.com/17122}{wikipedia}.
Non è molto rilevante oggigiorno, ma che termine!}.

Successivamente la corrispondente \printf viene chiamata con argomento \TT{'two'}.\\
Letteralmente, l'istruzione \TT{jmp DWORD PTR \$LN11@f[ecx*4]} corrisponde a 
\emph{salta alla DWORD che è memorizzata all'indirizzo} \TT{\$LN11@f + ecx * 4}.

\TT{npad} (\myref{sec:npad}) è una macro del linguaggio assembly che allinea la prossima label in modo tale che sia memorizzata 
ad un indirizzo allineato a 4 byte (or a 16 byte).

Ciò è molto utile in termini di performance poiché così il processore è in grado di recuperare valori a 32-bit dalla memoria attraverso il memory bus, 
la cache, etc., in maniera più efficiente se è allineata.

\clearpage
\mysubparagraph{\olly}
\myindex{\olly}

Esaminiamo questo esempio con \olly.
Il valore di input della funzione (2) viene caricato \EAX: 

\begin{figure}[H]
\centering
\myincludegraphics{patterns/08_switch/2_lot/olly1.png}
\caption{\olly: il valore di input è caricato in \EAX}
\label{fig:switch_lot_olly1}
\end{figure}

\clearpage
Il valore viene controllato, è maggiore di 4?
Se no, il \q{default} jump non viene innescato:
\begin{figure}[H]
\centering
\myincludegraphics{patterns/08_switch/2_lot/olly2.png}
\caption{\olly: 2 non è maggiore di 4: il salto non viene fatto}
\label{fig:switch_lot_olly2}
\end{figure}

\clearpage
Qui vediamo una jumptable:

\begin{figure}[H]
\centering
\myincludegraphics{patterns/08_switch/2_lot/olly3.png}
\caption{\olly: calcolo dell'indirizzo di destinazione mediante jumptable}
\label{fig:switch_lot_olly3}
\end{figure}

Qui abbiamo cliccato \q{Follow in Dump} $\rightarrow$ \q{Address constant}, così da vedere la \emph{jumptable} nella finestra dati.
Sono 5 valori a 32-bit \footnote{Sono sottolineati da \olly poiché
sono anche FIXUPs: \myref{subsec:relocs}, torneremo su questo argomento più avanti}.
\ECX adesso è 2, quindi il terzo elemento (avente indice 2\footnote{Per l'indicizzazione, vedi anche: \myref{arrays_at_one}}) della tabella.
E' anche possibile cliccare su \q{Follow in Dump} $\rightarrow$ 
\q{Memory address} e \olly mostrerà l'elemento a cui punta l'istruzione \JMP. 
In questo caso è \TT{0x010B103A}.

\clearpage
Dopo il salto ci troviamo a \TT{0x010B103A}: il codice che stampa \q{two} sarà ora eseguito:

\begin{figure}[H]
\centering
\myincludegraphics{patterns/08_switch/2_lot/olly4.png}
\caption{\olly: ora ci troviamo alla label \emph{case:}}
\label{fig:switch_lot_olly4}
\end{figure}


\myparagraph{\NonOptimizing GCC}
\label{switch_lot_GCC}

Vediamo il codice generato da GCC 4.4.1:

\lstinputlisting[caption=GCC 4.4.1,style=customasmx86]{patterns/08_switch/2_lot/lot_gcc.asm}

\myindex{x86!\Registers!JMP}

E' pressoché identico, con una leggera variazione: l'argomento \TT{arg\_0} è moltiplicato per 4
effettuando uno shift a sinistra di 2 bit (quasi identico alla moltiplicazione per 4)~(\myref{SHR}).
Successivamente l'indirizzo della label è preso dall'array \TT{off\_804855C}, memorizzato in 
\EAX, e infine \TT{JMP EAX} effettua il salto.

