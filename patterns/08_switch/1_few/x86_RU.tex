\subsubsection{x86}

\myparagraph{\NonOptimizing MSVC}

Это дает в итоге (MSVC 2010):

\lstinputlisting[caption=MSVC 2010,style=customasmx86]{patterns/08_switch/1_few/few_msvc.asm}

Наша функция с оператором switch(), с небольшим количеством вариантов, 
это практически аналог подобной конструкции:

\lstinputlisting[label=switch_few_ifelse,style=customc]{patterns/08_switch/1_few/few_analogue.c}

\myindex{\CLanguageElements!switch}
\myindex{\CLanguageElements!if}
Когда вариантов немного и мы видим подобный код, невозможно сказать с уверенностью, был ли
в оригинальном исходном коде switch(), либо просто набор операторов if().

\myindex{\SyntacticSugar}
То есть, switch() это синтаксический сахар для большого количества вложенных проверок 
при помощи if().

В самом выходном коде ничего особо нового, 
за исключением того, что компилятор зачем-то 
перекладывает входящую переменную ($a$) во временную в локальном стеке \TT{v64}\footnote{Локальные переменные в стеке с префиксом \TT{tv}~--- 
так MSVC называет внутренние переменные для своих нужд}.

Если скомпилировать это при помощи GCC 4.4.1, то будет почти то же самое, даже с максимальной оптимизацией (ключ \Othree).

\myparagraph{\Optimizing MSVC}

% TODO separate various kinds of \TT
% idea: enclose command lines in a specific environment, like \cmdline{} 
% assembly instructions in \asm{} (now both \TT and \q{} are used),
% variables in,  like \var{}
% messages (string constants) in something else, like \strconst
% to separate them all. Now they all use \TT, which is not best
% \INS{} for all instructions including operands? --DY

Попробуем включить оптимизацию кодегенератора MSVC (\Ox): \TT{cl 1.c /Fa1.asm /Ox}

\label{JMP_instead_of_RET}
\lstinputlisting[caption=MSVC,style=customasmx86]{patterns/08_switch/1_few/few_msvc_Ox.asm}

Вот здесь уже всё немного по-другому, причем не без грязных трюков.

\myindex{x86!\Instructions!JZ}
\myindex{x86!\Instructions!JE}
\myindex{x86!\Instructions!SUB}
Первое: \TT{а} помещается в \EAX и от него отнимается 0. Звучит абсурдно, но нужно это для того, чтобы проверить, 
0 ли в \EAX был до этого? Если да, то выставится флаг \ZF (что означает, что результат вычитания 0 от числа 
стал 0) и первый условный переход \JE (\emph{Jump if Equal} или его синоним \JZ~--- \emph{Jump if Zero}) 
сработает на метку \TT{\$LN4@f}, где выводится сообщение \TT{'zero'}.
Если первый переход не сработал, от значения отнимается по единице, 
и если на какой-то стадии в результате образуется 0, то сработает соответствующий переход.

И в конце концов, если ни один из условных переходов не сработал, управление передается \printf
со строковым аргументом \TT{'something unknown'}.

\label{jump_to_last_printf}
\myindex{\Stack}
Второе: мы видим две, мягко говоря, необычные вещи: указатель на сообщение помещается в переменную $a$, 
и затем \printf вызывается не через \CALL, а через \JMP. Объяснение этому простое. 
Вызывающая функция заталкивает в стек некоторое значение и через \CALL вызывает нашу функцию. 
\CALL в свою очередь заталкивает в стек адрес возврата (\ac{RA}) и делает безусловный переход на адрес нашей функции. 
Наша функция в самом начале (да и в любом её месте, потому что в теле функции нет ни одной инструкции, 
которая меняет что-то в стеке или в \ESP) имеет следующую разметку стека:

\begin{itemize}
\item\ESP --- хранится \ac{RA}
\item\TT{ESP+4} --- хранится значение $a$ 
\end{itemize}

С другой стороны, чтобы вызвать \printf, нам нужна почти такая же разметка стека, 
только в первом аргументе нужен указатель на строку. Что, собственно, этот код и делает.

Он заменяет свой первый аргумент на адрес строки, и затем передает управление \printf, как если бы вызвали не 
нашу функцию \ttf, а сразу \printf. 
\printf выводит некую строку на \gls{stdout}, затем исполняет инструкцию \RET, 
которая из стека достает \ac{RA} и управление передается в ту функцию, 
которая вызывала \ttf, минуя при этом конец функции \ttf.

\myindex{\CStandardLibrary!longjmp()}
\newcommand{\URLSJ}{\href{http://en.wikipedia.org/wiki/Setjmp.h}{wikipedia}}
% TODO \myref{}
Всё это возможно, потому что \printf вызывается в \ttf в самом конце. 
Всё это чем-то даже похоже на \TT{longjmp()}\footnote{\URLSJ}.
И всё это, разумеется, сделано для экономии времени исполнения.

Похожая ситуация с компилятором для ARM описана в секции \q{\PrintfSeveralArgumentsSectionName}~(\myref{ARM_B_to_printf}).

\clearpage
\mysubparagraph{Первый пример с \olly: a=1,2 и b=3,4}
\myindex{\olly}

Загружаем пример в \olly:

\begin{figure}[H]
\centering
\myincludegraphics{patterns/12_FPU/3_comparison/x86/MSVC/olly1_1.png}
\caption{\olly: первая \FLD исполнилась}
\label{fig:FPU_comparison_case1_olly1}
\end{figure}

Текущие параметры функции: $a=1,2$ и $b=3,4$ 
(их видно в стеке: 2 пары 32-битных значений).
$b$ (3,4) уже загружено в \ST{0}.
Сейчас будет исполняться \FCOMP. 
\olly показывает второй аргумент для \FCOMP, который сейчас находится в стеке.

\clearpage
\FCOMP отработал:

\begin{figure}[H]
\centering
\myincludegraphics{patterns/12_FPU/3_comparison/x86/MSVC/olly1_2.png}
\caption{\olly: \FCOMP исполнилась}
\label{fig:FPU_comparison_case1_olly2}
\end{figure}

Мы видим состояния condition-флагов \ac{FPU}: 
все нули.
Вытолкнутое значение отображается как \ST{7}. Почему это так, объяснялось ранее%
: 
\myref{FPU_is_rather_circular_buffer}.

\clearpage
\FNSTSW сработал:
\begin{figure}[H]
\centering
\myincludegraphics{patterns/12_FPU/3_comparison/x86/MSVC/olly1_3.png}
\caption{\olly: \FNSTSW исполнилась}
\label{fig:FPU_comparison_case1_olly3}
\end{figure}

Видно, что регистр \GTT{AX} содержит нули. Действительно, ведь все condition-флаги тоже содержали нули.

(\olly дизассемблирует команду \FNSTSW как \INS{FSTSW}~---%
 это синоним).

\clearpage
\TEST сработал:

\begin{figure}[H]
\centering
\myincludegraphics{patterns/12_FPU/3_comparison/x86/MSVC/olly1_4.png}
\caption{\olly: \TEST исполнилась}
\label{fig:FPU_comparison_case1_olly4}
\end{figure}

Флаг \GTT{PF} равен единице.
Всё верно: количество выставленных бит в 0~--- это 0, а 0~--- это четное число.

\olly дизассемблирует \INS{JP} как \ac{JPE}~--- это синонимы.
И она сейчас сработает.

\clearpage
\ac{JPE} сработала, \FLD загрузила в \ST{0} значение $b$ (3,4)%
:

\begin{figure}[H]
\centering
\myincludegraphics{patterns/12_FPU/3_comparison/x86/MSVC/olly1_5.png}
\caption{\olly: вторая \FLD исполнилась}
\label{fig:FPU_comparison_case1_olly5}
\end{figure}

Функция заканчивает свою работу.

\clearpage
\mysubparagraph{Второй пример с \olly: a=5,6 и b=-4}

Загружаем пример в \olly:

\begin{figure}[H]
\centering
\myincludegraphics{patterns/12_FPU/3_comparison/x86/MSVC/olly2_1.png}
\caption{\olly: первая \FLD исполнилась}
\label{fig:FPU_comparison_case2_olly1}
\end{figure}

Текущие параметры функции: $a=5,6$ и $b=-4$.
$b$ (-4) уже загружено в \ST{0}.
Сейчас будет исполняться \FCOMP. 
\olly показывает второй аргумент \FCOMP, который сейчас находится в стеке.


\clearpage
\FCOMP отработал:

\begin{figure}[H]
\centering
\myincludegraphics{patterns/12_FPU/3_comparison/x86/MSVC/olly2_2.png}
\caption{\olly: \FCOMP исполнилась}
\label{fig:FPU_comparison_case2_olly2}
\end{figure}

Мы видим значения condition-флагов \ac{FPU}: все нули, кроме \Czero.


\clearpage
\FNSTSW сработал:

\begin{figure}[H]
\centering
\myincludegraphics{patterns/12_FPU/3_comparison/x86/MSVC/olly2_3.png}
\caption{\olly: \FNSTSW исполнилась}
\label{fig:FPU_comparison_case2_olly3}
\end{figure}

Видно, что регистр \GTT{AX} содержит \GTT{0x100}: флаг \Czero стал на место 8-го бита.


\clearpage
\TEST сработал:

\begin{figure}[H]
\centering
\myincludegraphics{patterns/12_FPU/3_comparison/x86/MSVC/olly2_4.png}
\caption{\olly: \TEST исполнилась}
\label{fig:FPU_comparison_case2_olly4}
\end{figure}

Флаг \GTT{PF} равен нулю.
Всё верно: 
количество единичных бит в \GTT{0x100}~--- 1, а 1~--- нечетное число.

\ac{JPE} сейчас не сработает.

\clearpage
\ac{JPE} не сработала,  \FLD 
загрузила в \ST{0} значение $a$ (5,6)%
:

\begin{figure}[H]
\centering
\myincludegraphics{patterns/12_FPU/3_comparison/x86/MSVC/olly2_5.png}
\caption{\olly: вторая \FLD исполнилась}
\label{fig:FPU_comparison_case2_olly5}
\end{figure}

Функция заканчивает свою работу.


