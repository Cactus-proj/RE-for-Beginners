\subsubsection{x86}

\myparagraph{MSVC \NonOptimizing}

Résultat (MSVC 2010):

\lstinputlisting[caption=MSVC 2010,style=customasmx86]{patterns/08_switch/1_few/few_msvc.asm}

Notre fonction avec quelques cas dans switch() est en fait analogue à cette construction:

\lstinputlisting[label=switch_few_ifelse,style=customc]{patterns/08_switch/1_few/few_analogue.c}

\myindex{\CLanguageElements!switch}
\myindex{\CLanguageElements!if}

Si nous utilisons switch() avec quelques cas, il est impossible de savoir si il y
avait un vrai switch() dans le code source, ou un ensemble de directives if().
\myindex{\SyntacticSugar}

Ceci indique que switch() est comme un sucre syntaxique pour un grand nombre de if()
imbriqués.

Il n'y a rien de particulièrement nouveau pour nous dans le code généré, à l'exception
que le compilateur déplace la variable d'entrée $a$ dans une variable locale temporaire
\TT{tv64} \footnote{Les variables locales sur la pile sont préfixées avec \TT{tv}---c'est
ainsi que MSVC appelle les variables internes dont il a besoin.}.

Si nous compilons ceci avec GCC 4.4.1, nous obtenons presque le même résultat,
même avec le niveau d'optimisation le plus élevé (\Othree option).

\myparagraph{MSVC \Optimizing}

% TODO separate various kinds of \TT
% idea: enclose command lines in a specific environment, like \cmdline{} 
% assembly instructions in \asm{} (now both \TT and \q{} are used),
% variables in,  like \var{}
% messages (string constants) in something else, like \strconst
% to separate them all. Now they all use \TT, which is not best
% \INS{} for all instructions including operands? --DY
Maintenant compilons dans MSVC avec l'optimisation (\Ox): \TT{cl 1.c /Fa1.asm /Ox}

\label{JMP_instead_of_RET}
\lstinputlisting[caption=MSVC,style=customasmx86]{patterns/08_switch/1_few/few_msvc_Ox.asm}

Ici, nous voyons quelques hacks moches.

\myindex{x86!\Instructions!JZ}
\myindex{x86!\Instructions!JE}
\myindex{x86!\Instructions!SUB}

Premièrement: la valeurs de $a$ est mise dans \EAX et 0 en est soustrait. Ça semble
absurde, mais cela est fait pour vérifier si la valeur dans \EAX est 0. Si oui, le
flag \ZF est mis (e.g. soustraire de 0 est 0) et le premier saut conditionnel \JE
(\emph{Jump if Equal} saut si égal ou synonyme \JZ~---\emph{Jump if Zero} saut si zéro)
va être effectué et le déroulement du programme passera au label \TT{\$LN4@f}, où
le message \TT{'zero'} est affiché.
Si le premier saut n'est pas effectué, 1 est soustrait de la valeur d'entrée et si
à une étape le résultat est 0, le saut correspondant sera effectué.

Et si aucun saut n'est exécuté, l'exécution passera au \printf avec comme argument
la chaîne\\ \TT{'something unknown'}.

\label{jump_to_last_printf}
\myindex{\Stack}

Deuxièmement: nous voyons quelque chose d'inhabituel pour nous: un pointeur sur une
chaîne est mis dans la variable $a$ et ensuite \printf est appelé non pas par \CALL,
mais par \JMP. Il y a une explication simple à cela: l'\glslink{caller}{appelant}
pousse une valeur sur la pile et appelle notre fonction via \CALL.
\CALL lui même pousse l'adresse de retour (\ac{RA}) sur la pile et fait un saut
inconditionnel à l'adresse de notre fonction.
Notre fonction, à tout moment de son exécution (car elle ne contient pas d'instruction
qui modifie le pointeur de pile) à le schéma suivant pour sa pile:

\begin{itemize}
\item\ESP---pointe sur \ac{RA}
\item\TT{ESP+4}---pointe sur la variable $a$
\end{itemize}

D'un autre côté, lorsque nous appelons \printf ici nous avons exactement la même
disposition de pile, excepté pour le premier argument de \printf, qui doit pointer
sur la chaîne. Et c'est ce que fait notre code.

Il remplace le premier argument de la fonction par l'adresse de la chaîne et saute
à \printf, comme si nous n'avions pas appelé notre fonction \ttf, mais directement
\printf.
\printf affiche la chaîne sur la \glslink{stdout}{sortie standard} et ensuite exécute
l'instruction \RET qui POPs \ac{RA} de la pile et l'exécution est renvoyée non pas
à \ttf mais plutôt à l'\glslink{caller}{appelant} de \ttf, ignorant la fin de la
fonction \ttf.

\myindex{\CStandardLibrary!longjmp()}
\newcommand{\URLSJ}{\href{http://en.wikipedia.org/wiki/Setjmp.h}{Wikipédia}}

% TODO \myref{}
Tout ceci est possible car \printf est appelée, dans tous les cas, tout à la fin
de la fonction \ttf.
Dans un certain sens, c'est similaire à la fonction \TT{longjmp()}\footnote{\URLSJ}.
Et bien sûr, c'est fait dans un but de vitesse d'exécution.

Un cas similaire avec le compilateur ARM est décrit dans la section \q{\PrintfSeveralArgumentsSectionName},
ici~(\myref{ARM_B_to_printf}).

\clearpage
\subsubsection{MSVC: x86 + \olly}

Essayons de hacker notre programme dans \olly, pour le forcer à penser que \scanf
fonctionne toujours sans erreur.
Lorsque l'adresse d'une variable locale est passée à \scanf, la variable contient
initiallement toujours des restes de données aléatoires, dans ce cas \TT{0x6E494714}:

\begin{figure}[H]
\centering
\myincludegraphics{patterns/04_scanf/3_checking_retval/olly_1.png}
\caption{\olly: passer l'adresse de la variable à \scanf}
\label{fig:scanf_ex3_olly_1}
\end{figure}

\clearpage
Lorsque \scanf s'exécute dans la console, entrons quelque chose qui n'est pas du
tout un nombre, comme \q{asdasd}.
\scanf termine avec 0 dans \EAX, ce qui indique qu'une erreur s'est produite.

Nous pouvons vérifier la variable locale dans le pile et noter qu'elle n'a pas changé.
En effet, qu'aurait écrit \scanf ici?
Elle n'a simplement rien fait à part renvoyer zéro.

Essayons de \q{hacker} notre programme.
Clique-droit sur \EAX,
parmi les options il y a \q{Set to 1} (mettre à 1).
C'est ce dont nous avons besoin.

Nous avons maintenant 1 dans \EAX, donc la vérification suivante va s'exécuter comme
souhaiter et \printf va afficher la valeur de la variable dans la pile.

Lorsque nous lançons le programme (F9) nous pouvons voir ceci dans la fenêtre
de la console:

\lstinputlisting[caption=fenêtre console]{patterns/04_scanf/3_checking_retval/console.txt}

En effet, 1850296084 est la représentation en décimal du nombre dans la pile (\TT{0x6E494714})!


