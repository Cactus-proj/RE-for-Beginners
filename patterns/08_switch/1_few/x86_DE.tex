\subsubsection{x86}

\myparagraph{\NonOptimizing MSVC}

Ergebnis (MSVC 2010):

\lstinputlisting[caption=MSVC 2010,style=customasmx86]{patterns/08_switch/1_few/few_msvc.asm}
Unsere Funktionen mit ein paar Fällen in switch() ist analog zu dieser Konstruktion:

\lstinputlisting[label=switch_few_ifelse,style=customc]{patterns/08_switch/1_few/few_analogue.c}

\myindex{\CLanguageElements!switch}
\myindex{\CLanguageElements!if}
Wenn wir mit einem switch() mit einigen wenigen Fällen arbeiten, ist es unmöglich sicher zu sein, dass es sich
tatsächlich im Quellcode um ein switch() handelt und nicht um eine Reihe von if()-Anweisungen.

\myindex{\SyntacticSugar}
Das bedeutet, dass switch() ein syntaktischer Zucker für eine große Anzahl von verschachtelten if()-Anweisungen ist.

Hier ist nichts Neues für uns im erzeugten Code, mit der Ausnahme, dass der Compiler die Eingabevariable $a$ in einer
temporären Variable \TT{tv64} speichert\footnote{Lokale Variablen im Stack haben den Präfix \TT{tv}---so benennt MSVC
interne Variablen für seine Zwecke}.

Wenn wir diesen Code mit GCC 4.4.1 kompilieren, erhalten wir fast das gleiche Ergebnis, sogar unter Verwendung maximaler
Optimierung (Option \Othree).

\myparagraph{\Optimizing MSVC}

% TODO separate various kinds of \TT
% idea: enclose command lines in a specific environment, like \cmdline{} 
% assembly instructions in \asm{} (now both \TT and \q{} are used),
% variables in,  like \var{}
% messages (string constants) in something else, like \strconst
% to separate them all. Now they all use \TT, which is not best
% \INS{} for all instructions including operands? --DY
Aktivieren wir nun Optimierung in MSVC (\Ox): \TT{cl 1.c /Fa1.asm /Ox}

\label{JMP_instead_of_RET}
\lstinputlisting[caption=MSVC,style=customasmx86]{patterns/08_switch/1_few/few_msvc_Ox.asm}
Hier finden wir ein paar schmutzige Tricks.

\myindex{x86!\Instructions!JZ}
\myindex{x86!\Instructions!JE}
\myindex{x86!\Instructions!SUB}
Zunächst: der Wert von $a$ wird in \EAX abgelegt und 0 wird davon abgezogen. Klingt absurd, muss aber getan werden, um
zu prüfen, ob der Wert in \EAX null ist. Falls ja, wird das \ZF Flag gesetzt (denn 0 minus 0 ist 0) und der erste
bedingte Sprung \JE (\emph{Jump if Equal}) oder synonym \JZ~---\emph{Jump if Zero}) wird ausgeführt und der Control Flow
wird an das Label \TT{\$LN4@f} übergeben, an dem die Nachricht \TT{'zero'} ausgegeben wird.
Wenn der erste Sprung nicht ausgeführt wird, wird 1 vom Eingabewert abgezogen und wenn an irgendeinem Punkt der
Ausführung ein Ergebnis von 0 auftritt, wird der zugehörige Sprung ausgeführt. 

Falls aber keiner der Sprünge ausgeführt wird, wird der Control Flow mit dem String Argument \TT{'something unknown'}
an \printf übergeben.

\label{jump_to_last_printf}
\myindex{\Stack}
Zweitens: wir sehen etwas für uns Ungewohntes: ein Pointer auf einen String wird in der Variablen $a$ abgelegt und
anschließend wird \printf nicht über \CALL, sondern per \JMP aufgerufen. Es gibt hierfür eine einfache Erklärung:
Der \glslink{caller}{Aufrufer} legt einen Wert auf dem Stack ab und ruft unsere Funktion über \CALL auf.
\CALL selbst legt die Rücksprungadresse (\ac{RA}) auf dem Stack ab und macht einen unbedingten Sprung zur Adresse
unserer Funktion. Unsere Funktion hat an jedem Punkt der Ausführung (da sie keine Befehle enthält, die den Stackpointer
verändern) das folgende Stacklayout:

\begin{itemize}
\item\ESP---zeigt auf \ac{RA}
\item\TT{ESP+4}---zeigt auf die Variable $a$ 
\end{itemize}
Wenn wir andererseits \printf aufrufen benötigen wir hier exakt das gleiche Stacklayout mit dem Unterschied des ersten
Arguments von \printf, welches auf den auszugebenden String zeigt. Unser Code tut hier das Folgende:

Er ersetzt das erste Argument der Funktion mit der Adresse (d.i. dem Pointer) auf den String und springt zu \printf als
ob wir nicht unsere Funktion \ttf, sondern direkt \printf aufrufen würden. Die Funktion \printf gibt einen String auf
\gls{stdout} aus und führt dann den \RET Befehl aus, der die \ac{RA} vom Stack holt. Der Control Flow wird nicht an \ttf
übergeben, sondern an den Aufrufer von \ttf, womit effektiv die Funktion \ttf umgangen wird.

\myindex{\CStandardLibrary!longjmp()}
\newcommand{\URLSJ}{\href{http://en.wikipedia.org/wiki/Setjmp.h}{wikipedia}}

% TODO \myref{}
All dies ist möglich, da \printf in allen Fällen ganz am Ende der Funktion \ttf aufgerufen wird. In gewisser Weise
besteht Ähnlichkeit zur Funktion \TT{longjmp()}\footnote{\URLSJ}. Natürlich geschieht all dies um die
Ausführungsgeschwindigkeit zu erhöhen.

Ein ähnlicher Fall mit dem ARM Compiler wird im Abschnitt \q{\PrintfSeveralArgumentsSectionName}
beschrieben:~(\myref{ARM_B_to_printf}).

\clearpage
\subsubsection{MSVC: x86 + \olly}
Laden wir unser Programm in \olly und zwingen es dazu zu glauben, dass \scanf stets ohne Fehler arbeitet.
Wenn die Adresse einer lokalen Variablen an \scanf übergeben wird, enthält die Variable zu Beginn einen zufälligen Wert,
in diesem Fall \TT{0x6E494714}:

\begin{figure}[H]
\centering
\myincludegraphics{patterns/04_scanf/3_checking_retval/olly_1.png}
\caption{\olly: Adresse der Variablen an \scanf übergeben}
\label{fig:scanf_ex3_olly_1}
\end{figure}

\clearpage
Während \scanf ausgeführt wird, geben wir in der Konsole etwas ein, das definitiv keine Zahl ist, z.B. \q{asdasd}.
\scanf beendet sich mit 0 in \EAX, was anzeigt, dass ein Fehler aufgetreten ist.

Wir können auch die lokale Variable auf dem Stack überprüfen und stellen fest, dass sie sich nicht verändert hat.
Was könnte \scanf hier auch hineinschreiben? Die Funktion hat nichts getan außer 0 zurückzugeben.

Versuchen wir unser Programm zu modifizieren, d.i. zu \q{hacken}.
Rechtsklick auf \EAX, in den Optionen finden wir \q{Set to 1}. Das ist was wir brauchen.

Wir haben jetzt 1 in \EAX, sodass die folgende Überprüfung wie gewünscht ausgeführt wird und \printf den Wert der
Variablen auf dem Stack ausgibt.

Wenn wir das Programm laufen lassen (F9), sehen wir das Folgende im Konsolenfenster:

\lstinputlisting[caption=console window]{patterns/04_scanf/3_checking_retval/console.txt}

Und tatsächlich ist 1850296084 die dezimale Darstellung der Zahl auf dem Stack (\TT{0x6E494714})!


