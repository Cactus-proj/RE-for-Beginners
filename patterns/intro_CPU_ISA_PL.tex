\subsection{Krótkie wprowadzenie do CPU}

\ac{CPU} (procesor) jest urządzeniem, które wykonuje bezpośrednio kod maszynowy programu.

\textbf{Terminologia:}

\begin{description}
\item[Instrukcja]: prymitywny rozkaz \ac{CPU}.
Najprostsze przykłady: przenoszenie (kopiowanie) danych między rejestrami, korzystanie z pamięci (zapis/odczyt), proste operacje arytmetyczne.

Z reguły, każdy \ac{CPU} ma swój zestaw instrukcji (\ac{ISA}).

\item[Kod maszynowy]: kod wykonywany bezpośrednio przez \ac{CPU}. 
Każda instrukcja kodu maszynowego zwykle jest kodowana za pomocą kilku bajtów.
\item[Język asemblera]: kod maszynowy plus niektóre rozszerzenia (np. makra), stworzone po to, żeby ułatwić pracę programiście.
\item[Rejestr CPU]: Każdy \ac{CPU} ma swój zestaw rejestrów ogólnego przeznaczenia (\ac{GPR}).
$\approx 8$ w x86, $\approx 16$ w x86-64 i $\approx 16$ w ARM.
Najłatwiej traktować rejestr jako zmienną tymczasową bez określonego typu.
Wyobraź sobie, że pisząc w języku wyższego poziomu masz dostępne tylko 8 zmiennych o szerokości 32 (lub 64) bitów.
Te 8 zmiennych to właśnie rejestry. Wbrew pozorom można z nimi naprawdę wiele zrobić!
\end{description}

Dlaczego występują języki niższego i wyższego poziomu? Odpowiedź jest prosta: ludzie i procesory różnią się między sobą - dla człowieka jest o wiele łatwiej pisać w wysokopoziomowym języku programowania typu \CCpp, Java czy Python, а dla procesora łatwiej jest pracować na niższym poziomie abstrakcji.
Zapewne możnaby zbudować procesor, który wykonywałby kod wysokiego poziomu, ale jego budowa byłaby dużo bardziej skomplikowana niż  budowa procesorów jakie obecnie znamy.
I odwrotnie, pisanie w języku asemblera jest dla ludzi bardzo niewygodne, z uwagi na jego niski poziom i trudność kodowania bez popełniania całej masy drobnych błędów.
Program, który potrafi konwertować język wysokiego poziomu na kod asemblera, nazywamy \emph{kompilatorem}.

\myindex{ARM!\ARMMode}%
\myindex{ARM!\ThumbMode}%
\myindex{ARM!\ThumbTwoMode}%

\subsubsection{Kilka słów o różnicy między \ac{ISA}}
x86 od zawsze  zawierał instrukcje o różnej długości, więc kiedy nadeszła era 64-bitowej architektury, rozszerzenia x64 nie wpłynęły znacząco na \ac{ISA}.

ARM to procesor \ac{RISC} zaprojektowany tak, żeby zawierał wszystkie instrukcje tej samej długości, co miało sporo zalet w przeszłosci.
Na samym początku wszystkie instrukcje ARM były kodowane na czterech bajtach (obecnie \q{tryb ARM})%
\footnote{
Instrukcje o stałym rozmiarze są wygodne, bo dzięki temu można łatwo znaleźć adres następnej (lub poprzedniej) instrukcji. Dokładniejsze wyjaśnienie znajduje się w sekcji o operatorze switch()~(\myref{sec:SwitchARMLot}).
}.

Później się okazało, że nie jest to zbyt ekonomiczne, bo najczęściej używane przez procesor instrukcje\footnote{Są nimi MOV/PUSH/CALL/Jcc} mogą być zakodowane z wykorzystaniem mniejszej ilości informacji.
Więc dodano inną \ac{ISA} o nazwie "Thumb", gdzie każda instrukcja jest kodowana za pomocą tylko 2 bajtów.
Jednak nie \emph{wszystkie} instrukcje ARM mogą być zakodowane na 2 bajtach, więc zestaw instrukcji Thumb jest ograniczony.
Kod skompilowany dla trybu ARM i Thumb może współdziałać w jednym programie.

Później twórcy ARM stwierdzili, że Thumb można rozszerzyć: tak pojawił się Thumb-2 (w ARMv7).
Thumb-2 to wciąż dwubajtowe instrukcje, ale niektóre nowe instrukcje mają długość 4 bajtów.
Szeroko rozpowszechnioną i błędną opinią jest to, że Thumb-2 to mieszanina ARM i Thumb.
Tryb Thumb-2 został rozszerzony w celu wsparcia dla wszystkich możliwości procesora, by mógł
konkurować z trybem ARM---co zostało w pełni osiągnięte, gdyż większość aplikacji na urządzenia \idevices są skompilowane dla
zestawu instrukcji Thumb-2, (trzeba przyznać, że w dużej mierze jest to zasługa Xcode, który robi to domyślnie).

Później pojawił się 64-bitowy ARM. Jest to \ac{ISA} znowu z 4-bajtowymi instrukcjami, bez dodatkowego trybu Thumb.
Jednak nowe 64-bitowe wymagania wpłyneły na \ac{ISA} tak, że obecnie mamy 3 zestawy instrukcji ARM: tryb ARM, tryb Thumb/Thumb-2 i ARM64.
Te zestawy instrukcji częściowo się pokrywają, ale można powiedzieć, że są to różne zestawy a nie wariacje tego samego \ac{ISA}.
W tej książce postaramy się zaprezentować fragmenty kodu we wszystkich trzech trybach \ac{ISA}.
\myindex{PowerPC}%
\myindex{MIPS}%
\myindex{Alpha AXP}%
Istnieje jeszcze wiele innych \ac{RISC} \ac{ISA} z instrukcjami 32-bitowej długości~--- np. MIPS, PowerPC i Alpha AXP.

