\mysection{Déplombage de logiciel}
\myindex{\SoftwareCracking}

La grande majorité des logiciels peuvent être déplombés comme ça --- en cherchant
l'endroit où la protection est vérifiée, un dongle (\myref{dongles}), une clef de
licence, un numéro de série, etc.

Souvent, ça ressemble à ça:

\begin{lstlisting}[style=customasmx86]
...
call check_protection
jz   all_OK
call message_box_protection_missing
call exit
all_OK:
; proceed
...
\end{lstlisting}

Donc, si vous voyez un patch (ou ``crack''), qui déplombe un logiciel, et que ce
patch remplace un ou des octets 0x74/0x75 (JZ/JNZ) par 0xEB (JMP), c'est ça.

Le processus de déplombage de logiciel revient à une recherche de ce \INS{JMP}.

\myhrule{}

Il y a aussi les cas où le logiciel vérifie la protection de temps à autre, ceci
peut être un dongle, ou un serveur de licence qui peut être interrogé depuis Internet.
Dans ce cas, vous devez chercher une fonction qui vérifie la protection.
Puis, la modifier, pour y mettre \TT{xor eax, eax / retn}, ou \TT{mov eax, 1 / retn}.

Il est important de comprendre qu'après avoir patché le début d'une fonction,
souvent, il y a des octets résiduels qui suivent ces deux instructions.
Ces restes consistent en une partie d'une instruction et les instructions suivantes.

Ceci est un cas réel.
Le début de la fonction que nous voulons \emph{remplacer} par \TT{return 1;}

\begin{lstlisting}[style=customasmx86,caption=Before]
8BFF                           mov         edi,edi
55                             push        ebp
8BEC                           mov         ebp,esp
81EC68080000                   sub         esp,000000868
A110C00001                     mov         eax,[00100C010]
33C5                           xor         eax,ebp
8945FC                         mov         [ebp][-4],eax
53                             push        ebx
8B5D08                         mov         ebx,[ebp][8]
...
\end{lstlisting}

\begin{lstlisting}[style=customasmx86,caption=After]
B801000000                     mov         eax,1
C3                             retn
EC                             in          al,dx
68080000A1                     push        0A1000008
10C0                           adc         al,al
0001                           add         [ecx],al
33C5                           xor         eax,ebp
8945FC                         mov         [ebp][-4],eax
53                             push        ebx
8B5D08                         mov         ebx,[ebp][8]
...
\end{lstlisting}

Quelques instructions incorrectes apparaissent --- \INS{IN}, \INS{PUSH}, \INS{ADC}, \INS{ADD},
après lesquelles, le désassembleur Hiew (que j'ai utilisé) s'est synchronisé et
a continué de désassembler le reste.

Ceci n'est pas important --- toutes ces instructions qui suivent \INS{RETN} ne
seront jamais exécutées, à moins qu'un saut direct se produise quelque part, et
ça ne sera pas possible en général.

\myhrule{}

Il peut aussi y avoir une variable globale booléenne, un flag indiquant si le
logiciel est enregistré ou non.

\begin{lstlisting}[style=customasmx86]
init_etc proc
...
call check_protection_or_license_file
mov  is_demo, eax
...
retn
init_etc endp

...

save_file proc
...
mov  eax, is_demo
cmp  eax, 1
jz   all_OK1

call message_box_it_is_a_demo_no_saving_allowed
retn

:all_OK1
; continuer en sauvant le fichier

...

save_proc endp

somewhere_else proc

mov  eax, is_demo
cmp  eax, 1
jz   all_OK

; contrôler si le programme fonctionne depuis 15 minutes
; sortir si c'est le cas
; ou montrer un écran fixe

:all_OK2
; continuer

somewhere_else endp
\end{lstlisting}

Le début de la fonction \verb|check_protection_or_license_file()| peut être patché,
afin qu'elle renvoie toujours 1, ou, si c'est mieux pour une raison quelconques,
toutes les instructions \INS{JZ}/\INS{JNZ} peuvent être patchées de même

Plus sur le patching: \ref{patching}.
