\mysection{Взлом ПО}
\label{\SoftwareCracking}

Огромная часть ПО взламывается таким образом --- поиском того самого места, где проверяется защита, донгла (\myref{dongles}),
лицензионный ключ, серийный номер, итд.

Очень часто это так и выглядит:

\begin{lstlisting}[style=customasmx86]
...
call check_protection
jz   all_OK
call message_box_protection_missing
call exit
all_OK:
; proceed
...
\end{lstlisting}

Так что если вы видите патч (или ``крэк''), взламывающий некое ПО,
и этот патч заменяет байт(ы) 0x74/0x75 (JZ/JNZ) на 0xEB (JMP), то это оно и есть.

Собственно, процесс взлома большинства ПО сводится к поиску того самого \INS{JMP}-а.

\myhrule{}

Но бывает и так, что ПО проверяет защиту время от времени, это может быть донгла,
или ПО может через интернет проверять сервер лицензий.
Тогда приходится искать ф-цию, проверяющую защиту.
Затем пропатчить, вписав туда \TT{xor eax, eax / retn}, либо \TT{mov eax, 1 / retn}.

Важно понимать, что пропатчив начало ф-ции двумя инструкциями, обычно, за ними остается некий мусор.
Он состоит из куска одной из инструкций и нескольких последующих инструкций.

Вот реальный пример.
Начало некоей ф-ции, которую мы хотим \emph{заменить} на \TT{return 1;}

\begin{lstlisting}[style=customasmx86,caption=Было]
8BFF                           mov         edi,edi
55                             push        ebp
8BEC                           mov         ebp,esp
81EC68080000                   sub         esp,000000868
A110C00001                     mov         eax,[00100C010]
33C5                           xor         eax,ebp
8945FC                         mov         [ebp][-4],eax
53                             push        ebx
8B5D08                         mov         ebx,[ebp][8]
...
\end{lstlisting}

\begin{lstlisting}[style=customasmx86,caption=Стало]
B801000000                     mov         eax,1
C3                             retn
EC                             in          al,dx
68080000A1                     push        0A1000008
10C0                           adc         al,al
0001                           add         [ecx],al
33C5                           xor         eax,ebp
8945FC                         mov         [ebp][-4],eax
53                             push        ebx
8B5D08                         mov         ebx,[ebp][8]
...
\end{lstlisting}

Появляются несколько некорректных инструкций --- \INS{IN}, \INS{PUSH}, \INS{ADC}, \INS{ADD},
затем дизассемблер Hiew (котором я только что воспользовался)
синхронизируется и продолжает дизассемблировать корректно остальную часть ф-ции.

Всё это не важно --- все эти инструкции следующие за \INS{RETN} не будут исполняться никогда, если только на какую-то из них не будет прямого
перехода откуда-то, чего, конечно, в общем случае, не будет.

\myhrule{}

Также, нередко бывает некая глобальная булева переменная, хранящая флаг, зарегестрированно ли ПО или нет.

\begin{lstlisting}[style=customasmx86]
init_etc proc
...
call check_protection_or_license_file
mov  is_demo, eax
...
retn
init_etc endp

...

save_file proc
...
mov  eax, is_demo
cmp  eax, 1
jz   all_OK1

call message_box_it_is_a_demo_no_saving_allowed
retn

:all_OK1
; continue saving file

...

save_proc endp

somewhere_else proc

mov  eax, is_demo
cmp  eax, 1
jz   all_OK

; check if we run for 15 minutes
; exit if it is so
; or show nagging screen

:all_OK2
; continue

somewhere_else endp
\end{lstlisting}

Тут можно или патчить начало ф-ции \verb|check_protection_or_license_file()|,
чтобы она всегда возвращала 1, либо, если вдруг так лучше,
патчить все инструкции \INS{JZ}/\INS{JNZ}.

