\mysection{Zwracanie wartości}
\label{ret_val_func}

Inną prostą funkcją jest taka, która zwraca stałą wartość.

\lstinputlisting[label=lst:ret_val_func,caption=Kod w \CCpp,style=customc]{patterns/011_ret/1.c}

Skompilujmy ją.

\subsection{x86}

Poniżej efekt kompilacji z optymalizacją kompilatorami GCC i MSVC na x86:

\lstinputlisting[caption=\Optimizing GCC/MSVC (\assemblyOutput),style=customasmx86]{patterns/011_ret/1.s}

\myindex{x86!\Instructions!RET}
Są tu tylko dwie instrukcje: pierwsza zapisuje wartość 123 do rejestru EAX,
który umownie jest używany do przechowywania wartości zwracanej,
a drugą jest \RET, która zwraca sterowanie do \glslink{caller}{funkcji wywołującej}.

Funkcja wywołująca pobierze wynik z rejestru \EAX.

\subsection{ARM}

W kodzie maszynowym na ARM widać kilka różnic:

\lstinputlisting[caption=\OptimizingKeilVI (\ARMMode) ASM Output,style=customasmARM]{patterns/011_ret/1_Keil_ARM_O3.s}

ARM używa rejestru \Reg{0} do zwracania wartości z funkcji, więc 123 kopiowane jest do \Reg{0}.

\myindex{ARM!\Instructions!MOV}
\myindex{x86!\Instructions!MOV}
Warto zaznaczyć, że nazwa instrukcji \MOV jest myląca, zarówno na x86 jak i na ARM.

Dane nie są \emph{przenoszone}, tylko \emph{kopiowane}.

\subsection{MIPS}

\label{MIPS_leaf_function_ex1}

Wyjście asemblera GCC oznacza rejestry numerami:

\lstinputlisting[caption=\Optimizing GCC 4.4.5 (\assemblyOutput),style=customasmMIPS]{patterns/011_ret/MIPS.s}

\dots a \IDA używa nazw umownych:

\lstinputlisting[caption=\Optimizing GCC 4.4.5 (IDA),style=customasmMIPS]{patterns/011_ret/MIPS_IDA.lst}

Rejestr \$2 (nazwa umowna - \$V0) używany jest do przechowywanie wartości zwracanej z funkcji.
\myindex{MIPS!\Pseudoinstructions!LI}
\INS{LI} to skrót od ``Load Immediate'' i w architekturze MIPS jest odpowiednikiem instrukcji \MOV z x86.

\myindex{MIPS!\Instructions!J}
Drugą instrukcją jest instrukcja skoku (J lub JR), która zwraca sterowanie do \glslink{caller}{funkcji wywołującej}.

\myindex{MIPS!Branch delay slot}
Możesz się zastanawiać dlaczego instrukcje LI i J/JR są w odwrotnej kolejności.
Jest to efekt optymalizacji przetwarzania potokowego w architekturze \ac{RISC}, zwanej \q{branch delay slot}.

Przyczyną wprowadzanie takiego rozwiązania jest dziwactwo w niektórych architekturach typu RISC
i nie jest to dla nas istotne - po prostu przyjmijmy, że w asemblerze MIPS instrukcja następująca po instrukcji skoku jump/branch
jest wykonywana przed samym skokiem.

W rezultacie, instrukcje typu branch są zawsze zamienione z instrukcją, która jest wykonywana przed nimi.

W praktyce często występują funkcje, które jedynie zwracają 1 (\emph{true - prawdę}) lub 0 (\emph{false - fałsz}).

Najmniejsze UNIXowe narzędzia, \emph{/bin/true} i \emph{/bin/false} zwracają odpowiednio 0 i 1, jako kod wyjścia.
(Zero, jako kod wyjścia, oznacza zwykle sukces, natomiast kod niezerowy oznacza błąd.)
