\subsubsection{MSVC + \olly}
\myindex{\olly}
Prześledźmy działanie programu w \olly.
Gdy zatrzymamy się na pierwszej instrukcji w \ttf, używającej jednego z argumentów (pierwszego),
widać, że \EBP pokazuje na \glslink{stack frame}{ramkę stosu} (oznaczona czerwonym prostokątem).

Pierwszym elementem w ramce stosu jest zapisana wartość \EBP,
drugim jest \ac{RA} (adres powrotu), trzecim jest pierwszy argument funkcji, następnie drugi i trzeci.

Do odwołania się do pierwszego arugmentu należy dodać dokładnie 8 (dwa 32-bitowe słowa) do \EBP.

\olly potrafi to rozpoznać i dodał odpowiednie komentarze do elementów na stosie:

\q{RETURN from} czy \q{Arg1 = \dots}, etc.

Tak naprawdę - argumenty funkcji nie należą do ramki stosu funkcji, są elementami ramki stosu \glslink{caller}{funkcji wywołującej}.

Z tego powodu \olly oznaczył argumenty \q{Arg} jako elementy innej ramki stosu.

\begin{figure}[H]
\centering
\myincludegraphics{patterns/05_passing_arguments/olly.png}
\caption{\olly: inside of \ttf{} function}
\label{fig:passing_arguments_olly}
\end{figure}

