\subsection{x86}

\subsubsection{MSVC}

Poniżej wynik kompilacji (MSVC 2010 Express):

\lstinputlisting[label=src:passing_arguments_ex_MSVC_cdecl,caption=MSVC 2010 Express,style=customasmx86]{patterns/05_passing_arguments/msvc_EN.asm}

\myindex{x86!\Registers!EBP}

Na listingu widać, jak funkcja \main odkłada na stos 3 liczby i wywołuje \TT{f(int,int,int).}

Dostęp do argumentów \ttf uzyskuje za pomocą makr, jak np.:\\
\TT{\_a\$ = 8},
podobnie jak do zmiennych lokalnych, ale z dodatnim przesunięciem.
Niejako adresujemy pamięć \emph{poza stosem}, gdyż stos rośnie w dół, a my dodajemy wartość dodatnią  \TT{\_a\$} do rejestru \EBP (wskaźnik ramki stosu).

\myindex{x86!\Instructions!IMUL}
\myindex{x86!\Instructions!ADD}

Następnie wartość $a$ jest zapisywana do \EAX. Po wykonaniu instrukcji \IMUL, wartość w \EAX
jest iloczynem wartości z \EAX i wartości wskazywanej przez przesunięcie \TT{\_b}.

Kolejno wykonywana jest instrukcja \ADD, która dodaje wartość pokazywaną przez przesunięcie \TT{\_c} do \EAX.

Wartość w \EAX już nie musi być nigdzie zapisywana, gdyż jest to wynik funkcji, a w tej konwencji wywoływania jest on zwracany przez rejestr \EAX.
Po powrocie \glslink{caller}{funkcja wywołująca} pobiera wartość z \EAX i używa jako argumentu do \printf.

\subsubsection{MSVC + \olly}
\myindex{\olly}
Prześledźmy działanie programu w \olly.
Gdy zatrzymamy się na pierwszej instrukcji w \ttf, używającej jednego z argumentów (pierwszego),
widać, że \EBP pokazuje na \glslink{stack frame}{ramkę stosu} (oznaczona czerwonym prostokątem).

Pierwszym elementem w ramce stosu jest zapisana wartość \EBP,
drugim jest \ac{RA} (adres powrotu), trzecim jest pierwszy argument funkcji, następnie drugi i trzeci.

Do odwołania się do pierwszego arugmentu należy dodać dokładnie 8 (dwa 32-bitowe słowa) do \EBP.

\olly potrafi to rozpoznać i dodał odpowiednie komentarze do elementów na stosie:

\q{RETURN from} czy \q{Arg1 = \dots}, etc.

Tak naprawdę - argumenty funkcji nie należą do ramki stosu funkcji, są elementami ramki stosu \glslink{caller}{funkcji wywołującej}.

Z tego powodu \olly oznaczył argumenty \q{Arg} jako elementy innej ramki stosu.

\begin{figure}[H]
\centering
\myincludegraphics{patterns/05_passing_arguments/olly.png}
\caption{\olly: inside of \ttf{} function}
\label{fig:passing_arguments_olly}
\end{figure}



\subsubsection{GCC}

Skompilujmy ten sam przykład w GCC 4.4.1 i podejrzyjmy rezultat w programie \IDA:

\lstinputlisting[caption=GCC 4.4.1,style=customasmx86]{patterns/05_passing_arguments/gcc_PL.asm}

Efekt jest taki sam, z małymi różnicami, które już omówiliśmy wcześniej.

\glslink{stack pointer}{Wskaźnik stosu} nie jest przywracany po dwóch wywołaniach funkcji (f oraz \printf),
ponieważ przedostatnia instrukcja \TT{LEAVE} (\myref{x86_ins:LEAVE}) zajmie się tym na końcu.
