\subsection{x64}

\myindex{x86-64}
В x86-64 всё немного иначе, здесь аргументы функции (4 или 6) передаются через регистры, 
а \gls{callee} из читает их из регистров, а не из стека.

\subsubsection{MSVC}

\Optimizing MSVC:

\lstinputlisting[caption=\Optimizing MSVC 2012 x64,style=customasmx86]{patterns/05_passing_arguments/x64_MSVC_Ox_RU.asm}

Как видно, очень компактная функция \ttf берет аргументы прямо из регистров.

Инструкция \LEA используется здесь для сложения чисел. 
Должно быть компилятор посчитал, что это будет эффективнее использования \TT{ADD}.

\myindex{x86!\Instructions!LEA}
В самой \main{} \LEA{} также используется для подготовки первого и третьего аргумента: должно быть,
компилятор решил, что \LEA{} будет работать здесь быстрее, чем загрузка значения в регистр при помощи \MOV.

Попробуем посмотреть вывод неоптимизирующего MSVC:

\lstinputlisting[caption=MSVC 2012 x64,style=customasmx86]{patterns/05_passing_arguments/x64_MSVC_IDA_RU.asm}

Немного путанее: все 3 аргумента из регистров зачем-то сохраняются в стеке.

\myindex{Shadow space}
\label{shadow_space}
Это называется \q{shadow space} \footnote{\href{http://msdn.microsoft.com/en-us/library/zthk2dkh(v=vs.80).aspx}{MSDN}}: 
каждая функция в Win64 может (хотя и не обязана) сохранять значения 4-х регистров там.

Это делается по крайней мере из-за двух причин: 
1) в большой функции отвести целый регистр (а тем более 4 регистра) для входного аргумента 
слишком расточительно, так что к нему будет обращение через стек;

2) отладчик всегда знает, где найти аргументы функции в момент останова
\footnote{\href{http://msdn.microsoft.com/en-us/library/ew5tede7(v=VS.90).aspx}{MSDN}}.

Так что, какие-то большие функции могут сохранять входные аргументы в \q{shadow space} 
для использования в будущем, а небольшие функции, как наша, могут этого и не делать.

Место в стеке для \q{shadow space} выделяет именно \gls{caller}.

\subsubsection{GCC}

\Optimizing GCC также делает понятный код:

\lstinputlisting[caption=\Optimizing GCC 4.4.6 x64,style=customasmx86]{patterns/05_passing_arguments/x64_GCC_O3_RU.s}

\NonOptimizing GCC:

\lstinputlisting[caption=GCC 4.4.6 x64,style=customasmx86]{patterns/05_passing_arguments/x64_GCC_RU.s}

\myindex{Shadow space}
В соглашении о вызовах System V *NIX (\SysVABI) нет \q{shadow space}, но \gls{callee} тоже иногда
должен сохранять где-то аргументы, потому что, опять же, регистров может и не хватить на все действия.
Что мы здесь и видим.

\subsubsection{GCC: uint64\_t вместо int}

Наш пример работал с 32-битным \Tint, поэтому использовались 32-битные части регистров с префиксом \TT{E-}.

Его можно немного переделать, чтобы он заработал с 64-битными значениями:

\lstinputlisting[style=customc]{patterns/05_passing_arguments/ex64.c}

\lstinputlisting[caption=\Optimizing GCC 4.4.6 x64,style=customasmx86]{patterns/05_passing_arguments/ex64_GCC_O3_IDA_RU.asm}

Собствено, всё то же самое, только используются регистры \emph{целиком}, с префиксом \TT{R-}.

