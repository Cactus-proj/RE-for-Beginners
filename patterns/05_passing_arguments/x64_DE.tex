\subsection{x64}

\myindex{x86-64}

Die Geschichte bei x86-64 Funktions Argumenten ist ein wenig anders (zumindest für die ersten vier bis sechs)
sie werden über die Register übergeben z.b. der \gls{callee} liest direkt aus den Registern anstatt vom Stack 
zu lesen.

\subsubsection{MSVC}

\Optimizing MSVC:

\lstinputlisting[caption=\Optimizing MSVC 2012 x64,style=customasmx86]{patterns/05_passing_arguments/x64_MSVC_Ox_EN.asm}

Wie wir sehen können, die compact Funktion \ttf nimmt alle Argumente aus den Registern.

Die \LEA Instruktion wird hier für Addition benutzt,
scheinbar hat der Compiler die Instruktion für schneller befunden als
die \TT{ADD} Instruktion.

\myindex{x86!\Instructions!LEA}

\LEA wird auch benutzt in der \main Funktion um das erste und das dritte \ttf Argument vor zu bereiten.
Der Compiler muss entschieden haben das dies schneller abgearbeitet wird als die Werte in die Register 
zu laden mit der \MOV Instruktion.

Lasst uns einen Blick auf nicht optimierte MSVC Ausgabe werfen:

\lstinputlisting[caption=MSVC 2012 x64,style=customasmx86]{patterns/05_passing_arguments/x64_MSVC_IDA_EN.asm}

Es sieht ein bisschen wie ein Puzzle aus, weil alle drei Argumente aus den Registern auf dem Stack
gespeichert werden aus irgend einem Grund.

\myindex{Shadow space}
\label{shadow_space}
Dies bezeichnet man als \q{shadow space}

\footnote{\href{http://msdn.microsoft.com/en-us/library/zthk2dkh(v=vs.80).aspx}{MSDN}}: 
So wird sich wahrscheinlich jede Win64 EXE verhalten und alle vier Register Werte auf dem Stack speichern.

Das wird aus zwei Gründen so gemacht:

1) Es ist ziemlich übertrieben ein ganzes Register (oder gar vier Register) zu Reservieren für eine
Argument Übergabe, also werden die Argumente über den Stack zugänglich gemacht.
2) Der Debugger weiß immer wo die Funktions Argumente zu finden sind bei einem breakpoint\footnote{\href{http://msdn.microsoft.com/en-us/library/ew5tede7(v=VS.90).aspx}{MSDN}}.


Also, so können größere Funktionen ihre Eingabe Argumente im \q{shadows space} speichern wenn die Funktion
auf die Argumente während der Laufzeit zugreifen will, kleinere Funktionen (wie unsere) zeigen dieses Verhalten 
nicht. 

Es liegt in der Verantwortung vom \gls{caller} den \q{shadow space} auf dem Stack zu allozieren.

\subsubsection{GCC}

Optimierter GCC generiert mehr oder minder verständlichen Code:

\lstinputlisting[caption=\Optimizing GCC 4.4.6 x64,style=customasmx86]{patterns/05_passing_arguments/x64_GCC_O3_EN.s}

\NonOptimizing GCC:

\lstinputlisting[caption=GCC 4.4.6 x64,style=customasmx86]{patterns/05_passing_arguments/x64_GCC_EN.s}

\myindex{Shadow space}

Bei System V *NIX Systemen (\SysVABI) ist kein \q{shadow space} nötig, aber der \gls{callee} will vielleicht
seine Argumente irgendwo speichern im Fall das keine oder zu wenig Register frei sind.

\subsubsection{GCC: uint64\_t statt int}

Unser Beispiel funktioniert mit 32-Bit \Tint, weshalb auch 32-Bit Register Bereiche benutzt werden (mit dem Präfix \TT{E-}).

Es lassen sich auch ohne Probleme 64-Bit Werte benutzen:

\lstinputlisting{patterns/05_passing_arguments/ex64.c}

\lstinputlisting[caption=\Optimizing GCC 4.4.6 x64,style=customasmx86]{patterns/05_passing_arguments/ex64_GCC_O3_IDA_EN.asm}

Der Code ist der gleiche, aber diesmal werden die \emph{full size} 64-Bit Register benutzt (mit dem \TT{R-} Präfix).

