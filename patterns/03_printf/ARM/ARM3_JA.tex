\subsubsection{ARM: 3つの引数}

引数を渡すためのARMの伝統的なスキーム(呼び出し規約)は、次のように動作します。
最初の4つの引数は\Reg{0}-\Reg{3}レジスタに渡されます。 残りの引数はスタックを介して。 
これはfastcall~(\myref{fastcall})またはwin64~(\myref{sec:callingconventions_win64})の引数渡しスキームに似ています。

\myparagraph{32ビットARM}

\mysubparagraph{\NonOptimizingKeilVI (\ARMMode)}

\begin{lstlisting}[caption=\NonOptimizingKeilVI (\ARMMode),style=customasmARM]
.text:00000000 main
.text:00000000 10 40 2D E9   STMFD   SP!, {R4,LR}
.text:00000004 03 30 A0 E3   MOV     R3, #3
.text:00000008 02 20 A0 E3   MOV     R2, #2
.text:0000000C 01 10 A0 E3   MOV     R1, #1
.text:00000010 08 00 8F E2   ADR     R0, aADBDCD     ; "a=\%d; b=\%d; c=\%d"
.text:00000014 06 00 00 EB   BL      __2printf
.text:00000018 00 00 A0 E3   MOV     R0, #0          ; return 0
.text:0000001C 10 80 BD E8   LDMFD   SP!, {R4,PC}
\end{lstlisting}

したがって、最初の4つの引数は、\Reg{0}-\Reg{3}レジスタをこの順序で渡します。
\Reg{0}の \printf 形式文字列へのポインタ、\Reg{1}の1、\Reg{2}の2、\Reg{3}の3の順です。 
\GTT{0x18}の命令は\Reg{0}に0を書き込みます。 これは\emph{return 0} となるCの命令文です。 
珍しいことは何もありません。

\OptimizingKeilVI は同じコードを生成します。

\mysubparagraph{\OptimizingKeilVI (\ThumbMode)}

\begin{lstlisting}[caption=\OptimizingKeilVI (\ThumbMode),style=customasmARM]
.text:00000000 main
.text:00000000 10 B5        PUSH    {R4,LR}
.text:00000002 03 23        MOVS    R3, #3
.text:00000004 02 22        MOVS    R2, #2
.text:00000006 01 21        MOVS    R1, #1
.text:00000008 02 A0        ADR     R0, aADBDCD     ; "a=\%d; b=\%d; c=\%d"
.text:0000000A 00 F0 0D F8  BL      __2printf
.text:0000000E 00 20        MOVS    R0, #0
.text:00000010 10 BD        POP     {R4,PC}
\end{lstlisting}

ARMモードの最適化されていないコードとの大きな違いはありません。

\mysubparagraph{\OptimizingKeilVI (\ARMMode) + returnを削除してみる}
\label{ARM_B_to_printf}

\emph{return 0}を取り除いて例を少し修正しましょう:

\begin{lstlisting}[style=customc]
#include <stdio.h>

void main()
{
	printf("a=%d; b=%d; c=%d", 1, 2, 3);
};
\end{lstlisting}

結果はちょっと珍しくなりました。

\lstinputlisting[caption=\OptimizingKeilVI (\ARMMode),style=customasmARM]{patterns/03_printf/ARM/tmp3.asm}

\myindex{ARM!\Registers!Link Register}
\myindex{ARM!\Instructions!B}
\myindex{Function epilogue}
これはARMモード用に最適化された(\Othree)バージョンであり、今回は\INS{B}を使い慣れた\INS{BL}ではなく最後の命令と見なします。
この最適化されたバージョンと前のバージョン(最適化なしでコンパイルされたもの)との別の違いは、
関数のプロローグとエピローグ(\Reg{0} と \ac{LR}レジスタの値を保持する命令)の欠如です。 
\INS{B}命令は、x86の \JMP と同様に、\ac{LR}レジスタの操作なしで別のアドレスにジャンプするだけです。
それはなぜ機能するのでしょうか?実際、このコードは以前のコードと事実上同等です。
主な理由は2つあります。1)スタックも\ac{SP}(\gls{stack pointer})も変更されていません。 
2) \printf の呼び出しが最後の命令なので、その後は何も起こりません。
完了すると、 \printf 関数は\ac{LR}に格納されているアドレスにコントロールを返します。 
\ac{LR}は現在、関数が呼び出されたポイントのアドレスを格納しているので、 \printf からの制御はそのポイントに返されます。
したがって、\ac{LR}を変更する必要がないため、\ac{LR}を節約する必要はありません。 
\printf 以外の関数呼び出しがないため、\ac{LR}を変更する必要はありません。
さらに、この呼び出しの後、私たちは何もしません!これがそのような最適化が可能な理由です。

この最適化は、最後のステートメントが別の関数の呼び出しである関数でよく使用されます。
同様の例をここに示します:\myref{jump_to_last_printf}

% TBT
% A somewhat simpler case was described above: \myref{Boolector}.

\myparagraph{ARM64}

\mysubparagraph{\NonOptimizing GCC (Linaro) 4.9}

\lstinputlisting[caption=\NonOptimizing GCC (Linaro) 4.9,style=customasmARM]{patterns/03_printf/ARM/ARM3_O0_JA.lst}

\myindex{ARM!\Instructions!STP}

第1の命令\INS{STP} (\emph{ストアペア})は、\ac{FP}(X29)および\ac{LR}(X30)をスタックに保存します。 
2番目の\INS{ADD X29, SP, 0}命令がスタックフレームを形成します。 
\ac{SP}の値をX29に書き込むだけです。

\myindex{ARM!\Instructions!ADRP/ADD pair}
次に、使い慣れた\INS{ADRP}/\ADD 命令ペアを参照します。これは、文字列へのポインタを形成します。 
\myindex{ARM64!lo12}
すなわち、\emph{lo12}は、LC1アドレスの下位12ビットを \ADD 命令のオペコードに書き込みます。
\printf の文字列書式の\GTT{\%d}は32ビット \Tint なので、1,2および3は32ビットのレジスタ部分にロードされます。

\Optimizing GCC (Linaro) 4.9 は同じコードを生成します。
