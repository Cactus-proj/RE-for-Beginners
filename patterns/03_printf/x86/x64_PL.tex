\subsubsection{x64: 9 argumentów}

\myindex{x86-64}
\label{example_printf8_x64}
Zmieńmy nieco nasz przykład, by zobaczyć jak na x64 argumenty przekazywane są przez stos.
Zwiększymy liczbę argumentów do 9 (łańcuch znaków z formatem + 8 zmiennych typu \Tint):

\lstinputlisting[style=customc]{patterns/03_printf/2.c}

\myparagraph{MSVC}

Jak wspomnieliśmy poprzednio~--- pierwsze 4 argumenty na Win64 przekazywane są przez rejestry \RCX, \RDX, \Reg{8} i \Reg{9} a pozostałe przez stos\footnote{przyp. tłum. - rzecz wygląda inaczej w przypadku przekazywania argumentów zmiennoprzecinkowych, gdy może zostać wykorzystany rejestr wektorowy \XMM{0}-\XMM{3}}.

Widać to na wygenerowanym listingu. Stos został przygotowany przy pomocy instrukcji \MOV, zamiast \PUSH - wartości zostały odłożone ze wskazaniem adresu.

\lstinputlisting[caption=MSVC 2012 x64,style=customasmx86]{patterns/03_printf/x86/2_MSVC_x64_PL.asm}

Uważny czytelnik może się zastanawiać dlaczego dla wartości typu \Tint zostało zaalokowanych 8 bajtów, skoro wystarczą tylko 4?
Dla przypomnienia: 8 bajtów jest alokowanych dla każdego typu daych, krótszego niż 64 bity.
Powodem jest wygoda, łatwo w ten sposób wyliczyć adres dowolnego argumentu. Poza tym, wszystkie są przechowywane w pamięci na wyrównanych adresach.
W środowisku 32-bitowym jest podobnie - 4 bajty są zarezerwowane dla wszystkich typów nie dłuższych niż 4 bajty\footnote{przyp. tłum. - np. zmienna typu long long zajmie 8 bajtów.}.

% also for local variables?

\myparagraph{GCC}
Kompilacja na x86-64 systemach *NIX daje podobny wynik jak MSVC, jednak teraz 6 pierwszych argumentów jest przekazywanych przez rejestry \RDI, \RSI,
\RDX, \RCX, \Reg{8} i \Reg{9}, a cała reszta przez stos.

GCC generuje kod, który przechowuje wskaźnik stosu w \EDI zamiast w \RDI{}---co już zdążyliśmy zauważyć: \myref{hw_EDI_instead_of_RDI}.

Widzieliśmy też, że rejestr \EAX został wyzerowany przed wywołaniem funkcji \printf: \myref{SysVABI_input_EAX}.

\lstinputlisting[caption=\Optimizing GCC 4.4.6 x64,style=customasmx86]{patterns/03_printf/x86/2_GCC_x64_PL.s}

\myparagraph{GCC + GDB}
\myindex{GDB}

Wczytajmy plik wykonywalny do debuggera \ac{GDB}

\begin{lstlisting}
$ gcc -g 2.c -o 2
\end{lstlisting}

\begin{lstlisting}
$ gdb 2
GNU gdb (GDB) 7.6.1-ubuntu
...
Reading symbols from /home/dennis/polygon/2...done.
\end{lstlisting}

\begin{lstlisting}[caption=Ustawiamy breakpoint na funkcji \printf i zaczynamy wykonanie]
(gdb) b printf
Breakpoint 1 at 0x400410
(gdb) run
Starting program: /home/dennis/polygon/2 

Breakpoint 1, __printf (format=0x400628 "a=%d; b=%d; c=%d; d=%d; e=%d; f=%d; g=%d; h=%d\n") at printf.c:29
29	printf.c: No such file or directory.
\end{lstlisting}

Rejestry \RSI/\RDX/\RCX/\Reg{8}/\Reg{9} zostały ustawione na spodziewane wartości.
W \RIP przechowywany jest adres pierwszej instrukcji z funkcji \printf.

\begin{lstlisting}
(gdb) info registers
rax            0x0	0
rbx            0x0	0
rcx            0x3	3
rdx            0x2	2
rsi            0x1	1
rdi            0x400628	4195880
rbp            0x7fffffffdf60	0x7fffffffdf60
rsp            0x7fffffffdf38	0x7fffffffdf38
r8             0x4	4
r9             0x5	5
r10            0x7fffffffdce0	140737488346336
r11            0x7ffff7a65f60	140737348263776
r12            0x400440	4195392
r13            0x7fffffffe040	140737488347200
r14            0x0	0
r15            0x0	0
rip            0x7ffff7a65f60	0x7ffff7a65f60 <__printf>
...
\end{lstlisting}

\begin{lstlisting}[caption=Łańcuch znaków z formatem]
(gdb) x/s $rdi
0x400628:	"a=%d; b=%d; c=%d; d=%d; e=%d; f=%d; g=%d; h=%d\n"
\end{lstlisting}

Wyświetlimy zawartość stosu za pomocą komendy x/g ---\emph{g} oznacza \emph{giant word}, czyli wartość 64-bitową.

\begin{lstlisting}
(gdb) x/10g $rsp
0x7fffffffdf38:	0x0000000000400576	0x0000000000000006
0x7fffffffdf48:	0x0000000000000007	0x00007fff00000008
0x7fffffffdf58:	0x0000000000000000	0x0000000000000000
0x7fffffffdf68:	0x00007ffff7a33de5	0x0000000000000000
0x7fffffffdf78:	0x00007fffffffe048	0x0000000100000000
\end{lstlisting}

Jak w poprzednim przykładzie, pierwszy elementem na stosie jest adres powrotu \ac{RA}.
3 wartości przekazywane są przez stos: 6, 7, 8.
Widać, że 8 przekazywane jest z wypełnionymi 32 starszymim bitami: \GTT{0x00007fff00000008}.
Nie stanowi to problemu, ponieważ argument jest typu \Tint type, który jest 32-bitowy.
Starsze części rejestrów, albo elementów na stosie, mogą zawierać \q{losowe śmieci}.

Jeśli sprawdzimy gdzie zostanie zwrócone sterowanie po zakończeniu funkcji \printf, \ac{GDB} pokaże funkcję main:

\begin{lstlisting}[style=customasmx86]
(gdb) set disassembly-flavor intel
(gdb) disas 0x0000000000400576
Dump of assembler code for function main:
   0x000000000040052d <+0>:	push   rbp
   0x000000000040052e <+1>:	mov    rbp,rsp
   0x0000000000400531 <+4>:	sub    rsp,0x20
   0x0000000000400535 <+8>:	mov    DWORD PTR [rsp+0x10],0x8
   0x000000000040053d <+16>:	mov    DWORD PTR [rsp+0x8],0x7
   0x0000000000400545 <+24>:	mov    DWORD PTR [rsp],0x6
   0x000000000040054c <+31>:	mov    r9d,0x5
   0x0000000000400552 <+37>:	mov    r8d,0x4
   0x0000000000400558 <+43>:	mov    ecx,0x3
   0x000000000040055d <+48>:	mov    edx,0x2
   0x0000000000400562 <+53>:	mov    esi,0x1
   0x0000000000400567 <+58>:	mov    edi,0x400628
   0x000000000040056c <+63>:	mov    eax,0x0
   0x0000000000400571 <+68>:	call   0x400410 <printf@plt>
   0x0000000000400576 <+73>:	mov    eax,0x0
   0x000000000040057b <+78>:	leave  
   0x000000000040057c <+79>:	ret    
End of assembler dump.
\end{lstlisting}

Przeskoczmy na koniec funkcji \printf i wykonajmy jedną linię kodu z funkcji \main. W ramach tej linii rejestr \EAX został wyzerowany. Debugger pokazuje teraz na koniec bloku kodu. Możemy się upewnić, że \EAX rzeczywiście ma wartość 0, a \RIP pokazuje teraz na instrukcję \INS{LEAVE}, przedostatnią w funkcji \main.

\begin{lstlisting}
(gdb) finish
Run till exit from #0  __printf (format=0x400628 "a=%d; b=%d; c=%d; d=%d; e=%d; f=%d; g=%d; h=%d\n") at printf.c:29
a=1; b=2; c=3; d=4; e=5; f=6; g=7; h=8
main () at 2.c:6
6		return 0;
Value returned is $1 = 39
(gdb) next
7	};
(gdb) info registers
rax            0x0	0
rbx            0x0	0
rcx            0x26	38
rdx            0x7ffff7dd59f0	140737351866864
rsi            0x7fffffd9	2147483609
rdi            0x0	0
rbp            0x7fffffffdf60	0x7fffffffdf60
rsp            0x7fffffffdf40	0x7fffffffdf40
r8             0x7ffff7dd26a0	140737351853728
r9             0x7ffff7a60134	140737348239668
r10            0x7fffffffd5b0	140737488344496
r11            0x7ffff7a95900	140737348458752
r12            0x400440	4195392
r13            0x7fffffffe040	140737488347200
r14            0x0	0
r15            0x0	0
rip            0x40057b	0x40057b <main+78>
...
\end{lstlisting}
