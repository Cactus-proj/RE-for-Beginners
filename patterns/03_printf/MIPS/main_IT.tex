\subsection{MIPS}

\subsubsection{3 argomenti}
% to be sync: \subsubsection{3 integer arguments}

\myparagraph{\Optimizing GCC 4.4.5}

La differenza principale con l'esempio \q{\HelloWorldSectionName} è che in questo caso \printf è chiamata 
al posto di \puts, e 3 argomenti aggiuntivi sono passati attraverso i registri \$5\dots \$7 (o \$A0\dots \$A2).
Questo è il motivo per cui questi registri hanno il prefisso A-, che implica il loro uso per il passaggio di argomenti di funzioni.

\lstinputlisting[caption=\Optimizing GCC 4.4.5 (\assemblyOutput),style=customasmMIPS]{patterns/03_printf/MIPS/printf3.O3_IT.s}

\lstinputlisting[caption=\Optimizing GCC 4.4.5 (IDA),style=customasmMIPS]{patterns/03_printf/MIPS/printf3.O3.IDA_IT.lst}

\IDA ha fuso le coppie di istruzioni \INS{LUI} e \INS{ADDIU} in una unica pseudoistruzione \INS{LA}.
Questo è il motivo per cui non c'è nessuna istruzione all'indirizzo 0x1C: perchè \INS{LA} \emph{occupa} 8 byte.%

\myparagraph{\NonOptimizing GCC 4.4.5}

\NonOptimizing GCC è più verboso:

\lstinputlisting[caption=\NonOptimizing GCC 4.4.5 (\assemblyOutput),style=customasmMIPS]{patterns/03_printf/MIPS/printf3.O0_IT.s}

\lstinputlisting[caption=\NonOptimizing GCC 4.4.5 (IDA),style=customasmMIPS]{patterns/03_printf/MIPS/printf3.O0.IDA_IT.lst}

\subsubsection{8 argomenti}
% to be sync: \subsubsection{8 integer arguments}

Usiamo nuovamente l'esempio con 9 argomenti dalla sezione prcedente: \myref{example_printf8_x64}.

\lstinputlisting[style=customc]{patterns/03_printf/2.c}

\myparagraph{\Optimizing GCC 4.4.5}

Solo i primi 4 argomenti sono passati nei registri \$A0 \dots \$A3, gli altri sono passati tramite lo stack.
\myindex{MIPS!O32}

Questa è la calling convention O32 (che è la più comune nel mondo MIPS).
Altre calling conventions (come N32) possono usare i registri per scopi diversi.
% to be sync: Other calling conventions, or manually written assembly code, may use the registers for different purposes.

\myindex{MIPS!\Instructions!SW}

\INS{SW} è l'abbreviazione di \q{Store Word} (da un registro alla memoria).
MIPS manca di istruzioni per memorizzare un valore in memoria, è quindi necessario usare una coppia di istruzioni (LI/SW).

\lstinputlisting[caption=\Optimizing GCC 4.4.5 (\assemblyOutput),style=customasmMIPS]{patterns/03_printf/MIPS/printf8.O3_IT.s}

\lstinputlisting[caption=\Optimizing GCC 4.4.5 (IDA),style=customasmMIPS]{patterns/03_printf/MIPS/printf8.O3.IDA_IT.lst}

\myparagraph{\NonOptimizing GCC 4.4.5}

\NonOptimizing GCC è più verboso:

\lstinputlisting[caption=\NonOptimizing GCC 4.4.5 (\assemblyOutput),style=customasmMIPS]{patterns/03_printf/MIPS/printf8.O0_IT.s}

\lstinputlisting[caption=\NonOptimizing GCC 4.4.5 (IDA),style=customasmMIPS]{patterns/03_printf/MIPS/printf8.O0.IDA_IT.lst}
