\subsection{Routine di copia blocchi di memoria}
\label{loop_memcpy}

Le routine di copia della memoria, nel mondo reale, possono copiare 4 o 8 byte ad ogni iterazione, usando \ac{SIMD}, 
vettorizzazione, etc.
Ma per maggiore semplicità, questo esempio è il più semplice possibile.

\lstinputlisting[style=customc]{memcpy.c}

\subsubsection{Implementazione diretta}

\lstinputlisting[caption=GCC 4.9 x64 ottimizzato per la dimensione (-Os),style=customasmx86]{patterns/09_loops/memcpy/memcpy_GCC49_x64_Os_IT.s}

\lstinputlisting[caption=GCC 4.9 ARM64 ottimizzato per la dimensione (-Os),style=customasmARM]{patterns/09_loops/memcpy/memcpy_GCC49_ARM64_Os_IT.s}

\lstinputlisting[caption=\OptimizingKeilVI (\ThumbMode),style=customasmARM]{patterns/09_loops/memcpy/memcpy_Keil_Thumb_O3_IT.s}

\subsubsection{ARM in modalità ARM}

Keil in modalità ARM sfrutta appieno i suffissi condizionali:

\lstinputlisting[caption=\OptimizingKeilVI (\ARMMode),style=customasmARM]{patterns/09_loops/memcpy/memcpy_Keil_ARM_O3_IT.s}

Ecco perchè c'è solo un istruzione di diramazione anzichè 2.

\subsubsection{MIPS}

\lstinputlisting[caption=GCC 4.4.5 ottimizzato per la dimensione (-Os) (IDA),style=customasmMIPS]{patterns/09_loops/memcpy/memcpy_MIPS_Os_IDA_IT.lst}

\myindex{MIPS!\Instructions!LBU}
\myindex{MIPS!\Instructions!SB}

Qui abbiamo due nuove istruzioni: \INS{LBU} (\q{Load Byte Unsigned}) e \INS{SB} (\q{Store Byte}).

Esattamente come in ARM, tutti i registri in MIPS sono larghi 32 bit, non ci sono parti larghe un byte come in x86.

Quindi quando maneggiamo dei singoli byte, dobbiamo allocare interi registri a 32 bit per loro.

\INS{LBU} carica un byte e pulisce tutti gli altri bit (\q{Unsigned}).
\myindex{MIPS!\Instructions!LB}

D' altro canto, l' istruzione \INS{LB} (\q{Load Byte})  estende il segno del byte caricato a un valore su 32 bit.

\INS{SB} scrive solamente in memoria un byte dagli ultimi 8 bit del registro.

\subsubsection{Vettorizzazione}

\Optimizing GCC può fare molto di più con questo esempio: \myref{vec_memcpy}.
