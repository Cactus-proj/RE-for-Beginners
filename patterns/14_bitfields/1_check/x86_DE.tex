\subsubsection{x86}
\myindex{Windows!Win32}

Win32 API Beipsiel:

\begin{lstlisting}[style=customc]
	HANDLE fh;

	fh=CreateFile ("file", GENERIC_WRITE | GENERIC_READ, FILE_SHARE_READ, NULL, OPEN_ALWAYS, FILE_ATTRIBUTE_NORMAL, NULL);
\end{lstlisting}

We get (MSVC 2010):

\begin{lstlisting}[caption=MSVC 2010,style=customasmx86]
	push	0
	push	128		; 00000080H
	push	4
	push	0
	push	1
	push	-1073741824	; c0000000H
	push	OFFSET $SG78813
	call	DWORD PTR __imp__CreateFileA@28
	mov	DWORD PTR _fh$[ebp], eax
\end{lstlisting}

Schauen wir uns WinNT.h genauer an:

\begin{lstlisting}[caption=WinNT.h,style=customc]
#define GENERIC_READ                     (0x80000000L)
#define GENERIC_WRITE                    (0x40000000L)
#define GENERIC_EXECUTE                  (0x20000000L)
#define GENERIC_ALL                      (0x10000000L)
\end{lstlisting}
Alles eindeutig beschrieben: GENERIC\_READ | GENERIC\_WRITE = 0x80000000 |
0x40000000 = 0xC0000000. Dieser Wert wird als zweites Argument für die Funktion
\TT{CreateFile()}\footnote{\href{http://msdn.microsoft.com/en-us/library/aa363858(VS.85).aspx}{msdn.microsoft.com/en-us/library/aa363858(VS.85).aspx}}
verwendet.

Wie würde \TT{CreateFile()} diese Flags überprüfen?
\myindex{Windows!KERNEL32.DLL}
Wenn wir und die KERNEL32.DLL in Windows XP SP3 x86 anschauen, finden wir dieses
Codefragment in \TT{CreateFileW}:

\begin{lstlisting}[caption=KERNEL32.DLL (Windows XP SP3 x86),style=customasmx86]
.text:7C83D429     test    byte ptr [ebp+dwDesiredAccess+3], 40h
.text:7C83D42D     mov     [ebp+var_8], 1
.text:7C83D434     jz      short loc_7C83D417
.text:7C83D436     jmp     loc_7C810817
\end{lstlisting}

\myindex{x86!\Instructions!TEST}
Wir finden hier den \TEST Befehl, aber dieser nimmt nicht das ganze zweite
Argument, sonder nur das MSB (\TT{ebp+dwDesiredAccess+3}) und prüft es auf das
Flag \TT{0x40} (welches hier dem \TT{GENERIC\_WRITE} Flag entspricht).

\myindex{x86!\Instructions!AND}
\TEST ist prinzipiell der gleiche Befehl wie \AND, aber das Ergebnis wird nicht
gespeichert. (Erinnern wir uns, dass \CMP das gleiche macht wie \SUB, aber auch
ohne das Ergebnis zu speichern~(\myref{CMPandSUB})).

Die Logik dieses Codefragments ist die folgende:

\begin{lstlisting}[style=customc]
if ((dwDesiredAccess&0x40000000) == 0) goto loc_7C83D417
\end{lstlisting}

\myindex{x86!\Instructions!AND}
\myindex{x86!\Registers!ZF}
Wennn der \AND Befehl dieses Bit hinterlässt, wird das \ZF Flag gelöscht und der
bedingte Sprung \JZ wird nicht ausgeführt.
Der bedingte Sprung wird nur dann ausgeführt, wenn das Bit \TT{0x40000000} in
der Variable \TT{dwDesiredAccess} fehlt~---dann ist das Ergebnis von \AND 0, \ZF
wird gesetzt und der bedingte Sprung wird ausgeführt.

Schauen wir es uns mit GCC 4.4.1 unter Linux an:

\begin{lstlisting}[style=customc]
#include <stdio.h>
#include <fcntl.h>

void main()
{
	int handle;

	handle=open ("file", O_RDWR | O_CREAT);
};
\end{lstlisting}

Wir erhalten folgenden Code:

\lstinputlisting[caption=GCC 4.4.1,style=customasmx86]{patterns/14_bitfields/1_check/check.asm}

\myindex{Linux!libc.so.6}
\myindex{syscall}
Wenn wir uns die Funktion \TT{open()} in der Bibliothek \TT{libc.so.6}
anschauen, gibt es nur einen syscall:

\lstinputlisting[caption=open() (libc.so.6),style=customasmx86]{patterns/14_bitfields/1_check/tmp1.asm}

Die Bitfields für \TT{open()} werden also offenbar irgendwo im Linux Kernel
geprüft.

Natürlich ist es ein Leichtes sowohl GLibc als auch den Quellcode des Linux
Kernels herunterzuladen, aber wir wollen wir Sache ohne den Quellcode verstehen.

Wenn also in Linux 2.6 der syscall \TT{sys\_open} verwendet wird, wird die
Kontrolle an \TT{do\_sys\_open} übergeben und anschließend von dort aus---an die
Funktion \TT{do\_filp\_open()} (diese befindet sich im Verzeichnisbaum des
Kernel-Quellcodes in \TT{fs/namei.c}).

\newcommand{\URLREGPARM}{\href{http://www.ohse.de/uwe/articles/gcc-attributes.html\#func-regparm}{ohse.de/uwe/articles/gcc-attributes.html\#func-regparm}}

\myindex{fastcall}
\label{regparm}
Neben der Übergabe von Argumenten über den Stack gibt es auch die Möglichkeit
einige von ihnen direkt über Register zu übergeben. Dies wird auch
fastcall~(\myref{fastcall}) genannt. 
Es ist schneller, da die CPU nicht auf den Stack im Speicher zugreifen muss, um
die Funktionsargumente einzulesen.
GCC kennt dafür die Option \emph{regparm}\footnote{\URLREGPARM}, mithilfe derer es
möglich ist, die Anzahl der Argumente anzugeben, die über Register übergeben
werden sollen.

\newcommand{\URLKERNELNEWB}{\href{http://kernelnewbies.org/Linux_2_6_20\#head-042c62f290834eb1fe0a1942bbf5bb9a4accbc8f}{kernelnewbies.org/Linux\_2\_6\_20\#head-042c62f290834eb1fe0a1942bbf5bb9a4accbc8f}}
\newcommand{\CALLINGHFILE}{arch/x86/include/asm/calling.h}

Der Linux 2.6 Kernel wird mit der Option \TT{-mregparm=3}
kompiliert~\footnote{\URLKERNELNEWB} \footnote{Siehe auch die \TT{\CALLINGHFILE}
Datei im Kernel Verzeichnisbaum}.

Dies bedeutet, dass die erste 3 Argumente über die Register \EAX, \EDX und \ECX
übergeben werden und der Rest über den Stack. Ist die Anzahl der Argumente
kleiner als 3 wird natürlich nur ein Teil der genannten Register verwendet.

Laden wir also den Linux Kernel 2.6.31 herunter, kompilieren ihn in Ubuntu
(\TT{make vmlinux}) und öffnen ihn in \IDA, so finden wir die Funktion
\TT{do\_filp\_open()}. Am Beginn derselben finden wir den folgenden Code (die
Kommentare stammen vom Autor):

\lstinputlisting[caption=do\_filp\_open() (Linux Kernel 2.6.31),style=customasmx86]{patterns/14_bitfields/1_check/check2_DE.asm}
GCC speichert die Werte der ersten drei Argumente auf dem lokalen Stack.
Der Compiler würde diese Register ansonsten nicht verwenden und das wäre für den
\gls{register allocator} des Compilers nicht umsetzbar.

Finden wir dieses Codefragment:

\lstinputlisting[caption=do\_filp\_open() (Linux Kernel 2.6.31),style=customasmx86]{patterns/14_bitfields/1_check/check3.asm}

\TT{0x40}---enspricht dem \TT{O\_CREAT} Makro.
\TT{open\_flag} wird auf Anwesenheit des \TT{0x40} Bits hin überprüft und, wenn
das Bit 1 ist, wird der folgende \JNZ Befehl ausgelöst.
