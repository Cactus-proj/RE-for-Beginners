\mysection{Случай со структурой LARGE\_INTEGER}
\label{LargeInteger}

\myindex{Microsoft}
\myindex{LARGE\_INTEGER}
Представьте: конец 1980-х, вы Microsoft, и вы разрабатываете новую \emph{серьезную} \ac{OS} (Windows NT),
которая будет конкурировать с Юниксами.
Целевые платформы это и 32-битные и 64-битные процессоры.
\myindex{FILETIME}
И вам нужен 64-битный целочисленный тип для самых разных целей, начиная со структуры 
FILETIME\footnote{\url{https://docs.microsoft.com/en-us/windows/desktop/api/minwinbase/ns-minwinbase-filetime}}.

Проблема: пока еще не все компиляторы с Си/Си++ поддерживают 64-битные целочисленные (это конец 1980-х).
Конечно, это изменится в (ближайшем) будущем, но не сейчас.
Что вы будете делать?

Во время чтения, попробуйте остановиться (и/или закрыть эту книгу) и подумать, как вы можете решить эту проблему.

\clearpage

И вот что сделали в Microsoft, что-то вроде этого\footnote{Это не копипаста, я сам это написал}:

\begin{lstlisting}
union ULARGE_INTEGER
{
    struct backward_compatibility
    {
        DWORD LowPart;
        DWORD HighPart;
    };
#ifdef NEW_FANCY_COMPILER_SUPPORTING_64_BIT
    ULONGLONG QuadPart;
#endif
};
\end{lstlisting}

Это кусок из 8-и байт, к которому можно обратиться через 64-битное целочисленное QuadPart (если скомпилированно новым компилятором),
или используя два 32-битных целочисленных (если скомпилированно старым компилятором).

Поле QuadPart просто отсутствует здесь, если компилируется старым компилятором.

Порядок существенен: первое поле (LowPart) транслируется в младшие 4 байта 64-битного значения, второе поле (HighPart) ---
в старшие 4 байта.

В Microsoft также добавили ф-ции для арифметических операций, в той же манере, что уже было описано раннее:
\ref{sec:64bit_in_32_env}.

\myindex{Windows 2000}
Вот что можно найти в утекших исходных файлах Windows 2000:

\begin{lstlisting}[style=customasmx86,caption=Архитектура i386]
;++
;
; LARGE\_INTEGER
; RtlLargeIntegerAdd (
;    IN LARGE\_INTEGER Addend1,
;    IN LARGE\_INTEGER Addend2
;    )
;
; Routine Description:
;
;    This function adds a signed large integer to a signed large integer and
;    returns the signed large integer result.
;
; Arguments:
;
;    (TOS+4) = Addend1 - first addend value
;    (TOS+12) = Addend2 - second addend value
;
; Return Value:
;
;    The large integer result is stored in (edx:eax)
;
;--

cPublicProc _RtlLargeIntegerAdd ,4
cPublicFpo 4,0

        mov     eax,[esp]+4             ; (eax)=add1.low
        add     eax,[esp]+12            ; (eax)=sum.low
        mov     edx,[esp]+8             ; (edx)=add1.hi
        adc     edx,[esp]+16            ; (edx)=sum.hi
        stdRET    _RtlLargeIntegerAdd

stdENDP _RtlLargeIntegerAdd
\end{lstlisting}

\begin{lstlisting}[caption=Архитектура MIPS]
        LEAF_ENTRY(RtlLargeIntegerAdd)

        lw      t0,4 * 4(sp)            // get low part of addend2 value
        lw      t1,4 * 5(sp)            // get high part of addend2 value
        addu    t0,t0,a2                // add low parts of large integer
        addu    t1,t1,a3                // add high parts of large integer
        sltu    t2,t0,a2                // generate carry from low part
        addu    t1,t1,t2                // add carry to high part
        sw      t0,0(a0)                // store low part of result
        sw      t1,4(a0)                // store high part of result
        move    v0,a0                   // set function return register
        j       ra                      // return

        .end    RtlLargeIntegerAdd
\end{lstlisting}

Теперь две 64-битных архитектуры:

\myindex{Itanium}
\myindex{DEC Alpha}
\begin{lstlisting}[caption=Архитектура Itanium]
        LEAF_ENTRY(RtlLargeIntegerAdd)

        add         v0 = a0, a1                 // add both quadword arguments
        LEAF_RETURN

        LEAF_EXIT(RtlLargeIntegerAdd)
\end{lstlisting}

\begin{lstlisting}[caption=Архитектура DEC Alpha]
        LEAF_ENTRY(RtlLargeIntegerAdd)

        addq    a0, a1, v0              // add both quadword arguments
        ret     zero, (ra)              // return

        .end    RtlLargeIntegerAdd
\end{lstlisting}

В Itanium и DEC Alpha не нужно использовать 32-битные инструкии, потому что уже есть 64-битные.

\myindex{Windows Research Kernel}
И вот что можно найти в Windows Research Kernel:

\begin{lstlisting}[style=customc]
DECLSPEC_DEPRECATED_DDK         // Use native \_\_int64 math
__inline
LARGE_INTEGER
NTAPI
RtlLargeIntegerAdd (
    LARGE_INTEGER Addend1,
    LARGE_INTEGER Addend2
    )
{
    LARGE_INTEGER Sum;

    Sum.QuadPart = Addend1.QuadPart + Addend2.QuadPart;
    return Sum;
}
\end{lstlisting}

Все эти ф-ции могут быть убраны (в будущем), но сейчас они работают с полем QuadPart.
Если этот фрагмент кода будет скомпилирован современным 32-битным компилятором (поддерживающем 64-битные целочисленные),
он сгенерирует два 32-битных сложения.
С этого момента, поля LowPart/HighPart можно убрать из структуры  LARGE\_INTEGER.

Нужно ли использовать такую технику сегодня?
Вероятно нет, но если кому-то вдруг понадобится 128-битный тип целочисленных, вы можете его сделать точно так же.

Так же, нужно сказать, это работает благодаря порядку байт \emph{little-endian} (\ref{sec:endianness})
(все архитектуры под которые разрабатывалась Windows NT, \emph{little-endian}).
Этот трюк не сработает на \emph{big-endian}-архитектуре.

