\mysection{Metoda}

Kiedy autor tej książki uczył się C, a później \Cpp, pisał niewielkie kawałki kodu, kompilował
i patrzył jak wyglądają w asemblerze. Tak było o wiele łatwiej zrozumieć co się dzieje w programie.%
\footnote{Szczerze mówiąc, dalej tak robi, kiedy nie rozumie jak jakiś kod działa.
Ostatni przykład, z 2019 roku: \verb|p += p+(i&1)+2;| z ``SAT0W'', SAT-solvera autorstwa D. Knutha.}.
Robił to wystarczająco dużo razy, żeby związek między kodem w \CCpp a tym co generuje kompilator wbił się w jego podświadomość bardzo głęboko.
Po tym, patrząc na kod w asemblerze, od razu ogólnikowo rozumiał to co było napisane w C.
Możliwe, że ta metoda pomoże komuś jeszcze.

%Tu jest dużo przykładów do x86/x64 zarówno jak i do ARM.
%Ci, którzy są wystarczająco dobrze obeznani z jedną z architektur, mogą przekartkować te strony.

Czasami wykorzystam stare kompilatory, żeby otrzymać bardzo krótki lub prosty kawałek kodu.

À propos, jest świetna strona, gdzie możesz zrobić to samo, używając różnych kompilatorów - bez konieczności instalowania ich u siebie: \url{http://godbolt.org/}.

\section*{\Exercises}

Kiedy autor tej książki uczył się asemblera, często kompilował krótkie funkcje w C i przepisywał
je stopniowo na assemblera, starając się uzyskać jak najkrótszy kodu.
Prawdopodbnie nie warto tego robić w praktycznych zastosowaniach,
ponieważ trudno jest konkurować ze współczesnymi kompilatorami pod kątem wydajności. Jest to jednak bardzo dobry sposób na zrozumienie asemblera.
Możesz wziąć dowolny fragment kodu w asemblerze z tej książki i postarać się uczynić go krótszym.
Ale nie zapomnij przetestować swojego rezultatu.

\section*{Poziomy optymalizacji i debuggowanie}

Kod źródłowy można kompilować różnymi kompilatorami z różnym poziomami optymalizacji.
W typowym kompilatorze jest tych poziomów około trzech, gdzie poziom zerowy oznacza wyłączoną optymalizację.
Optymalizować można rozmiar programu lub jego szybkość.
Kompilator, który nie dokonuje optymalizacji, działa szybciej i generuje bardziej przejrzysty kod (choć i większy objętościowo).
Kompilator, który dokonuje optymalizacji, działa wolniej i stara się wygenerować jak najszybszy kod (co nie zawsze znaczy, że kod będzie krótszy).
Obok poziomów i kierunków optymalizacji kompilator może załączać do pliku wynikowego dodatkowe informacje dla debuggera,
tworząc w ten sposób kod, który jest prostszy w debuggowaniu.
Bardzo ważną cechą kodu debuggowanego jest to, że może on zawierać powiązanie
między każdą linią kodu źródłowego a adresem w kodzie maszynowym.
Kompilatory, dokonując optymalizacji, zwykle generują kod, gdzie całe linie kodu źródłowego mogą zostać pominięte nie będą nawet widoczne w kodzie maszynowym.
Praktykujący reverse engineer z reguły ma styczność z obiema wersjami, jako że niektórzy developerzy włączają optymalizację, a niektórzy - nie.

Dlatego będziemy pracować z przykładami kodu w obu wariantach.
