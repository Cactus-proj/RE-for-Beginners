\subsection*{mini-FAQ}

% TBT
%\par Q: Is this book simpler/easier than others?
%\par A: No, it is at about the same level as other books of this subject.

\par איזה ידע קודם נדרש לקריאת הספר?
\par רצויה הבנה בסיסית של C/C++.
%TBT
\par יש אפשרות לרכוש עותק פיזי של הספר?
\par לצערי לא, עד כה לא נמצאה הוצאה לאור שמעוניינת בהפצת הספר. בינתיים, ניתן לבקש להדפיס עותק מבית דפוס באזורך.

\par יש גירסת epub/mobi לספר?
\par הספר תלוי במידה רבה ב- TeX/LaTeX ולכן כל המרה ל-HTML (הפורמטים epub ו-mobi מבוססים על HTML) לא תהיה פשוטה.

\par יש סיבה ללמוד אסמבלי כיום?
\par כל עוד אינך עוסק/ת בפיתוח מערכות הפעלה, את/ה ככל הנראה לא תפתח/י באסמבלי. מהדרים חדשים, נניח למשל, החל משנת 2010, טובים בהרבה בביצוע אופטימזציות מאשר בני אדם1 {מאמר מצויין בנושא: \InSqBrackets{\AgnerFog}}.

יתרה מזאת, מעבדים חדשים הם רכיבים מאוד מסובכים וידיעת אסמבלי לא תסייע רבות בהבנת מבנם הפנימי.

עם זאת, צריך לציין כי ישנם לפחות שני תחומים בהם הבנת אסמבלי עשויה להיות מועילה:
בראש ובראשונה, מחקר פוגענים/אבטחה. בנוסף, זו דרך טובה לקבל הבנה טובה יותר של הקוד המהודר ובכך להקל על דיבאגינג.

הספר מיועד עבור אלו שרוצים להבין שפת אסמבלי ולא לפתח בה, לכן זו הסיבה שישנן דוגמאות רבות של פלטי מהדרים בתוכו.

\par לחצתי על קישור בתוך מסמך PDF, איך אני חוזר אחורה?
\par ב Adobe Acrobat Reader יש ללחוץ על Alt+חץ שמאלה. בתוכנת Evince יש ללחוץ על הכפתור ">".

\par האם אני רשאי/ת להדפיס/להשתמש בספר למטרות הוראה?
\par בוודאי! בדיוק לשם כך רשיון הספר הוא Creative Commons license (CC BY-SA 4.00).

\par למה הספר בחינם? עשית עבודה מעולה. זה חשוד בעיני, כמו דברים אחרים שמגיעים בחינם.
\par מחווייתי האישית, כותבים של ספרות טכנית עושים זאת לרוב מסיבות של פרסום-עצמי. זה בלתי אפשרי לקבל סכומים כסף הגונים מעבודה כזאת.

\par איך למצוא עבודה בהנדסה לאחור?
\par מפעם לפעם עולים פרסומים של משרות בתחום באתר reddit RE\FNURLREDDIT{} (\RedditHiringThread{}). נסה/י לחפש שם.

משרות שקשורות בדרך כלשהי לתחום ניתן למצוא גם בתת הפורום \q{netsec} של reddit: \NetsecHiringThread{}.

% TBT
%\par Q: Compilers' versions in the book are outdated already...
%\par A: No need to follow all steps precisely.
%Use the compilers you already have installed on your \ac{OS}.
%Also, there is: \href{https://godbolt.org/}{Compiler Explorer}.

\par יש לי שאלה...
\par ניתן לשלוח לי אימייל (\EMAIL).

