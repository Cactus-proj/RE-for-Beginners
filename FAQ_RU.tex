\subsection*{mini-ЧаВО}

\par Q: Эта книга проще/легче других?
\par A: Нет, примерно на таком же уровне, как и остальные книги посвященные этой теме.

\par Q: Мне страшно начинать читать эту книгу, здесь более 1000 страниц.
"... для начинающих" в названии звучит слегка саркастично.
\par A: Основная часть книги это масса разных листингов.
И эта книга действительно для начинающих, тут многого (пока) не хватает.

\par Q: Что необходимо знать перед чтением книги?
\par A: Желательно иметь базовое понимание Си/Си++.

\par Q: Должен ли я изучать сразу x86/x64/ARM и MIPS? Это не многовато?
\par A: Для начала, вы можете читать только о x86/x64, пропуская/пролистывая части о ARM/MIPS.

\par Q: Возможно ли купить русскую/английскую бумажную книгу?
\par A: К сожалению нет, пока ни один издатель не заинтересовался в издании русской или английской версии.
А пока вы можете распечатать/переплести её в вашем любимом копи-шопе/копи-центре.
\url{https://yurichev.com/news/20200222_printed_RE4B/}.

\par Q: Существует ли версия epub/mobi?
\par A: Книга очень сильно завязана на специфические для TeX/LaTeX хаки, поэтому преобразование в HTML (epub/mobi это набор HTML)
легким не будет.

\par Q: Зачем в наше время нужно изучать язык ассемблера?
\par A: Если вы не разработчик \ac{OS}, вам наверное не нужно писать на ассемблере: современные компиляторы (2010-ые) оптимизируют код намного лучше человека
\footnote{Очень хороший текст на эту тему: \InSqBrackets{\AgnerFog}}.

К тому же, современные \ac{CPU} это крайне сложные устройства и знание ассемблера вряд ли
поможет узнать их внутренности.

Но все-таки остается по крайней мере две области, где знание ассемблера может хорошо помочь:
1) исследование malware (\emph{зловредов}) с целью анализа; 2) лучшее понимание
вашего скомпилированного кода в процессе отладки.
Таким образом, эта книга предназначена для тех, кто хочет скорее понимать ассемблер,
нежели писать на нем, и вот почему здесь масса примеров, связанных с результатами
работы компиляторов.

\par Q: Я кликнул на ссылку внутри PDF-документа, как теперь вернуться назад?
\par A: В Adobe Acrobat Reader нажмите сочетание Alt+LeftArrow. В Evince кликните на ``<''.

\par Q: Могу ли я распечатать эту книгу? Использовать её для обучения?
\par A: Конечно, поэтому книга и лицензирована под лицензией Creative Commons (CC BY-SA 4.0).

\par Q: Почему эта книга бесплатная? Вы проделали большую работу. Это подозрительно, как и многие другие бесплатные вещи.
\par A: По моему опыту, авторы технической литературы делают это, в основном ради саморекламы.
Такой работой заработать приличные деньги невозможно.

\par Q: Как можно найти работу reverse engineer-а?
\par A: На reddit, посвященному RE\FNURLREDDIT, время от времени бывают hiring thread.
Посмотрите там.

В смежном субреддите \q{netsec} имеется похожий тред.

\par Q: Версии компиляторов в книге уже устарели давно...
\par A: Следовать всем шагам в точности не обязательно.
Пользуйтесь теми компиляторами, которые уже инсталлированы в вашу \ac{OS}.
Также, всегда есть: \href{https://godbolt.org/}{Compiler Explorer}.

\par Q: У меня есть вопрос...
\par A: Напишите мне его емейлом (\EMAILS).
