\mysection{Appels à assert()}
\myindex{\CStandardLibrary!assert()}

Parfois, la présence de la macro \TT{assert()} est aussi utile:
En général, cette macro laisse le nom du fichier source, le numéro de ligne et une
condition dans le code.

L'information la plus utile est contenue dans la condition d'assert, nous pouvons
en déduire les noms de variables ou les noms de champ de la structure. Les autres
informations utiles sont les noms de fichier---nous pouvons essayer d'en déduire
le type de code dont il s'agit ici.
Il est aussi possible de reconnaître les bibliothèques open-source connues d'après
les noms de fichier.

\lstinputlisting[caption=Exemple d'appels à assert() informatifs,style=customasmx86]{digging_into_code/assert_examples.lst}

Il est recommandé de \q{googler} à la fois les conditions et les noms de fichier,
qui peuvent nous conduire à une bibliothèque open-source.
Par exemple, si nous \q{googlons} \q{sp->lzw\_nbits <= BITS\_MAX}, cela va comme
prévu nous donner du code open-source relatif à la compression LZW.
