\mysection{Boucles}

À chaque fois que votre programme travaille avec des sortes de fichier, ou un buffer
d'une certaine taille, il doit s'agir d'un sorte de boucle de déchiffrement/traitement
à l'intérieur du code.

Ceci est un exemple réel de sortie de l'outil \tracer.
Il y avait un code qui chargeait une sorte de fichier chiffré de 258 octets.
Je l'ai lancé dans l'intention d'obtenir le nombre d'exécution de chaque instruction
(l'outil \ac{DBI} irait beaucoup mieux de nos jours).
Et j'ai rapidement trouvé un morceau de code qui était exécuté 259/258 fois:

\lstinputlisting{digging_into_code/crypto_loop.txt}

Il s'avère qu'il s'agit de la boucle de déchiffrement.

