\subsubsection{Base64}
\myindex{Base64}

L'encodage base64 est très répandu dans les cas où vous devez transférer des données
binaires sous forme de chaîne de texte.

Pour l'essentiel, cet algorithme encode 3 octets binaires en 4 caractères imprimables:
toutes les 26 lettres Latin (à la fois minuscule et majuscule), chiffres, signe plus
(\q{+}) et signe slash (\q{/}), 64 caractères en tout.

Une particularité des chaînes base64 est qu'elles se terminent souvent (mais pas
toujours) par 1 ou 2 symbole égal (\q{=}) pour l'alignement, par exemple:

\begin{lstlisting}
AVjbbVSVfcUMu1xvjaMgjNtueRwBbxnyJw8dpGnLW8ZW8aKG3v4Y0icuQT+qEJAp9lAOuWs=
\end{lstlisting}

\begin{lstlisting}
WVjbbVSVfcUMu1xvjaMgjNtueRwBbxnyJw8dpGnLW8ZW8aKG3v4Y0icuQT+qEJAp9lAOuQ==
\end{lstlisting}

Le signe égal (\q{=}) ne se rencontre jamais au milieu des chaînes encodées en base64.

maintenant, un exemple d'encodage manuel.
Encodons les octets hexadécimaux 0x00, 0x11, 0x22, 0x33 en une chaîne base64:

\lstinputlisting{digging_into_code/strings/base64_ex.sh}

Mettons ces 4 octets au forme binaire, puis regroupons les dans des groupes de 6-bit:

\begin{lstlisting}
|  00  ||  11  ||  22  ||  33  ||      ||      |
00000000000100010010001000110011????????????????
| A  || B  || E  || i  || M  || w  || =  || =  |
\end{lstlisting}

Les trois premiers octets (0x00, 0x11, 0x22) peuvent être encodés dans 4 caractères
base64 (``ABEi''), mais le dernier (0x33) --- ne le peut pas, donc il est encodé
en utilisant deux caractères (``Mw'') et de symbole (``='') de padding est ajouté
deux fois pour compléter le dernier groupe à 4 caractères.
De ce fait, la longueur de toutes les chaînes en base64 correctes est toujours divisible
par 4.

\myindex{XML}
\myindex{PGP}
Base64 est souvent utilisé lorsque des données binaires doivent être stockées dans
du XML.  Les clefs PGP ``Armored'' (i.e., au format texte) et les signatures sont
encodées en utilisant base64.

Certains essayent d'utiliser base64 pour masquer des chaînes:
\url{http://blog.sec-consult.com/2016/01/deliberately-hidden-backdoor-account-in.html}%
\footnote{\url{http://archive.is/nDCas}}.

\myindex{base64scanner}
Il existe des utilitaires pour rechercher des chaînes base64 dans des fichiers binaires
arbitraires.
L'un d'entre eux est base64scanner\footnote{\url{https://github.com/DennisYurichev/base64scanner}}.

\myindex{UseNet}
\myindex{FidoNet}
\myindex{Uuencoding}
\myindex{Phrack}
Un autre système d'encodage qui était très répandu sur UseNet et FidoNet est l'Uuencoding.
Les fichiers binaires sont toujours encodés au format Uuencode dans le magazine Phrack.
Il offre à peu près la même fonctionnalité, mais il est différent de base64 dans
le sens où le nom de fichier est aussi stocké dans l'entête.

\myindex{Tor}
\myindex{base32}
À propos: base64 à un petit frère: base32, alphabet qui a ~10 chiffres et ~26 caractères Latin.
Un usage répandu est les adresses onion%
\footnote{\url{https://trac.torproject.org/projects/tor/wiki/doc/HiddenServiceNames}},
comme: \\
\url{http://3g2upl4pq6kufc4m.onion/}.
\ac{URL} ne peut pas avoir de mélange de casse de caractères Latin, donc, c'est apparemment
pourquoi les développeurs de Tor ont utilisé base32.

