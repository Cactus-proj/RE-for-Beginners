% TODO move section...

\subsection{Quelques schémas de fichier binaire}

Tous les exemples ici ont été préparé sur Windows, avec la page de code 437 activée%
dans la console.
L'intérieur des fichiers binaires peut avoir l'air différent avec une autre page
de code.

\clearpage
\subsubsection{Tableaux}

Parfois, nous pouvons clairement localiser visuellement un tableau de valeurs 16/32/64-bit,
dans un éditeur hexadécimal.

Voici un exemple de tableau de valeurs 16-bit.
Nous voyons que le premier octet d'une paire est 7 ou 8, et que le second semble
aléatoire:

\begin{figure}[H]
\centering
\myincludegraphics{digging_into_code/binary/16bit_array.png}
\caption{FAR: tableau de valeurs 16-bit}
\end{figure}

J'ai utilisé un fichier contenant un signal 12-canaux numérisé en utilisant 16-bit \ac{ADC}.

\clearpage
\myindex{MIPS}
\par Et voici un exemple de code MIPS très typique.

Comme nous pouvons nous en souvenir, chaque instruction MIPS (et aussi ARM en mode
ARM ou ARM64) a une taille de 32 bits (ou 4 octets), donc un tel code est un tableau
de valeurs 32-bit.

En regardant cette copie d'écran, nous voyons des sortes de schémas.

Les lignes rouge verticales ont été ajoutées pour la clarté:

\begin{figure}[H]
\centering
\myincludegraphics{digging_into_code/binary/typical_MIPS_code.png}
\caption{Hiew: code MIPS très typique}
\end{figure}

Il y a un autre exemple de tel schéma ici dans le livre:
\myref{Oracle_SYM_files_example}.

\clearpage
\subsubsection{Fichiers clairsemés}

Ceci est un fichier clairsemé avec des données éparpillées dans un fichier presque vide.
Chaque caractère espace est en fait l'octet zéro (qui rend comme un espace).
Ceci est un fichier pour programmer des FPGA (Altera Stratix GX device).
Bien sûr, de tels fichiers peuvent être compressés facilement, mais des formats comme
celui-ci sont très populaire dans les logiciels scientifiques et d'ingénierie, où
l'efficience des accès est importante, tandis que la compacité ne l'est pas.

\begin{figure}[H]
\centering
\myincludegraphics{digging_into_code/binary/sparse_FPGA.png}
\caption{FAR: Fichier clairsemé}
\end{figure}

\clearpage
\subsubsection{Fichiers compressés}

% FIXME \ref{} ->
Ce fichier est juste une archive compressée.
Il a une entropie relativement haute et visuellement, il à l'air chaotique.
Ceci est ce à quoi ressemble les fichiers compressés et/ou chiffrés.

\begin{figure}[H]
\centering
\myincludegraphics{digging_into_code/binary/compressed.png}
\caption{FAR: Fichier compressé}
\end{figure}

\clearpage
\subsubsection{\ac{CDFS}}

Les fichiers d'installation d'un \ac{OS} sont en général distribués sous forme de
fichiers ISO, qui sont des copies de disques CD/DVD.
Le système de fichiers utilisé est appelé \ac{CDFS}, ce que vous voyez ici sont des
noms de fichiers mixés avec des données additionnelles.
Ceci peut-être la taille des fichiers, des pointeurs sur d'autres répertoires, des
attributs de fichier, etc.
C'est l'aspect typique de ce à quoi ressemble un système de fichiers en interne.

\begin{figure}[H]
\centering
\myincludegraphics{digging_into_code/binary/cdfs.png}
\caption{FAR: Fichier ISO: \ac{CD} d'installation d'Ubuntu 15}
\end{figure}

\clearpage
\subsubsection{Code exécutable x86 32-bit}

Voici l'allure de code exécutable x86 32-bit.
Il n'a pas une grande entropie, car certains octets reviennent plus souvent que d'autres.

\begin{figure}[H]
\centering
\myincludegraphics{digging_into_code/binary/x86_32.png}
\caption{FAR: Code exécutable x86 32-bit}
\end{figure}

% TODO: Read more about x86 statistics: \ref{}. % FIXME blog post about decryption...

\clearpage
\subsubsection{Fichiers graphique BMP}

% TODO: bitmap, bit, group of bits...

Les fichiers BMP ne sont pas compressés, donc chaque octet (ou groupe d'octet) représente
chaque pixel.
J'ai trouvé cette image quelque part dans mon installation de Windows 8.1:

\begin{figure}[H]
\centering
\myincludegraphicsSmall{digging_into_code/binary/bmp.png}
\caption{Image exemple}
\end{figure}

Vous voyez que cette image a des pixels qui ne doivent pas pouvoir être compressés
beaucoup (autour du centre), mais il y a de longues lignes monochromes au haut et
en bas.
En effet, de telles lignes ressemblent à des lignes lorsque l'on regarde le fichier:

\begin{figure}[H]
\centering
\myincludegraphics{digging_into_code/binary/bmp_FAR.png}
\caption{Fragment de fichier BMP}
\end{figure}

