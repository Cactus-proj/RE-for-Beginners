% TODO move section...

\subsection{Некоторые паттерны в бинарных файлах}

Все примеры здесь были подготовлены в Windows с активной кодовой страницей 437
в консоли.
Двоичные файлы внутри могут визуально выглядеть иначе если установлена другая кодовая страница.

\clearpage
\subsubsection{Массивы}

Иногда мы можем легко заметить массив 16/32/64-битных значений визуально, в шестнадцатеричном 
редакторе.

Вот пример массива 16-битных значений.
Мы видим, что каждый первый байт в паре всегда равен 7 или 8, а второй выглядит случайным:

\begin{figure}[H]
\centering
\myincludegraphics{digging_into_code/binary/16bit_array.png}
\caption{FAR: массив 16-битных значений}
\end{figure}

Для примера я использовал файл содержащий 12-канальный сигнал оцифрованный при помощи 16-битного \ac{ADC}.

\clearpage
\myindex{MIPS}
\par А вот пример очень типичного MIPS-кода.

Как мы наверное помним, каждая инструкция в MIPS (а также в ARM в режиме ARM, или ARM64) имеет 
длину 32 бита (или 4 байта),
так что такой код это массив 32-битных значений.

Глядя на этот скриншот, можно увидеть некий узор.
Вертикальные красные линии добавлены для ясности:

\begin{figure}[H]
\centering
\myincludegraphics{digging_into_code/binary/typical_MIPS_code.png}
\caption{Hiew: очень типичный код для MIPS}
\end{figure}

Еще пример таких файлов в этой книге: 
\myref{Oracle_SYM_files_example}.

\clearpage
\subsubsection{Разреженные файлы}

Это разреженный файл, в котором данные разбросаны посреди почти пустого файла.
Каждый символ пробела здесь на самом деле нулевой байт (который выглядит как пробел).
Это файл для программирования FPGA (чип Altera Stratix GX).
Конечно, такие файлы легко сжимаются, но подобные форматы очень популярны в научном и инженерном ПО, где быстрый доступ важен, а компактность --- не очень.

\begin{figure}[H]
\centering
\myincludegraphics{digging_into_code/binary/sparse_FPGA.png}
\caption{FAR: Разреженный файл}
\end{figure}

\clearpage
\subsubsection{Сжатый файл}

% FIXME \ref{} ->
Этот файл это просто некий сжатый архив.
Он имеет довольно высокую энтропию и визуально выглядит просто хаотичным.
Так выглядят сжатые и/или зашифрованные файлы.

\begin{figure}[H]
\centering
\myincludegraphics{digging_into_code/binary/compressed.png}
\caption{FAR: Сжатый файл}
\end{figure}

\clearpage
\subsubsection{\ac{CDFS}}

Инсталляции \ac{OS} обычно распространяются в ISO-файлах, которые суть копии CD/DVD-дисков.
Используемая файловая система называется \ac{CDFS}, здесь видны имена файлов и какие-то допольнительные данные.
Это могут быть длины файлов, указатели на другие директории, атрибуты файлов, итд.
Так может выглядеть типичная файловая система внутри.

\begin{figure}[H]
\centering
\myincludegraphics{digging_into_code/binary/cdfs.png}
\caption{FAR: ISO-файл: инсталляционный \ac{CD} Ubuntu 15}
\end{figure}

\clearpage
\subsubsection{32-битный x86 исполняемый код}

Так выглядит 32-битный x86 исполняемый код.
У него не очень высокая энтропия, потому что некоторые байты встречаются чаще других.

\begin{figure}[H]
\centering
\myincludegraphics{digging_into_code/binary/x86_32.png}
\caption{FAR: Исполняемый 32-битных x86 код}
\end{figure}

% TODO: Read more about x86 statistics: \ref{}. % FIXME blog post about decryption...

\clearpage
\subsubsection{Графические BMP-файлы}

% TODO: bitmap, bit, group of bits...

BMP-файлы не сжаты, так что каждый байт (или группа байт) описывают каждый пиксель.
Я нашел эту картинку где-то внутри заинсталлированной Windows 8.1:

\begin{figure}[H]
\centering
\myincludegraphicsSmall{digging_into_code/binary/bmp.png}
\caption{Пример картинки}
\end{figure}

Вы видите, что эта картинка имеет пиксели, которые вряд ли могут быть хорошо сжаты (в районе центра),
но здесь есть длинные одноцветные линии вверху и внизу.
Действительно, линии вроде этих выглядят как линии при просмотре этого файла:

\begin{figure}[H]
\centering
\myincludegraphics{digging_into_code/binary/bmp_FAR.png}
\caption{Фрагмент BMP-файла}
\end{figure}

