% FIXME comparison!
\subsection{Memory \q{snapshots} comparing}
\label{snapshots_comparing}

Die Technik zwei Memory Snapshots zu vergleichen ist recht einfach, das hat man auch oft benutzt um 8-Bit Computerspiele und
\q{high score}'s  zu hacken.

Zum Beispiel, wenn man ein geladenes Spiel auf einem 8-Bit Computer hat ( auf den Maschinen ist nicht viel Speicher 
vorhanden, jedoch braucht das Spiel noch weniger Speicher) und du weißt was du im Spiel hast, sagen wir 100 Patronen, 
nun kann man einen \q{snapshot} vom gesamten Speicher machen und diesen Irgendwohin speichern. Dann verschiesst man 
eine Patrone, dann geht der Patronen Z\"ahler auf 99, nun erstellt man den zweiten Snapshot und Vergleich die beiden: 
Nun muss es irgendwo ein Byte geben das vorher 100 war und jetzt 99 ist. 

Betrachtet man den Fakt das diese 8-Bit Spiele oftmals in Assembler geschrieben wurden und diese Variablen meist global 
waren, konnte man ziemlich einfach bestimmen welche Adressen im Speicher den Kugelz\"ahler beinhalten. Wenn man nach allen 
Referenzen der Adresse im dissassembelten Spiel code sucht, ist es nicht schwer den Code \glslink{decrement}{decrementing} 
zu finden und dann eine \gls{NOP} Instruktion an diese Stelle zu schreiben, oder gar mehrere \gls{NOP}-s, und dann hat man 
ein Spiel bei dem man f\"ur immer 100 Kugeln hat. %<-- das kacke der ganze block
\myindex{BASIC!POKE}
Spiele auf 8-Bit Computern wurden allgemein an konstanten Adressen geladen, zus\"atzlich gab es nicht viele unterschiedliche
Versionen des Spiels (  Es war meist eine Version f\"ur lange Zeit popul\"ar ), dadurch wussten enthusiastische Gamer welche
Bytes (durch das benutzen von Basic Instruktionen wie \gls{POKE}) \"uberschrieben werden mussten um das Spiel zu hacken. 
Das hat wiederum zu \q{cheat} listen gef\"uhrt die in Magazinen f\"ur 8-Bit Games erschienen, die dann \gls{POKE} Instruktionen enthielten.

% Considering the fact that these 8-bit games were often written in assembly language and such variables were global,
% it can be said for sure which address in memory has holding the bullet count. If you searched for all references to the
% address in the disassembled game code, it was not very hard to find a piece of code \glslink{decrement}{decrementing} the bullet count,
% then to write a \gls{NOP} instruction there, or a couple of \gls{NOP}-s, 
% and then have a game with 100 bullets forever.
% \myindex{BASIC!POKE}
% Games on these 8-bit computers were commonly loaded at the constant
% address, also, there were not much different versions of each game (commonly just one version was popular for a long span of time),
% so enthusiastic gamers knew which bytes must be overwritten (using the BASIC's instruction \gls{POKE}) at which address in
% order to hack it. This led to \q{cheat} lists that contained \gls{POKE} instructions, published in magazines related to
% 8-bit games.

\myindex{MS-DOS}

Es ist auch einfach \q{high score} Dateien zu modifizieren, das funktioniert nicht nur bei 8-Bit Spielen. Man achte 
auf seinen Highscore Z\"ahler, dann macht man ein Backup der Datei. Wenn sich der \q{high score} Z\"ahler \"andert, vergleicht man die 
zwei Dateien miteinander, das kann man sogar mit dem DOS Tool FC\footnote{MS-DOS Utility zum vergleichen von  Dateien} (\q{high score} Dateien,
sind oft in Bin\"arer Form). 

Es wird beim Vergleichen der Dateien einen Punkt geben wo einige Bytes sich unterscheiden und 
es wird leicht sein, die Punkte zu sehen die die Bytes des Punktez\"ahler beinhalten. 
Jedoch sind sich die Spiele Entwickler solcher Tricks bewusst und bauen Wege ein um das Programm
vor solchen Manipulationen zu sch\"utzen. 

Ein \"ahnliches Beispiel findet man auch in dem Buch \myref{Millenium_DOS_game}.

% TODO: пример с какой-то простой игрушкой?

% TBT 

\subsubsection{Windows registry}

Es ist auch m\"oglich die Windows Regestry zu vergleichen vor und nach der Programm Installation.

Es ist eine sehr popul\"are Methode Regestry Elemente zu finden die vom Programm benutzt werden.
Vielleicht ist das auch der Grund warum die \q{windows regestry cleaner} Shareware so popul\"ar ist.

% TBT

\subsubsection{Blink-comparator}

Der Vergleich von Datei- oder Speichersnapshots erinnert ein wenig an einen Blinkkomparator
\footnote{\url{http://go.yurichev.com/17348}}
ein Ger\"at das in der Vergangenheit von Astronomen benutzt wurde, um sich bewegende Astronomische
Objekte zu finden.

Ein Blinkkomperator erlaubt es schnell zwischen Photographie zu wechseln die zu unterschiedlicher
Zeit aufgenommen wurden, so kann ein Astronom Unterschiede zwischen Fotografien visuell erkennen.

Ach \"ubrigens, Pluto wurde durch einen solchen Blink-Komparator 1930 entdeckt.
