% FIXME comparison!
\subsection{Comparer des \q{snapshots} mémoire}
\label{snapshots_comparing}

La technique consistant à comparer directement deux états mémoire afin de voir les
changements était souvent utilisée pour tricher avec les jeux sur ordinateurs 8-bit
et pour modifier le fichiers des \q{meilleurs scores}.

Par exemple, si vous avez chargé un jeu sur un ordinateur 8-bit (il n'y a pas beaucoup
de mémoire dedans, mais le jeu utilise en général encore moins de mémoire), et que
vous savez que vous avez maintenant, disons, 100 balles, vous pouvez faire un \q{snapshot}
de toute la mémoire et le sauver quelque part. Puis, vous tirez une fois, le compteur
de balles descend à 99, faites un second \q{snapshot} et puis comparer les deux:
il doit y avoir quelque part un octet qui était à 100 au début, et qui est maintenant
à 99.

En considérant le fait que ces jeux 8-bit étaient souvent écrits en langage d'assemblage
et que de telles variables étaient globales, on peut déterminer avec certitude quelle
adresse en mémoire contenait le compteur de balles. Si vous cherchiez toutes les références
à cette adresse dans le code du jeu désassemblé, il n'était pas très difficile de
trouver un morceau de code \glslink{decrement}{décrémentant} le compteur de balles,
puis d'y écrire une, ou plusieurs, instruction \gls{NOP}, et d'avoir un jeu avec
toujours 100 balles.
\myindex{BASIC!POKE}
Les jeux sur ces ordinateurs 8-bit étaient en général chargés à une adresse constante,
aussi, il n'y avait pas beaucoup de versions ce chaque jeu (souvent, une seule version
était répandue pour un long moment), donc les joueurs enthousiastes savaient à quelles
adresses se trouvaient les octets qui devaient être modifiés (en utilisant l'instruction
BASIC \gls{POKE}) pour le bidouiller. Ceci à conduit à des listes de \q{cheat} qui
contenaient les instructions \gls{POKE} publiées dans des magazines relatifs aux
jeux 8-bit. Voir aussi: \href{http://go.yurichev.com/17114}{Wikipédia}.

\myindex{MS-DOS}

De même, il est facile de modifier le fichier des \q{meilleurs scores}, ceci ne fonctionne
pas seulement avec des jeux 8-bit. Notez votre score et sauvez le fichier quelque part.
Lorsque le décompte des \q{meilleurs scores} devient différent, comparez juste les
deux fichiers, ça peut même être fait avec l'utilitaire DOS FC\footnote{Utilitaire
MS-DOS pour comparer des fichiers binaires.} (les fichiers des \q{meilleurs scores}
sont souvent au format binaire).

Il y aura un endroit où quelques octets seront différents et il est facile de voir
lesquels contiennent le score.
Toutefois, les développeurs de jeux étaient conscient de ces trucs et pouvaient protéger
le programme contre ça.

Exemple quelque peu similaire dans ce livre: \myref{Millenium_DOS_game}.

% TODO: пример с какой-то простой игрушкой?

\subsubsection{Une histoire vraie de 1999}

\myindex{ICQ}
C'était un temps de l'engouement pour la messagerie ICQ, au moins dans les pays de
l'ex-URSS.
Cette messagerie avait une particularité --- certains utilisateurs ne voulaient pas
partager leur état en ligne avec tout le monde.
Et vous deviez demander une \emph{autorisation} à cet utilisateur.
Il pouvait vous autoriser à voir son état, ou pas.

Voici ce que j'ai fait:

\begin{itemize}
\item Ajouté un utilisateur.
\item Un utiliseur est apparu dans la liste de contact, dans la section ``attente d'autorisation''.
\item Fermé ICQ.
\item Sauvegardé la base de données ICQ.
\item Ouvert à nouveau ICQ.
\item L'utilisateur m'a \emph{autorisé}.
\item Refermé ICQ et comparé les deux base de données.
\end{itemize}

Il s'est avéré que: les deux bases de données ne différaient que d'un octet.
Dans la première version: \verb|RESU\x03|, dans la seconde: \verb|RESU\x02|.
(``RESU'', signifie probablement ``USER'', i.e., un entête d'une structure où toutes
les informations à propos d'un utilisateur étaient stockées.)
Cela signifie que l'information sur l'autorisation n'était pas stockée sur le serveur,
mais sur le client. Vraisemblablement, la valeur 2/3 reflétait l'état de l'\q{autorisation}.

\subsubsection{Registres de Windows}

Il est aussi possible de comparer les registres de Windows avant et après l'installation
d'un programme.

C'est une méthode courante  que de trouver quels sont les éléments des registres
utilisés par le programme. Peut-être que ceci est la raison pour laquelle le shareware
de \q{nettoyage des registres windows} est si apprécié.

À propos, voici comment sauver les registres de Windows dans des fichiers texte:

\begin{lstlisting}
reg export HKLM HKLM.reg
reg export HKCU HKCU.reg
reg export HKCR HKCR.reg
reg export HKU HKU.reg
reg export HKCC HKCC.reg
\end{lstlisting}

\myindex{UNIX!diff}
Ils peuvent être comparés en utilisant diff...

\subsubsection{Comparateur à clignotement}

La comparaison de fichiers ou d'images mémoire nous rappelle le comparateur à clignotement
\footnote{\url{http://go.yurichev.com/17348}}:
Un dispositif utilisé autrefois par les astronomes pour trouver les objets célestes
changeant de position.

Les comparateurs à clignotement permettent d'alterner rapidement entre deux photographies
prisent à des moments différents, de façon à faire apparaître les différences visuellement.

À propos, Pluton a été découverte avec un comparateur à clignotement en 1930.
