\mysection{Строки}
\label{sec:digging_strings}

\subsection{Текстовые строки}

\subsubsection{\CCpp}

\label{C_strings}
Обычные строки в Си заканчиваются нулем (\ac{ASCIIZ}-строки).

Причина, почему формат строки в Си именно такой (оканчивающийся нулем) вероятно историческая.
В [Dennis M. Ritchie, \emph{The Evolution of the Unix Time-sharing System}, (1979)]
мы можем прочитать:

\begin{framed}
\begin{quotation}
A minor difference was that the unit of I/O was the word, not the byte, because the PDP-7 was a word-addressed
machine. In practice this meant merely that all programs dealing with character streams ignored null
characters, because null was used to pad a file to an even number of characters.
\end{quotation}
\end{framed}

\myindex{Hiew}
Строки выглядят в Hiew или FAR Manager точно так же:

\begin{lstlisting}[style=customc]
int main()
{
	printf ("Hello, world!\n");
};
\end{lstlisting}

\begin{figure}[H]
\centering
\includegraphics[width=0.6\textwidth]{digging_into_code/strings/C-string.png}
\caption{Hiew}
\end{figure}

% FIXME видно \n в конце, потом пробел

\subsubsection{Borland Delphi}
\myindex{Pascal}
\myindex{Borland Delphi}
Когда кодируются строки в Pascal и Delphi, сама строка предваряется 8-битным или 32-битным значением, в котором закодирована длина строки.

Например:

\begin{lstlisting}[caption=Delphi,style=customasmx86]
CODE:00518AC8                 dd 19h
CODE:00518ACC aLoading___Plea db 'Loading... , please wait.',0

...

CODE:00518AFC                 dd 10h
CODE:00518B00 aPreparingRun__ db 'Preparing run...',0
\end{lstlisting}

\subsubsection{Unicode}

\myindex{Unicode}
Нередко уникодом называют все способы передачи символов, когда символ занимает 2 байта или 16 бит.
Это распространенная терминологическая ошибка.
Уникод --- это стандарт, присваивающий номер каждому символу многих письменностей мира, но не описывающий
способ кодирования.

\myindex{UTF-8}
\myindex{UTF-16LE}
Наиболее популярные способы кодирования: 
UTF-8 (наиболее часто используется в Интернете и *NIX-системах) и UTF-16LE (используется в Windows).

\myparagraph{UTF-8}

\myindex{UTF-8}
UTF-8 это один из очень удачных способов кодирования символов.
Все символы латиницы кодируются так же, как и в ASCII-кодировке, а символы, выходящие за пределы
ASCII-7-таблицы, кодируются несколькими байтами.
0 кодируется, как и прежде, нулевым байтом, так что все стандартные
функции Си продолжают работать с UTF-8-строками так же как и с обычными строками.

Посмотрим, как символы из разных языков кодируются в UTF-8 и как это выглядит в FAR, в кодировке 437

\footnote{Пример и переводы на разные языки были взяты здесь: 
\url{http://www.columbia.edu/~fdc/utf8/}}:

\begin{figure}[H]
\centering
\includegraphics[width=0.6\textwidth]{digging_into_code/strings/multilang_sampler.png}
\end{figure}

% FIXME: cut it
\begin{figure}[H]
\centering
\myincludegraphics{digging_into_code/strings/multilang_sampler_UTF8.png}
\caption{FAR: UTF-8}
\end{figure}

Видно, что строка на английском языке выглядит точно так же, как и в ASCII-кодировке.
В венгерском языке используются латиница плюс латинские буквы с диакритическими знаками.
Здесь видно, что эти буквы кодируются несколькими байтами, они подчеркнуты красным.
То же самое с исландским и польским языками.
В самом начале имеется также символ валюты \q{Евро}, который кодируется тремя байтами.
Остальные системы письма здесь никак не связаны с латиницей.
По крайней мере о русском, арабском, иврите и хинди мы можем сказать, что здесь видны повторяющиеся
байты, что не удивительно, ведь, обычно буквы из одной и той же системы письменности расположены в одной
или нескольких таблицах уникода, поэтому часто их коды начинаются с одних и тех же цифр.

В самом начале, перед строкой \q{How much?}, видны три байта, которые на самом деле \ac{BOM}.
\ac{BOM} показывает, какой способ кодирования будет сейчас использоваться.

\myparagraph{UTF-16LE}

\myindex{UTF-16LE}
\myindex{Windows!Win32}
Многие функции win32 в Windows имеют суффикс \TT{-A} и \TT{-W}.
Первые функции работают с обычными строками, вторые с UTF-16LE-строками (\emph{wide}).
Во втором случае, каждый символ обычно хранится в 16-битной переменной типа \emph{short}.

Cтроки с латинскими буквами выглядят в Hiew или FAR как перемежающиеся с нулевыми байтами:

\begin{lstlisting}[style=customc]
int wmain()
{
	wprintf (L"Hello, world!\n");
};
\end{lstlisting}

\begin{figure}[H]
\centering
\includegraphics[width=0.6\textwidth]{digging_into_code/strings/UTF16-string.png}
\caption{Hiew}
\end{figure}

Подобное можно часто увидеть в системных файлах \gls{Windows NT}:

\begin{figure}[H]
\centering
\includegraphics[width=0.6\textwidth]{digging_into_code/strings/ntoskrnl_UTF16.png}
\caption{Hiew}
\end{figure}

\myindex{IDA}
В \IDA, уникодом называется именно строки с символами, занимающими 2 байта:

\begin{lstlisting}[style=customasmx86]
.data:0040E000 aHelloWorld:
.data:0040E000                 unicode 0, <Hello, world!>
.data:0040E000                 dw 0Ah, 0
\end{lstlisting}

Вот как может выглядеть строка на русском языке (\q{И снова здравствуйте!}) закодированная в UTF-16LE:

\begin{figure}[H]
\centering
\includegraphics[width=0.6\textwidth]{digging_into_code/strings/russian_UTF16.png}
\caption{Hiew: UTF-16LE}
\end{figure}

То что бросается в глаза --- это то что символы перемежаются ромбиками (который имеет код 4).
Действительно, в уникоде кирилличные символы находятся в четвертом блоке.
Таким образом, все кирилличные символы в UTF-16LE находятся в диапазоне \TT{0x400-0x4FF}.

Вернемся к примеру, где одна и та же строка написана разными языками.
Здесь посмотрим в кодировке UTF-16LE.

% FIXME: cut it
\begin{figure}[H]
\centering
\myincludegraphics{digging_into_code/strings/multilang_sampler_UTF16.png}
\caption{FAR: UTF-16LE}
\end{figure}

Здесь мы также видим \ac{BOM} в самом начале.
Все латинские буквы перемежаются с нулевыми байтами.
Некоторые буквы с диакритическими знаками (венгерский и исландский языки) также подчеркнуты красным.

% subsection:
\input{digging_into_code/strings/base64_RU}



\subsection{Поиск строк в бинарном файле}

\epigraph{Actually, the best form of Unix documentation is frequently running the
\textbf{strings} command over a program’s object code. Using \textbf{strings}, you can get
a complete list of the program’s hard-coded file name, environment variables,
undocumented options, obscure error messages, and so forth.}{The Unix-Haters Handbook}

\myindex{UNIX!strings}
Стандартная утилита в UNIX \emph{strings} это самый простой способ увидеть строки в файле.
Например, это строки найденные в исполняемом файле sshd из OpenSSH 7.2:

\lstinputlisting{digging_into_code/sshd_strings.txt}

Тут опции, сообщения об ошибках, пути к файлам, импортируемые модули, функции, и еще какие-то странные строки (ключи?)
Присутствует также нечитаемый шум---иногда в x86-коде бывают целые куски состоящие из печатаемых ASCII-символов,
вплоть до ~8 символов.

Конечно, OpenSSH это опенсорсная программа.
Но изучение читаемых строк внутри некоторого неизвестного бинарного файла это зачастую самый первый шаг в анализе.
\myindex{UNIX!grep}

Также можно использовать \emph{grep}.

\myindex{Hiew}
\myindex{Sysinternals}
В Hiew есть такая же возможность (Alt-F6), также как и в Sysinternals ProcessMonitor.

\subsection{Сообщения об ошибках и отладочные сообщения}

Очень сильно помогают отладочные сообщения, если они имеются. В некотором смысле, отладочные сообщения, 
это отчет о том, что сейчас происходит в программе.
Зачастую, это \printf-подобные функции, 
которые пишут куда-нибудь в лог, а бывает так что и не пишут ничего, но вызовы остались, так как эта сборка --- не
отладочная, а \emph{release}.

\myindex{\oracle}
Если в отладочных сообщениях дампятся значения некоторых локальных или глобальных переменных, 
это тоже может помочь, как минимум, узнать их имена. 
Например, в \oracle одна из таких функций: \TT{ksdwrt()}.

Осмысленные текстовые строки вообще очень сильно могут помочь. 
Дизассемблер \IDA может сразу указать, из какой функции и из какого её места используется эта строка. 
Встречаются и смешные случаи
\footnote{\href{http://blog.yurichev.com/node/32}{blog.yurichev.com}}.

Сообщения об ошибках также могут помочь найти то что нужно. 
В \oracle сигнализация об ошибках проходит при помощи вызова некоторой группы функций. \\
Тут еще немного об этом: \href{http://blog.yurichev.com/node/43}{blog.yurichev.com}.

\myindex{Error messages}
Можно довольно быстро найти, какие функции сообщают о каких ошибках, и при каких условиях.

Это, кстати, одна из причин, почему в защите софта от копирования, 
бывает так, что сообщение об ошибке заменяется 
невнятным кодом или номером ошибки. Мало кому приятно, если взломщик быстро поймет, 
из-за чего именно срабатывает защита от копирования, просто по сообщению об ошибке.

Один из примеров шифрования сообщений об ошибке, здесь: \myref{examples_SCO}.

\subsection{Подозрительные магические строки}

Некоторые магические строки, используемые в бэкдорах выглядят очень подозрительно.
Например, в домашних роутерах TP-Link WR740 был бэкдор
\footnote{\url{http://sekurak.pl/tp-link-httptftp-backdoor/}, на русском: \url{http://m.habrahabr.ru/post/172799/}}.
Бэкдор активировался при посещении следующего URL:\\
\url{http://192.168.0.1/userRpmNatDebugRpm26525557/start_art.html}.\\
Действительно, строка \q{userRpmNatDebugRpm26525557} присутствует в прошивке.

Эту строку нельзя было нагуглить до распространения информации о бэкдоре.

Вы не найдете ничего такого ни в одном \ac{RFC}.

Вы не найдете ни одного алгоритма, который бы использовал такие странные последовательности байт.

И это не выглядит как сообщение об ошибке, или отладочное сообщение.

Так что проверить использование подобных странных строк --- это всегда хорошая идея.
\\
\myindex{base64}
Иногда такие строки кодируются при помощи 
base64\footnote{Например, бэкдор в кабельном модеме Arris: 
\url{http://www.securitylab.ru/analytics/461497.php}}.
Так что неплохая идея их всех декодировать и затем просмотреть глазами, пусть даже бегло.
\\
\myindex{Security through obscurity}

Более точно, такой метод сокрытия бэкдоров называется \q{security through obscurity} (безопасность через
запутанность).
