\chapter{Trouver des choses importantes/intéressantes dans le code}

Le minimalisme n'est pas une caractéristique prépondérante des logiciels modernes.

\myindex{\Cpp!STL}

Pas parce que les programmeurs écrivent beaucoup, mais parce que de nombreuses bibliothèques
sont couramment liées statiquement aux fichiers exécutable.
Si toutes les bibliothèques externes étaient déplacées dans des fichiers DLL externes,
le monde serait différent. (Une autre raison pour C++ sont la \ac{STL} et autres
bibliothèques templates.)

\newcommand{\FOOTNOTEBOOST}{\footnote{\url{http://go.yurichev.com/17036}}}
\newcommand{\FOOTNOTELIBPNG}{\footnote{\url{http://go.yurichev.com/17037}}}

Ainsi, il est très important de déterminer l'origine de la fonction, si elle provient
d'une bibliothèque standard ou d'une bibliothèque bien connue (comme Boost\FOOTNOTEBOOST,
libpng\FOOTNOTELIBPNG), ou si elle est liée à ce que l'on essaye de trouver dans
le code.

Il est simplement absurde de tout récrire le code en \CCpp pour trouver ce que l'on
cherche.

Une des premières tâches d'un rétro-ingénieur est de trouver rapidement le code dont
il a besoin.

\myindex{\GrepUsage}

Le dés-assembleur \IDA nous permet de chercher parmi les chaînes de texte, les séquences
d'octets et les constantes.
Il est même possible d'exporter le code dans un fichier texte .lst ou .asm et d'utiliser
\TT{grep}, \TT{awk}, etc.

Lorsque vous essayez de comprendre ce que fait un certain code, ceci peut être facile
avec une bibliothèque open-source comme libpng.
Donc, lorsque vous voyez certaines constantes ou chaînes de texte qui vous semblent
familières, il vaut toujours la peine de les \emph{googler}.
Et si vous trouvez le projet open-source où elles sont utilisées, alors il suffit
de comparer les fonctions.
Ceci peut permettre de résoudre certaines parties du problème.

Par exemple, si un programme utilise des fichiers XML, la premières étape peut-être
de déterminer quelle bibliothèque XML est utilisée pour le traitement, puisque les
bibliothèques standards (ou bien connues) sont en général utilisées au lieu de code
fait maison.

\myindex{SAP}
\myindex{Windows!PDB}

Par exemple, j'ai essayé une fois de comprendre comment la compression/décompression
des paquets réseau fonctionne dans SAP 6.0.
C'est un logiciel gigantesque, mais un .\gls{PDB} détaillé avec des informations
de débogage est présent, et c'est pratique.
J'en suis finalement arrivé à l'idée que l'une des fonctions, qui était appelée par
\emph{CsDecomprLZC}, effectuait la décompression des paquets réseau.
Immédiatement, j'ai essayé de googler le nom et rapidement trouvé que la fonction
était utilisée dans MaxDB (c'est un projet open-source de SAP)
\footnote{Plus sur ce sujet dans la section concernée~(\myref{sec:SAPGUI})}.

\url{http://www.google.com/search?q=CsDecomprLZC}

Étonnement, les logiciels MaxDB et SAP 6.0 partagent du code comme ceci pour la compression/
décompression des paquets réseau.

\mysection{Identification de fichiers exécutables}

\subsection{Microsoft Visual C++}
\label{MSVC_versions}

Les versions de MSVC et des DLLs peuvent être importées:

%\small
\begin{center}
\begin{tabular}{ | l | l | l | l | l | }
\hline
\HeaderColor Marketing ver. &
\HeaderColor Internal ver. &
\HeaderColor CL.EXE ver. &
\HeaderColor DLLs imported &
\HeaderColor Release date \\
\hline
% 4.0, April 1995
% 97 & 5.0 & February 1997
6		&  6.0	& 12.00	& msvcrt.dll	& June 1998		\\
		&	&	& msvcp60.dll	&			\\
\hline
.NET (2002)	&  7.0	& 13.00	& msvcr70.dll	& February 13, 2002	\\
		&	&	& msvcp70.dll	&			\\
\hline
.NET 2003	&  7.1	& 13.10 & msvcr71.dll	& April 24, 2003	\\
		&	&	& msvcp71.dll	&			\\
\hline
2005		&  8.0	& 14.00 & msvcr80.dll	& November 7, 2005	\\
		&	&	& msvcp80.dll	&			\\
\hline
2008		&  9.0	& 15.00 & msvcr90.dll	& November 19, 2007	\\
		&	&	& msvcp90.dll	&			\\
\hline
2010		& 10.0	& 16.00 & msvcr100.dll	& April 12, 2010 	\\
		&	&	& msvcp100.dll	&			\\
\hline
2012		& 11.0	& 17.00 & msvcr110.dll	& September 12, 2012 	\\
		&	&	& msvcp110.dll	&			\\
\hline
2013		& 12.0	& 18.00 & msvcr120.dll	& October 17, 2013 	\\
		&	&	& msvcp120.dll	&			\\
\hline
\end{tabular}
\end{center}
%\normalsize

msvcp*.dll contient des fonctions relatives à \Cpp{}, donc si elle est importées,
il s'agit probablement d'un programme \Cpp.

\subsubsection{Mangling de nom}

Les noms commencent en général par le symbole \TT{?}.

Vous trouverez plus d'informations le \glslink{name mangling}{mangling de nom} de
MSVC ici: \myref{namemangling}.

\subsection{GCC}
\myindex{GCC}

À part les cibles *NIX, GCC est aussi présent dans l'environnement win32, sous la
forme de Cygwin et MinGW.

\subsubsection{Mangling de nom}

Les noms commencent en général par le symbole \TT{\_Z}.
Vous trouverez plus d'informations le \glslink{name mangling}{mangling de nom} de
GCC ici: \myref{namemangling}.
\subsubsection{Cygwin}
\myindex{Cygwin}

cygwin1.dll est souvent importée.

\subsubsection{MinGW}
\myindex{MinGW}

msvcrt.dll peut être importée.

\subsection{Intel Fortran}
\myindex{Fortran}

libifcoremd.dll, libifportmd.dll et libiomp5md.dll (support OpenMP) peuvent être importées.

libifcoremd.dll a beaucoup de fonctions préfixées par \TT{for\_}, qui signifie \emph{Fortran}.

\subsection{Watcom, OpenWatcom}
\myindex{Watcom}
\myindex{OpenWatcom}

\subsubsection{Mangling de nom}

Les noms commencent usuellement par le symbole \TT{W}.

Par exemple, ceci est la façon dont la méthode nommées \q{method} de la classe \q{class}
qui n'a pas d'argument et qui renvoie \Tvoid est encodée:

\begin{lstlisting}
W?method$_class$n__v
\end{lstlisting}

\subsection{Borland}
\myindex{Borland Delphi}
\myindex{Borland C++Builder}

Voici un exemple de \glslink{name mangling}{mangling de nom} de Delphi de Borland
et de C++Builder:

\lstinputlisting{digging_into_code/identification/borland_mangling.txt}

Les noms commencent toujours avec le symbole \TT{@}, puis nous avons le nom de la
classe, de la méthode et les types des arguments de méthode encodés.

Ces noms peuvent être dans des imports .exe, des exports .dll, des données de débogage,
etc.

Les Borland Visual Component Libraries (VCL) sont stockées dans des fichiers .bpl
au lieu de .dll, par exemple, vcl50.dll, rtl60.dll.

Une autre DLL qui peut être importée: BORLNDMM.DLL.

\subsubsection{Delphi}

Presque tous les exécutables Delpi ont la chaîne de texte \q{Boolean} au début de
leur segment de code, ainsi que d'autres noms de type.

Ceci est le début très typique du segment \TT{CODE} d'un programme Delphi, ce bloc
vient juste après l'entête de fichier win32 PE:

\lstinputlisting{digging_into_code/identification/delphi.txt}

Les 4 premiers octets du segment de données (\TT{DATA}) peuvent être \TT{00 00 00 00},
\TT{32 13 8B C0} ou \TT{FF FF FF FF}.%

Cette information peut être utile lorsque l'on fait face à des exécutables Delphi
préparés/chiffrés.

\subsection{Autres DLLs connues}

\myindex{OpenMP}
\begin{itemize}
\item vcomp*.dll---implémentation d'OpenMP de Microsoft.
\end{itemize}


% binary files might be also here

\mysection{Communication avec le monde extérieur (niveau fonction)}
Il est souvent recommandé de suivre les arguments de la fonction et sa valeur de
retour dans un débogueur ou \ac{DBI}.
Par exemple, l'auteur a essayé une fois de comprendre la signification d'une fonction
obscure, qui s'est avérée être un tri à bulles mal implémenté\footnote{\url{https://yurichev.com/blog/weird_sort_KLEE/}}.
(Il fonctionnait correctement, mais plus lentement.)
En même temps, regarder les entrées et sorties de cette fonction aide instantanément
à comprendre ce quelle fait.

Souvent, lorsque vous voyez une division par la multiplication (\myref{sec:divisionbymult}),
mais avez oublié tous les détails du mécanisme, vous pouvez seulement observer l'entrée
et la sortie, et trouver le diviseur rapidement.

% sections:
\mysection{Communication avec le monde extérieur (win32)}

Parfois, il est suffisant d'observer les entrées/sorties d'une fonction pour comprendre
ce qu'elle fait.
Ainsi, vous pouvez gagner du temps.

Accès aux fichiers et au registre:
pour les analyses très basiques, l'utilitaire, Process Monitor\footnote{\url{http://technet.microsoft.com/en-us/sysinternals/bb896645.aspx}}
de SysInternals peut aider.

Pour l'analyse basique des accès au réseau, Wireshark\footnote{\url{http://www.wireshark.org/}}
peut être utile.

Mais vous devrez de toutes façons regarder à l'intérieur, \\
\\
Les premières choses à chercher sont les fonctions des \ac{API}s de l'\ac{OS} et
des bibliothèques standards qui sont utilisées.

Si le programme est divisé en un fichier exécutable et un groupe de fichiers DLL,
parfois le nom des fonctions dans ces DLLs peut aider.

Si nous sommes intéressés par exactement ce qui peut conduire à appeler \TT{MessageBox()}
avec un texte spécifique, nous pouvons essayer de trouver ce texte dans le segment
de données, trouver sa référence et trouver les points depuis lesquels le contrôle
peut être passé à l'appel à \TT{MessageBox()} qui nous intéresse.

\myindex{\CStandardLibrary!rand()}
Si nous parlons d'un jeu vidéo et que nous sommes intéressés par les évènements qui
y sont plus ou moins aléatoires, nous pouvons essayer de trouver la fonction \rand
ou sa remplaçante (comme l'algorithme du twister de Mersenne) et trouver les points
depuis lesquels ces fonctions sont appelées, et plus important, comment les résultats
sont utilisés.
% BUG in varioref: http://tex.stackexchange.com/questions/104261/varioref-vref-or-vpageref-at-page-boundary-may-loop
Un exemple: \myref{chap:color_lines}.

Mais si ce n'est pas un jeu, et que \rand est toujours utilisé, il est intéressant
de savoir pourquoi.
Il a y des cas d'utilisation inattendu de \rand dans des algorithmes de compression
de données (pour une imitation du chiffrement):
\href{http://blog.yurichev.com/node/44}{blog.yurichev.com}.

\subsection{Fonctions souvent utilisées dans l'API Windows}

Ces fonctions peuvent être parmi les fonctions importées.
Il est utile de noter que toutes les fonctions ne sont pas forcément utilisées dans
du code écrit par le programmeur.
Beaucoup de fonctions peuvent être appelées depuis des fonctions de bibliothèque
et du code \ac{CRT}.

Certaines fonctions peuvent avoir le suffixe \GTT{-A} pour la version ASCII et \GTT{-W}
pour la version Unicode.

\begin{itemize}

\item
Accès au registre (advapi32.dll): 
RegEnumKeyEx, RegEnumValue, RegGetValue, RegOpenKeyEx, RegQueryValueEx.

\item
Accès au text des fichiers .ini (kernel32.dll):
GetPrivateProfileString.

\item
Boites de dialogue (user32.dll):
MessageBox, MessageBoxEx, CreateDialog, SetDlgItemText, GetDlgItemText.

\item
Accès aux resources (\myref{PEresources}): (user32.dll): LoadMenu.

\item
Réseau TCP/IP (ws2\_32.dll):
WSARecv, WSASend.

\item
Accès fichier (kernel32.dll):
CreateFile, ReadFile, ReadFileEx, WriteFile, WriteFileEx.

\item
Accès haut niveau à Internet (wininet.dll):
WinHttpOpen.

\item
Vérifier la signature digitale d'uin fichier exécutable (wintrust.dll):
WinVerifyTrust.

\item
La bibliothèque MSVC standard (si elle est liée dynamiquement) (msvcr*.dll):
assert, itoa, ltoa, open, printf, read, strcmp, atol, atoi, fopen, fread, fwrite, memcmp, rand,
strlen, strstr, strchr.

\end{itemize}

\subsection{Étendre la période d'essai}

Les fonctions d'accès au registre sont des cibles fréquentes pour ceux qui veulent
essayer de craquer des logiciels avec période d'essai, qui peuvent sauvegarder la
date et l'heure dans un registre.

Des autres cibles courantes sont les fonctions GetLocalTime() et GetSystemTime():
un logiciel avec période d'essai, à chaque démarrage, doit de toutes façons vérifier
la date et l'heure d'une certaine façon.

\subsection{Supprimer la boite de dialogue nag}

Une manière répandue de trouver ce qui cause l'apparition de la boite de dialogue
nag est d'intercepter les fonctions MessageBox(), CreateDialog() et CreateWindow().

\subsection{tracer: Intercepter toutes les fonctions dans un module spécifique}
\myindex{tracer}

\myindex{x86!\Instructions!INT3}
Il y a un point d'arrêt INT3 dans \tracer, qui peut être déclenché seulement une
fois, toutefois, il peut être mis pour toutes les fonctions dans une DLL spécifique.

\begin{lstlisting}
--one-time-INT3-bp:somedll.dll!.*
\end{lstlisting}

Ou, mettons un point d'arrêt INT3 sur toutes les fonctions avec le préfixe \TT{xml}
dans leur nom:

\begin{lstlisting}
--one-time-INT3-bp:somedll.dll!xml.*
\end{lstlisting}

Le revers de la médaille est que de tels points d'arrêt ne sont déclenchés qu'une fois.
Tracer montrera l'appel à une fonction, s'il se produit, mais seulement une fois.
Un autre inconvénient---il est impossible de voir les arguments de la fonction.

Néanmoins, cette fonctionnalité est très utile lorsque vous avez qu'un programme
utilise une DLL, mais que vous ne savez pas quelles fonctions sont effectivement
utilisées.
Et il y a beaucoup de fonctions.

\par
\myindex{Cygwin}
Par exemple, regardons ce qu'utilise l'utilitaire uptime de Cygwin:

\begin{lstlisting}
tracer -l:uptime.exe --one-time-INT3-bp:cygwin1.dll!.*
\end{lstlisting}

Ainsi nous pouvons voir quelles sont les fonctions de la bibliothèque cygwin1.dll
qui sont appelées au moins une fois, et depuis où:

\lstinputlisting{digging_into_code/uptime_cygwin.txt}


\mysection{Chaînes}
\label{sec:digging_strings}

\subsection{Chaînes de texte}

\subsubsection{\CCpp}

\label{C_strings}
Les chaînes C normales sont terminées par un zéro (chaînes \ac{ASCIIZ}).

La raison pour laquelle le format des chaînes C est ce qu'il est (terminé par zéro)
est apparemment historique:
Dans [Dennis M. Ritchie, \emph{The Evolution of the Unix Time-sharing System}, (1979)]
nous lisons:

\begin{framed}
\begin{quotation}
A minor difference was that the unit of I/O was the word, not the byte, because the PDP-7 was a word-addressed
machine. In practice this meant merely that all programs dealing with character streams ignored null
characters, because null was used to pad a file to an even number of characters.
\end{quotation}
\end{framed}
Une différence mineure était que l'unité d'E/S était le mot, pas l'octet, car le PDP-7 était une machine
adressée par mot. En pratique, cela signifiait que tous les programmes ayant à faire
avec des flux de caractères ignoraient le caractère nul, car nul était utilisé pour compléter un fichier
ayant un nombre impair de caractères.

\myindex{Hiew}

Dans Hiew ou FAR Manager ces chaînes ressemblent à ceci:

\begin{lstlisting}[style=customc]
int main()
{
	printf ("Hello, world!\n");
};
\end{lstlisting}

\begin{figure}[H]
\centering
\includegraphics[width=0.6\textwidth]{digging_into_code/strings/C-string.png}
\caption{Hiew}
\end{figure}

% FIXME видно \n в конце, потом пробел

\subsubsection{Borland Delphi}
\myindex{Pascal}
\myindex{Borland Delphi}

Une chaîne en Pascal et en Delphi de Borland est précédée par sa longueur sur 8-bit
ou 32-bit.

Par exemple:

\begin{lstlisting}[caption=Delphi,style=customasmx86]
CODE:00518AC8                 dd 19h
CODE:00518ACC aLoading___Plea db 'Loading... , please wait.',0

...

CODE:00518AFC                 dd 10h
CODE:00518B00 aPreparingRun__ db 'Preparing run...',0
\end{lstlisting}

\subsubsection{Unicode}

\myindex{Unicode}

Souvent, ce qui est appelé Unicode est la méthode pour encoder des chaînes où chaque
caractère occupe 2 octets ou 16 bits.
Ceci est une erreur de terminologie répandue.
Unicode est un standard pour assigner un nombre à chaque caractère dans un des nombreux
systèmes d'écriture dans le monde, mais ne décrit pas la méthode d'encodage.

\myindex{UTF-8}
\myindex{UTF-16LE}
Les méthodes d'encodage les plus répandues sont: UTF-8 (est répandue sur Internet
et les systèmes *NIX) et UTF-16LE (est utilisé dans Windows).

\myparagraph{UTF-8}

\myindex{UTF-8}
UTF-8 est l'une des méthodes les plus efficace pour l'encodage des caractères.
Tous les symboles Latin sont encodés comme en ASCII, et les symboles après la table
ASCII sont encodés en utilisant quelques octets.
0 est encodé comme avant, donc toutes les fonctions C de chaîne standard fonctionnent
avec des chaînes UTF-8 comme avec tout autre chaîne.

Voyons comment les symboles de divers langages sont encodés en UTF-8 et de quoi ils
ont l'air en FAR, en utilisant la page de code 437%
\footnote{L'exemple et les traductions ont été pris d'ici:
\url{http://go.yurichev.com/17304}}:

\begin{figure}[H]
\centering
\includegraphics[width=0.6\textwidth]{digging_into_code/strings/multilang_sampler.png}
\end{figure}

% FIXME: cut it
\begin{figure}[H]
\centering
\myincludegraphics{digging_into_code/strings/multilang_sampler_UTF8.png}
\caption{FAR: UTF-8}
\end{figure}

Comme vous le voyez, la chaîne en anglais est la même qu'en ASCII.

Le hongrois utilise certains symboles Latin et des symboles avec des signes diacritiques.

Ces symboles sont encodés en utilisant plusieurs octets, qui sont soulignés en rouge.
C'est le même principe avec l'islandais et le polonais.

Il y a aussi le symbole de l'\q{Euro} au début, qui est encodé avec 3 octets.

Les autres systèmes d'écritures n'ont de point commun avec Latin.

Au moins en russe, arabe hébreux et hindi, nous pouvons voir des octets récurrents,
et ce n'est pas une surprise: tous les symboles d'un système d'écriture sont en général
situés dans la même table Unicode, donc leur code débute par le même nombre.

Au début, avant la chaîne \q{How much?}, nous voyons 3 octets, qui sont en fait le
\ac{BOM}. Le \ac{BOM} défini le système d'encodage à utiliser.

\myparagraph{UTF-16LE}

\myindex{UTF-16LE}
\myindex{Windows!Win32}
De nombreuses fonctions win32 de Windows ont le suffixes \TT{-A} et \TT{-W}.
Le premier type de fonctions fonctionne avec les chaînes normales, l'autre, avec
des chaîne UTF-16LE (\emph{large}).

Dans le second cas, chaque symbole est en général stocké dans une valeur 16-bit de
type \emph{short}.

Les symboles Latin dans les chaînes UTF-16 dans Hiew ou FAR semblent être séparés
avec un octet zéro:

\begin{lstlisting}[style=customc]
int wmain()
{
	wprintf (L"Hello, world!\n");
};
\end{lstlisting}

\begin{figure}[H]
\centering
\includegraphics[width=0.6\textwidth]{digging_into_code/strings/UTF16-string.png}
\caption{Hiew}
\end{figure}

Nous voyons souvent ceci dans les fichiers système de \gls{Windows NT}:

\begin{figure}[H]
\centering
\includegraphics[width=0.6\textwidth]{digging_into_code/strings/ntoskrnl_UTF16.png}
\caption{Hiew}
\end{figure}

\myindex{IDA}
Les chaînes avec des caractères qui occupent exactement 2 octets sont appelées \q{Unicode}
dans \IDA:

\begin{lstlisting}[style=customasmx86]
.data:0040E000 aHelloWorld:
.data:0040E000                 unicode 0, <Hello, world!>
.data:0040E000                 dw 0Ah, 0
\end{lstlisting}

Voici comment une chaîne en russe est encodée en UTF-16LE:

\begin{figure}[H]
\centering
\includegraphics[width=0.6\textwidth]{digging_into_code/strings/russian_UTF16.png}
\caption{Hiew: UTF-16LE}
\end{figure}

Ce que nous remarquons facilement, c'est que les symboles sont intercalés par le
caractère diamant (qui a le code ASCII 4). En effet, les symboles cyrilliques sont
situés dans le quatrième plan Unicode.
Ainsi, tous les symboles cyrillique en UTF-16LE sont situés dans l'intervalle \TT{0x400-0x4FF}.

Retournons à l'exemple avec la chaîne écrite dans de multiple langages.
Voici à quoi elle ressemble en UTF-16LE.

% FIXME: cut it
\begin{figure}[H]
\centering
\myincludegraphics{digging_into_code/strings/multilang_sampler_UTF16.png}
\caption{FAR: UTF-16LE}
\end{figure}

Ici nous pouvons aussi voir le \ac{BOM} au début.
Tous les caractères Latin sont intercalés avec un octet à zéro.

Certains caractères avec signe diacritique (hongrois et islandais) sont aussi soulignés en rouge.

% subsection:
\subsubsection{Base64}
\myindex{Base64}

L'encodage base64 est très répandu dans les cas où vous devez transférer des données
binaires sous forme de chaîne de texte.

Pour l'essentiel, cet algorithme encode 3 octets binaires en 4 caractères imprimables:
toutes les 26 lettres Latin (à la fois minuscule et majuscule), chiffres, signe plus
(\q{+}) et signe slash (\q{/}), 64 caractères en tout.

Une particularité des chaînes base64 est qu'elles se terminent souvent (mais pas
toujours) par 1 ou 2 symbole égal (\q{=}) pour l'alignement, par exemple:

\begin{lstlisting}
AVjbbVSVfcUMu1xvjaMgjNtueRwBbxnyJw8dpGnLW8ZW8aKG3v4Y0icuQT+qEJAp9lAOuWs=
\end{lstlisting}

\begin{lstlisting}
WVjbbVSVfcUMu1xvjaMgjNtueRwBbxnyJw8dpGnLW8ZW8aKG3v4Y0icuQT+qEJAp9lAOuQ==
\end{lstlisting}

Le signe égal (\q{=}) ne se rencontre jamais au milieu des chaînes encodées en base64.

maintenant, un exemple d'encodage manuel.
Encodons les octets hexadécimaux 0x00, 0x11, 0x22, 0x33 en une chaîne base64:

\lstinputlisting{digging_into_code/strings/base64_ex.sh}

Mettons ces 4 octets au forme binaire, puis regroupons les dans des groupes de 6-bit:

\begin{lstlisting}
|  00  ||  11  ||  22  ||  33  ||      ||      |
00000000000100010010001000110011????????????????
| A  || B  || E  || i  || M  || w  || =  || =  |
\end{lstlisting}

Les trois premiers octets (0x00, 0x11, 0x22) peuvent être encodés dans 4 caractères
base64 (``ABEi''), mais le dernier (0x33) --- ne le peut pas, donc il est encodé
en utilisant deux caractères (``Mw'') et de symbole (``='') de padding est ajouté
deux fois pour compléter le dernier groupe à 4 caractères.
De ce fait, la longueur de toutes les chaînes en base64 correctes est toujours divisible
par 4.

\myindex{XML}
\myindex{PGP}
Base64 est souvent utilisé lorsque des données binaires doivent être stockées dans
du XML.  Les clefs PGP ``Armored'' (i.e., au format texte) et les signatures sont
encodées en utilisant base64.

Certains essayent d'utiliser base64 pour masquer des chaînes:
\url{http://blog.sec-consult.com/2016/01/deliberately-hidden-backdoor-account-in.html}%
\footnote{\url{http://archive.is/nDCas}}.

\myindex{base64scanner}
Il existe des utilitaires pour rechercher des chaînes base64 dans des fichiers binaires
arbitraires.
L'un d'entre eux est base64scanner\footnote{\url{https://github.com/DennisYurichev/base64scanner}}.

\myindex{UseNet}
\myindex{FidoNet}
\myindex{Uuencoding}
\myindex{Phrack}
Un autre système d'encodage qui était très répandu sur UseNet et FidoNet est l'Uuencoding.
Les fichiers binaires sont toujours encodés au format Uuencode dans le magazine Phrack.
Il offre à peu près la même fonctionnalité, mais il est différent de base64 dans
le sens où le nom de fichier est aussi stocké dans l'entête.

\myindex{Tor}
\myindex{base32}
À propos: base64 à un petit frère: base32, alphabet qui a ~10 chiffres et ~26 caractères Latin.
Un usage répandu est les adresses onion%
\footnote{\url{https://trac.torproject.org/projects/tor/wiki/doc/HiddenServiceNames}},
comme: \\
\url{http://3g2upl4pq6kufc4m.onion/}.
\ac{URL} ne peut pas avoir de mélange de casse de caractères Latin, donc, c'est apparemment
pourquoi les développeurs de Tor ont utilisé base32.





\subsection{Trouver des chaînes dans un binaire}

\epigraph{Actually, the best form of Unix documentation is frequently running the
\textbf{strings} command over a program’s object code. Using \textbf{strings}, you can get
a complete list of the program’s hard-coded file name, environment variables,
undocumented options, obscure error messages, and so forth.}{The Unix-Haters Handbook}
En fait, la meilleure forme de documentation Unix est de lancer la commande
\textbf{strings} sur le code objet d'un programme. En utilisant \textbf{strings},
vous obtenez une liste complète des noms de fichiers codés en dur dans le programme,
les variables d'environnement, les options non documentées, les messages d'erreurs
méconnus et ainsi de suite.

\myindex{UNIX!strings}
L'utilitaire standard UNIX \emph{strings} est un moyen rapide et facile de voir les
chaînes dans un fichier.
Par exemple, voici quelques chaînes du fichier exécutable sshd d'OpenSSH 7.2:

\lstinputlisting{digging_into_code/sshd_strings.txt}

Il y a des options, des messages d'erreur, des chemins de fichier, des modules et
des fonctions importés dynamiquement, ainsi que d'autres chaînes étranges (clefs?).
Il y a aussi du bruit illisible---le code x86 à parfois des fragments constitués de
caractères ASCII imprimables, jusqu'à ~8 caractères.

Bien sûr, OpenSSH est un programme open-source.
Mais regarder les chaînes lisibles dans un binaire inconnu est souvent une première
étape d'analyse.
\myindex{UNIX!grep}

\emph{grep} peut aussi être utilisé.

\myindex{Hiew}
\myindex{Sysinternals}
Hiew a la même capacité (Alt-F6), ainsi que ProcessMonitor de Sysinternals.

\subsection{Messages d'erreur/de débogage}

Les messages de débogage sont très utiles s'il sont présents.
Dans un certain sens, les messages de débogage rapportent ce qui est en train de
se passer dans le programme. Souvent, ce sont des fonctions \printf-like, qui écrivent
des fichiers de log, ou parfois elles n'écrivent rien du tout mais les appels sont
toujours présents puisque le build n'est pas un de débogage mais de \emph{release}.
\myindex{\oracle}

Si des variables locales ou globales sont affichées dans les messages, ça peut être
aussi utile, puisqu'il est possible d'obtenir au moins le nom de la variable.
Par exemple, une telle fonction dans \oracle est \TT{ksdwrt()}.

Des chaînes de texte significatives sont souvent utiles.
Le dés-assembleur \IDA peut montrer depuis quelles fonctions et depuis quel endroit
cette chaîne particulière est utilisée.
Des cas drôles arrivent parfois\footnote{\href{http://go.yurichev.com/17223}{blog.yurichev.com}}.

Le message d'erreur peut aussi nous aider.
Dans \oracle, les erreurs sont rapportées en utilisant un groupe de fonctions.\\
Vous pouvez en lire plus ici: \href{http://go.yurichev.com/17224}{blog.yurichev.com}.

\myindex{Error messages}

Il est possible de trouver rapidement quelle fonction signale une erreur et dans
quelles conditions.

À propos, ceci est souvent la raison pour laquelle les systèmes de protection contre
la copie utilisent des messages d'erreur inintelligibles ou juste des numéros d'erreur.
Personne n'est content lorsque le copieur de logiciel comprend rapidement pourquoi
la protection contre la copie est déclenchée seulement par les messages d'erreur.

Un exemple de messages d'erreur chiffrés se trouve ici: \myref{examples_SCO}.

\subsection{Chaînes magiques suspectes}

Certaines chaînes magique sont d'habitude utilisées dans les porte dérobées semblent
vraiment suspectes.

Par exemple, il y avait une porte dérobée dans le routeur personnel TP-Link WR740%
\footnote{\url{http://sekurak.pl/tp-link-httptftp-backdoor/}}.
La porte dérobée était activée en utilisant l'URL suivante:\\
\url{http://192.168.0.1/userRpmNatDebugRpm26525557/start_art.html}.\\

En effet, la chaîne \q{userRpmNatDebugRpm26525557} est présente dans le firmware.

Cette chaîne n'était pas googlable jusqu'à la large révélation d'information concernant
la porte dérobée.

Vous ne trouverez ceci dans aucun \ac{RFC}.

Vous ne trouverez pas d'algorithme informatique qui utilise une séquence d'octets
aussi étrange.

Et elle ne ressemble pas à un erreur ou un message de débogage.

Donc, c'est une bonne idée d'inspecter l'utilisation de ce genre de chaînes bizarres.\\
\\
\myindex{base64}

Parfois, de telles chaînes sont encodées en utilisant base64.

Donc, c'est une bonne idée de les décoder toutes et de les inspecter visuellement,
même un coup d'\oe{}il doit suffire.\\
\\
\myindex{Sécurité par l'obscurité}
Plus précisément, cette méthode de cacher des accès non documentés est appelée \q{sécurité par l'obscurité}.


\mysection{Appels à assert()}
\myindex{\CStandardLibrary!assert()}

Parfois, la présence de la macro \TT{assert()} est aussi utile:
En général, cette macro laisse le nom du fichier source, le numéro de ligne et une
condition dans le code.

L'information la plus utile est contenue dans la condition d'assert, nous pouvons
en déduire les noms de variables ou les noms de champ de la structure. Les autres
informations utiles sont les noms de fichier---nous pouvons essayer d'en déduire
le type de code dont il s'agit ici.
Il est aussi possible de reconnaître les bibliothèques open-source connues d'après
les noms de fichier.

\lstinputlisting[caption=Exemple d'appels à assert() informatifs,style=customasmx86]{digging_into_code/assert_examples.lst}

Il est recommandé de \q{googler} à la fois les conditions et les noms de fichier,
qui peuvent nous conduire à une bibliothèque open-source.
Par exemple, si nous \q{googlons} \q{sp->lzw\_nbits <= BITS\_MAX}, cela va comme
prévu nous donner du code open-source relatif à la compression LZW.

\mysection{Constantes}

Les humains, programmeurs inclus, utilisent souvent des nombres ronds, comme 10, 100,
1000, dans la vie courante comme dans le code.

Le rétro ingénieur pratiquant connaît en général bien leur représentation décimale:
10=0xA, 100=0x64, 1000=0x3E8, 10000=0x2710.

Les constantes \TT{0xAAAAAAAA} (0b10101010101010101010101010101010) et \\
\TT{0x55555555} (0b01010101010101010101010101010101)  sont aussi répandues---elles
sont composées d'alternance de bits.

Cela peut aider à distinguer un signal d'un signal dans lequel tous les bits sont
à 1 (0b1111 \dots) ou à 0 (0b0000 \dots).
Par exemple, la constante \TT{0x55AA} est utilisée au moins dans le secteur de boot,
\ac{MBR}, et dans la \ac{ROM} de cartes d'extention de compatible IBM.

Certains algorithmes, particulièrement ceux de chiffrement, utilisent des constantes
distinctes, qui sont faciles à trouver dans le code en utilisant \IDA.

\myindex{MD5}
\newcommand{\URLMD}{http://go.yurichev.com/17111}

Par exemple, l'algorithme MD5\footnote{\href{\URLMD}{Wikipédia}} initialise ses propres
variables internes comme ceci:

\begin{verbatim}
var int h0 := 0x67452301
var int h1 := 0xEFCDAB89
var int h2 := 0x98BADCFE
var int h3 := 0x10325476
\end{verbatim}

Si vous trouvez ces quatre constantes utilisées à la suite dans du code, il est
très probable que cette fonction soit relatives à MD5.

\par Un autre exemple sont les algorithmes CRC16/CRC32, ces algorithmes de calcul
utilisent souvent des tables pré-calculées comme celle-ci:

\begin{lstlisting}[caption=linux/lib/crc16.c,style=customc]
/** CRC table for the CRC-16. The poly is 0x8005 (x^16 + x^15 + x^2 + 1) */
u16 const crc16_table[256] = {
	0x0000, 0xC0C1, 0xC181, 0x0140, 0xC301, 0x03C0, 0x0280, 0xC241,
	0xC601, 0x06C0, 0x0780, 0xC741, 0x0500, 0xC5C1, 0xC481, 0x0440,
	0xCC01, 0x0CC0, 0x0D80, 0xCD41, 0x0F00, 0xCFC1, 0xCE81, 0x0E40,
	...
\end{lstlisting}

Voir aussi la table pré-calculée pour CRC32: \myref{sec:CRC32}.

Dans les algorithmes CRC sans table, des polynômes bien connus sont utilisés, par
exemple 0xEDB88320 pour CRC32.

\subsection{Nombres magiques}
\label{magic_numbers}

\newcommand{\FNURLMAGIC}{\footnote{\href{http://go.yurichev.com/17112}{Wikipédia}}}

De nombreux formats de fichier définissent un entête standard où un \emph{nombre(s) magique}\FNURLMAGIC{}
est utilisé, unique ou même plusieurs.

\myindex{MS-DOS}

Par exemple, tous les exécutables Win32 et MS-DOS débutent par ces deux caractères \q{MZ}\footnote{\href{http://go.yurichev.com/17113}{Wikipédia}}.

\myindex{MIDI}

Au début d'un fichier MIDI, la signature \q{MThd} doit être présente.
Si nous avons un programme qui utilise des fichiers MIDI pour quelque chose, il
est très probable qu'il doit vérifier la validité du fichier en testant au moins
les 4 premiers octets.

Ça peut être fait comme ceci:
(\emph{buf} pointe sur le début du fichier chargé en mémoire)

\begin{lstlisting}[style=customasmx86]
cmp [buf], 0x6468544D ; "MThd"
jnz _error_not_a_MIDI_file
\end{lstlisting}

\myindex{\CStandardLibrary!memcmp()}
\myindex{x86!\Instructions!CMPSB}

\dots ou en appelant une fonction pour comparer des blocs de mémoire comme \TT{memcmp()}
ou tout autre code équivalent jusqu'à une instruction \TT{CMPSB} (\myref{REPE_CMPSx}).

Lorsque vous trouvez un tel point, vous pouvez déjà dire que le chargement du fichier
MIDI commence, ainsi, vous pouvez voir l'endroit où se trouve le buffer avec le contenu
du fichier MIDI, ce qui est utilisé dans le buffer et comment.

\subsubsection{Dates}

\myindex{UFS2}
\myindex{FreeBSD}
\myindex{HASP}

Souvent, on peut rencontrer des nombres comme \TT{0x19870116}, qui ressemble clairement
à une date (année 1987, 1er mois (janvier), 16ème jour).
Ça peut être la date de naissance de quelqu'un (un programmeur, une de ses relations,
un enfant), ou une autre date importante.
La date peut aussi être écrite dans l'ordre inverse, comme \TT{0x16011987}.
Les dates au format américain sont aussi courante, comme \TT{0x01161987}.

Un exemple célèbre est \TT{0x19540119} (nombre magique utilisé dans la structure
du super-bloc UFS2), qui est la date de naissance de Marshall Kirk McKusick, éminent
contributeur FreeBSD.

\myindex{Stuxnet}
Stuxnet utilise le nombre ``19790509'' (pas comme un nombre 32-bit, mais comme une
chaîne, toutefois), et ça a conduit à spéculer que le malware était relié à Israël%
\footnote{C'est la date d'exécution de Habib Elghanian, juif persan.}.

Aussi, des nombre comme ceux-ci sont très répandus dans dans le chiffrement niveau
amateur, par exemple, extrait de la \emph{fonction secrète} des entrailles du dongle
HASP3\footnote{\url{https://web.archive.org/web/20160311231616/http://www.woodmann.com/fravia/bayu3.htm}}:

\begin{lstlisting}[style=customc]
void xor_pwd(void) 
{ 
	int i; 
	
	pwd^=0x09071966;
	for(i=0;i<8;i++) 
	{ 
		al_buf[i]= pwd & 7; pwd = pwd >> 3; 
	} 
};

void emulate_func2(unsigned short seed)
{ 
	int i, j; 
	for(i=0;i<8;i++) 
	{ 
		ch[i] = 0; 
		
		for(j=0;j<8;j++)
		{ 
			seed *= 0x1989; 
			seed += 5; 
			ch[i] |= (tab[(seed>>9)&0x3f]) << (7-j); 
		}
	} 
}
\end{lstlisting}

\subsubsection{DHCP}

Ceci s'applique aussi aux protocoles réseaux.
Par exemple, les paquets réseau du protocole DHCP contiennent un soi-disant \emph{nombre
magique}: \TT{0x63538263}.
Tout code qui génère des paquets DHCP doit contenir quelque part cette constante
à insérer dans les paquets.
Si nous la trouvons dans du code, nous pouvons trouver ce qui s'y passe, et pas seulement ça.
Tout programme qui peut recevoir des paquet DHCP doit vérifier le \emph{cookie magique},
et le comparer à cette constante.

Par exemple, prenons le fichier dhcpcore.dll de Windows 7 x64 et cherchons cette constante.
Et nous la trouvons, deux fois:
Il semble que la constante soit utilisée dans deux fonctions avec des noms parlants\\
\TT{DhcpExtractOptionsForValidation()} et \TT{DhcpExtractFullOptions()}:

\begin{lstlisting}[caption=dhcpcore.dll (Windows 7 x64),style=customasmx86]
.rdata:000007FF6483CBE8 dword_7FF6483CBE8 dd 63538263h          ; DATA XREF: DhcpExtractOptionsForValidation+79
.rdata:000007FF6483CBEC dword_7FF6483CBEC dd 63538263h          ; DATA XREF: DhcpExtractFullOptions+97
\end{lstlisting}

Et ici sont les endroits où ces constantes sont accédées:

\begin{lstlisting}[caption=dhcpcore.dll (Windows 7 x64),style=customasmx86]
.text:000007FF6480875F  mov     eax, [rsi]
.text:000007FF64808761  cmp     eax, cs:dword_7FF6483CBE8
.text:000007FF64808767  jnz     loc_7FF64817179
\end{lstlisting}

Et:

\begin{lstlisting}[caption=dhcpcore.dll (Windows 7 x64),style=customasmx86]
.text:000007FF648082C7  mov     eax, [r12]
.text:000007FF648082CB  cmp     eax, cs:dword_7FF6483CBEC
.text:000007FF648082D1  jnz     loc_7FF648173AF
\end{lstlisting}

\subsection{Constantes spécifiques}

Parfois, il y a une constante spécifique pour un certain type de code.
Par exemple, je me suis plongé une fois dans du code, où le nombre 12 était rencontré
anormalement souvent.
La taille de nombreux tableaux était 12 ou un multiple de 12 (24, etc.).
Il s'est avéré que ce code prenait des fichiers audio de 12 canaux en entrée et les
traitait.

Et vice versa: par exemple, si un programme fonctionne avec des champs de texte qui
ont une longueur de 120 octets, il doit y avoir une constante 120 ou 119 quelque
part dans le code.
Si UTF-16 est utilisé, alors $2 \cdot 120$.
Si le code fonctionne avec des paquets réseau de taille fixe, c'est une bonne idée
de chercher cette constante dans le code.

C'est aussi vrai pour le chiffrement amateur (clefs de licence, etc.).
Si le bloc chiffré a une taille de $n$ octets, vous pouvez essayer de trouver des
occurrences de ce nombre à travers le code.
Aussi, si vous voyez un morceau e code qui est répété $n$ fois dans une boucle durant
l'exécution, ceci peut être une routine de chiffrement/déchiffrement.

\subsection{Chercher des constantes}

C'est facile dans \IDA: Alt-B or Alt-I.
\myindex{binary grep}
Et pour chercher une constante dans un grand nombre de fichiers, ou pour chercher
dans des fichiers non exécutables, il y a un petit utilitaire appelé \emph{binary grep}\footnote{\BGREPURL}.


\mysection{Trouver les bonnes instructions}

Si le programme utilise des instructions FPU et qu'il n'y en a que quelques une dans
le code, on peut essayer de les vérifier chacunes manuellement avec un débogueur.

\par Par exemple, nous pouvons être intéressés de comprendre comment Microsoft Excel
calcule la formule entrée par l'utilisateur.
Par exemple, l'opération de division.

\myindex{\GrepUsage}
\myindex{x86!\Instructions!FDIV}

Si nous chargeons excel.exe (d'Office 2010) version 14.0.4756.1000 dans \IDA, faisons
un listing complet et cherchons chaque instruction \FDIV (sauf celle qui utilisent
une constante comme second opérande---évidemment, elles ne nous intéressent pas):

\begin{lstlisting}
cat EXCEL.lst | grep fdiv | grep -v dbl_ > EXCEL.fdiv
\end{lstlisting}

\dots nous voyons alors qu'il y en a 144.

\par Nous pouvons entrer une chaîne comme \TT{=(1/3)} dans Excel et vérifier chaque
instruction.

\myindex{tracer}

\par En vérifiant chaque instruction dans un débogueur ou \tracer
(on peut vérifier 4 instructions à la fois),
nous avons de la chance et l'instruction que nous cherchons n'est que la 14ème:

\begin{lstlisting}[style=customasmx86]
.text:3011E919 DC 33          fdiv    qword ptr [ebx]
\end{lstlisting}

\begin{lstlisting}
PID=13944|TID=28744|(0) 0x2f64e919 (Excel.exe!BASE+0x11e919)
EAX=0x02088006 EBX=0x02088018 ECX=0x00000001 EDX=0x00000001
ESI=0x02088000 EDI=0x00544804 EBP=0x0274FA3C ESP=0x0274F9F8
EIP=0x2F64E919
FLAGS=PF IF
FPU ControlWord=IC RC=NEAR PC=64bits PM UM OM ZM DM IM
FPU StatusWord=
FPU ST(0): 1.000000
\end{lstlisting}

\ST{0} contient le premier argument (1) et le second est dans \TT{[EBX]}.\\
\\
\myindex{x86!\Instructions!FDIV}

L'instruction après \FDIV (\TT{FSTP}) écrit le résultat en mémoire:\\

\begin{lstlisting}[style=customasmx86]
.text:3011E91B DD 1E          fstp    qword ptr [esi]
\end{lstlisting}

Si nous mettons un point d'arrêt dessus, nous voyons le résultat:

\begin{lstlisting}
PID=32852|TID=36488|(0) 0x2f40e91b (Excel.exe!BASE+0x11e91b)
EAX=0x00598006 EBX=0x00598018 ECX=0x00000001 EDX=0x00000001
ESI=0x00598000 EDI=0x00294804 EBP=0x026CF93C ESP=0x026CF8F8
EIP=0x2F40E91B
FLAGS=PF IF
FPU ControlWord=IC RC=NEAR PC=64bits PM UM OM ZM DM IM
FPU StatusWord=C1 P
FPU ST(0): 0.333333
\end{lstlisting}

Pour blaguer, nous pouvons modifier le résultat au vol:

\begin{lstlisting}
tracer -l:excel.exe bpx=excel.exe!BASE+0x11E91B,set(st0,666)
\end{lstlisting}

\begin{lstlisting}
PID=36540|TID=24056|(0) 0x2f40e91b (Excel.exe!BASE+0x11e91b)
EAX=0x00680006 EBX=0x00680018 ECX=0x00000001 EDX=0x00000001
ESI=0x00680000 EDI=0x00395404 EBP=0x0290FD9C ESP=0x0290FD58
EIP=0x2F40E91B
FLAGS=PF IF
FPU ControlWord=IC RC=NEAR PC=64bits PM UM OM ZM DM IM
FPU StatusWord=C1 P
FPU ST(0): 0.333333
Set ST0 register to 666.000000
\end{lstlisting}

Excel affiche 666 dans la cellule, achevant de nous convaincre que nous avons trouvé
le bon endroit.

\begin{figure}[H]
\centering
\includegraphics[width=0.6\textwidth]{digging_into_code/Excel_prank.png}
\caption{La blague a fonctionné}
\end{figure}

Si nous essayons la même version d'Excel, mais en x64, nous allons y trouver seulement
12 instructions \FDIV, et celle que nous cherchons est la troisième.

\begin{lstlisting}
tracer.exe -l:excel.exe bpx=excel.exe!BASE+0x1B7FCC,set(st0,666)
\end{lstlisting}

\myindex{x86!\Instructions!DIVSD}

Il semble que le compilateur a remplacé beaucoup d'opérations de division de types
\Tfloat et \Tdouble, par des instructions SSE comme \TT{DIVSD} (\TT{DIVSD} est présent
268 fois en tout).

\mysection{Patterns de code suspect}

\subsection{instructions XOR}
\myindex{x86!\Instructions!XOR}

Des instructions comme \TT{XOR op, op} (par exemple, \TT{XOR EAX, EAX}) sont utilisées
en général pour mettre la valeur d'un registre à zéro, mais si les opérandes sont
différentes, l'opération \q{ou exclusif} est exécutée.

Cette opération est rare en programmation courante, mais répandu en cryptographie,
y compris amateur.
C'est particulièrement suspect si le second opérande est un grand nombre.

Ceci peut indiquer du chiffrement/déchiffrement, du calcul de somme de contrôle, etc.\\
\\

Une exception à cette observation, qu'il est utile de noter, est le \q{canari} (\myref{subsec:BO_protection}).
Sa génération et sa vérification sont souvent effectuées en utilisant des instructions
\XOR. \\
\\
\myindex{AWK}

Ce script awk peut être utilisé pour traité les fichiers listing (.lst) d'\IDA:

\lstinputlisting{digging_into_code/awk.sh}

Il est aussi utile de noter que ce type de script peut aussi rapporter du code mal
désassemblé
(\myref{sec:incorrectly_disasmed_code}).

\subsection{Code assembleur écrit à la main}

\myindex{Function prologue}
\myindex{Function epilogue}
\myindex{x86!\Instructions!LOOP}
\myindex{x86!\Instructions!RCL}

Les compilateurs modernes ne génèrent pas les instructions \TT{LOOP} et \TT{RCL}.
D'un autre côté, ces instructions sont très connues des codeurs qui aiment écrire
directement en langage d'assemblage.
Si vous les rencontrez, on peut dire qu'il est très probable que ce morceau de code
ait été écrit à la main.
De telles instructions sont marquées avec un (M) dans la liste des instructions de
cet appendice: \myref{sec:x86_instructions}.

\par
De même, les prologue/épilogue de fonction sont rares dans de l'assembleur écrit
à la main.

\par
Il n'y a généralement pas de système fixé pour le passage des arguments aux fonctions
dans du code écrit à la main.

\par
Exemple du noyau de Windows 2003
(ntoskrnl.exe file):

\lstinputlisting[style=customasmx86]{digging_into_code/ntoskrnl.lst}

En effet, si nous regardons dans le code source de \ac{WRK} v1.2, ce code peut être
trouvé facilement dans le fichier \\
\emph{WRK-v1.2\textbackslash{}base\textbackslash{}ntos\textbackslash{}ke\textbackslash{}i386\textbackslash{}cpu.asm}.

\par 
D'après l'instruction \INS{RCL} que j'ai pu trouver dans le fichier ntoskrnl.exe de Windows
2003 x86 (compilé avec MS Visual C compiler).
Elle apparaît seulement une fois ici, dans la fonction \TT{RtlExtendedLargeIntegerDivide()}
et ça pourrait être un cas de code assembleur en ligne.

\mysection{Utilisation de nombres magiques lors du tracing}

Souvent, notre but principal est de comprendre comment le programme utilise une valeur
qui a été soit lue d'un fichier ou reçue par le réseau. Le tracing manuel d'une valeur
est souvent une tâche laborieuse. Une des techniques les plus simple pour ceci (bien
que non sûre à 100\%) est d'utiliser votre propre \emph{nombre magique}.

Ceci ressemble à la tomodensitométrie aux rayons X: un agent de radio-contraste est
injecté dans le sang du patient, qui est utilisé pour augmenter la visibilité de
la structure interne du patient aux rayons X.
C'est bien connu comment le sang circule dans les reins d'humains en bonne santé
et si l'agent est dans le sang, il peut être vu facilement en tomographie comment
le sang circule et si il y a des calculs ou des tumeurs.

Nous pouvons prendre un nombre 32-bit comme \TT{0x0badf00d}, ou la date de naissance
de quelqu'un comme \TT{0x11101979} et écrire ce nombre de 4 octets quelque part dans
un fichier utilisé par le programme que nous investiguons.

\myindex{\GrepUsage}
\myindex{tracer}

Puis, en suivant ce programme avec \tracer en mode \emph{code coverage}, avec l'aide
de \emph{grep} ou simplement en cherchant dans le fichier texte (résultant de l'investigation),
nous pouvons facilement voir où la valeur a été utilisée et comment.

Exemple de résultats de \tracer \emph{grepable} en mode \emph{cc}:

\begin{lstlisting}[style=customasmx86]
0x150bf66 (_kziaia+0x14), e=       1 [MOV EBX, [EBP+8]] [EBP+8]=0xf59c934
0x150bf69 (_kziaia+0x17), e=       1 [MOV EDX, [69AEB08h]] [69AEB08h]=0
0x150bf6f (_kziaia+0x1d), e=       1 [FS: MOV EAX, [2Ch]]
0x150bf75 (_kziaia+0x23), e=       1 [MOV ECX, [EAX+EDX*4]] [EAX+EDX*4]=0xf1ac360
0x150bf78 (_kziaia+0x26), e=       1 [MOV [EBP-4], ECX] ECX=0xf1ac360
\end{lstlisting}
% TODO: good example!

Cela peut aussi être utilisé pour des paquets réseau.
Il est important que le \emph{nombre magique} soit unique et ne soit pas présent
dans le code du programme.

\newcommand{\DOSBOXURL}{\href{http://blog.yurichev.com/node/55}{blog.yurichev.com}}

\myindex{DosBox}
\myindex{MS-DOS}
À part \tracer, DosBox (émulateur MS-DOS) en mode heavydebug est capable d'écrire
de l'information à propos de l'état de tous les registres pour chaque instruction
du programme exécutée dans un fichier texte\footnote{Voir aussi mon article de blog
sur cette fonctionnalité de DosBox: \DOSBOXURL{}}, donc cette technique peut être
utile également pour des programmes DOS.

\mysection{Boucles}

À chaque fois que votre programme travaille avec des sortes de fichier, ou un buffer
d'une certaine taille, il doit s'agir d'un sorte de boucle de déchiffrement/traitement
à l'intérieur du code.

Ceci est un exemple réel de sortie de l'outil \tracer.
Il y avait un code qui chargeait une sorte de fichier chiffré de 258 octets.
Je l'ai lancé dans l'intention d'obtenir le nombre d'exécution de chaque instruction
(l'outil \ac{DBI} irait beaucoup mieux de nos jours).
Et j'ai rapidement trouvé un morceau de code qui était exécuté 259/258 fois:

\lstinputlisting{digging_into_code/crypto_loop.txt}

Il s'avère qu'il s'agit de la boucle de déchiffrement.


% TODO move section...

\subsection{Quelques schémas de fichier binaire}

Tous les exemples ici ont été préparé sur Windows, avec la page de code 437 activée%
dans la console.
L'intérieur des fichiers binaires peut avoir l'air différent avec une autre page
de code.

\clearpage
\subsubsection{Tableaux}

Parfois, nous pouvons clairement localiser visuellement un tableau de valeurs 16/32/64-bit,
dans un éditeur hexadécimal.

Voici un exemple de tableau de valeurs 16-bit.
Nous voyons que le premier octet d'une paire est 7 ou 8, et que le second semble
aléatoire:

\begin{figure}[H]
\centering
\myincludegraphics{digging_into_code/binary/16bit_array.png}
\caption{FAR: tableau de valeurs 16-bit}
\end{figure}

J'ai utilisé un fichier contenant un signal 12-canaux numérisé en utilisant 16-bit \ac{ADC}.

\clearpage
\myindex{MIPS}
\par Et voici un exemple de code MIPS très typique.

Comme nous pouvons nous en souvenir, chaque instruction MIPS (et aussi ARM en mode
ARM ou ARM64) a une taille de 32 bits (ou 4 octets), donc un tel code est un tableau
de valeurs 32-bit.

En regardant cette copie d'écran, nous voyons des sortes de schémas.

Les lignes rouge verticales ont été ajoutées pour la clarté:

\begin{figure}[H]
\centering
\myincludegraphics{digging_into_code/binary/typical_MIPS_code.png}
\caption{Hiew: code MIPS très typique}
\end{figure}

Il y a un autre exemple de tel schéma ici dans le livre:
\myref{Oracle_SYM_files_example}.

\clearpage
\subsubsection{Fichiers clairsemés}

Ceci est un fichier clairsemé avec des données éparpillées dans un fichier presque vide.
Chaque caractère espace est en fait l'octet zéro (qui rend comme un espace).
Ceci est un fichier pour programmer des FPGA (Altera Stratix GX device).
Bien sûr, de tels fichiers peuvent être compressés facilement, mais des formats comme
celui-ci sont très populaire dans les logiciels scientifiques et d'ingénierie, où
l'efficience des accès est importante, tandis que la compacité ne l'est pas.

\begin{figure}[H]
\centering
\myincludegraphics{digging_into_code/binary/sparse_FPGA.png}
\caption{FAR: Fichier clairsemé}
\end{figure}

\clearpage
\subsubsection{Fichiers compressés}

% FIXME \ref{} ->
Ce fichier est juste une archive compressée.
Il a une entropie relativement haute et visuellement, il à l'air chaotique.
Ceci est ce à quoi ressemble les fichiers compressés et/ou chiffrés.

\begin{figure}[H]
\centering
\myincludegraphics{digging_into_code/binary/compressed.png}
\caption{FAR: Fichier compressé}
\end{figure}

\clearpage
\subsubsection{\ac{CDFS}}

Les fichiers d'installation d'un \ac{OS} sont en général distribués sous forme de
fichiers ISO, qui sont des copies de disques CD/DVD.
Le système de fichiers utilisé est appelé \ac{CDFS}, ce que vous voyez ici sont des
noms de fichiers mixés avec des données additionnelles.
Ceci peut-être la taille des fichiers, des pointeurs sur d'autres répertoires, des
attributs de fichier, etc.
C'est l'aspect typique de ce à quoi ressemble un système de fichiers en interne.

\begin{figure}[H]
\centering
\myincludegraphics{digging_into_code/binary/cdfs.png}
\caption{FAR: Fichier ISO: \ac{CD} d'installation d'Ubuntu 15}
\end{figure}

\clearpage
\subsubsection{Code exécutable x86 32-bit}

Voici l'allure de code exécutable x86 32-bit.
Il n'a pas une grande entropie, car certains octets reviennent plus souvent que d'autres.

\begin{figure}[H]
\centering
\myincludegraphics{digging_into_code/binary/x86_32.png}
\caption{FAR: Code exécutable x86 32-bit}
\end{figure}

% TODO: Read more about x86 statistics: \ref{}. % FIXME blog post about decryption...

\clearpage
\subsubsection{Fichiers graphique BMP}

% TODO: bitmap, bit, group of bits...

Les fichiers BMP ne sont pas compressés, donc chaque octet (ou groupe d'octet) représente
chaque pixel.
J'ai trouvé cette image quelque part dans mon installation de Windows 8.1:

\begin{figure}[H]
\centering
\myincludegraphicsSmall{digging_into_code/binary/bmp.png}
\caption{Image exemple}
\end{figure}

Vous voyez que cette image a des pixels qui ne doivent pas pouvoir être compressés
beaucoup (autour du centre), mais il y a de longues lignes monochromes au haut et
en bas.
En effet, de telles lignes ressemblent à des lignes lorsque l'on regarde le fichier:

\begin{figure}[H]
\centering
\myincludegraphics{digging_into_code/binary/bmp_FAR.png}
\caption{Fragment de fichier BMP}
\end{figure}


% FIXME comparison!
\subsection{Comparer des \q{snapshots} mémoire}
\label{snapshots_comparing}

La technique consistant à comparer directement deux états mémoire afin de voir les
changements était souvent utilisée pour tricher avec les jeux sur ordinateurs 8-bit
et pour modifier le fichiers des \q{meilleurs scores}.

Par exemple, si vous avez chargé un jeu sur un ordinateur 8-bit (il n'y a pas beaucoup
de mémoire dedans, mais le jeu utilise en général encore moins de mémoire), et que
vous savez que vous avez maintenant, disons, 100 balles, vous pouvez faire un \q{snapshot}
de toute la mémoire et le sauver quelque part. Puis, vous tirez une fois, le compteur
de balles descend à 99, faites un second \q{snapshot} et puis comparer les deux:
il doit y avoir quelque part un octet qui était à 100 au début, et qui est maintenant
à 99.

En considérant le fait que ces jeux 8-bit étaient souvent écrits en langage d'assemblage
et que de telles variables étaient globales, on peut déterminer avec certitude quelle
adresse en mémoire contenait le compteur de balles. Si vous cherchiez toutes les références
à cette adresse dans le code du jeu désassemblé, il n'était pas très difficile de
trouver un morceau de code \glslink{decrement}{décrémentant} le compteur de balles,
puis d'y écrire une, ou plusieurs, instruction \gls{NOP}, et d'avoir un jeu avec
toujours 100 balles.
\myindex{BASIC!POKE}
Les jeux sur ces ordinateurs 8-bit étaient en général chargés à une adresse constante,
aussi, il n'y avait pas beaucoup de versions ce chaque jeu (souvent, une seule version
était répandue pour un long moment), donc les joueurs enthousiastes savaient à quelles
adresses se trouvaient les octets devaient être modifiés (en utilisant l'instruction
BASIC \gls{POKE}) pour le bidouiller. Ceci à conduit à des listes de \q{cheat} qui
contenaient les instructions \gls{POKE} publiées dans des magazines relatifs aux
jeux 8-bit. Voir aussi: \href{http://go.yurichev.com/17114}{Wikipédia}.

\myindex{MS-DOS}

De même, il est facile de modifier le fichier des \q{meilleurs scores}, ceci ne fonctionne
pas seulement avec des jeux 8-bit. Notez votre score et sauvez le fichier quelque part.
Lorsque le décompte des \q{meilleurs scores} devient différent, comparez juste les
deux fichiers, ça peut même être fait avec l'utilitaire DOS FC\footnote{Utilitaire
MS-DOS pour comparer des fichiers binaires.} (les fichiers des \q{meilleurs scores}
sont souvent au format binaire).

Il y aura un endroit où quelques octets seront différents et il est facile de voir
lesquels contiennent le score.
Toutefois, les développeurs de jeux étaient conscient de ces trucs et pouvaient protéger
le programme contre ça.

Exemple quelque peu similaire dans ce livre: \myref{Millenium_DOS_game}.

% TODO: пример с какой-то простой игрушкой?

\subsubsection{Une histoire vraie de 1999}

\myindex{ICQ}
C'était un temps de l'engouement pour la messagerie ICQ, au moins dans les pays de
l'ex-URSS.
Cette messagerie avait une particularité --- certains utilisateurs ne voulaient pas
partager leur état en ligne avec tout le monde.
Et vous deviez demander une \emph{autorisation} à cet utilisateur.
Il pouvait vous autoriser à voir son état, ou pas.

Voici ce que j'ai fait:

\begin{itemize}
\item Ajouté un utilisateur.
\item Un utiliseur est apparu dans la liste de contact, dans la section ``attente d'autorisation''.
\item Fermé ICQ.
\item Sauvegardé la base de données ICQ.
\item Ouvert à nouveau ICQ.
\item L'utilisateur m'a \emph{autorisé}.
\item Refermé ICQ et comparé les deux base de données.
\end{itemize}

Il s'est avéré que: les deux bases de données ne différaient que d'un octet.
Dans la première version: \verb|RESU\x03|, dans la seconde: \verb|RESU\x02|.
(``RESU'', signifie probablement ``USER'', i.e., un entête d'une structure où toutes
les informations à propos d'un utilisateur étaient stockées.)
Cela signifie que l'information sur l'autorisation n'était pas stockée sur le serveur,
mais sur le client. Vraisemblablement, la valeur 2/3 reflétait l'état de l'\q{autorisation}.

\subsubsection{Registres de Windows}

Il est aussi possible de comparer les registres de Windows avant et après l'installation
d'un programme.

C'est une méthode courante  que de trouver quels sont les éléments des registres
utilisés par le programme. Peut-être que ceci est la raison pour laquelle le shareware
de \q{nettoyage des registres windows} est si apprécié.

À propos, voici comment sauver les registres de Windows dans des fichiers texte:

\begin{lstlisting}
reg export HKLM HKLM.reg
reg export HKCU HKCU.reg
reg export HKCR HKCR.reg
reg export HKU HKU.reg
reg export HKCC HKCC.reg
\end{lstlisting}

\myindex{UNIX!diff}
Ils peuvent être comparés en utilisant diff...

\subsubsection{Comparateur à clignotement}

La comparaison de fichiers ou d'images mémoire nous rappelle le comparateur à clignotement
\footnote{\url{http://go.yurichev.com/17348}}:
Un dispositif utilisé autrefois par les astronomes pour trouver les objets célestes
changeant de position.

Les comparateurs à clignotement permet d'alterner rapidement entre deux photographies
prisent à des moments différents, de façon à faire apparaître les différences visuellement.

À propos, Pluton a été découverte avec un comparateur à clignotement en 1930.

\mysection{Détection de l'\ac{ISA}}
\label{ISA_detect}

Souvent, vous avez à faire à un binaire avec un \ac{ISA} inconnu.
Peut-être que la manière la plus facile de détecter l'\ac{ISA} est d'en essayer plusieurs
dans \IDA, objdump ou un autre désassembleur.

Pour réussir ceci, il faut comprendre la différence entre du code incorrectement
et celui correctement désassemblé.

% subsection:
\renewcommand{\CURPATH}{digging_into_code/incorrect_disassembly}
\subsection{Code mal désassemblé}
\label{sec:incorrectly_disasmed_code}

Un rétro ingénieur pratiquant a souvent à faire avec du code mal désassemblé.

\subsubsection{Désassemblage depuis une adresse de début incorrecte (x86)}

Contrairement à ARM et MIPS (où toute instruction a une longueur de 2 ou 4 octets),
les instructions x86 ont une taille variable, donc tout désassembleur démarrant à
une mauvaise adresse qui se trouve au milieu d'une instruction x86 pourra produire
un résultat incorrect.

À titre d'exemple:

\lstinputlisting[style=customasmx86]{\CURPATH/x86_wrong_start_FR.asm}

Il y a des instructions incorrectement désassemblées au début, mais finalement le
désassembleur revient sur la bonne voie.

\subsubsection{À quoi ressemble du bruit aléatoire désassemblé?}

Des propriétés répandues qui peuvent être repérées facilement sont:

\begin{itemize}
\item Dispersion d'instructions inhabituellement grande.
\myindex{x86!\Instructions!PUSH}
\myindex{x86!\Instructions!MOV}
\myindex{x86!\Instructions!CALL}
\myindex{x86!\Instructions!IN}
\myindex{x86!\Instructions!OUT}
Les instructions x86 les plus fréquentes sont \PUSH{}, \MOV{}, \CALL{}, mais ici nous
voyons des instructions de tous les groupes d'instructions: \ac{FPU}, \INS{IN}/\INS{OUT},
instructions systèmes et rares.

\item Valeurs grandes et aléatoires, d'offsets et immédiates.

\item Sauts ayant des offsets incorrects, sautant au milieu d'autres instructions
\end{itemize}

\lstinputlisting[caption=\randomNoise{} (x86),style=customasmx86]{\CURPATH/x86.asm}

\myindex{x86-64}
\lstinputlisting[caption=\randomNoise{} (x86-64),style=customasmx86]{\CURPATH/x64.asm}

\myindex{ARM}
\lstinputlisting[caption=\randomNoise{} (ARM (\ARMMode)),style=customasmARM]{\CURPATH/ARM.asm}

\lstinputlisting[caption=\randomNoise{} (ARM (\ThumbMode)),style=customasmARM]{\CURPATH/ARM_thumb.asm}

\myindex{MIPS}
\lstinputlisting[caption=\randomNoise{} (MIPS little endian),style=customasmMIPS]{\CURPATH/MIPS.asm}

Il est important de garder à l'esprit que du code de dépaquetage et de déchiffrement
construit intelligemment (y compris auto-modifiant) peut avoir l'air aléatoire, mais
s'exécute toujours correctement.
% TODO таких примеров тоже бы добавить



\subsection{Code désassemblé correctement}
\label{correctly_disasmed_code}

Chaque \ac{ISA} a une douzaine d'instructions les plus utilisées, toutes les autres
le sont beaucoup moins souvent.

Concernant le x86, il est intéressant de savoir le fait que les instructions d'appel
de fonctions (\PUSH/\CALL/\ADD) et \MOV sont les morceaux de code les plus fréquemment
exécutées dans presque tous les programmes que nous utilisons.
Autrement dit, le \ac{CPU} est très occupé à passer de l'information entre les niveaux
d'abstraction, ou, on peut dire qu'il est très occupé à commuter entre ces niveaux.
Indépendamment du type d'\ac{ISA}.
Ceci a un coût de diviser les problèmes entre plusieurs niveaux d'abstraction (ainsi
les êtres humain peuvent travailler plus facilement avec).



\mysection{Autres choses}

\subsection{Idée générale}

Un rétro-ingénieur doit essayer se se mettre dans la peau d'un programmeur aussi
souvent que possible.
Pour adopter son point de vue et se demander comment il aurait résolu des tâches
d'un cas spécifique.

\subsection{Ordre des fonctions dans le code binaire}

Toutes les fonctions situées dans un unique fichier .c ou .cpp sont compilées dans
le fichier objet (.o) correspondant.
Plus tard, l'éditeur de liens mets tous les fichiers dont il a besoin ensemble, sans
changer l'ordre ni les fonctions.
Par conséquent, si vous voyez deux ou plus fonctions consécutives, cela signifie
qu'elles étaient situées dans le même fichier source (à moins que vous ne soyez en
limite de deux fichiers objet, bien sûr).
Ceci signifie que ces fonctions ont quelque chose en commun, qu'elles sont des fonctions
du même niveau d'\ac{API}, de la même bibliothèque, etc.

\myindex{CryptoPP}
Ceci est une histoire vraie de pratique: il était une fois, alors que je cherchais
des fonctions relatives à Twofish dans un programme lié à la bibliothèque CryptoPP,
en particulier des fonctions de chiffrement/déchiffrement.\\
J'ai trouvé la fonction \verb|Twofish::Base::UncheckedSetKey()| mais pas d'autres.
Après avoir cherché dans le code source
\verb|twofish.cpp|\footnote{\url{https://github.com/weidai11/cryptopp/blob/b613522794a7633aa2bd81932a98a0b0a51bc04f/twofish.cpp}},
il devint clair que toutes les fonctions étaient situées dans ce module (\verb|twofish.cpp|).\\
Donc j'ai essayé toutes les fonctions qui suivaient \verb|Twofish::Base::UncheckedSetKey()|---comme elles arrivaient,\\
une a été \verb|Twofish::Enc::ProcessAndXorBlock()|, une autre---\verb|Twofish::Dec::ProcessAndXorBlock()|.

\subsection{Fonctions minuscules}

Les fonctions minuscules comme les fonctions vides (\myref{empty_func})
ou les fonctions qui renvoient juste ``true'' (1) ou ``false'' (0) (\myref{ret_val_func})
sont très communes, et presque tous les compilateurs corrects tendent à ne mettre
qu'une seule fonction de ce genre dans le code de l'exécutable résultant, même si
il y avait plusieurs fonctions similaires dans le code source.
Donc, à chaque fois que vous voyez une fonction minuscule consistant seulement en
\TT{mov eax, 1 / ret} qui est référencée (et peut être appelée) dans plusieurs endroits
qui ne semblent pas reliés les uns au autres, ceci peut résulter d'une telle optimisation.%

\subsection{\Cpp}

Les données \ac{RTTI}~(\myref{RTTI})- peuvent être utiles pour l'identification des
classes \Cpp.

%\subsection{Crash on purpose}
\subsection{Crash délibéré}

Souvent, vous voulez savoir quelle fonction a été exécutée, et laquelle ne l'a pas
été.
Vous pouvez utiliser un débogueur, mais sur des architectures exotiques, il peut
ne pas en avoir, donc la façon la plus simple est d'y mettre un opcode invalide,
ou quelque chose comme \INS{INT3} (0xCC).
Le crash signalera le fait que l'instruction a été exécutée.

Un autre exemple de crash délibéré: \myref{dmalloc_KILL_PROCESS}.

