\chapter{Trouver des choses importantes/intéressantes dans le code}

Le minimalisme n'est pas une caractéristique prépondérante des logiciels modernes.

\myindex{\Cpp!STL}

Pas parce que les programmeurs écrivent beaucoup, mais parce que de nombreuses bibliothèques
sont couramment liées statiquement aux fichiers exécutable.
Si toutes les bibliothèques externes étaient déplacées dans des fichiers DLL externes,
le monde serait différent. (Une autre raison pour C++ sont la \ac{STL} et autres
bibliothèques templates.)

\newcommand{\FOOTNOTEBOOST}{\footnote{\url{http://go.yurichev.com/17036}}}
\newcommand{\FOOTNOTELIBPNG}{\footnote{\url{http://go.yurichev.com/17037}}}

Ainsi, il est très important de déterminer l'origine de la fonction, si elle provient
d'une bibliothèque standard ou d'une bibliothèque bien connue (comme Boost\FOOTNOTEBOOST,
libpng\FOOTNOTELIBPNG), ou si elle est liée à ce que l'on essaye de trouver dans
le code.

Il est simplement absurde de tout récrire le code en \CCpp pour trouver ce que l'on
cherche.

Une des premières tâches d'un rétro-ingénieur est de trouver rapidement le code dont
il a besoin.

\myindex{\GrepUsage}

Le dés-assembleur \IDA nous permet de chercher parmi les chaînes de texte, les séquences
d'octets et les constantes.
Il est même possible d'exporter le code dans un fichier texte .lst ou .asm et d'utiliser
\TT{grep}, \TT{awk}, etc.

Lorsque vous essayez de comprendre ce que fait un certain code, ceci peut être facile
avec une bibliothèque open-source comme libpng.
Donc, lorsque vous voyez certaines constantes ou chaînes de texte qui vous semblent
familières, il vaut toujours la peine de les \emph{googler}.
Et si vous trouvez le projet open-source où elles sont utilisées, alors il suffit
de comparer les fonctions.
Ceci peut permettre de résoudre certaines parties du problème.

Par exemple, si un programme utilise des fichiers XML, la premières étape peut-être
de déterminer quelle bibliothèque XML est utilisée pour le traitement, puisque les
bibliothèques standards (ou bien connues) sont en général utilisées au lieu de code
fait maison.

\myindex{SAP}
\myindex{Windows!PDB}

Par exemple, j'ai essayé une fois de comprendre comment la compression/décompression
des paquets réseau fonctionne dans SAP 6.0.
C'est un logiciel gigantesque, mais un .\gls{PDB} détaillé avec des informations
de débogage est présent, et c'est pratique.
J'en suis finalement arrivé à l'idée que l'une des fonctions, qui était appelée par
\emph{CsDecomprLZC}, effectuait la décompression des paquets réseau.
Immédiatement, j'ai essayé de googler le nom et rapidement trouvé que la fonction
était utilisée dans MaxDB (c'est un projet open-source de SAP)
\footnote{Plus sur ce sujet dans la section concernée~(\myref{sec:SAPGUI})}.

\url{http://www.google.com/search?q=CsDecomprLZC}

Étonnement, les logiciels MaxDB et SAP 6.0 partagent du code comme ceci pour la compression/
décompression des paquets réseau.

\mysection{Identification de fichiers exécutables}

\subsection{Microsoft Visual C++}
\label{MSVC_versions}

Les versions de MSVC et des DLLs peuvent être importées:

%\small
\begin{center}
\begin{tabular}{ | l | l | l | l | l | }
\hline
\HeaderColor Marketing ver. &
\HeaderColor Internal ver. &
\HeaderColor CL.EXE ver. &
\HeaderColor DLLs imported &
\HeaderColor Release date \\
\hline
% 4.0, April 1995
% 97 & 5.0 & February 1997
6		&  6.0	& 12.00	& msvcrt.dll	& June 1998		\\
		&	&	& msvcp60.dll	&			\\
\hline
.NET (2002)	&  7.0	& 13.00	& msvcr70.dll	& February 13, 2002	\\
		&	&	& msvcp70.dll	&			\\
\hline
.NET 2003	&  7.1	& 13.10 & msvcr71.dll	& April 24, 2003	\\
		&	&	& msvcp71.dll	&			\\
\hline
2005		&  8.0	& 14.00 & msvcr80.dll	& November 7, 2005	\\
		&	&	& msvcp80.dll	&			\\
\hline
2008		&  9.0	& 15.00 & msvcr90.dll	& November 19, 2007	\\
		&	&	& msvcp90.dll	&			\\
\hline
2010		& 10.0	& 16.00 & msvcr100.dll	& April 12, 2010 	\\
		&	&	& msvcp100.dll	&			\\
\hline
2012		& 11.0	& 17.00 & msvcr110.dll	& September 12, 2012 	\\
		&	&	& msvcp110.dll	&			\\
\hline
2013		& 12.0	& 18.00 & msvcr120.dll	& October 17, 2013 	\\
		&	&	& msvcp120.dll	&			\\
\hline
\end{tabular}
\end{center}
%\normalsize

msvcp*.dll contient des fonctions relatives à \Cpp{}, donc si elle est importées,
il s'agit probablement d'un programme \Cpp.

\subsubsection{Mangling de nom}

Les noms commencent en général par le symbole \TT{?}.

Vous trouverez plus d'informations le \glslink{name mangling}{mangling de nom} de
MSVC ici: \myref{namemangling}.

\subsection{GCC}
\myindex{GCC}

À part les cibles *NIX, GCC est aussi présent dans l'environnement win32, sous la
forme de Cygwin et MinGW.

\subsubsection{Mangling de nom}

Les noms commencent en général par le symbole \TT{\_Z}.
Vous trouverez plus d'informations le \glslink{name mangling}{mangling de nom} de
GCC ici: \myref{namemangling}.
\subsubsection{Cygwin}
\myindex{Cygwin}

cygwin1.dll est souvent importée.

\subsubsection{MinGW}
\myindex{MinGW}

msvcrt.dll peut être importée.

\subsection{Intel Fortran}
\myindex{Fortran}

libifcoremd.dll, libifportmd.dll et libiomp5md.dll (support OpenMP) peuvent être importées.

libifcoremd.dll a beaucoup de fonctions préfixées par \TT{for\_}, qui signifie \emph{Fortran}.

\subsection{Watcom, OpenWatcom}
\myindex{Watcom}
\myindex{OpenWatcom}

\subsubsection{Mangling de nom}

Les noms commencent usuellement par le symbole \TT{W}.

Par exemple, ceci est la façon dont la méthode nommées \q{method} de la classe \q{class}
qui n'a pas d'argument et qui renvoie \Tvoid est encodée:

\begin{lstlisting}
W?method$_class$n__v
\end{lstlisting}

\subsection{Borland}
\myindex{Borland Delphi}
\myindex{Borland C++Builder}

Voici un exemple de \glslink{name mangling}{mangling de nom} de Delphi de Borland
et de C++Builder:

\lstinputlisting{digging_into_code/identification/borland_mangling.txt}

Les noms commencent toujours avec le symbole \TT{@}, puis nous avons le nom de la
classe, de la méthode et les types des arguments de méthode encodés.

Ces noms peuvent être dans des imports .exe, des exports .dll, des données de débogage,
etc.

Les Borland Visual Component Libraries (VCL) sont stockées dans des fichiers .bpl
au lieu de .dll, par exemple, vcl50.dll, rtl60.dll.

Une autre DLL qui peut être importée: BORLNDMM.DLL.

\subsubsection{Delphi}

Presque tous les exécutables Delpi ont la chaîne de texte \q{Boolean} au début de
leur segment de code, ainsi que d'autres noms de type.

Ceci est le début très typique du segment \TT{CODE} d'un programme Delphi, ce bloc
vient juste après l'entête de fichier win32 PE:

\lstinputlisting{digging_into_code/identification/delphi.txt}

Les 4 premiers octets du segment de données (\TT{DATA}) peuvent être \TT{00 00 00 00},
\TT{32 13 8B C0} ou \TT{FF FF FF FF}.%

Cette information peut être utile lorsque l'on fait face à des exécutables Delphi
préparés/chiffrés.

\subsection{Autres DLLs connues}

\myindex{OpenMP}
\begin{itemize}
\item vcomp*.dll---implémentation d'OpenMP de Microsoft.
\end{itemize}


% binary files might be also here

\mysection{Communication avec le monde extérieur (niveau fonction)}
Il est souvent recommandé de suivre les arguments de la fonction et sa valeur de
retour dans un débogueur ou \ac{DBI}.
Par exemple, l'auteur a essayé une fois de comprendre la signification d'une fonction
obscure, qui s'est avérée être un tri à bulles mal implémenté\footnote{\url{https://yurichev.com/blog/weird_sort/}}.
(Il fonctionnait correctement, mais plus lentement.)
En même temps, regarder les entrées et sorties de cette fonction aide instantanément
à comprendre ce quelle fait.

Souvent, lorsque vous voyez une division par la multiplication (\myref{sec:divisionbymult}),
mais avez oublié tous les détails du mécanisme, vous pouvez seulement observer l'entrée
et la sortie, et trouver le diviseur rapidement.

% sections:
\mysection{Communication avec le monde extérieur (win32)}

Parfois, il est suffisant d'observer les entrées/sorties d'une fonction pour comprendre
ce qu'elle fait.
Ainsi, vous pouvez gagner du temps.

Accès aux fichiers et au registre:
pour les analyses très basiques, l'utilitaire, Process Monitor\footnote{\url{http://technet.microsoft.com/en-us/sysinternals/bb896645.aspx}}
de SysInternals peut aider.

Pour l'analyse basique des accès au réseau, Wireshark\footnote{\url{http://www.wireshark.org/}}
peut être utile.

Mais vous devrez de toutes façons regarder à l'intérieur, \\
\\
Les premières choses à chercher sont les fonctions des \ac{API}s de l'\ac{OS} et
des bibliothèques standards qui sont utilisées.

Si le programme est divisé en un fichier exécutable et un groupe de fichiers DLL,
parfois le nom des fonctions dans ces DLLs peut aider.

Si nous sommes intéressés par exactement ce qui peut conduire à appeler \TT{MessageBox()}
avec un texte spécifique, nous pouvons essayer de trouver ce texte dans le segment
de données, trouver sa référence et trouver les points depuis lesquels le contrôle
peut être passé à l'appel à \TT{MessageBox()} qui nous intéresse.

\myindex{\CStandardLibrary!rand()}
Si nous parlons d'un jeu vidéo et que nous sommes intéressés par les évènements qui
y sont plus ou moins aléatoires, nous pouvons essayer de trouver la fonction \rand
ou sa remplaçante (comme l'algorithme du twister de Mersenne) et trouver les points
depuis lesquels ces fonctions sont appelées, et plus important, comment les résultats
sont utilisés.
% BUG in varioref: http://tex.stackexchange.com/questions/104261/varioref-vref-or-vpageref-at-page-boundary-may-loop
Un exemple: \myref{chap:color_lines}.

Mais si ce n'est pas un jeu, et que \rand est toujours utilisé, il est intéressant
de savoir pourquoi.
Il a y des cas d'utilisation inattendu de \rand dans des algorithmes de compression
de données (pour une imitation du chiffrement):
\href{http://blog.yurichev.com/node/44}{blog.yurichev.com}.

\subsection{Fonctions souvent utilisées dans l'API Windows}

Ces fonctions peuvent être parmi les fonctions importées.
Il est utile de noter que toutes les fonctions ne sont pas forcément utilisées dans
du code écrit par le programmeur.
Beaucoup de fonctions peuvent être appelées depuis des fonctions de bibliothèque
et du code \ac{CRT}.

Certaines fonctions peuvent avoir le suffixe \GTT{-A} pour la version ASCII et \GTT{-W}
pour la version Unicode.

\begin{itemize}

\item
Accès au registre (advapi32.dll): 
RegEnumKeyEx, RegEnumValue, RegGetValue, RegOpenKeyEx, RegQueryValueEx.

\item
Accès au text des fichiers .ini (kernel32.dll):
GetPrivateProfileString.

\item
Boites de dialogue (user32.dll):
MessageBox, MessageBoxEx, CreateDialog, SetDlgItemText, GetDlgItemText.

\item
Accès aux resources (\myref{PEresources}): (user32.dll): LoadMenu.

\item
Réseau TCP/IP (ws2\_32.dll):
WSARecv, WSASend.

\item
Accès fichier (kernel32.dll):
CreateFile, ReadFile, ReadFileEx, WriteFile, WriteFileEx.

\item
Accès haut niveau à Internet (wininet.dll):
WinHttpOpen.

\item
Vérifier la signature digitale d'uin fichier exécutable (wintrust.dll):
WinVerifyTrust.

\item
La bibliothèque MSVC standard (si elle est liée dynamiquement) (msvcr*.dll):
assert, itoa, ltoa, open, printf, read, strcmp, atol, atoi, fopen, fread, fwrite, memcmp, rand,
strlen, strstr, strchr.

\end{itemize}

\subsection{Étendre la période d'essai}

Les fonctions d'accès au registre sont des cibles fréquentes pour ceux qui veulent
essayer de craquer des logiciels avec période d'essai, qui peuvent sauvegarder la
date et l'heure dans un registre.

Des autres cibles courantes sont les fonctions GetLocalTime() et GetSystemTime():
un logiciel avec période d'essai, à chaque démarrage, doit de toutes façons vérifier
la date et l'heure d'une certaine façon.

\subsection{Supprimer la boite de dialogue nag}

Une manière répandue de trouver ce qui cause l'apparition de la boite de dialogue
nag est d'intercepter les fonctions MessageBox(), CreateDialog() et CreateWindow().

\subsection{tracer: Intercepter toutes les fonctions dans un module spécifique}
\myindex{tracer}

\myindex{x86!\Instructions!INT3}
Il y a un point d'arrêt INT3 dans \tracer, qui peut être déclenché seulement une
fois, toutefois, il peut être mis pour toutes les fonctions dans une DLL spécifique.

\begin{lstlisting}
--one-time-INT3-bp:somedll.dll!.*
\end{lstlisting}

Ou, mettons un point d'arrêt INT3 sur toutes les fonctions avec le préfixe \TT{xml}
dans leur nom:

\begin{lstlisting}
--one-time-INT3-bp:somedll.dll!xml.*
\end{lstlisting}

Le revers de la médaille est que de tels points d'arrêt ne sont déclenchés qu'une fois.
Tracer montrera l'appel à une fonction, s'il se produit, mais seulement une fois.
Un autre inconvénient---il est impossible de voir les arguments de la fonction.

Néanmoins, cette fonctionnalité est très utile lorsque vous avez qu'un programme
utilise une DLL, mais que vous ne savez pas quelles fonctions sont effectivement
utilisées.
Et il y a beaucoup de fonctions.

\par
\myindex{Cygwin}
Par exemple, regardons ce qu'utilise l'utilitaire uptime de Cygwin:

\begin{lstlisting}
tracer -l:uptime.exe --one-time-INT3-bp:cygwin1.dll!.*
\end{lstlisting}

Ainsi nous pouvons voir quelles sont les fonctions de la bibliothèque cygwin1.dll
qui sont appelées au moins une fois, et depuis où:

\lstinputlisting{digging_into_code/uptime_cygwin.txt}


%% TBT \mysection{Strings}
\label{sec:digging_strings}

\subsection{Text strings}

\subsubsection{\CCpp}

\label{C_strings}
The normal C strings are zero-terminated (\ac{ASCIIZ}-strings).

The reason why the C string format is as it is (zero-terminated) is apparently historical.
In [Dennis M. Ritchie, \emph{The Evolution of the Unix Time-sharing System}, (1979)]
we read:

\begin{framed}
\begin{quotation}
A minor difference was that the unit of I/O was the word, not the byte, because the PDP-7 was a word-addressed
machine. In practice this meant merely that all programs dealing with character streams ignored null
characters, because null was used to pad a file to an even number of characters.
\end{quotation}
\end{framed}

\myindex{Hiew}

In Hiew or FAR Manager these strings look like this:

\begin{lstlisting}[style=customc]
int main()
{
	printf ("Hello, world!\n");
};
\end{lstlisting}

\begin{figure}[H]
\centering
\includegraphics[width=0.6\textwidth]{digging_into_code/strings/C-string.png}
\caption{Hiew}
\end{figure}

% FIXME видно \n в конце, потом пробел

\subsubsection{Borland Delphi}
\myindex{Pascal}
\myindex{Borland Delphi}

The string in Pascal and Borland Delphi is preceded by an 8-bit or 32-bit string length.

For example:

\begin{lstlisting}[caption=Delphi,style=customasmx86]
CODE:00518AC8                 dd 19h
CODE:00518ACC aLoading___Plea db 'Loading... , please wait.',0

...

CODE:00518AFC                 dd 10h
CODE:00518B00 aPreparingRun__ db 'Preparing run...',0
\end{lstlisting}

\subsubsection{Unicode}

\myindex{Unicode}

Often, what is called Unicode is a methods for encoding strings where each character occupies 2 bytes or 16 bits.
This is a common terminological mistake.
Unicode is a standard for assigning a number to each character in the many writing systems of the 
world, but does not describe the encoding method.

\myindex{UTF-8}
\myindex{UTF-16LE}
The most popular encoding methods are: UTF-8 (is widespread in Internet and *NIX systems) and UTF-16LE (is used in Windows).

\myparagraph{UTF-8}

\myindex{UTF-8}
UTF-8 is one of the most successful methods for
encoding characters.
All Latin symbols are encoded just like in ASCII,
and the symbols beyond the ASCII table are encoded using several bytes.
0 is encoded as
before, so all standard C string functions work with UTF-8 strings just like any other string.

Let's see how the symbols in various languages are encoded in UTF-8 and how it looks like in FAR, using the 437 codepage
\footnote{The example and translations was taken from here: 
\url{http://www.columbia.edu/~fdc/utf8/}}:

\begin{figure}[H]
\centering
\includegraphics[width=0.6\textwidth]{digging_into_code/strings/multilang_sampler.png}
\end{figure}

% FIXME: cut it
\begin{figure}[H]
\centering
\myincludegraphics{digging_into_code/strings/multilang_sampler_UTF8.png}
\caption{FAR: UTF-8}
\end{figure}

As you can see, the English language string looks the same as it is in ASCII.

The Hungarian language uses some Latin symbols plus symbols with diacritic marks.

These symbols are encoded using several bytes, these are underscored with red.
It's the same story with the Icelandic and Polish languages.

There is also the \q{Euro} currency symbol at the start, which is encoded with 3 bytes.

The rest of the writing systems here have no connection with Latin.

At least in Russian, Arabic, Hebrew and Hindi we can see some recurring bytes, and that is not surprise:
all symbols from a writing system are usually located in the same Unicode table, so their code begins with
the same numbers.

At the beginning, before the \q{How much?} string we see 3 bytes, which are in fact the \ac{BOM}.
The \ac{BOM} defines the encoding system to be
used.

\myparagraph{UTF-16LE}

\myindex{UTF-16LE}
\myindex{Windows!Win32}
Many win32 functions in Windows have the suffixes \TT{-A} and \TT{-W}.
The first type of functions works
with normal strings, the other with UTF-16LE strings (\emph{wide}).

In the second case, each symbol is usually stored in a 16-bit value of type \emph{short}.

The Latin symbols in UTF-16 strings look in Hiew or FAR like they are interleaved with zero byte:

\begin{lstlisting}[style=customc]
int wmain()
{
	wprintf (L"Hello, world!\n");
};
\end{lstlisting}

\begin{figure}[H]
\centering
\includegraphics[width=0.6\textwidth]{digging_into_code/strings/UTF16-string.png}
\caption{Hiew}
\end{figure}

We can see this often in \gls{Windows NT} system files:

\begin{figure}[H]
\centering
\includegraphics[width=0.6\textwidth]{digging_into_code/strings/ntoskrnl_UTF16.png}
\caption{Hiew}
\end{figure}

\myindex{IDA}
Strings with characters that occupy exactly 2 bytes are called \q{Unicode} in \IDA:

\begin{lstlisting}[style=customasmx86]
.data:0040E000 aHelloWorld:
.data:0040E000                 unicode 0, <Hello, world!>
.data:0040E000                 dw 0Ah, 0
\end{lstlisting}

Here is how the Russian language string is encoded in UTF-16LE:

\begin{figure}[H]
\centering
\includegraphics[width=0.6\textwidth]{digging_into_code/strings/russian_UTF16.png}
\caption{Hiew: UTF-16LE}
\end{figure}

What we can easily spot is that the symbols are interleaved by the diamond character (which has the ASCII code of 4).
Indeed, the Cyrillic symbols are located in the fourth Unicode plane.
Hence, all Cyrillic symbols in UTF-16LE are located in the \TT{0x400-0x4FF} range.

Let's go back to the example with the string written in multiple languages.
Here is how it looks like in UTF-16LE.

% FIXME: cut it
\begin{figure}[H]
\centering
\myincludegraphics{digging_into_code/strings/multilang_sampler_UTF16.png}
\caption{FAR: UTF-16LE}
\end{figure}

Here we can also see the \ac{BOM} at the beginning.
All Latin characters are interleaved with a zero byte.

Some characters with diacritic marks (Hungarian and Icelandic languages) are also underscored in red.

% subsection:
\input{digging_into_code/strings/base64_EN}



\subsection{Finding strings in binary}

\epigraph{Actually, the best form of Unix documentation is frequently running the
\textbf{strings} command over a program’s object code. Using \textbf{strings}, you can get
a complete list of the program’s hard-coded file name, environment variables,
undocumented options, obscure error messages, and so forth.}{The Unix-Haters Handbook}

\myindex{UNIX!strings}
The standard UNIX \emph{strings} utility is quick-n-dirty way to see strings in file.
For example, these are some strings from OpenSSH 7.2 sshd executable file:

\lstinputlisting{digging_into_code/sshd_strings.txt}

There are options, error messages, file paths, imported dynamic modules and functions, some other strange strings (keys?)
There is also unreadable noise---x86 code sometimes has chunks consisting of printable ASCII characters, up to ~8 characters.

Of course, OpenSSH is open-source program.
But looking at readable strings inside of some unknown binary is often a first step of analysis.
\myindex{UNIX!grep}

\emph{grep} can be applied as well.

\myindex{Hiew}
\myindex{Sysinternals}
Hiew has the same capability (Alt-F6), as well as Sysinternals ProcessMonitor.

\subsection{Error/debug messages}

Debugging messages are very helpful if present.
In some sense, the debugging messages are reporting
what's going on in the program right now. Often these are \printf-like functions,
which write to log-files, or sometimes do not writing anything but the calls are still present 
since the build is not a debug one but \emph{release} one.
\myindex{\oracle}

If local or global variables are dumped in debug messages, it might be helpful as well 
since it is possible to get at least the variable names.
For example, one of such function in \oracle is \TT{ksdwrt()}.

Meaningful text strings are often helpful.
The \IDA disassembler may show from which function and from which point this specific string is used.
Funny cases sometimes happen\footnote{\href{http://go.yurichev.com/17223}{blog.yurichev.com}}.

The error messages may help us as well.
In \oracle, errors are reported using a group of functions.\\
You can read more about them here: \href{http://go.yurichev.com/17224}{blog.yurichev.com}.

\myindex{Error messages}

It is possible to find quickly which functions report errors and in which conditions.

By the way, this is often the reason why copy-protection systems use inarticulate cryptic error messages 
or just error numbers.
No software author is happy if the software cracker can quickly understands copy-protection's inner workings
judging by error messages it can produce.

One example of encrypted error messages is here: \myref{examples_SCO}.

\subsection{Suspicious magic strings}

Some magic strings which are usually used in backdoors look pretty suspicious.

For example, there was a backdoor in the TP-Link WR740 home router\footnote{\url{http://sekurak.pl/tp-link-httptftp-backdoor/}}.
The backdoor can activated using the following URL:\\
\url{http://192.168.0.1/userRpmNatDebugRpm26525557/start_art.html}.\\

Indeed, the \q{userRpmNatDebugRpm26525557} string is present in the firmware.

This string was not googleable until the wide disclosure of information about the backdoor.

You would not find this in any \ac{RFC}.

You would not find any computer science algorithm which uses such strange byte sequences.

And it doesn't look like an error or debugging message.

So it's a good idea to inspect the usage of such weird strings.\\
\\
\myindex{base64}

Sometimes, such strings are encoded using base64.

So it's a good idea to decode them all and to scan them visually, even a glance should be enough.\\
\\
\myindex{Security through obscurity}
More precise, this method of hiding backdoors is called \q{security through obscurity}.


\mysection{Appels à assert()}
\myindex{\CStandardLibrary!assert()}

Parfois, la présence de la macro \TT{assert()} est aussi utile:
En général, cette macro laisse le nom du fichier source, le numéro de ligne et une
condition dans le code.

L'information la plus utile est contenue dans la condition d'assert, nous pouvons
en déduire les noms de variables ou les noms de champ de la structure. Les autres
informations utiles sont les noms de fichier---nous pouvons essayer d'en déduire
le type de code dont il s'agit ici.
Il est aussi possible de reconnaître les bibliothèques open-source connues d'après
les noms de fichier.

\lstinputlisting[caption=Exemple d'appels à assert() informatifs,style=customasmx86]{digging_into_code/assert_examples.lst}

Il est recommandé de \q{googler} à la fois les conditions et les noms de fichier,
qui peuvent nous conduire à une bibliothèque open-source.
Par exemple, si nous \q{googlons} \q{sp->lzw\_nbits <= BITS\_MAX}, cela va comme
prévu nous donner du code open-source relatif à la compression LZW.

\mysection{Constantes}

Les humains, programmeurs inclus, utilisent souvent des nombres ronds, comme 10, 100,
1000, dans la vie courante comme dans le code.

Le rétro ingénieur pratiquant connaît en général bien leur représentation décimale:
10=0xA, 100=0x64, 1000=0x3E8, 10000=0x2710.

Les constantes \TT{0xAAAAAAAA} (0b10101010101010101010101010101010) et \\
\TT{0x55555555} (0b01010101010101010101010101010101)  sont aussi répandues---elles
sont composées d'alternance de bits.

Cela peut aider à distinguer un signal d'un signal dans lequel tous les bits sont
à 1 (0b1111 \dots) ou à 0 (0b0000 \dots).
Par exemple, la constante \TT{0x55AA} est utilisée au moins dans le secteur de boot,
\ac{MBR}, et dans la \ac{ROM} de cartes d'extention de compatible IBM.

Certains algorithmes, particulièrement ceux de chiffrement, utilisent des constantes
distinctes, qui sont faciles à trouver dans le code en utilisant \IDA.

\myindex{MD5}
\newcommand{\URLMD}{http://go.yurichev.com/17111}

Par exemple, l'algorithme MD5\footnote{\href{\URLMD}{Wikipédia}} initialise ses propres
variables internes comme ceci:

\begin{verbatim}
var int h0 := 0x67452301
var int h1 := 0xEFCDAB89
var int h2 := 0x98BADCFE
var int h3 := 0x10325476
\end{verbatim}

Si vous trouvez ces quatre constantes utilisées à la suite dans du code, il est
très probable que cette fonction soit relatives à MD5.

\par Un autre exemple sont les algorithmes CRC16/CRC32, ces algorithmes de calcul
utilisent souvent des tables pré-calculées comme celle-ci:

\begin{lstlisting}[caption=linux/lib/crc16.c,style=customc]
/** CRC table for the CRC-16. The poly is 0x8005 (x^16 + x^15 + x^2 + 1) */
u16 const crc16_table[256] = {
	0x0000, 0xC0C1, 0xC181, 0x0140, 0xC301, 0x03C0, 0x0280, 0xC241,
	0xC601, 0x06C0, 0x0780, 0xC741, 0x0500, 0xC5C1, 0xC481, 0x0440,
	0xCC01, 0x0CC0, 0x0D80, 0xCD41, 0x0F00, 0xCFC1, 0xCE81, 0x0E40,
	...
\end{lstlisting}

Voir aussi la table pré-calculée pour CRC32: \myref{sec:CRC32}.

Dans les algorithmes CRC sans table, des polynômes bien connus sont utilisés, par
exemple 0xEDB88320 pour CRC32.

\subsection{Nombres magiques}
\label{magic_numbers}

\newcommand{\FNURLMAGIC}{\footnote{\href{http://go.yurichev.com/17112}{Wikipédia}}}

De nombreux formats de fichier définissent un entête standard où un \emph{nombre(s) magique}\FNURLMAGIC{}
est utilisé, unique ou même plusieurs.

\myindex{MS-DOS}

Par exemple, tous les exécutables Win32 et MS-DOS débutent par ces deux caractères \q{MZ}\footnote{\href{http://go.yurichev.com/17113}{Wikipédia}}.

\myindex{MIDI}

Au début d'un fichier MIDI, la signature \q{MThd} doit être présente.
Si nous avons un programme qui utilise des fichiers MIDI pour quelque chose, il
est très probable qu'il doit vérifier la validité du fichier en testant au moins
les 4 premiers octets.

Ça peut être fait comme ceci:
(\emph{buf} pointe sur le début du fichier chargé en mémoire)

\begin{lstlisting}[style=customasmx86]
cmp [buf], 0x6468544D ; "MThd"
jnz _error_not_a_MIDI_file
\end{lstlisting}

\myindex{\CStandardLibrary!memcmp()}
\myindex{x86!\Instructions!CMPSB}

\dots ou en appelant une fonction pour comparer des blocs de mémoire comme \TT{memcmp()}
ou tout autre code équivalent jusqu'à une instruction \TT{CMPSB} (\myref{REPE_CMPSx}).

Lorsque vous trouvez un tel point, vous pouvez déjà dire que le chargement du fichier
MIDI commence, ainsi, vous pouvez voir l'endroit où se trouve le buffer avec le contenu
du fichier MIDI, ce qui est utilisé dans le buffer et comment.

\subsubsection{Dates}

\myindex{UFS2}
\myindex{FreeBSD}
\myindex{HASP}

Souvent, on peut rencontrer des nombres comme \TT{0x19870116}, qui ressemble clairement
à une date (année 1987, 1er mois (janvier), 16ème jour).
Ça peut être la date de naissance de quelqu'un (un programmeur, une de ses relations,
un enfant), ou une autre date importante.
La date peut aussi être écrite dans l'ordre inverse, comme \TT{0x16011987}.
Les dates au format américain sont aussi courante, comme \TT{0x01161987}.

Un exemple célèbre est \TT{0x19540119} (nombre magique utilisé dans la structure
du super-bloc UFS2), qui est la date de naissance de Marshall Kirk McKusick, éminent
contributeur FreeBSD.

\myindex{Stuxnet}
Stuxnet utilise le nombre ``19790509'' (pas comme un nombre 32-bit, mais comme une
chaîne, toutefois), et ça a conduit à spéculer que le malware était relié à Israël%
\footnote{C'est la date d'exécution de Habib Elghanian, juif persan.}.

Aussi, des nombre comme ceux-ci sont très répandus dans dans le chiffrement niveau
amateur, par exemple, extrait de la \emph{fonction secrète} des entrailles du dongle
HASP3\footnote{\url{https://web.archive.org/web/20160311231616/http://www.woodmann.com/fravia/bayu3.htm}}:

\begin{lstlisting}[style=customc]
void xor_pwd(void) 
{ 
	int i; 
	
	pwd^=0x09071966;
	for(i=0;i<8;i++) 
	{ 
		al_buf[i]= pwd & 7; pwd = pwd >> 3; 
	} 
};

void emulate_func2(unsigned short seed)
{ 
	int i, j; 
	for(i=0;i<8;i++) 
	{ 
		ch[i] = 0; 
		
		for(j=0;j<8;j++)
		{ 
			seed *= 0x1989; 
			seed += 5; 
			ch[i] |= (tab[(seed>>9)&0x3f]) << (7-j); 
		}
	} 
}
\end{lstlisting}

\subsubsection{DHCP}

Ceci s'applique aussi aux protocoles réseaux.
Par exemple, les paquets réseau du protocole DHCP contiennent un soi-disant \emph{nombre
magique}: \TT{0x63538263}.
Tout code qui génère des paquets DHCP doit contenir quelque part cette constante
à insérer dans les paquets.
Si nous la trouvons dans du code, nous pouvons trouver ce qui s'y passe, et pas seulement ça.
Tout programme qui peut recevoir des paquet DHCP doit vérifier le \emph{cookie magique},
et le comparer à cette constante.

Par exemple, prenons le fichier dhcpcore.dll de Windows 7 x64 et cherchons cette constante.
Et nous la trouvons, deux fois:
Il semble que la constante soit utilisée dans deux fonctions avec des noms parlants\\
\TT{DhcpExtractOptionsForValidation()} et \TT{DhcpExtractFullOptions()}:

\begin{lstlisting}[caption=dhcpcore.dll (Windows 7 x64),style=customasmx86]
.rdata:000007FF6483CBE8 dword_7FF6483CBE8 dd 63538263h          ; DATA XREF: DhcpExtractOptionsForValidation+79
.rdata:000007FF6483CBEC dword_7FF6483CBEC dd 63538263h          ; DATA XREF: DhcpExtractFullOptions+97
\end{lstlisting}

Et ici sont les endroits où ces constantes sont accédées:

\begin{lstlisting}[caption=dhcpcore.dll (Windows 7 x64),style=customasmx86]
.text:000007FF6480875F  mov     eax, [rsi]
.text:000007FF64808761  cmp     eax, cs:dword_7FF6483CBE8
.text:000007FF64808767  jnz     loc_7FF64817179
\end{lstlisting}

Et:

\begin{lstlisting}[caption=dhcpcore.dll (Windows 7 x64),style=customasmx86]
.text:000007FF648082C7  mov     eax, [r12]
.text:000007FF648082CB  cmp     eax, cs:dword_7FF6483CBEC
.text:000007FF648082D1  jnz     loc_7FF648173AF
\end{lstlisting}

\subsection{Constantes spécifiques}

Parfois, il y a une constante spécifique pour un certain type de code.
Par exemple, je me suis plongé une fois dans du code, où le nombre 12 était rencontré
anormalement souvent.
La taille de nombreux tableaux était 12 ou un multiple de 12 (24, etc.).
Il s'est avéré que ce code prenait des fichiers audio de 12 canaux en entrée et les
traitait.

Et vice versa: par exemple, si un programme fonctionne avec des champs de texte qui
ont une longueur de 120 octets, il doit y avoir une constante 120 ou 119 quelque
part dans le code.
Si UTF-16 est utilisé, alors $2 \cdot 120$.
Si le code fonctionne avec des paquets réseau de taille fixe, c'est une bonne idée
de chercher cette constante dans le code.

C'est aussi vrai pour le chiffrement amateur (clefs de licence, etc.).
Si le bloc chiffré a une taille de $n$ octets, vous pouvez essayer de trouver des
occurrences de ce nombre à travers le code.
Aussi, si vous voyez un morceau e code qui est répété $n$ fois dans une boucle durant
l'exécution, ceci peut être une routine de chiffrement/déchiffrement.

\subsection{Chercher des constantes}

C'est facile dans \IDA: Alt-B or Alt-I.
\myindex{binary grep}
Et pour chercher une constante dans un grand nombre de fichiers, ou pour chercher
dans des fichiers non exécutables, il y a un petit utilitaire appelé \emph{binary grep}\footnote{\BGREPURL}.


%% TBT \subsection{Instructions}
\label{sec:x86_instructions}

Instructions marked as (M) are not usually generated by the compiler: if you see one of them, it is probably
a hand-written piece of assembly code, or a compiler intrinsic (\myref{sec:compiler_intrinsic}).

% TODO ? обратные инструкции

Only the most frequently used instructions are listed here.
You can read \myref{x86_manuals} for a full documentation.

Do you have to know all instruction's opcodes by heart?
No, only those which are used for code patching (\myref{x86_patching}).
All the rest of the opcodes don't need to be memorized.

\subsubsection{Prefixes}

\myindex{x86!\Prefixes!LOCK}
\myindex{x86!\Prefixes!REP}
\myindex{x86!\Prefixes!REPE/REPNE}
\begin{description}
\label{x86_lock}
\item[LOCK] forces CPU to make exclusive access to the RAM in multiprocessor environment.
For the sake of simplification, it can be said that when an instruction
with this prefix is executed, all other CPUs in a multiprocessor system are stopped.
Most often
it is used for critical sections, semaphores, mutexes.
Commonly used with ADD, AND, BTR, BTS, CMPXCHG, OR, XADD, XOR.
You can read more about critical sections here (\myref{critical_sections}).

\item[REP] is used with the MOVSx and STOSx instructions:
execute the instruction in a loop, the counter is located in the CX/ECX/RCX register.
For a detailed description, read more about the MOVSx (\myref{REP_MOVSx}) 
and STOSx (\myref{REP_STOSx}) instructions.

The instructions prefixed by REP are sensitive to the DF flag, which is used to set the direction.

\item[REPE/REPNE] (\ac{AKA} REPZ/REPNZ) used with CMPSx and
SCASx instructions:
execute the last instruction in a loop, the count is set in the \TT{CX}/\TT{ECX}/\TT{RCX} register.
It terminates prematurely if ZF is 0 (REPE) or if ZF is 1 (REPNE).

For a detailed description, you can read more about the CMPSx (\myref{REPE_CMPSx}) 
and SCASx (\myref{REPNE_SCASx}) instructions.

Instructions prefixed by REPE/REPNE are sensitive to the DF flag, which is used to set the direction.

\end{description}

\subsubsection{Most frequently used instructions}

These can be memorized in the first place.

\begin{description}
% in order to keep them easily sorted...
\input{appendix/x86/instructions/ADC}
\input{appendix/x86/instructions/ADD}
\input{appendix/x86/instructions/AND}
\input{appendix/x86/instructions/CALL}
\input{appendix/x86/instructions/CMP}
\input{appendix/x86/instructions/DEC}
\myindex{x86!\Instructions!IMUL}
  \item[IMUL] \RU{умножение с учетом знаковых значений}\EN{signed multiply}\FR{multiplication signée}
  \EN{\IMUL often used instead of \MUL, read more about it:}%
  \RU{\IMUL часто используется вместо \MUL, читайте об этом больше:}%
  \FR{\IMUL est souvent utilisé à la place de \MUL, voir ici:} \myref{IMUL_over_MUL}.


\input{appendix/x86/instructions/INC}
\input{appendix/x86/instructions/JCXZ}
\input{appendix/x86/instructions/JMP}
\input{appendix/x86/instructions/Jcc}
\input{appendix/x86/instructions/LAHF}
\input{appendix/x86/instructions/LEAVE}
\myindex{x86!\Instructions!LEA}
\item[LEA] (\emph{Load Effective Address}) \RU{сформировать адрес}\EN{form an address}

\label{sec:LEA}

\newcommand{\URLAM}{\href{http://en.wikipedia.org/wiki/Addressing_mode}{wikipedia}}

\RU{Это инструкция, которая задумывалась вовсе не для складывания 
и умножения чисел, 
а для формирования адреса например, из указателя на массив и прибавления индекса к нему
\footnote{См. также: \URLAM}.}
\EN{This instruction was intended not for summing values and multiplication 
but for forming an address, 
e.g., for calculating the address of an array element by adding the array address, element index, with 
multiplication of element size\footnote{See also: \URLAM}.}
\par
\RU{То есть, разница между \MOV и \LEA в том, что \MOV формирует адрес в памяти 
и загружает значение из памяти, либо записывает его туда, а \LEA только формирует адрес.}
\EN{So, the difference between \MOV and \LEA is that \MOV forms a memory address and loads a value
from memory or stores it there, but \LEA just forms an address.}
\par
\RU{Тем не менее, её можно использовать для любых других вычислений}
\EN{But nevertheless, it is can be used for any other calculations}.
\par
\RU{\LEA удобна тем, что производимые ею вычисления не модифицируют флаги \ac{CPU}. 
Это может быть очень важно для \ac{OOE} процессоров (чтобы было меньше зависимостей между данными).}
\EN{\LEA is convenient because the computations performed by it does not alter \ac{CPU} flags.
This may be very important for \ac{OOE} processors (to create less data dependencies).}

\RU{Помимо всего прочего, начиная минимум с Pentium, инструкция LEA исполняется за 1 такт.}%
\EN{Aside from this, starting at least at Pentium, LEA instruction is executed in 1 cycle.}

\begin{lstlisting}[style=customc]
int f(int a, int b)
{
	return a*8+b;
};
\end{lstlisting}

\begin{lstlisting}[caption=\Optimizing MSVC 2010,style=customasmx86]
_a$ = 8		; size = 4
_b$ = 12	; size = 4
_f	PROC
	mov	eax, DWORD PTR _b$[esp-4]
	mov	ecx, DWORD PTR _a$[esp-4]
	lea	eax, DWORD PTR [eax+ecx*8]
	ret	0
_f	ENDP
\end{lstlisting}

\myindex{Intel C++}
Intel C++ \RU{использует LEA даже больше}\EN{uses LEA even more}:

\begin{lstlisting}[style=customc]
int f1(int a)
{
	return a*13;
};
\end{lstlisting}

\begin{lstlisting}[caption=Intel C++ 2011,style=customasmx86]
_f1	PROC NEAR 
        mov       ecx, DWORD PTR [4+esp]      ; ecx = a
	lea       edx, DWORD PTR [ecx+ecx*8]  ; edx = a*9
	lea       eax, DWORD PTR [edx+ecx*4]  ; eax = a*9 + a*4 = a*13
        ret                                
\end{lstlisting}

\RU{Эти две инструкции вместо одной IMUL будут работать быстрее.}
\EN{These two instructions performs faster than one IMUL.}


\input{appendix/x86/instructions/MOVSB_W_D_Q}
\input{appendix/x86/instructions/MOVSX}
\input{appendix/x86/instructions/MOVZX}
\input{appendix/x86/instructions/MOV}
\myindex{x86!\Instructions!MUL}
  \item[MUL] \RU{умножение с учетом беззнаковых значений}\EN{unsigned multiply}\FR{multiplier sans signe}.
  \EN{\IMUL often used instead of \MUL, read more about it:}%
  \RU{\IMUL часто используется вместо \MUL, читайте об этом больше:}%
  \FR{\IMUL est souvent utilisée au lieu de \MUL, en lire plus ici:} \myref{IMUL_over_MUL}.


\input{appendix/x86/instructions/NEG}
\input{appendix/x86/instructions/NOP}
\input{appendix/x86/instructions/NOT}
\input{appendix/x86/instructions/OR}
\input{appendix/x86/instructions/POP}
\input{appendix/x86/instructions/PUSH}
\input{appendix/x86/instructions/RET}
\input{appendix/x86/instructions/SAHF}
\input{appendix/x86/instructions/SBB}
\input{appendix/x86/instructions/SCASB_W_D_Q}
\input{appendix/x86/instructions/SHx}
\input{appendix/x86/instructions/SHRD}
\input{appendix/x86/instructions/STOSB_W_D_Q}
\input{appendix/x86/instructions/SUB}
\input{appendix/x86/instructions/TEST}
\input{appendix/x86/instructions/XOR}
\end{description}

\subsubsection{Less frequently used instructions}

\begin{description}
\input{appendix/x86/instructions/BSF}
\input{appendix/x86/instructions/BSR}
\input{appendix/x86/instructions/BSWAP}
\input{appendix/x86/instructions/BTC}
\input{appendix/x86/instructions/BTR}
\input{appendix/x86/instructions/BTS}
\input{appendix/x86/instructions/BT}
\input{appendix/x86/instructions/CBW_CWDE_CDQ}
\input{appendix/x86/instructions/CLD}
\input{appendix/x86/instructions/CLI}
\input{appendix/x86/instructions/CLC}
\input{appendix/x86/instructions/CMC}
\input{appendix/x86/instructions/CMOVcc}
\input{appendix/x86/instructions/CMPSB_W_D_Q}
\input{appendix/x86/instructions/CPUID}
\input{appendix/x86/instructions/DIV}
\input{appendix/x86/instructions/IDIV}
\myindex{x86!\Instructions!INT}
\myindex{MS-DOS}

\item[INT] (M): \INS{INT x} \RU{аналогична}\EN{is analogous to} \INS{PUSHF; CALL dword ptr [x*4]} 
\RU{в 16-битной среде}\EN{in 16-bit environment}.
  \RU{Она активно использовалась в MS-DOS, работая как сисколл. Аргументы записывались в регистры
  AX/BX/CX/DX/SI/DI и затем происходил переход на таблицу векторов прерываний (расположенную в самом
  начале адресного пространства)}
  \EN{It was widely used in MS-DOS, functioning as a syscall vector. The registers AX/BX/CX/DX/SI/DI were filled
  with the arguments and then the flow jumped to the address in the Interrupt Vector Table 
  (located at the beginning of the address space)}.
  \RU{Она была очень популярна потому что имела короткий опкод (2 байта) и программе использующая
  сервисы MS-DOS не нужно было заморачиваться узнавая адреса всех функций этих сервисов}
  \EN{It was popular because INT has a short opcode (2 bytes) and the program which needs
  some MS-DOS services is not bother to determine the address of the service's entry point}.
\myindex{x86!\Instructions!IRET}
  \RU{Обработчик прерываний возвращал управление назад при помощи инструкции IRET}
  \EN{The interrupt handler returns the control flow to caller using the IRET instruction}.

  \RU{Самое используемое прерывание в MS-DOS было 0x21, там была основная часть его \ac{API}}
  \EN{The most busy MS-DOS interrupt number was 0x21, serving a huge part of its \ac{API}}.
  \RU{См. также}\EN{See also}: [Ralf Brown \emph{Ralf Brown's Interrupt List}], 
  \RU{самый крупный список всех известных прерываний и вообще там много информации о MS-DOS}
  \EN{for the most comprehensive interrupt lists and other MS-DOS information}.

\myindex{x86!\Instructions!SYSENTER}
\myindex{x86!\Instructions!SYSCALL}
  \RU{Во времена после MS-DOS, эта инструкция все еще использовалась как сискол, и в Linux
  и в Windows (\myref{syscalls}), но позже была заменена инструкцией SYSENTER или SYSCALL}
  \EN{In the post-MS-DOS era, this instruction was still used as syscall both in Linux and 
  Windows (\myref{syscalls}), but was later replaced by the SYSENTER or SYSCALL instructions}.

\item[INT 3] (M): \RU{эта инструкция стоит немного в стороне от}\EN{this instruction is somewhat close to} 
\INS{INT}, \RU{она имеет собственный 1-байтный опкод}\EN{it has its own 1-byte opcode} (\GTT{0xCC}), 
\RU{и активно используется в отладке}\EN{and is actively used while debugging}.
\RU{Часто, отладчик просто записывает байт}\EN{Often, the debuggers just write the} \GTT{0xCC} 
\RU{по адресу в памяти где устанавливается точка останова, и когда исключение поднимается, оригинальный байт
будет восстановлен и оригинальная инструкция по этому адресу исполнена заново}\EN{byte at the address of 
the breakpoint to be set, and when an exception is raised,
the original byte is restored and the original instruction at this address is re-executed}. \\
\RU{В}\EN{As of} \gls{Windows NT}, \RU{исключение}\EN{an} \GTT{EXCEPTION\_BREAKPOINT} \RU{поднимается,
когда \ac{CPU} исполняет эту инструкцию}\EN{exception is to be raised when the \ac{CPU} executes this instruction}.
\RU{Это отладочное событие может быть перехвачено и обработано отладчиком, если он загружен}
\EN{This debugging event may be intercepted and handled by a host debugger, if one is loaded}.
\RU{Если он не загружен, Windows предложит запустить один из зарегистрированных в системе отладчиков}
\EN{If it is not loaded, Windows offers to run one of the registered system debuggers}.
\RU{Если}\EN{If} \ac{MSVS} \RU{установлена, его отладчик может быть загружен и подключен к процессу}\EN{is installed, 
its debugger may be loaded and connected to the process}.
\RU{В целях защиты от}\EN{In order to protect from} \gls{reverse engineering}, \RU{множество анти-отладочных методов
проверяют целостность загруженного кода}\EN{a lot of anti-debugging methods check integrity of the loaded code}.

\RU{В }\ac{MSVC} \RU{есть}\EN{has} \gls{compiler intrinsic} \RU{для этой инструкции}\EN{for the instruction}:
\GTT{\_\_debugbreak()}\footnote{\href{http://msdn.microsoft.com/en-us/library/f408b4et.aspx}{MSDN}}.

\RU{В win32 также имеется функция в}\EN{There is also a win32 function in} kernel32.dll \RU{с названием}\EN{named}
\GTT{DebugBreak()}\footnote{\href{http://msdn.microsoft.com/en-us/library/windows/desktop/ms679297(v=vs.85).aspx}{MSDN}},
\RU{которая также исполняет}\EN{which also executes} \GTT{INT 3}.


\input{appendix/x86/instructions/IN}
\input{appendix/x86/instructions/IRET}
\input{appendix/x86/instructions/LOOP}
\input{appendix/x86/instructions/OUT}
\input{appendix/x86/instructions/POPA}
\input{appendix/x86/instructions/POPCNT}
\input{appendix/x86/instructions/POPF}
\input{appendix/x86/instructions/PUSHA}
\input{appendix/x86/instructions/PUSHF}
\input{appendix/x86/instructions/RCx}
\myindex{x86!\Instructions!ROL}
\myindex{x86!\Instructions!ROR}
\label{ROL_ROR}
\item[ROL/ROR] (M) cyclic shift
  
ROL: rotate left:

\input{rotate_left}

ROR: rotate right:

\input{rotate_right}

Despite the 
fact that almost all \ac{CPU}s have these instructions, there are no corresponding
operations in \CCpp, so the compilers of these \ac{PL}s usually do not generate these 
instructions.

For the programmer's convenience, at least \ac{MSVC} has the pseudofunctions (compiler intrinsics)
\emph{\_rotl()} and \emph{\_rotr()}\FNMSDNROTxURL{},
which are translated by the compiler directly to these instructions.


\input{appendix/x86/instructions/SAL}
\input{appendix/x86/instructions/SAR}
\input{appendix/x86/instructions/SETcc}
\input{appendix/x86/instructions/STC}
\input{appendix/x86/instructions/STD}
\input{appendix/x86/instructions/STI}
\input{appendix/x86/instructions/SYSCALL}
\input{appendix/x86/instructions/SYSENTER}
\input{appendix/x86/instructions/UD2}
\input{appendix/x86/instructions/XCHG}
\end{description}

\subsubsection{FPU instructions}

\TT{-R} suffix in the mnemonic usually implies that the operands are reversed,
\TT{-P} suffix implies that one element is popped
from the stack after the instruction's execution, \TT{-PP} suffix implies that two elements are popped.

\TT{-P} instructions are often useful when we do not need the value in the FPU stack to be 
present anymore after the operation.

\begin{description}
\input{appendix/x86/instructions/FABS}
\input{appendix/x86/instructions/FADD} % + FADDP
\input{appendix/x86/instructions/FCHS}
\input{appendix/x86/instructions/FCOM} % + FCOMP + FCOMPP
\input{appendix/x86/instructions/FDIVR} % + FDIVRP
\input{appendix/x86/instructions/FDIV} % + FDIVP
\input{appendix/x86/instructions/FILD}
\input{appendix/x86/instructions/FIST} % + FISTP
\input{appendix/x86/instructions/FLD1}
\input{appendix/x86/instructions/FLDCW}
\input{appendix/x86/instructions/FLDZ}
\input{appendix/x86/instructions/FLD}
\input{appendix/x86/instructions/FMUL} % + FMULP
\input{appendix/x86/instructions/FSINCOS}
\input{appendix/x86/instructions/FSQRT}
\input{appendix/x86/instructions/FSTCW} % + FNSTCW
\input{appendix/x86/instructions/FSTSW} % + FNSTSW
\input{appendix/x86/instructions/FST}
\input{appendix/x86/instructions/FSUBR} % + FSUBRP
\input{appendix/x86/instructions/FSUB} % + FSUBP
\myindex{x86!\Instructions!FUCOM}
\myindex{x86!\Instructions!FUCOMP}
\myindex{x86!\Instructions!FUCOMPP}
  \item[FUCOM] ST(i): compare ST(0) and ST(i)
  \item[FUCOM] compare ST(0) and ST(1)
  \item[FUCOMP] compare ST(0) and ST(1); pop one element from stack.
  \item[FUCOMPP] compare ST(0) and ST(1); pop two elements from stack.
 
The instructions perform just like FCOM, but an exception is raised only if one of the operands is SNaN,
  while QNaN numbers are processed smoothly.
 % + FUCOMP + FUCOMPP
\input{appendix/x86/instructions/FXCH}
\end{description}

%\subsubsection{SIMD instructions}

% TODO

%\begin{description}
%\input{appendix/x86/instructions/DIVSD}
%\input{appendix/x86/instructions/MOVDQA}
%\input{appendix/x86/instructions/MOVDQU}
%\input{appendix/x86/instructions/PADDD}
%\input{appendix/x86/instructions/PCMPEQB}
%\input{appendix/x86/instructions/PLMULHW}
%\input{appendix/x86/instructions/PLMULLD}
%\input{appendix/x86/instructions/PMOVMSKB}
%\input{appendix/x86/instructions/PXOR}
%\end{description}

% SHLD !
% SHRD !
% BSWAP !
% CMPXCHG
% XADD !
% CMPXCHG8B
% RDTSC !
% PAUSE!

% xsave
% fnclex, fnsave
% movsxd, movaps, wait, sfence, lfence, pushfq
% prefetchw
% REP RETN
% REP BSF
% movnti, movntdq, rdmsr, wrmsr
% ldmxcsr, stmxcsr, invlpg
% swapgs
% movq, movd
% mulsd
% POR
% IRETQ
% pslldq
% psrldq
% cqo, fxrstor, comisd, xrstor, wbinvd, movntq
% fprem
% addsb, subsd, frndint

% rare:
%\item[ENTER]
%\item[LES]
% LDS
% XLAT

\subsubsection{Instructions having printable ASCII opcode}

(In 32-bit mode.)

\label{printable_x86_opcodes}
\myindex{Shellcode}
These can be suitable for shellcode construction.
See also: \myref{subsec:EICAR}.

% FIXME: start at 0x20...
\begin{center}
\begin{longtable}{ | l | l | l | }
\hline
\HeaderColor ASCII character & 
\HeaderColor hexadecimal code & 
\HeaderColor x86 instruction \\
\hline
0	 &30	 &XOR \\
1	 &31	 &XOR \\
2	 &32	 &XOR \\
3	 &33	 &XOR \\
4	 &34	 &XOR \\
5	 &35	 &XOR \\
7	 &37	 &AAA \\
8	 &38	 &CMP \\
9	 &39	 &CMP \\
:	 &3a	 &CMP \\
;	 &3b	 &CMP \\
<	 &3c	 &CMP \\
=	 &3d	 &CMP \\
?	 &3f	 &AAS \\
@	 &40	 &INC \\
A	 &41	 &INC \\
B	 &42	 &INC \\
C	 &43	 &INC \\
D	 &44	 &INC \\
E	 &45	 &INC \\
F	 &46	 &INC \\
G	 &47	 &INC \\
H	 &48	 &DEC \\
I	 &49	 &DEC \\
J	 &4a	 &DEC \\
K	 &4b	 &DEC \\
L	 &4c	 &DEC \\
M	 &4d	 &DEC \\
N	 &4e	 &DEC \\
O	 &4f	 &DEC \\
P	 &50	 &PUSH \\
Q	 &51	 &PUSH \\
R	 &52	 &PUSH \\
S	 &53	 &PUSH \\
T	 &54	 &PUSH \\
U	 &55	 &PUSH \\
V	 &56	 &PUSH \\
W	 &57	 &PUSH \\
X	 &58	 &POP \\
Y	 &59	 &POP \\
Z	 &5a	 &POP \\
\lbrack{}	 &5b	 &POP \\
\textbackslash{}	 &5c	 &POP \\
\rbrack{}	 &5d	 &POP \\
\verb|^|	 &5e	 &POP \\
\_	 &5f	 &POP \\
\verb|`|	 &60	 &PUSHA \\
a	 &61	 &POPA \\

h	 &68	 &PUSH\\
i	 &69	 &IMUL\\
j	 &6a	 &PUSH\\
k	 &6b	 &IMUL\\
p	 &70	 &JO\\
q	 &71	 &JNO\\
r	 &72	 &JB\\
s	 &73	 &JAE\\
t	 &74	 &JE\\
u	 &75	 &JNE\\
v	 &76	 &JBE\\
w	 &77	 &JA\\
x	 &78	 &JS\\
y	 &79	 &JNS\\
z	 &7a	 &JP\\
\hline
\end{longtable}
\end{center}

Also:

\begin{center}
\begin{longtable}{ | l | l | l | }
\hline
\HeaderColor ASCII character & 
\HeaderColor hexadecimal code & 
\HeaderColor x86 instruction \\
\hline
f	 &66	 &(in 32-bit mode) switch to\\
   & & 16-bit operand size \\
g	 &67	 &in 32-bit mode) switch to\\
   & & 16-bit address size \\
\hline
\end{longtable}
\end{center}

\myindex{x86!\Instructions!AAA}
\myindex{x86!\Instructions!AAS}
\myindex{x86!\Instructions!CMP}
\myindex{x86!\Instructions!DEC}
\myindex{x86!\Instructions!IMUL}
\myindex{x86!\Instructions!INC}
\myindex{x86!\Instructions!JA}
\myindex{x86!\Instructions!JAE}
\myindex{x86!\Instructions!JB}
\myindex{x86!\Instructions!JBE}
\myindex{x86!\Instructions!JE}
\myindex{x86!\Instructions!JNE}
\myindex{x86!\Instructions!JNO}
\myindex{x86!\Instructions!JNS}
\myindex{x86!\Instructions!JO}
\myindex{x86!\Instructions!JP}
\myindex{x86!\Instructions!JS}
\myindex{x86!\Instructions!POP}
\myindex{x86!\Instructions!POPA}
\myindex{x86!\Instructions!PUSH}
\myindex{x86!\Instructions!PUSHA}
\myindex{x86!\Instructions!XOR}

In summary:
AAA, AAS, CMP, DEC, IMUL, INC, JA, JAE, JB, JBE, JE, JNE, JNO, JNS, JO, JP, JS, POP, POPA, PUSH, PUSHA, 
XOR.


%% TBT \mysection{Suspicious code patterns}

\subsection{XOR instructions}
\myindex{x86!\Instructions!XOR}

Instructions like \TT{XOR op, op} (for example, \TT{XOR EAX, EAX}) 
are usually used for setting the register value
to zero, but if the operands are different, the \q{exclusive or} operation
is executed.

This operation is rare in common programming, but widespread in cryptography,
including amateur one.
It's especially suspicious if the
second operand is a big number.

This may point to encrypting/decrypting, checksum computing, etc.\\
\\

One exception to this observation worth noting is the \q{canary} (\myref{subsec:BO_protection}). 
Its generation and checking are often done using the \XOR instruction. \\
\\
\myindex{AWK}

This AWK script can be used for processing \IDA{} listing (.lst) files:

\lstinputlisting{digging_into_code/awk.sh}

It is also worth noting that this kind of script can also match incorrectly disassembled code 
(\myref{sec:incorrectly_disasmed_code}).

\subsection{Hand-written assembly code}

\myindex{Function prologue}
\myindex{Function epilogue}
\myindex{x86!\Instructions!LOOP}
\myindex{x86!\Instructions!RCL}

Modern compilers do not emit the \TT{LOOP} and \TT{RCL} instructions.
On the other hand, these instructions are well-known to coders who like to code directly in assembly language.
If you spot these, it can be said that there is a high probability that this fragment of code was hand-written.
Such instructions are marked as (M) in the instructions list in this appendix: \myref{sec:x86_instructions}.

\par
Also the function prologue/epilogue are not commonly present in hand-written assembly.

\par
Commonly there is no fixed system for passing arguments to functions in the hand-written code.

\par
Example from the Windows 2003 kernel 
(ntoskrnl.exe file):

\lstinputlisting[style=customasmx86]{digging_into_code/ntoskrnl.lst}

Indeed, if we look in the 
\ac{WRK} v1.2 source code, this code
can be found easily in file \\
\emph{WRK-v1.2\textbackslash{}base\textbackslash{}ntos\textbackslash{}ke\textbackslash{}i386\textbackslash{}cpu.asm}.

\par 
As of \INS{RCL}, I could find it in ntoskrnl.exe file from Windows 2003 x86 (MS Visual C compiler).
It is occurred only once, in \TT{RtlExtendedLargeIntegerDivide()} function, and this might be inline assembler code case.


%% TBT \mysection{Using magic numbers while tracing}

Often, our main goal is to understand how the program uses a value that has been either read from file or received via network. 
The manual tracing of a value is often a very labor-intensive task. One of the simplest techniques for this (although not 100\% reliable) 
is to use your own \emph{magic number}.

This resembles X-ray computed tomography is some sense: a radiocontrast agent is injected into the patient's blood,
which is then used to improve the visibility of the patient's internal structure in to the X-rays.
It is well known how the blood of healthy humans
percolates in the kidneys and if the agent is in the blood, it can be easily seen on tomography, how blood is percolating,
and are there any stones or tumors.

We can take a 32-bit number like \TT{0x0badf00d}, or someone's birth date like \TT{0x11101979}
and write this 4-byte number to some point in a file used by the program we investigate.

\myindex{\GrepUsage}
\myindex{tracer}

Then, while tracing this program with \tracer in \emph{code coverage} mode, with the help of \emph{grep}
or just by searching in the text file (of tracing results), we can easily see where the value has been used and how.

Example 
of \emph{grepable} \tracer results in \emph{cc} mode:

\begin{lstlisting}[style=customasmx86]
0x150bf66 (_kziaia+0x14), e=       1 [MOV EBX, [EBP+8]] [EBP+8]=0xf59c934 
0x150bf69 (_kziaia+0x17), e=       1 [MOV EDX, [69AEB08h]] [69AEB08h]=0 
0x150bf6f (_kziaia+0x1d), e=       1 [FS: MOV EAX, [2Ch]] 
0x150bf75 (_kziaia+0x23), e=       1 [MOV ECX, [EAX+EDX*4]] [EAX+EDX*4]=0xf1ac360 
0x150bf78 (_kziaia+0x26), e=       1 [MOV [EBP-4], ECX] ECX=0xf1ac360 
\end{lstlisting}
% TODO: good example!

This can be used for network packets as well.
It is important for the \emph{magic number} to be unique and not to be present in the program's code.

\newcommand{\DOSBOXURL}{\href{http://blog.yurichev.com/node/55}{blog.yurichev.com}}

\myindex{DosBox}
\myindex{MS-DOS}
Aside of 
the \tracer, DosBox (MS-DOS emulator) in heavydebug mode
is able to write information about all registers' states for each executed instruction of the program to a plain text file\footnote{See also my 
blog post about this DosBox feature: \DOSBOXURL{}}, so this technique may be useful for DOS programs as well.


\mysection{Boucles}

À chaque fois que votre programme travaille avec des sortes de fichier, ou un buffer
d'une certaine taille, il doit s'agir d'un sorte de boucle de déchiffrement/traitement
à l'intérieur du code.

Ceci est un exemple réel de sortie de l'outil \tracer.
Il y avait un code qui chargeait une sorte de fichier chiffré de 258 octets.
Je l'ai lancé dans l'intention d'obtenir le nombre d'exécution de chaque instruction
(l'outil \ac{DBI} irait beaucoup mieux de nos jours).
Et j'ai rapidement trouvé un morceau de code qui était exécuté 259/258 fois:

\lstinputlisting{digging_into_code/crypto_loop.txt}

Il s'avère qu'il s'agit de la boucle de déchiffrement.


% TODO move section...

\subsection{Quelques schémas de fichier binaire}

Tous les exemples ici ont été préparé sur Windows, avec la page de code 437 activée%
dans la console.
L'intérieur des fichiers binaires peut avoir l'air différent avec une autre page
de code.

\clearpage
\subsubsection{Tableaux}

Parfois, nous pouvons clairement localiser visuellement un tableau de valeurs 16/32/64-bit,
dans un éditeur hexadécimal.

Voici un exemple de tableau de valeurs 16-bit.
Nous voyons que le premier octet d'une paire est 7 ou 8, et que le second semble
aléatoire:

\begin{figure}[H]
\centering
\myincludegraphics{digging_into_code/binary/16bit_array.png}
\caption{FAR: tableau de valeurs 16-bit}
\end{figure}

J'ai utilisé un fichier contenant un signal 12-canaux numérisé en utilisant 16-bit \ac{ADC}.

\clearpage
\myindex{MIPS}
\par Et voici un exemple de code MIPS très typique.

Comme nous pouvons nous en souvenir, chaque instruction MIPS (et aussi ARM en mode
ARM ou ARM64) a une taille de 32 bits (ou 4 octets), donc un tel code est un tableau
de valeurs 32-bit.

En regardant cette copie d'écran, nous voyons des sortes de schémas.

Les lignes rouge verticales ont été ajoutées pour la clarté:

\begin{figure}[H]
\centering
\myincludegraphics{digging_into_code/binary/typical_MIPS_code.png}
\caption{Hiew: code MIPS très typique}
\end{figure}

Il y a un autre exemple de tel schéma ici dans le livre:
\myref{Oracle_SYM_files_example}.

\clearpage
\subsubsection{Fichiers clairsemés}

Ceci est un fichier clairsemé avec des données éparpillées dans un fichier presque vide.
Chaque caractère espace est en fait l'octet zéro (qui rend comme un espace).
Ceci est un fichier pour programmer des FPGA (Altera Stratix GX device).
Bien sûr, de tels fichiers peuvent être compressés facilement, mais des formats comme
celui-ci sont très populaire dans les logiciels scientifiques et d'ingénierie, où
l'efficience des accès est importante, tandis que la compacité ne l'est pas.

\begin{figure}[H]
\centering
\myincludegraphics{digging_into_code/binary/sparse_FPGA.png}
\caption{FAR: Fichier clairsemé}
\end{figure}

\clearpage
\subsubsection{Fichiers compressés}

% FIXME \ref{} ->
Ce fichier est juste une archive compressée.
Il a une entropie relativement haute et visuellement, il à l'air chaotique.
Ceci est ce à quoi ressemble les fichiers compressés et/ou chiffrés.

\begin{figure}[H]
\centering
\myincludegraphics{digging_into_code/binary/compressed.png}
\caption{FAR: Fichier compressé}
\end{figure}

\clearpage
\subsubsection{\ac{CDFS}}

Les fichiers d'installation d'un \ac{OS} sont en général distribués sous forme de
fichiers ISO, qui sont des copies de disques CD/DVD.
Le système de fichiers utilisé est appelé \ac{CDFS}, ce que vous voyez ici sont des
noms de fichiers mixés avec des données additionnelles.
Ceci peut-être la taille des fichiers, des pointeurs sur d'autres répertoires, des
attributs de fichier, etc.
C'est l'aspect typique de ce à quoi ressemble un système de fichiers en interne.

\begin{figure}[H]
\centering
\myincludegraphics{digging_into_code/binary/cdfs.png}
\caption{FAR: Fichier ISO: \ac{CD} d'installation d'Ubuntu 15}
\end{figure}

\clearpage
\subsubsection{Code exécutable x86 32-bit}

Voici l'allure de code exécutable x86 32-bit.
Il n'a pas une grande entropie, car certains octets reviennent plus souvent que d'autres.

\begin{figure}[H]
\centering
\myincludegraphics{digging_into_code/binary/x86_32.png}
\caption{FAR: Code exécutable x86 32-bit}
\end{figure}

% TODO: Read more about x86 statistics: \ref{}. % FIXME blog post about decryption...

\clearpage
\subsubsection{Fichiers graphique BMP}

% TODO: bitmap, bit, group of bits...

Les fichiers BMP ne sont pas compressés, donc chaque octet (ou groupe d'octet) représente
chaque pixel.
J'ai trouvé cette image quelque part dans mon installation de Windows 8.1:

\begin{figure}[H]
\centering
\myincludegraphicsSmall{digging_into_code/binary/bmp.png}
\caption{Image exemple}
\end{figure}

Vous voyez que cette image a des pixels qui ne doivent pas pouvoir être compressés
beaucoup (autour du centre), mais il y a de longues lignes monochromes au haut et
en bas.
En effet, de telles lignes ressemblent à des lignes lorsque l'on regarde le fichier:

\begin{figure}[H]
\centering
\myincludegraphics{digging_into_code/binary/bmp_FAR.png}
\caption{Fragment de fichier BMP}
\end{figure}


%% TBT % FIXME comparison!
\subsection{Memory \q{snapshots} comparing}
\label{snapshots_comparing}

The technique of the straightforward comparison of two memory snapshots in order to see changes was often used to hack
8-bit computer games and for hacking \q{high score} files.

For example, if you had a loaded game on an 8-bit computer (there isn't much memory on these, but the game usually
consumes even less memory) and you know that you have now, let's say, 100 bullets, you can do a \q{snapshot}
of all memory and back it up to some place. Then shoot once, the bullet count goes to 99, do a second \q{snapshot}
and then compare both: it must be a byte somewhere which has been 100 at the beginning, and now it is 99.

Considering the fact that these 8-bit games were often written in assembly language and such variables were global,
it can be said for sure which address in memory has holding the bullet count. If you searched for all references to the
address in the disassembled game code, it was not very hard to find a piece of code \glslink{decrement}{decrementing} the bullet count,
then to write a \gls{NOP} instruction there, or a couple of \gls{NOP}-s, 
and then have a game with 100 bullets forever.
\myindex{BASIC!POKE}
Games on these 8-bit computers were commonly loaded at the constant
address, also, there were not much different versions of each game (commonly just one version was popular for a long span of time),
so enthusiastic gamers knew which bytes must be overwritten (using the BASIC's instruction \gls{POKE}) at which address in
order to hack it. This led to \q{cheat} lists that contained \gls{POKE} instructions, published in magazines related to
8-bit games. See also: \href{http://go.yurichev.com/17114}{wikipedia}.

\myindex{MS-DOS}

Likewise, it is easy to modify \q{high score} files, this does not work with just 8-bit games. Notice 
your score count and back up the file somewhere. When the \q{high score} count gets different, just compare the two files,
it can even be done with the DOS utility FC\footnote{MS-DOS utility for comparing binary files} (\q{high score} files
are often in binary form).

There will be a point where a couple of bytes are different and it is easy to see which ones are
holding the score number.
However, game developers are fully aware of such tricks and may defend the program against it.

Somewhat similar example in this book is: \myref{Millenium_DOS_game}.

% TODO: пример с какой-то простой игрушкой?

\subsubsection{A real story from 1999}

\myindex{ICQ}
There was a time of ICQ messenger's popularity, at least in ex-USSR countries.
The messenger had a peculiarity --- some users didn't want to share their online status with everyone.
And you had to ask an \emph{authorization} from that user.
That user could allow you seeing his/her status, or maybe not.

This is what the author of these lines did:

\begin{itemize}
\item Added a user.
\item A user appeared in a contact-list, in a ``wait for authorization'' section.
\item Closed ICQ.
\item Backed up the ICQ database.
\item Loaded ICQ again.
\item User \emph{authorized}.
\item Closed ICQ and compared two databases.
\end{itemize}

It turned out: two database differed by only one byte.
In the first version: \verb|RESU\x03|, in the second: \verb|RESU\x02|.
(``RESU'', presumably, means ``USER'', i.e., a header of a structure where all the information about user was stored.)
That means the information about authorization was stored not at the server, but at the client.
Presumably, 2/3 value reflected \emph{authorization} status.

\subsubsection{Windows registry}

It is also possible to compare the Windows registry before and after a program installation.

It is a very popular method of finding which registry elements are used by the program.
Perhaps, this is the reason why the \q{windows registry cleaner} shareware is so popular.

By the way, this is how to dump Windows registry to text files:

\begin{lstlisting}
reg export HKLM HKLM.reg
reg export HKCU HKCU.reg
reg export HKCR HKCR.reg
reg export HKU HKU.reg
reg export HKCC HKCC.reg
\end{lstlisting}

\myindex{UNIX!diff}
They can be compared using diff...

\subsubsection{Engineering software, CADs, etc}

If a software uses proprietary files, you can also investigate something here as well.
You save file.
Then you add a dot or line or another primitive.
Save file, compare.
Or move dot, save file, compare.

\subsubsection{Blink-comparator}

Comparison of files or memory snapshots remind us blink-comparator
\footnote{\url{http://go.yurichev.com/17348}}:
a device used by astronomers in past, intended to find moving celestial objects.

Blink-comparator allows to switch quickly between two photographies shot in different time,
so astronomer would spot the difference visually.

By the way, Pluto was discovered by blink-comparator in 1930.

%% TBT \input{digging_into_code/ISA_detect_EN}

\mysection{Autres choses}

\subsection{Idée générale}

Un rétro-ingénieur doit essayer se se mettre dans la peau d'un programmeur aussi
souvent que possible.
Pour prendre son point de vue et se demander comment il aurait résolu des taches
d'un cas spécifique.

\subsection{Ordre des fonctions dans le code binaire}

Toutes les fonctions situées dans un unique fichier .c ou .cpp sont compilées dans
le fichier objet (.o) correspondant.
Plus tard, l'éditeur de liens mets tous les fichiers dont il a besoin ensemble, sans
changer l'ordre ni les fonctions.
Par conséquent, si vous voyez deux ou plus fonctions consécutives, cela signifie
qu'elles étaient situées dans le même fichier source (à moins que vous ne soyez en
limite de deux fichiers objet, bien sûr).
Ceci signifie que ces fonctions ont quelque chose en commun, qu'elles sont des fonctions
du même niveau d'\ac{API}, de la même bibliothèque, etc.

\myindex{CryptoPP}
Ceci est une histoire vraie de pratique: il était une fois, alors que je cherchais
des fonctions relatives à Twofish dans un programme lié à la bibliothèque CryptoPP,
en particulier des fonctions de chiffrement/déchiffrement.\\
J'ai trouvé la fonction \verb|Twofish::Base::UncheckedSetKey()| mais pas d'autres.
Après avoir cherché dans le code source
\verb|twofish.cpp|\footnote{\url{https://github.com/weidai11/cryptopp/blob/b613522794a7633aa2bd81932a98a0b0a51bc04f/twofish.cpp}},
il devint clair que toutes les fonctions étaient situées dans ce module (\verb|twofish.cpp|).\\
Donc j'ai essayé toutes les fonctions qui suivaient \verb|Twofish::Base::UncheckedSetKey()|---comme elles arrivaient,\\
une a été \verb|Twofish::Enc::ProcessAndXorBlock()|, une autre---\verb|Twofish::Dec::ProcessAndXorBlock()|.

\subsection{Fonctions minuscules}

Les fonctions minuscules comme les fonctions vides (\myref{empty_func})
ou les fonctions qui renvoient juste ``true'' (1) ou ``false'' (0) (\myref{ret_val_func})
sont très communes, et presque tous les compilateurs corrects tendent à ne mettre
qu'une seule fonction de ce genre dans le code de l'exécutable résultant, même si
il y avait plusieurs fonctions similaires dans le code source.
Donc, à chaque fois que vous voyez une fonction minuscule consistant seulement en
\TT{mov eax, 1 / ret} qui est référencée (et peut être appelée) dans plusieurs endroits
qui ne semblent pas reliés les uns au autres, ceci peut résulter d'une telle optimisation.%

\subsection{\Cpp}

Les données \ac{RTTI}~(\myref{RTTI})- peuvent être utiles pour l'identification des
classes \Cpp.

%\subsection{Crash on purpose}
\subsection{Crash délibéré}

Souvent, vous voulez savoir quelle fonction a été exécutée, et laquelle ne l'a pas
été.
Vous pouvez utiliser un débogueur, mais sur des architectures exotiques, il peut
ne pas en avoir, donc la façon la plus simple est d'y mettre un opcode invalide,
ou quelque chose comme \INS{INT3} (0xCC).
Le crash signalera le fait que l'instruction a été exécutée.

Un autre exemple de crash délibéré: \myref{dmalloc_KILL_PROCESS}.

