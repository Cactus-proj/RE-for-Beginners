\mysection{Identification de fichiers exécutables}

\subsection{Microsoft Visual C++}
\label{MSVC_versions}

Les versions de MSVC et des DLLs peuvent être importées:

%\small
\begin{center}
\begin{tabular}{ | l | l | l | l | l | }
\hline
\HeaderColor Marketing ver. & 
\HeaderColor Internal ver. & 
\HeaderColor CL.EXE ver. &
\HeaderColor DLLs imported &
\HeaderColor Release date \\
\hline
% 4.0, April 1995
% 97 & 5.0 & February 1997
6		&  6.0	& 12.00	& msvcrt.dll	& June 1998		\\
		&	&	& msvcp60.dll	&			\\
\hline
.NET (2002)	&  7.0	& 13.00	& msvcr70.dll	& February 13, 2002	\\
		&	&	& msvcp70.dll	&			\\
\hline
.NET 2003	&  7.1	& 13.10 & msvcr71.dll	& April 24, 2003	\\
		&	&	& msvcp71.dll	&			\\
\hline
2005		&  8.0	& 14.00 & msvcr80.dll	& November 7, 2005	\\
		&	&	& msvcp80.dll	&			\\
\hline
2008		&  9.0	& 15.00 & msvcr90.dll	& November 19, 2007	\\
		&	&	& msvcp90.dll	&			\\
\hline
2010		& 10.0	& 16.00 & msvcr100.dll	& April 12, 2010 	\\
		&	&	& msvcp100.dll	&			\\
\hline
2012		& 11.0	& 17.00 & msvcr110.dll	& September 12, 2012 	\\
		&	&	& msvcp110.dll	&			\\
\hline
2013		& 12.0	& 18.00 & msvcr120.dll	& October 17, 2013 	\\
		&	&	& msvcp120.dll	&			\\
\hline
\end{tabular}
\end{center}
%\normalsize

msvcp*.dll contient des fonctions relatives à \Cpp{}, donc si elle est importées,
il s'agit probablement d'un programme \Cpp.

\subsubsection{Mangling de nom}

Les noms commencent en général par le symbole \TT{?}.

Vous trouverez plus d'informations le \glslink{name mangling}{mangling de nom} de
MSVC ici: \myref{namemangling}.

\subsection{GCC}
\myindex{GCC}

À part les cibles *NIX, GCC est aussi présent dans l'environnement win32, sous la
forme de Cygwin et MinGW.

\subsubsection{Mangling de nom}

Les noms commencent en général par le symbole \TT{\_Z}.
Vous trouverez plus d'informations le \glslink{name mangling}{mangling de nom} de
GCC ici: \myref{namemangling}.
\subsubsection{Cygwin}
\myindex{Cygwin}

cygwin1.dll est souvent importée.

\subsubsection{MinGW}
\myindex{MinGW}

msvcrt.dll peut être importée.

\subsection{Intel Fortran}
\myindex{Fortran}

libifcoremd.dll, libifportmd.dll et libiomp5md.dll (support OpenMP) peuvent être importées.

libifcoremd.dll a beaucoup de fonctions préfixées par \TT{for\_}, qui signifie \emph{Fortran}.

\subsection{Watcom, OpenWatcom}
\myindex{Watcom}
\myindex{OpenWatcom}

\subsubsection{Mangling de nom}

Les noms commencent usuellement par le symbole \TT{W}.

Par exemple, ceci est la façon dont la méthode nommées \q{method} de la classe \q{class}
qui n'a pas d'argument et qui renvoie \Tvoid est encodée:

\begin{lstlisting}
W?method$_class$n__v
\end{lstlisting}

\subsection{Borland}
\myindex{Borland Delphi}
\myindex{Borland C++Builder}

Voici un exemple de \glslink{name mangling}{mangling de nom} de Delphi de Borland
et de C++Builder:

\lstinputlisting{digging_into_code/identification/borland_mangling.txt}

Les noms commencent toujours avec le symbole \TT{@}, puis nous avons le nom de la
classe, de la méthode et les types des arguments de méthode encodés.

Ces noms peuvent être dans des imports .exe, des exports .dll, des données de débogage,
etc.

Les Borland Visual Component Libraries (VCL) sont stockées dans des fichiers .bpl
au lieu de .dll, par exemple, vcl50.dll, rtl60.dll.

Une autre DLL qui peut être importée: BORLNDMM.DLL.

\subsubsection{Delphi}

Presque tous les exécutables Delpi ont la chaîne de texte \q{Boolean} au début de
leur segment de code, ainsi que d'autres noms de type.

Ceci est le début très typique du segment \TT{CODE} d'un programme Delphi, ce bloc
vient juste après l'entête de fichier win32 PE:

\lstinputlisting{digging_into_code/identification/delphi.txt}

Les 4 premiers octets du segment de données (\TT{DATA}) peuvent être \TT{00 00 00 00},
\TT{32 13 8B C0} ou \TT{FF FF FF FF}.%

Cette information peut être utile lorsque l'on fait face à des exécutable Delphi
préparés/chiffrés.

\subsection{Autres DLLs connues}

\myindex{OpenMP}
\begin{itemize}
\item vcomp*.dll---implémentation d'OpenMP de Microsoft.
\end{itemize}

