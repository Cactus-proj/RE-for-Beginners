\chapter{Поиск в коде того что нужно}

Современное ПО, в общем-то, минимализмом не отличается.

\myindex{\Cpp!STL}
Но не потому, что программисты слишком много пишут, 
а потому что к исполняемым файлам обыкновенно прикомпилируют все подряд библиотеки. 
Если бы все вспомогательные библиотеки всегда выносили во внешние DLL, мир был бы иным.
(Еще одна причина для Си++ --- \ac{STL} и прочие библиотеки шаблонов.)

\newcommand{\FOOTNOTEBOOST}{\footnote{\url{http://www.boost.org/}}}
\newcommand{\FOOTNOTELIBPNG}{\footnote{\url{http://www.libpng.org/pub/png/libpng.html}}}

Таким образом, очень полезно сразу понимать, какая функция из стандартной библиотеки или 
более-менее известной (как Boost\FOOTNOTEBOOST, libpng\FOOTNOTELIBPNG), 
а какая --- имеет отношение к тому что мы пытаемся найти в коде.

Переписывать весь код на \CCpp, чтобы разобраться в нем, безусловно, не имеет никакого смысла.

Одна из важных задач reverse engineer-а это быстрый поиск в коде того что собственно его интересует, а что -- второстепенно.

\myindex{\GrepUsage}
Дизассемблер \IDA позволяет делать поиск как минимум строк, последовательностей байт, констант.
Можно даже сделать экспорт кода в текстовый файл .lst или .asm и затем натравить на него \TT{grep}, \TT{awk}, итд.

Когда вы пытаетесь понять, что делает тот или иной код, это запросто может быть какая-то 
опенсорсная библиотека вроде libpng. Поэтому, когда находите константы, или текстовые строки, которые 
выглядят явно знакомыми, всегда полезно их \emph{погуглить}.
А если вы найдете искомый опенсорсный проект где это используется, 
то тогда будет достаточно будет просто сравнить вашу функцию с ней. 
Это решит часть проблем.

К примеру, если программа использует какие-то XML-файлы, первым шагом может быть
установление, какая именно XML-библиотека для этого используется, ведь часто используется какая-то
стандартная (или очень известная) вместо самодельной.

\myindex{SAP}
\myindex{Windows!PDB}
К примеру, автор этих строк однажды пытался разобраться как происходит компрессия/декомпрессия сетевых пакетов в SAP 6.0. 
Это очень большая программа, но к ней идет подробный .\gls{PDB}-файл с отладочной информацией, и это очень удобно. 
Он в конце концов пришел к тому что одна из функций декомпрессирующая пакеты называется CsDecomprLZC(). 
Не сильно раздумывая, он решил погуглить и оказалось, что функция с таким же названием имеется в MaxDB
(это опен-сорсный проект SAP) \footnote{Больше об этом в соответствующей секции~(\myref{sec:SAPGUI})}.

\url{http://www.google.com/search?q=CsDecomprLZC}

Каково же было мое удивление, когда оказалось, что в MaxDB используется точно такой же алгоритм, 
скорее всего, с таким же исходником.

\input{digging_into_code/identification/exec_RU}
% binary files might be also here

\mysection{Связь с внешним миром (на уровне функции)}

Очень желательно следить за аргументами ф-ции и возвращаемыми значениями, в отладчике или \ac{DBI}.
Например, автор этих строк однажды пытался понять значение некоторой очень запутанной ф-ции, которая, как потом оказалось,
была неверно реализованной пузырьковой сортировкой\footnote{\url{https://yurichev.com/blog/weird_sort_KLEE/}}.
(Она работала правильно, но медленнее.)
В то же время, наблюдение за входами и выходами этой ф-ции помогает мгновенно понять, что она делает.

Часто, когда вы видите деление через умножение (\myref{sec:divisionbymult}),
но забыли все детали о том, как оно работает, вы можете просто наблюдать за входом и выходом, и так быстро найти делитель.

% sections:
\mysection{Связь с внешним миром (win32)}

Иногда, чтобы понять, что делает та или иная функция, можно её не разбирать, а просто посмотреть на её входы и выходы.
Так можно сэкономить время.

Обращения к файлам и реестру: 
для самого простого анализа может помочь утилита Process Monitor\footnote{\url{http://technet.microsoft.com/en-us/sysinternals/bb896645.aspx}}
от SysInternals.

Для анализа обращения программы к сети, может помочь  Wireshark\footnote{\url{http://www.wireshark.org/}}.

Затем всё-таки придётся смотреть внутрь. \\
\\
Первое на что нужно обратить внимание, это какие функции из \ac{API} \ac{OS}
и какие функции стандартных библиотек используются.
Если программа поделена на главный исполняемый файл и группу DLL-файлов, то имена функций в этих DLL, бывает так, могут помочь.

Если нас интересует, что именно приводит к вызову \TT{MessageBox()} с определенным текстом, 
то первое, что можно попробовать сделать: найти в сегменте данных этот текст, найти ссылки на него, и найти, 
откуда может передаться управление к интересующему нас вызову \TT{MessageBox()}.

\myindex{\CStandardLibrary!rand()}
Если речь идет о компьютерной игре, и нам интересно какие события в ней более-менее случайны, 
мы можем найти функцию \rand или её заменитель (как алгоритм Mersenne twister), и посмотреть, 
из каких мест эта функция вызывается и что самое главное: как используется результат этой функции.%

% BUG in varioref: http://tex.stackexchange.com/questions/104261/varioref-vref-or-vpageref-at-page-boundary-may-loop
Один пример: \myref{chap:color_lines}. 

Но если это не игра, а \rand используется, то также весьма любопытно, зачем. 
Бывают неожиданные случаи вроде использования \rand в алгоритме для сжатия данных (для имитации шифрования):
\href{http://blog.yurichev.com/node/44}{blog.yurichev.com}.

\subsection{Часто используемые функции Windows API}

Это функции которые можно увидеть в числе импортируемых.
Но также нельзя забывать, что далеко не все они были использованы в коде написанном автором.
Немалая часть может вызываться из библиотечных функций и \ac{CRT}-кода.
	
Многие ф-ции могут иметь суффикс \GTT{-A} для ASCII-версии и \GTT{-W} для Unicode-версии.

\begin{itemize}

\item
Работа с реестром (advapi32.dll): 
RegEnumKeyEx, RegEnumValue, RegGetValue, RegOpenKeyEx, RegQueryValueEx.

\item
Работа с текстовыми .ini-файлами (kernel32.dll):\\
GetPrivateProfileString.

\item
Диалоговые окна (user32.dll):\\
MessageBox, MessageBoxEx, CreateDialog, SetDlgItemText, GetDlgItemText.

\item
Работа с ресурсами (\myref{PEresources}): (user32.dll):
LoadMenu.

\item
Работа с TCP/IP-сетью (ws2\_32.dll):
WSARecv, WSASend.

\item
Работа с файлами (kernel32.dll):
CreateFile, ReadFile, ReadFileEx, WriteFile, WriteFileEx.

\item
Высокоуровневая работа с Internet
(wininet.dll):
WinHttpOpen.

\item
Проверка цифровой подписи исполняемого файла (wintrust.dll):
WinVerifyTrust.

\item
Стандартная библиотека MSVC (в случае динамического связывания)%
 (msvcr*.dll):
assert, itoa, ltoa, open, printf, read, strcmp, atol, atoi, fopen, fread, fwrite, memcmp, rand,
strlen, strstr, strchr.

\end{itemize}

\subsection{Расширение триального периода}

Ф-ции доступа к реестру это частая цель тех, кто пытается расширить триальный период ПО, которое
может сохранять дату/время инсталляции в реестре.

Другая популярная цель это ф-ции GetLocalTime() и GetSystemTime():
триальное ПО, при каждом запуске, должно как-то проверять текущую дату/время.

% FIXME language!
\subsection{Удаление nag-окна}

Популярный метод поиска того, что заставляет выводить nag-окно это перехват ф-ций MessageBox(),
CreateDialog() и CreateWindow().

\subsection{tracer: Перехват всех функций в отдельном модуле}
\myindex{tracer}

\myindex{x86!\Instructions!INT3}
В \tracer есть поддержка точек останова INT3, хотя и срабатывающие только один раз, но зато их можно установить на все
сразу функции в некоей DLL.

\begin{lstlisting}
--one-time-INT3-bp:somedll.dll!.*
\end{lstlisting}

Либо, поставим INT3-прерывание на все функции, имена которых начинаются с префикса \TT{xml}:

\begin{lstlisting}
--one-time-INT3-bp:somedll.dll!xml.*
\end{lstlisting}

В качестве обратной стороны медали, такие прерывания срабатывают только один раз.
Tracer покажет вызов какой-либо функции, если он случится, но только один раз.
Еще один недостаток --- увидеть аргументы функции также нельзя.

Тем не менее, эта возможность очень удобна для тех ситуаций, 
когда вы знаете что некая программа использует некую DLL,
но не знаете какие именно функции в этой DLL.

И функций много. 

\par
\myindex{Cygwin}
Например, попробуем узнать, что использует cygwin-утилита uptime:

\begin{lstlisting}
tracer -l:uptime.exe --one-time-INT3-bp:cygwin1.dll!.*
\end{lstlisting}

Так мы можем увидеть все функции из библиотеки cygwin1.dll, которые были вызваны хотя бы один раз, и откуда:

\lstinputlisting{digging_into_code/uptime_cygwin.txt}


\mysection{Строки}
\label{sec:digging_strings}

\subsection{Текстовые строки}

\subsubsection{\CCpp}

\label{C_strings}
Обычные строки в Си заканчиваются нулем (\ac{ASCIIZ}-строки).

Причина, почему формат строки в Си именно такой (оканчивающийся нулем) вероятно историческая.
В [Dennis M. Ritchie, \emph{The Evolution of the Unix Time-sharing System}, (1979)]
мы можем прочитать:

\begin{framed}
\begin{quotation}
A minor difference was that the unit of I/O was the word, not the byte, because the PDP-7 was a word-addressed
machine. In practice this meant merely that all programs dealing with character streams ignored null
characters, because null was used to pad a file to an even number of characters.
\end{quotation}
\end{framed}

\myindex{Hiew}
Строки выглядят в Hiew или FAR Manager точно так же:

\begin{lstlisting}[style=customc]
int main()
{
	printf ("Hello, world!\n");
};
\end{lstlisting}

\begin{figure}[H]
\centering
\includegraphics[width=0.6\textwidth]{digging_into_code/strings/C-string.png}
\caption{Hiew}
\end{figure}

% FIXME видно \n в конце, потом пробел

\subsubsection{Borland Delphi}
\myindex{Pascal}
\myindex{Borland Delphi}
Когда кодируются строки в Pascal и Delphi, сама строка предваряется 8-битным или 32-битным значением, в котором закодирована длина строки.

Например:

\begin{lstlisting}[caption=Delphi,style=customasmx86]
CODE:00518AC8                 dd 19h
CODE:00518ACC aLoading___Plea db 'Loading... , please wait.',0

...

CODE:00518AFC                 dd 10h
CODE:00518B00 aPreparingRun__ db 'Preparing run...',0
\end{lstlisting}

\subsubsection{Unicode}

\myindex{Unicode}
Нередко уникодом называют все способы передачи символов, когда символ занимает 2 байта или 16 бит.
Это распространенная терминологическая ошибка.
Уникод --- это стандарт, присваивающий номер каждому символу многих письменностей мира, но не описывающий
способ кодирования.

\myindex{UTF-8}
\myindex{UTF-16LE}
Наиболее популярные способы кодирования: 
UTF-8 (наиболее часто используется в Интернете и *NIX-системах) и UTF-16LE (используется в Windows).

\myparagraph{UTF-8}

\myindex{UTF-8}
UTF-8 это один из очень удачных способов кодирования символов.
Все символы латиницы кодируются так же, как и в ASCII-кодировке, а символы, выходящие за пределы
ASCII-7-таблицы, кодируются несколькими байтами.
0 кодируется, как и прежде, нулевым байтом, так что все стандартные
функции Си продолжают работать с UTF-8-строками так же как и с обычными строками.

Посмотрим, как символы из разных языков кодируются в UTF-8 и как это выглядит в FAR, в кодировке 437

\footnote{Пример и переводы на разные языки были взяты здесь: 
\url{http://www.columbia.edu/~fdc/utf8/}}:

\begin{figure}[H]
\centering
\includegraphics[width=0.6\textwidth]{digging_into_code/strings/multilang_sampler.png}
\end{figure}

% FIXME: cut it
\begin{figure}[H]
\centering
\myincludegraphics{digging_into_code/strings/multilang_sampler_UTF8.png}
\caption{FAR: UTF-8}
\end{figure}

Видно, что строка на английском языке выглядит точно так же, как и в ASCII-кодировке.
В венгерском языке используются латиница плюс латинские буквы с диакритическими знаками.
Здесь видно, что эти буквы кодируются несколькими байтами, они подчеркнуты красным.
То же самое с исландским и польским языками.
В самом начале имеется также символ валюты \q{Евро}, который кодируется тремя байтами.
Остальные системы письма здесь никак не связаны с латиницей.
По крайней мере о русском, арабском, иврите и хинди мы можем сказать, что здесь видны повторяющиеся
байты, что не удивительно, ведь, обычно буквы из одной и той же системы письменности расположены в одной
или нескольких таблицах уникода, поэтому часто их коды начинаются с одних и тех же цифр.

В самом начале, перед строкой \q{How much?}, видны три байта, которые на самом деле \ac{BOM}.
\ac{BOM} показывает, какой способ кодирования будет сейчас использоваться.

\myparagraph{UTF-16LE}

\myindex{UTF-16LE}
\myindex{Windows!Win32}
Многие функции win32 в Windows имеют суффикс \TT{-A} и \TT{-W}.
Первые функции работают с обычными строками, вторые с UTF-16LE-строками (\emph{wide}).
Во втором случае, каждый символ обычно хранится в 16-битной переменной типа \emph{short}.

Cтроки с латинскими буквами выглядят в Hiew или FAR как перемежающиеся с нулевыми байтами:

\begin{lstlisting}[style=customc]
int wmain()
{
	wprintf (L"Hello, world!\n");
};
\end{lstlisting}

\begin{figure}[H]
\centering
\includegraphics[width=0.6\textwidth]{digging_into_code/strings/UTF16-string.png}
\caption{Hiew}
\end{figure}

Подобное можно часто увидеть в системных файлах \gls{Windows NT}:

\begin{figure}[H]
\centering
\includegraphics[width=0.6\textwidth]{digging_into_code/strings/ntoskrnl_UTF16.png}
\caption{Hiew}
\end{figure}

\myindex{IDA}
В \IDA, уникодом называется именно строки с символами, занимающими 2 байта:

\begin{lstlisting}[style=customasmx86]
.data:0040E000 aHelloWorld:
.data:0040E000                 unicode 0, <Hello, world!>
.data:0040E000                 dw 0Ah, 0
\end{lstlisting}

Вот как может выглядеть строка на русском языке (\q{И снова здравствуйте!}) закодированная в UTF-16LE:

\begin{figure}[H]
\centering
\includegraphics[width=0.6\textwidth]{digging_into_code/strings/russian_UTF16.png}
\caption{Hiew: UTF-16LE}
\end{figure}

То что бросается в глаза --- это то что символы перемежаются ромбиками (который имеет код 4).
Действительно, в уникоде кирилличные символы находятся в четвертом блоке.
Таким образом, все кирилличные символы в UTF-16LE находятся в диапазоне \TT{0x400-0x4FF}.

Вернемся к примеру, где одна и та же строка написана разными языками.
Здесь посмотрим в кодировке UTF-16LE.

% FIXME: cut it
\begin{figure}[H]
\centering
\myincludegraphics{digging_into_code/strings/multilang_sampler_UTF16.png}
\caption{FAR: UTF-16LE}
\end{figure}

Здесь мы также видим \ac{BOM} в самом начале.
Все латинские буквы перемежаются с нулевыми байтами.
Некоторые буквы с диакритическими знаками (венгерский и исландский языки) также подчеркнуты красным.

% subsection:
\input{digging_into_code/strings/base64_RU}



\subsection{Поиск строк в бинарном файле}

\epigraph{Actually, the best form of Unix documentation is frequently running the
\textbf{strings} command over a program’s object code. Using \textbf{strings}, you can get
a complete list of the program’s hard-coded file name, environment variables,
undocumented options, obscure error messages, and so forth.}{The Unix-Haters Handbook}

\myindex{UNIX!strings}
Стандартная утилита в UNIX \emph{strings} это самый простой способ увидеть строки в файле.
Например, это строки найденные в исполняемом файле sshd из OpenSSH 7.2:

\lstinputlisting{digging_into_code/sshd_strings.txt}

Тут опции, сообщения об ошибках, пути к файлам, импортируемые модули, функции, и еще какие-то странные строки (ключи?)
Присутствует также нечитаемый шум---иногда в x86-коде бывают целые куски состоящие из печатаемых ASCII-символов,
вплоть до ~8 символов.

Конечно, OpenSSH это опенсорсная программа.
Но изучение читаемых строк внутри некоторого неизвестного бинарного файла это зачастую самый первый шаг в анализе.
\myindex{UNIX!grep}

Также можно использовать \emph{grep}.

\myindex{Hiew}
\myindex{Sysinternals}
В Hiew есть такая же возможность (Alt-F6), также как и в Sysinternals ProcessMonitor.

\subsection{Сообщения об ошибках и отладочные сообщения}

Очень сильно помогают отладочные сообщения, если они имеются. В некотором смысле, отладочные сообщения, 
это отчет о том, что сейчас происходит в программе.
Зачастую, это \printf-подобные функции, 
которые пишут куда-нибудь в лог, а бывает так что и не пишут ничего, но вызовы остались, так как эта сборка --- не
отладочная, а \emph{release}.

\myindex{\oracle}
Если в отладочных сообщениях дампятся значения некоторых локальных или глобальных переменных, 
это тоже может помочь, как минимум, узнать их имена. 
Например, в \oracle одна из таких функций: \TT{ksdwrt()}.

Осмысленные текстовые строки вообще очень сильно могут помочь. 
Дизассемблер \IDA может сразу указать, из какой функции и из какого её места используется эта строка. 
Встречаются и смешные случаи
\footnote{\href{http://blog.yurichev.com/node/32}{blog.yurichev.com}}.

Сообщения об ошибках также могут помочь найти то что нужно. 
В \oracle сигнализация об ошибках проходит при помощи вызова некоторой группы функций. \\
Тут еще немного об этом: \href{http://blog.yurichev.com/node/43}{blog.yurichev.com}.

\myindex{Error messages}
Можно довольно быстро найти, какие функции сообщают о каких ошибках, и при каких условиях.

Это, кстати, одна из причин, почему в защите софта от копирования, 
бывает так, что сообщение об ошибке заменяется 
невнятным кодом или номером ошибки. Мало кому приятно, если взломщик быстро поймет, 
из-за чего именно срабатывает защита от копирования, просто по сообщению об ошибке.

Один из примеров шифрования сообщений об ошибке, здесь: \myref{examples_SCO}.

\subsection{Подозрительные магические строки}

Некоторые магические строки, используемые в бэкдорах выглядят очень подозрительно.
Например, в домашних роутерах TP-Link WR740 был бэкдор
\footnote{\url{http://sekurak.pl/tp-link-httptftp-backdoor/}, на русском: \url{http://m.habrahabr.ru/post/172799/}}.
Бэкдор активировался при посещении следующего URL:\\
\url{http://192.168.0.1/userRpmNatDebugRpm26525557/start_art.html}.\\
Действительно, строка \q{userRpmNatDebugRpm26525557} присутствует в прошивке.

Эту строку нельзя было нагуглить до распространения информации о бэкдоре.

Вы не найдете ничего такого ни в одном \ac{RFC}.

Вы не найдете ни одного алгоритма, который бы использовал такие странные последовательности байт.

И это не выглядит как сообщение об ошибке, или отладочное сообщение.

Так что проверить использование подобных странных строк --- это всегда хорошая идея.
\\
\myindex{base64}
Иногда такие строки кодируются при помощи 
base64\footnote{Например, бэкдор в кабельном модеме Arris: 
\url{http://www.securitylab.ru/analytics/461497.php}}.
Так что неплохая идея их всех декодировать и затем просмотреть глазами, пусть даже бегло.
\\
\myindex{Security through obscurity}

Более точно, такой метод сокрытия бэкдоров называется \q{security through obscurity} (безопасность через
запутанность).

\input{digging_into_code/assert_RU}
\mysection{Константы}

Люди, включая программистов, часто используют круглые числа вроде 10, 100, 1000, в т.ч. и в коде.

Практикующие реверсеры, обычно, хорошо знают их в шестнадцатеричном представлении:
10=0xA, 100=0x64, 1000=0x3E8, 10000=0x2710.

Иногда попадаются константы \TT{0xAAAAAAAA} (0b10101010101010101010101010101010) и\\
\TT{0x55555555} (0b01010101010101010101010101010101) --- это чередующиеся биты.
Это помогает отличить некоторый сигнал от сигнала где все биты включены (0b1111 \dots) или выключены (0b0000 \dots).

Например, константа \TT{0x55AA} используется как минимум в бут-секторе, \ac{MBR}, 
и в \ac{ROM} плат-расширений IBM-компьютеров.

Некоторые алгоритмы, особенно криптографические, используют хорошо различимые константы, 
которые при помощи \IDA легко находить в коде.

\myindex{MD5}

Например, алгоритм MD5 инициализирует свои внутренние переменные так:

\begin{verbatim}
var int h0 := 0x67452301
var int h1 := 0xEFCDAB89
var int h2 := 0x98BADCFE
var int h3 := 0x10325476
\end{verbatim}

Если в коде найти использование этих четырех констант подряд --- очень высокая вероятность что эта функция имеет отношение к MD5.

\par
Еще такой пример это алгоритмы CRC16/CRC32, часто, алгоритмы вычисления контрольной суммы по CRC 
используют заранее заполненные таблицы, вроде:

\begin{lstlisting}[caption=linux/lib/crc16.c,style=customc]
/** CRC table for the CRC-16. The poly is 0x8005 (x^16 + x^15 + x^2 + 1) */
u16 const crc16_table[256] = {
	0x0000, 0xC0C1, 0xC181, 0x0140, 0xC301, 0x03C0, 0x0280, 0xC241,
	0xC601, 0x06C0, 0x0780, 0xC741, 0x0500, 0xC5C1, 0xC481, 0x0440,
	0xCC01, 0x0CC0, 0x0D80, 0xCD41, 0x0F00, 0xCFC1, 0xCE81, 0x0E40,
	...
\end{lstlisting}

См. также таблицу CRC32: \myref{sec:CRC32}.

В бестабличных алгоритмах CRC используются хорошо известные полиномы, например 0xEDB88320 для CRC32.

\subsection{Магические числа}
\label{magic_numbers}

Немало форматов файлов определяет стандартный заголовок файла где используются \emph{магическое число} (magic number), один или даже несколько.

\myindex{MS-DOS}
Скажем, все исполняемые файлы для Win32 и MS-DOS начинаются с двух символов \q{MZ}.

\myindex{MIDI}
В начале MIDI-файла должно быть \q{MThd}. Если у нас есть использующая для чего-нибудь MIDI-файлы программа,
наверняка она будет проверять MIDI-файлы на правильность хотя бы проверяя первые 4 байта.

Это можно сделать при помощи:
(\emph{buf} указывает на начало загруженного в память файла)

\begin{lstlisting}[style=customasmx86]
cmp [buf], 0x6468544D ; "MThd"
jnz _error_not_a_MIDI_file
\end{lstlisting}

\myindex{\CStandardLibrary!memcmp()}
\myindex{x86!\Instructions!CMPSB}
\dots либо вызвав функцию сравнения блоков памяти \TT{memcmp()} или любой аналогичный код, 
вплоть до инструкции \TT{CMPSB} (\myref{REPE_CMPSx}).

Найдя такое место мы получаем как минимум информацию о том, где начинается загрузка MIDI-файла, во-вторых, 
мы можем увидеть где располагается буфер с содержимым файла, и что еще оттуда берется, и как используется.

\subsubsection{Даты}

\myindex{UFS2}
\myindex{FreeBSD}
\myindex{HASP}

Часто, можно встретить число вроде \TT{0x19870116}, которое явно выглядит как дата (1987-й год, 1-й месяц (январь), 16-й день).
Это может быть чей-то день рождения (программиста, его/её родственника, ребенка), либо какая-то другая важная дата.
Дата может быть записана и в другом порядке, например \TT{0x16011987}.
Даты в американском стиле также популярны, например \TT{0x01161987}.

Известный пример это \TT{0x19540119} (магическое число используемое в структуре суперблока UFS2), это день рождения Маршала Кирка МакКузика, видного разработчика FreeBSD.

\myindex{Stuxnet}
В Stuxnet используется число ``19790509'' (хотя и не как 32-битное число, а как строка), и это привело к догадкам,
что этот зловред связан с Израелем\footnote{Это дата казни персидского еврея Habib Elghanian-а}.

Также, числа вроде таких очень популярны в любительской криптографии, например, это отрывок из внутренностей \emph{секретной функции} донглы HASP3
\footnote{\url{https://web.archive.org/web/20160311231616/http://www.woodmann.com/fravia/bayu3.htm}}:

\begin{lstlisting}[style=customc]
void xor_pwd(void) 
{ 
	int i; 
	
	pwd^=0x09071966;
	for(i=0;i<8;i++) 
	{ 
		al_buf[i]= pwd & 7; pwd = pwd >> 3; 
	} 
};

void emulate_func2(unsigned short seed)
{ 
	int i, j; 
	for(i=0;i<8;i++) 
	{ 
		ch[i] = 0; 
		
		for(j=0;j<8;j++)
		{ 
			seed *= 0x1989; 
			seed += 5; 
			ch[i] |= (tab[(seed>>9)&0x3f]) << (7-j); 
		}
	} 
}
\end{lstlisting}

\subsubsection{DHCP}

Это касается также и сетевых протоколов. 
Например, сетевые пакеты протокола DHCP содержат так называемую \emph{magic cookie}: \TT{0x63538263}. 
Какой-либо код, генерирующий пакеты по протоколу DHCP где-то и как-то должен внедрять в пакет также и эту константу. 
Найдя её в коде мы сможем найти место где происходит это и не только это. 
Любая программа, получающая DHCP-пакеты, должна где-то как-то проверять \emph{magic cookie}, 
сравнивая это поле пакета с константой.

Например, берем файл dhcpcore.dll из Windows 7 x64 и ищем эту константу. 
И находим, два раза: оказывается, эта константа используется в функциях с красноречивыми названиями \\
\TT{DhcpExtractOptionsForValidation()} и \TT{DhcpExtractFullOptions()}:

\begin{lstlisting}[caption=dhcpcore.dll (Windows 7 x64),style=customasmx86]
.rdata:000007FF6483CBE8 dword_7FF6483CBE8 dd 63538263h          ; DATA XREF: DhcpExtractOptionsForValidation+79
.rdata:000007FF6483CBEC dword_7FF6483CBEC dd 63538263h          ; DATA XREF: DhcpExtractFullOptions+97
\end{lstlisting}

А вот те места в функциях где происходит обращение к константам:

\begin{lstlisting}[caption=dhcpcore.dll (Windows 7 x64),style=customasmx86]
.text:000007FF6480875F  mov     eax, [rsi]
.text:000007FF64808761  cmp     eax, cs:dword_7FF6483CBE8
.text:000007FF64808767  jnz     loc_7FF64817179
\end{lstlisting}

И:

\begin{lstlisting}[caption=dhcpcore.dll (Windows 7 x64),style=customasmx86]
.text:000007FF648082C7  mov     eax, [r12]
.text:000007FF648082CB  cmp     eax, cs:dword_7FF6483CBEC
.text:000007FF648082D1  jnz     loc_7FF648173AF
\end{lstlisting}

\subsection{Специфические константы}

Иногда, бывают какие-то специфические константы для некоторого типа кода.
Например, однажды автор сих строк пытался разобраться с кодом, где подозрительно часто встречалось число 12.
Размеры многих массивов также были 12, или кратные 12 (24, итд).
Оказалось, этот код брал на вход 12-канальный аудиофайл и обрабатывал его.

И наоборот: например, если программа работает с текстовым полем длиной 120 байт, значит где-то в коде должна
быть константа 120, или 119.
Если используется UTF-16, то тогда $2 \cdot 120$.
Если код работает с сетевыми пакетами фиксированной длины, то хорошо бы и такую константу поискать в коде.

Это также справедливо для любительской криптографии (ключи с лицензией, итд).
Если зашифрованный блок имеет размер в $n$ байт, вы можете попробовать поискать это число в коде.
Также, если вы видите фрагмент кода, который при исполнении, повторяется $n$ раз в цикле,
это может быть ф-ция шифрования/дешифрования.

\subsection{Поиск констант}

В \IDA это очень просто, Alt-B или Alt-I.

\myindex{binary grep}
А для поиска константы в большом количестве файлов, либо для поиска их в неисполняемых файлах, имеется небольшая утилита
\emph{binary grep}\footnote{\BGREPURL}.


\subsection{Инструкции}

В ARM имеется также для некоторых инструкций суффикс \emph{-S}, указывающий, 
что эта инструкция будет модифицировать флаги.
Инструкции без этого суффикса не модифицируют флаги.
\myindex{ARM!\Instructions!ADD}
\myindex{ARM!\Instructions!ADDS}
\myindex{ARM!\Instructions!CMP}
Например, инструкция \TT{ADD} в отличие от \TT{ADDS}
сложит два числа, но флаги не изменит.
Такие инструкции удобно использовать
между \CMP где выставляются флаги и, например, инструкциями перехода, где флаги используются.
Они также лучше в смысле анализа зависимостей данных (data dependency analysis) 
(потому что меньшее количество регистров модифицируется во время исполнения).

% ADD
% ADDAL
% ADDCC
% ADDS
% ADR
% ADREQ
% ADRGT
% ADRHI
% ADRNE
% ASRS
% B
% BCS
% BEQ
% BGE
% BIC
% BL
% BLE
% BLEQ
% BLGT
% BLHI
% BLS
% BLT
% BLX
% BNE
% BX
% CMP
% IDIV
% IT
% LDMCSFD
% LDMEA
% LDMED
% LDMFA
% LDMFD
% LDMGEFD
% LDR.W
% LDR
% LDRB.W
% LDRB
% LDRSB
% LSL.W
% LSL
% LSLS
% MLA
% MOV
% MOVT.W
% MOVT
% MOVW
% MULS
% MVNS
% ORR
% POP
% PUSH
% RSB
% SMMUL
% STMEA
% STMED
% STMFA
% STMFD
% STMIA
% STMIB
% STR
% SUB
% SUBEQ
% SXTB
% TEST
% TST
% VADD
% VDIV
% VLDR
% VMOV
% VMOVGT
% VMRS
% VMUL
%\myindex{ARM!Optional operators!ASR
%\myindex{ARM!Optional operators!LSL
%\myindex{ARM!Optional operators!LSR
%\myindex{ARM!Optional operators!ROR
%\myindex{ARM!Optional operators!RRX

% AArch64
% RET is BR X30 or BR LR but with additional hint to CPU

\subsubsection{Таблица условных кодов}

% TODO rework this!
\small
\begin{center}
\begin{tabular}{ | l | l | l | }
\hline
\HeaderColor Код & 
\HeaderColor Описание & 
\HeaderColor Флаги \\
\hline
EQ                              & равно                                                         & Z == 1 \\
\hline
NE                              & не равно                                                      & Z == 0 \\
\hline
CS \ac{AKA} HS (Higher or Same) & перенос / беззнаковое, больше или равно                       & C == 1 \\
\hline
CC \ac{AKA} LO (LOwer)          & нет переноса / беззнаковое, меньше чем                        & C == 0 \\
\hline
MI                              & минус, отрицательный знак / меньше чем                        & N == 1 \\
\hline
PL                              & плюс, положительный знак или ноль /                           & N == 0 \\
                                & больше чем или равно                                          & \\
\hline
VS                              & переполнение & V == 1 \\
\hline
VC                              & нет переполнения & V == 0 \\
\hline
HI                              & беззнаковое, больше чем                                       & C == 1 и \\
                                &                                                               & Z == 0 \\
\hline
LS                              & беззнаковое, меньше или равно                                 & C == 0 или \\
                                &                                                               & Z == 1 \\
\hline
GE                              & знаковое, больше чем или равно                                & N == V \\
\hline
LT                              & знаковое, меньше чем                                          & N != V \\
\hline
GT                              & знаковое, больше чем                                          & Z == 0 и \\
                                &                                                               & N == V \\
\hline
LE                              & знаковое, меньше чем или равно                                & Z == 1 или \\
                                &                                                               & N != V \\
\hline
Нету / AL                       & Всегда                                                        & Любые \\
\hline
\end{tabular}
\end{center}
\normalsize


\mysection{Подозрительные паттерны кода}

\subsection{Инструкции XOR}
\myindex{x86!\Instructions!XOR}

Инструкции вроде \TT{XOR op, op} (например, \TT{XOR EAX, EAX}) 
обычно используются для обнуления регистра,
однако, если операнды разные, то применяется операция именно \q{исключающего или}.
Эта операция очень редко применяется в обычном программировании, но применяется очень часто в криптографии,
включая любительскую.

Особенно подозрительно, если второй операнд --- это большое число.
Это может указывать на шифрование, вычисление контрольной суммы, итд.  \\
\\
Одно из исключений из этого наблюдения о котором стоит сказать, то, что генерация и проверка значения \q{канарейки}
(\myref{subsec:BO_protection}) часто происходит, используя инструкцию \XOR.  \\
\\
\myindex{AWK}
Этот AWK-скрипт можно использовать для обработки листингов (.lst) созданных \IDA{}:

\lstinputlisting{digging_into_code/awk.sh}

Нельзя также забывать, что подобный скрипт может захватить и неверно дизассемблированный код 
(\myref{sec:incorrectly_disasmed_code}).

\subsection{Вручную написанный код на ассемблере}

\myindex{Function prologue}
\myindex{Function epilogue}
\myindex{x86!\Instructions!LOOP}
\myindex{x86!\Instructions!RCL}
Современные компиляторы не генерируют инструкции \TT{LOOP} и \TT{RCL}. 
С другой стороны, эти инструкции хорошо знакомы кодерам, предпочитающим писать прямо на ассемблере. 
Подобные инструкции отмечены как (M) в списке инструкций в приложении: 
\myref{sec:x86_instructions}.
Если такие инструкции встретились, можно сказать с какой-то вероятностью, что этот фрагмент кода написан вручную.

\par
Также, пролог/эпилог функции обычно не встречается в ассемблерном коде, написанном вручную.

\par
Как правило, во вручную написанном коде, нет никакого четкого метода передачи аргументов в функцию.

\par
Пример из ядра Windows 2003 (файл ntoskrnl.exe):

\lstinputlisting[style=customasmx86]{digging_into_code/ntoskrnl.lst}

Действительно, если заглянуть в исходные коды 
\ac{WRK} v1.2, данный код можно найти в файле \\
\emph{WRK-v1.2\textbackslash{}base\textbackslash{}ntos\textbackslash{}ke\textbackslash{}i386\textbackslash{}cpu.asm}.

\mysection{Использование magic numbers для трассировки}

Нередко бывает нужно узнать, как используется то или иное значение, прочитанное из файла либо взятое из пакета,
принятого по сети. Часто, ручное слежение за нужной переменной это трудный процесс. Один из простых методов (хотя и не
полностью надежный на 100\%) это использование вашей собственной \emph{magic number}.

Это чем-то напоминает компьютерную томографию: пациенту перед сканированием вводят в кровь 
рентгеноконтрастный препарат, хорошо отсвечивающий в рентгеновских лучах.
Известно, как кровь нормального человека
расходится, например, по почкам, и если в этой крови будет препарат, то при томографии будет хорошо видно,
достаточно ли хорошо кровь расходится по почкам и нет ли там камней, например, и прочих образований.

Мы можем взять 32-битное число вроде \TT{0x0badf00d}, либо чью-то дату рождения вроде \TT{0x11101979} 
и записать это, занимающее 4 байта число, в какое-либо место файла используемого исследуемой нами программой.

\myindex{\GrepUsage}
\myindex{tracer}
Затем, при трассировке этой программы, в том числе, при помощи \tracer в режиме 
\emph{code coverage}, а затем при помощи
\emph{grep} или простого поиска по текстовому файлу с результатами трассировки, мы можем легко увидеть, в каких местах кода использовалось 
это значение, и как.

Пример результата работы \tracer в режиме \emph{cc}, к которому легко применить утилиту \emph{grep}:

\begin{lstlisting}[style=customasmx86]
0x150bf66 (_kziaia+0x14), e=       1 [MOV EBX, [EBP+8]] [EBP+8]=0xf59c934 
0x150bf69 (_kziaia+0x17), e=       1 [MOV EDX, [69AEB08h]] [69AEB08h]=0 
0x150bf6f (_kziaia+0x1d), e=       1 [FS: MOV EAX, [2Ch]] 
0x150bf75 (_kziaia+0x23), e=       1 [MOV ECX, [EAX+EDX*4]] [EAX+EDX*4]=0xf1ac360 
0x150bf78 (_kziaia+0x26), e=       1 [MOV [EBP-4], ECX] ECX=0xf1ac360 
\end{lstlisting}
% TODO: good example!
Это справедливо также и для сетевых пакетов.
Важно только, чтобы наш \emph{magic number} был как можно более уникален и не присутствовал в самом коде.

\newcommand{\DOSBOXURL}{\href{http://blog.yurichev.com/node/55}{blog.yurichev.com}}

\myindex{DosBox}
\myindex{MS-DOS}
Помимо \tracer, такой эмулятор MS-DOS как DosBox, в режиме heavydebug, может писать в отчет информацию обо всех
состояниях регистра на каждом шаге исполнения программы\footnote{См. также мой пост в блоге об этой возможности в 
DosBox: \DOSBOXURL{}}, так что этот метод может пригодиться и для исследования программ под DOS.


\mysection{Циклы}

Когда ваша программа работает с некоторым файлом, или буфером некоторой длины,
внутри кода где-то должен быть цикл с дешифровкой/обработкой.

Вот реальный пример выхода инструмента \tracer.
Был код, загружающий некоторый зашифрованный файл размером 258 байт.
Я могу запустить его с целью подсчета, сколько раз была исполнена каждая инструкция (в наше время \ac{DBI} послужила бы куда лучше).
И я могу быстро найти место в коде, которое было исполнено 259/258 раз:

\lstinputlisting{digging_into_code/crypto_loop.txt}

Как потом оказалось, это цикл дешифрования.


% TODO move section...

\subsection{Некоторые паттерны в бинарных файлах}

Все примеры здесь были подготовлены в Windows с активной кодовой страницей 437
в консоли.
Двоичные файлы внутри могут визуально выглядеть иначе если установлена другая кодовая страница.

\clearpage
\subsubsection{Массивы}

Иногда мы можем легко заметить массив 16/32/64-битных значений визуально, в шестнадцатеричном 
редакторе.

Вот пример массива 16-битных значений.
Мы видим, что каждый первый байт в паре всегда равен 7 или 8, а второй выглядит случайным:

\begin{figure}[H]
\centering
\myincludegraphics{digging_into_code/binary/16bit_array.png}
\caption{FAR: массив 16-битных значений}
\end{figure}

Для примера я использовал файл содержащий 12-канальный сигнал оцифрованный при помощи 16-битного \ac{ADC}.

\clearpage
\myindex{MIPS}
\par А вот пример очень типичного MIPS-кода.

Как мы наверное помним, каждая инструкция в MIPS (а также в ARM в режиме ARM, или ARM64) имеет 
длину 32 бита (или 4 байта),
так что такой код это массив 32-битных значений.

Глядя на этот скриншот, можно увидеть некий узор.
Вертикальные красные линии добавлены для ясности:

\begin{figure}[H]
\centering
\myincludegraphics{digging_into_code/binary/typical_MIPS_code.png}
\caption{Hiew: очень типичный код для MIPS}
\end{figure}

Еще пример таких файлов в этой книге: 
\myref{Oracle_SYM_files_example}.

\clearpage
\subsubsection{Разреженные файлы}

Это разреженный файл, в котором данные разбросаны посреди почти пустого файла.
Каждый символ пробела здесь на самом деле нулевой байт (который выглядит как пробел).
Это файл для программирования FPGA (чип Altera Stratix GX).
Конечно, такие файлы легко сжимаются, но подобные форматы очень популярны в научном и инженерном ПО, где быстрый доступ важен, а компактность --- не очень.

\begin{figure}[H]
\centering
\myincludegraphics{digging_into_code/binary/sparse_FPGA.png}
\caption{FAR: Разреженный файл}
\end{figure}

\clearpage
\subsubsection{Сжатый файл}

% FIXME \ref{} ->
Этот файл это просто некий сжатый архив.
Он имеет довольно высокую энтропию и визуально выглядит просто хаотичным.
Так выглядят сжатые и/или зашифрованные файлы.

\begin{figure}[H]
\centering
\myincludegraphics{digging_into_code/binary/compressed.png}
\caption{FAR: Сжатый файл}
\end{figure}

\clearpage
\subsubsection{\ac{CDFS}}

Инсталляции \ac{OS} обычно распространяются в ISO-файлах, которые суть копии CD/DVD-дисков.
Используемая файловая система называется \ac{CDFS}, здесь видны имена файлов и какие-то допольнительные данные.
Это могут быть длины файлов, указатели на другие директории, атрибуты файлов, итд.
Так может выглядеть типичная файловая система внутри.

\begin{figure}[H]
\centering
\myincludegraphics{digging_into_code/binary/cdfs.png}
\caption{FAR: ISO-файл: инсталляционный \ac{CD} Ubuntu 15}
\end{figure}

\clearpage
\subsubsection{32-битный x86 исполняемый код}

Так выглядит 32-битный x86 исполняемый код.
У него не очень высокая энтропия, потому что некоторые байты встречаются чаще других.

\begin{figure}[H]
\centering
\myincludegraphics{digging_into_code/binary/x86_32.png}
\caption{FAR: Исполняемый 32-битных x86 код}
\end{figure}

% TODO: Read more about x86 statistics: \ref{}. % FIXME blog post about decryption...

\clearpage
\subsubsection{Графические BMP-файлы}

% TODO: bitmap, bit, group of bits...

BMP-файлы не сжаты, так что каждый байт (или группа байт) описывают каждый пиксель.
Я нашел эту картинку где-то внутри заинсталлированной Windows 8.1:

\begin{figure}[H]
\centering
\myincludegraphicsSmall{digging_into_code/binary/bmp.png}
\caption{Пример картинки}
\end{figure}

Вы видите, что эта картинка имеет пиксели, которые вряд ли могут быть хорошо сжаты (в районе центра),
но здесь есть длинные одноцветные линии вверху и внизу.
Действительно, линии вроде этих выглядят как линии при просмотре этого файла:

\begin{figure}[H]
\centering
\myincludegraphics{digging_into_code/binary/bmp_FAR.png}
\caption{Фрагмент BMP-файла}
\end{figure}


% FIXME comparison!
\subsection{Сравнение \q{снимков} памяти}
\label{snapshots_comparing}

Метод простого сравнения двух снимков памяти для поиска изменений часто применялся для взлома игр 
на 8-битных компьютерах и взлома файлов с записанными рекордными очками.

К примеру, если вы имеете загруженную игру на 8-битном компьютере (где самой памяти не очень много, но игра
занимает еще меньше), и вы знаете что сейчас у вас, условно, 100 пуль, вы можете сделать \q{снимок} всей
памяти и сохранить где-то. Затем просто стреляете куда угодно, у вас станет 99 пуль, сделать второй \q{снимок},
и затем сравнить эти два снимка: где-то наверняка должен быть байт, который в начале был 100, а затем стал 99.

Если учесть, что игры на тех маломощных домашних компьютерах обычно были написаны на ассемблере и подобные
переменные там были глобальные, то можно с уверенностью сказать, какой адрес в памяти всегда отвечает за количество
пуль. Если поискать в дизассемблированном коде игры все обращения по этому адресу, несложно найти код,
отвечающий за уменьшение пуль и записать туда инструкцию \gls{NOP}
или несколько \gls{NOP}-в, так мы получим игру в которой у игрока всегда будет 100 пуль, например.

\myindex{BASIC!POKE}
А так как игры на тех домашних 8-битных 
компьютерах всегда загружались по одним и тем же адресам, и версий одной игры редко когда было больше одной продолжительное время,
то геймеры-энтузиасты знали, по какому адресу (используя инструкцию языка BASIC \gls{POKE}) что записать после загрузки
игры, чтобы хакнуть её. Это привело к появлению списков \q{читов} состоящих из инструкций \gls{POKE}, публикуемых
в журналах посвященным 8-битным играм. См. также: \href{http://go.yurichev.com/17114}{wikipedia}.

\myindex{MS-DOS}
Точно так же легко модифицировать файлы с сохраненными рекордами (кто сколько очков набрал), впрочем, это может
сработать не только с 8-битными играми. Нужно заметить, какой у вас сейчас рекорд и где-то сохранить файл
с очками. Затем, когда очков станет другое количество, просто сравнить два файла, можно даже
DOS-утилитой FC\footnote{утилита MS-DOS для сравнения двух файлов побайтово} (файлы рекордов, часто, бинарные).

Где-то будут отличаться несколько байт, и легко будет увидеть, какие именно отвечают за количество очков. 
Впрочем, разработчики игр полностью осведомлены о таких хитростях и могут защититься от этого.

В каком-то смысле похожий пример в этой книге здесь: \myref{Millenium_DOS_game}.

% TODO: пример с какой-то простой игрушкой?

\subsubsection{Реальная история из 1999}

\myindex{ICQ}
В то время был популярен мессенджер ICQ, по крайней мере, в странах бывшего СССР.
У мессенджера была особенность --- некоторые пользователи не хотели, чтобы все знали, в онлайне они или нет.
И для начала у того пользователя нужно было запросить \emph{авторизацию}.
Тот человек мог разрешить вам видеть свой статус, а мог и не разрешить.

Автор сих строк сделал следующее.

\begin{itemize}
\item Добавил человека. Он появился в контакт-листе, в разделе ``wait for authorization''.
\item Выгрузил ICQ.
\item Сохранил базу ICQ в другом месте.
\item Загрузил ICQ снова.
\item Человек \emph{авторизировал}.
\item Выгрузил ICQ и сравнил две базы.
\end{itemize}

Выяснилось: базы отличались только одним байтом.
В первой версии: \verb|RESU\x03|, во второй \verb|RESU\x02|.
(``RESU'', надо думать, означало ``USER'', т.е., заголовок структуры, где хранилась информация о пользователе.)
Это означало, что информация об авторизации хранилась не на сервере, а в клиенте.
Вероятно, значение 2/3 отражало статус \emph{авторизированности}.

\subsubsection{Реестр Windows}

А еще можно вспомнить сравнение реестра Windows до инсталляции программы и после.
Это также популярный метод поиска, какие элементы реестра программа использует.

Наверное это причина, почему так популярны shareware-программы для очистки реестра в Windows.

Кстати, вот как сдампить реестр в Windows в текстовые файлы:

\begin{lstlisting}
reg export HKLM HKLM.reg
reg export HKCU HKCU.reg
reg export HKCR HKCR.reg
reg export HKU HKU.reg
reg export HKCC HKCC.reg
\end{lstlisting}

\myindex{UNIX!diff}
Затем их можно сравнивать используя diff...

\subsubsection{Инженерное ПО, CAD-ы, итд}

Если некое ПО использует закрытые (проприетарные) файлы, то и тут можно попытаться что-то выяснить.
Сохраняете файл.
Затем добавили точку, или линию, или еще какой примитив.
Сохранили файл, сравнили.
Или сдвинули точку в сторону, сохранили файл, сравнили.

\subsubsection{Блинк-компаратор}

Сравнение файлов или слепков памяти вообще, немного напоминает блинк-компаратор
\footnote{\url{http://go.yurichev.com/17349}}:
устройство, которое раньше использовали астрономы для поиска движущихся небесных объектов.

Блинк-компаратор позволял быстро переключаться между двух отснятых в разное время кадров,
и астроном мог увидеть разницу визуально.

Кстати, при помощи блинк-компаратора, в 1930 был открыт Плутон.


\mysection{Определение \ac{ISA}}
\label{ISA_detect}

Часто, вы можете иметь дело с бинарным файлом для неизвестной \ac{ISA}.
Вероятно, простейший способ определить \ac{ISA} это пробовать разные в IDA, objdump или другом дизассемблере.

Чтобы этого достичь, нужно понимать разницу между некорректно дизассемблированным кодом, и корректно дизассемблированным.

% subsection:
\renewcommand{\CURPATH}{digging_into_code/incorrect_disassembly}
\subsubsection{std::string}
\myindex{\Cpp!STL!std::string}
\label{std_string}

\myparagraph{Как устроена структура}

Многие строковые библиотеки \InSqBrackets{\CNotes 2.2} обеспечивают структуру содержащую ссылку 
на буфер собственно со строкой, переменную всегда содержащую длину строки 
(что очень удобно для массы функций \InSqBrackets{\CNotes 2.2.1}) и переменную содержащую текущий размер буфера.

Строка в буфере обыкновенно оканчивается нулем: это для того чтобы указатель на буфер можно было
передавать в функции требующие на вход обычную сишную \ac{ASCIIZ}-строку.

Стандарт \Cpp не описывает, как именно нужно реализовывать std::string,
но, как правило, они реализованы как описано выше, с небольшими дополнениями.

Строки в \Cpp это не класс (как, например, QString в Qt), а темплейт (basic\_string), 
это сделано для того чтобы поддерживать 
строки содержащие разного типа символы: как минимум \Tchar и \emph{wchar\_t}.

Так что, std::string это класс с базовым типом \Tchar.

А std::wstring это класс с базовым типом \emph{wchar\_t}.

\mysubparagraph{MSVC}

В реализации MSVC, вместо ссылки на буфер может содержаться сам буфер (если строка короче 16-и символов).

Это означает, что каждая короткая строка будет занимать в памяти по крайней мере $16 + 4 + 4 = 24$ 
байт для 32-битной среды либо $16 + 8 + 8 = 32$ 
байта в 64-битной, а если строка длиннее 16-и символов, то прибавьте еще длину самой строки.

\lstinputlisting[caption=пример для MSVC,style=customc]{\CURPATH/STL/string/MSVC_RU.cpp}

Собственно, из этого исходника почти всё ясно.

Несколько замечаний:

Если строка короче 16-и символов, 
то отдельный буфер для строки в \glslink{heap}{куче} выделяться не будет.

Это удобно потому что на практике, основная часть строк действительно короткие.
Вероятно, разработчики в Microsoft выбрали размер в 16 символов как разумный баланс.

Теперь очень важный момент в конце функции main(): мы не пользуемся методом c\_str(), тем не менее,
если это скомпилировать и запустить, то обе строки появятся в консоли!

Работает это вот почему.

В первом случае строка короче 16-и символов и в начале объекта std::string (его можно рассматривать
просто как структуру) расположен буфер с этой строкой.
\printf трактует указатель как указатель на массив символов оканчивающийся нулем и поэтому всё работает.

Вывод второй строки (длиннее 16-и символов) даже еще опаснее: это вообще типичная программистская ошибка 
(или опечатка), забыть дописать c\_str().
Это работает потому что в это время в начале структуры расположен указатель на буфер.
Это может надолго остаться незамеченным: до тех пока там не появится строка 
короче 16-и символов, тогда процесс упадет.

\mysubparagraph{GCC}

В реализации GCC в структуре есть еще одна переменная --- reference count.

Интересно, что указатель на экземпляр класса std::string в GCC указывает не на начало самой структуры, 
а на указатель на буфера.
В libstdc++-v3\textbackslash{}include\textbackslash{}bits\textbackslash{}basic\_string.h 
мы можем прочитать что это сделано для удобства отладки:

\begin{lstlisting}
   *  The reason you want _M_data pointing to the character %array and
   *  not the _Rep is so that the debugger can see the string
   *  contents. (Probably we should add a non-inline member to get
   *  the _Rep for the debugger to use, so users can check the actual
   *  string length.)
\end{lstlisting}

\href{http://gcc.gnu.org/onlinedocs/libstdc++/libstdc++-html-USERS-4.4/a01068.html}{исходный код basic\_string.h}

В нашем примере мы учитываем это:

\lstinputlisting[caption=пример для GCC,style=customc]{\CURPATH/STL/string/GCC_RU.cpp}

Нужны еще небольшие хаки чтобы сымитировать типичную ошибку, которую мы уже видели выше, из-за
более ужесточенной проверки типов в GCC, тем не менее, printf() работает и здесь без c\_str().

\myparagraph{Чуть более сложный пример}

\lstinputlisting[style=customc]{\CURPATH/STL/string/3.cpp}

\lstinputlisting[caption=MSVC 2012,style=customasmx86]{\CURPATH/STL/string/3_MSVC_RU.asm}

Собственно, компилятор не конструирует строки статически: да в общем-то и как
это возможно, если буфер с ней нужно хранить в \glslink{heap}{куче}?

Вместо этого в сегменте данных хранятся обычные \ac{ASCIIZ}-строки, а позже, во время выполнения, 
при помощи метода \q{assign}, конструируются строки s1 и s2
.
При помощи \TT{operator+}, создается строка s3.

Обратите внимание на то что вызов метода c\_str() отсутствует,
потому что его код достаточно короткий и компилятор вставил его прямо здесь:
если строка короче 16-и байт, то в регистре EAX остается указатель на буфер,
а если длиннее, то из этого же места достается адрес на буфер расположенный в \glslink{heap}{куче}.

Далее следуют вызовы трех деструкторов, причем, они вызываются только если строка длиннее 16-и байт:
тогда нужно освободить буфера в \glslink{heap}{куче}.
В противном случае, так как все три объекта std::string хранятся в стеке,
они освобождаются автоматически после выхода из функции.

Следовательно, работа с короткими строками более быстрая из-за м\'{е}ньшего обращения к \glslink{heap}{куче}.

Код на GCC даже проще (из-за того, что в GCC, как мы уже видели, не реализована возможность хранить короткую
строку прямо в структуре):

% TODO1 comment each function meaning
\lstinputlisting[caption=GCC 4.8.1,style=customasmx86]{\CURPATH/STL/string/3_GCC_RU.s}

Можно заметить, что в деструкторы передается не указатель на объект,
а указатель на место за 12 байт (или 3 слова) перед ним, то есть, на настоящее начало структуры.

\myparagraph{std::string как глобальная переменная}
\label{sec:std_string_as_global_variable}

Опытные программисты на \Cpp знают, что глобальные переменные \ac{STL}-типов вполне можно объявлять.

Да, действительно:

\lstinputlisting[style=customc]{\CURPATH/STL/string/5.cpp}

Но как и где будет вызываться конструктор \TT{std::string}?

На самом деле, эта переменная будет инициализирована даже перед началом \main.

\lstinputlisting[caption=MSVC 2012: здесь конструируется глобальная переменная{,} а также регистрируется её деструктор,style=customasmx86]{\CURPATH/STL/string/5_MSVC_p2.asm}

\lstinputlisting[caption=MSVC 2012: здесь глобальная переменная используется в \main,style=customasmx86]{\CURPATH/STL/string/5_MSVC_p1.asm}

\lstinputlisting[caption=MSVC 2012: эта функция-деструктор вызывается перед выходом,style=customasmx86]{\CURPATH/STL/string/5_MSVC_p3.asm}

\myindex{\CStandardLibrary!atexit()}
В реальности, из \ac{CRT}, еще до вызова main(), вызывается специальная функция,
в которой перечислены все конструкторы подобных переменных.
Более того: при помощи atexit() регистрируется функция, которая будет вызвана в конце работы программы:
в этой функции компилятор собирает вызовы деструкторов всех подобных глобальных переменных.

GCC работает похожим образом:

\lstinputlisting[caption=GCC 4.8.1,style=customasmx86]{\CURPATH/STL/string/5_GCC.s}

Но он не выделяет отдельной функции в которой будут собраны деструкторы: 
каждый деструктор передается в atexit() по одному.

% TODO а если глобальная STL-переменная в другом модуле? надо проверить.



\subsection{Корректно дизассемблированный код}
\label{correctly_disasmed_code}

Каждая \ac{ISA} имеет десяток самых используемых инструкций, остальные используются куда реже.

Интересно знать тот факт, что в x86, инструкции вызовов ф-ций (\PUSH/\CALL/\ADD) и \MOV
это наиболее часто исполняющиеся инструкции в коде почти во всем ПО что мы используем.
Другими словами, \ac{CPU} очень занят передачей информации между уровнями абстракции, или, можно сказать, очень занят
переключением между этими уровнями.
Вне зависимости от \ac{ISA}.
Это цена расслоения программ на разные уровни абстракций (чтобы человеку было легче с ними управляться).



\mysection{Прочее}

\subsection{Общая идея}

Нужно стараться как можно чаще ставить себя на место программиста и задавать себе вопрос, 
как бы вы сделали ту или иную вещь в этом случае и в этой программе.

\subsection{Порядок функций в бинарном коде}

Все функции расположеные в одном .c или .cpp файле компилируются в соответствующий объектный (.o) файл.
Линкер впоследствии складывает все нужные объектные файлы вместе, не меняя порядок ф-ций в них.
Как следствие, если вы видите в коде две или более идущих подряд ф-ций, то это означает, что и в исходном коде они 
были расположены в одном и том же файле (если только вы не на границе двух объектных файлов, конечно).
Это может означать, что эти ф-ции имеют что-то общее между собой, что они из одного слоя \ac{API}, из одной библиотеки, итд.

\myindex{CryptoPP}
Это реальная история из практики: однажды автор искал в прикомпилированной библиотеке CryptoPP ф-ции связанные
с алгоритмом Twofish, особенно шифрования/дешифрования.\\
Я нашел ф-цию \verb|Twofish::Base::UncheckedSetKey()|, но не остальные.
Заглянув в исходники \verb|twofish.cpp|
\footnote{\url{https://github.com/weidai11/cryptopp/blob/b613522794a7633aa2bd81932a98a0b0a51bc04f/twofish.cpp}}, стало ясно, что все ф-ции расположены в одном модуле (\verb|twofish.cpp|).
Так что я просто попробовал посмотреть ф-ции следующие за \\
\verb|Twofish::Base::UncheckedSetKey()| --- так и оказалось,\\
одна из них была \verb|Twofish::Enc::ProcessAndXorBlock()|, другая --- \verb|Twofish::Dec::ProcessAndXorBlock()|.

\subsection{Крохотные функции}

Крохотные ф-ции, такие как пустые ф-ции (\myref{empty_func})
или ф-ции возвращающие только ``true'' (1) или ``false'' (0) (\myref{ret_val_func}) очень часто встречаются,
и почти все современные компиляторы, как правило, помещают только одну такую ф-цию в исполняемый код,
даже если в исходном их было много одинаковых.
Так что если вы видите ф-цию состояющую только из \TT{mov eax, 1 / ret}, которая может вызываться из разных мест,
которые, судя по всему, друг с другом никак не связаны, это может быть результат подобной оптимизации.

\subsection{\Cpp}

\ac{RTTI}~(\myref{RTTI})-информация также может быть полезна для идентификации 
классов в \Cpp.

\subsection{Намеренный сбой}

Часто, нужно знать, какая ф-ция была исполнена, а какая --- нет.
Вы можете использовать отладчик, но на экзотических архитектурах его может и не быть, так что простейший способ это вписать
туда неверный опкод или что-то вроде \INS{INT3} (0xCC).
Сбой будет сигнализировать о том, что эта инструкция была исполнена.

Еще один пример намеренного сбоя: \myref{dmalloc_KILL_PROCESS}.

