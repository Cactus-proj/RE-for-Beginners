\mysection{Chaînes}
\label{sec:digging_strings}

\subsection{Chaînes de texte}

\subsubsection{\CCpp}

\label{C_strings}
Les chaînes C normales sont terminées par un zéro (chaînes \ac{ASCIIZ}).

La raison pour laquelle le format des chaînes C est ce qu'il est (terminé par zéro)
est apparemment historique:
Dans [Dennis M. Ritchie, \emph{The Evolution of the Unix Time-sharing System}, (1979)]
nous lisons:

\begin{framed}
\begin{quotation}
A minor difference was that the unit of I/O was the word, not the byte, because the PDP-7 was a word-addressed
machine. In practice this meant merely that all programs dealing with character streams ignored null
characters, because null was used to pad a file to an even number of characters.
\end{quotation}
\end{framed}
Une différence mineure était que l'unité d'E/S était le mot, pas l'octet, car le PDP-7 était une machine
adressée par mot. En pratique, cela signifiait que tous les programmes ayant à faire
avec des flux de caractères ignoraient le caractère nul, car nul était utilisé pour compléter un fichier
ayant un nombre impair de caractères.

\myindex{Hiew}

Dans Hiew ou FAR Manager ces chaînes ressemblent à ceci:

\begin{lstlisting}[style=customc]
int main()
{
	printf ("Hello, world!\n");
};
\end{lstlisting}

\begin{figure}[H]
\centering
\includegraphics[width=0.6\textwidth]{digging_into_code/strings/C-string.png}
\caption{Hiew}
\end{figure}

% FIXME видно \n в конце, потом пробел

\subsubsection{Borland Delphi}
\myindex{Pascal}
\myindex{Borland Delphi}

Une chaîne en Pascal et en Delphi de Borland est précédée par sa longueur sur 8-bit
ou 32-bit.

Par exemple:

\begin{lstlisting}[caption=Delphi,style=customasmx86]
CODE:00518AC8                 dd 19h
CODE:00518ACC aLoading___Plea db 'Loading... , please wait.',0

...

CODE:00518AFC                 dd 10h
CODE:00518B00 aPreparingRun__ db 'Preparing run...',0
\end{lstlisting}

\subsubsection{Unicode}

\myindex{Unicode}

Souvent, ce qui est appelé Unicode est la méthode pour encoder des chaînes où chaque
caractère occupe 2 octets ou 16 bits.
Ceci est une erreur de terminologie répandue.
Unicode est un standard pour assigner un nombre à chaque caractère dans un des nombreux
systèmes d'écriture dans le monde, mais ne décrit pas la méthode d'encodage.

\myindex{UTF-8}
\myindex{UTF-16LE}
Les méthodes d'encodage les plus répandues sont: UTF-8 (est répandue sur Internet
et les systèmes *NIX) et UTF-16LE (est utilisé dans Windows).

\myparagraph{UTF-8}

\myindex{UTF-8}
UTF-8 est l'une des méthodes les plus efficace pour l'encodage des caractères.
Tous les symboles Latin sont encodés comme en ASCII, et les symboles après la table
ASCII sont encodés en utilisant quelques octets.
0 est encodé comme avant, donc toutes les fonctions C de chaîne standard fonctionnent
avec des chaînes UTF-8 comme avec tout autre chaîne.

Voyons comment les symboles de divers langages sont encodés en UTF-8 et de quoi ils
ont l'air en FAR, en utilisant la page de code 437%
\footnote{L'exemple et les traductions ont été pris d'ici:
\url{http://go.yurichev.com/17304}}:

\begin{figure}[H]
\centering
\includegraphics[width=0.6\textwidth]{digging_into_code/strings/multilang_sampler.png}
\end{figure}

% FIXME: cut it
\begin{figure}[H]
\centering
\myincludegraphics{digging_into_code/strings/multilang_sampler_UTF8.png}
\caption{FAR: UTF-8}
\end{figure}

Comme vous le voyez, la chaîne en anglais est la même qu'en ASCII.

Le hongrois utilise certains symboles Latin et des symboles avec des signes diacritiques.

Ces symboles sont encodés en utilisant plusieurs octets, qui sont soulignés en rouge.
C'est le même principe avec l'islandais et le polonais.

Il y a aussi le symbole de l'\q{Euro} au début, qui est encodé avec 3 octets.

Les autres systèmes d'écritures n'ont de point commun avec Latin.

Au moins en russe, arabe hébreux et hindi, nous pouvons voir des octets récurrents,
et ce n'est pas une surprise: tous les symboles d'un système d'écriture sont en général
situés dans la même table Unicode, donc leur code débute par le même nombre.

Au début, avant la chaîne \q{How much?}, nous voyons 3 octets, qui sont en fait le
\ac{BOM}. Le \ac{BOM} défini le système d'encodage à utiliser.

\myparagraph{UTF-16LE}

\myindex{UTF-16LE}
\myindex{Windows!Win32}
De nombreuses fonctions win32 de Windows ont le suffixes \TT{-A} et \TT{-W}.
Le premier type de fonctions fonctionne avec les chaînes normales, l'autre, avec
des chaîne UTF-16LE (\emph{large}).

Dans le second cas, chaque symbole est en général stocké dans une valeur 16-bit de
type \emph{short}.

Les symboles Latin dans les chaînes UTF-16 dans Hiew ou FAR semblent être séparés
avec un octet zéro:

\begin{lstlisting}[style=customc]
int wmain()
{
	wprintf (L"Hello, world!\n");
};
\end{lstlisting}

\begin{figure}[H]
\centering
\includegraphics[width=0.6\textwidth]{digging_into_code/strings/UTF16-string.png}
\caption{Hiew}
\end{figure}

Nous voyons souvent ceci dans les fichiers système de \gls{Windows NT}:

\begin{figure}[H]
\centering
\includegraphics[width=0.6\textwidth]{digging_into_code/strings/ntoskrnl_UTF16.png}
\caption{Hiew}
\end{figure}

\myindex{IDA}
Les chaînes avec des caractères qui occupent exactement 2 octets sont appelées \q{Unicode}
dans \IDA:

\begin{lstlisting}[style=customasmx86]
.data:0040E000 aHelloWorld:
.data:0040E000                 unicode 0, <Hello, world!>
.data:0040E000                 dw 0Ah, 0
\end{lstlisting}

Voici comment une chaîne en russe est encodée en UTF-16LE:

\begin{figure}[H]
\centering
\includegraphics[width=0.6\textwidth]{digging_into_code/strings/russian_UTF16.png}
\caption{Hiew: UTF-16LE}
\end{figure}

Ce que nous remarquons facilement, c'est que les symboles sont intercalés par le
caractère diamant (qui a le code ASCII 4). En effet, les symboles cyrilliques sont
situés dans le quatrième plan Unicode.
Ainsi, tous les symboles cyrillique en UTF-16LE sont situés dans l'intervalle \TT{0x400-0x4FF}.

Retournons à l'exemple avec la chaîne écrite dans de multiple langages.
Voici à quoi elle ressemble en UTF-16LE.

% FIXME: cut it
\begin{figure}[H]
\centering
\myincludegraphics{digging_into_code/strings/multilang_sampler_UTF16.png}
\caption{FAR: UTF-16LE}
\end{figure}

Ici nous pouvons aussi voir le \ac{BOM} au début.
Tous les caractères Latin sont intercalés avec un octet à zéro.

Certains caractères avec signe diacritique (hongrois et islandais) sont aussi soulignés en rouge.

% subsection:
\subsubsection{Base64}
\myindex{Base64}

L'encodage base64 est très répandu dans les cas où vous devez transférer des données
binaires sous forme de chaîne de texte.

Pour l'essentiel, cet algorithme encode 3 octets binaires en 4 caractères imprimables:
toutes les 26 lettres Latin (à la fois minuscule et majuscule), chiffres, signe plus
(\q{+}) et signe slash (\q{/}), 64 caractères en tout.

Une particularité des chaînes base64 est qu'elles se terminent souvent (mais pas
toujours) par 1 ou 2 symbole égal (\q{=}) pour l'alignement, par exemple:

\begin{lstlisting}
AVjbbVSVfcUMu1xvjaMgjNtueRwBbxnyJw8dpGnLW8ZW8aKG3v4Y0icuQT+qEJAp9lAOuWs=
\end{lstlisting}

\begin{lstlisting}
WVjbbVSVfcUMu1xvjaMgjNtueRwBbxnyJw8dpGnLW8ZW8aKG3v4Y0icuQT+qEJAp9lAOuQ==
\end{lstlisting}

Le signe égal (\q{=}) ne se rencontre jamais au milieu des chaînes encodées en base64.

maintenant, un exemple d'encodage manuel.
Encodons les octets hexadécimaux 0x00, 0x11, 0x22, 0x33 en une chaîne base64:

\lstinputlisting{digging_into_code/strings/base64_ex.sh}

Mettons ces 4 octets au forme binaire, puis regroupons les dans des groupes de 6-bit:

\begin{lstlisting}
|  00  ||  11  ||  22  ||  33  ||      ||      |
00000000000100010010001000110011????????????????
| A  || B  || E  || i  || M  || w  || =  || =  |
\end{lstlisting}

Les trois premiers octets (0x00, 0x11, 0x22) peuvent être encodés dans 4 caractères
base64 (``ABEi''), mais le dernier (0x33) --- ne le peut pas, donc il est encodé
en utilisant deux caractères (``Mw'') et de symbole (``='') de padding est ajouté
deux fois pour compléter le dernier groupe à 4 caractères.
De ce fait, la longueur de toutes les chaînes en base64 correctes est toujours divisible
par 4.

\myindex{XML}
\myindex{PGP}
Base64 est souvent utilisé lorsque des données binaires doivent être stockées dans
du XML.  Les clefs PGP ``Armored'' (i.e., au format texte) et les signatures sont
encodées en utilisant base64.

Certains essayent d'utiliser base64 pour masquer des chaînes:
\url{http://blog.sec-consult.com/2016/01/deliberately-hidden-backdoor-account-in.html}%
\footnote{\url{http://archive.is/nDCas}}.

\myindex{base64scanner}
Il existe des utilitaires pour rechercher des chaînes base64 dans des fichiers binaires
arbitraires.
L'un d'entre eux est base64scanner\footnote{\url{https://github.com/DennisYurichev/base64scanner}}.

\myindex{UseNet}
\myindex{FidoNet}
\myindex{Uuencoding}
\myindex{Phrack}
Un autre système d'encodage qui était très répandu sur UseNet et FidoNet est l'Uuencoding.
Les fichiers binaires sont toujours encodés au format Uuencode dans le magazine Phrack.
Il offre à peu près la même fonctionnalité, mais il est différent de base64 dans
le sens où le nom de fichier est aussi stocké dans l'entête.

\myindex{Tor}
\myindex{base32}
À propos: base64 à un petit frère: base32, alphabet qui a ~10 chiffres et ~26 caractères Latin.
Un usage répandu est les adresses onion%
\footnote{\url{https://trac.torproject.org/projects/tor/wiki/doc/HiddenServiceNames}},
comme: \\
\url{http://3g2upl4pq6kufc4m.onion/}.
\ac{URL} ne peut pas avoir de mélange de casse de caractères Latin, donc, c'est apparemment
pourquoi les développeurs de Tor ont utilisé base32.





\subsection{Trouver des chaînes dans un binaire}

\epigraph{Actually, the best form of Unix documentation is frequently running the
\textbf{strings} command over a program’s object code. Using \textbf{strings}, you can get
a complete list of the program’s hard-coded file name, environment variables,
undocumented options, obscure error messages, and so forth.}{The Unix-Haters Handbook}
En fait, la meilleure forme de documentation Unix est de lancer la commande
\textbf{strings} sur le code objet d'un programme. En utilisant \textbf{strings},
vous obtenez une liste complète des noms de fichiers codés en dur dans le programme,
les variables d'environnement, les options non documentées, les messages d'erreurs
méconnus et ainsi de suite.

\myindex{UNIX!strings}
L'utilitaire standard UNIX \emph{strings} est un moyen rapide et facile de voir les
chaînes dans un fichier.
Par exemple, voici quelques chaînes du fichier exécutable sshd d'OpenSSH 7.2:

\lstinputlisting{digging_into_code/sshd_strings.txt}

Il y a des options, des messages d'erreur, des chemins de fichier, des modules et
des fonctions importés dynamiquement, ainsi que d'autres chaînes étranges (clefs?).
Il y a aussi du bruit illisible---le code x86 à parfois des fragments constitués de
caractères ASCII imprimables, jusqu'à ~8 caractères.

Bien sûr, OpenSSH est un programme open-source.
Mais regarder les chaînes lisibles dans un binaire inconnu est souvent une première
étape d'analyse.
\myindex{UNIX!grep}

\emph{grep} peut aussi être utilisé.

\myindex{Hiew}
\myindex{Sysinternals}
Hiew a la même capacité (Alt-F6), ainsi que ProcessMonitor de Sysinternals.

\subsection{Messages d'erreur/de débogage}

Les messages de débogage sont très utiles s'il sont présents.
Dans un certain sens, les messages de débogage rapportent ce qui est en train de
se passer dans le programme. Souvent, ce sont des fonctions \printf-like, qui écrivent
des fichiers de log, ou parfois elles n'écrivent rien du tout mais les appels sont
toujours présents puisque le build n'est pas un de débogage mais de \emph{release}.
\myindex{\oracle}

Si des variables locales ou globales sont affichées dans les messages, ça peut être
aussi utile, puisqu'il est possible d'obtenir au moins le nom de la variable.
Par exemple, une telle fonction dans \oracle est \TT{ksdwrt()}.

Des chaînes de texte significatives sont souvent utiles.
Le dés-assembleur \IDA peut montrer depuis quelles fonctions et depuis quel endroit
cette chaîne particulière est utilisée.
Des cas drôles arrivent parfois\footnote{\href{http://go.yurichev.com/17223}{blog.yurichev.com}}.

Le message d'erreur peut aussi nous aider.
Dans \oracle, les erreurs sont rapportées en utilisant un groupe de fonctions.\\
Vous pouvez en lire plus ici: \href{http://go.yurichev.com/17224}{blog.yurichev.com}.

\myindex{Error messages}

Il est possible de trouver rapidement quelle fonction signale une erreur et dans
quelles conditions.

À propos, ceci est souvent la raison pour laquelle les systèmes de protection contre
la copie utilisent des messages d'erreur inintelligibles ou juste des numéros d'erreur.
Personne n'est content lorsque le copieur de logiciel comprend comment fonctionne
la protection contre la copie seulement en lisant les messages d'erreur.

Un exemple de messages d'erreur chiffrés se trouve ici: \myref{examples_SCO}.

\subsection{Chaînes magiques suspectes}

Certaines chaînes magique sont d'habitude utilisées dans les porte dérobées semblent
vraiment suspectes.

Par exemple, il y avait une porte dérobée dans le routeur personnel TP-Link WR740%
\footnote{\url{http://sekurak.pl/tp-link-httptftp-backdoor/}}.
La porte dérobée était activée en utilisant l'URL suivante:\\
\url{http://192.168.0.1/userRpmNatDebugRpm26525557/start_art.html}.\\

En effet, la chaîne \q{userRpmNatDebugRpm26525557} est présente dans le firmware.

Cette chaîne n'était pas googlable jusqu'à la large révélation d'information concernant
la porte dérobée.

Vous ne trouverez ceci dans aucun \ac{RFC}.

Vous ne trouverez pas d'algorithme informatique qui utilise une séquence d'octets
aussi étrange.

Et elle ne ressemble pas à un erreur ou un message de débogage.

Donc, c'est une bonne idée d'inspecter l'utilisation de ce genre de chaînes bizarres.\\
\\
\myindex{base64}

Parfois, de telles chaînes sont encodées en utilisant base64.

Donc, c'est une bonne idée de les décoder toutes et de les inspecter visuellement,
même un coup d'\oe{}il doit suffire.\\
\\
\myindex{Sécurité par l'obscurité}
Plus précisément, cette méthode de cacher des accès non documentés est appelée \q{sécurité par l'obscurité}.

