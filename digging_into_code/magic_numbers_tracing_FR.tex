\mysection{Utilisation de nombres magiques lors du tracing}

Souvent, notre but principal est de comprendre comment le programme utilise une valeur
qui a été soit lue d'un fichier ou reçue par le réseau. Le tracing manuel d'une valeur
est souvent une tâche laborieuse. Une des techniques les plus simple pour ceci (bien
que non sûre à 100\%) est d'utiliser votre propre \emph{nombre magique}.

Ceci ressemble à la tomodensitométrie aux rayons X: un agent de radio-contraste est
injecté dans le sang du patient, qui est utilisé pour augmenter la visibilité de
la structure interne du patient aux rayons X.
C'est bien connu comment le sang circule dans les reins d'humains en bonne santé
et si l'agent est dans le sang, il peut être vu facilement en tomographie comment
le sang circule et si il y a des calculs ou des tumeurs.

Nous pouvons prendre un nombre 32-bit comme \TT{0x0badf00d}, ou la date de naissance
de quelqu'un comme \TT{0x11101979} et écrire ce nombre de 4 octets quelque part dans
un fichier utilisé par le programme que nous investiguons.

\myindex{\GrepUsage}
\myindex{tracer}

Puis, en suivant ce programme avec \tracer en mode \emph{code coverage}, avec l'aide
de \emph{grep} ou simplement en cherchant dans le fichier texte (résultant de l'investigation),
nous pouvons facilement voir où la valeur a été utilisée et comment.

Exemple de résultats de \tracer \emph{grepable} en mode \emph{cc}:

\begin{lstlisting}[style=customasmx86]
0x150bf66 (_kziaia+0x14), e=       1 [MOV EBX, [EBP+8]] [EBP+8]=0xf59c934
0x150bf69 (_kziaia+0x17), e=       1 [MOV EDX, [69AEB08h]] [69AEB08h]=0
0x150bf6f (_kziaia+0x1d), e=       1 [FS: MOV EAX, [2Ch]]
0x150bf75 (_kziaia+0x23), e=       1 [MOV ECX, [EAX+EDX*4]] [EAX+EDX*4]=0xf1ac360
0x150bf78 (_kziaia+0x26), e=       1 [MOV [EBP-4], ECX] ECX=0xf1ac360
\end{lstlisting}
% TODO: good example!

Cela peut aussi être utilisé pour des paquets réseau.
Il est important que le \emph{nombre magique} soit unique et ne soit pas présent
dans le code du programme.

\newcommand{\DOSBOXURL}{\href{http://go.yurichev.com/17222}{blog.yurichev.com}}

\myindex{DosBox}
\myindex{MS-DOS}
À part \tracer, DosBox (émulateur MS-DOS) en mode heavydebug est capable d'écrire
de l'information à propos de l'état de tous les registres pour chaque instruction
du programme exécutée dans un fichier texte\footnote{Voir aussi mon article de blog
sur cette fonctionnalité de DosBox: \DOSBOXURL{}}, donc cette technique peut être
utile également pour des programmes DOS.
