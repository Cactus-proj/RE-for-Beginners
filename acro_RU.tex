	\acro{OS}[ОС]{Операционная Система}
	\acro{FAQ}[ЧаВО]{Часто задаваемые вопросы}
	\acro{OOP}[ООП]{Объектно-Ориентированное Программирование}
	\acro{PL}[ЯП]{Язык Программирования}
	\acro{PRNG}[ГПСЧ]{Генератор псевдослучайных чисел}
	\acro{ROM}[ПЗУ]{Постоянное запоминающее устройство}
	\acro{ALU}[АЛУ]{Арифметико-логическое устройство}
	\acro{PID}{ID программы/процесса}
	\acro{LF}{Line feed (подача строки) (10 или '\textbackslash{}n' в \CCpp)}
	\acro{CR}{Carriage return (возврат каретки) (13 или '\textbackslash{}r' в \CCpp)}
	\acro{LIFO}{Last In First Out (последним вошел, первым вышел)}
	\acro{MSB}{Most significant bit (самый старший бит)} % NOT BYTE!
	\acro{LSB}{Least significant bit (самый младший бит)} % NOT BYTE!
	\acro{NSA}[АНБ]{Агентство национальной безопасности}
	\acro{CFB}{Режим обратной связи по шифротексту (Cipher Feedback)}
	\acro{CSPRNG}{Криптографически стойкий генератор псевдослучайных чисел (cryptographically secure pseudorandom number generator)}
        \acro{PC}{Program Counter. IP/EIP/RIP \InENRU x86/64. PC \InENRU ARM.}
        \acro{SP}{\gls{stack pointer}. SP/ESP/RSP \InENRU x86/x64. SP \InENRU ARM.}
