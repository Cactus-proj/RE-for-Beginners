\mysection{\SignedNumbersSectionName}
\label{sec:signednumbers}
\myindex{Signed numbers}

\newcommand{\URLS}{\href{http://go.yurichev.com/17117}{wikipedia}}

Il existe plusieurs méthodes pour représenter les nombres signées\footnote{\URLS},
mais le \q{complément à deux} est la plus populaire sur les ordinateurs.

Voici une table pour quelques valeurs d'octet:

\begin{center}
\begin{tabular}{ | l | l | l | l | }
\hline
\HeaderColor binaire & \HeaderColor hexadécimal & \HeaderColor non-signé & \HeaderColor signé \\
\hline
01111111 & 0x7f & 127 & 127 \\
\hline
01111110 & 0x7e & 126 & 126 \\
\hline
\multicolumn{4}{ |c| }{...} \\
\hline
00000110 & 0x6 & 6 & 6 \\
\hline
00000101 & 0x5 & 5 & 5 \\
\hline
00000100 & 0x4 & 4 & 4 \\
\hline
00000011 & 0x3 & 3 & 3 \\
\hline
00000010 & 0x2 & 2 & 2 \\
\hline
00000001 & 0x1 & 1 & 1 \\
\hline
00000000 & 0x0 & 0 & 0 \\
\hline
11111111 & 0xff & 255 & -1 \\
\hline
11111110 & 0xfe & 254 & -2 \\
\hline
11111101 & 0xfd & 253 & -3 \\
\hline
11111100 & 0xfc & 252 & -4 \\
\hline
11111011 & 0xfb & 251 & -5 \\
\hline
11111010 & 0xfa & 250 & -6 \\
\hline
\multicolumn{4}{ |c| }{...} \\
\hline
10000010 & 0x82 & 130 & -126 \\
\hline
10000001 & 0x81 & 129 & -127 \\
\hline
10000000 & 0x80 & 128 & -128 \\
\hline
\end{tabular}
\end{center}

\myindex{x86!\Instructions!JA}
\myindex{x86!\Instructions!JB}
\myindex{x86!\Instructions!JL}
\myindex{x86!\Instructions!JG}
La différence entre nombres signé et non-signé est que si l'on représente \TT{0xFFFFFFFE}
et \TT{0x00000002} comme non signées, alors le premier nombre (4294967294) est plus
grand que le second (2).
Si nous les représentons comme signés, le premier devient $-2$, et il est plus petit
que le second.
C'est la raison pour laquelle les sauts conditionnels~(\myref{sec:Jcc}) existent
à la fois pour des opérations signées (p. ex. \JG, \JL) et non-signées (\INS{JA}, \JB).

Par souci de simplicité, voici ce qu'il faut retenir:

\begin{itemize}
\item Les nombres peuvent être signés ou non-signés.

\item Types \CCpp signés:

  \begin{itemize}
    \item \TT{int64\_t} (-9,223,372,036,854,775,808 .. 9,223,372,036,854,775,807)
	  (-~9.2..~9.2 quintillions) ou \\
                \TT{0x8000000000000000..0x7FFFFFFFFFFFFFFF}),
    \item \Tint (-2,147,483,648..2,147,483,647 (-~2.15..~2.15Gb) ou \TT{0x80000000..0x7FFFFFFF}),
    \item \Tchar (-128..127 ou \TT{0x80..0x7F}),
    \item \TT{ssize\_t}.
   \end{itemize}

	Non-signés:
  \begin{itemize}
	  \item \TT{uint64\_t} (0..18,446,744,073,709,551,615 
		  (~18 quintillions) ou \TT{0..0xFFFFFFFFFFFFFFFF}),
   \item \TT{unsigned int} (0..4,294,967,295 (~4.3Gb) ou \TT{0..0xFFFFFFFF}),
   \item \TT{unsigned char} (0..255 ou \TT{0..0xFF}),
   \item \TT{size\_t}.
  \end{itemize}

\item Les types signés ont le signe dans le \ac{MSB}: 1 signifie \q{moins}, 0 signifie \q{plus}.

\item Étendre à un type de données plus large est facile:
\myref{subsec:sign_extending_32_to_64}.

\label{sec:signednumbers:negation}
\item La négation est simple: il suffit d'inverser tous les bits et d'ajouter 1.

Nous pouvons garder à l'esprit qu'un nombre de signe opposé se trouve de l'autre côté,
à la même distance de zéro.
L'addition d'un est nécessaire car zéro se trouve au milieu.

\myindex{x86!\Instructions!IDIV}
\myindex{x86!\Instructions!DIV}
\myindex{x86!\Instructions!IMUL}
\myindex{x86!\Instructions!MUL}
\myindex{x86!\Instructions!CBW}
\myindex{x86!\Instructions!CWD}
\myindex{x86!\Instructions!CWDE}
\myindex{x86!\Instructions!CDQ}
\myindex{x86!\Instructions!CDQE}
\myindex{x86!\Instructions!MOVSX}
\myindex{x86!\Instructions!SAR}
\item
	Les opérations d'addition et de soustraction fonctionnent bien pour les valeurs signées et non-signées.
	Mais pour la multiplication et la division, le x86 possède des instructions différentes:
	\TT{IDIV}/\TT{IMUL} pour les signés et \TT{DIV}/\TT{MUL} pour les non-signés.
\item
	Voici d'autres instructions qui fonctionnent avec des nombres signés:\\
	\TT{CBW/CWD/CWDE/CDQ/CDQE} (\myref{ins:CBW_CWD_etc}), \TT{MOVSX} (\myref{MOVSX}), \TT{SAR} (\myref{ins:SAR}).
\end{itemize}

Une table avec quelques valeurs négatives et positives (\myref{signed_tbl}) ressemble
à un thermomètre avec une échelle Celsius.
C'est pourquoi l'addition et la soustraction fonctionnent bien pour les nombres signés
et non-signés:
si le premier opérande est représenté par une marque sur un thermomètre, et que l'on
doit ajouter un second opérande, et qu'il est positif, nous devons juste augmenter
la marque sur le thermomètre de la valeur du second opérande.
Si le second opérande est négatif, alors nous baissons la marque de la valeur absolue
du second opérande.

L'addition de deux nombres négatifs fonctionne comme suit.
Par exemple, nous devons ajouter -2 et -3 en utilisant des registres 16-bit.
-2 et -3 sont respectivement 0xfffe et 0xfffd.
si nous les ajoutons comme nombres non-signés, nous obtenons 0xfffe+0xfffd=0x1fffb.
Mais nous travaillons avec des registres 16-bit, le résultat est \emph{tronqué},
le premier 1 est perdu, et il reste 0xfffb et c'est -5.
Ceci fonctionne car -2 (ou 0xfffe) peut être représenté en utilisant des mots simples
comme suit:
``il manque 2 à la valeur maximale d'un registre 16-bit + 1''.
-3 peut être représenté comme ``\dots il manque 3 à la valeur maximale jusqu'à \dots''.
La valeur maximale d'un registre 16-bit + 1 est 0x10000.
Pendant l'addition de deux nombres et en \emph{tronquant} modulo $2^{16}$, il manquera
$2+3=5$.

% subsections:
\input{fundamentals/MUL_IMUL_FR}
\input{fundamentals/one_more_FR}

\subsection{-1}

Vous savez maintenant que $-1$ est lorsque tous les bits sont mis à 1.
Souvent, vous pouvez trouver la constante $-1$ dans toute sorte de code qui nécessite
une constante avec tous les bits à 1, par exemple, un masque.

Par exemple: \myref{using_OR_instead_of_MOV}.

