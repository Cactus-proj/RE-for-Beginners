\mysection{\SignedNumbersSectionName}
\label{sec:signednumbers}
\myindex{Signed numbers}

\newcommand{\URLS}{\href{http://go.yurichev.com/17117}{wikipedia}}

There are several methods for representing signed numbers\footnote{\URLS}, 
but \q{two's complement} is the most popular one in computers.

Here is a table for some byte values:

\begin{center}
\begin{tabular}{ | l | l | l | l | }
\hline
\HeaderColor binary & \HeaderColor hexadecimal & \HeaderColor unsigned & \HeaderColor signed \\
\hline
01111111 & 0x7f & 127 & 127 \\
\hline
01111110 & 0x7e & 126 & 126 \\
\hline
\multicolumn{4}{ |c| }{...} \\
\hline
00000110 & 0x6 & 6 & 6 \\
\hline
00000101 & 0x5 & 5 & 5 \\
\hline
00000100 & 0x4 & 4 & 4 \\
\hline
00000011 & 0x3 & 3 & 3 \\
\hline
00000010 & 0x2 & 2 & 2 \\
\hline
00000001 & 0x1 & 1 & 1 \\
\hline
00000000 & 0x0 & 0 & 0 \\
\hline
11111111 & 0xff & 255 & -1 \\
\hline
11111110 & 0xfe & 254 & -2 \\
\hline
11111101 & 0xfd & 253 & -3 \\
\hline
11111100 & 0xfc & 252 & -4 \\
\hline
11111011 & 0xfb & 251 & -5 \\
\hline
11111010 & 0xfa & 250 & -6 \\
\hline
\multicolumn{4}{ |c| }{...} \\
\hline
10000010 & 0x82 & 130 & -126 \\
\hline
10000001 & 0x81 & 129 & -127 \\
\hline
10000000 & 0x80 & 128 & -128 \\
\hline
\end{tabular}
\end{center}

\myindex{x86!\Instructions!JA}
\myindex{x86!\Instructions!JB}
\myindex{x86!\Instructions!JL}
\myindex{x86!\Instructions!JG}
The difference between signed and unsigned numbers is that if we represent \TT{0xFFFFFFFE} and \TT{0x00000002} 
as unsigned, then the first number (4294967294) is bigger than the second one (2). 
If we represent them both as signed, the first one becomes $-2$, and it is smaller than the second (2). 
That is the reason why conditional jumps~(\myref{sec:Jcc}) are present both for signed (e.g. \JG, \JL) 
and unsigned (\INS{JA}, \JB) operations.

For the sake of simplicity, this is what one needs to know:

\begin{itemize}
\item Numbers can be signed or unsigned.

\item \CCpp signed types:

  \begin{itemize}
    \item \TT{int64\_t} (-9,223,372,036,854,775,808 .. 9,223,372,036,854,775,807)
	  (-~9.2..~9.2 quintillions) or \\
                \TT{0x8000000000000000..0x7FFFFFFFFFFFFFFF}),
    \item \Tint (-2,147,483,648..2,147,483,647 (-~2.15..~2.15Gb) or \\
	    \TT{0x80000000..0x7FFFFFFF}),
    \item \Tchar (-128..127 or \TT{0x80..0x7F}),
    \item \TT{ssize\_t}.
   \end{itemize}

	Unsigned:
  \begin{itemize}
	  \item \TT{uint64\_t} (0..18,446,744,073,709,551,615 
		  (~18 quintillions) or \TT{0..0xFFFFFFFFFFFFFFFF}),
   \item \TT{unsigned int} (0..4,294,967,295 (~4.3Gb) or \TT{0..0xFFFFFFFF}),
   \item \TT{unsigned char} (0..255 or \TT{0..0xFF}), 
   \item \TT{size\_t}.
  \end{itemize}

\item Signed types have the sign in the \ac{MSB}: 1 means \q{minus}, 0 means \q{plus}.

\item Promoting to a larger data types is simple:
\myref{subsec:sign_extending_32_to_64}.

\label{sec:signednumbers:negation}
\item Negation is simple: just invert all bits and add 1.

We can keep in mind that a number of inverse sign is located on the opposite side at the same proximity from zero.
The addition of one is needed because zero is present in the middle.

\myindex{x86!\Instructions!IDIV}
\myindex{x86!\Instructions!DIV}
\myindex{x86!\Instructions!IMUL}
\myindex{x86!\Instructions!MUL}
\myindex{x86!\Instructions!CBW}
\myindex{x86!\Instructions!CWD}
\myindex{x86!\Instructions!CWDE}
\myindex{x86!\Instructions!CDQ}
\myindex{x86!\Instructions!CDQE}
\myindex{x86!\Instructions!MOVSX}
\myindex{x86!\Instructions!SAR}
\item 
	The addition and subtraction operations work well for both signed and unsigned values.
	But for multiplication and division operations, x86 has different instructions:
	\TT{IDIV}/\TT{IMUL} for signed and \TT{DIV}/\TT{MUL} for unsigned.
\item
	Here are some more instructions that work with signed numbers:\\
	\TT{CBW/CWD/CWDE/CDQ/CDQE} (\myref{ins:CBW_CWD_etc}), \TT{MOVSX} (\myref{MOVSX}), \TT{SAR} (\myref{ins:SAR}).
\end{itemize}

A table of some negative and positive values (\myref{signed_tbl}) looks like thermometer with Celsius scale.
This is why addition and subtraction works equally well for both signed and unsigned numbers:
if the first addend is represented as mark on thermometer, and one need to add a second addend,
and it's positive, we just shift mark up on thermometer by the value of second addend.
If the second addend is negative, then we shift mark down to absolute value of the second addend.

Addition of two negative numbers works as follows.
For example, we need to add -2 and -3 using 16-bit registers.
-2 and -3 is 0xfffe and 0xfffd respectively.
If we add these numbers as unsigned, we will get 0xfffe+0xfffd=0x1fffb.
But we work on 16-bit registers, so the result is \emph{cut off}, the first 1 is dropped,
0xfffb is left, and this is -5.
This works because -2 (or 0xfffe) can be represented using plain English like this:
``2 lacks in this value up to maximal value in 16-bit register + 1''.
-3 can be represented as ``\dots 3 lacks in this value up to \dots''.
Maximal value of 16-bit register + 1 is 0x10000.
During addition of two numbers and \emph{cutting off} by $2^{16}$ modulo, $2+3=5$ \emph{will be lacking}.

% subsections:
\input{fundamentals/MUL_IMUL_EN}
\input{fundamentals/one_more_EN}

\subsection{-1}

Now you see that $-1$ is when all bits are set.
Often, you can find the $-1$ constant in all sorts of code, where a constant with all bits set are needed, for example, a mask.

For example: \myref{using_OR_instead_of_MOV}.

