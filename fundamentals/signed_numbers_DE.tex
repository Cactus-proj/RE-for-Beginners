\mysection{\SignedNumbersSectionName}
\label{sec:signednumbers}
\myindex{Signed numbers}

Es gibt verschiedene Arten vorzeichenbehaftete Zahlen darzustellen,
jedoch ist das \q{Zweierkomplement} die am weitesten verbreitete in Computern.

Hier ist eine Tabelle für einige Byte-Werte:

\begin{center}
\begin{tabular}{ | l | l | l | l | }
\hline
\HeaderColor binär & \HeaderColor hexadezimal & \HeaderColor vorzeichenlos & \HeaderColor vorzeichenbehaftet \\
\hline
01111111 & 0x7f & 127 & 127 \\
\hline
01111110 & 0x7e & 126 & 126 \\
\hline
\multicolumn{4}{ |c| }{...} \\
\hline
00000110 & 0x6 & 6 & 6 \\
\hline
00000101 & 0x5 & 5 & 5 \\
\hline
00000100 & 0x4 & 4 & 4 \\
\hline
00000011 & 0x3 & 3 & 3 \\
\hline
00000010 & 0x2 & 2 & 2 \\
\hline
00000001 & 0x1 & 1 & 1 \\
\hline
00000000 & 0x0 & 0 & 0 \\
\hline
11111111 & 0xff & 255 & -1 \\
\hline
11111110 & 0xfe & 254 & -2 \\
\hline
11111101 & 0xfd & 253 & -3 \\
\hline
11111100 & 0xfc & 252 & -4 \\
\hline
11111011 & 0xfb & 251 & -5 \\
\hline
11111010 & 0xfa & 250 & -6 \\
\hline
\multicolumn{4}{ |c| }{...} \\
\hline
10000010 & 0x82 & 130 & -126 \\
\hline
10000001 & 0x81 & 129 & -127 \\
\hline
10000000 & 0x80 & 128 & -128 \\
\hline
\end{tabular}
\end{center}

\myindex{x86!\Instructions!JA}
\myindex{x86!\Instructions!JB}
\myindex{x86!\Instructions!JL}
\myindex{x86!\Instructions!JG}
Der Unterschied zwischen vorzeichenbehafteten und vorzeichenlosen Zahlen ist, dass wenn \TT{0xFFFFFFFE}
und \TT{0x00000002} ohne Vorzeichen repräsentiert werden, die erste Zahl (4294967294) größer ist als
die zweite Zahl (2).
Wenn beide Zahlen als vorzeichenbehaftet repräsentiert werden, wird die erste Zahl $-2$ und ist kleiner
als die zweite Zahl (2).
Das ist der Grund, warum bedingte Sprünge ~(\myref{sec:Jcc}) sowohl für vorzeichenbehaftete (e.g. \JG, \JL)
als auch vorzeichenlose (\INS{JA}, \JB) Operationen vorhanden sind.

Aus Gründen der Einfachheit ist hier was man wissen muss:

\begin{itemize}
\item Zahlen können mit oder ohne Vorzeichen sein.

\item vorzeichenbehaftete Datentypen in \CCpp:

  \begin{itemize}
    \item \TT{int64\_t} (-9,223,372,036,854,775,808 .. 9,223,372,036,854,775,807)
	  (-~9.2..~9.2 Quintillionen) oder \\
                \TT{0x8000000000000000..0x7FFFFFFFFFFFFFFF}),
    \item \Tint (-2,147,483,648..2,147,483,647 (-~2.15..~2.15Gb) oder \\
	    \TT{0x80000000..0x7FFFFFFF}),
    \item \Tchar (-128..127 oder \TT{0x80..0x7F}),
    \item \TT{ssize\_t}.
   \end{itemize}

	Vorzeichenlos:
  \begin{itemize}
	  \item \TT{uint64\_t} (0..18,446,744,073,709,551,615 
		  (~18 Quintillionen) oder \TT{0..0xFFFFFFFFFFFFFFFF}),
   \item \TT{unsigned int} (0..4,294,967,295 (~4.3Gb) oder \TT{0..0xFFFFFFFF}),
   \item \TT{unsigned char} (0..255 oder \TT{0..0xFF}), 
   \item \TT{size\_t}.
  \end{itemize}

\item Vorzeichenbehaftete Typen haben das Vorzeichen am \ac{MSB}: 1 bedeutet \q{Minus}, 0 bedeutet \q{Plus}.

\item Erweitern auf größere Datentypen ist einfach:
\myref{subsec:sign_extending_32_to_64}.

\label{sec:signednumbers:negation}
\item Negieren ist einfach: es müssen lediglich alle Bits invertiert und anschließend 1 addiert werden.

%TODO Not sure how to translate this into an understandable in german.
%We can keep in mind that a number of inverse sign is located on the opposite side at the same proximity from zero.
%The addition of one is needed because zero is present in the middle.

\myindex{x86!\Instructions!IDIV}
\myindex{x86!\Instructions!DIV}
\myindex{x86!\Instructions!IMUL}
\myindex{x86!\Instructions!MUL}
\myindex{x86!\Instructions!CBW}
\myindex{x86!\Instructions!CWD}
\myindex{x86!\Instructions!CWDE}
\myindex{x86!\Instructions!CDQ}
\myindex{x86!\Instructions!CDQE}
\myindex{x86!\Instructions!MOVSX}
\myindex{x86!\Instructions!SAR}
\item 
	Die Addition und Subtraktion funktioniert sowohl für Zahlen mit als auch ohne Vorzeichen.
	Für Multiplikation und Division gibt es bei x86 unterschiedliche Anweisungen:
	\TT{IDIV}/\TT{IMUL} für vorzeichenbehaftete und \TT{DIV}/\TT{MUL} für vorzeichenlose Zahlen.
\item
	Hier sind einige weitere Anweisungen die mit vorzeichenbehafteten Zahlen arbeiten:\\
	\TT{CBW/CWD/CWDE/CDQ/CDQE} (\myref{ins:CBW_CWD_etc}), \TT{MOVSX} (\myref{MOVSX}), \TT{SAR} (\myref{ins:SAR}).
\end{itemize}

Eine Tabelle mit negativen und positiven Werten (\myref{signed_tbl}) sieht aus wie ein Thermometer mit Celsius-Skala.
Das ist der Grund warum Addition und Subtraktion für Zahlen mit und ohne Vorzeichen gleich funktioniert:
wenn der erste Summand eine Markierung auf dem Thermometer ist und ein weiterer Summand addiert werden soll,
der positiv ist, muss lediglich die Markierung auf dem Thermometer um den Wert des zweiten Summanden nach
oben verschoben werden.
Ist der zweite Summand negativ, wird die Markierung um den entsprechenden, absoluten Wert nach unten verschoben.

Die Addition zweier negativer Zahlen funktioniert wie folgt:
Wenn beispielsweise -2 und -3 unter Verwendung eines 16-Bit-Registers addiert werden sollen,
ist die Darstellung 0xfffe beziehungsweise 0xfffd.
Wenn diese Werte ohne Vorzeichen addiert werden, ist das Ergebnis 0xfffe+0xfffd=0x1fffb.
Allerdings sollen 16-Bit-Register verwendet werden, also wird beim Ergebis die erste 1 abgeschnitten.
Es bleibt 0xfffb was -5 entspricht.
Dies funktioniert, weil -2 (oder 0xfffe) in natürlicher Sprache wie folgt repräsentiert werden kann:
``2 fehlt bei diesem Werte bis zum maximalen Wert in einem 16-Bit-Register + 1''.
-3 kann repräsentiert werden als ``\dots 3 fehlt in diesem Wert bis zu \dots''.
Der maximale Wert eines 16-Bit-Registers + 1 ist 0x10000.
Bei der Addition der beiden Zahlen und \emph{Abschneiden}  durch $2^{16}$ modulo,
wird $2+3=5$ \emph{fehlen}.

% subsections:
\input{fundamentals/MUL_IMUL_DE}
\input{fundamentals/one_more_DE}

% TBT
%\subsection{-1}

%Now you see that $-1$ is when all bits are set.
%Often, you can find the $-1$ constant in all sorts of code, where a constant with all bits set are needed, for example, a mask.

%For example: \myref{using_OR_instead_of_MOV}.

