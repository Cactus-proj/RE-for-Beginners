\mysection{Integer overflow}

I intentionally put this section after the section about signed number representation.

First, take a look at this implementation of \emph{itoa()} function from \InSqBrackets{\KRBook}:

\begin{lstlisting}[style=customc]
void itoa(int n, char s[])
{
        int i, sign;
        if ((sign = n) < 0) /* record sign */
                n = -n; /* make n positive */
        i = 0;
        do { /* generate digits in reverse order */
                s[i++] = n % 10 + '0'; /* get next digit */
        } while ((n /= 10) > 0); /* delete it */
        if (sign < 0)
                s[i++] = '-';
        s[i] = '\0';
        strrev(s);
}
\end{lstlisting}

( The full source code: \url{\GitHubBlobMasterURL/fundamentals/itoa_KR.c} )

It has a subtle bug. Try to find it. You can download source code, compile it, etc.
The answer on the next page.

\clearpage

From \InSqBrackets{\KRBook}:

\begin{framed}
\begin{quotation}
Exercise 3-4. In a two's complement number representation, our version of \emph{itoa}
does not handle the largest negative number, that is, the value
of \emph{n} equal to $-(2^{wordsize-1})$. Explain why not. Modify it to
print that value correctly, regardless of the machine on which
it runs.
\end{quotation}
\end{framed}

The answer is: the function cannot process largest negative number (INT\_MIN or 0x80000000 or -2147483648) correctly.

How to change sign? Invert all bits and add 1.
If to invert all bits in INT\_MIN value (0x80000000), this is 0x7fffffff. Add 1 and this is 0x80000000 again.
So changing sign has no effect.
This is an important artifact of two's complement system.

Further reading:

\begin{itemize}
\item blexim -- Basic Integer Overflows\footnote{\url{http://phrack.org/issues/60/10.html}}

\item Yannick Moy, Nikolaj Bjørner, and David Sielaff -- Modular Bug-finding for Integer Overflows in the Large: Sound, Efficient, Bit-precise Static Analysis\footnote{\url{https://yurichev.com/mirrors/SMT/z3prefix.pdf}}
\end{itemize}

