\newglossaryentry{tail call}
{
  name={\PLph},
  description={\PLph}
}

\newglossaryentry{endianness}
{
  name={endianess},
  description={kolejność bajtów}
}

\newglossaryentry{caller}
{
  name={caller},
  description={funkcja wywołująca}
}

\newglossaryentry{callee}
{
  name={callee},
  description={funkcja wywoływana}
}

\newglossaryentry{debuggee}
{
  name={\PLph},
  description={\PLph}
}

\newglossaryentry{leaf function}
{
  name={funkcja liść},
  description={Funkcja, która nie wywołuje żadnej innej}
}

\newglossaryentry{link register}
{
  name={rejestr powrotu},
  description=(RISC) {Rejestr, w który zwykle przechowywany jest adres powrotu. Dzięki temu można wywoływać funkcje-liście (leaf functions) bez używania stosu - a więc szybciej}
}

\newglossaryentry{anti-pattern}
{
  name={anti-pattern},
  description={coś powszechnie uznanego jako zła praktyka}
}

\newglossaryentry{stack pointer}
{
  name=wskaźnik stosu,
  description={rejestr pokazujący na miejsce na stosie}
}

\newglossaryentry{decrement}
{
  name={\PLph},
  description={\PLph}
}

\newglossaryentry{increment}
{
  name={inkrementować},
  description={zwiększać o 1}
}

\newglossaryentry{loop unwinding}
{
  name={\PLph},
  description={\PLph}
}

\newglossaryentry{register allocator}
{
  name={\PLph},
  description={\PLph}
}

\newglossaryentry{quotient}
{
  name={\PLph},
  description={\PLph}
}

\newglossaryentry{product}
{
  name={\PLph},
  description={\PLph}
}

\newglossaryentry{NOP}
{
  name={\PLph},
  description={\PLph}
}

\newglossaryentry{POKE}
{
  name={\PLph},
  description={\PLph}
}

\newglossaryentry{keygenme}
{
  name={\PLph},
  description={\PLph}
} 

\newglossaryentry{dongle}
{
  name={\PLph},
  description={\PLph}
}

\newglossaryentry{thunk function}
{
  name={thunk function},
  description={prosta funkcja, której jedynym zadaniem jest wywołanie innej funkcji}
}

\newglossaryentry{user mode}
{
  name={\PLph},
  description={\PLph}
}

\newglossaryentry{kernel mode}
{
  name={\PLph},
  description={\PLph}
}

\newglossaryentry{Windows NT}
{
  name={\PLph},
  description={\PLph}
}

\newglossaryentry{atomic operation}
{
  name={\PLph},
  description={\PLph}
}

\newglossaryentry{NaN}
{
  name={\PLph},
  description={\PLph}
}

\newglossaryentry{basic block}
{
  name={\PLph},
  description={\PLph}
}

\newglossaryentry{NEON}
{
  name={\PLph},
  description={\PLph}
}

\newglossaryentry{reverse engineering}
{
  name={inżynieria wsteczna},
  description={proces odkrywania jak dana rzecz działa, czasami w celu jej sklonowania}
}

\newglossaryentry{compiler intrinsic}
{
  name={\PLph},
  description={\PLph}
}

\newglossaryentry{heap}
{
  name={heap},
  description={(kopiec, sterta) - przeważnie duży kawałek pamięci, zapewniony aplikacji przez \ac{OS} na jej własne potrzeby. malloc()/free() pracują ze stertą}
}

\newglossaryentry{name mangling}
{
  name={\PLph},
  description={\PLph}
}

\newglossaryentry{xoring}
{
  name={\PLph},
  description={\PLph}
}

\newglossaryentry{security cookie}
{
  name={\PLph},
  description={\PLph}
}

\newglossaryentry{tracer}
{
  name={\PLph},
  description={\PLph}
}

\newglossaryentry{GiB}
{
  name=GiB,
  description={gibibajt: $2^{10}$ (1024) mebibajtów, $2^{20}$ (1048576) kibibajtów lub $2^{30}$ (1073741824) bajtów}
}

\newglossaryentry{CP/M}
{
  name=CP/M,
  description={\PLph}
}

\newglossaryentry{stack frame}
{
  name={\PLph},
  description={\PLph}
}

\newglossaryentry{jump offset}
{
  name={\PLph},
  description={\PLph}
}

\newglossaryentry{integral type}
{
  name={\PLph},
  description={\PLph}
}

\newglossaryentry{real number}
{
  name={real number},
  description={\PLph}
}

\newglossaryentry{PDB}
{
  name={\PLph},
  description={\PLph}
}

\newglossaryentry{NTAPI}
{
  name={\PLph},
  description={\PLph}
}

\newglossaryentry{stdout}
{
  name={stdout},
  description={standardowe wyjście}
}

\newglossaryentry{word}
{
  name={\PLph},
  description={\PLph}
}

\newglossaryentry{arithmetic mean}
{
  name={\PLph},
  description={\PLph}
}

\newglossaryentry{padding}
{
  name={\PLph},
  description={\PLph}
}

