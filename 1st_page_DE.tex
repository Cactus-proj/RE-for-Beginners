\vspace*{\fill}

\iffalse
\huge
	Bitte nehmen Sie an der kurzen Umfrage teil, unter
\normalsize

\bigskip
\bigskip
\bigskip

\dots \url{https://beginners.re/survey.html}.
Dies kann sehr hilfreich für den Autor sein!

\bigskip
\bigskip
\bigskip
\fi

\huge
	\DE{Meine Leistungen}
\normalsize

\bigskip
\bigskip
\bigskip

Das vorliegende Buch ist \href{http://beginners.re/}{kostenlos} und
\href{https://github.com/DennisYurichev/RE-for-beginners/}{als OpenSource erhältlich}.
Manchmal muss ich jedoch auch Geld verdienen, aus diesem Grund entschuldige ich mich im Voraus
für das Platzieren der Werbung an dieser Stelle.

\iffalse
\Large Benötigen Sie Dokumentationen? \normalsize

Ich kann versuchen Dokumentationen, Referenzen und Handbücher für einige APIs,
Sprachen, Frameworks und so weiter zu schreiben.

Manchmal bin ich gut im Finden von präzisen und klaren Beispielen für jedes API- oder Sprachfeature.
Dieses Buch ist ein Beispiel dafür.
Ich kann versuchen dies in einer ausführlichen und zuverlässigen Art zu tun.

Auf der anderen Seite ist mein Englisch weit entfernt davon fließend zu sein und
ich könnte lange brauchen um mich tief in Produkte einzuarbeiten die ich nicht kenne.

Ich wäre aber erfreut existierende Dokumentationsprojekte zu überarbeiten.
Eine Beispielreferenz die ich bewundere ist Wolfram Mathematica: \url{http://reference.wolfram.com/language/}.
\fi

\Large Reverse engineering \normalsize

Ich kann keine Vollzeit-Jobs annehmen. Meistens arbeite ich von zuhause aus an kleinen Aufgaben wie:

\large Entschlüsseln von Datenbanken, welche unbekannte Datentypen verwalten \normalsize

Aufgrund einer Geheimhaltungsvereinbarung kann ich nicht viel über den letzten Auftrag
sagen, aber der Inhalt des Abschnitts \myref{encrypted_DB1} entstammt realen Arbeiten von mir.

\large Nachprogrammieren ausführbarer Dateien wie alte EXE- oder DLL-Dateien in C/C++ \normalsize

\large Dongle \normalsize

Gelegentlich realisiere ich Ersatz für
\href{https://en.wikipedia.org/wiki/Software_protection_dongle}{Kopierschutzstecker} oder Dongle-Emulatoren.
In der Regel ist dies nicht erlaubt, deswegen bestehen die folgenden Bedingungen:

\begin{itemize}
\item die Herstellerfirma der Software existiert nach meinem besten Wissen nicht mehr;
\item die Software ist älter als 10 Jahre;
\item Sie haben einen Dongle um die Informationen auszulesen. Mit anderen Worten, kann ich Ihnen
nur helfen wenn Sie noch sehr alte Software benutzen mit der Sie komplett zufrieden sind, jedoch
einen Defekt des Dongles fürchten und keine Firma Ersatz liefern kann.
\end{itemize}

Dies schließt alte MS-DOS- und UNIX-Software mit ein. Produkte für exotischere Computer-Architekturen
(wie MIPS, DEC Alpha, PowerPC) akzeptiere ich ebenfalls.

Beispiele meiner Arbeit finden Sie hier:

\begin{itemize}
\item Mein Buch über Reverse Engineering beinhaltet einen Teil über Kopierschutzstecker: \myref{dongles}.
\item \href{http://yurichev.com/writings/z3_rockey.pdf}{Finden von unbekannten Algorithmen durch Eingangs-/-Ausgangs-Paare
und Z3 SMT-Solver-Artikel}
\item \href{http://yurichev.com/blog/56/}{über MicroPhar (93c46-basierte Dongle) Emulation in DosBox}.
\item \href{http://conus.info/dongle/src/microph.asm}{Quellcode vom DOS MicroPhar-Emulator mit der EMM386 I/O API}
\end{itemize}

\large Kontaktieren Sie mich \normalsize

E-Mail: \GTT{\EMAIL}.

\large Möchten Sie immer noch einen Reverse Engineer / Security-Forscher in Vollzeit engagieren? \normalsize

Sie könnten es hier versuchen: \href{https://www.reddit.com/r/ReverseEngineering/comments/49cza0/rreverseengineerings_2015_triannual_hiring_thread/}{Reddit RE Thread}.
Außerdem gibt es ein russischsprachiges Forum mit einem \href{https://forum.reverse4you.org/forumdisplay.php?f=252}{Abschnitt für RE-Jobs}.

\vspace*{\fill}
\vfill
