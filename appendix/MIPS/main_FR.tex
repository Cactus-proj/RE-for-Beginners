\mysection{MIPS}

\subsection{\Registers}
\label{MIPS_registers_ref}

\myindex{MIPS!O32}
(Convention d'appel O32)

\subsubsection{Registres à usage général \ac{GPR}}

%\small
\begin{center}
\begin{tabular}{ | l | l | l | }
\hline
\HeaderColor Numéro & 
\HeaderColor Pseudonom & 
\HeaderColor Description \\
\hline
\$0             & \$ZERO          & Toujours à zéro. Écrire dans ce registre est comme \ac{NOP}. \\
\hline
\$1             & \$AT            & Utilisé comme un registre temporaire \\
                &                 & pour les macros assembleurs et les pseudo-instructions. \\
\hline
\$2 \dots \$3   & \$V0 \dots \$V1 & Le résultat des fonctions est renvoyé par ce registre. \\
\hline
\$4 \dots \$7   & \$A0 \dots \$A3 & Arguments de fonctions. \\
\hline
\$8 \dots \$15  & \$T0 \dots \$T7 & Utilisé pour des données temporaires. \\
\hline
\$16 \dots \$23 & \$S0 \dots \$S7 & Utilisé pour des données temporaireu\AsteriskOne{}. \\
\hline
\$24 \dots \$25 & \$T8 \dots \$T9 & Utilisé pour des données temporaires. \\
\hline
\$26 \dots \$27 & \$K0 \dots \$K1 & Réservé pour le noyau de l'\ac{OS}. \\
\hline
\$28            & \$GP            & Pointeur global\AsteriskTwo{}. \\
\hline
\$29            & \$SP            & \ac{SP}\AsteriskOne{}. \\
\hline
\$30            & \$FP            & \ac{FP}\AsteriskOne{}. \\
\hline
\$31            & \$RA            & \ac{RA}. \\
\hline
n/a             & PC              & \ac{PC}. \\
\hline
n/a             & HI              & Partie 32-bit haute d'une multiplication ou du reste d'une division\AsteriskThree{}. \\
\hline
n/a             & LO              & Partie 32-bit basse d'une multiplication ou du reste d'une division\AsteriskThree{}. \\
\hline
\end{tabular}
\end{center}
%\normalsize

\subsubsection{Registres en virgule flottante}
\label{MIPS_FPU_registers}

\begin{center}
\begin{tabular}{ | l | l | l | }
\hline
\HeaderColor Nom & \HeaderColor Description \\
\hline
\$F0..\$F1   & Le résultat d'une est renvoyé ici. \\
\hline
\$F2..\$F3   & Non utilisé. \\
\hline
\$F4..\$F11  & Utilisé pour des données temporaires. \\
\hline
\$F12..\$F15 & Deux premiers arguments de fonction. \\
\hline
\$F16..\$F19 & Utilisé pour des données temporaires. \\
\hline
\$F20..\$F31 & Utilisé pour des données temporaires.\AsteriskOne{}. \\
\hline
%fcr31 & Control/status register. \\
%\hline
\end{tabular}
\end{center}

\AsteriskOne{} --- L'\glslink{callee}{appelée} doit préservé le contenu.\\
\AsteriskTwo{} --- L'\glslink{callee}{appelée} doit préserver le contenu (sauf dans du code \ac{PIC}).\\
\myindex{MIPS!\Instructions!MFLO}
\myindex{MIPS!\Instructions!MFHI}
\AsteriskThree{} --- Accessible en utilisant les instructions \TT{MFHI} et \TT{MFLO}.\\

\subsection{\Instructions}

Il y a 3 types d'instructions:

\begin{itemize}

\item type-R: celles qui ont 3 registres, Les instructions-R ont habituellement la
forme suivante:

\begin{lstlisting}
instruction destination, source1, source2
\end{lstlisting}

Une chose importante à garder à l'esprit est que lorsque le premier et le second
registre sont les même, \IDA peut montrer l'instruction sous une forme plus courte:

\begin{lstlisting}
instruction destination/source1, source2
\end{lstlisting}

Cela nous rappelle quelque peu la syntaxe Intel pour le langage d'assemblage x86.

\item type-I: celles qui ont 2 registres et une valeur immédiate 16-bit.

\item type-J: instructions de saut/branchement, elles ont 26 bits pour encoder l'offset.

\end{itemize}

\subsubsection{Instructions da saut}

Quelle est la différence entre les instructions -B (BEQ, B, etc.) et le -J (JAL, JALR, etc.)?

Les instructions-B ont un type-I, ainsi, l'offset de l'instruction-B est encodé
comme une valeur 16-bit immédiate.
JR et JALR sont des types-R et sautent à une adresse absolue spécifiée dans un registre.
J et JAL sont des type-J, ainsi l'offset est encodé en une valeur 26-bit immédiate.

En bref, les instructions-B peuvent encoder une condition (B est en fait une pseudo-instruction
pour \TT{BEQ \$ZERO, \$ZERO, LABEL}), tandis que les instructions-J ne le peuvent pas.
