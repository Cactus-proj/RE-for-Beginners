\subsection{Инструкции}

В ARM имеется также для некоторых инструкций суффикс \emph{-S}, указывающий, 
что эта инструкция будет модифицировать флаги.
Инструкции без этого суффикса не модифицируют флаги.
\myindex{ARM!\Instructions!ADD}
\myindex{ARM!\Instructions!ADDS}
\myindex{ARM!\Instructions!CMP}
Например, инструкция \TT{ADD} в отличие от \TT{ADDS}
сложит два числа, но флаги не изменит.
Такие инструкции удобно использовать
между \CMP где выставляются флаги и, например, инструкциями перехода, где флаги используются.
Они также лучше в смысле анализа зависимостей данных (data dependency analysis) 
(потому что меньшее количество регистров модифицируется во время исполнения).

% ADD
% ADDAL
% ADDCC
% ADDS
% ADR
% ADREQ
% ADRGT
% ADRHI
% ADRNE
% ASRS
% B
% BCS
% BEQ
% BGE
% BIC
% BL
% BLE
% BLEQ
% BLGT
% BLHI
% BLS
% BLT
% BLX
% BNE
% BX
% CMP
% IDIV
% IT
% LDMCSFD
% LDMEA
% LDMED
% LDMFA
% LDMFD
% LDMGEFD
% LDR.W
% LDR
% LDRB.W
% LDRB
% LDRSB
% LSL.W
% LSL
% LSLS
% MLA
% MOV
% MOVT.W
% MOVT
% MOVW
% MULS
% MVNS
% ORR
% POP
% PUSH
% RSB
% SMMUL
% STMEA
% STMED
% STMFA
% STMFD
% STMIA
% STMIB
% STR
% SUB
% SUBEQ
% SXTB
% TEST
% TST
% VADD
% VDIV
% VLDR
% VMOV
% VMOVGT
% VMRS
% VMUL
%\myindex{ARM!Optional operators!ASR
%\myindex{ARM!Optional operators!LSL
%\myindex{ARM!Optional operators!LSR
%\myindex{ARM!Optional operators!ROR
%\myindex{ARM!Optional operators!RRX

% AArch64
% RET is BR X30 or BR LR but with additional hint to CPU

\subsubsection{Таблица условных кодов}

% TODO rework this!
\small
\begin{center}
\begin{tabular}{ | l | l | l | }
\hline
\HeaderColor Код & 
\HeaderColor Описание & 
\HeaderColor Флаги \\
\hline
EQ                              & равно                                                         & Z == 1 \\
\hline
NE                              & не равно                                                      & Z == 0 \\
\hline
CS \ac{AKA} HS (Higher or Same) & перенос / беззнаковое, больше или равно                       & C == 1 \\
\hline
CC \ac{AKA} LO (LOwer)          & нет переноса / беззнаковое, меньше чем                        & C == 0 \\
\hline
MI                              & минус, отрицательный знак / меньше чем                        & N == 1 \\
\hline
PL                              & плюс, положительный знак или ноль /                           & N == 0 \\
                                & больше чем или равно                                          & \\
\hline
VS                              & переполнение & V == 1 \\
\hline
VC                              & нет переполнения & V == 0 \\
\hline
HI                              & беззнаковое, больше чем                                       & C == 1 и \\
                                &                                                               & Z == 0 \\
\hline
LS                              & беззнаковое, меньше или равно                                 & C == 0 или \\
                                &                                                               & Z == 1 \\
\hline
GE                              & знаковое, больше чем или равно                                & N == V \\
\hline
LT                              & знаковое, меньше чем                                          & N != V \\
\hline
GT                              & знаковое, больше чем                                          & Z == 0 и \\
                                &                                                               & N == V \\
\hline
LE                              & знаковое, меньше чем или равно                                & Z == 1 или \\
                                &                                                               & N != V \\
\hline
Нету / AL                       & Всегда                                                        & Любые \\
\hline
\end{tabular}
\end{center}
\normalsize

