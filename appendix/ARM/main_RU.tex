\mysection{ARM}
\myindex{ARM}

\subsection{Терминология}

ARM изначально разрабатывался как 32-битный \ac{CPU}, 
поэтому \emph{слово} здесь, в отличие от x86, 32-битное.

\begin{description}
	\item[byte] 8-бит.
		Для определения переменных и массива байт используется директива ассемблера DCB.
	\item[halfword] 16-бит. \dittoclosing директива ассемблера DCW.
	\item[word] 32-бит. \dittoclosing директива ассемблера DCD.
	\item[doubleword] 64-бит.
	\item[quadword] 128-бит.
\end{description}

\subsection{Версии}

\begin{itemize}
\item ARMv4: появился режим Thumb.

\item ARMv6: использовался в iPhone 1st gen., iPhone 3G 
(Samsung 32-bit RISC ARM 1176JZ(F)-S поддерживающий Thumb-2)

\item ARMv7: появился Thumb-2 (2003).
Использовался в iPhone 3GS, iPhone 4, iPad 1st gen. (ARM Cortex-A8), iPad 2 (Cortex-A9),
iPad 3rd gen.

\item ARMv7s: Добавлены новые инструкции.
Использовался в iPhone 5, iPhone 5c, iPad 4th gen. (Apple A6).

\item ARMv8: 64-битный процессор, \ac{AKA} ARM64 \ac{AKA} AArch64.
Использовался в iPhone 5S, iPad Air (Apple A7).
В 64-битном режиме, режима Thumb больше нет, только режим ARM (4-байтные инструкции).
\end{itemize}

% sections
\subsection{32-битный ARM (AArch32)}

\subsubsection{Регистры общего пользования}

\begin{itemize}
\myindex{ARM!\Registers!R0}
	\item R0 --- результат функции обычно возвращается через R0
	\item R1...R12 --- \ac{GPR}s
	\item R13 --- \ac{AKA} SP (\gls{stack pointer})
\myindex{ARM!\Registers!Link Register}
	\item R14 --- \ac{AKA} LR (\gls{link register})
	\item R15 --- \ac{AKA} PC (program counter)
\end{itemize}

\myindex{ARM!\Registers!scratch registers}
\Reg{0}-\Reg{3} называются также \q{scratch registers}: аргументы функции обычно передаются через них,
и эти значения не обязательно восстанавливать перед выходом из функции.

\subsubsection{Current Program Status Register (CPSR)}

\begin{center}
\begin{tabular}{ | l | l | }
\hline
\headercolor\ Бит &
\headercolor\ Описание \\
\hline
0..4           & M --- processor mode \\
\hline
5              & T --- Thumb state \\
\hline
6              & F --- FIQ disable \\
\hline
7              & I --- IRQ disable \\
\hline
8              & A --- imprecise data abort disable \\
\hline
9              & E --- data endianness \\
\hline
10..15, 25, 26 & IT --- if-then state \\
\hline
16..19         & GE --- greater-than-or-equal-to \\
\hline
20..23         & DNM --- do not modify \\
\hline
24             & J --- Java state \\
\hline
27             & Q --- sticky overflow \\
\hline
28             & V --- overflow \\
\hline
29             & C --- carry/borrow/extend \\
\hline
\myindex{ARM!\Registers!Z}
30             & Z --- zero bit \\
\hline
31             & N --- negative/less than \\
\hline
\end{tabular}
\end{center}

% TODO
% \myindex{ARM!\Registers!APSR}
% \subsubsection{Application Program Status Register (APSR)}

% TODO
% \myindex{ARM!\Registers!FPSCR}
% \subsubsection{Floating-Point Status and Control Register (FPPSR)}
% http://infocenter.arm.com/help/index.jsp?topic=/com.arm.doc.ddi0344b/Chdfafia.html

\subsubsection{Регистры VPF (для чисел с плавающей точкой) и NEON}
\label{ARM_VFP_registers}

% http://infocenter.arm.com/help/index.jsp?topic=/com.arm.doc.dht0002a/ch01s03s02.html

\myindex{ARM!D-\registers{}}
\myindex{ARM!S-\registers{}}
\begin{center}
\begin{tabular}{ | l | l | l | l | }
\hline
0..31\textsuperscript{bits} & 32..64 & 65..96 & 97..127 \\
\hline
\multicolumn{4}{ | c | }{Q0\textsuperscript{128 bits}} \\
\hline
\multicolumn{2}{ | c | }{D0\textsuperscript{64 bits}} & \multicolumn{2}{ c | }{D1} \\
\hline
S0\textsuperscript{32 bits} & S1 & S2 & S3 \\
\hline
\end{tabular}
\end{center}

S-регистры 32-битные, используются для хранения чисел с одинарной точностью.

D-регистры 64-битные, используются для хранения чисел с двойной точностью.

D- и S-регистры занимают одно и то же место в памяти CPU --- можно
обращаться к D-регистрам через S-регистры (хотя это и бессмысленно).

Точно также, \gls{NEON} Q-регистры имеют размер 128 бит и занимают то же физическое место 
в памяти CPU что и остальные регистры, предназначенные для чисел с плавающей точкой.

В VFP присутствует 32 S-регистров: S0..S31.

В VPFv2 были добавлены 16 D-регистров, которые занимают то же место что и S0..S31.

В VFPv3 (\gls{NEON} или \q{Advanced SIMD}) добавили еще 16 D-регистров, в итоге это D0..D31, но регистры D16..D31 не делят место с другими S-регистрами.

В \gls{NEON} или \q{Advanced SIMD} были добавлены также 16 128-битных Q-регистров, делящих место с регистрами D0..D31.

\subsection{64-битный ARM (AArch64)}

\subsubsection{Регистры общего пользования}
\label{ARM64_GPRs}

Количество регистров было удвоено со времен AArch32.

\begin{itemize}
\myindex{ARM!\Registers!X0}
	\item X0 --- результат функции обычно возвращается через X0
        \item X0...X7 --- Здесь передаются аргументы функции.
	\item X8
	\item X9...X15 --- временные регистры, вызываемая функция может их использовать и не восстанавливать их.
	\item X16
	\item X17
	\item X18
	\item X19...X29 --- вызываемая функция может их использовать, но должна восстанавливать их по завершению.
	\item X29 --- используется как \ac{FP} (как минимум в GCC)
	\item X30 --- \q{Procedure Link Register} \ac{AKA} \ac{LR} (\gls{link register}).
	\item X31 --- регистр, всегда содержащий ноль \ac{AKA} XZR или \q{Zero Register}. Его 32-битная часть называется WZR.
	\item \ac{SP}, больше не регистр общего пользования.
\end{itemize}

См.также: \ARMPCS.

32-битная часть каждого X-регистра также доступна как W-регистр (W0, W1, итд).

\begin{center}
\begin{tabular}{ | l | l | }
\hline
	Старшие 32 бита & младшие 32 бита \\
\hline
\multicolumn{2}{ | c | }{X0} \\
\hline
\multicolumn{1}{ | c | }{} & \multicolumn{1}{ c | }{W0} \\
\hline
\end{tabular}
\end{center}


\subsection{Инструкции}

В ARM имеется также для некоторых инструкций суффикс \emph{-S}, указывающий, 
что эта инструкция будет модифицировать флаги.
Инструкции без этого суффикса не модифицируют флаги.
\myindex{ARM!\Instructions!ADD}
\myindex{ARM!\Instructions!ADDS}
\myindex{ARM!\Instructions!CMP}
Например, инструкция \TT{ADD} в отличие от \TT{ADDS}
сложит два числа, но флаги не изменит.
Такие инструкции удобно использовать
между \CMP где выставляются флаги и, например, инструкциями перехода, где флаги используются.
Они также лучше в смысле анализа зависимостей данных (data dependency analysis) 
(потому что меньшее количество регистров модифицируется во время исполнения).

% ADD
% ADDAL
% ADDCC
% ADDS
% ADR
% ADREQ
% ADRGT
% ADRHI
% ADRNE
% ASRS
% B
% BCS
% BEQ
% BGE
% BIC
% BL
% BLE
% BLEQ
% BLGT
% BLHI
% BLS
% BLT
% BLX
% BNE
% BX
% CMP
% IDIV
% IT
% LDMCSFD
% LDMEA
% LDMED
% LDMFA
% LDMFD
% LDMGEFD
% LDR.W
% LDR
% LDRB.W
% LDRB
% LDRSB
% LSL.W
% LSL
% LSLS
% MLA
% MOV
% MOVT.W
% MOVT
% MOVW
% MULS
% MVNS
% ORR
% POP
% PUSH
% RSB
% SMMUL
% STMEA
% STMED
% STMFA
% STMFD
% STMIA
% STMIB
% STR
% SUB
% SUBEQ
% SXTB
% TEST
% TST
% VADD
% VDIV
% VLDR
% VMOV
% VMOVGT
% VMRS
% VMUL
%\myindex{ARM!Optional operators!ASR
%\myindex{ARM!Optional operators!LSL
%\myindex{ARM!Optional operators!LSR
%\myindex{ARM!Optional operators!ROR
%\myindex{ARM!Optional operators!RRX

% AArch64
% RET is BR X30 or BR LR but with additional hint to CPU

\subsubsection{Таблица условных кодов}

% TODO rework this!
\small
\begin{center}
\begin{tabular}{ | l | l | l | }
\hline
\HeaderColor Код & 
\HeaderColor Описание & 
\HeaderColor Флаги \\
\hline
EQ                              & равно                                                         & Z == 1 \\
\hline
NE                              & не равно                                                      & Z == 0 \\
\hline
CS \ac{AKA} HS (Higher or Same) & перенос / беззнаковое, больше или равно                       & C == 1 \\
\hline
CC \ac{AKA} LO (LOwer)          & нет переноса / беззнаковое, меньше чем                        & C == 0 \\
\hline
MI                              & минус, отрицательный знак / меньше чем                        & N == 1 \\
\hline
PL                              & плюс, положительный знак или ноль /                           & N == 0 \\
                                & больше чем или равно                                          & \\
\hline
VS                              & переполнение & V == 1 \\
\hline
VC                              & нет переполнения & V == 0 \\
\hline
HI                              & беззнаковое, больше чем                                       & C == 1 и \\
                                &                                                               & Z == 0 \\
\hline
LS                              & беззнаковое, меньше или равно                                 & C == 0 или \\
                                &                                                               & Z == 1 \\
\hline
GE                              & знаковое, больше чем или равно                                & N == V \\
\hline
LT                              & знаковое, меньше чем                                          & N != V \\
\hline
GT                              & знаковое, больше чем                                          & Z == 0 и \\
                                &                                                               & N == V \\
\hline
LE                              & знаковое, меньше чем или равно                                & Z == 1 или \\
                                &                                                               & N != V \\
\hline
Нету / AL                       & Всегда                                                        & Любые \\
\hline
\end{tabular}
\end{center}
\normalsize


