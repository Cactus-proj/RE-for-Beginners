\subsection{Отладочные регистры}

Применяются для работы с т.н. аппаратными точками останова (\textit{hardware breakpoints}).

\begin{itemize}
	\item DR0 --- адрес точки останова \#1
	\item DR1 --- адрес точки останова \#2
	\item DR2 --- адрес точки останова \#3
	\item DR3 --- адрес точки останова \#4
	\item DR6 --- здесь отображается причина останова
	\item DR7 --- здесь можно задать типы точек останова
\end{itemize}

\subsubsection{DR6}
\myindex{x86!\Registers!DR6}

\begin{center}
\begin{tabular}{ | l | l | }
\hline
\headercolor\ Бит (маска) &
\headercolor\ Описание \\
\hline
0 (1)       &  B0 --- сработала точка останова \#1 \\
\hline
1 (2)       &  B1 --- сработала точка останова \#2 \\
\hline
2 (4)       &  B2 --- сработала точка останова \#3 \\
\hline
3 (8)       &  B3 --- сработала точка останова \#4 \\
\hline
13 (0x2000) &  BD --- была попытка модифицировать один из регистров DRx. \\
            &  может быть выставлен если бит GD выставлен. \\
\hline
14 (0x4000) &  BS --- точка останова типа single step (флаг TF был выставлен в EFLAGS). \\
	    &  Наивысший приоритет. Другие биты также могут быть выставлены. \\
\hline
% TODO: describe BT
15 (0x8000) &  BT (флаг переключения задачи) \\
\hline
\end{tabular}
\end{center}

N.B. Точка останова single step это срабатывающая после каждой инструкции.
Может быть включена выставлением флага TF в EFLAGS (\myref{EFLAGS}).

\subsubsection{DR7}
\myindex{x86!\Registers!DR7}

В этом регистре задаются типы точек останова.

%\small
\begin{center}
\begin{tabular}{ | l | l | }
\hline
\headercolor\ Бит (маска) &
\headercolor\ Описание \\
\hline
0 (1)       &  L0 --- разрешить точку останова \#1 для текущей задачи \\
\hline
1 (2)       &  G0 --- разрешить точку останова \#1 для всех задач \\
\hline
2 (4)       &  L1 --- разрешить точку останова \#2 для текущей задачи \\
\hline
3 (8)       &  G1 --- разрешить точку останова \#2 для всех задач \\
\hline
4 (0x10)    &  L2 --- разрешить точку останова \#3 для текущей задачи \\
\hline
5 (0x20)    &  G2 --- разрешить точку останова \#3 для всех задач \\
\hline
6 (0x40)    &  L3 --- разрешить точку останова \#4 для текущей задачи \\
\hline
7 (0x80)    &  G3 --- разрешить точку останова \#4 для всех задач \\
\hline
8 (0x100)   &  LE --- не поддерживается, начиная с P6 \\
\hline
9 (0x200)   &  GE --- не поддерживается, начиная с P6 \\
\hline
13 (0x2000) &  GD --- исключение будет вызвано если какая-либо инструкция MOV \\
            & попытается модифицировать один из DRx-регистров \\
\hline
16,17 (0x30000)    &  точка останова \#1: R/W --- тип \\
\hline
18,19 (0xC0000)    &  точка останова \#1: LEN --- длина \\
\hline
20,21 (0x300000)   &  точка останова \#2: R/W --- тип \\
\hline
22,23 (0xC00000)   &  точка останова \#2: LEN --- длина \\
\hline
24,25 (0x3000000)  &  точка останова \#3: R/W --- тип \\
\hline
26,27 (0xC000000)  &  точка останова \#3: LEN --- длина \\
\hline
28,29 (0x30000000) &  точка останова \#4: R/W --- тип \\
\hline
30,31 (0xC0000000) &  точка останова \#4: LEN --- длина \\
\hline
\end{tabular}
\end{center}
%\normalsize

Так задается тип точки останова (R/W):

\begin{itemize}
\item 00 --- исполнение инструкции
\item 01 --- запись в память
\item 10 --- обращения к I/O-портам (недоступно из user-mode)
\item 11 --- обращение к памяти (чтение или запись)
\end{itemize}

N.B.: отдельного типа для чтения из памяти действительно нет. \\
\\
Так задается длина точки останова (LEN):

\begin{itemize}
\item 00 --- 1 байт
\item 01 --- 2 байта
\item 10 --- не определено для 32-битного режима, 8 байт для 64-битного
\item 11 --- 4 байта
\end{itemize}
