\subsection{Инструкции}
\label{sec:x86_instructions}

Инструкции, отмеченные как (M) обычно не генерируются компилятором: если вы видите её, очень может быть
это вручную написанный фрагмент кода, либо это т.н. compiler intrinsic
(\myref{sec:compiler_intrinsic}).

% TODO ? обратные инструкции

Только наиболее используемые инструкции перечислены здесь
Обращайтесь к \myref{x86_manuals} для полной документации.

Нужно ли заучивать опкоды инструкций на память?
Нет, только те, которые часто используются для модификации кода
(\myref{x86_patching}).
Остальные запоминать нет смысла.

\subsubsection{Префиксы}

\myindex{x86!\Prefixes!LOCK}
\myindex{x86!\Prefixes!REP}
\myindex{x86!\Prefixes!REPE/REPNE}
\begin{description}
\label{x86_lock}
\item[LOCK] используется чтобы предоставить эксклюзивный доступ к памяти в многопроцессорной среде.
Для упрощения, можно сказать, что когда исполняется инструкция с этим префиксом, остальные процессоры
в системе останавливаются.
Чаще все это используется для критических секций, семафоров, мьютексов.
Обычно используется с ADD, AND, BTR, BTS, CMPXCHG, OR, XADD, XOR.
Читайте больше о критических секциях (\myref{critical_sections}).

\item[REP] используется с инструкциями MOVSx и STOSx:
инструкция будет исполняться в цикле, счетчик расположен в регистре CX/ECX/RCX.
Для более детального описания, читайте больше об инструкциях
MOVSx (\myref{REP_MOVSx}) и STOSx (\myref{REP_STOSx}).

Работа инструкций с префиксом REP зависит от флага DF, он задает направление.

\item[REPE/REPNE] (\ac{AKA} REPZ/REPNZ) используется с инструкциями CMPSx и
SCASx:
инструкция будет исполняться в цикле, счетчик расположен в регистре \TT{CX}/\TT{ECX}/\TT{RCX}.
Выполнение будет прервано если ZF будет 0 (REPE) либо если ZF будет 1 (REPNE).

Для более детального описания, читайте больше об инструкциях CMPSx (\myref{REPE_CMPSx}) 
и SCASx (\myref{REPNE_SCASx}).

Работа инструкций с префиксами REPE/REPNE зависит от флага DF, он задает направление.

\end{description}

\subsubsection{Наиболее часто используемые инструкции}

Их можно заучить в первую очередь.

\begin{description}
% in order to keep them easily sorted...
\input{appendix/x86/instructions/ADC}
\input{appendix/x86/instructions/ADD}
\input{appendix/x86/instructions/AND}
\input{appendix/x86/instructions/CALL}
\input{appendix/x86/instructions/CMP}
\input{appendix/x86/instructions/DEC}
\myindex{x86!\Instructions!IMUL}
  \item[IMUL] \RU{умножение с учетом знаковых значений}\EN{signed multiply}\FR{multiplication signée}
  \EN{\IMUL often used instead of \MUL, read more about it:}%
  \RU{\IMUL часто используется вместо \MUL, читайте об этом больше:}%
  \FR{\IMUL est souvent utilisé à la place de \MUL, voir ici:} \myref{IMUL_over_MUL}.


\input{appendix/x86/instructions/INC}
\input{appendix/x86/instructions/JCXZ}
\input{appendix/x86/instructions/JMP}
\item[Jcc] (где cc --- condition code)

\myindex{x86!\Instructions!JAE}
\myindex{x86!\Instructions!JA}
\myindex{x86!\Instructions!JBE}
\myindex{x86!\Instructions!JB}
\myindex{x86!\Instructions!JC}
\myindex{x86!\Instructions!JE}
\myindex{x86!\Instructions!JGE}
\myindex{x86!\Instructions!JG}
\myindex{x86!\Instructions!JLE}
\myindex{x86!\Instructions!JL}
\myindex{x86!\Instructions!JNAE}
\myindex{x86!\Instructions!JNA}
\myindex{x86!\Instructions!JNBE}
\myindex{x86!\Instructions!JNB}
\myindex{x86!\Instructions!JNC}
\myindex{x86!\Instructions!JNE}
\myindex{x86!\Instructions!JNGE}
\myindex{x86!\Instructions!JNG}
\myindex{x86!\Instructions!JNLE}
\myindex{x86!\Instructions!JNL}
\myindex{x86!\Instructions!JNO}
\myindex{x86!\Instructions!JNS}
\myindex{x86!\Instructions!JNZ}
\myindex{x86!\Instructions!JO}
\myindex{x86!\Instructions!JPO}
\myindex{x86!\Instructions!JP}
\myindex{x86!\Instructions!JS}
\myindex{x86!\Instructions!JZ}

Немало этих инструкций имеют синонимы (отмечены с AKA), это сделано для удобства.
Синонимичные инструкции транслируются в один и тот же опкод.
Опкод имеет т.н. \gls{jump offset}.

\label{Jcc}
\begin{description}
\item[JAE] \ac{AKA} JNC: переход если больше или равно (беззнаковый): CF=0
\item[JA] \ac{AKA} JNBE: переход если больше (беззнаковый): CF=0 и ZF=0
\item[JBE] переход если меньше или равно (беззнаковый): CF=1 или ZF=1
\item[JB] \ac{AKA} JC: переход если меньше (беззнаковый): CF=1
\item[JC] \ac{AKA} JB: переход если CF=1
\item[JE] \ac{AKA} JZ: переход если равно или ноль: ZF=1
\item[JGE] переход если больше или равно (знаковый): SF=OF
\item[JG] переход если больше (знаковый): ZF=0 и SF=OF
\item[JLE] переход если меньше или равно (знаковый): ZF=1 или SF$\neq$OF
\item[JL] переход если меньше (знаковый): SF$\neq$OF
\item[JNAE] \ac{AKA} JC: переход если не больше или равно (беззнаковый) CF=1
\item[JNA] переход если не больше (беззнаковый) CF=1 и ZF=1
\item[JNBE] переход если не меньше или равно (беззнаковый): CF=0 и ZF=0
\item[JNB] \ac{AKA} JNC: переход если не меньше (беззнаковый): CF=0
\item[JNC] \ac{AKA} JAE: переход если CF=0, синонимично JNB.
\item[JNE] \ac{AKA} JNZ: переход если не равно или не ноль: ZF=0
\item[JNGE] переход если не больше или равно (знаковый): SF$\neq$OF
\item[JNG] переход если не больше (знаковый): ZF=1 или SF$\neq$OF
\item[JNLE] переход если не меньше (знаковый): ZF=0 и SF=OF
\item[JNL] переход если не меньше (знаковый): SF=OF
\item[JNO] переход если не переполнение: OF=0
\item[JNS] переход если флаг SF сброшен
\item[JNZ] \ac{AKA} JNE: переход если не равно или не ноль: ZF=0
\item[JO] переход если переполнение: OF=1
\item[JPO] переход если сброшен флаг PF (Jump Parity Odd)
\item[JP] \ac{AKA} \ac{JPE}: переход если выставлен флаг PF
\item[JS] переход если выставлен флаг SF
\item[JZ] \ac{AKA} JE: переход если равно или ноль: ZF=1
\end{description}


\input{appendix/x86/instructions/LAHF}
\input{appendix/x86/instructions/LEAVE}
\myindex{x86!\Instructions!LEA}
\item[LEA] (\emph{Load Effective Address}) \RU{сформировать адрес}\EN{form an address}

\label{sec:LEA}

\newcommand{\URLAM}{\href{http://en.wikipedia.org/wiki/Addressing_mode}{wikipedia}}

\RU{Это инструкция, которая задумывалась вовсе не для складывания 
и умножения чисел, 
а для формирования адреса например, из указателя на массив и прибавления индекса к нему
\footnote{См. также: \URLAM}.}
\EN{This instruction was intended not for summing values and multiplication 
but for forming an address, 
e.g., for calculating the address of an array element by adding the array address, element index, with 
multiplication of element size\footnote{See also: \URLAM}.}
\par
\RU{То есть, разница между \MOV и \LEA в том, что \MOV формирует адрес в памяти 
и загружает значение из памяти, либо записывает его туда, а \LEA только формирует адрес.}
\EN{So, the difference between \MOV and \LEA is that \MOV forms a memory address and loads a value
from memory or stores it there, but \LEA just forms an address.}
\par
\RU{Тем не менее, её можно использовать для любых других вычислений}
\EN{But nevertheless, it is can be used for any other calculations}.
\par
\RU{\LEA удобна тем, что производимые ею вычисления не модифицируют флаги \ac{CPU}. 
Это может быть очень важно для \ac{OOE} процессоров (чтобы было меньше зависимостей между данными).}
\EN{\LEA is convenient because the computations performed by it does not alter \ac{CPU} flags.
This may be very important for \ac{OOE} processors (to create less data dependencies).}

\RU{Помимо всего прочего, начиная минимум с Pentium, инструкция LEA исполняется за 1 такт.}%
\EN{Aside from this, starting at least at Pentium, LEA instruction is executed in 1 cycle.}

\begin{lstlisting}[style=customc]
int f(int a, int b)
{
	return a*8+b;
};
\end{lstlisting}

\begin{lstlisting}[caption=\Optimizing MSVC 2010,style=customasmx86]
_a$ = 8		; size = 4
_b$ = 12	; size = 4
_f	PROC
	mov	eax, DWORD PTR _b$[esp-4]
	mov	ecx, DWORD PTR _a$[esp-4]
	lea	eax, DWORD PTR [eax+ecx*8]
	ret	0
_f	ENDP
\end{lstlisting}

\myindex{Intel C++}
Intel C++ \RU{использует LEA даже больше}\EN{uses LEA even more}:

\begin{lstlisting}[style=customc]
int f1(int a)
{
	return a*13;
};
\end{lstlisting}

\begin{lstlisting}[caption=Intel C++ 2011,style=customasmx86]
_f1	PROC NEAR 
        mov       ecx, DWORD PTR [4+esp]      ; ecx = a
	lea       edx, DWORD PTR [ecx+ecx*8]  ; edx = a*9
	lea       eax, DWORD PTR [edx+ecx*4]  ; eax = a*9 + a*4 = a*13
        ret                                
\end{lstlisting}

\RU{Эти две инструкции вместо одной IMUL будут работать быстрее.}
\EN{These two instructions performs faster than one IMUL.}


\input{appendix/x86/instructions/MOVSB_W_D_Q}
\input{appendix/x86/instructions/MOVSX}
\input{appendix/x86/instructions/MOVZX}
\input{appendix/x86/instructions/MOV}
\myindex{x86!\Instructions!MUL}
  \item[MUL] \RU{умножение с учетом беззнаковых значений}\EN{unsigned multiply}\FR{multiplier sans signe}.
  \EN{\IMUL often used instead of \MUL, read more about it:}%
  \RU{\IMUL часто используется вместо \MUL, читайте об этом больше:}%
  \FR{\IMUL est souvent utilisée au lieu de \MUL, en lire plus ici:} \myref{IMUL_over_MUL}.


\input{appendix/x86/instructions/NEG}
\input{appendix/x86/instructions/NOP}
\input{appendix/x86/instructions/NOT}
\input{appendix/x86/instructions/OR}
\input{appendix/x86/instructions/POP}
\input{appendix/x86/instructions/PUSH}
\input{appendix/x86/instructions/RET}
\input{appendix/x86/instructions/SAHF}
\input{appendix/x86/instructions/SBB}
\input{appendix/x86/instructions/SCASB_W_D_Q}
\input{appendix/x86/instructions/SHx}
\input{appendix/x86/instructions/SHRD}
\input{appendix/x86/instructions/STOSB_W_D_Q}
\input{appendix/x86/instructions/SUB}
\input{appendix/x86/instructions/TEST}
\input{appendix/x86/instructions/XOR}
\end{description}

\subsubsection{Реже используемые инструкции}

\begin{description}
\input{appendix/x86/instructions/BSF}
\input{appendix/x86/instructions/BSR}
\input{appendix/x86/instructions/BSWAP}
\input{appendix/x86/instructions/BTC}
\input{appendix/x86/instructions/BTR}
\input{appendix/x86/instructions/BTS}
\input{appendix/x86/instructions/BT}
\input{appendix/x86/instructions/CBW_CWDE_CDQ}
\input{appendix/x86/instructions/CLD}
\input{appendix/x86/instructions/CLI}
\input{appendix/x86/instructions/CLC}
\input{appendix/x86/instructions/CMC}
\input{appendix/x86/instructions/CMOVcc}
\input{appendix/x86/instructions/CMPSB_W_D_Q}
\input{appendix/x86/instructions/CPUID}
\input{appendix/x86/instructions/DIV}
\input{appendix/x86/instructions/IDIV}
\myindex{x86!\Instructions!INT}
\myindex{MS-DOS}

\item[INT] (M): \INS{INT x} \RU{аналогична}\EN{is analogous to} \INS{PUSHF; CALL dword ptr [x*4]} 
\RU{в 16-битной среде}\EN{in 16-bit environment}.
  \RU{Она активно использовалась в MS-DOS, работая как сисколл. Аргументы записывались в регистры
  AX/BX/CX/DX/SI/DI и затем происходил переход на таблицу векторов прерываний (расположенную в самом
  начале адресного пространства)}
  \EN{It was widely used in MS-DOS, functioning as a syscall vector. The registers AX/BX/CX/DX/SI/DI were filled
  with the arguments and then the flow jumped to the address in the Interrupt Vector Table 
  (located at the beginning of the address space)}.
  \RU{Она была очень популярна потому что имела короткий опкод (2 байта) и программе использующая
  сервисы MS-DOS не нужно было заморачиваться узнавая адреса всех функций этих сервисов}
  \EN{It was popular because INT has a short opcode (2 bytes) and the program which needs
  some MS-DOS services is not bother to determine the address of the service's entry point}.
\myindex{x86!\Instructions!IRET}
  \RU{Обработчик прерываний возвращал управление назад при помощи инструкции IRET}
  \EN{The interrupt handler returns the control flow to caller using the IRET instruction}.

  \RU{Самое используемое прерывание в MS-DOS было 0x21, там была основная часть его \ac{API}}
  \EN{The most busy MS-DOS interrupt number was 0x21, serving a huge part of its \ac{API}}.
  \RU{См. также}\EN{See also}: [Ralf Brown \emph{Ralf Brown's Interrupt List}], 
  \RU{самый крупный список всех известных прерываний и вообще там много информации о MS-DOS}
  \EN{for the most comprehensive interrupt lists and other MS-DOS information}.

\myindex{x86!\Instructions!SYSENTER}
\myindex{x86!\Instructions!SYSCALL}
  \RU{Во времена после MS-DOS, эта инструкция все еще использовалась как сискол, и в Linux
  и в Windows (\myref{syscalls}), но позже была заменена инструкцией SYSENTER или SYSCALL}
  \EN{In the post-MS-DOS era, this instruction was still used as syscall both in Linux and 
  Windows (\myref{syscalls}), but was later replaced by the SYSENTER or SYSCALL instructions}.

\item[INT 3] (M): \RU{эта инструкция стоит немного в стороне от}\EN{this instruction is somewhat close to} 
\INS{INT}, \RU{она имеет собственный 1-байтный опкод}\EN{it has its own 1-byte opcode} (\GTT{0xCC}), 
\RU{и активно используется в отладке}\EN{and is actively used while debugging}.
\RU{Часто, отладчик просто записывает байт}\EN{Often, the debuggers just write the} \GTT{0xCC} 
\RU{по адресу в памяти где устанавливается точка останова, и когда исключение поднимается, оригинальный байт
будет восстановлен и оригинальная инструкция по этому адресу исполнена заново}\EN{byte at the address of 
the breakpoint to be set, and when an exception is raised,
the original byte is restored and the original instruction at this address is re-executed}. \\
\RU{В}\EN{As of} \gls{Windows NT}, \RU{исключение}\EN{an} \GTT{EXCEPTION\_BREAKPOINT} \RU{поднимается,
когда \ac{CPU} исполняет эту инструкцию}\EN{exception is to be raised when the \ac{CPU} executes this instruction}.
\RU{Это отладочное событие может быть перехвачено и обработано отладчиком, если он загружен}
\EN{This debugging event may be intercepted and handled by a host debugger, if one is loaded}.
\RU{Если он не загружен, Windows предложит запустить один из зарегистрированных в системе отладчиков}
\EN{If it is not loaded, Windows offers to run one of the registered system debuggers}.
\RU{Если}\EN{If} \ac{MSVS} \RU{установлена, его отладчик может быть загружен и подключен к процессу}\EN{is installed, 
its debugger may be loaded and connected to the process}.
\RU{В целях защиты от}\EN{In order to protect from} \gls{reverse engineering}, \RU{множество анти-отладочных методов
проверяют целостность загруженного кода}\EN{a lot of anti-debugging methods check integrity of the loaded code}.

\RU{В }\ac{MSVC} \RU{есть}\EN{has} \gls{compiler intrinsic} \RU{для этой инструкции}\EN{for the instruction}:
\GTT{\_\_debugbreak()}\footnote{\href{http://msdn.microsoft.com/en-us/library/f408b4et.aspx}{MSDN}}.

\RU{В win32 также имеется функция в}\EN{There is also a win32 function in} kernel32.dll \RU{с названием}\EN{named}
\GTT{DebugBreak()}\footnote{\href{http://msdn.microsoft.com/en-us/library/windows/desktop/ms679297(v=vs.85).aspx}{MSDN}},
\RU{которая также исполняет}\EN{which also executes} \GTT{INT 3}.


\input{appendix/x86/instructions/IN}
\input{appendix/x86/instructions/IRET}
\input{appendix/x86/instructions/LOOP}
\input{appendix/x86/instructions/OUT}
\input{appendix/x86/instructions/POPA}
\input{appendix/x86/instructions/POPCNT}
\input{appendix/x86/instructions/POPF}
\input{appendix/x86/instructions/PUSHA}
\input{appendix/x86/instructions/PUSHF}
\input{appendix/x86/instructions/RCx}
\myindex{x86!\Instructions!ROL}
\myindex{x86!\Instructions!ROR}
\label{ROL_ROR}
\item[ROL/ROR] (M) циклический сдвиг
  
ROL: вращать налево:

\input{rotate_left}

ROR: вращать направо:

\input{rotate_right}

Не смотря на то что многие \ac{CPU} имеют эти инструкции, в \CCpp нет соответствующих операций,
так что компиляторы с этих \ac{PL} обычно не генерируют код использующий эти инструкции.

Чтобы программисту были доступны эти инструкции, в \ac{MSVC} есть псевдофункции
(compiler intrinsics)
\emph{\_rotl()} и \emph{\_rotr()}\FNMSDNROTxURL{},
которые транслируются компилятором напрямую в эти инструкции.


\input{appendix/x86/instructions/SAL}
\input{appendix/x86/instructions/SAR}
\input{appendix/x86/instructions/SETcc}
\input{appendix/x86/instructions/STC}
\input{appendix/x86/instructions/STD}
\input{appendix/x86/instructions/STI}
\input{appendix/x86/instructions/SYSCALL}
\input{appendix/x86/instructions/SYSENTER}
\input{appendix/x86/instructions/UD2}
\input{appendix/x86/instructions/XCHG}
\end{description}

\subsubsection{Инструкции FPU}

Суффикс \TT{-R} в названии инструкции обычно означает, что операнды поменяны местами, суффикс \TT{-P} означает
что один элемент выталкивается из стека после исполнения инструкции, суффикс \TT{-PP} означает, что
выталкиваются два элемента.

\TT{-P} инструкции часто бывают полезны, когда нам уже больше не нужно хранить значение в 
FPU-стеке после операции.

\begin{description}
\input{appendix/x86/instructions/FABS}
\input{appendix/x86/instructions/FADD} % + FADDP
\input{appendix/x86/instructions/FCHS}
\input{appendix/x86/instructions/FCOM} % + FCOMP + FCOMPP
\input{appendix/x86/instructions/FDIVR} % + FDIVRP
\input{appendix/x86/instructions/FDIV} % + FDIVP
\input{appendix/x86/instructions/FILD}
\input{appendix/x86/instructions/FIST} % + FISTP
\input{appendix/x86/instructions/FLD1}
\input{appendix/x86/instructions/FLDCW}
\input{appendix/x86/instructions/FLDZ}
\input{appendix/x86/instructions/FLD}
\input{appendix/x86/instructions/FMUL} % + FMULP
\input{appendix/x86/instructions/FSINCOS}
\input{appendix/x86/instructions/FSQRT}
\input{appendix/x86/instructions/FSTCW} % + FNSTCW
\input{appendix/x86/instructions/FSTSW} % + FNSTSW
\input{appendix/x86/instructions/FST}
\input{appendix/x86/instructions/FSUBR} % + FSUBRP
\input{appendix/x86/instructions/FSUB} % + FSUBP
\myindex{x86!\Instructions!FUCOM}
\myindex{x86!\Instructions!FUCOMP}
\myindex{x86!\Instructions!FUCOMPP}
  \item[FUCOM] ST(i): сравнить ST(0) и ST(i)
  \item[FUCOM] сравнить ST(0) и ST(1)
  \item[FUCOMP] сравнить ST(0) и ST(1); вытолкнуть один элемент из стека.
  \item[FUCOMPP] сравнить ST(0) и ST(1); вытолкнуть два элемента из стека.
 
Инструкция работает так же, как и FCOM, за тем исключением что исключение срабатывает только
  если один из операндов SNaN, но числа QNaN нормально обрабатываются.
 % + FUCOMP + FUCOMPP
\input{appendix/x86/instructions/FXCH}
\end{description}

%\subsubsection{SIMD-инструкции}

% TODO

%\begin{description}
%\input{appendix/x86/instructions/DIVSD}
%\input{appendix/x86/instructions/MOVDQA}
%\input{appendix/x86/instructions/MOVDQU}
%\input{appendix/x86/instructions/PADDD}
%\input{appendix/x86/instructions/PCMPEQB}
%\input{appendix/x86/instructions/PLMULHW}
%\input{appendix/x86/instructions/PLMULLD}
%\input{appendix/x86/instructions/PMOVMSKB}
%\input{appendix/x86/instructions/PXOR}
%\end{description}

% SHLD !
% SHRD !
% BSWAP !
% CMPXCHG
% XADD !
% CMPXCHG8B
% RDTSC !
% PAUSE!

% xsave
% fnclex, fnsave
% movsxd, movaps, wait, sfence, lfence, pushfq
% prefetchw
% REP RETN
% REP BSF
% movnti, movntdq, rdmsr, wrmsr
% ldmxcsr, stmxcsr, invlpg
% swapgs
% movq, movd
% mulsd
% POR
% IRETQ
% pslldq
% psrldq
% cqo, fxrstor, comisd, xrstor, wbinvd, movntq
% fprem
% addsb, subsd, frndint

% rare:
%\item[ENTER]
%\item[LES]
% LDS
% XLAT

\subsubsection{Инструкции с печатаемым ASCII-опкодом}

(В 32-битном режиме.)

\label{printable_x86_opcodes}
\myindex{Shellcode}
Это может пригодиться для создания шеллкодов.
См. также: \myref{subsec:EICAR}.

% FIXME: start at 0x20...
\begin{center}
\begin{longtable}{ | l | l | l | }
\hline
\HeaderColor ASCII-символ & 
\HeaderColor шестнадцатеричный код & 
\HeaderColor x86-инструкция \\
\hline
0	 &30	 &XOR \\
1	 &31	 &XOR \\
2	 &32	 &XOR \\
3	 &33	 &XOR \\
4	 &34	 &XOR \\
5	 &35	 &XOR \\
7	 &37	 &AAA \\
8	 &38	 &CMP \\
9	 &39	 &CMP \\
:	 &3a	 &CMP \\
;	 &3b	 &CMP \\
<	 &3c	 &CMP \\
=	 &3d	 &CMP \\
?	 &3f	 &AAS \\
@	 &40	 &INC \\
A	 &41	 &INC \\
B	 &42	 &INC \\
C	 &43	 &INC \\
D	 &44	 &INC \\
E	 &45	 &INC \\
F	 &46	 &INC \\
G	 &47	 &INC \\
H	 &48	 &DEC \\
I	 &49	 &DEC \\
J	 &4a	 &DEC \\
K	 &4b	 &DEC \\
L	 &4c	 &DEC \\
M	 &4d	 &DEC \\
N	 &4e	 &DEC \\
O	 &4f	 &DEC \\
P	 &50	 &PUSH \\
Q	 &51	 &PUSH \\
R	 &52	 &PUSH \\
S	 &53	 &PUSH \\
T	 &54	 &PUSH \\
U	 &55	 &PUSH \\
V	 &56	 &PUSH \\
W	 &57	 &PUSH \\
X	 &58	 &POP \\
Y	 &59	 &POP \\
Z	 &5a	 &POP \\
\lbrack{}	 &5b	 &POP \\
\textbackslash{}	 &5c	 &POP \\
\rbrack{}	 &5d	 &POP \\
\verb|^|	 &5e	 &POP \\
\_	 &5f	 &POP \\
\verb|`|	 &60	 &PUSHA \\
a	 &61	 &POPA \\

h	 &68	 &PUSH\\
i	 &69	 &IMUL\\
j	 &6a	 &PUSH\\
k	 &6b	 &IMUL\\
p	 &70	 &JO\\
q	 &71	 &JNO\\
r	 &72	 &JB\\
s	 &73	 &JAE\\
t	 &74	 &JE\\
u	 &75	 &JNE\\
v	 &76	 &JBE\\
w	 &77	 &JA\\
x	 &78	 &JS\\
y	 &79	 &JNS\\
z	 &7a	 &JP\\
\hline
\end{longtable}
\end{center}

А также:

\begin{center}
\begin{longtable}{ | l | l | l | }
\hline
\HeaderColor ASCII-символ & 
\HeaderColor шестнадцатеричный код & 
\HeaderColor x86-инструкция \\
\hline
f	 &66	 &(в 32-битном режиме) переключиться на\\
   & & 16-битный размер операнда \\
g	 &67	 &(в 32-битном режиме) переключиться на\\
   & & 16-битный размер адреса \\
\hline
\end{longtable}
\end{center}

\myindex{x86!\Instructions!AAA}
\myindex{x86!\Instructions!AAS}
\myindex{x86!\Instructions!CMP}
\myindex{x86!\Instructions!DEC}
\myindex{x86!\Instructions!IMUL}
\myindex{x86!\Instructions!INC}
\myindex{x86!\Instructions!JA}
\myindex{x86!\Instructions!JAE}
\myindex{x86!\Instructions!JB}
\myindex{x86!\Instructions!JBE}
\myindex{x86!\Instructions!JE}
\myindex{x86!\Instructions!JNE}
\myindex{x86!\Instructions!JNO}
\myindex{x86!\Instructions!JNS}
\myindex{x86!\Instructions!JO}
\myindex{x86!\Instructions!JP}
\myindex{x86!\Instructions!JS}
\myindex{x86!\Instructions!POP}
\myindex{x86!\Instructions!POPA}
\myindex{x86!\Instructions!PUSH}
\myindex{x86!\Instructions!PUSHA}
\myindex{x86!\Instructions!XOR}

В итоге:
AAA, AAS, CMP, DEC, IMUL, INC, JA, JAE, JB, JBE, JE, JNE, JNO, JNS, JO, JP, JS, POP, POPA, PUSH, PUSHA, 
XOR.

