\subsection{Регистры общего пользования}

Ко многим регистрам можно обращаться как к частям размером в байт или 16-битное слово.

Это всё --- наследие от более старых процессоров Intel (вплоть до 8-битного 8080),
все еще поддерживаемое для обратной совместимости.

Старые 8-битные процессоры 8080 имели 16-битные регистры, разделенные на две части.

Программы, написанные для 8080 имели доступ к младшему байту 16-битного регистра, к старшему
байту или к целому 16-битному регистру.

Вероятно, эта возможность была оставлена в 8086 для более простого портирования.
В \ac{RISC} процессорах, такой возможности, как правило, нет.

\myindex{x86-64}
Регистры, имеющие префикс \TT{R-} появились только в x86-64, а префикс \TT{E-} ---в 80386.

Таким образом, R-регистры 64-битные, а E-регистры --- 32-битные.

В x86-64 добавили еще 8 \ac{GPR}: R8-R15.

N.B.: В документации от Intel, для обращения к самому младшему байту к имени регистра
нужно добавлять суффикс \emph{L}: \emph{R8L}, но \ac{IDA} называет эти регистры добавляя суффикс \emph{B}: \emph{R8B}.

\subsubsection{RAX/EAX/AX/AL}
\RegTableOne{RAX}{EAX}{AX}{AH}{AL}

\ac{AKA} аккумулятор.
Результат функции обычно возвращается через этот регистр.

\subsubsection{RBX/EBX/BX/BL}
\RegTableOne{RBX}{EBX}{BX}{BH}{BL}

\subsubsection{RCX/ECX/CX/CL}
\RegTableOne{RCX}{ECX}{CX}{CH}{CL}

\ac{AKA} счетчик:
используется в этой роли в инструкциях с префиксом REP и в инструкциях сдвига
(SHL/SHR/RxL/RxR).

\subsubsection{RDX/EDX/DX/DL}
\RegTableOne{RDX}{EDX}{DX}{DH}{DL}

\subsubsection{RSI/ESI/SI/SIL}
\RegTableTwo{RSI}{ESI}{SI}{SIL}

\ac{AKA} \q{source index}. Используется как источник в инструкциях
REP MOVSx, REP CMPSx.

\subsubsection{RDI/EDI/DI/DIL}
\RegTableTwo{RDI}{EDI}{DI}{DIL}

\ac{AKA} \q{destination index}. Используется как указатель на место назначения в инструкции
REP MOVSx, REP STOSx.

% TODO навести тут порядок
\subsubsection{R8/R8D/R8W/R8L}
\RegTableFour{R8}{R8D}{R8W}{R8L}

\subsubsection{R9/R9D/R9W/R9L}
\RegTableFour{R9}{R9D}{R9W}{R9L}

\subsubsection{R10/R10D/R10W/R10L}
\RegTableFour{R10}{R10D}{R10W}{R10L}

\subsubsection{R11/R11D/R11W/R11L}
\RegTableFour{R11}{R11D}{R11W}{R11L}

\subsubsection{R12/R12D/R12W/R12L}
\RegTableFour{R12}{R12D}{R12W}{R12L}

\subsubsection{R13/R13D/R13W/R13L}
\RegTableFour{R13}{R13D}{R13W}{R13L}

\subsubsection{R14/R14D/R14W/R14L}
\RegTableFour{R14}{R14D}{R14W}{R14L}

\subsubsection{R15/R15D/R15W/R15L}
\RegTableFour{R15}{R15D}{R15W}{R15L}

\subsubsection{RSP/ESP/SP/SPL}
\RegTableFour{RSP}{ESP}{SP}{SPL}

\ac{AKA} \gls{stack pointer}. Обычно всегда указывает на текущий стек, кроме тех случаев,
когда он не инициализирован.

\subsubsection{RBP/EBP/BP/BPL}
\RegTableFour{RBP}{EBP}{BP}{BPL}

\ac{AKA} frame pointer. Обычно используется для доступа к локальным переменным функции и аргументам,
Больше о нем: (\myref{stack_frame}).

\subsubsection{RIP/EIP/IP}

\begin{center}
\begin{tabular}{ | l | l | l | l | l | l | l | l | l |}
\hline
\RegHeaderTop \\
\hline
\RegHeader \\
\hline
\multicolumn{8}{ | c | }{RIP\textsuperscript{x64}} \\
\hline
\multicolumn{4}{ | c | }{} & \multicolumn{4}{ c | }{EIP} \\
\hline
\multicolumn{6}{ | c | }{} & \multicolumn{2}{ c | }{IP} \\
\hline
\end{tabular}
\end{center}

\ac{AKA} \q{instruction pointer}
\footnote{Иногда называется также \q{program counter}}.
Обычно всегда указывает на инструкцию, которая сейчас будет исполняться
Напрямую модифицировать регистр нельзя, хотя можно делать так (что равноценно):

\begin{lstlisting}
MOV EAX, ...
JMP EAX
\end{lstlisting}

\RU{Либо}\EN{Or}:

\begin{lstlisting}
PUSH value
RET
\end{lstlisting}

\subsubsection{CS/DS/ES/SS/FS/GS}

16-битные регистры, содержащие селектор кода (CS), 
данных (DS), стека (SS).\\
\\
\myindex{TLS}
\myindex{Windows!TIB}
FS в win32 указывает на \ac{TLS}, а в Linux на эту роль был выбран GS.
Это сделано для более быстрого доступа к \ac{TLS} и прочим структурам там вроде \ac{TIB}.
\\
В прошлом эти регистры использовались как сегментные регистры
(\myref{8086_memory_model}).

\subsubsection{Регистр флагов}
\myindex{x86!\Registers!\Flags}
\label{EFLAGS}
\ac{AKA} EFLAGS.

\small
\begin{center}
\begin{tabular}{ | l | l | l | }
\hline
\headercolor{} Бит (маска) &
\headercolor{} Аббревиатура (значение) &
\headercolor{} Описание \\
\hline
0 (1) & CF (Carry) & Флаг переноса. \\
      &            & Инструкции CLC/STC/CMC используются \\
      &            & для установки/сброса/инвертирования этого флага \\
\hline
2 (4) & PF (Parity) & Флаг четности (\myref{parity_flag}). \\
\hline
4 (0x10) & AF (Adjust) & Существует только для работы с \ac{BCD}-числами \\
\hline
6 (0x40) & ZF (Zero) & Выставляется в 0 \\
         &           & если результат последней операции был равен 0. \\
\hline
7 (0x80) & SF (Sign) & Флаг знака. \\
\hline
8 (0x100) & TF (Trap) & Применяется при отладке. \\
&         &             Если включен, то после исполнения каждой инструкции \\
&         &             будет сгенерировано исключение. \\
\hline
9 (0x200) & IF (Interrupt enable) & Разрешены ли прерывания. \\
          &                       & Инструкции CLI/STI используются \\
	  &                       & для установки/сброса этого флага \\
\hline
10 (0x400) & DF (Direction) & Задается направление для инструкций \\
           &                & REP MOVSx/CMPSx/LODSx/SCASx.\\
           &                & Инструкции CLD/STD используются \\
	   &                & для установки/сброса этого флага \\
	   &                & См.также: \myref{memmove_and_DF}. \\
\hline
11 (0x800) & OF (Overflow) & Переполнение. \\
\hline
12, 13 (0x3000) & IOPL (I/O privilege level)\textsuperscript{i286} & \\
\hline
14 (0x4000) & NT (Nested task)\textsuperscript{i286} & \\
\hline
16 (0x10000) & RF (Resume)\textsuperscript{i386} & Применяется при отладке. \\
             &                  & Если включить, \\
	     &                  & CPU проигнорирует хардварную точку останова в DRx. \\
\hline
17 (0x20000) & VM (Virtual 8086 mode)\textsuperscript{i386} & \\
\hline
18 (0x40000) & AC (Alignment check)\textsuperscript{i486} & \\
\hline
19 (0x80000) & VIF (Virtual interrupt)\textsuperscript{i586} & \\
\hline
20 (0x100000) & VIP (Virtual interrupt pending)\textsuperscript{i586} & \\
\hline
21 (0x200000) & ID (Identification)\textsuperscript{i586} & \\
\hline
\end{tabular}
\end{center}
\normalsize

Остальные флаги зарезервированы.

\subsection{Регистры FPU}

\myindex{x86!FPU}
8 80-битных регистров работающих как стек: ST(0)-ST(7).
N.B.: \ac{IDA} называет ST(0) просто ST.
Числа хранятся в формате IEEE 754.

Формат значения \emph{long double}:

\bigskip
% a hack used here! http://tex.stackexchange.com/questions/73524/bytefield-package
\begin{center}
\begingroup
\makeatletter
\let\saved@bf@bitformatting\bf@bitformatting
\renewcommand*{\bf@bitformatting}{%
	\ifnum\value{header@val}=21 %
	\value{header@val}=62 %
	\else\ifnum\value{header@val}=22 %
	\value{header@val}=63 %
	\else\ifnum\value{header@val}=23 %
	\value{header@val}=64 %
	\else\ifnum\value{header@val}=30 %
	\value{header@val}=78 %
	\else\ifnum\value{header@val}=31 %
	\value{header@val}=79 %
	\fi\fi\fi\fi\fi
	\saved@bf@bitformatting
}%
\begin{bytefield}[bitwidth=0.03\linewidth]{32}
	\bitheader[endianness=big]{0,21,22,23,30,31} \\
	\bitbox{1}{S} &
	\bitbox{8}{экспонента} &
	\bitbox{1}{I} &
	\bitbox{22}{мантисса}
\end{bytefield}
\endgroup
\end{center}

\begin{center}
( S --- знак, I --- целочисленная часть )
\end{center}

\label{FPU_control_word}
\subsubsection{Регистр управления}

Регистр, при помощи которого можно задавать поведение \ac{FPU}.

%\small
\begin{center}
\begin{tabular}{ | l | l | l | }
\hline
Бит &
Аббревиатура (значение) &
Описание \\
\hline
0   & IM (Invalid operation Mask) & \\
\hline
1   & DM (Denormalized operand Mask) & \\
\hline
2   & ZM (Zero divide Mask) & \\
\hline
3   & OM (Overflow Mask) & \\
\hline
4   & UM (Underflow Mask) & \\
\hline
5   & PM (Precision Mask) & \\
\hline
7   & IEM (Interrupt Enable Mask) & Разрешение исключений, по умолчанию 1 (запрещено) \\
\hline
8, 9 & PC (Precision Control) & Управление точностью \\
     &                        & 00 ~--- 24 бита (REAL4) \\
     &                        & 10 ~--- 53 бита (REAL8) \\
     &                        & 11 ~--- 64 бита (REAL10) \\
\hline
10, 11 & RC (Rounding Control) & Управление округлением \\
       &                       & 00 ~--- (по умолчанию) округлять к ближайшему \\
       &                       & 01 ~--- округлять к $-\infty$ \\
       &                       & 10 ~--- округлять к $+\infty$ \\
       &                       & 11 ~--- округлять к 0 \\
\hline
12 & IC (Infinity Control) & 0 ~--- (по умолчанию) считать $+\infty$ и $-\infty$ за беззнаковое \\
   &                       & 1 ~--- учитывать и $+\infty$ и $-\infty$ \\
\hline
\end{tabular}
\end{center}
%\normalsize

Флагами PM, UM, OM, ZM, DM, IM 
задается, генерировать ли исключения в случае соответствующих ошибок.

\subsubsection{Регистр статуса}

\label{FPU_status_word}
Регистр только для чтения.

\small
\begin{center}
\begin{tabular}{ | l | l | l | }
\hline
Бит &
Аббревиатура (значение) &
Описание \\
\hline
15   & B (Busy) & Работает ли сейчас FPU (1)
или закончил и результаты готовы (0) \\
\hline
14   & C3 & \\
\hline
13, 12, 11 & TOP & указывает, какой сейчас регистр является нулевым \\
\hline
10 & C2 & \\
\hline
9  & C1 & \\
\hline
8  & C0 & \\
\hline
7  & IR (Interrupt Request) & \\
\hline
6  & SF (Stack Fault) & \\
\hline
5  & P (Precision) & \\
\hline
4  & U (Underflow) & \\
\hline
3  & O (Overflow) & \\
\hline
2  & Z (Zero) & \\
\hline
1  & D (Denormalized) & \\
\hline
0  & I (Invalid operation) & \\
\hline
\end{tabular}
\end{center}
\normalsize

Биты SF, P, U, O, Z, D, I сигнализируют об исключениях.

О C3, C2, C1, C0 читайте больше тут: (\myref{Czero_etc}).

N.B.:когда используется регистр ST(x), FPU прибавляет $x$ к TOP по модулю 8 и получается номер
внутреннего регистра.

\subsubsection{Слово ``тэг''}

Этот регистр отражает текущее содержимое регистров чисел.

\begin{center}
\begin{tabular}{ | l | l | l | }
\hline
Бит & Аббревиатура (значение) \\
\hline
15, 14 & Tag(7) \\
\hline
13, 12 & Tag(6) \\
\hline
11, 10 & Tag(5) \\
\hline
9, 8 & Tag(4) \\
\hline
7, 6 & Tag(3) \\
\hline
5, 4 & Tag(2) \\
\hline
3, 2 & Tag(1) \\
\hline
1, 0 & Tag(0) \\
\hline
\end{tabular}
\end{center}

Каждый тэг содержит информацию о физическом регистре FPU (R(x)), но не логическом (ST(x)).

Для каждого тэга:

\begin{itemize}
\item 00 ~--- Регистр содержит ненулевое значение
\item 01 ~--- Регистр содержит 0
\item 10 ~--- Регистр содержит специальное число (\ac{NAN}, $\infty$, или денормализованное число)
\item 11 ~--- Регистр пуст
\end{itemize}

\subsection{Регистры SIMD}

\subsubsection{Регистры MMX}

8 64-битных регистров: MM0..MM7.

\subsubsection{Регистры SSE и AVX}

\myindex{x86-64}
SSE: 8 128-битных регистров: XMM0..XMM7.
В x86-64 добавлено еще 8 регистров: XMM8..XMM15.

AVX это расширение всех регистры до 256 бит.

\subsection{Отладочные регистры}

Применяются для работы с т.н. аппаратными точками останова (\textit{hardware breakpoints}).

\begin{itemize}
	\item DR0 --- адрес точки останова \#1
	\item DR1 --- адрес точки останова \#2
	\item DR2 --- адрес точки останова \#3
	\item DR3 --- адрес точки останова \#4
	\item DR6 --- здесь отображается причина останова
	\item DR7 --- здесь можно задать типы точек останова
\end{itemize}

\subsubsection{DR6}
\myindex{x86!\Registers!DR6}

\begin{center}
\begin{tabular}{ | l | l | }
\hline
\headercolor\ Бит (маска) &
\headercolor\ Описание \\
\hline
0 (1)       &  B0 --- сработала точка останова \#1 \\
\hline
1 (2)       &  B1 --- сработала точка останова \#2 \\
\hline
2 (4)       &  B2 --- сработала точка останова \#3 \\
\hline
3 (8)       &  B3 --- сработала точка останова \#4 \\
\hline
13 (0x2000) &  BD --- была попытка модифицировать один из регистров DRx. \\
            &  может быть выставлен если бит GD выставлен. \\
\hline
14 (0x4000) &  BS --- точка останова типа single step (флаг TF был выставлен в EFLAGS). \\
	    &  Наивысший приоритет. Другие биты также могут быть выставлены. \\
\hline
% TODO: describe BT
15 (0x8000) &  BT (флаг переключения задачи) \\
\hline
\end{tabular}
\end{center}

N.B. Точка останова single step это срабатывающая после каждой инструкции.
Может быть включена выставлением флага TF в EFLAGS (\myref{EFLAGS}).

\subsubsection{DR7}
\myindex{x86!\Registers!DR7}

В этом регистре задаются типы точек останова.

%\small
\begin{center}
\begin{tabular}{ | l | l | }
\hline
\headercolor\ Бит (маска) &
\headercolor\ Описание \\
\hline
0 (1)       &  L0 --- разрешить точку останова \#1 для текущей задачи \\
\hline
1 (2)       &  G0 --- разрешить точку останова \#1 для всех задач \\
\hline
2 (4)       &  L1 --- разрешить точку останова \#2 для текущей задачи \\
\hline
3 (8)       &  G1 --- разрешить точку останова \#2 для всех задач \\
\hline
4 (0x10)    &  L2 --- разрешить точку останова \#3 для текущей задачи \\
\hline
5 (0x20)    &  G2 --- разрешить точку останова \#3 для всех задач \\
\hline
6 (0x40)    &  L3 --- разрешить точку останова \#4 для текущей задачи \\
\hline
7 (0x80)    &  G3 --- разрешить точку останова \#4 для всех задач \\
\hline
8 (0x100)   &  LE --- не поддерживается, начиная с P6 \\
\hline
9 (0x200)   &  GE --- не поддерживается, начиная с P6 \\
\hline
13 (0x2000) &  GD --- исключение будет вызвано если какая-либо инструкция MOV \\
            & попытается модифицировать один из DRx-регистров \\
\hline
16,17 (0x30000)    &  точка останова \#1: R/W --- тип \\
\hline
18,19 (0xC0000)    &  точка останова \#1: LEN --- длина \\
\hline
20,21 (0x300000)   &  точка останова \#2: R/W --- тип \\
\hline
22,23 (0xC00000)   &  точка останова \#2: LEN --- длина \\
\hline
24,25 (0x3000000)  &  точка останова \#3: R/W --- тип \\
\hline
26,27 (0xC000000)  &  точка останова \#3: LEN --- длина \\
\hline
28,29 (0x30000000) &  точка останова \#4: R/W --- тип \\
\hline
30,31 (0xC0000000) &  точка останова \#4: LEN --- длина \\
\hline
\end{tabular}
\end{center}
%\normalsize

Так задается тип точки останова (R/W):

\begin{itemize}
\item 00 --- исполнение инструкции
\item 01 --- запись в память
\item 10 --- обращения к I/O-портам (недоступно из user-mode)
\item 11 --- обращение к памяти (чтение или запись)
\end{itemize}

N.B.: отдельного типа для чтения из памяти действительно нет. \\
\\
Так задается длина точки останова (LEN):

\begin{itemize}
\item 00 --- 1 байт
\item 01 --- 2 байта
\item 10 --- не определено для 32-битного режима, 8 байт для 64-битного
\item 11 --- 4 байта
\end{itemize}


% TODO: control registers
