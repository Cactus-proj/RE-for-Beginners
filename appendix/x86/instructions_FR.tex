\subsection{Instructions}
\label{sec:x86_instructions}

Les instructions marquées avec un (M) ne sont généralement pas générées par le compilateur:
si vous rencontrez l'une d'entre elles, il s'agit probablement de code assembleur
écrit à la main, ou de fonctions intrinsèques (\myref{sec:compiler_intrinsic}).

% TODO ? обратные инструкции

Seules les instructions les plus fréquemment utilisées sont listées ici.
Vous pouvez lire \myref{x86_manuals} pour une documentation complète.

Devez-vous connaître tous les opcodes des instructions par c\oe{}ur?
Non, seulement ceux qui sont utilisés pour patcher du code
(\myref{x86_patching}).
Tout le reste des opcodes n'a pas besoin d'être mémorisé.

\subsubsection{Préfixes}

\myindex{x86!\Prefixes!LOCK}
\myindex{x86!\Prefixes!REP}
\myindex{x86!\Prefixes!REPE/REPNE}
\begin{description}
\label{x86_lock}
\item[LOCK] force le CPU à faire un accès exclusif à la RAM dans un environnement multi-processeurs.
Par simplification, on peut dire que lorsqu'une instruction avec ce préfixe est exécutée,
tous les autres CPU dans un système multi-processeur sont stoppés.
Le plus souvent, c'est utilisé pour les sections critiques, les sémaphores et les mutex.
Couramment utilisé avec ADD, AND, BTR, BTS, CMPXCHG, OR, XADD, XOR.
Vous pouvez en lire plus sur les sections critiques ici (\myref{critical_sections}).

\item[REP] est utilisé avec les instructions MOVSx et STOSx:
exécute l'instruction dans une boucle, le compteur est situé dans le registre CX/ECX/RCX.
Pour une description plus détaillée de ces instructions, voir MOVSx (\myref{REP_MOVSx})
et STOSx (\myref{REP_STOSx}).

Les instructions préfixées par REP sont sensibles au flag DF, qui est utilisé pour définir la direction.

\item[REPE/REPNE] (\ac{AKA} REPZ/REPNZ) utilisé avec les instructions CMPSx et SCASx:
exécute la dernière instruction dans une boucle, le compteur est mis dans le registre \TT{CX}/\TT{ECX}/\TT{RCX}.
Elle s'arrête prématurément si ZF vaut 0 (REPE) ou si ZF vaut 1 (REPNE).

Pour une description plus détaillée de ces instructions, voir CMPSx (\myref{REPE_CMPSx})
et SCASx (\myref{REPNE_SCASx}).

Les instructions préfixées par REPE/REPNE sont sensibles au flag DF, qui est utilisé pour définir la direction.

\end{description}

\subsubsection{Instructions les plus fréquemment utilisées}

Celles-ci peuvent être mémorisées en premier.

\begin{description}
% in order to keep them easily sorted...
\input{appendix/x86/instructions/ADC_FR}
\input{appendix/x86/instructions/ADD}
\input{appendix/x86/instructions/AND}
\input{appendix/x86/instructions/CALL}
\input{appendix/x86/instructions/CMP}
\input{appendix/x86/instructions/DEC_FR}
\myindex{x86!\Instructions!IMUL}
  \item[IMUL] \RU{умножение с учетом знаковых значений}\EN{signed multiply}\FR{multiplication signée}
  \EN{\IMUL often used instead of \MUL, read more about it:}%
  \RU{\IMUL часто используется вместо \MUL, читайте об этом больше:}%
  \FR{\IMUL est souvent utilisé à la place de \MUL, voir ici:} \myref{IMUL_over_MUL}.


\input{appendix/x86/instructions/INC_FR}
\input{appendix/x86/instructions/JCXZ}
\input{appendix/x86/instructions/JMP}
\item[Jcc] (où cc ---  condition code)

\myindex{x86!\Instructions!JAE}
\myindex{x86!\Instructions!JA}
\myindex{x86!\Instructions!JBE}
\myindex{x86!\Instructions!JB}
\myindex{x86!\Instructions!JC}
\myindex{x86!\Instructions!JE}
\myindex{x86!\Instructions!JGE}
\myindex{x86!\Instructions!JG}
\myindex{x86!\Instructions!JLE}
\myindex{x86!\Instructions!JL}
\myindex{x86!\Instructions!JNAE}
\myindex{x86!\Instructions!JNA}
\myindex{x86!\Instructions!JNBE}
\myindex{x86!\Instructions!JNB}
\myindex{x86!\Instructions!JNC}
\myindex{x86!\Instructions!JNE}
\myindex{x86!\Instructions!JNGE}
\myindex{x86!\Instructions!JNG}
\myindex{x86!\Instructions!JNLE}
\myindex{x86!\Instructions!JNL}
\myindex{x86!\Instructions!JNO}
\myindex{x86!\Instructions!JNS}
\myindex{x86!\Instructions!JNZ}
\myindex{x86!\Instructions!JO}
\myindex{x86!\Instructions!JPO}
\myindex{x86!\Instructions!JP}
\myindex{x86!\Instructions!JS}
\myindex{x86!\Instructions!JZ}

Beaucoup de ces instructions ont des synonymes (notés avec AKA), qui ont été ajoutés
par commodité. Ils sont codés avec le même opcode.
L'opcode a un \glslink{jump offset}{offset de saut}.

\label{Jcc}
\begin{description}
\item[JAE] \ac{AKA} JNC: saut si supérieur ou égal (non signé): C=0
\item[JA] \ac{AKA} JNBE: saut si supérieur (non signé): CF=0 et ZF=0
\item[JBE] saut si inférieur ou égal (non signé): CF=1 ou ZF=1
\item[JB] \ac{AKA} JC: saut si inférieur(non signé): CF=1
\item[JC] \ac{AKA} JB: saut si CF=1
\item[JE] \ac{AKA} JZ: saut si égal ou zéro: ZF=1
\item[JGE] saut si supérieur ou égal (signé): SF=OF
\item[JG] saut si supérieur (signé): ZF=0 et SF=OF
\item[JLE] saut si inférieur ou égal (signé): ZF=1 ou SF$\neq$OF
\item[JL] saut si inférieur (signé): SF$\neq$OF
\item[JNAE] \ac{AKA} JC: saut si non supérieur ou égal (non signé): CF=1
\item[JNA] saut si non supérieur (non signé): CF=1 et ZF=1
\item[JNBE] saut si non inférieur ou égal (non signé): CF=0 et ZF=0
\item[JNB] \ac{AKA} JNC: saut si non inférieur (non signé): CF=0
\item[JNC] \ac{AKA} JAE: saut si CF=0, synonyme de JNB.
\item[JNE] \ac{AKA} JNZ: saut si non égal ou non zéro: ZF=0
\item[JNGE] saut si non supérieur ou égal (signé): SF$\neq$OF
\item[JNG] saut si non supérieur (signé): ZF=1 ou SF$\neq$OF
\item[JNLE] saut si non inférieur ou égal (signé): ZF=0 et SF=OF
\item[JNL] saut si non inférieur (signé): SF=OF
\item[JNO] saut si non débordement: OF=0
\item[JNS] saut si le flag SF vaut zéro
\item[JNZ] \ac{AKA} JNE: saut si non égal ou non zéro: ZF=0
\item[JO] saut si débordement: OF=1
\item[JPO] saut si le flag PF vaut 0 (Jump Parity Odd)
\item[JP] \ac{AKA} \ac{JPE}: saut si le flag PF est mis
\item[JS] saut si le flag SF est mis
\item[JZ] \ac{AKA} JE: saut si égal ou zéro: ZF=1
\end{description}


\input{appendix/x86/instructions/LAHF}
\input{appendix/x86/instructions/LEAVE}
\myindex{x86!\Instructions!LEA}
\item[LEA] (\emph{Load Effective Address}) forme une adresse

\label{sec:LEA}

\newcommand{\URLAM}{\href{http://en.wikipedia.org/wiki/Addressing_mode}{Wikipédia}}

Cette instruction n'a pas été conçue pour sommer des valeurs et/ou les multiplier,
mais pour former une adresse, e.g., pour calculer l'adresse d'un élément d'un tableau
en ajoutant l'adresse du tableau, l'index de l'élément multiplié par la taille de
l'élément\footnote{Voir aussi: \URLAM}.
\par
Donc, la différence entre \MOV et \LEA est que \MOV forme une adresse mémoire et
charge une valeur depuis la mémoire ou l'y stocke, alors que \LEA forme simplement une adresse.
\par
Mais néanmoins, elle peut être utilisée pour tout autre calcul.
\par
\LEA est pratique car le calcul qu'elle effectue n'altère pas les flags du \ac{CPU}.
Ceci peut être très important pour les processeurs \ac{OOE} (afin de créer moins de dépendances).
À part ça, au moins à partir du Pentium, l'instruction \LEA est exécutée en 1 cycle.

\begin{lstlisting}[style=customc]
int f(int a, int b)
{
	return a*8+b;
};
\end{lstlisting}

\begin{lstlisting}[caption=MSVC 2010 \Optimizing,style=customasmx86]
_a$ = 8		; size = 4
_b$ = 12	; size = 4
_f	PROC
	mov	eax, DWORD PTR _b$[esp-4]
	mov	ecx, DWORD PTR _a$[esp-4]
	lea	eax, DWORD PTR [eax+ecx*8]
	ret	0
_f	ENDP
\end{lstlisting}

\myindex{Intel C++}
Intel C++ utilise encore plus LEA:

\begin{lstlisting}[style=customc]
int f1(int a)
{
	return a*13;
};
\end{lstlisting}

\begin{lstlisting}[caption=Intel C++ 2011,style=customasmx86]
_f1	PROC NEAR
        mov       ecx, DWORD PTR [4+esp]      ; ecx = a
	lea       edx, DWORD PTR [ecx+ecx*8]  ; edx = a*9
	lea       eax, DWORD PTR [edx+ecx*4]  ; eax = a*9 + a*4 = a*13
        ret
\end{lstlisting}

Ces deux instructions sont plus rapide qu'un IMUL.


\input{appendix/x86/instructions/MOVSB_W_D_Q_FR}
\input{appendix/x86/instructions/MOVSX}
\input{appendix/x86/instructions/MOVZX}
\input{appendix/x86/instructions/MOV_FR}
\myindex{x86!\Instructions!MUL}
  \item[MUL] \RU{умножение с учетом беззнаковых значений}\EN{unsigned multiply}\FR{multiplier sans signe}.
  \EN{\IMUL often used instead of \MUL, read more about it:}%
  \RU{\IMUL часто используется вместо \MUL, читайте об этом больше:}%
  \FR{\IMUL est souvent utilisée au lieu de \MUL, en lire plus ici:} \myref{IMUL_over_MUL}.


\input{appendix/x86/instructions/NEG}
\input{appendix/x86/instructions/NOP_FR}
\input{appendix/x86/instructions/NOT}
\input{appendix/x86/instructions/OR}
\input{appendix/x86/instructions/POP}
\input{appendix/x86/instructions/PUSH}
\input{appendix/x86/instructions/RET_FR}
\input{appendix/x86/instructions/SAHF}
\input{appendix/x86/instructions/SBB}
\input{appendix/x86/instructions/SCASB_W_D_Q_FR}
\input{appendix/x86/instructions/SHx}
\input{appendix/x86/instructions/SHRD}
\input{appendix/x86/instructions/STOSB_W_D_Q_FR}
\input{appendix/x86/instructions/SUB}
\input{appendix/x86/instructions/TEST}
\input{appendix/x86/instructions/XOR}
\end{description}

\subsubsection{Instructions les moins fréquemment utilisées}

\begin{description}
\input{appendix/x86/instructions/BSF}
\input{appendix/x86/instructions/BSR}
\input{appendix/x86/instructions/BSWAP}
\input{appendix/x86/instructions/BTC}
\input{appendix/x86/instructions/BTR}
\input{appendix/x86/instructions/BTS}
\input{appendix/x86/instructions/BT}
\input{appendix/x86/instructions/CBW_CWDE_CDQ_FR}
\input{appendix/x86/instructions/CLD}
\input{appendix/x86/instructions/CLI}
\input{appendix/x86/instructions/CLC}
\input{appendix/x86/instructions/CMC}
\input{appendix/x86/instructions/CMOVcc_FR}
\input{appendix/x86/instructions/CMPSB_W_D_Q_FR}
\input{appendix/x86/instructions/CPUID}
\input{appendix/x86/instructions/DIV}
\input{appendix/x86/instructions/IDIV}
\myindex{x86!\Instructions!INT}
\myindex{MS-DOS}

\item[INT] (M): \INS{INT x} est similaire à \INS{PUSHF; CALL dword ptr [x*4]}
en environnement 16-bit.
  Elle était énormément utilisée dans MS-DOS, fonctionnant comme un vecteur syscall.
  Les registres AX/BX/CX/DX/SI/DI étaient remplis avec les arguments et le flux sautait
  à l'adresse dans la table des vecteurs d'interruption (Interrupt Vector Table,
  située au début de l'espace d'adressage).
  Elle était répandue car INT a un opcode court (2 octets) et le programme qui a
  besoin d'un service MS-DOS ne doit pas déterminer l'adresse du point d'entrée de
  ce service.
\myindex{x86!\Instructions!IRET}
  Le gestionnaire d'interruption renvoie le contrôle du flux à l'appelant en utilisant
  l'instruction IRET.

  Le numéro d'interruption les plus utilisé était 0x21, servant une grande partie
  de on \ac{API}.
  Voir aussi: [Ralf Brown \emph{Ralf Brown's Interrupt List}],
  pour les listes d'interruption plus exhaustives et d'autres informations sur MS-DOS.

\myindex{x86!\Instructions!SYSENTER}
\myindex{x86!\Instructions!SYSCALL}
  Durant l'ère post-MS-DOS, cette instruction était toujours utilisée comme un syscall
  à la fois dans Linux et Windows (\myref{syscalls}), mais fût remplacée plus tard
  par les instructions SYSENTER ou SYSCALL.

\item[INT 3] (M): cette instruction est proche de
\INS{INT}, elle a son propre opcode d'1 octet (\GTT{0xCC}),
et est très utilisée pour le débogage.
Souvent, les débogueurs écrivent simplement l'octet \GTT{0xCC} à l'adresse du point
d'arrêt à mettre, et lorsqu'une exception est levée, l'octet original est restauré
et l'instruction originale à cette adresse est ré-exécutée. \\
Depuis \gls{Windows NT}, une exception \GTT{EXCEPTION\_BREAKPOINT} est déclenchée
lorsque le \ac{CPU} exécute cette instruction.
Cet évènement de débogage peut être intercepté et géré par un débogueur hôte, si
il y en a un de chargé.
S'il n'y en a pas de charger, Windows propose de lancer un des débogueurs enregistré
dans le système.
Si \ac{MSVS} est installé, son débogueur peut être chargé et connecté au processus.
Afin de protéger contre le \gls{reverse engineering}, de nombreuses méthodes anti-débogage
vérifient l'intégrité du code chargé.

\ac{MSVC} possède une \glslink{compiler intrinsic}{fonction intrinsèque} pour l'instruction:
\GTT{\_\_debugbreak()}\footnote{\href{http://msdn.microsoft.com/en-us/library/f408b4et.aspx}{MSDN}}.


Il y a aussi une fonction win32 dans kernel32.dll appelée
\GTT{DebugBreak()}\footnote{\href{http://msdn.microsoft.com/en-us/library/windows/desktop/ms679297(v=vs.85).aspx}{MSDN}},
qui exécute aussi \GTT{INT 3}.


\input{appendix/x86/instructions/IN}
\input{appendix/x86/instructions/IRET}
\input{appendix/x86/instructions/LOOP_FR}
\input{appendix/x86/instructions/OUT}
\input{appendix/x86/instructions/POPA}
\input{appendix/x86/instructions/POPCNT}
\input{appendix/x86/instructions/POPF}
\input{appendix/x86/instructions/PUSHA}
\input{appendix/x86/instructions/PUSHF}
\input{appendix/x86/instructions/RCx}
\myindex{x86!\Instructions!ROL}
\myindex{x86!\Instructions!ROR}
\label{ROL_ROR}
\item[ROL/ROR] (M) décalage cyclique

ROL: rotation à gauche:

\input{rotate_left}

ROR: rotation à droite:

\input{rotate_right}

En dépit du fait que presque presque tous les \ac{CPU}s aient ces instructions, il n'y
a pas d'opération correspondante en \CCpp, donc les compilateurs de ces \ac{PL}s
ne génèrent en général pas ces instructions.

Par commodité pour le programmeur, au moins \ac{MSVC} fourni les pseudo-fonctions
(fonctions intrinsèques du compilateur)
\emph{\_rotl()} et \emph{\_rotr()}\FNMSDNROTxURL{},
qui sont traduites directement par le compilateur en ces instructions.


\input{appendix/x86/instructions/SAL}
\input{appendix/x86/instructions/SAR}
\input{appendix/x86/instructions/SETcc}
\input{appendix/x86/instructions/STC}
\input{appendix/x86/instructions/STD_FR}
\input{appendix/x86/instructions/STI}
\input{appendix/x86/instructions/SYSCALL}
\input{appendix/x86/instructions/SYSENTER}
\input{appendix/x86/instructions/UD2}
\input{appendix/x86/instructions/XCHG}
\end{description}

\subsubsection{Instructions FPU}

Le suffixe \TT{-R} dans le mnémonique signifie en général que les opérandes sont
inversés, le suffixe \TT{-P} implique qu'un élément est supprimé de la pile après
l'exécution de l'instruction, le suffixe \TT{-PP} implique que deux éléments sont
supprimés.

Les instructions \TT{-P} sont souvent utiles lorsque nous n'avons plus besoin que
la valeur soit présente dans la pile FPU après l'opération.

\begin{description}
\input{appendix/x86/instructions/FABS}
\input{appendix/x86/instructions/FADD} % + FADDP
\input{appendix/x86/instructions/FCHS}
\input{appendix/x86/instructions/FCOM_FR} % + FCOMP + FCOMPP
\input{appendix/x86/instructions/FDIVR} % + FDIVRP
\input{appendix/x86/instructions/FDIV} % + FDIVP
\input{appendix/x86/instructions/FILD}
\input{appendix/x86/instructions/FIST_FR} % + FISTP
\input{appendix/x86/instructions/FLD1}
\input{appendix/x86/instructions/FLDCW}
\input{appendix/x86/instructions/FLDZ}
\input{appendix/x86/instructions/FLD}
\input{appendix/x86/instructions/FMUL} % + FMULP
\input{appendix/x86/instructions/FSINCOS}
\input{appendix/x86/instructions/FSQRT}
\input{appendix/x86/instructions/FSTCW_FR} % + FNSTCW
\input{appendix/x86/instructions/FSTSW_FR} % + FNSTSW
\input{appendix/x86/instructions/FST}
\input{appendix/x86/instructions/FSUBR} % + FSUBRP
\input{appendix/x86/instructions/FSUB} % + FSUBP
\myindex{x86!\Instructions!FUCOM}
\myindex{x86!\Instructions!FUCOMP}
\myindex{x86!\Instructions!FUCOMPP}
  \item[FUCOM] ST(i): compare ST(0) et ST(i)
  \item[FUCOM] compare ST(0) et ST(1)
  \item[FUCOMP] compare ST(0) et ST(1); supprime un élément de la pile.
  \item[FUCOMPP] compare ST(0) et ST(1); supprime deux éléments de la pile.

  L'instruction se comporte comme FCOM, mais une exception est levée seulement si
  un opérande est SNaN, tandis que les nombres QNaN sont traités normalement.
 % + FUCOMP + FUCOMPP
\input{appendix/x86/instructions/FXCH}
\end{description}

%\subsubsection{\RU{SIMD-инструкции}\EN{SIMD instructions}}

% TODO

%\begin{description}
%\input{appendix/x86/instructions/DIVSD}
%\input{appendix/x86/instructions/MOVDQA}
%\input{appendix/x86/instructions/MOVDQU}
%\input{appendix/x86/instructions/PADDD}
%\input{appendix/x86/instructions/PCMPEQB}
%\input{appendix/x86/instructions/PLMULHW}
%\input{appendix/x86/instructions/PLMULLD}
%\input{appendix/x86/instructions/PMOVMSKB}
%\input{appendix/x86/instructions/PXOR}
%\end{description}

% SHLD !
% SHRD !
% BSWAP !
% CMPXCHG
% XADD !
% CMPXCHG8B
% RDTSC !
% PAUSE!

% xsave
% fnclex, fnsave
% movsxd, movaps, wait, sfence, lfence, pushfq
% prefetchw
% REP RETN
% REP BSF
% movnti, movntdq, rdmsr, wrmsr
% ldmxcsr, stmxcsr, invlpg
% swapgs
% movq, movd
% mulsd
% POR
% IRETQ
% pslldq
% psrldq
% cqo, fxrstor, comisd, xrstor, wbinvd, movntq
% fprem
% addsb, subsd, frndint

% rare:
%\item[ENTER]
%\item[LES]
% LDS
% XLAT

\subsubsection{Instructions ayant un opcode affichable en ASCII}

(En mode 32-bit.)

\label{printable_x86_opcodes}
\myindex{Shellcode}
Elles peuvent être utilisées pour la création de shellcode.
Voir aussi: \myref{subsec:EICAR}.

% FIXME: start at 0x20...
\begin{center}
\begin{longtable}{ | l | l | l | }
\hline
\HeaderColor caractère ASCII & 
\HeaderColor code hexadécimal & 
\HeaderColor instruction x86 \\
\hline
0	 &30	 &XOR \\
1	 &31	 &XOR \\
2	 &32	 &XOR \\
3	 &33	 &XOR \\
4	 &34	 &XOR \\
5	 &35	 &XOR \\
7	 &37	 &AAA \\
8	 &38	 &CMP \\
9	 &39	 &CMP \\
:	 &3a	 &CMP \\
;	 &3b	 &CMP \\
<	 &3c	 &CMP \\
=	 &3d	 &CMP \\
?	 &3f	 &AAS \\
@	 &40	 &INC \\
A	 &41	 &INC \\
B	 &42	 &INC \\
C	 &43	 &INC \\
D	 &44	 &INC \\
E	 &45	 &INC \\
F	 &46	 &INC \\
G	 &47	 &INC \\
H	 &48	 &DEC \\
I	 &49	 &DEC \\
J	 &4a	 &DEC \\
K	 &4b	 &DEC \\
L	 &4c	 &DEC \\
M	 &4d	 &DEC \\
N	 &4e	 &DEC \\
O	 &4f	 &DEC \\
P	 &50	 &PUSH \\
Q	 &51	 &PUSH \\
R	 &52	 &PUSH \\
S	 &53	 &PUSH \\
T	 &54	 &PUSH \\
U	 &55	 &PUSH \\
V	 &56	 &PUSH \\
W	 &57	 &PUSH \\
X	 &58	 &POP \\
Y	 &59	 &POP \\
Z	 &5a	 &POP \\
\lbrack{}	 &5b	 &POP \\
\textbackslash{}	 &5c	 &POP \\
\rbrack{}	 &5d	 &POP \\
\verb|^|	 &5e	 &POP \\
\_	 &5f	 &POP \\
\verb|`|	 &60	 &PUSHA \\
a	 &61	 &POPA \\
f	 &66	 &en mode 32-bit, change pour une\\
   & & taille d'opérande de 16-bit \\
g	 &67	 &en mode 32-bit, change pour une\\
   & & taille d'adresse 16-bit \\
h	 &68	 &PUSH\\
i	 &69	 &IMUL\\
j	 &6a	 &PUSH\\
k	 &6b	 &IMUL\\
p	 &70	 &JO\\
q	 &71	 &JNO\\
r	 &72	 &JB\\
s	 &73	 &JAE\\
t	 &74	 &JE\\
u	 &75	 &JNE\\
v	 &76	 &JBE\\
w	 &77	 &JA\\
x	 &78	 &JS\\
y	 &79	 &JNS\\
z	 &7a	 &JP\\
\hline
\end{longtable}
\end{center}

De même:
\begin{center}
\begin{longtable}{ | l | l | l | }
\hline
\HeaderColor caractère ASCII & 
\HeaderColor code hexadécimal & 
\HeaderColor instruction x86 \\
\hline
f	 &66	 &en mode 32-bit, change pour une\\
   & & taille d'opérande de 16-bit \\
g	 &67	 &en mode 32-bit, change pour une\\
   & & taille d'adresse 16-bit \\
\hline
\end{longtable}
\end{center}

\myindex{x86!\Instructions!AAA}
\myindex{x86!\Instructions!AAS}
\myindex{x86!\Instructions!CMP}
\myindex{x86!\Instructions!DEC}
\myindex{x86!\Instructions!IMUL}
\myindex{x86!\Instructions!INC}
\myindex{x86!\Instructions!JA}
\myindex{x86!\Instructions!JAE}
\myindex{x86!\Instructions!JB}
\myindex{x86!\Instructions!JBE}
\myindex{x86!\Instructions!JE}
\myindex{x86!\Instructions!JNE}
\myindex{x86!\Instructions!JNO}
\myindex{x86!\Instructions!JNS}
\myindex{x86!\Instructions!JO}
\myindex{x86!\Instructions!JP}
\myindex{x86!\Instructions!JS}
\myindex{x86!\Instructions!POP}
\myindex{x86!\Instructions!POPA}
\myindex{x86!\Instructions!PUSH}
\myindex{x86!\Instructions!PUSHA}
\myindex{x86!\Instructions!XOR}

En résumé:
AAA, AAS, CMP, DEC, IMUL, INC, JA, JAE, JB, JBE, JE, JNE, JNO, JNS, JO, JP, JS, POP, POPA, PUSH, PUSHA, 
XOR.

