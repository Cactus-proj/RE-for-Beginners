\subsection{\RU{Инструкции}\EN{Instructions}}
\label{sec:x86_instructions}

\RU{Инструкции, отмеченные как (M) обычно не генерируются компилятором: если вы видите её, очень может быть
это вручную написанный фрагмент кода, либо это т.н. compiler intrinsic}
\EN{Instructions marked as (M) are not usually generated by the compiler: if you see one of them, it is probably
a hand-written piece of assembly code, or a compiler intrinsic} (\myref{sec:compiler_intrinsic}).

% TODO ? обратные инструкции

\RU{Только наиболее используемые инструкции перечислены здесь}
\EN{Only the most frequently used instructions are listed here}.
\EN{You can read \myref{x86_manuals} for a full documentation.}%
\RU{Обращайтесь к \myref{x86_manuals} для полной документации.}

\RU{Нужно ли заучивать опкоды инструкций на память?}\EN{Do you have to know all instruction's opcodes by heart?}
\RU{Нет, только те, которые часто используются для модификации кода}\EN{No, only those
which are used for code patching} (\myref{x86_patching}).
\RU{Остальные запоминать нет смысла.}\EN{All the rest of the opcodes don't need to be memorized.}

\subsubsection{\RU{Префиксы}\EN{Prefixes}}

\myindex{x86!\Prefixes!LOCK}
\myindex{x86!\Prefixes!REP}
\myindex{x86!\Prefixes!REPE/REPNE}
\begin{description}
\label{x86_lock}
\item[LOCK] \RU{используется чтобы предоставить эксклюзивный доступ к памяти в многопроцессорной среде}
\EN{forces CPU to make exclusive access to the RAM in multiprocessor environment}.
\RU{Для упрощения, можно сказать, что когда исполняется инструкция с этим префиксом, остальные процессоры
в системе останавливаются}\EN{For the sake of simplification, it can be said that when an instruction
with this prefix is executed, all other CPUs in a multiprocessor system are stopped}.
\RU{Чаще все это используется для критических секций, семафоров, мьютексов}\EN{Most often
it is used for critical sections, semaphores, mutexes}.
\RU{Обычно используется с}\EN{Commonly used with} ADD, AND, BTR, BTS, CMPXCHG, OR, XADD, XOR.
\RU{Читайте больше о критических секциях}\EN{You can read more about critical sections here} (\myref{critical_sections}).

\item[REP] \RU{используется с инструкциями}\EN{is used with the} MOVSx \AndENRU STOSx\EN{ instructions}:
\RU{инструкция будет исполняться в цикле, счетчик расположен в регистре CX/ECX/RCX}
\EN{execute the instruction in a loop, the counter is located in the CX/ECX/RCX register}.
\RU{Для более детального описания, читайте больше об инструкциях}
\EN{For a detailed description, read more about the} MOVSx (\myref{REP_MOVSx}) 
\AndENRU STOSx (\myref{REP_STOSx})\EN{ instructions}.

\RU{Работа инструкций с префиксом REP зависит от флага DF, он задает направление}
\EN{The instructions prefixed by REP are sensitive to the DF flag, which is used to set the direction}.

\item[REPE/REPNE] (\ac{AKA} REPZ/REPNZ) \RU{используется с инструкциями}\EN{used with} CMPSx \AndENRU
SCASx\EN{ instructions}:
\RU{инструкция будет исполняться в цикле, счетчик расположен в регистре \TT{CX}/\TT{ECX}/\TT{RCX}}
\EN{execute the last instruction in a loop, the count is set in the \TT{CX}/\TT{ECX}/\TT{RCX} register}. 
\RU{Выполнение будет прервано если ZF будет 0 (REPE) либо если ZF будет 1 (REPNE)}
\EN{It terminates prematurely if ZF is 0 (REPE) or if ZF is 1 (REPNE)}.

\RU{Для более детального описания, читайте больше об инструкциях}
\EN{For a detailed description, you can read more about the} CMPSx (\myref{REPE_CMPSx}) 
\AndENRU SCASx (\myref{REPNE_SCASx})\EN{ instructions}.

\RU{Работа инструкций с префиксами REPE/REPNE зависит от флага DF, он задает направление}
\EN{Instructions prefixed by REPE/REPNE are sensitive to the DF flag, which is used to set the direction}.

\end{description}

\subsubsection{\RU{Наиболее часто используемые инструкции}\EN{Most frequently used instructions}}

\RU{Их можно заучить в первую очередь}\EN{These can be memorized in the first place}.

\begin{description}
% in order to keep them easily sorted...
\input{appendix/x86/instructions/ADC}
\input{appendix/x86/instructions/ADD}
\input{appendix/x86/instructions/AND}
\input{appendix/x86/instructions/CALL}
\input{appendix/x86/instructions/CMP}
\input{appendix/x86/instructions/DEC}
\myindex{x86!\Instructions!IMUL}
  \item[IMUL] \RU{умножение с учетом знаковых значений}\EN{signed multiply}\FR{multiplication signée}
  \EN{\IMUL often used instead of \MUL, read more about it:}%
  \RU{\IMUL часто используется вместо \MUL, читайте об этом больше:}%
  \FR{\IMUL est souvent utilisé à la place de \MUL, voir ici:} \myref{IMUL_over_MUL}.


\input{appendix/x86/instructions/INC}
\input{appendix/x86/instructions/JCXZ}
\input{appendix/x86/instructions/JMP}
\input{appendix/x86/instructions/Jcc}
\input{appendix/x86/instructions/LAHF}
\input{appendix/x86/instructions/LEAVE}
\myindex{x86!\Instructions!LEA}
\item[LEA] (\emph{Load Effective Address}) \RU{сформировать адрес}\EN{form an address}

\label{sec:LEA}

\newcommand{\URLAM}{\href{http://en.wikipedia.org/wiki/Addressing_mode}{wikipedia}}

\RU{Это инструкция, которая задумывалась вовсе не для складывания 
и умножения чисел, 
а для формирования адреса например, из указателя на массив и прибавления индекса к нему
\footnote{См. также: \URLAM}.}
\EN{This instruction was intended not for summing values and multiplication 
but for forming an address, 
e.g., for calculating the address of an array element by adding the array address, element index, with 
multiplication of element size\footnote{See also: \URLAM}.}
\par
\RU{То есть, разница между \MOV и \LEA в том, что \MOV формирует адрес в памяти 
и загружает значение из памяти, либо записывает его туда, а \LEA только формирует адрес.}
\EN{So, the difference between \MOV and \LEA is that \MOV forms a memory address and loads a value
from memory or stores it there, but \LEA just forms an address.}
\par
\RU{Тем не менее, её можно использовать для любых других вычислений}
\EN{But nevertheless, it is can be used for any other calculations}.
\par
\RU{\LEA удобна тем, что производимые ею вычисления не модифицируют флаги \ac{CPU}. 
Это может быть очень важно для \ac{OOE} процессоров (чтобы было меньше зависимостей между данными).}
\EN{\LEA is convenient because the computations performed by it does not alter \ac{CPU} flags.
This may be very important for \ac{OOE} processors (to create less data dependencies).}

\RU{Помимо всего прочего, начиная минимум с Pentium, инструкция LEA исполняется за 1 такт.}%
\EN{Aside from this, starting at least at Pentium, LEA instruction is executed in 1 cycle.}

\begin{lstlisting}[style=customc]
int f(int a, int b)
{
	return a*8+b;
};
\end{lstlisting}

\begin{lstlisting}[caption=\Optimizing MSVC 2010,style=customasmx86]
_a$ = 8		; size = 4
_b$ = 12	; size = 4
_f	PROC
	mov	eax, DWORD PTR _b$[esp-4]
	mov	ecx, DWORD PTR _a$[esp-4]
	lea	eax, DWORD PTR [eax+ecx*8]
	ret	0
_f	ENDP
\end{lstlisting}

\myindex{Intel C++}
Intel C++ \RU{использует LEA даже больше}\EN{uses LEA even more}:

\begin{lstlisting}[style=customc]
int f1(int a)
{
	return a*13;
};
\end{lstlisting}

\begin{lstlisting}[caption=Intel C++ 2011,style=customasmx86]
_f1	PROC NEAR 
        mov       ecx, DWORD PTR [4+esp]      ; ecx = a
	lea       edx, DWORD PTR [ecx+ecx*8]  ; edx = a*9
	lea       eax, DWORD PTR [edx+ecx*4]  ; eax = a*9 + a*4 = a*13
        ret                                
\end{lstlisting}

\RU{Эти две инструкции вместо одной IMUL будут работать быстрее.}
\EN{These two instructions performs faster than one IMUL.}


\input{appendix/x86/instructions/MOVSB_W_D_Q}
\input{appendix/x86/instructions/MOVSX}
\input{appendix/x86/instructions/MOVZX}
\input{appendix/x86/instructions/MOV}
\myindex{x86!\Instructions!MUL}
  \item[MUL] \RU{умножение с учетом беззнаковых значений}\EN{unsigned multiply}\FR{multiplier sans signe}.
  \EN{\IMUL often used instead of \MUL, read more about it:}%
  \RU{\IMUL часто используется вместо \MUL, читайте об этом больше:}%
  \FR{\IMUL est souvent utilisée au lieu de \MUL, en lire plus ici:} \myref{IMUL_over_MUL}.


\input{appendix/x86/instructions/NEG}
\input{appendix/x86/instructions/NOP}
\input{appendix/x86/instructions/NOT}
\input{appendix/x86/instructions/OR}
\input{appendix/x86/instructions/POP}
\input{appendix/x86/instructions/PUSH}
\input{appendix/x86/instructions/RET}
\input{appendix/x86/instructions/SAHF}
\input{appendix/x86/instructions/SBB}
\input{appendix/x86/instructions/SCASB_W_D_Q}
\input{appendix/x86/instructions/SHx}
\input{appendix/x86/instructions/SHRD}
\input{appendix/x86/instructions/STOSB_W_D_Q}
\input{appendix/x86/instructions/SUB}
\input{appendix/x86/instructions/TEST}
\input{appendix/x86/instructions/XOR}
\end{description}

\subsubsection{\RU{Реже используемые инструкции}\EN{Less frequently used instructions}}

\begin{description}
\input{appendix/x86/instructions/BSF}
\input{appendix/x86/instructions/BSR}
\input{appendix/x86/instructions/BSWAP}
\input{appendix/x86/instructions/BTC}
\input{appendix/x86/instructions/BTR}
\input{appendix/x86/instructions/BTS}
\input{appendix/x86/instructions/BT}
\input{appendix/x86/instructions/CBW_CWDE_CDQ}
\input{appendix/x86/instructions/CLD}
\input{appendix/x86/instructions/CLI}
\input{appendix/x86/instructions/CLC}
\input{appendix/x86/instructions/CMC}
\input{appendix/x86/instructions/CMOVcc}
\input{appendix/x86/instructions/CMPSB_W_D_Q}
\input{appendix/x86/instructions/CPUID}
\input{appendix/x86/instructions/DIV}
\input{appendix/x86/instructions/IDIV}
\myindex{x86!\Instructions!INT}
\myindex{MS-DOS}

\item[INT] (M): \INS{INT x} \RU{аналогична}\EN{is analogous to} \INS{PUSHF; CALL dword ptr [x*4]} 
\RU{в 16-битной среде}\EN{in 16-bit environment}.
  \RU{Она активно использовалась в MS-DOS, работая как сисколл. Аргументы записывались в регистры
  AX/BX/CX/DX/SI/DI и затем происходил переход на таблицу векторов прерываний (расположенную в самом
  начале адресного пространства)}
  \EN{It was widely used in MS-DOS, functioning as a syscall vector. The registers AX/BX/CX/DX/SI/DI were filled
  with the arguments and then the flow jumped to the address in the Interrupt Vector Table 
  (located at the beginning of the address space)}.
  \RU{Она была очень популярна потому что имела короткий опкод (2 байта) и программе использующая
  сервисы MS-DOS не нужно было заморачиваться узнавая адреса всех функций этих сервисов}
  \EN{It was popular because INT has a short opcode (2 bytes) and the program which needs
  some MS-DOS services is not bother to determine the address of the service's entry point}.
\myindex{x86!\Instructions!IRET}
  \RU{Обработчик прерываний возвращал управление назад при помощи инструкции IRET}
  \EN{The interrupt handler returns the control flow to caller using the IRET instruction}.

  \RU{Самое используемое прерывание в MS-DOS было 0x21, там была основная часть его \ac{API}}
  \EN{The most busy MS-DOS interrupt number was 0x21, serving a huge part of its \ac{API}}.
  \RU{См. также}\EN{See also}: [Ralf Brown \emph{Ralf Brown's Interrupt List}], 
  \RU{самый крупный список всех известных прерываний и вообще там много информации о MS-DOS}
  \EN{for the most comprehensive interrupt lists and other MS-DOS information}.

\myindex{x86!\Instructions!SYSENTER}
\myindex{x86!\Instructions!SYSCALL}
  \RU{Во времена после MS-DOS, эта инструкция все еще использовалась как сискол, и в Linux
  и в Windows (\myref{syscalls}), но позже была заменена инструкцией SYSENTER или SYSCALL}
  \EN{In the post-MS-DOS era, this instruction was still used as syscall both in Linux and 
  Windows (\myref{syscalls}), but was later replaced by the SYSENTER or SYSCALL instructions}.

\item[INT 3] (M): \RU{эта инструкция стоит немного в стороне от}\EN{this instruction is somewhat close to} 
\INS{INT}, \RU{она имеет собственный 1-байтный опкод}\EN{it has its own 1-byte opcode} (\GTT{0xCC}), 
\RU{и активно используется в отладке}\EN{and is actively used while debugging}.
\RU{Часто, отладчик просто записывает байт}\EN{Often, the debuggers just write the} \GTT{0xCC} 
\RU{по адресу в памяти где устанавливается точка останова, и когда исключение поднимается, оригинальный байт
будет восстановлен и оригинальная инструкция по этому адресу исполнена заново}\EN{byte at the address of 
the breakpoint to be set, and when an exception is raised,
the original byte is restored and the original instruction at this address is re-executed}. \\
\RU{В}\EN{As of} \gls{Windows NT}, \RU{исключение}\EN{an} \GTT{EXCEPTION\_BREAKPOINT} \RU{поднимается,
когда \ac{CPU} исполняет эту инструкцию}\EN{exception is to be raised when the \ac{CPU} executes this instruction}.
\RU{Это отладочное событие может быть перехвачено и обработано отладчиком, если он загружен}
\EN{This debugging event may be intercepted and handled by a host debugger, if one is loaded}.
\RU{Если он не загружен, Windows предложит запустить один из зарегистрированных в системе отладчиков}
\EN{If it is not loaded, Windows offers to run one of the registered system debuggers}.
\RU{Если}\EN{If} \ac{MSVS} \RU{установлена, его отладчик может быть загружен и подключен к процессу}\EN{is installed, 
its debugger may be loaded and connected to the process}.
\RU{В целях защиты от}\EN{In order to protect from} \gls{reverse engineering}, \RU{множество анти-отладочных методов
проверяют целостность загруженного кода}\EN{a lot of anti-debugging methods check integrity of the loaded code}.

\RU{В }\ac{MSVC} \RU{есть}\EN{has} \gls{compiler intrinsic} \RU{для этой инструкции}\EN{for the instruction}:
\GTT{\_\_debugbreak()}\footnote{\href{http://msdn.microsoft.com/en-us/library/f408b4et.aspx}{MSDN}}.

\RU{В win32 также имеется функция в}\EN{There is also a win32 function in} kernel32.dll \RU{с названием}\EN{named}
\GTT{DebugBreak()}\footnote{\href{http://msdn.microsoft.com/en-us/library/windows/desktop/ms679297(v=vs.85).aspx}{MSDN}},
\RU{которая также исполняет}\EN{which also executes} \GTT{INT 3}.


\input{appendix/x86/instructions/IN}
\input{appendix/x86/instructions/IRET}
\input{appendix/x86/instructions/LOOP}
\input{appendix/x86/instructions/OUT}
\input{appendix/x86/instructions/POPA}
\input{appendix/x86/instructions/POPCNT}
\input{appendix/x86/instructions/POPF}
\input{appendix/x86/instructions/PUSHA}
\input{appendix/x86/instructions/PUSHF}
\input{appendix/x86/instructions/RCx}
\input{appendix/x86/instructions/ROx}
\input{appendix/x86/instructions/SAL}
\input{appendix/x86/instructions/SAR}
\input{appendix/x86/instructions/SETcc}
\input{appendix/x86/instructions/STC}
\input{appendix/x86/instructions/STD}
\input{appendix/x86/instructions/STI}
\input{appendix/x86/instructions/SYSCALL}
\input{appendix/x86/instructions/SYSENTER}
\input{appendix/x86/instructions/UD2}
\input{appendix/x86/instructions/XCHG}
\end{description}

\subsubsection{\RU{Инструкции FPU}\EN{FPU instructions}}

\RU{Суффикс \TT{-R} в названии инструкции обычно означает, что операнды поменяны местами, суффикс \TT{-P} означает
что один элемент выталкивается из стека после исполнения инструкции, суффикс \TT{-PP} означает, что
выталкиваются два элемента}%
\EN{\TT{-R} suffix in the mnemonic usually implies that the operands are reversed,
\TT{-P} suffix implies that one element is popped
from the stack after the instruction's execution, \TT{-PP} suffix implies that two elements are popped}.

\TT{-P} \RU{инструкции часто бывают полезны, когда нам уже больше не нужно хранить значение в 
FPU-стеке после операции.}%
\EN{instructions are often useful when we do not need the value in the FPU stack to be 
present anymore after the operation.}

\begin{description}
\input{appendix/x86/instructions/FABS}
\input{appendix/x86/instructions/FADD} % + FADDP
\input{appendix/x86/instructions/FCHS}
\input{appendix/x86/instructions/FCOM} % + FCOMP + FCOMPP
\input{appendix/x86/instructions/FDIVR} % + FDIVRP
\input{appendix/x86/instructions/FDIV} % + FDIVP
\input{appendix/x86/instructions/FILD}
\input{appendix/x86/instructions/FIST} % + FISTP
\input{appendix/x86/instructions/FLD1}
\input{appendix/x86/instructions/FLDCW}
\input{appendix/x86/instructions/FLDZ}
\input{appendix/x86/instructions/FLD}
\input{appendix/x86/instructions/FMUL} % + FMULP
\input{appendix/x86/instructions/FSINCOS}
\input{appendix/x86/instructions/FSQRT}
\input{appendix/x86/instructions/FSTCW} % + FNSTCW
\input{appendix/x86/instructions/FSTSW} % + FNSTSW
\input{appendix/x86/instructions/FST}
\input{appendix/x86/instructions/FSUBR} % + FSUBRP
\input{appendix/x86/instructions/FSUB} % + FSUBP
\input{appendix/x86/instructions/FUCOM} % + FUCOMP + FUCOMPP
\input{appendix/x86/instructions/FXCH}
\end{description}

%\subsubsection{\RU{SIMD-инструкции}\EN{SIMD instructions}}

% TODO

%\begin{description}
%\input{appendix/x86/instructions/DIVSD}
%\input{appendix/x86/instructions/MOVDQA}
%\input{appendix/x86/instructions/MOVDQU}
%\input{appendix/x86/instructions/PADDD}
%\input{appendix/x86/instructions/PCMPEQB}
%\input{appendix/x86/instructions/PLMULHW}
%\input{appendix/x86/instructions/PLMULLD}
%\input{appendix/x86/instructions/PMOVMSKB}
%\input{appendix/x86/instructions/PXOR}
%\end{description}

% SHLD !
% SHRD !
% BSWAP !
% CMPXCHG
% XADD !
% CMPXCHG8B
% RDTSC !
% PAUSE!

% xsave
% fnclex, fnsave
% movsxd, movaps, wait, sfence, lfence, pushfq
% prefetchw
% REP RETN
% REP BSF
% movnti, movntdq, rdmsr, wrmsr
% ldmxcsr, stmxcsr, invlpg
% swapgs
% movq, movd
% mulsd
% POR
% IRETQ
% pslldq
% psrldq
% cqo, fxrstor, comisd, xrstor, wbinvd, movntq
% fprem
% addsb, subsd, frndint

% rare:
%\item[ENTER]
%\item[LES]
% LDS
% XLAT

\subsubsection{\RU{Инструкции с печатаемым ASCII-опкодом}\EN{Instructions having printable ASCII opcode}}

(\RU{В 32-битном режиме}\EN{In 32-bit mode}).

\label{printable_x86_opcodes}
\myindex{Shellcode}
\RU{Это может пригодиться для создания шеллкодов}\EN{These can be suitable for shellcode construction}.
\RU{См. также}\EN{See also}: \myref{subsec:EICAR}.

% FIXME: start at 0x20...
\begin{center}
\begin{longtable}{ | l | l | l | }
\hline
\HeaderColor ASCII\RU{-символ}\EN{ character} & 
\HeaderColor \RU{шестнадцатеричный код}\EN{hexadecimal code} & 
\HeaderColor x86\RU{-инструкция}\EN{ instruction} \\
\hline
0	 &30	 &XOR \\
1	 &31	 &XOR \\
2	 &32	 &XOR \\
3	 &33	 &XOR \\
4	 &34	 &XOR \\
5	 &35	 &XOR \\
7	 &37	 &AAA \\
8	 &38	 &CMP \\
9	 &39	 &CMP \\
:	 &3a	 &CMP \\
;	 &3b	 &CMP \\
<	 &3c	 &CMP \\
=	 &3d	 &CMP \\
?	 &3f	 &AAS \\
@	 &40	 &INC \\
A	 &41	 &INC \\
B	 &42	 &INC \\
C	 &43	 &INC \\
D	 &44	 &INC \\
E	 &45	 &INC \\
F	 &46	 &INC \\
G	 &47	 &INC \\
H	 &48	 &DEC \\
I	 &49	 &DEC \\
J	 &4a	 &DEC \\
K	 &4b	 &DEC \\
L	 &4c	 &DEC \\
M	 &4d	 &DEC \\
N	 &4e	 &DEC \\
O	 &4f	 &DEC \\
P	 &50	 &PUSH \\
Q	 &51	 &PUSH \\
R	 &52	 &PUSH \\
S	 &53	 &PUSH \\
T	 &54	 &PUSH \\
U	 &55	 &PUSH \\
V	 &56	 &PUSH \\
W	 &57	 &PUSH \\
X	 &58	 &POP \\
Y	 &59	 &POP \\
Z	 &5a	 &POP \\
\lbrack{}	 &5b	 &POP \\
\textbackslash{}	 &5c	 &POP \\
\rbrack{}	 &5d	 &POP \\
\verb|^|	 &5e	 &POP \\
\_	 &5f	 &POP \\
\verb|`|	 &60	 &PUSHA \\
a	 &61	 &POPA \\

h	 &68	 &PUSH\\
i	 &69	 &IMUL\\
j	 &6a	 &PUSH\\
k	 &6b	 &IMUL\\
p	 &70	 &JO\\
q	 &71	 &JNO\\
r	 &72	 &JB\\
s	 &73	 &JAE\\
t	 &74	 &JE\\
u	 &75	 &JNE\\
v	 &76	 &JBE\\
w	 &77	 &JA\\
x	 &78	 &JS\\
y	 &79	 &JNS\\
z	 &7a	 &JP\\
\hline
\end{longtable}
\end{center}

\RU{А также}\EN{Also}:

\begin{center}
\begin{longtable}{ | l | l | l | }
\hline
\HeaderColor ASCII\RU{-символ}\EN{ character} & 
\HeaderColor \RU{шестнадцатеричный код}\EN{hexadecimal code} & 
\HeaderColor x86\RU{-инструкция}\EN{ instruction} \\
\hline
f	 &66	 &\RU{(в 32-битном режиме) переключиться на}\EN{(in 32-bit mode) switch to}\\
   & & \RU{16-битный размер операнда}\EN{16-bit operand size} \\
g	 &67	 &\RU{(в 32-битном режиме) переключиться на}\EN{in 32-bit mode) switch to}\\
   & & \RU{16-битный размер адреса}\EN{16-bit address size} \\
\hline
\end{longtable}
\end{center}

\myindex{x86!\Instructions!AAA}
\myindex{x86!\Instructions!AAS}
\myindex{x86!\Instructions!CMP}
\myindex{x86!\Instructions!DEC}
\myindex{x86!\Instructions!IMUL}
\myindex{x86!\Instructions!INC}
\myindex{x86!\Instructions!JA}
\myindex{x86!\Instructions!JAE}
\myindex{x86!\Instructions!JB}
\myindex{x86!\Instructions!JBE}
\myindex{x86!\Instructions!JE}
\myindex{x86!\Instructions!JNE}
\myindex{x86!\Instructions!JNO}
\myindex{x86!\Instructions!JNS}
\myindex{x86!\Instructions!JO}
\myindex{x86!\Instructions!JP}
\myindex{x86!\Instructions!JS}
\myindex{x86!\Instructions!POP}
\myindex{x86!\Instructions!POPA}
\myindex{x86!\Instructions!PUSH}
\myindex{x86!\Instructions!PUSHA}
\myindex{x86!\Instructions!XOR}

\RU{В итоге}\EN{In summary}:
AAA, AAS, CMP, DEC, IMUL, INC, JA, JAE, JB, JBE, JE, JNE, JNO, JNS, JO, JP, JS, POP, POPA, PUSH, PUSHA, 
XOR.

