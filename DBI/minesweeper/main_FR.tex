\section{Cracker Minesweeper avec PIN}

\newcommand{\GitHubMinesweeperURL}{\GitHubTreeMasterURL/DBI/minesweeper}

Dans ce livre, j'ai expliqué comment cracker Minesweeper pour Windows XP: \myref{minesweeper_winxp}.

Le Minesweeper de Windows Vista et 7 est différent: il a probablement été (r)écrit
en C++, et l'information de la case n'est maintenant plus stockée dans un tableau
global, mais plutôt dans des blocs du heap alloués par malloc.

Ceci est un cas où nous pouvons essayer l'outil PIN DBI.

\subsection{Intercepter tous les appels à rand()}

Tout d'abord, puisque Minesweeper dispose les mines aléatoirement, il doit appeler
rand() ou une fonction similaire.
Essayons d'intercepter tous les appels à rand(): \url{\GitHubMinesweeperURL/minesweeper1.cpp}.

Nous pouvons maintenant le lancer:

\begin{lstlisting}
c:\pin-3.2-81205-msvc-windows\pin.exe -t minesweeper1.dll -- C:\PATH\TO\MineSweeper.exe
\end{lstlisting}

Durant le démarrage, PIN cherche tous les appels à la fonction rand() et ajoute un
hook juste après chaque appel.
Le hook est la fonction RandAfter() que nous avons défini: elle logue la valeur et l'adresse
de retour.
Voici un log que j'ai obtenu en lançant la configuration 9*9 standard (10 mines):
 \url{\GitHubMinesweeperURL/minesweeper1.out.10mines}.
La fonction rand() a été appelée de nombreuses fois depuis différents endroits, mais
a été appelée depuis 0x10002770d exactement 10 fois.
J'ai changé la configuration de Minesweeper à 16*16 (40 mines) et rand() a été appelée
40 fois depuis 0x10002770d.
Donc oui, c'est ce que l'on cherche.
Lorsque je charge minesweeper.exe (depuis Windows 7) dans IDA et une fois que le
PDB est récupéré depuis le site web de Microsoft, la fonction qui appelle rand()
en 0x10002770d est appelée Board::placeMines().

\subsection{Remplacer les appels à rand() par notre function}

Essayons maintenant de remplacer la fonction rand() avec notre version, qui renvoie
toujours zéro: \url{\GitHubMinesweeperURL/minesweeper2.cpp}.
Durant le démarrage, PIN remplace tous les appels à la fonction rand() par des appels
à notre fonction, qui écrit dans le log et renvoie zéro.
Ok, je l'ai lancé et ai cliqué sur la case la plus en haut à gauche.

\begin{figure}[H]
\centering
\myincludegraphicsSmall{DBI/minesweeper/minesweeper1.png}
%\caption{TODO}
\end{figure}

Oui, contrairement à Minesweeper de Windows XP, les mines sont placées aléatoirement
\emph{après} que l'utilisateur ai cliqué sur une case, afin de garantir qu'il n'y
a pas de mine sur la première case cliquée par l'utilisateur.
Donc Minesweeper a placé les mines dans des cases autres que celle la plus en
haut à gauche (où j'ai cliqué).

Maintenant j'ai cliqué sur la case la plus en haut à droite:

\begin{figure}[H]
\centering
\myincludegraphicsSmall{DBI/minesweeper/minesweeper2.png}
%\caption{TODO}
\end{figure}

Ceci peut-être une sorte de blague? Je ne sais pas.

J'ai cliqué sur la 5ème case (droite du milieu) de la 1ère ligne:

\begin{figure}[H]
\centering
\myincludegraphicsSmall{DBI/minesweeper/minesweeper3.png}
%\caption{TODO}
\end{figure}

C'est bien, car Minesweeper peut effectuer un placement correct même avec un \ac{PRNG}
aussi mauvais!

\subsection{Regarder comment les mines sont placées}

Comment pouvons-nous obtenir des informations sur où les mines sont placées?
Le résultat de rand() semble être inutile: elle renvoie zéro à chaque fois, mais
Minesweeper a réussi à placer les mines dans des cases différentes, quoique, alignées.

Ce Minesweeper est aussi écrit dans la tradition C++, donc il n'a pas de tableau
global.

Mettons-nous dans la peau du programmeur.
Il doit y avoir une boucle comme:

\begin{lstlisting}
for (int i; i<mines_total; i++)
{
	// get coordinates using rand()
	// put a cell: in other words, modify a block allocated in heap
};
\end{lstlisting}

Comment pouvons-nous obtenir des information sur le bloc de qui est modifié à la
2nde étape?
Ce que nous devons faire:
1) suivre toutes les allocations dans la heap en interceptant malloc()/realloc()/free().
2) suivre toutes les écritures en mémoire (lent).
3) suivre les appels à rand().

Maintenant l'algorithme:
1) suivre tous les blocs du heap qui sont modifiés entre le 1er et le 2nd appel à
rand() depuis 0x10002770d;
2) à chaque fois qu'un bloc du heap est libéré, afficher son contenu.

Suivre toutes les écritures en mémoire est lent, mais après le 2nd appel à rand(),
nous n'avons plus besoin de les suivre (puisque nous avons déjà obtenu une liste
de blocs intéressants à ce point), donc nous arrêtons.

Maintenant le code: \url{\GitHubMinesweeperURL/minesweeper3.cpp}.

Il s'avère que seulement 4 blocs de heap sont modifiés entre les deux premiers appels
à rand(), voici à quoi ils ressemblent:

\begin{lstlisting}
free(0x20aa6360)
free(): we have this block in our records, size=0x28
0x20AA6360: 36 00 00 00 4E 00 00 00-2D 00 00 00 29 00 00 00 "6...N...-...)..."
0x20AA6370: 06 00 00 00 37 00 00 00-35 00 00 00 19 00 00 00 "....7...5......."
0x20AA6380: 46 00 00 00 0B 00 00 00-                        "F.......        "

...

free(0x20af9d10)
free(): we have this block in our records, size=0x18
0x20AF9D10: 0A 00 00 00 0A 00 00 00-0A 00 00 00 00 00 00 00 "................"
0x20AF9D20: 60 63 AA 20 00 00 00 00-                        "`c. ....        "

...

free(0x20b28b20)
free(): we have this block in our records, size=0x140
0x20B28B20: 02 00 00 00 03 00 00 00-04 00 00 00 05 00 00 00 "................"
0x20B28B30: 07 00 00 00 08 00 00 00-0C 00 00 00 0D 00 00 00 "................"
0x20B28B40: 0E 00 00 00 0F 00 00 00-10 00 00 00 11 00 00 00 "................"
0x20B28B50: 12 00 00 00 13 00 00 00-14 00 00 00 15 00 00 00 "................"
0x20B28B60: 16 00 00 00 17 00 00 00-18 00 00 00 1A 00 00 00 "................"
0x20B28B70: 1B 00 00 00 1C 00 00 00-1D 00 00 00 1E 00 00 00 "................"
0x20B28B80: 1F 00 00 00 20 00 00 00-21 00 00 00 22 00 00 00 ".... ...!..."..."
0x20B28B90: 23 00 00 00 24 00 00 00-25 00 00 00 26 00 00 00 "#...\$...%...&..."
0x20B28BA0: 27 00 00 00 28 00 00 00-2A 00 00 00 2B 00 00 00 "'...(...*...+..."
0x20B28BB0: 2C 00 00 00 2E 00 00 00-2F 00 00 00 30 00 00 00 ",......./...0..."
0x20B28BC0: 31 00 00 00 32 00 00 00-33 00 00 00 34 00 00 00 "1...2...3...4..."
0x20B28BD0: 38 00 00 00 39 00 00 00-3A 00 00 00 3B 00 00 00 "8...9...:...;..."
0x20B28BE0: 3C 00 00 00 3D 00 00 00-3E 00 00 00 3F 00 00 00 "<...=...>...?..."
0x20B28BF0: 40 00 00 00 41 00 00 00-42 00 00 00 43 00 00 00 "@...A...B...C..."
0x20B28C00: 44 00 00 00 45 00 00 00-47 00 00 00 48 00 00 00 "D...E...G...H..."
0x20B28C10: 49 00 00 00 4A 00 00 00-4B 00 00 00 4C 00 00 00 "I...J...K...L..."
0x20B28C20: 4D 00 00 00 4F 00 00 00-50 00 00 00 50 00 00 00 "M...O...P...P..."
0x20B28C30: 50 00 00 00 50 00 00 00-50 00 00 00 50 00 00 00 "P...P...P...P..."
0x20B28C40: 50 00 00 00 50 00 00 00-50 00 00 00 50 00 00 00 "P...P...P...P..."
0x20B28C50: 50 00 00 00 00 00 00 00-00 00 00 00 00 00 00 00 "P..............."

...

free(0x20af9cf0)
free(): we have this block in our records, size=0x18
0x20AF9CF0: 43 00 00 00 50 00 00 00-10 00 00 00 20 00 74 00 "C...P....... .t."
0x20AF9D00: 20 8B B2 20 00 00 00 00-                        " .. ....        "
\end{lstlisting}

Nous voyons facilement que les plus gros blocs (avec une taille de 0x28 et 0x140) sont
juste des tableaux de valeurs jusqu'à $\approx$ 0x50.
Attendez... 0x50 est 80 en représentation décimale. et 9*9=81 (configuration standard
de Minesweeper).

Après une rapide investigation, j'ai trouvé que chaque élément 32-bit est en fait
les coordonnées d'une case.
Une case est représentée en utilisant un seul nombre, c'est un nombre dans un tableau-2D.
Ligne et colonne de chaque mine sont décodées comme ceci: \emph{row=n / WIDTH; col=n \% HEIGHT;}

Lorsque j'ai essayé de décoder ces deux blocs les plus gros, j'ai obtenu ces
cartes de case:

\begin{lstlisting}
try_to_dump_cells(). unique elements=0xa
......*..
..*......
.......*.
.........
.....*...
*.......*
**.......
.......*.
......*..

...

try_to_dump_cells(). unique elements=0x44
*.****.**
...******
*******.*
*********
*****.***
.*******.
..*******
*******.*
******.**
\end{lstlisting}

Il semble que le premier bloc soit juste une liste des mines placées, tandis que le
second bloc est une liste des cases libres, mais le second semble quelque peu désynchroniser
du premier, et une version inversée du premier ne coïncide que partiellement.
Néanmoins, la première carte est correcte - nous pouvons jeter un coup d'\oe{}il
dans le fichier de log alors que Minesweeper est encore chargé et presque toutes
les cases sont cachées, et cliquer tranquillement sur les cases marquées d'un point
ici.

Il semble donc que lorsque l'utilisateur clique pour l première fois quelque part,
Minesweeper place les 10 mines, puis détruit le bloc avec leurs liste (peut-être
copie-t-il toutes les données dans un autre bloc avant?), donc nous pouvons les voir
lors de l'appel à free().

Un autre fait: la méthode Array<NodeType>::Add(NodeType) modifie les blocs que nous
avons observé, et est appelée depuis de nombreux endroits, Board::placeMines() incluse.
Mais c'est cool: je ne suis jamais allé dans les détails, tout a été résolu simplement
en utilisant PIN.

Les fichiers: \url{\GitHubMinesweeperURL}.

\subsection{Exercice}

Essayez de comprendre comment le résultat de rand() est converti en coordonnée(s).
Pour blaguer, faite que rand() renvoie des résultats tels que les mines
soient placées en formant un symbole ou une figure.
