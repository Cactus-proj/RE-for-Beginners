\section{Utiliser PIN DBI pour intercepter les XOR}

\newcommand{\RepoPinXORURL}{\RepoURL/DBI/XOR/files/}

PIN d'Intel est un outil \ac{DBI}.
Cela signifie qu'il prend un binaire compilé et y insère vos instructions, où vous
voulez.

Essayons d'intercepter toutes les instructions XOR.
Elles sont utilisées intensément en cryptographie, et nous pouvons essayer de lancer l'archiveur
WinRAR en mode chiffrement avec l'espoir que des instructions sont effectivement utilisées durant
le chiffrement.

Voici le code source de mon outil PIN: \url{\RepoPinXORURL/XOR_ins.cpp}.

Le code est presque auto-documenté: il scanne le fichier exécutable en entrée à la recherche
des instructions XOR/PXOR et insère un appel à notre fonction avant chaque.
La fonction log\_info() vérifie d'abord si les opérandes sont différents (puisque
l'instruction XOR est souvent utilisée pour effacer simplement un registre, comme
XOR EAX, EAX), et si ils sont différents, il incrémente un compteur à cette EIP/RIP,
afin que les statistiques soient collectées.

J'ai préparé deux fichiers pour tester: test1.bin (30720 octets) et test2.bin (5547752 octets),
je vais les compresser avec RAR avec un mot de passe et voir les différences dans
les statistiques.

Vous devez aussi désactiver \ac{ASLR}
\footnote{\url{https://stackoverflow.com/q/9560993}},
afin que l'outil PIN rapporte les mêmes RIPs que dans l'exécutable RAR.

Maintenant, lançons-le:

\begin{lstlisting}
c:\pin-3.2-81205-msvc-windows\pin.exe -t XOR_ins.dll -- rar a -pLongPassword tmp.rar test1.bin
c:\pin-3.2-81205-msvc-windows\pin.exe -t XOR_ins.dll -- rar a -pLongPassword tmp.rar test2.bin
\end{lstlisting}

Maintenant voici les statistiques pour test1.bin: \\
\url{\RepoPinXORURL/XOR_ins.out.test1}.
... et pour test2.bin: \\
\url{\RepoPinXORURL/XOR_ins.out.test2}.
Jusqu'ici, vous pouvez ignorer toutes les adresses autres que ip=0x1400xxxxx, qui
sont dans d'autres DLLs.

Maintenant, regardons la différence: \url{\RepoPinXORURL/XOR_ins.diff}.

Certaines instructions XOR sont exécutées plus souvent pour test2.bin (qui est plus
gros) que pour test1.bin (qui est plus petit).
Donc elles sont clairement liées à la taille du fichier!

Le premier bloc de différence est:

\begin{lstlisting}
< ip=0x140017b21 count=0xd84
< ip=0x140017b48 count=0x81f
< ip=0x140017b59 count=0x858
< ip=0x140017b6a count=0xc13
< ip=0x140017b7b count=0xefc
< ip=0x140017b8a count=0xefd
< ip=0x140017b92 count=0xb86
< ip=0x140017ba1 count=0xf01
---
> ip=0x140017b21 count=0x9eab5
> ip=0x140017b48 count=0x79863
> ip=0x140017b59 count=0x862e8
> ip=0x140017b6a count=0x99495
> ip=0x140017b7b count=0xa891c
> ip=0x140017b8a count=0xa89f4
> ip=0x140017b92 count=0x8ed72
> ip=0x140017ba1 count=0xa8a8a
\end{lstlisting}

C'est en effet une sorte de boucle à l'intérieur de RAR.EXE:

\begin{lstlisting}
.text:0000000140017B21 loc_140017B21:
.text:0000000140017B21                 xor     r11d, [rbx]
.text:0000000140017B24                 mov     r9d, [rbx+4]
.text:0000000140017B28                 add     rbx, 8
.text:0000000140017B2C                 mov     eax, r9d
.text:0000000140017B2F                 shr     eax, 18h
.text:0000000140017B32                 movzx   edx, al
.text:0000000140017B35                 mov     eax, r9d
.text:0000000140017B38                 shr     eax, 10h
.text:0000000140017B3B                 movzx   ecx, al
.text:0000000140017B3E                 mov     eax, r9d
.text:0000000140017B41                 shr     eax, 8
.text:0000000140017B44                 mov     r8d, [rsi+rdx*4]
.text:0000000140017B48                 xor     r8d, [rsi+rcx*4+400h]
.text:0000000140017B50                 movzx   ecx, al
.text:0000000140017B53                 mov     eax, r11d
.text:0000000140017B56                 shr     eax, 18h
.text:0000000140017B59                 xor     r8d, [rsi+rcx*4+800h]
.text:0000000140017B61                 movzx   ecx, al
.text:0000000140017B64                 mov     eax, r11d
.text:0000000140017B67                 shr     eax, 10h
.text:0000000140017B6A                 xor     r8d, [rsi+rcx*4+1000h]
.text:0000000140017B72                 movzx   ecx, al
.text:0000000140017B75                 mov     eax, r11d
.text:0000000140017B78                 shr     eax, 8
.text:0000000140017B7B                 xor     r8d, [rsi+rcx*4+1400h]
.text:0000000140017B83                 movzx   ecx, al
.text:0000000140017B86                 movzx   eax, r9b
.text:0000000140017B8A                 xor     r8d, [rsi+rcx*4+1800h]
.text:0000000140017B92                 xor     r8d, [rsi+rax*4+0C00h]
.text:0000000140017B9A                 movzx   eax, r11b
.text:0000000140017B9E                 mov     r11d, r8d
.text:0000000140017BA1                 xor     r11d, [rsi+rax*4+1C00h]
.text:0000000140017BA9                 sub     rdi, 1
.text:0000000140017BAD                 jnz     loc_140017B21
\end{lstlisting}

Que fait-elle? Aucune idée à ce stade.

La suivante:

\begin{lstlisting}
< ip=0x14002c4f1 count=0x4fce
---
> ip=0x14002c4f1 count=0x4463be
\end{lstlisting}

0x4fce est 20430, qui est proche de la taille de test1.bin (30720 octets).
0x4463be est 4481982, qui est proche de la taille de test2.bin (5547752 octets).
Par égal, mais proche.

Ceci est un morceau de code avec cette instruction XOR:

\begin{lstlisting}
.text:000000014002C4EA loc_14002C4EA:
.text:000000014002C4EA                 movzx   eax, byte ptr [r8]
.text:000000014002C4EE                 shl     ecx, 5
.text:000000014002C4F1                 xor     ecx, eax
.text:000000014002C4F3                 and     ecx, 7FFFh
.text:000000014002C4F9                 cmp     [r11+rcx*4], esi
.text:000000014002C4FD                 jb      short loc_14002C507
.text:000000014002C4FF                 cmp     [r11+rcx*4], r10d
.text:000000014002C503                 ja      short loc_14002C507
.text:000000014002C505                 inc     ebx
\end{lstlisting}

Le corps de la boucle peut être écrit comme:

\begin{lstlisting}
state = input_byte ^ (state<<5) & 0x7FFF}.
\end{lstlisting}

\emph{state} est ensuite utilisé comme un index dans une table. Est-ce une sorte de \ac{CRC}?
Je ne sais pas, mais ça pourrait être une routine effectuant une somme de contrôle.
Ou peut-être une routine \ac{CRC} optimisée?
Une idée?

Le bloc suivant:

\begin{lstlisting}
< ip=0x14004104a count=0x367
< ip=0x140041057 count=0x367
---
> ip=0x14004104a count=0x24193
> ip=0x140041057 count=0x24193
\end{lstlisting}

\begin{lstlisting}
.text:0000000140041039 loc_140041039:
.text:0000000140041039                 mov     rax, r10
.text:000000014004103C                 add     r10, 10h
.text:0000000140041040                 cmp     byte ptr [rcx+1], 0
.text:0000000140041044                 movdqu  xmm0, xmmword ptr [rax]
.text:0000000140041048                 jz      short loc_14004104E
.text:000000014004104A                 pxor    xmm0, xmm1
.text:000000014004104E
.text:000000014004104E loc_14004104E:
.text:000000014004104E                 movdqu  xmm1, xmmword ptr [rcx+18h]
.text:0000000140041053                 movsxd  r8, dword ptr [rcx+4]
.text:0000000140041057                 pxor    xmm1, xmm0
.text:000000014004105B                 cmp     r8d, 1
.text:000000014004105F                 jle     short loc_14004107C
.text:0000000140041061                 lea     rdx, [rcx+28h]
.text:0000000140041065                 lea     r9d, [r8-1]
.text:0000000140041069
.text:0000000140041069 loc_140041069:
.text:0000000140041069                 movdqu  xmm0, xmmword ptr [rdx]
.text:000000014004106D                 lea     rdx, [rdx+10h]
.text:0000000140041071                 aesenc  xmm1, xmm0
.text:0000000140041076                 sub     r9, 1
.text:000000014004107A                 jnz     short loc_140041069
.text:000000014004107C
\end{lstlisting}

Ce morceau possède les instructions PXOR et AESENC (la dernière est une instruction
de chiffrement \ac{AES}).
Donc oui, nous avons trouvé une fonction de chiffrement, RAR utilise \ac{AES}.

Il y a ensuite un autre gros bloc d'instructions XOR presque contigus:

\begin{lstlisting}
< ip=0x140043e10 count=0x23006
---
> ip=0x140043e10 count=0x23004
499c510
< ip=0x140043e56 count=0x22ffd
---
> ip=0x140043e56 count=0x23002
\end{lstlisting}

Mais le compteur n'est pas très différent pendant la compression/chiffrement de test1.bin/test2.bin.
Qu'y a-t-il à ces adresses?

\begin{lstlisting}
.text:0000000140043E07                 xor     ecx, r9d
.text:0000000140043E0A                 mov     r11d, eax
.text:0000000140043E0D                 and     ecx, r10d
.text:0000000140043E10                 xor     ecx, r8d
.text:0000000140043E13                 rol     eax, 8
.text:0000000140043E16                 and     eax, esi
.text:0000000140043E18                 ror     r11d, 8
.text:0000000140043E1C                 add     edx, 5A827999h
.text:0000000140043E22                 ror     r10d, 2
.text:0000000140043E26                 add     r8d, 5A827999h
.text:0000000140043E2D                 and     r11d, r12d
.text:0000000140043E30                 or      r11d, eax
.text:0000000140043E33                 mov     eax, ebx
\end{lstlisting}

Googlons la constante 5A827999h... ceci ressemble à du SHA-1! mais pourquoi RAR utiliserait-il
SHA-1 pendant le chiffrement?

Voici la réponse:

\begin{lstlisting}
In comparison, WinRAR uses its own key derivation scheme that requires (password length * 2 + 11)*4096 SHA-1 transformations. That’s why it takes longer to brute-force attack encrypted WinRAR archives.
\end{lstlisting}
( \url{http://www.tomshardware.com/reviews/password-recovery-gpu,2945-8.html} )

C'est la génération de la clef: le mot de passe entré est hashé plusieurs fois et
le hash est utilisé comme clef \ac{AES}.
C'est pourquoi nous voyons que le comptage de l'instruction XOR est presque inchangé
lorsque nous passons au fichier de test plus gros.

C'est tout ce qu'il faut faire, ça m'a pris quelques heures d'écrire cet outil et
d'obtenir au moins 3 éléments: 1) c'est probablement une somme de contrôle; 2) chiffrement
\ac{AES}; 3) calcul de somme SHA-1.
La première fonction est encore un mystère pour moi.

Cependant, ceci est impressionnant, car je ne me suis pas plongé dans le code de
RAR (qui est propriétaire, bien sûr). Je n'ai même pas jeté un coup d'\oe{}il dans
le code source de UnRAR (qui est disponible).

Les fichiers, incluant les fichiers de test et l'exécutable RAR que j'ai utilisé (win64, 5.40): \\
\url{\RepoPinXORURL}.

