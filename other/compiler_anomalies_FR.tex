\mysection{Anomalies des compilateurs}
\label{anomaly:Intel}
\myindex{\CompilerAnomaly}

\subsection{\oracle 11.2 et Intel C++ 10.1}

\myindex{Intel C++}
\myindex{\oracle}
\myindex{x86!\Instructions!JZ}

Le compilateur Intel C++ 10.1, qui fut utilisé pour la compilation de \oracle 11.2 pour Linux86, 
émettait parfois deux instructions \JZ successives, sans que la seconde instruction soit jamais 
référencée. Elle était donc inutile.

\lstinputlisting[caption=kdli.o from libserver11.a,style=customasmx86]{other/kdli.lst}

\begin{lstlisting}[caption=from the same code,style=customasmx86]
.text:0811A2A5                   loc_811A2A5: ; CODE XREF: kdliSerLengths+11C
.text:0811A2A5                                ; kdliSerLengths+1C1
.text:0811A2A5 8B 7D 08              mov     edi, [ebp+arg_0]
.text:0811A2A8 8B 7F 10              mov     edi, [edi+10h]
.text:0811A2AB 0F B6 57 14           movzx   edx, byte ptr [edi+14h]
.text:0811A2AF F6 C2 01              test    dl, 1
.text:0811A2B2 75 3E                 jnz     short loc_811A2F2
.text:0811A2B4 83 E0 01              and     eax, 1
.text:0811A2B7 74 1F                 jz      short loc_811A2D8
.text:0811A2B9 74 37                 jz      short loc_811A2F2
.text:0811A2BB 6A 00                 push    0
.text:0811A2BD FF 71 08              push    dword ptr [ecx+8]
.text:0811A2C0 E8 5F FE FF FF        call    len2nbytes
\end{lstlisting}

Il s'agit probablement d'un bug du générateur de code du compilateur qui ne fut pas découvert durant 
les tests de celui-ci car le code produit fonctionnait conformément aux résultats attendus.

% TBT
% Another example from \oracle 11.1.0.6.0 for win32.

%\begin{lstlisting}
%.text:0051FBF8 85 C0                             test    eax, eax
%.text:0051FBFA 0F 84 8F 00 00 00                 jz      loc_51FC8F
%.text:0051FC00 74 1D                             jz      short loc_51FC1F
%\end{lstlisting}

\input{other/anomaly2_FR}
%
\subsection{ftol2() dans MSVC 2012}

Je viens de trouver ceci dans la fonction ftol2() de la bibliothèque C/C++ standard
(routine de conversion float-to-long) de Microsoft Visual Studio 2012.

% TODO translate:
\lstinputlisting[style=customasmx86]{other/ftol2_EN.asm}

Notez les deux \INS{FSTP}-s (stocker un float avec pop) identiques à la fin.
D'abord, j'ai cru qu'il s'agissait d'une anomalie du compilateur (je collectionne
de tels cas tout comme certains collectionnent les papillons), mais il semble qu'il
s'agisse d'un morceau d'assembleur écrit à la main, dans msvcrt.lib il y a un fichier
objet avec cette fonction dedans et on peut y trouver cette chaîne:\\
\verb|f:\dd\vctools\crt_bld\SELF_X86\crt\prebuild\tran\i386\ftol2.asm| ---
qui est sans doute un chemin vers le fichier sur l'ordinateur du développeur où
msvcrt.lib a été généré.

Donc, bogue, typo induite par l'éditeur de texte ou fonctionnalité?
% to be sync
%So, bug, text editor-induced typo, or it was done by intent?
%The code working correctly, anyway.



\subsection{Résumé}

Des anomalies constatées dans d'autres compilateurs figurent également dans ce livre: 
\myref{anomaly:LLVM}, \myref{loops_iterators_loop_anomaly}, \myref{Keil_anomaly},
\myref{MSVC2013_anomaly},
\myref{MSVC_double_JMP_anomaly},
\myref{MSVC2012_anomaly}.

Ces cas sont exposés dans ce livre afin de démontrer que ces compilateurs comportent leurs propres 
erreurs et qu'il convient de ne pas toujours se triturer le cerveau en tentant de comprendre 
pourquoi le compilateur a généré un code aussi étrange.

