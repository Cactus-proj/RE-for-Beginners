\subsection{Loi de De Morgan et décompilation}
\myindex{De Morgan's laws}

Parfois l'optimiseur du compilateur peut utiliser la loi de De Morgan pour rendre
le code plus court/rapide.

Par exemple, ceci:

\begin{lstlisting}[style=customc]
void f(int a, int b, int c, int d)
{
	if (a>0 && b>0)
		printf ("both a and b are positive\n");
	else if (c>0 && d>0)
		printf ("both c and d are positive\n");
	else
		printf ("something else\n");
};
\end{lstlisting}

\dots ça semble assez anodin, lorsque c'est compilé avec GCC 5.4.0 x64 avec optimisation:

\begin{lstlisting}[style=customasmx86]
; \verb|int __fastcall f(int a, int b, int c, int d)|
                public f
f               proc near
                test    edi, edi
                jle     short loc_8
                test    esi, esi
                jg      short loc_30

loc_8:
                test    edx, edx
                jle     short loc_20
                test    ecx, ecx
                jle     short loc_20
                mov     edi, offset s   ; "both c and d are positive"
                jmp     puts

loc_20:
                mov     edi, offset aSomethingElse ; "something else"
                jmp     puts

loc_30:
                mov     edi, offset aAAndBPositive ; "both a and b are positive"

loc_35:
                jmp     puts
f               endp
\end{lstlisting}

\dots ça semble donc anodin, mais Hex-Rays 2.2.0 n'arrive pas vraiment à détecter
qu'en fait des opérations double AND ont été utilisées dans le code source:

\begin{lstlisting}[style=customc]
int __fastcall f(int a, int b, int c, int d)
{
  int result;

  if ( a > 0 && b > 0 )
  {
    result = puts("both a and b are positive");
  }
  else if ( c <= 0 || d <= 0 )
  {
    result = puts("something else");
  }
  else
  {
    result = puts("both c and d are positive");
  }
  return result;
}
\end{lstlisting}

L'expression \verb~c <= 0 || d <= 0~ est la contraposée de \verb|c>0 && d>0| puisque
$\overline{A \cup B} = \overline{A} \cap \overline{B}$ et
$\overline{A \cap B} = \overline{A} \cup \overline{B}$,
Autrement dit,
\verb~!(cond1 || cond2) == !cond1 && !cond2~ et \verb~!(cond1 && cond2) == !cond1 || !cond2~.

Ça vaut la peine de garder ces règles à l'esprit, puisque ces optimisations du
compilateur sont utilisées massivement presque partout.

C'est parfois une bonne idée d'inverser une condition, afin de mieux comprendre
le code.
Ceci est un morceau de code réel décompilé par Hex-Rays:

\begin{lstlisting}[style=customc]
	for (int i=0; i<12; i++)
	{
		if (v1[i-12] != 0.0 || v1[i] != 0.0)
		{
			v108=min(v108, (float)v0[i*24 -2]);
			v113=max(v113, (float)v0[i*24]);
		};
	}
\end{lstlisting}

\dots qui peut être récrit comme:

\begin{lstlisting}[style=customc]
	for (int i=0; i<12; i++)
	{
		if (v1[i-12] == 0.0 && v1[i] == 0.0)
			continue;

		v108=min(v108, (float)v0[i*24 -2]);
		v113=max(v113, (float)v0[i*24]);
	}
\end{lstlisting}

Lequel est le meilleur? Je ne sais pas encore, mais pour une meilleure compréhension,
c'est bien de regarder les deux.

