\mysection{Compiler Anomalien}
\label{anomaly:Intel}
\myindex{\CompilerAnomaly}

\subsection{\oracle 11.2 und Intel C++ 10.1}

\myindex{Intel C++}
\myindex{\oracle}
\myindex{x86!\Instructions!JZ}

Der Intel C++ 10.1-Compiler, der für \oracle 11.2 für Linux 86 genutzt wurde, kann
zwei \JZ in einer Reihe ausgeben. Es gibt keine Referenz zum zweiten \JZ. Das zweite
ist also ohne Bedeutung.

\lstinputlisting[caption=kdli.o from libserver11.a,style=customasmx86]{other/kdli.lst}

\begin{lstlisting}[caption=from the same code,style=customasmx86]
.text:0811A2A5                   loc_811A2A5: ; CODE XREF: kdliSerLengths+11C
.text:0811A2A5                                ; kdliSerLengths+1C1
.text:0811A2A5 8B 7D 08              mov     edi, [ebp+arg_0]
.text:0811A2A8 8B 7F 10              mov     edi, [edi+10h]
.text:0811A2AB 0F B6 57 14           movzx   edx, byte ptr [edi+14h]
.text:0811A2AF F6 C2 01              test    dl, 1
.text:0811A2B2 75 3E                 jnz     short loc_811A2F2
.text:0811A2B4 83 E0 01              and     eax, 1
.text:0811A2B7 74 1F                 jz      short loc_811A2D8
.text:0811A2B9 74 37                 jz      short loc_811A2F2
.text:0811A2BB 6A 00                 push    0
.text:0811A2BD FF 71 08              push    dword ptr [ecx+8]
.text:0811A2C0 E8 5F FE FF FF        call    len2nbytes
\end{lstlisting}

Dies ist vermutlich ein Fehler im Codegenerator der während der Tests nicht
gefunden wurde. Der resultierende Code funktioniert trotzdem.

% TBT
% Another example from \oracle 11.1.0.6.0 for win32.

%\begin{lstlisting}
%.text:0051FBF8 85 C0                             test    eax, eax
%.text:0051FBFA 0F 84 8F 00 00 00                 jz      loc_51FC8F
%.text:0051FC00 74 1D                             jz      short loc_51FC1F
%\end{lstlisting}

\input{other/anomaly2_DE}
%\input{other/anomaly3_DE}

\subsection{Zusammenfassung}

Andere Compiler-Anomalien in diesem Buch:
\myref{anomaly:LLVM}, \myref{loops_iterators_loop_anomaly}, \myref{Keil_anomaly},
\myref{MSVC2013_anomaly},
\myref{MSVC_double_JMP_anomaly},
\myref{MSVC2012_anomaly}.

Diese Beispiele werden in diesem Buch gezeigt, um zu verdeutlichen, das solche Fehler
in den Compilern möglich sind und es gelegentlich keinen Sinn ergibt sich den Kopf
darüber zu zerbrechen warum der Compiler diesen \q{seltsamen} Code erzeugte.
