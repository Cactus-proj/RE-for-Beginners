\section{Mon expérience avec Hex-Rays 2.2.0}
\myindex{Hex-Rays}
\label{hex_rays}

\subsection{Bugs}

Il y a plusieurs bugs.

Tout d'abord, Hex-Rays est perdu lorsque des instructions \ac{FPU} sont mélangées
(par le générateur de code du compilateur) avec des autres.

Par exemple, ceci:

\begin{lstlisting}
f               proc    near

        	lea     eax, [esp+4]
	        fild    dword ptr [eax]
                lea     eax, [esp+8]
        	fild    dword ptr [eax]
	        fabs
                fcompp
        	fnstsw  ax
	        test    ah, 1
                jz      l01

        	mov     eax, 1
	        retn
l01:
                mov     eax, 2
	        retn

f               endp
\end{lstlisting}

\dots sera correctement décompilé en:

\begin{lstlisting}
signed int __cdecl f(signed int a1, signed int a2)
{
  signed int result; // eax@2

  if ( fabs((double)a2) >= (double)a1 )
    result = 2;
  else
    result = 1;
  return result;
}
\end{lstlisting}

Mais commentons une des instructions à la fin:

\begin{lstlisting}
...
l01:
	        ;mov    eax, 2
        	retn
...
\end{lstlisting}

\dots nous obtenons ce bug évident:

\begin{lstlisting}
void __cdecl f(char a1, char a2)
{
  fabs((double)a2);
}
\end{lstlisting}

Ceci est un autre bug:

\begin{lstlisting}
extrn f1:dword
extrn f2:dword

f               proc    near

	        fld     dword ptr [esp+4]
        	fadd    dword ptr [esp+8]
                fst     dword ptr [esp+12]
	        fcomp   ds:const_100
                fld     dword ptr [esp+16]      ; commenter cette instruction et ça sera OK
	        fnstsw  ax
        	test    ah, 1

                jnz     short l01

	        call    f1
        	retn
l01:
	        call    f2
        	retn

f               endp

...

const_100       dd 42C80000h            ; 100.0
\end{lstlisting}

Résultat:

\begin{lstlisting}
int __cdecl f(float a1, float a2, float a3, float a4)
{
  double v5; // st7@1
  char v6; // c0@1
  int result; // eax@2

  v5 = a4;
  if ( v6 )
    result = f2(v5);
  else
    result = f1(v5);
  return result;
}
\end{lstlisting}

La variable \TT{v6} a un type \TT{char} et si vous essayez de compiler ce code, le
compilateur vous avertira à propos de l'utilisation de variable avant son initialisation.

Un autre bug: l'instruction \INS{FPATAN} est correctement décompilée en \TT{atan2()},
mais les arguments sont échangés.

\subsection{Particularités bizarres}

Hex-Rays converti trop souvent des \TT{int} 32-bit en 64-bit.
Voici un exemple:

\begin{lstlisting}
f               proc    near

	        mov     eax, [esp+4]
                cdq
	        xor     eax, edx
        	sub     eax, edx
                ; EAX=abs(a1)

	        sub     eax, [esp+8]
        	; EAX=EAX-a2

                ; EAX à ce point est converti en 64-bit (RAX)

	        cdq
        	xor     eax, edx
                sub     eax, edx
	        ; EAX=abs(abs(a1)-a2)

                retn

f               endp
\end{lstlisting}

Résultat:

\begin{lstlisting}
int __cdecl f(int a1, int a2)
{
  __int64 v2; // rax@1

  v2 = abs(a1) - a2;
  return (HIDWORD(v2) ^ v2) - HIDWORD(v2);
}
\end{lstlisting}

Peut-être est-ce le résultat de l'instruction \INS{CDQ}? Je ne suis pas sûr.
Quoiqu'il en soit, à chaque fois que vous voyez le type \TT{\_\_int64} dans du code
32-bit, soyez attentif.

Ceci est aussi bizarre:

\begin{lstlisting}
f               proc    near

	        mov     esi, [esp+4]

        	lea     ebx, [esi+10h]
                cmp     esi, ebx
	        jge     short l00

                cmp     esi, 1000
	        jg      short l00

                mov     eax, 2
	        retn

l00:
	        mov     eax, 1
        	retn

f               endp
\end{lstlisting}

Résultat:

\begin{lstlisting}
signed int __cdecl f(signed int a1)
{
  signed int result; // eax@3

  if ( __OFSUB__(a1, a1 + 16) ^ 1 && a1 <= 1000 )
    result = 2;
  else
    result = 1;
  return result;
}
\end{lstlisting}

Le code est correct, mais il requiert une intervention manuelle.

Parfois, Hex-Rays ne remplace pas le code de la division par la multiplication:

\begin{lstlisting}
f               proc    near

        	mov     eax, [esp+4]
	        mov     edx, 2AAAAAABh
                imul    edx
        	mov     eax, edx

	        retn

f               endp
\end{lstlisting}

Résultat:

\begin{lstlisting}
int __cdecl f(int a1)
{
  return (unsigned __int64)(715827883i64 * a1) >> 32;
}
\end{lstlisting}

Ceci peut être remplacé manuellement.

Beaucoup de ces particularités peuvent être résolues en ré-arrangeant les instructions,
recompilant le code assembleur et en le renvoyant dans Hex-Rays.

\subsection{Silence}

\begin{lstlisting}
extrn some_func:dword

f               proc    near

        	mov     ecx, [esp+4]
	        mov     eax, [esp+8]
                push    eax
        	call    some_func
	        add     esp, 4

                ; use ECX
        	mov     eax, ecx

	        retn

f               endp
\end{lstlisting}

Résultat:

\begin{lstlisting}
int __cdecl f(int a1, int a2)
{
  int v2; // ecx@1

  some_func(a2);
  return v2;
}
\end{lstlisting}

La variable \TT{v2} (de ECX) est perdue \dots
Oui, ce code est incorrect (la valeur de ECX n'est pas sauvée lors de l'appel d'une
autre fonction), mais il serait bon que Hex-Rays donne un warning.

Un autre:

\begin{lstlisting}
extrn some_func:dword

f               proc    near

	        call    some_func
        	jnz     l01

                mov     eax, 1
	        retn
l01:
	        mov     eax, 2
        	retn

f               endp
\end{lstlisting}

Résultat:

\begin{lstlisting}
signed int f()
{
  char v0; // zf@1
  signed int result; // eax@2

  some_func();
  if ( v0 )
    result = 1;
  else
    result = 2;
  return result;
}
\end{lstlisting}

Encore une fois, un warning serait utile.

En tout cas, à chaque fois que vous voyez une variable de type \TT{char}, ou une
variable qui est utilisée sans initialisation, c'est un signe clair que quelque chose
s'est mal passé et nécessite une intervention manuelle.

\subsection{Virgule}
\myindex{\CLanguageElements!Comma}

La virgule en \CCpp a mauvaise presse, car elle peut conduire à du code confus.

Quiz rapide, que renvoie cette fonction \CCpp?

\begin{lstlisting}
int f()
{
	return 1, 2;
};
\end{lstlisting}

C'est 2: lorsque le compilateur rencontre une expression avec des virgules, il génère
du code qui exécute toutes les sous-expressions, et \TT{renvoie} la valeur de la
dernière.

J'ai vu quelque chose comme ça dans du code en production:

\begin{lstlisting}
if (cond)
	return global_var=123, 456; // 456 is returned
else
	return global_var=789, 321; // 321 is returned
\end{lstlisting}

Il semble que le programmeur voulait rendre le code plus court sans parenthèses supplémentaires.
Autrement dit, la virgule permet de grouper plusieurs expressions en une seule, sans
déclaration/bloc de code dans des parenthèses.

\myindex{Scheme}
\myindex{Racket}
La virgule en C/C++ est proche du \TT{begin} en Scheme/Racket: \url{https://docs.racket-lang.org/guide/begin.html}.

Peut-être que le seul usage largement accepté de la virgule est dans les déclarations
\TT{for()}:

\begin{lstlisting}
char *s="hello, world";
for(int i=0; *s; s++, i++);
; i = string lenght
\end{lstlisting}

À la fois \TT{s++} et \TT{i++} sont exécutés à chaque itération de la boucle.

Plus d'information:\\
\url{http://stackoverflow.com/questions/52550/what-does-the-comma-operator-do-in-c}.

\myindex{\CLanguageElements!Short-circuit}
J'ai écrit tout ceci car Hex-Rays produit (au moins dans mon cas) du code qui est
riche tant en virgules qu'en expression raccourcies:
Par exemple, ceci est une sortie réelle de Hex-Rays:

\begin{lstlisting}
 if ( a >= b || (c = a, (d[a] - e) >> 2 > f) )
    {
    	...
\end{lstlisting}

Ceci est correct, compile et fonctionne, et Dieu puisse vous aider à la comprendre.
La voici récrite:

\begin{lstlisting}
if (cond1 || (comma_expr, cond2))
{
	...
\end{lstlisting}

Le raccourci est effectif ici: d'abord \TT{cond1} est testé, si c'est \TT{true},
le corps du \TT{if()} est exécuté, le reste de l'expression du \TT{if()} est complètement
ignoré.
Si \TT{cond1} est \TT{false}, \TT{comma\_expr} est exécuté (dans l'exemple précédent,
\TT{a} est copié dans \TT{c}), puis \TT{cond2} est testée.
Si \TT{cond2} est \TT{true}, le corps du \TT{if()} est exécuté, ou pas.
Autrement dit, le corps du \TT{if()} est exécuté si \TT{cond1} est \TT{true} ou si \TT{cond2} est \TT{true},
mais si ce dernier est \TT{true}, \TT{comma\_expr} est aussi exécutée.

Maintenant, vous pouvez voir pourquoi la virgule est si célèbre.

\textbf{Un mot sur les raccourcis.}
Une idée fausse répandue de débutant est que les sous-conditions sont testées dans
un ordre indéterminé, ce qui n'est pas vrai.
Dans l'expression \TT{a | b | c},  $a$, $b$ et $c$ sont évalués dans un ordre indéterminé,
donc c'est pourquoi \TT{||} a été ajouté à C/C++, pour appliquer des raccourcis explicitement.

\subsection{Types de donnée}

Les types de donnée sont un problème pour les décompilateurs.

Hex-Rays peut ne pas voir les tableaux dans la pile locale, si ils n'ont pas été déterminés
avant la décompilation. Même histoire avec les tableaux globaux.

Un autre problème se pose avec les fonctions trop grosses, où un slot unique dans
la pile locale peut être utilisé par plusieurs variables durant l'exécution de la
fonction.
Ce n'est pas un cas rare que lorsqu'un slot est utilisé pour une variable \TT{int},
puis un pointeur, puis une variable \TT{float}.
Hex-Rays le décompile correctement: il créé une variable avec le même type, puis
la caste sur un autre type dans diverses parties de la fonction.
J'ai résolu ce problème en découpant les grosses fonctions en plusieurs plus petites.
Met les variables locales comme des globales, etc., etc.
Et n'oubliez pas les tests.

\subsection{Expressions longues et confuses}

Parfois, lors de la ré-écriture, vous pouvez vous retrouvez avec des expressions
longues et difficiles à comprendre dans des constructions \TT{if()} comme:

\begin{lstlisting}
if ((! (v38 && v30 <= 5 && v27 != -1)) && ((! (v38 && v30 <= 5) && v27 != -1) || (v24 >= 5 || v26)) && v25)
{
...
}
\end{lstlisting}

Wolfram Mathematica peut minimiser certaines d'entre elles, en utilisant la fonction
\TT{BooleanMinimize[]}:

\begin{lstlisting}
In[1]:= BooleanMinimize[(! (v38 && v30 <= 5 && v27 != -1)) && v38 && v30 <= 5 && v25 == 0]

Out[1]:= v38 && v25 == 0 && v27 == -1 && v30 <= 5
\end{lstlisting}

Il y a encore une meilleure voie, pour trouver les sous-expressions communes:

\begin{lstlisting}
In[2]:= Experimental`OptimizeExpression[(! (v38 && v30 <= 5 && 
      v27 != -1)) && ((! (v38 && v30 <= 5) && 
      v27 != -1) || (v24 >= 5 || v26)) && v25]

Out[2]= Experimental`OptimizedExpression[
 Block[{Compile`$1, Compile`$2}, Compile`$1 = v30 <= 5; 
  Compile`$2 = 
   v27 != -1; ! (v38 && Compile`$1 && 
      Compile`$2) && ((! (v38 && Compile`$1) && Compile`$2) || 
     v24 >= 5 || v26) && v25]]
\end{lstlisting}

Mathematica ajoute deux nouvelles variables: \TT{Compile`\$1} et  \TT{Compile`\$2},
qui vont être ré-utilisées plusieurs fois dans l'expression.
Donc, nous pouvons ajouter deux variables supplémentaires.

\subsection{Loi de De Morgan et décompilation}
\myindex{De Morgan's laws}

Parfois l'optimiseur du compilateur peut utiliser la loi de De Morgan pour rendre
le code plus court/rapide.

Par exemple, ceci:

\begin{lstlisting}[style=customc]
void f(int a, int b, int c, int d)
{
	if (a>0 && b>0)
		printf ("both a and b are positive\n");
	else if (c>0 && d>0)
		printf ("both c and d are positive\n");
	else
		printf ("something else\n");
};
\end{lstlisting}

\dots ça semble assez anodin, lorsque c'est compilé avec GCC 5.4.0 x64 avec optimisation:

\begin{lstlisting}[style=customasmx86]
; \verb|int __fastcall f(int a, int b, int c, int d)|
                public f
f               proc near
                test    edi, edi
                jle     short loc_8
                test    esi, esi
                jg      short loc_30

loc_8:
                test    edx, edx
                jle     short loc_20
                test    ecx, ecx
                jle     short loc_20
                mov     edi, offset s   ; "both c and d are positive"
                jmp     puts

loc_20:
                mov     edi, offset aSomethingElse ; "something else"
                jmp     puts

loc_30:
                mov     edi, offset aAAndBPositive ; "both a and b are positive"

loc_35:
                jmp     puts
f               endp
\end{lstlisting}

\dots ça semble donc anodin, mais Hex-Rays 2.2.0 n'arrive pas vraiment à détecter
qu'en fait des opérations double AND ont été utilisées dans le code source:

\begin{lstlisting}[style=customc]
int __fastcall f(int a, int b, int c, int d)
{
  int result;

  if ( a > 0 && b > 0 )
  {
    result = puts("both a and b are positive");
  }
  else if ( c <= 0 || d <= 0 )
  {
    result = puts("something else");
  }
  else
  {
    result = puts("both c and d are positive");
  }
  return result;
}
\end{lstlisting}

L'expression \verb~c <= 0 || d <= 0~ est la contraposée de \verb|c>0 && d>0| puisque
$\overline{A \cup B} = \overline{A} \cap \overline{B}$ et
$\overline{A \cap B} = \overline{A} \cup \overline{B}$,
Autrement dit,
\verb~!(cond1 || cond2) == !cond1 && !cond2~ et \verb~!(cond1 && cond2) == !cond1 || !cond2~.

Ça vaut la peine de garder ces règles à l'esprit, puisque ces optimisations du
compilateur sont utilisées massivement presque partout.

C'est parfois une bonne idée d'inverser une condition, afin de mieux comprendre
le code.
Ceci est un morceau de code réel décompilé par Hex-Rays:

\begin{lstlisting}[style=customc]
	for (int i=0; i<12; i++)
	{
		if (v1[i-12] != 0.0 || v1[i] != 0.0)
		{
			v108=min(v108, (float)v0[i*24 -2]);
			v113=max(v113, (float)v0[i*24]);
		};
	}
\end{lstlisting}

\dots qui peut être récrit comme:

\begin{lstlisting}[style=customc]
	for (int i=0; i<12; i++)
	{
		if (v1[i-12] == 0.0 && v1[i] == 0.0)
			continue;

		v108=min(v108, (float)v0[i*24 -2]);
		v113=max(v113, (float)v0[i*24]);
	}
\end{lstlisting}

Lequel est le meilleur? Je ne sais pas encore, mais pour une meilleure compréhension,
c'est bien de regarder les deux.



\subsection{Mon plan}

\begin{itemize}
\item Séparer les grosses fonctions (et ne pas oublier de tester).
Parfois c'est utile de créer des nouvelles fonctions à partir des corps de boucles.

\item Vérifier/tester le type des variables, tableaux, etc.

\item Si vous voyez un résultat étrange, une variable \TT{dangling} (qui est utilisée
avant son initialisation), essayez d'échanger les instructions manuellement, recompilez
et repassez-le à Hex-Rays.
\end{itemize}

\subsection{Résumé}

Quoiqu'il en soit, la qualité de Hex-Rays 2.2.0 est très, très bonne.
Il rend la vie plus facile.

