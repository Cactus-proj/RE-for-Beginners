\mysection{Compiler's anomalies}
\label{anomaly:Intel}
\myindex{\CompilerAnomaly}

\subsection{\oracle 11.2 and Intel C++ 10.1}

\myindex{Intel C++}
\myindex{\oracle}
\myindex{x86!\Instructions!JZ}

Intel C++ 10.1, which was used for \oracle 11.2 Linux86 compilation, may emit two \JZ in row,
and there are no references to the second \JZ. The second \JZ is thus meaningless.

\lstinputlisting[caption=kdli.o from libserver11.a,style=customasmx86]{other/kdli.lst}

\begin{lstlisting}[caption=from the same code,style=customasmx86]
.text:0811A2A5                   loc_811A2A5: ; CODE XREF: kdliSerLengths+11C
.text:0811A2A5                                ; kdliSerLengths+1C1
.text:0811A2A5 8B 7D 08              mov     edi, [ebp+arg_0]
.text:0811A2A8 8B 7F 10              mov     edi, [edi+10h]
.text:0811A2AB 0F B6 57 14           movzx   edx, byte ptr [edi+14h]
.text:0811A2AF F6 C2 01              test    dl, 1
.text:0811A2B2 75 3E                 jnz     short loc_811A2F2
.text:0811A2B4 83 E0 01              and     eax, 1
.text:0811A2B7 74 1F                 jz      short loc_811A2D8
.text:0811A2B9 74 37                 jz      short loc_811A2F2
.text:0811A2BB 6A 00                 push    0
.text:0811A2BD FF 71 08              push    dword ptr [ecx+8]
.text:0811A2C0 E8 5F FE FF FF        call    len2nbytes
\end{lstlisting}

It is supposedly a code generator bug that was not found by tests, because 
resulting code works correctly anyway.

Another example from \oracle 11.1.0.6.0 for win32.

\begin{lstlisting}
.text:0051FBF8 85 C0                             test    eax, eax
.text:0051FBFA 0F 84 8F 00 00 00                 jz      loc_51FC8F
.text:0051FC00 74 1D                             jz      short loc_51FC1F
\end{lstlisting}

\input{other/anomaly2_EN}
\subsection{ftol2() in MSVC 2012}

Just found this in ftol2() standard C/C++ library function (float-to-long conversion routine) in Microsoft Visual Studio 2012.

\lstinputlisting[style=customasmx86]{other/ftol2_EN.asm}

Note two identical \INS{FSTP}-s (\emph{float store with pop}) at the end. 
First I thought it was compiler anomaly (I'm collecting such cases just as someone do with butterflies),
but it seems, it's handwritten assembler piece, in msvcrt.lib there is an object file with this function in it,
and we can find this string in it:\\
\verb|f:\dd\vctools\crt_bld\SELF_X86\crt\prebuild\tran\i386\ftol2.asm| ---
that was probably a path to the file on developer's computer where msvcrt.lib was built.

So, bug, text editor-induced typo, or it was done by intent?
The code working correctly, anyway.



\subsection{Summary}

Other compiler anomalies here in this book: 
\myref{anomaly:LLVM}, \myref{loops_iterators_loop_anomaly}, \myref{Keil_anomaly},
\myref{MSVC2013_anomaly},
\myref{MSVC_double_JMP_anomaly},
\myref{MSVC2012_anomaly}.

Such cases are demonstrated here in this book, to show that such compilers errors are possible and sometimes
one should not to rack one's brain while thinking why did the compiler generate such strange code.

