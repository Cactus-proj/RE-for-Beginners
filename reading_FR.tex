\chapter{Livres/blogs qui valent le détour}

\mysection{Livres et autres matériels}

\subsection{Rétro-ingénierie}

\input{RE_books}

Également, les livres de Kris Kaspersky.

\subsection{Windows}

\begin{itemize}
\item \Russinovich
\item Peter Ferrie -- The ``Ultimate'' Anti-Debugging Reference\footnote\url{http://pferrie.host22.com/papers/antidebug.pdf}}
\end{itemize}

\EN{Blogs}\ES{Blogs}\RU{Блоги}\FR{Blogs}\DE{Blogs}\PL{Blogi}:

\begin{itemize}
\item \href{http://go.yurichev.com/17025}{Microsoft: Raymond Chen}
\item \href{http://go.yurichev.com/17026}{nynaeve.net}
\end{itemize}



\subsection{\CCpp}

\label{CCppBooks}

\begin{itemize}

\item \KRBook

\item \CNineNineStd\footnote{\AlsoAvailableAs \url{http://www.open-std.org/jtc1/sc22/WG14/www/docs/n1256.pdf}}

\item \TCPPPL

\item \CppOneOneStd\footnote{\AlsoAvailableAs \url{http://www.open-std.org/jtc1/sc22/wg21/docs/papers/2013/n3690.pdf}.}

\item \AgnerFogCPP\footnote{\AlsoAvailableAs \url{http://agner.org/optimize/optimizing_cpp.pdf}.}

\item \ParashiftCPPFAQ\footnote{\AlsoAvailableAs \url{http://www.parashift.com/c++-faq-lite/index.html}}

\item \CNotes\footnote{\AlsoAvailableAs \url{http://yurichev.com/C-book.html}}

\item JPL Institutional Coding Standard for the C Programming Language\footnote{\AlsoAvailableAs \url{https://yurichev.com/mirrors/C/JPL_Coding_Standard_C.pdf}}

\RU{\item Евгений Зуев --- Редкая профессия\footnote{\AlsoAvailableAs \url{https://yurichev.com/mirrors/C++/Redkaya_professiya.pdf}}}

\end{itemize}



\subsection{Architecture x86 / x86-64}

\label{x86_manuals}
\begin{itemize}
\item Manuels Intel\footnote{\AlsoAvailableAs \url{http://www.intel.com/content/www/us/en/processors/architectures-software-developer-manuals.html}}

\item Manuels AMD\footnote{\AlsoAvailableAs \url{http://developer.amd.com/resources/developer-guides-manuals/}}

\item \AgnerFog{}\footnote{\AlsoAvailableAs \url{http://agner.org/optimize/microarchitecture.pdf}}

\item \AgnerFogCC{}\footnote{\AlsoAvailableAs \url{http://www.agner.org/optimize/calling_conventions.pdf}}

\item \IntelOptimization

\item \AMDOptimization
\end{itemize}

Quelque peu vieux, mais toujours intéressant à lire :

\MAbrash\footnote{\AlsoAvailableAs \url{https://github.com/jagregory/abrash-black-book}}
(il est connu pour son travail sur l'optimisation bas niveau pour des projets tels que Windows NT 3.1 et id Quake).

\subsection{ARM}

\begin{itemize}
\item Manuels ARM\footnote{\AlsoAvailableAs \url{http://infocenter.arm.com/help/index.jsp?topic=/com.arm.doc.subset.architecture.reference/index.html}}

\item \ARMSevenRef

\item \ARMSixFourRefURL

\item \ARMCookBook\footnote{\AlsoAvailableAs \url{http://go.yurichev.com/17273}}
\end{itemize}

\subsection{Langage d'assemblage}

Richard Blum --- Professional Assembly Language.

\subsection{Java}

\JavaBook.

\subsection{UNIX}

\TAOUP

\subsection{Programmation en général}

\begin{itemize}

\item \RobPikePractice

\item \HenryWarren
Certaines personnes disent que les trucs et astuces de ce livre ne sont plus pertinents
aujourd'hui, car ils n'étaient valables que pour les \ac{CPU}s \ac{RISC}, où les instructions
de branchement sont coûteuses.
Néanmoins, ils peuvent énormément aider à comprendre l'algèbre booléenne et toutes les
mathématiques associées.

\end{itemize}

%subsection:
\input{crypto_reading}

\iffalse
\subsection{Dédicace}

Comme le dit la première page de ce livre, ``Ce livre est dédicacé à Robert Jourdain,
John Socha, Ralf Brown et Peter Abel''.
Ce sont les auteurs de livres bien connus relatif au langage d'assemblage et de
référence des années 1980 et 1990:

\begin{itemize}
\item Robert Jourdain -- Programmer's problem solver for the IBM PC, XT, \& AT (1986)

\item Peter Norton and John Socha -- The Peter Norton Programmer's Guide to the IBM PC (1985),
Peter Norton's Assembly Language Book for the IBM PC (1989).
En fait, John Socha est l'auteur réel de ces livres, on peut dire qu'il était un
écrivain fantome. % un nègre? Vérifier si Peter Norton existe et qui signe les livres
Il est aussi l'auteur de Norton Commander.

\item Ralph Brown était connu pour ``Ralf Brown's Interrupt List''\footnote{\url{http://www.ctyme.com/rbrown.htm}}.

\item Peter Abel -- IBM PC assembly language and programming (1991)
\end{itemize}

Ces livres sont dépassés, bien sûr.
Mais peut-être que quelqu'un se rappellera de ``ce temps là''.
\fi
