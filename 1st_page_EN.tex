% To translators: don't bother to translate this... english-only version.

\begin{center}
\LARGE{} This is my own bulletin board \normalsize{}
\end{center}

I would be happy to hear from anyone interested in my consulting services.
The area of my expertise is programming in general, reverse engineering, maybe technical writing...

-> \EMAIL{}

\myhrule{}

\textbf{This book is probably outdated already}.
(Unless it was just downloaded from \url{https://beginners.re/}.)

The book is \href{https://github.com/DennisYurichev/RE-for-beginners/commits/master}{changing too often},
content being added, bugs are (hopefully) being fixed.
The latest version is always at \url{https://beginners.re/}.

This PDF you currently reading was compiled at \today{}.

\myhrule{}

My dear readers! From time to time, I have questions, I don't know who (or where) to ask.
Or I'm just lazy...
Can you please help me?

\myhrule{}

What do you use on Linux in place of Adobe Acrobat Pro?
To edit contents, add bookmarks, notes, highlight text...

\myhrule{}

What does tilde means in versions number in .dsc files, which describing Ubuntu packages?
Some kind of wildcard?
And what does \verb|>>| means?
\emph{Pipe} is just \emph{OR}?
For example:

\begin{lstlisting}
Build-Depends: cython-dbg | python-pyrex, ca-certificates, debhelper (>> 8.1.0~), python (>= 2.6.6-3), python-all-dev (>= 2.6.6-3), python-all-dbg (>= 2.6.6-3), python-configobj (>= 4.7.2+ds-2), python-docutils, python-paramiko, python-pycurl-dbg, python-subunit, python-testtools (>= 0.9.5~)
\end{lstlisting}

\myhrule{}

What do you use on unrooted Android device as a SSH server, that can write to external micro-SD card?
I was happy with SSHDroid, but now it's outdated...
For instance, when Total Commander writes to micro-SD, a dialog box appears, whether to allow this or not.
You allow it, and it can write to any folder on flash.
This is not a case with SSH-servers that I saw.

\myhrule{}

A win32 process A is running.
Process B is attaching to it as a debugger, or opens it using OpenProcess().
ReadProcessMemory() works OK, but fails if it tries to read uncommited memory pages of process A.

The problem: how to force the Windows Memory Manager to commit a page in process A from userland of process B?
I can inject a read instruction into process A, run it, and the page would be committed, but this is not the solution.

\myhrule{}

If you know something, please help me: \EMAIL{}

