% To translators: don't bother to translate this... english-only version.

\begin{center}
\LARGE{} Это моя собственная доска объявления \normalsize{}
\end{center}

\iffalse
\myhrule{}

В связи с миграцией, раздаю даром. В Киеве, на Подоле. Самовывоз отсюда.

\begin{itemize}
\item Что остаётся от моих старых компов.
Все эти старые железки примечательны тем, что я при помощи оных писал вот эти вот книги, в указанные периоды времени.
\begin{itemize}
%\item (2011-2019) два дисплея 21.5" Acer P226HQLBD Black, 1920 x 1080
\item (2011-2019) дисплей 21.5" Acer P226HQLBD Black, 1920 x 1080
%\item (2017-2019) клава HyperX Alloy Cherry MX Brown, юзалась активно
%\item (2019) ноутбучный SATA-винт (2.5) Seagate на 160GB, 5400 RPM
\item (2019) ноутбучная память DDR3 1GB SO-DIMM PC3-8500 1.5v
        \item (2019) ноутбучная память DDR3 2GB SO-DIMM PC3-8500 1.5v
        \item (2019) ноутбучный проц Intel® Core™2 Duo processor P8600 (2.40GHz, 3MB L2, 1066MHz FSB, TDP 25W)
        \end{itemize}

        \item Бумажные книги на русском:
        \begin{itemize}
        \item Б.Страуструп про С++ --- почти новая
        \item Структура и интерпретация компьютерных программ --- новая.
        \item Идеальный код (Beautiful code)
        \item Прочее: \url{https://yurichev.com/giveaway/books/}
        \end{itemize}

        \item FPGA-борда: Nios II Development Kit, Stratix II Edition: \href{https://www.intel.com/content/www/us/en/programmable/products/boards_and_kits/dev-kits/altera/kit-niosii-2s60.html}{info}.
        На ней работала ломалка хешей для Oracle RDBMS, подключенная к инету: \url{https://yurichev.com/ops_FPGA.html}.
        Увы, без USB-бластера.

        \item Осциллограф DSO Quad: \url{http://wiki.seeedstudio.com/DSO_Quad/}
        \item Глюкометр Contour Plus, новый, юзался только 3 раза
        %\item Сканер Epson Perfection V19, юзался немного
        %\item Велик Centurion backfire comp m6, юзался несколько месяцев

        \item (Старая) палатка Pinguin Gemini 150 Extreme
        \item Прочее: \url{https://yurichev.com/giveaway/}

        \end{itemize}

Эта страница будет обновляться время то времени...
В случае невостребованности, всё это выкидывается...
\fi

        \myhrule{}

\textbf{Эта книга наверняка уже устарела}.
(Если только не была скачана прямо сейчас с \url{https://beginners.re/}.)

Книга \href{\RepoURL/ChangeLog}{меняется очень часто},
контент добавляется, ошибки (будем надеяться) исправляются.
Также, в первую очередь книга пишется на английском, а перевод на русский немного запаздывает.
Последняя версия всегда на \url{https://beginners.re/}.

А PDF-файл, который вы сейчас читаете, был скомпилирован \today{}.

\myhrule{}

Если вы распечатали эту книгу на бумаге, не могли бы вы прислать мне её фотографию, для коллекции?\\
\EMAILS{}.
Сама коллекция, пока что: \url{https://yurichev.com/news/20200222_printed_RE4B/}.

\myhrule{}

Мои дорогие читатели! Время от времени, у меня появляются вопросы, и я не знаю, кого (или где) спросить.
Или я просто ленив...
Поможете мне?

\myhrule{}

Нужно из Киева закачать на сервак на hetzner ~10TB. Что делать?

\myhrule{}

Есть у меня какие-нибудь знакомые в Украине, покупающие Биткоины в обмен на наличные доллары-евро?

\myhrule{}

В Windows 10, по умолчанию, если процесс падает, не появляется окно с выбором JIT-отладчика.
Как включить его?

\myhrule{}

Есть большой граф, например, миллион узлов (вершин).
Нужно его визуализировать как-то, чтобы пользователь мог ходить по графу при помощи мыши.
Нажал на линк (ребро), переместился на другой узел (вершину).
Вот примерно как в IDA.
Может быть, при помощи JavaScript.
Есть какие-то опенсорсные готовые решения?

\myhrule{}

Помните ли вы игру ``The Incredible Machine'' под DOS?
Знаете ли о машинах Руби Голдберга?
Что в наше время можно использовать для симуляции оных?
Может быть, какой-нибудь физический движок?

\myhrule{}

Где можно накачать баз и телефонных справочников, которые используются на сайте \url{http://nomerorg.website/}?

\myhrule{}

Кто-нибудь может помочь мне с Low Fragmentation Heap в Windows?

\myhrule{}

Соц.опрос: как вы используете DLL injection кроме как для перехвата вызовов API?

\myhrule{}

Блютузовые наушники ERGO BT-590 имеют сенсорные кнопки, слишком чувствительные, и их легко задеть одеждой.
Как в Андроиде сделать так, чтобы Андроид игнорировал сообщения от наушников о нажатии кнопок?

\myhrule{}

Если знаете что-то, пожалуйста помогите мне: \EMAILS{}.

