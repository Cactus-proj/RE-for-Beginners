% To translators: don't bother to translate this... english-only version.

\begin{center}
\LARGE{} Это моя собственная доска объявления \normalsize{}
\end{center}

\textbf{Эта книга наверняка уже устарела}.
(Если только не была скачана прямо сейчас с \url{https://beginners.re/}.)

Книга \href{https://github.com/DennisYurichev/RE-for-beginners/commits/master}{меняется очень часто},
контент добавляется, ошибки (будем надеяться) исправляются.
Также, в первую очередь книга пишется на английском, а перевод на русский немного запаздывает.
Последняя версия всегда на \url{https://beginners.re/}.

А PDF-файл, который вы сейчас читаете, был скомпилирован \today{}.

\myhrule{}

Если вы распечатали эту книгу на бумаге, не могли бы вы прислать мне её фотографию, для коллекции?\\
\EMAIL{}, Telegram: @yurichev.

\myhrule{}

Мои дорогие читатели! Время от времени, у меня появляются вопросы, и я не знаю, кого (или где) спросить.
Или я просто ленив...
Поможете мне?

\myhrule{}

В самом начале 1990-х вышла книга на русском, где были законы Мерфи, Паркинсона, итд...
Как называлась?
Забыл.

\myhrule{}

Какой меломанский HiFi mp3-плеер за \$200-300 выбрать?
Hifiman HM-601 меня устраивал целиком...

\myhrule{}

Нужно проиндексировать пачку текстов. Потом сделать поиск по ним. Желателен простенький query-язык.
Какую поискать легковесную библиотеку для этого?
Желательно Питон или С++.

\myhrule{}

Как инсталлировать и запускать Cyc?

\myhrule{}

Как инсталлировать VMware Remote Console 10.0.4 на Ubuntu 19? Инсталлятор просто молча останавливается. Это известный симптом?

Или что вы используете для запуска VMware Workstation-машин на удаленном хосте на Ubuntu?

\myhrule{}

Что означает тильда в номерах версий, в .dsc-файлах, которые описывают пакеты Ubuntu?
Это wildcard (символ подстановки)?
И что означает \verb|>>|?
\emph{Труба} это просто \emph{ИЛИ}?
Например:

\begin{lstlisting}
Build-Depends: cython-dbg | python-pyrex, ca-certificates, debhelper (>> 8.1.0~), python (>= 2.6.6-3), python-all-dev (>= 2.6.6-3), python-all-dbg (>= 2.6.6-3), python-configobj (>= 4.7.2+ds-2), python-docutils, python-paramiko, python-pycurl-dbg, python-subunit, python-testtools (>= 0.9.5~)
\end{lstlisting}

\myhrule{}

Win32-процесс А запущен.
Процесс Б аттачится к нему как отладчик, либо открывает его используя OpenProcess().
ReadProcessMemory() работает, но не работает, если пытается читать незакоммиченные страницы процесса А.

Проблема: как заставить менеджер памяти Windows закоммитить страницу в процессе А из userland-а процесса Б?
Я могу всунуть в процесс А инструкцию чтения, запустить её, страница закоммитится, но это не решение.

\myhrule{}

Если знаете что-то, пожалуйста помогите мне: \EMAIL{}, Telegram: @yurichev, Skype: dennis.yurichev
