% To translators: don't bother to translate this... english-only version.

\begin{center}
\LARGE{} Это моя собственная доска объявления \normalsize{}
\end{center}

\textbf{Эта книга наверняка уже устарела}.
(Если только не была скачана прямо сейчас с \url{https://beginners.re/}.)

Книга \href{https://github.com/DennisYurichev/RE-for-beginners/commits/master}{меняется очень часто},
контент добавляется, ошибки (будем надеяться) исправляются.
Также, в первую очередь книга пишется на английском, а перевод на русский немного запаздывает.
Последняя версия всегда на \url{https://beginners.re/}.

А PDF-файл, который вы сейчас читаете, был скомпилирован \today{}.

\myhrule{}

Мои дорогие читатели! Время от времени, у меня появляются вопросы, и я не знаю, кого (или где) спросить.
Или я просто ленив...
Поможете мне?

\myhrule{}

Как инсталлировать и запускать Cyc?

\myhrule{}

Midnight Commander: хочу добавить в меню (F2) команду вроде "cd path/DD/MM/YYYY".
Что делать?
Так не получилось: \verb|cd ... $(date +%m) ...|

\myhrule{}

Как инсталлировать VMware Remote Console 10.0.4 на Ubuntu 19? Инсталлятор просто молча останавливается. Это известный симптом?

Или что вы используете для запуска VMware Workstation-машин на удаленном хосте на Ubuntu?

\myhrule{}

Что использовать в Линуксе вместо place of Adobe Acrobat Pro?
Редактировать содеражние, добавлять закладки, заметки, подсвечивать текст...

\myhrule{}

Что означает тильда в номерах версий, в .dsc-файлах, которые описывают пакеты Ubuntu?
Это wildcard (символ подстановки)?
И что означает \verb|>>|?
\emph{Труба} это просто \emph{ИЛИ}?
Например:

\begin{lstlisting}
Build-Depends: cython-dbg | python-pyrex, ca-certificates, debhelper (>> 8.1.0~), python (>= 2.6.6-3), python-all-dev (>= 2.6.6-3), python-all-dbg (>= 2.6.6-3), python-configobj (>= 4.7.2+ds-2), python-docutils, python-paramiko, python-pycurl-dbg, python-subunit, python-testtools (>= 0.9.5~)
\end{lstlisting}

\myhrule{}

Что использовать на нерутованном Андроиде как SSH-сервер, который умеет писать на внешнюю micro-SD?
SSHDroid нормально работал, но теперь он устарел...
Например, когда Total Commander пишет на micro-SD, появляется окно с вопросом, разрешить или нет.
Вы разрешаете, и он начинает писать в любые директории на флешке.
В случае с виденными мною SSH-серверами этого что-то не происходит.

\myhrule{}

Win32-процесс А запущен.
Процесс Б аттачится к нему как отладчик, либо открывает его используя OpenProcess().
ReadProcessMemory() работает, но не работает, если пытается читать незакоммиченные страницы процесса А.

Проблема: как заставить менеджер памяти Windows закоммитить страницу в процессе А из userland-а процесса Б?
Я могу всунуть в процесс А инструкцию чтения, запустить её, страница закоммитится, но это не решение.

\myhrule{}

Если знаете что-то, пожалуйста помогите мне: \EMAIL{}

