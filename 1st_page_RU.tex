% To translators: don't bother to translate this... english-only version.

\begin{center}
\LARGE{} Это моя собственная доска объявления \normalsize{}
\end{center}

\textbf{Если вы наблюдали меня в ЖЖ в 2011-2014, прочтите \href{http://link.yurichev.com/46001}{это}.}

\myhrule{}

\textbf{Эта книга наверняка уже устарела}.
(Если только не была скачана прямо сейчас с \url{https://beginners.re/}.)

Книга \href{\GitHubURL/commits/master}{меняется очень часто},
контент добавляется, ошибки (будем надеяться) исправляются.
Также, в первую очередь книга пишется на английском, а перевод на русский немного запаздывает.
Последняя версия всегда на \url{https://beginners.re/}.

А PDF-файл, который вы сейчас читаете, был скомпилирован \today{}.

\myhrule{}

Аукцион!
Продам в Украине (Киеве) тому, кто предложит больше денег.

\begin{itemize}
\item Что остаётся от моих компов. Все эти железки примечательны тем, что я при помощи оных писал вот эти вот книги, в указанные периоды времени.
	\begin{itemize}
	\item (2011-2019) два дисплея 21.5" Acer P226HQLBD Black, 1920 x 1080 
	\item (2017-2019) клава HyperX Alloy Cherry MX Brown, юзалась активно
	\end{itemize}

\item Бумажные книги на русском:
	\begin{itemize}
	\item Б.Страуструп про С++ --- почти новая
	\item Структура и интерпретация компьютерных программ --- новая.
	\item Идеальный код (Beautiful code)
	\end{itemize}

\item FPGA-борда: Nios II Development Kit, Stratix II Edition: \href{https://www.intel.com/content/www/us/en/programmable/products/boards_and_kits/dev-kits/altera/kit-niosii-2s60.html}{info}.
На ней работала ломалка хешей для Oracle RDBMS, подключенная к инету: \url{https://yurichev.com/ops_FPGA.html}.
\item Осциллограф DSO Quad: \url{http://wiki.seeedstudio.com/DSO_Quad/}
\item Глюкометр Contour Plus, новый, юзался только 3 раза
\item Сканер Epson Perfection V19, юзался немного
\item Велик Centurion backfire comp m6, юзался несколько месяцев

\item (Старая) палатка Pinguin Gemini 150 Extreme

Могу всё это отправлять Новой Поштой по Украине...

\end{itemize}

\myhrule{}

Может в шахматы?
Мой аккаунт: \url{https://lichess.org/@/DY1979}.
Я обычно играю по переписке (``correspondence''), 14 дней на ход.
Пришлите мне \textit{challenge}...
Я играю примерно как 4-й или 3-й разряд...

\myhrule{}

Есть у меня какие-нибудь знакомые в Украине, покупающие Биткоины в обмен на наличные доллары-евро?

\myhrule{}

Если вы распечатали эту книгу на бумаге, не могли бы вы прислать мне её фотографию, для коллекции?\\
\EMAIL{}, Telegram: @yurichev.

\myhrule{}

Мои дорогие читатели! Время от времени, у меня появляются вопросы, и я не знаю, кого (или где) спросить.
Или я просто ленив...
Поможете мне?

\myhrule{}

Помните ли вы игру ``The Incredible Machine'' под DOS?
Знаете ли о машинах Руби Голдберга?
Что в наше время можно использовать для симуляции оных?
Может быть, какой-нибудь физический движок?

\myhrule{}

Есть вопросы по математической статистике. Кто-нибудь может помочь?

\myhrule{}

Где можно накачать баз и телефонных справочников, которые используются на сайте \url{http://nomerorg.website/}?

\myhrule{}

Блютузовые наушники ERGO BT-590 имеют сенсорные кнопки, слишком чувствительные, и их легко задеть одеждой.
Как в Андроиде сделать так, чтобы Андроид игнорировал сообщения от наушников о нажатии кнопок?

\myhrule{}

Кто-нибудь может мне помочь с CBMC? Есть вопросы...

\myhrule{}

В самом начале 1990-х вышла книга на русском, где были законы Мерфи, Паркинсона, итд...
Как называлась?
Забыл.

\myhrule{}

Какой меломанский HiFi mp3-плеер за \$200-300 выбрать?
Hifiman HM-601 меня устраивал целиком...

\myhrule{}

Нужно проиндексировать пачку текстов. Потом сделать поиск по ним. Желателен простенький query-язык.
Какую поискать легковесную библиотеку для этого?
Желательно Питон или С++.

\myhrule{}

Как инсталлировать и запускать Cyc?

\myhrule{}

Если знаете что-то, пожалуйста помогите мне: \EMAIL{}, Telegram: @yurichev
