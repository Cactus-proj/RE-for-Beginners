% To translators: don't bother to translate this... english-only version.

\begin{center}
\LARGE{} Это моя собственная доска объявления \normalsize{}
\end{center}

\textbf{Эта книга наверняка уже устарела}.
(Если только не была скачана прямо сейчас с \url{https://beginners.re/}.)

Книга \href{\RepoURL/ChangeLog}{меняется очень часто},
контент добавляется, ошибки (будем надеяться) исправляются.
Также, в первую очередь книга пишется на английском, а перевод на русский немного запаздывает.
Последняя версия всегда на \url{https://beginners.re/}.

А PDF-файл, который вы сейчас читаете, был скомпилирован \today{}.

\myhrule{}

Если вы распечатали эту книгу на бумаге, не могли бы вы прислать мне её фотографию, для коллекции?\\
\EMAILS{}.
Сама коллекция, пока что: \url{https://yurichev.com/news/20200222_printed_RE4B/}.

\myhrule{}

Мои дорогие читатели! Время от времени, у меня появляются вопросы, и я не знаю, кого (или где) спросить.
Или я просто ленив...
Поможете мне?

\myhrule{}

Есть большой граф, например, миллион узлов (вершин).
Нужно его визуализировать как-то, чтобы пользователь мог ходить по графу при помощи мыши.
Нажал на линк (ребро), переместился на другой узел (вершину).
Вот примерно как в IDA.
Может быть, при помощи JavaScript.
Есть какие-то опенсорсные готовые решения?

\myhrule{}

Кто-нибудь может помочь мне с Low Fragmentation Heap в Windows?

\myhrule{}

Соц.опрос: как вы используете DLL injection кроме как для перехвата вызовов API?

\myhrule{}

Если знаете что-то, пожалуйста помогите мне: \EMAILS{}.

