\subsection*{mini-FAQ}

% TBT
%\par Q: Is this book simpler/easier than others?
%\par A: No, it is at about the same level as other books of this subject.
% TBT
%\par Q: I'm too frightened to start reading this book, there are more than 1000 pages.
%\par A: All sorts of listings are the bulk of the book.

\par Q: Quali sono i prerequisiti per leggere questo libro?
\par A: È consigliato avere almeno una conoscenza base di C/C++.

\par Q: Dovrei veramente imparare x86/x64/ARM e MIPS allo stesso tempo? Non è troppo?
\par A: Chi inizia può leggere semplicemente su x86/x64, e saltare o sfogliare velocemente le parti su ARM e MIPS.

\par Q: Posso acquistare la versione in Russo/Inglese del libro?
\par A: Sfortunatamente no, nessun editore (al momento) è interessato nel pubblicare questo libro.
Nel frattempo puoi chiedere alla tua copisteria di fiducia di stamparlo.

\par Q: C'è una versione EPUB/MOBI?
\par A: Il libro dipende fortemente da alcuni hacks in TeX/LaTeX, quindi convertire il tutto in HTML (EPUB/MOBI è un set di HTMLs)
non sarebbe facile.

\par Q: Perchè qualcuno dovrebbe studiare assembly al giorno d'oggi?
\par A: A meno che tu non sia uno sviluppatore di \ac{sistemi operativi}, probabilmente non avrai mai bisogno di scrivere codice assembly \textemdash{}i compilatori moderni sono migliori dell'uomo nell'effettuare ottimizzazioni \footnote{Un testo consigliato relativamente a questo argomento: \InSqBrackets{\AgnerFog}}.

Inoltre, le \ac{CPU} moderne sono dispositivi molto complessi e la semplice conoscenza di assembly non basta per capire il loro funzionamento interno.

Ci sono però almeno due aree in cui una buona conoscenza di assembly può tornare utile: analisi malware/ricercatore in ambito sicurezza e per avere una miglior comprensione del codice compilato durante il debugging di un programma.
Questo libro è perciò pensato per quelle persone che vogliono capire il linguaggio assembly piuttosto che imparare a programmare con esso.

\par Q: Ho cliccato su un link all'interno del PDF, come torno indietro?
\par A: In Adobe Acrobat Reader clicca Alt+FrecciaSinistra. In Evince clicca il pulsante ``<''.

\par Q: Posso stampare questo libro / usarlo per insegnare?
\par A: Certamente! Il libro è rilasciato sotto licenza Creative Commons (CC BY-SA 4.0).

\par Q: Perchè questo libro è gratis? Hai svolto un ottimo lavoro. È sospetto come molte altre cose gratis.
\par A: Per mia esperienza, gli autori di libri tecnici fanno queste cose per auto-pubblicizzarsi. Non è possibile ottenere un buon ricavato da un lavoro così oneroso.

\par Q: Come si fa ad ottenere un lavoro nel campo del reverse engineering?
\par A: Ci sono threads di assunzione che appaiono di tanto in tanto su reddit RE\FNURLREDDIT{}.
Prova a guardare lì.

Qualcosa di simile si può anche trovare nel subreddit \q{netsec}.

% TBT
%\par Q: Compilers' versions in the book are outdated already...
%\par A: No need to follow all steps precisely.
%Use the compilers you already have installed on your \ac{OS}.
%Also, there is: \href{https://godbolt.org/}{Compiler Explorer}.

\par Q: Avrei una domanda...
\par A: Inviamela tramite e-mail (\EMAILS).
