\chapter{Books/blogs worth reading}

\mysection{Books and other materials}

\subsection{Reverse Engineering}

\input{RE_books}

Also, Kris Kaspersky's books.

\subsection{Windows}

\begin{itemize}
\item \Russinovich
\item Peter Ferrie -- The ``Ultimate'' Anti-Debugging Reference\footnote\url{http://pferrie.host22.com/papers/antidebug.pdf}}
\end{itemize}

\EN{Blogs}\ES{Blogs}\RU{Блоги}\FR{Blogs}\DE{Blogs}\PL{Blogi}:

\begin{itemize}
\item \href{http://go.yurichev.com/17025}{Microsoft: Raymond Chen}
\item \href{http://go.yurichev.com/17026}{nynaeve.net}
\end{itemize}



\subsection{\CCpp}

\label{CCppBooks}

\begin{itemize}

\item \KRBook

\item \CNineNineStd\footnote{\AlsoAvailableAs \url{http://www.open-std.org/jtc1/sc22/WG14/www/docs/n1256.pdf}}

\item \TCPPPL

\item \CppOneOneStd\footnote{\AlsoAvailableAs \url{http://www.open-std.org/jtc1/sc22/wg21/docs/papers/2013/n3690.pdf}.}

\item \AgnerFogCPP\footnote{\AlsoAvailableAs \url{http://agner.org/optimize/optimizing_cpp.pdf}.}

\item \ParashiftCPPFAQ\footnote{\AlsoAvailableAs \url{http://www.parashift.com/c++-faq-lite/index.html}}

\item \CNotes\footnote{\AlsoAvailableAs \url{http://yurichev.com/C-book.html}}

\item JPL Institutional Coding Standard for the C Programming Language\footnote{\AlsoAvailableAs \url{https://yurichev.com/mirrors/C/JPL_Coding_Standard_C.pdf}}

\RU{\item Евгений Зуев --- Редкая профессия\footnote{\AlsoAvailableAs \url{https://yurichev.com/mirrors/C++/Redkaya_professiya.pdf}}}

\end{itemize}



\subsection{x86 / x86-64}

\label{x86_manuals}
\begin{itemize}
\item Intel manuals\footnote{\AlsoAvailableAs \url{http://www.intel.com/content/www/us/en/processors/architectures-software-developer-manuals.html}}

\item AMD manuals\footnote{\AlsoAvailableAs \url{http://developer.amd.com/resources/developer-guides-manuals/}}

\item \AgnerFog{}\footnote{\AlsoAvailableAs \url{http://agner.org/optimize/microarchitecture.pdf}}

\item \AgnerFogCC{}\footnote{\AlsoAvailableAs \url{http://www.agner.org/optimize/calling_conventions.pdf}}

\item \IntelOptimization

\item \AMDOptimization
\end{itemize}

Somewhat outdated, but still interesting to read:

\MAbrash\footnote{\AlsoAvailableAs \url{https://github.com/jagregory/abrash-black-book}}
(he is known for his work on low-level optimization for such projects as Windows NT 3.1 and id Quake).

\subsection{ARM}

\begin{itemize}
\item ARM manuals\footnote{\AlsoAvailableAs \url{http://infocenter.arm.com/help/index.jsp?topic=/com.arm.doc.subset.architecture.reference/index.html}}

\item \ARMSevenRef

\item \ARMSixFourRefURL

\item \ARMCookBook\footnote{\AlsoAvailableAs \url{http://go.yurichev.com/17273}}
\end{itemize}

\subsection{Assembly language}

Richard Blum --- Professional Assembly Language.

\subsection{Java}

\JavaBook.

\subsection{UNIX}

\TAOUP

\subsection{Programming in general}

\begin{itemize}

\item \RobPikePractice

\item \HenryWarren.
Some people say tricks and hacks from the book are not relevant today because they were good only for \ac{RISC} \ac{CPU}s,
where branching instructions are expensive.
Nevertheless, these can help immensely to understand boolean algebra and what all the mathematics near it.

\item (For hard-core geeks with computer science and mathematical background) \TAOCP.
Some people arguing, if it worth for an average programmer to try hard to read these quite hard fundamental books.
I would say, it's worth just to skim them, to learn what \ac{CS} consists of.

\end{itemize}

% subsection:
\input{crypto_reading}

\iffalse
\subsection{Dedication}

As the first page of this book says, ``This book is dedicated to Robert Jourdain, John Socha, Ralf Brown and Peter Abel''.
These are authors of well-known assembly language related books and references from 1980's and 1990's:

\begin{itemize}
\item Robert Jourdain -- Programmer's problem solver for the IBM PC, XT, \& AT (1986)

\item Peter Norton and John Socha -- The Peter Norton Programmer's Guide to the IBM PC (1985), Peter Norton's Assembly Language Book for the IBM PC (1989).
In fact, John Socha is a real author of these books, it can be said, he was ghostwriter.
He is also the author of Norton Commander.

\item Ralph Brown was known for ``Ralf Brown's Interrupt List''\footnote{\url{http://www.ctyme.com/rbrown.htm}}.

\item Peter Abel -- IBM PC assembly language and programming (1991)
\end{itemize}

These are outdated books, of course.
But maybe someone will recall ``those times''.
\fi

